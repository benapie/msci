\chapter{Combinational logic design}

\section{Circuits}

\begin{definition}
    A \textbf{circuit} has:
    \begin{enumerate}
        \item one or more discrete valued input terminals;
        \item one or more discrete valued output terminals;
        \item a specification of the relationship between inputs and outputs;
        \item a specification of the delay between inputs changing and outputs responding.
    \end{enumerate}
    
    A circuit is made up of \textbf{elements} and \textbf{nodes}:
    \begin{enumerate}
        \item an element is itself a circuit with inputs, outputs, and specifications; and
        \item a node is a wire joining elements, whose voltage conveys a discrete valued variable.
    \end{enumerate}
\end{definition}

Any logic circuit can be constructed with just these three operators: AND, OR, and NOT, they form a functionally complete set. Charles Sanders Peirce showed that NOR gates alone form a functionally complete set.


% above and below definition needs to be combined

\begin{definition}
    A \textbf{circuit} is a network that processes \textbf{discrete-valued variables}. A circuit can be viewed as a black box (Figure \ref{fig:black_box}) which has \textbf{inputs} and \textbf{outputs}. Inside the \textbf{black box} we will have specifications for how it will behave. Inside of this black box, circuits consist of
    \begin{enumerate}
        \item \textbf{elements}, these are circuits themselves with their own input, outputs, and specifications; and
        \item \textbf{nodes}, these are wires whose voltages represent the discrete-value variables and are classified as
        \begin{enumerate}
            \item \textbf{inputs}, nodes that receive values from the external world;
            \item \textbf{outputs}, nodes that deliver values to the external world; or
            \item \textbf{internal}, nodes that are neither inputs or outputs.
        \end{enumerate}
    \end{enumerate}
\end{definition}

\begin{figure}
    \centering
    \begin{tikzpicture}[scale=1]
        \draw % rectangle
            (0,0) rectangle (4,2)
        ;
        \draw[->] (-0.5,0.5) -- (0,0.5);
        \draw[->] (-0.5,1) -- (0,1);
        \draw[->] (-0.5,1.5) -- (0,1.5);
        \draw[->] (4,0.6) -- (4.5,0.6);
        \draw[->] (4,1.4) -- (4.5,1.4);
        \node[left] at (-0.5,1) {inputs};
        \node[right] at (4.5,1) {outputs};
    \end{tikzpicture}
    \caption{Circuit as a black box with inputs and outputs.}
    \label{fig:black_box}
\end{figure}

\section{Boolean equations}

The algebra of 0/1 variables... used for specifying the function of a combinatorional circuit and used to analyse and simply the circuits required to give a specified truth table.

Variables are represented by letters, eg. A,B,C.... The complement or inverse of a variable is written with a bar. 

A variable or its complement is called a literal. 

The AND of several literals is called a product or implicant. Products may be written A.B.C, ABC,  AcapBcapC or AlandBlandC. A minterm is a product involving all the inputs to a function.

The OR of several literals is called a sum or implicant, e.g. A+B+C, or A+C. Sums may be written A+B+C, AcupBcupC or AlorBlorC.

A maxterm is a sum involving all the inputs to a function.

\begin{description}
    \item[sum of products] every boolean expression can be written as minterms ORed together: (A.B.C)+(A.barBdotbarC)
\end{description}

% product of sum form and sum of products todo

\section{Boolean algebra}

% restructure

\section{Karnaugh maps}

Karnaugh maps are a graphical way of representing equations to make identifying terms in a truth table easier.

\begin{enumerate}
    \item Create the map so that neighbouring terms differ in the negation of one variable
    \item Circle all rectangular blocks of ones in the map using as few circles as possible
    \item each circle must be a power of 2 in each dimension (i.e. 1,2,4).
    \item Read off the implicants that were circled.
\end{enumerate}

If a Boolean expression is minimal then it is the sum of prime implicants: implicants that cannot be combined with each other. Each circle represents an implicant. Largest possible circles represent prime implicants.

\begin{example}
    \begin{table}
        \centering
        \begin{tabular}{cccc}
            \toprule
            A & B & C & Y \\
            \midrule
            0 & 0 & 0 & 1 \\
            0 & 0 & 1 & 1 \\
            0 & 1 & 0 & 0 \\
            0 & 1 & 1 & 0 \\
            1 & 0 & 0 & 0 \\
            1 & 0 & 1 & 0 \\
            1 & 1 & 0 & 0 \\
            1 & 1 & 1 & 0 \\
            \bottomrule
        \end{tabular}
        \caption{Caption}
        % \label{tab:my_label}
    \end{table}
    
    \begin{figure}
        \centering
        \begin{karnaugh-map}[4][2][1][$AB$][$C$]
            \manualterms{1,0,0,0,1,0,0,0}
            \implicant{0}{4}
        \end{karnaugh-map}
        \caption{Caption}
        % \label{fig:my_label}
    \end{figure}
\end{example}

\section{Circuits}

Combinatorial circuits, where output depends only on current input
\begin{enumerate}
    \item adders, add the contents of two register;
    \item decorders, use a binary number to activate a single line; and
    \item multiplexors, use a binary number to select an input.
\end{enumerate}

Sequential circuits, output depends on state and input, that is, the history of inputs
\begin{enumerate}
    \item latches / flip-flops, basic memory element.
\end{enumerate}

% figure out how to make logic diagrams in latex

\section{Timing and advanced adders}

In any physical gate or circuit there is a delay between the input changing and the output adjusting appropriately. 

\begin{remark}
    Output voltages do not change instantaneous between one level and another, there is a continuous change as it changes.
\end{remark}

\begin{definition}
    We define \textbf{propagation delay} $t_{pd}$ as the \textbf{maximum duration} before the output is stable after an input is changed.
\end{definition}

\begin{definition}
    We define the \textbf{contamination delay} $t_{cd}$ as the \textbf{minimum duration} before the output is stable after an input is changed.
\end{definition}

So we can bound the time for a given circuit to have stable outputs $T_c\in[t_{cd},t_{pd}]$. 

The delay between inputs and outputs is due to
\begin{enumerate}
    \item the capacitance and resistance in the circuit; and
    \item the speed of light limitation.
\end{enumerate}

$t_{pd}$ and $t_{cd}$ may be different because
\begin{enumerate}
    \item a circuit may have different inputs and outputs, some of which are faster than others; and
    \item a circuits speed is dependent on its temperature.
\end{enumerate}

\begin{definition}
    In a circuit, the \textbf{critical path} is the path determining the propagation delay of the circuit, in other words, the longest path in the circuit.
\end{definition}

\begin{definition}
    The \textbf{short path} is the path determining the contamination delay of the circuit, in other words, the shortest path in the circuit.
\end{definition}

\begin{example}
    Following are circuits with the critical path (red) and short path (blue) highlighted, with contamination delays and propagation delays given for each gate.
    \begin{enumerate}
        \item $\operatorname{tpd}{and}$
        \begin{center}
            \begin{circuitikz}
                \coordinate (O) at (0,0);
                \draw
                    (2,0) node[and port] (and) {}
                    (4,1) node[nor port] (nor) {}
                    (0,1.5) node[left] (A) {$A$} -| (and.in 1)
                    (O |- and.in 2) node[left] (B) {$B$} |- (and.in 2)
                    (nor.out) node[right] (S) {$S$}
                    (A) -| (nor.in 1)
                    (and.out) -| (nor.in 2)
                ;
                \draw[preaction={draw,red,-,double=red,double distance=2\pgflinewidth,}] % critical path
                    (B) |- (and.in 2)
                    (and.out) -| (nor.in 2)
                ;
                \draw[preaction={draw,blue,-,double=blue,double distance=2\pgflinewidth,}] % short path
                    (A) -| (nor.in 1)
                ;
            \end{circuitikz}
        \end{center}
        
        \item \hspace{0em}
        \begin{center}
            \begin{circuitikz}
                \coordinate (O) at (0,0);
                \draw
                    (2,2) node[and port] (and1) {}
                    (4,1) node[or port] (or) {}
                    (6,0) node[and port] (and2) {}
                    (and1.in 1) node[left] (A) {$A$}
                    (and1.in 2) node[left] (B) {$B$}
                    (and1.in 1 |- or.in 2) node[left] (C) {$C$} |- (or.in 2)
                    (and1.in 1 |- and2.in 2) node[left] (D) {$D$} |- (and2.in 2)
                    (and2.out) node[right] (S) {$S$}
                    (and1.out) |- (or.in 1)
                    (or.out) |- (and2.in 1)
                ;
                \draw[preaction={draw,red,-,double=red,double distance=2\pgflinewidth,}] % critical path
                    (and1.out) |- (or.in 1)
                    (or.out) |- (and2.in 1)
                ;
                \draw[preaction={draw,blue,-,double=blue,double distance=2\pgflinewidth,}] % short path
                    (D) -- (and2.in 2)
                ;
                
            \end{circuitikz}
        \end{center}
    \end{enumerate}
\end{example}

% draw logic circuit for ((A and B) or C) and D and show the critical path and short path for the circuit as example

\begin{definition}
    Sometimes the output of a circuit can temporarily move to an incorrect value before stabilising after inputs have been changed, we call this a \textbf{glitch} or a \textbf{hazard}.
    
    As long as we wait for propagation delay to elapse before we depend on the output, glitches are not an issue.
\end{definition}

% example of a glitch

% 4 multiplexor example, 3 designs with different delays.

\begin{example}
    The full adder circuit we have looked at takes 3 gate delays for the carry out to be computer.
\end{example}

