\chapter{MIPS}

\begin{definition}
    MIPS is a 32-bit RISC processor with approximately 80 instructions in the set, 32 general purpose registers \$r0-\$r31, The MIPS processor has a super-pipelined architecture, each instruction is broken down into a sequence of micro instructions.
\end{definition}

Design principles of MIPS:
\begin{enumerate}
    \item Simplicity favours regularity, consistent instruction format. There is a constant number if operants (two sources and one destination), this makes it easier to encode and handle in hardware.
    
    \item Make the common case fast, MIPS includes only simple, commonly,= used instructions. Hardware to decode and execute instructions can be simple, small, and fast. More complexxx instructions (that are less common) are performed using multiple simple instructions. This is a design principle for RISC proessors.
\end{enumerate}

Registers
\begin{table}
    \centering
    \begin{tabular}{ccc}
        \toprule
        Name & Register Number & Usage \\
        \midrule
        \$0 & 0 & the constant value 0 \\
        \$at & 1 & assembler temporary \\
        \$v0-\$v1 & 2-3 & function return values \\
        \$a0-\$a3 & 4-7 & function arguments \\
        \$t0-\$t7 & 8-15 & temporary variables \\
        \$s0-\$s7 & 16-23 & saved variables \\
        \$t8-\$t9 & 24-25 & more temporary variables \\
        \$k0-\$k1 & 26-27 & OS temporaries \\
        \$a0-\$a3 & 4-7 & function arguments \\
        \bottomrule
    \end{tabular}
    \caption{Registers used in MIPs.}
    \label{tab:mips_register}
\end{table}

There are 3 instruction formats:
\begin{enumerate}
    \item R-Type, register operands;
    \item I-Type, immediate operand; and
    \item J-Type, for jumping.
\end{enumerate}

\subsection{R-type instructions}
\begin{center}
    \begin{tabular}{cccccc}
        \toprule
            \texttt{op} & \texttt{rs} & \texttt{rt} & \texttt{rd} & \texttt{shamt} & \texttt{funct} \\
        \bottomrule
        \addlinespace
        6 bits & 5 bits & 5 bits & 5 bits & 5 bits & 6 bits \\
    \end{tabular}
\end{center}
R-type instruction have 3 register operands:
\begin{enumerate}
    \item \texttt{rs}, \texttt{rt}: source registers
    \item \texttt{rd}: desitination register.
\end{enumerate}
Other fields:
\begin{enumerate}
    \item \texttt{op}: the opeartion code (opcode);
    \item \texttt{funct}: the function, with opcode, tells the computer what operation to perform
    \item \texttt{shamt}: the shift amount for shift instructions, otherwise 0.
\end{enumerate}

\subsection{I-type instructions}

\begin{center}
    \begin{tabular}{cccc}
        \toprule
        \texttt{op} & \texttt{rs} & \texttt{rt} & \texttt{imm} \\
        \bottomrule
        \addlinespace
        6 bits & 5 bits & 5 bits & 16 bits \\
    \end{tabular}
\end{center}

This instruction type has 3 operands:
\begin{enumerate}
    \item \texttt{rs},\texttt{rt}: register operands;
    \item \texttt{imm}: 16-bit two's complement immediate.
\end{enumerate}
Other fields:
\begin{enumerate}
    \item \texttt{op}, the opcode, operation is completely determined by opcode.
\end{enumerate}

\begin{example}
    Following are examples of I-type instructions.
    \begin{enumerate}
        \item Add value in register 17 and 5 and put the answer in register 16.
        \begin{center}
            \ttfamily 
            addi \$s0, \$s1, 5
        \end{center}
        
        \item Load immediate value 5 and put in register 16.
        \begin{center}
            \ttfamily 
            li \$s0, 5
        \end{center}
    \end{enumerate}
\end{example}

\subsection{J-type insturctions}

\begin{center}
    \begin{tabular}{cc}
        \toprule
        \texttt{op} & \texttt{addr} \\
        \bottomrule
        \addlinespace
        6 bits & 26 bits
    \end{tabular}
\end{center}

This is rarely used in assembler.

\section{Addressing}

We can address our operands in four ways:
\begin{enumerate}
    \item register only;
    \item immediate (16-bit two's complement integer);
    \item base addressing, address of operand is given by $\text{base address }+\text{ signed immediate}$; and
    \item PC-relative, jumps so far from current position.
\end{enumerate}

OS calls

\begin{table}
    \centering
    \begin{tabular}{rcp{20em}}
        \toprule
        Service & Code (in \$v0) & Arguments / results \\
        \midrule
        \texttt{print\_int} & 1 & \texttt{\$a0}, integer to be printed \\
        \texttt{print\_float} & 2 & \texttt{\$f12}, float to be printed \\
        \texttt{print\_double} & 3 & \texttt{\$f12}, double to be printed \\
        \texttt{print\_string} & 4 & \texttt{\$a0}, address of string in memory \\
        \texttt{read\_int} & 5 & integer return in \$v0 \\
        \texttt{read\_float} & 6 & float returned in \$v0 \\
        \texttt{read\_string} & 7 & double return in \$v0 \\
        \texttt{read\_string} & 8 & \texttt{\$a0}, memory address of string input buffer;\newline\texttt{\$a1}, length of string buffer (n) \\
        \texttt{sbrk} & 9 & \texttt{\$a0}, amount, address in \$v0 \\
        \texttt{exit} & 10 \\
        \bottomrule
    \end{tabular}
    \caption{Caption}
    \label{tab:my_label}
\end{table}

\begin{definition}
    An architecture consists of an instruction set and implies an architectural state:
    \begin{enumerate}
        \item the state of all architecture defined registers; and
        \item the value of the program counter.
    \end{enumerate}
\end{definition}
