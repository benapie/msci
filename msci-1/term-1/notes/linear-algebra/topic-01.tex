\clearpage

\chapter{Vectors in $\mathbb R^n$}

\section{Vectors in $\R^n$}

\begin{definition}
    The space $\mathbb R^n$ is defined to be
    \[
        \mathbb R^n=
        \left\{
            \begin{pmatrix}
                x_1\\
                x_2\\
                \vdots\\
                x_n
            \end{pmatrix}
            : x_i \in \mathbb R
        \right\}    
        .
    \]
    The elements of $\mathbb R^n$ are called \textbf{vectors}, denoted as $\bm x$ ($\bm v$, etc.). 
\end{definition}


\begin{figure}
    \centering
    \begin{tikzpicture}
        \draw[help lines, color=gray!50, dashed] (-0.9,-0.9) grid (3.9,2.9);
        \draw[->,ultra thick] (-1,0) -- (4,0) node[right]{$x$};
        \draw[->,ultra thick] (0,-1) -- (0,3) node[above]{$y$};
        \filldraw (3,2) circle (2pt);
        \begin{scope}[very thick, decoration={markings, mark=at position 0.5 with {\arrow{>}}}] 
            \draw[postaction={decorate}] (0,0) -- (3,2);
        \end{scope}
    \end{tikzpicture}
    \caption{Visualisation of a vector.}
    \label{fig:vector_vis}
\end{figure}

We can visualise a vector in the following ways.

\begin{enumerate}
    \item As a point in $n$-dimensional space.
    \item An arrow pointing to that point. See Fig.\ref{fig:vector_vis}.
    \item An \emph{equivalence class} of arrows in the plane parallel and of the same length of 2.
\end{enumerate}

Adding structure to $\mathbb R^n$, there are \textbf{natural operations} on vectors:

\begin{enumerate}
    \item addition: is a function
    \[+:\mathbb R^n\times\mathbb R^n\to\mathbb R^n\]
    that is defined by
    \[
        \bm u+\bm v=
        \begin{pmatrix}
            u_1\\
            u_2\\
            \vdots\\
            u_n
        \end{pmatrix}
        +
        \begin{pmatrix}
            v_1\\
            v_2\\
            \vdots\\
            v_n
        \end{pmatrix}
        =
        \begin{pmatrix}
            u_1+v_1\\
            u_2+v_2\\
            \vdots\\
            u_n+v_n
        \end{pmatrix}
        ;
    \]
    
    \item scalar multiplication: is a function
    \[\cdot:\mathbb R\times\mathbb R^n\to\mathbb R^n\]
    that is defined by
    \[
        \lambda\bm u=\lambda
        \begin{pmatrix}
            u_1 \\
            u_2 \\
            \vdots \\
            u_n \\
        \end{pmatrix}
        =
        \begin{pmatrix}
            \lambda u_1 \\
            \lambda u_2 \\
            \vdots \\
            \lambda u_n \\
        \end{pmatrix}
        .
    \]
\end{enumerate}

Both of these come with a \textbf{geometric interpretation} which are more intuitive:

\begin{figure}
    \centering
    \begin{tikzpicture}
        \begin{scope}[very thick, decoration={markings, mark=at position 0.5 with {\arrow{>}}}] 
            \draw[postaction={decorate}] (0,0) -- node [left, xshift = -3pt] {$\underline u$} (1,3);
            \draw[postaction={decorate}] (1,3) -- node [above, yshift = 3pt] {$\underline v$} (5,4);
            \draw[postaction={decorate}] (0,0) -- node [below, yshift = -3pt] {$\underline v$} (4,1);
            \draw[postaction={decorate}] (4,1) -- node [right, xshift = 3pt] {$\underline u$} (5,4);
            \draw[postaction={decorate}] (0,0) -- node [above, xshift = -14pt] {$\underline u+\underline v$} (5,4);
        \end{scope}
    \end{tikzpicture}
    \caption{Visualisation of adding two vectors.}
    \label{fig:vector_add_vis}
\end{figure}

\begin{enumerate}
    \item see Fig.\ref{fig:vector_add_vis}; and
    \item scaling of a vector.
\end{enumerate}

\begin{proposition}
    \label{pro:axioms_of_real_vector_space}
    These operations satisfy the axioms of a real vector space: 
    \textbf{Axioms for addition}:
    \begin{enumerate}
        \item associativity: $\bf u(\bf v+\bf w)=(\bf u+\bf v)+\bf w$ for all $\bf u,\bf v,\bf w\in\mathbb R^n$;
        \item existence of additive identity: there is a vector $\bf 0$ in $\mathbb R^n$ so that $\bf v+\bf0=\bf0=\bf0+\bf v$ for all $\bf v$ in $\mathbb R^n$;
        \item existence of additive inverses: for each $\bf v$ in $\mathbb R^n$ there is an element in $\mathbb R^n$ such that $\bf v+(-\bf v)=\bf0=(-\bf v)+\bf v$;
        \item commutativity: $\bf v+\bf w=\bf w+\bf v$ for all $\bf v,\bf w\in\mathbb R^n$.
    \end{enumerate}
 
    \textbf{Axioms for multiplication}:
    \begin{enumerate}
        \item associativity: $(\lambda\mu)\bf v=\lambda(\mu\bf v)$ for $\lambda,\mu\in\mathbb R$ and $\bf v$ in $\mathbb R^n$;
        \item $0\bf v=\bf0$ for all $v\in\mathbb R^n$;
        \item $1\bf v=\bf v$ for all $v\in\mathbb R^n$;
        \item distributivity: $(\lambda+\mu)\bf v=\lambda\bf v+\mu\bf v$ and $\lambda(\bf v+\bf w)=\lambda\bf v+\lambda\bf w$ for $\lambda,\mu\in\mathbb R$ and $\bf v,\bf w\in\mathbb R^n$.
    \end{enumerate}
\end{proposition}

\begin{definition}
    We set
    \[
        \bf{e}_1=
        \begin{pmatrix}
            1\\
            0\\
            \vdots\\
            0\\
        \end{pmatrix}
        ,\bf{e}_2=
        \begin{pmatrix}
            0\\
            1\\
            \vdots\\
            0\\
        \end{pmatrix}
        ,\ldots,\bf{e}_n=
        \begin{pmatrix}
            0\\
            0\\
            \vdots\\
            1\\
        \end{pmatrix}
    \]
    and call these vectors the \textbf{standard basis vectors}.
\end{definition}

Note

\begin{align*}
    \begin{pmatrix}
        x_1\\
        x_2\\
        \vdots\\
        x_n
    \end{pmatrix}
    &=
    \begin{pmatrix}
        x_1\\
        0\\
        \vdots\\
        0
    \end{pmatrix}
    +
    \begin{pmatrix}
        0\\
        x_2\\
        \vdots\\
        0
    \end{pmatrix}
    +
    \ldots
    +
    \begin{pmatrix}
        0\\
        0\\
        \vdots\\
        x_n
    \end{pmatrix}\\
    &=x_1
    \begin{pmatrix}
        1\\
        0\\
        \vdots\\
        0
    \end{pmatrix}
    +
    x_2
    \begin{pmatrix}
        0\\
        1\\
        \vdots\\
        0
    \end{pmatrix}
    + \ldots + x_n
    \begin{pmatrix}
        0\\
        0\\
        \vdots\\
        1
    \end{pmatrix}\\
    &=x_1e_1+x_2e_2+\ldots+x_ne_n\\
    &=\sum^n_{i=1}{x_ie_i}.
\end{align*}

so that any vector can be expressed in terms of the standard basis vectors.


\section{The scalar product in $\mathbb R^n$}

\begin{definition}
    The \textbf{scalar} or \textbf{dot product} is a function that takes:
    \[\mathbb{R}^n\times\mathbb{R}^n\to\mathbb{R}\]
    defined as follows. For $\bm v,\bm w\in\mathbb{R}^n$ we define
    \[
        \bm v\cdot\bm w=
        \begin{pmatrix}
            v_1\\
            \vdots\\
            v_n\\
        \end{pmatrix}
        \cdot
        \begin{pmatrix}
            w_1\\
            \vdots\\
            w_n\\
        \end{pmatrix}
        =v_1w_1+\ldots+v_nw_n.
    \]
    This is a real number, a scalar.
\end{definition}

\begin{example}
    \[
        \begin{pmatrix}
            1\\
            2\\
            1\\
            2\\
        \end{pmatrix}
        \cdot
        \begin{pmatrix}
            1\\
            2\\
            3\\
            -4\\
        \end{pmatrix}
        =1+4+3-8=0
    \]
\end{example}

\begin{definition}
    Suppose
    \[
        \bf v=
        \begin{pmatrix}
            v_1\\
            \vdots\\
            v_n\\
        \end{pmatrix}
    \]
    then $\boldsymbol v\cdot \boldsymbol v=v_1^2+\ldots+v_n^2$. We define the \textbf{length} $|\bf v|$ of a vector by
    \[
        |\bf v|=\sqrt{\bf v\cdot\bf v}.
    \]
\end{definition}

A vector is \textbf{non-trivial} if it has a positive length. It is clear to see that the only \textbf{trivial} vector in $\mathbb R^n$ is $\bm 0$ (the zero vector).

\begin{example}
    \[
        \begin{vmatrix}
            \begin{pmatrix}
                -3\\4
            \end{pmatrix}
        \end{vmatrix}
        =
        \sqrt{
            \begin{pmatrix}
                -3\\4
            \end{pmatrix}
            \cdot
            \begin{pmatrix}
                -3\\4
            \end{pmatrix}
        }
        =\sqrt{9+16}=\sqrt{25}=5
    \]
\end{example}

\begin{lemma}
    Properties of the scalar product.
    \begin{enumerate}
        \item Symmetry
        
        $\bm u\cdot\bm v=\bm v\cdot\bm u$ for all $\bm u\,\bm v\in\mathbb R^n$;
        
        \item Linearity in the first factor:
        \begin{enumerate}
            \item $(\bm u+\bm v)\cdot\bm w=\bm u\cdot w+\bm v\cdot w$ for all $\bm u,\bm v,\bm w\in\mathbb R^n$;
            \item $(\lambda\bm u)\cdot v=\lambda(\bm u\cdot\bm v)$ for all $\lambda\in\mathbb R$ and all $\bm u,\bm v\in\mathbb R^n$;
        \end{enumerate}
        
        \item Linearity in the second factor:
        \begin{enumerate}
            \item $\bm u\cdot(\bm v+\bm w)=\bm u\cdot\bm w+\bm u+\bm v$ for all $\bm u,\bm v,\bm w\in\mathbb R^n$;
            \item $\bm u\cdot(\lambda\bm v)=\lambda(\bm u\cdot\bm v)$ for all $\lambda\in\mathbb R$ and all $\bm u,\bm v\in\mathbb R^n$.
        \end{enumerate}
        
        \item Positivity:
        
        $\bm v\cdot\bm v\geq0$ for all $\bm v$. $\bm v\cdot\bm v=0$ if and only if $\bm v=\bm0$.
    \end{enumerate}
\end{lemma}

\begin{definition}
    Let $\bm v$ be a non-trivial vector in $\mathbb R^2$ with length $|\bm v|=r$. Consider the right angled triangle with hypotenuse $\bm v$ and base on the $x$-axis. Let $\theta\in[0,2\pi)$ be the angle at origin $O$ measured in an anti-clockwise direction from the $x$-axis to $\bm v$. Using basic trigonometry, it is clear that
    \begin{align*}
        \cos\theta&=\dfrac xr,&\sin\theta&=\dfrac yr.
    \end{align*}
    We define the \textbf{polar coordinates} of $v$ to be $(r,\theta)\in\mathbb R_+\times[0,2\pi)$. In \textbf{cartesian} vector for then
    \[
        \bm v=r
        \begin{pmatrix}
            \cos\theta\\
            \sin\theta
        \end{pmatrix}.
    \]
    If $\bm v$ is trivial, $\theta$ is not defined.
\end{definition}

\begin{example}
    Find the polar coordinates of the following vectors.
    \begin{enumerate}
        \item
        \[
            \bm v=
            \begin{pmatrix}
                2\\
                3
            \end{pmatrix}
        \]
        then $|v|=\sqrt{2^2+3^2}=\sqrt{13}$ and $\theta=\arccos{\frac 2{\sqrt{13}}}=\arcsin{\frac 3{\sqrt{13}}}$. The polar coordinates are $(\sqrt{13}, \arccos{\frac2{\sqrt{13}}})$ where $\arccos:[-1,1]\to [-\frac\pi2,\frac\pi2]$.
        
        \item
        \[
            \bm v=
            \begin{pmatrix}
                3\\
                -3
            \end{pmatrix}
        \]
        then $|v|=\sqrt{3^2+3^2}=3\sqrt{2}$ and $\theta=\arccos{\sfrac1{\sqrt 2}}=\arcsin{-\sfrac1{\sqrt 2}}=\sfrac{7\pi}4$. The polar coordinates are $(3\sqrt 2, \sfrac{7\pi}4)$.
    \end{enumerate}
\end{example}

\begin{lemma}
    Suppose $\bm v,\bm w\in\mathbb R^2$ with $r=|\bm v|>0, s=|w|>0$. Let $\theta$ be the angle between $\bm v$ and $\bm w$, then
    \[
            \bm v\cdot\bm w=rs\cos\theta.
    \]
\end{lemma}

\begin{proof}
    Suppose
    \[
        \bm v=r
        \begin{pmatrix}
            \cos\phi\\
            \sin\phi
        \end{pmatrix}
        ,\bm w=s
        \begin{pmatrix}
            \cos\mu\\
            \sin\mu
        \end{pmatrix}
    \]
    then $\theta=|\phi-\mu|$.
    Now
    \begin{align*}
        \bm v\cdot\bm w&=rs\cos\phi\cos\mu+rs\sin\phi\sin\mu+rs\sin\phi\sin\mu\\
        &=rs(\cos\phi\cos\mu+\sin\phi\sin\mu)\\
        &=rs(\cos{(\phi-\mu)})\\
        &=rs\cos\theta
    \end{align*}
\end{proof}

\begin{example}
    Find the angle between the vectors
    \[
        \bm v=
        \begin{pmatrix}
            2\\
            2
        \end{pmatrix}
        ,\bm w=
        \begin{pmatrix}
            0\\
            1
        \end{pmatrix}
        .
    \]
    \begin{align*}
        \cos\theta&=\dfrac{\bm v\cdot\bm w}{|\bm v||\bm w|}\\
        &=\dfrac{(2)(0)+(2)(1)}{\sqrt{2^2+2^2}\sqrt{0^2+1^2}}\\
        &=\dfrac 1{\sqrt{2}}
    \end{align*}
    so
    \[
        \theta=\dfrac\pi 4.
    \]
\end{example}

\begin{example}
    Find the angle $\theta\in[0,\pi]$ between the vectors
    \[
        \bm v=
        \begin{pmatrix}
            1\\2\\1\\3
        \end{pmatrix}
        ,\bm w=
        \begin{pmatrix}
            1\\3\\-1\\3
        \end{pmatrix}.
    \]
    \begin{align*}
        \cos\theta&=\dfrac{\bm v\cdot\bm w}{|\bm v||\bm w|}\\
        &=\dfrac{1+6-1+9}{\sqrt{1+4+1+9}\sqrt{1+9+1+9}}\\
        &=\dfrac{15}{\sqrt{15}\sqrt{20}}\\
        &=\dfrac{15}{10\sqrt{3}}\\
        &=\dfrac{\sqrt 3}2
    \end{align*}
    Hence $\theta=\sfrac\pi6$.
\end{example}

\begin{theorem}[Cauchy-Schwarz Inequality]
    Suppose that $\bm v,\bm w\in\mathbb R^n$ then
    \[|\bm v||\bm w|\geq|\bm v\cdot\bm w|\]
    with equality if and only if $\bm v$ and $\bm w$ are colinear (one is a scalar multiple of the other).
\end{theorem}

Two vectors are said to be \textbf{orthogonal} or \textbf{perpendicular} if the angle between them is $\sfrac\pi2$. So the \textbf{non-trivial} vectors $\bm u$ and $\bm v$ are orthogonal if and only if
\[\bm u\cdot\bm v=0.\]

\begin{example}
    The vectors
    $
        \begin{pmatrix}
            1\\3\\-5
        \end{pmatrix}
    $
    and
    $
        \begin{pmatrix}
            4\\2\\2
        \end{pmatrix}
    $
    are orthogonal since
    $
        \begin{pmatrix}
            1\\3\\-5
        \end{pmatrix}
        \cdot
        \begin{pmatrix}
            4\\2\\2
        \end{pmatrix}
        =4+6-10=0
    $.
\end{example}

\begin{example}
    Let
    $
        \bm u=
        \begin{pmatrix}
            1\\1\\0
        \end{pmatrix}
        ,\bm v=
        \begin{pmatrix}
            1\\0\\1
        \end{pmatrix}
        \in\mathbb R^3.
    $
    Find all unit vectors in $\mathbb R^3$ that make an angle of $\frac\pi 4$ with both $\bm u$ and $\bm v$.
    
    Call solution
    \[
        \bm x=
        \begin{pmatrix}
            x_1\\x_2\\x_3
        \end{pmatrix}
    \]
    then
    \begin{align*}
        &\bm x\text{ is a unit vector}\\
        \iff&|\bm x|=1\\
        \iff&x_1^2+x_2^2+x_3^2=1\\
    \end{align*}
    and
    \begin{align*}
        \cos{\frac\pi 4}&=\frac1{\sqrt2}\\
        &=\dfrac{\bm u\cdot \bm x}{|\bm u||\bm x|}\\
        &=\dfrac{x_1+x_2}{\sqrt2}\iff x_1+x_2=1
    \end{align*},
    \begin{align*}
        \cos{\frac\pi 4}&=\frac1{\sqrt2}\\
        &=\dfrac{\bm v\cdot \bm x}{|\bm v||\bm x|}\\
        &=\dfrac{x_1+x_3}{\sqrt2}\iff x_1+x_3=1
    \end{align*}.
    So $x_2=x_3=1-x_1$, substituting this into $x_1^2+x_2^2+x_3^2=1$
    \begin{align*}
        (1-x_2)^2+x_2^2+x_2^2&=1\\
        3x_2^2-x_2+1&=1\\
        x_2(3x_2-2)&=0
    \end{align*}
    hence $x_2=0$ or $x_2=\frac23$, and the solutiosn are
    \begin{align*}
        \bm x&=
        \begin{pmatrix}
            1\\0\\0
        \end{pmatrix}
        ,&\bm x&=
        \begin{pmatrix}
            \sfrac13\\\sfrac23\\\sfrac23
        \end{pmatrix}
        .
    \end{align*}
\end{example}

\section{The vector product in $\mathbb R^3$}

\begin{definition}
    The cross product is a function
    \[\mathbb R^3\times\mathbb R^3\to\mathbb R^3\]
    defined by
    \[
        \begin{pmatrix}
            x_1\\x_2\\x_3
        \end{pmatrix}
        \times
        \begin{pmatrix}
            y_1\\y_2\\y_3
        \end{pmatrix}
        =
        \begin{pmatrix}
            x_2y_3-x_3y_2\\
            x_3y_1-x_1y_3\\
            x_1y_2-x_2y_1
        \end{pmatrix}
        .
    \]
    Note that the vector product is only defined for vectors in $\mathbb R^3$.
\end{definition}

\begin{lemma}
    Properties of the cross product.
    \begin{enumerate}
        \item Anti-symmetry:
        
        $\bm u\times\bm v=-\bm v\times\bm u$ for all $\bm u,\bm v\in\mathbb R^3$.
        
        \item Linear in the first factor, for all $\bm u,\bm v,\bm w\in\mathbb R^3,\lambda\in\mathbb R$:
        \begin{enumerate}
            \item $(\bm u+\bm v)\times\bm w=\bm u\times\bm w+\bm v\times\bm w$; and
            \item $(\lambda\bm u)\times\bm v=\lambda(\bm u\times\bm v)$.
        \end{enumerate}
        
        \item Linear in the second factor, for all $\bm u,\bm v,\bm w\in\mathbb R^3,\lambda\in\mathbb R$:
        \begin{enumerate}
            \item $\bm u\times(\bm v+\bm z)=\bm u\times\bm v+\bm u\times\bm w$
            \item $\bm u\times(\lambda\bm v)=\lambda(\bm u\times\bm v)$
        \end{enumerate}
        
        \item Orthogonality to the input
        
        $\bm u\cdot(\bm u\times\bm v)=\bm v\cdot(\bm u\times\bm v)=0$
    \end{enumerate}
\end{lemma}

\begin{example}
    \[
        \begin{pmatrix}
            1\\2\\3
        \end{pmatrix}
        \times
        \begin{pmatrix}
            4\\-2\\1
        \end{pmatrix}
        =
        \begin{pmatrix}
            (2)(1)-(-2)(3)\\
            (4)(3)-(1)(1)\\
            (1)(-2)-(4)(2)
        \end{pmatrix}
        =
        \begin{pmatrix}
            8\\11\\-10
        \end{pmatrix}
    \]
\end{example}

\begin{lemma}
    If $\bm u,\bm v\in\mathbb R^3$ then 
    \[|\bm u\times\bm v|=|\bm u||\bm v|\sin\theta\]
    where $\theta\in[0,\pi]$ is the angle between $\bm u$ and $\bm v$. Hence if $\bm u\times\bm v=\bm 0$ we can conclude that $\bm u$ and $\bm v$ are colinear.
\end{lemma}

\begin{proof}
    \begin{align*}
        |\bm u\times\bm v|^2&=(\bm u\times\bm v)\cdot(\bm u\times\bm v)\\
        &=(u_2v_3-u_3v_2)^2+(u_3v_1-u_1v_3)^2+(u_1v_2-u_2v_1)^2\\
        &=(u_1^2+u_2^2+u_3^2)(v_1^2+v_2^2+v_3^2)-(u_1v_1+u_2v_2+u_3v_3)^2\\
        &=|\bm u|^2|\bm v|^2-(\bm u\cdot\bm v)^2\\
        &=|\bm u|^2|\bm v|^2-|\bm u|^2|\bm v|^2\cos^2\theta\\
        &=|\bm u|^2|\bm v|^2(1-\cos^2\theta)\\
        &=|\bm u|^2|\bm v|^2\sin^2\theta\\
        &\implies|\bm u\times\bm v|=|\bm u||\bm v|\sin\theta
    \end{align*}
\end{proof}

\section{Planes in $\mathbb R$}

\begin{definition}
    A flat, two-dimensional surface which extends infinitely far is known as a \textbf{plane}. A plane is the two-dimensional analogue of a point (zero dimensions), a line (two dimensions), and a three dimensional space.
\end{definition}

We will look at two forms of defining planes: parametric form and using a normal vector.

\begin{definition}
    A plane $\Pi\in\mathbb R^3$ can  be described in \textbf{parametric form} as the collection of all points corresponding to the vectors of the form
    \[\bm a+\lambda\bm d_1+\mu\bm d_2.\]
    We can write
    \[\Pi=\{\bm a+\lambda\bm d_1+\mu\bm d_2:\lambda,\mu\in\mathbb R\}.\]
    Here $\bm a$ is any point on the plane $\Pi$ and $\bm d_1,\bm d_2$ are the direction vectors of the plane.
\end{definition}

\begin{definition}
    A plane $\Pi\in\mathbb R^3$ can be described in \textbf{point-normal form} as the collection of all points $\bm r$ satisfying
    \[(\bm r-\bm r_0)\cdot\bm n=0\]
    where $\bm n\neq0$ is a normal direction vector to the plane and $r_0$ is a point on the plane. 
    Suppose that
    \[
        \bm n=
        \begin{pmatrix}
            a\\b\\c
        \end{pmatrix},
        \bm r=
        \begin{pmatrix}
            x\\y\\z
        \end{pmatrix}
    \]
    then
    \begin{align*}
        (\bm r-\bm r_0)\cdot\bm n&=0\\
        \bm r\cdot\bm n&=\bm r_0\cdot\bm n\\
        ax+by+cz& =l
    \end{align*}
    where $l\in\mathbb R$ (for $l=0$ the plane will cross the origin). This gives us an alternative description of the plane $\Pi\in\mathbb R^3$:
    \[
        \Pi=\left\{
            \begin{pmatrix}
                x\\y\\z
            \end{pmatrix}
            \mid ax+by+cz=l
        \right\}.
    \]
\end{definition}

Starting with the equation
\[ax+by+cz=l\]
we can find a parametric description of the solution as follows. Given that $c\neq0$, we can solve for $z$ and obtain
\[z=\dfrac{l-ax-by}c,\]
so for every $x,y\in\mathbb R$ there is a unique $z$ such that
\[
    \begin{pmatrix}
        x\\y\\z
    \end{pmatrix}
    \in\Pi.
\]
The general solution to the equation is
\[
    \left\{
        \begin{pmatrix}
            0\\0\\\sfrac lc
        \end{pmatrix}
        +x
        \begin{pmatrix}
            1\\0\\\sfrac{-a}c
        \end{pmatrix}
        +y
        \begin{pmatrix}
            0\\1\\\sfrac{-b}c
        \end{pmatrix}
        \mid x,y\in\mathbb R
    \right\}.
\]
which is in parametric form.

\begin{example}
    \begin{enumerate}
        \item Find the equation of the plane passing through
        $\begin{pmatrix}
            1\\-1\\1
        \end{pmatrix}$
        with normal direction
        $\begin{pmatrix}
            1\\2\\3
        \end{pmatrix}$.
        \begin{align*}
            (\bm r-\bm r_0)\cdot\bm n&=0\\
            \bm r\cdot \bm n&=\bm a\cdot\bm n\\
            \bm r
            \cdot
            \begin{pmatrix}
                1\\2\\3
            \end{pmatrix}
            &=
            \begin{pmatrix}
                1\\-1\\1
            \end{pmatrix}
            \cdot
            \begin{pmatrix}
                1\\2\\3
            \end{pmatrix}
            \\
            x+2y+3z&=2
        \end{align*}
        where 
        $
            \bm r=
            \begin{pmatrix}
                x\\y\\z
            \end{pmatrix}
            .
        $
        
        
        \item Find a normal vector to and a point on the plane given by
        \[2x-3y+5z=12.\]
        A normal vector is
        $
            \begin{pmatrix}
                2\\-3\\5
            \end{pmatrix}
        $
        and a point on the plane is
        $
            \begin{pmatrix}
                6\\0\\0
            \end{pmatrix}
            .
        $
        
        \item Find the equation of the plane $\Pi$ passing through the points
        \[
            \bm a=
            \begin{pmatrix}
                1\\-1\\1
            \end{pmatrix}
            ,\bm b=
            \begin{pmatrix}
                2\\0\\3
            \end{pmatrix}
            ,\bm c=
            \begin{pmatrix}
                1\\3\\2
            \end{pmatrix}
            .
        \]
        We take our normal vector to be
        \begin{align*}
            (\bm a-\bm b)\times(\bm a-\bm c)&=
            \begin{pmatrix}
                -1\\-1\\-2
            \end{pmatrix}
            \times
            \begin{pmatrix}
                0\\-4\\-1
            \end{pmatrix}
            \\
            &=
            \begin{pmatrix}
                1-8\\-1-0\\4-0
            \end{pmatrix}
            \\
            &=
            \begin{pmatrix}
                -7\\-1\\4
            \end{pmatrix}
        \end{align*}
        and
        \[\bm a\cdot\bm n=-7+1+4=-2.\]
        Therefore, the equation which defines plane $\Pi$ is
        \[
            \bm r\cdot
            \begin{pmatrix}
                -7\\-1\\4
            \end{pmatrix}
            =-2.
        \]
    \end{enumerate}
\end{example}

\section{Lines in $\mathbb R^3$}

\begin{definition}
    We can define a line $L\in\mathbb R^3$ can be described in three ways:
    \begin{enumerate}
        \item in \textbf{parametric form}, let $\bm a$ be a point on $L$ and $\bm d$ be the direction vector for $L$:
        \[
            L=\left\{\bm a+\lambda\bm d\mid\lambda\in\mathbb R\right\};
        \]
        
        \item as the \textbf{intersection of two planes}, consider the equations
        \begin{alignat*}{4}
            ax&+by&+cy&=l\\
            bx&+ey&+fz&=m
        \end{alignat*}
        that define $\Pi_1$ and $\Pi_2$ respectively, therefore
        \[L=\Pi_1\cap\Pi_2\]
        and we can find the direction vector of our line as
        \[
            \bm d=
            \begin{pmatrix}
                a\\b\\c
            \end{pmatrix}
            \times
            \begin{pmatrix}
                d\\e\\f
            \end{pmatrix}
        \]
        and a point on the plane can be found somewhat ad-hoc (for now); and
        
        \item in a cheeky third description, given a parametric description of $L$, solve for the free variable $\lambda$, eg
        \[
            \begin{pmatrix}
                x\\y\\z
            \end{pmatrix}
            =
            \begin{pmatrix}
                a_1\\a_2\\a_3
            \end{pmatrix}
            +\lambda
            \begin{pmatrix}
                d_1\\d_2\\d_3
            \end{pmatrix}
        \]
        so
        \[
            \lambda=\dfrac{x-a_1}{d_1}=\dfrac{y-a_2}{d_2}=\dfrac{z-a_3}{d_3}
        \]
        for $d_1,d_2,d_3\neq0$, if one of the direction vectors is trivial then we just remove that term from the equation.
    \end{enumerate}
\end{definition}

\begin{example}
    Find a parametric description of the line given as the intersection of the two planes defined by
    \begin{align*}
        x+5z&=1,&2x-y-z&=3.
    \end{align*}
    First we find the direction vector for our line,
    \[
        \bm d=\bm n_1\times\bm n_2=
        \begin{pmatrix}
            1\\0\\5
        \end{pmatrix}
        \times
        \begin{pmatrix}
            2\\-1\\-1
        \end{pmatrix}
        =
        \begin{pmatrix}
            0+5\\10+1\\-1-0
        \end{pmatrix}
        =\begin{pmatrix}
            5\\11\\-1
        \end{pmatrix}
        ,
    \]
    and then we need some point on $L$, so we set $k=0$ then $x=1$ and
    \[2-y=3\iff y=-1,\]
    so
    \[
        L=
        \left\{
            \begin{pmatrix}
                1\\-1\\0
            \end{pmatrix}
            +\lambda
            \begin{pmatrix}
                5\\11\\-1
            \end{pmatrix}
            \mid\lambda\in\mathbb R
        \right\}.
    \]
\end{example}

\section{$3$ by $3$ linear systems and the triple product in $\mathbb R^3$}

Consider three linear equations in three variables, written as
\begin{align*}
    \bm n_1\cdot\bm x&=b_1\\
    \bm n_2\cdot\bm x&=b_2\\
    \bm n_3\cdot\bm x&=b_3
\end{align*}
with $\bm n_1,\bm n_2,\bm n_3\in\mathbb R^3$ and $b_1,b_2,b_3\in\mathbb R$. The solution set of the last two equations is a line $L$ with direction vector $\bm d=\bm n_2\times\bm n_3$ unless $\bm n_2$ and $\bm n_3$ are colinear, in which case the solution set is either a plane defined by either of the two equations or the empty set. This line $L$ intersects the plane $\Pi$ defined by the first equation at a unique point if it is not parallel to $\Pi$, that is, if $L$ is not orthogonal to $\bm n_1$. Therefore, we see that the system of linear equations has a unique solution if and only if
\[\bm n_1\cdot(\bm n_2\times\bm n_3)\neq0.\]

This does not mean, however, that if $\bm n_1\cdot(\bm n_2\times\bm n_3)=0$ that there is no solution. In this case, a solution could be a line, a plane, or no solution. 

\begin{definition}
    The \textbf{scalar triple product} is a function
    \[\mathbb R^3\times\mathbb R^3\times\mathbb R^3\to\mathbb R\]
    that, given three vectors $\bm a,\bm b,\bm c\in\mathbb R^3$, is defined by
    \[[\bm a,\bm b,\bm c]=\bm a\cdot(\bm b\times\bm c).\]
\end{definition}

\begin{lemma}
    For any $\bm a,\bm b,\bm c\in\mathbb R^3$ we have
    \[[\bm a,\bm b,\bm c]=[\bm b,\bm c,\bm a]=[\bm c,\bm a,\bm b]=-[\bm b,\bm a,\bm c]=-[\bm c,\bm b,\bm a]=-[\bm a,\bm c,\bm b].\]
\end{lemma}

\begin{proof}
    Set
    \[
        \bm a=
        \begin{pmatrix}
            a_1\\a_2\\a_3
        \end{pmatrix}
        ,\bm b=
        \begin{pmatrix}
            b_1\\b_2\\b_3
        \end{pmatrix}
        ,\bm c=
        \begin{pmatrix}
            c_1\\c_2\\c_3
        \end{pmatrix}
    \]
    then
    \begin{align*}
        [\bm a,\bm b,\bm c]&=
        \begin{pmatrix}
            a_1\\a_2\\a_3
        \end{pmatrix}
        \cdot
        \begin{pmatrix}
            b_2c_3-b_3c_2\\
            b_3c_1-b_1c_3\\
            b_1c_2-b_2c_1
        \end{pmatrix}
        \\
        &=a_1b_2c_3+a_2b_3c_1+a_3b_1c_2-a_1b_3c_2-a_2b_1c_3-a_3b_2c_1.
    \end{align*}
    It is clear that this expression remains unchanged if we cyclically permute $\bm a,\bm b,\bm c$ while we obtain a sign if we interchange two if the vectors.
\end{proof}

\begin{example}
    \begin{enumerate}
        \item 
        \[
            \left[
                \begin{pmatrix}
                    1\\1\\0
                \end{pmatrix}
                ,
                \begin{pmatrix}
                    1\\2\\3
                \end{pmatrix}
                ,
                \begin{pmatrix}
                    1\\1\\1
                \end{pmatrix}
            \right]=
            \begin{pmatrix}
                1\\1\\0
            \end{pmatrix}
            \cdot
            \begin{pmatrix}
                -1\\2\\-1
            \end{pmatrix}
            =1
        \]
        
        \item 
        \[
            \left[
                \begin{pmatrix}
                    1\\1\\1
                \end{pmatrix}
                ,
                \begin{pmatrix}
                    1\\1\\0
                \end{pmatrix}
                ,
                \begin{pmatrix}
                    1\\2\\3
                \end{pmatrix}
            \right]=
            \begin{pmatrix}
                1\\1\\1
            \end{pmatrix}
            \cdot
            \begin{pmatrix}
                3\\-3\\1
            \end{pmatrix}
            =1
        \]
    \end{enumerate}
\end{example}

