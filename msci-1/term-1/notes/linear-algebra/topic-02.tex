\chapter{Gauss-Jordan elimination}

\section{Solving equations and Gauss-Jordan elimination}

One method of systematically 

We consider a system of $M$ linear equations in $N$ unknowns $x_1,x_2,\ldots,x_n$.

\begin{alignat*}{5}
    a_{11}x_1&+a_{12}x_2&+\ldots&+a_{1n}x_n&=b_1\\
    a_{21}x_1&+a_{22}x_2&+\ldots&+a_{2n}x_n&=b_2\\
    \vdots && \vdots && \vdots \\
    a_{m1}x_1&+a_{m2}x_2&+\ldots&+a_{mn}x_n&=b_m\\
\end{alignat*}

Set 
\[
    \bm a_r=
    \begin{pmatrix}
        a_{r1}\\
        a_{r2}\\
        \vdots\\
        a_{rn}
    \end{pmatrix}
    ,\quad\bm x=
    \begin{pmatrix}
        x_1\\
        x_2\\
        \vdots\\
        x_n
    \end{pmatrix}
\]

and then we can write our system of linear equations as \[\bm a_r\cdot\bm x=b_r\qquad\text{for }r=1,2,\ldots,m\] or \[\sum_{i=1}^na_{ri}x_i=b_r\qquad\text{for }r=1,2,\ldots,m.\]

We can also write the system of equations as an augmented matrix.

\[
    \begin{pmatrix}[cccc|c]
        a_{11}&a_{12}&\ldots&a_{1n}&b_1\\
        a_{21}&a_{22}&\ldots&a_{2n}&b_2\\
        \vdots&\vdots&\vdots&\vdots&\vdots\\
        a_{m1}&a_{m2}&\ldots&a_{mn}&b_m\\
    \end{pmatrix}
\]

The left side of this augmented matrix $A$ is a $(n\times m)$ matrix known as the \textbf{coefficient matrix}. If we write 
\[
    \bm b=
    \begin{pmatrix}
        b_1\\b_2\\\vdots\\b_m
    \end{pmatrix}
\]
then we can write the augmented matrix as
\[
    \begin{pmatrix}[c|c]
        A&\bm b
    \end{pmatrix}
    .
\]

If $\bm b=\bm0$ then we call the system \textbf{homogeneous}, if $\bm b\neq\bm0$ then the system is \textbf{inhomogeneous}. All homogeneous systems have at least one solution, namely $\bm x=\bm0$.

\section{Row reduced echelon form}

\begin{definition}
    A matrix is said to be in row-reduced echelon form (RREF) if it satisfies the following:
    \begin{enumerate}
        \item in any non-zero row, the first non-zero entry is a 1 (called the leading 1);
        \item if a row has its leading 1 in the $r$th column then
        \begin{enumerate}
            \item all other entries in the $r$th column must be $0$; and
            \item the leading 1 of the subsequent row must occur to the right of the $r$th column;
        \end{enumerate}
        \item all zero rows come after all non-zero rows.
    \end{enumerate}
\end{definition}

\begin{example}
    The matrix
    \[
        \begin{pmatrix}
            0&0&1&\star&\star&0&0&\star&\star&0\\
            0&0&0&0&0&1&0&\star&\star&0\\
            0&0&0&0&0&0&1&\star&\star&0\\
            0&0&0&0&0&0&0&0&0&1\\
            0&0&0&0&0&0&0&0&0&0\\
            0&0&0&0&0&0&0&0&0&0\\
        \end{pmatrix}
    \]
    is in row reduced echelon form.
\end{example}

If an augmented matrix is in RREF then it is easy to tread off solutions to the associated system of linear equations.

\begin{example}
    Solve the linear system given by
    \[
        \begin{pmatrix}[cccccccc|c]
            0&1&2&0&0&3&0&7&2\\
            0&0&0&1&0&2&0&4&3\\
            0&0&0&0&1&-5&0&-2&5\\
            0&0&0&0&0&0&1&-1&7\\
            0&0&0&0&0&0&0&0&0\\
        \end{pmatrix}
        .
    \]
    This gives equations
    \[
        \linsys{
            x_2+x_3+3x_6+7x_8=2,
            x_4+2x_6+4x_8=3,
            x_5-5x_6-2x_8=5,
            x_7-x_8=7
        }
        .
    \]
    The unknowns not corresponding to leading $1$s are $x_1,x_3,x_6,x_8$ and we call these free variables. For any values of $x_1,x_3,x_6,x_8$, there is a unique solution to the system given by
    \[
        \begin{pmatrix}
            x_1\\x_2\\x_3\\x_4\\x_5\\x_6\\x_7\\x_8\\
        \end{pmatrix}
        =
        \begin{pmatrix}
            x_1\\
            2-2x_3-3x_6-7x_8\\
            x_3\\
            3-2x_6-4x_8\\
            5+5x_6+2x_8\\
            x_6\\
            7+x_8\\
            x_8\\
        \end{pmatrix}
    \]
\end{example}

\begin{example}
    Give a parametric description of the 3 dimensional space inside $\mathbb R^6$ defined by
    \[
        \begin{pmatrix}[cccccc|c]
            1&2&4&0&0&-1&3\\
            0&0&0&1&0&5&2\\
            0&0&0&0&1&-2&1\\
            0&0&0&0&0&0&0\\
        \end{pmatrix}
        .
    \]
    For any choice of $x_2,x_3,x_6\in\mathbb R$ there is a real solution of the form
    \begin{align*}
        \begin{pmatrix}
            x_1\\x_2\\x_3\\x_4\\x_5\\x_6\\
        \end{pmatrix}
        &=\begin{pmatrix}
            3-2x_2-4x_3+x_6\\
            x_2\\
            x_3\\
            2-5x_6\\
            1+2x_6\\
            x_6\\
        \end{pmatrix}
        \\
        &=\begin{pmatrix}
            3\\0\\0\\2\\1\\0\\
        \end{pmatrix}
        +x_2\begin{pmatrix}
            -2\\1\\0\\0\\0\\0\\
        \end{pmatrix}
        +x_3\begin{pmatrix}
            -4\\0\\0\\0\\0\\0\\
        \end{pmatrix}
        +x_6\begin{pmatrix}
            1\\0\\0\\-5\\2\\1
        \end{pmatrix}
    \end{align*}
    .
\end{example}

\section{Elementary row operations}

\begin{theorem}
    Any matrix can be brought into RREF by applying a finite sequence of \textbf{elementary row operations} as defined below.\label{the:rrefbyero}
\end{theorem}

\begin{definition}
    There are three \textbf{elementary row operations} (EROs):
    \begin{enumerate}
        \item $P_{rs}$, swap the $r$th row with the $s$th row;
        \item $M_r(\lambda)$, multiply the $r$th row by $\lambda\neq0,\lambda\in\mathbb R$; and
        \item $A_{rs}(\lambda)$, add $\lambda$ times the $r$th row to the $s$th row ($\lambda\neq0,\lambda\in\mathbb R$).
    \end{enumerate}
\end{definition}

\begin{remark}
    You may see ERO $A_{rs}$ without the argument $\lambda$ being present, in this case take $\lambda=1$.
\end{remark}

\begin{example}
    Bring the following matrix to RREF using EROs.
    \[
        \begin{pmatrix}
            1&3&2\\
            2&1&-1\\
            -1&1&2\\
        \end{pmatrix}
    \]
    \begin{align*}
        \begin{pmatrix}
            1&3&2\\
            2&1&-1\\
            -1&1&2\\
        \end{pmatrix}
        &\eroarrow{A_{12}(-2)}{}
        \begin{pmatrix}
            1&3&2\\
            0&-5&-5\\
            -1&1&2\\
        \end{pmatrix}
        \\
        &\eroarrow{A_{13}(1)}{}
        \begin{pmatrix}
            1&3&2\\
            0&-5&-5\\
            0&4&4\\
        \end{pmatrix}
        \\
        &\eroarrow{M_{2}(-\sfrac15)}{}
        \begin{pmatrix}
            1&3&2\\
            0&1&1\\
            0&4&4\\
        \end{pmatrix}
        \\
        &\eroarrow{A_{21}(-3)}{}
        \begin{pmatrix}
            1&0&-1\\
            0&1&1\\
            0&4&4\\
        \end{pmatrix}
        \\
        &\eroarrow{A_{23}(-4)}{}
        \begin{pmatrix}
            1&0&-1\\
            0&1&1\\
            0&0&0\\
        \end{pmatrix}
        \\
    \end{align*}
\end{example}

\begin{proof}[Proof of Theorem \ref{the:rrefbyero}]
    Firstly, we defined a \textbf{basic routine} to execute on the matrix:
    \begin{enumerate}
        \item start with a matrix $A$ and go along until you find the first non-zero column;
        \item apply $P_{1n}$ (if necessary) to make the first entry of this column non-zero;
        \item apply $M_1{\lambda}$ (if necessary) to make the first entry $1$; and
        \item use $A_{1r}(\lambda)$ as many times as necessary to make every other entry in the column $0$.
    \end{enumerate}
    The initial matrix is of the form
    \[
        \begin{pmatrix}
            \star&\star&\ldots&\star\\
            \star&\star&\ldots&\star\\
            \vdots&\vdots&\ddots&\vdots\\
            \star&\star&\ldots&\star\\
        \end{pmatrix}
    \]
    and after applying the basic routine is of the form
    \[
        \begin{pmatrix}
            1&\star&\ldots&\star\\
            0&\star&\ldots&\star\\
            \vdots&\vdots&\ddots&\vdots\\
            0&\star&\ldots&\star\\
        \end{pmatrix}
    \]
    or, if the first non-zero column is not the first one,
    \[
        \begin{pmatrix}
            0&\ldots&0&1&\star&\ldots&\star\\
            0&\ldots&0&0&\star&\ldots&\star\\
            \vdots&\ddots&\vdots&\vdots&\vdots&\ddots&\vdots\\
            0&\ldots&0&0&\star&\ldots&\star\\
        \end{pmatrix}
        .
    \]
    Now we can define an algorithm to convert any matrix into RREF using only EROs. We start with matrix $A$.
    \begin{enumerate}
        \item Apply the basic routine defined above to $A$, to get a new matrix $A_1$ of the form
        \[
            A_1=
            \begin{pmatrix}
                1&\star&\ldots&\star\\
                0&\star&\ldots&\star\\
                \vdots&\vdots&\ddots&\vdots\\
                0&\star&\ldots&\star\\
            \end{pmatrix}
        \]
        \item
        \begin{enumerate}
            \item Apply the basic routine to the submatrix of $A_1$ obtained from ignoring the first row. This gives a new matrix $\tilde{A_2}$ of the form
            \[
                \tilde{A}_2=
                \begin{pmatrix}
                    1&\star&\ldots&\star&\star&\star&\ldots&\star\\
                    0&0&\ldots&0&1&\star&\ldots&\star\\
                    0&0&\ldots&0&0&\star&\ldots&\star\\
                    \vdots&\vdots&\ddots&\vdots&\vdots&\vdots&\ddots&\vdots\\
                    0&0&\ldots&0&0&\star&\vdots&\star\\
                \end{pmatrix}
                .
            \]
            
            \item Apply the ERO $A_{21}(\lambda)$ to $\tilde{A}_2$ (if necessary) such that the entry above the first 1 in the second row is a zero. This gives a new matrix $A_2$ of the form
            \[
                A_2=
                \begin{pmatrix}
                    1&\star&\ldots&\star&0&\star&\ldots&\star\\
                    0&0&\ldots&0&1&\star&\ldots&\star\\
                    0&0&\ldots&0&0&\star&\ldots&\star\\
                    \vdots&\vdots&\ddots&\vdots&\vdots&\vdots&\ddots&\vdots\\
                    0&0&\ldots&0&0&\star&\vdots&\star\\
                \end{pmatrix}
                .
            \]
        \end{enumerate}
        \item Apply the last step inductively until the matrix $A_m$ is the reduced echelon form of $A$.
    \end{enumerate}
\end{proof}

\section{Using Gauss-Jordan elimination to solve linear equations}

Consider the system of $m$ linear equations in $n$ unknowns, defined by \[\sum_{i=1}^n{a_{ri}x_i}=b_r,\qquad1\leq r\leq m.\]

Two such systems are said to be \textbf{equivalent} if they share the same solution set. We can write the set of equations as an augmented matrix:

\[
    \begin{pmatrix}[cccc|c]
        a_{11}&a_{12}&\ldots&a_{1n}&b_1\\
        a_{21}&a_{22}&\ldots&a_{2n}&b_2\\
        \vdots&\vdots&\vdots&\ddots&\vdots\\
        a_{m1}&a_{m2}&\ldots&a_{mn}&b_m\\
    \end{pmatrix}
    .
\]

\begin{theorem}
    The ERO's change a linear system to an equivalent system, that is, ERO's do not change the solution set.
\end{theorem}

\begin{proof}
    Here we will consider each ERO and show that it will not change the solution set of a linear system.
    \begin{enumerate}
        \item Consider $P_{rs}$, the ERO that swaps two rows. The order of the equations clearly does not effect the solution set.
        
        \item Consider $M_r(\lambda)$, the ERO that multiples the $r$th row by $\lambda$. Looking at the $r$th row \[\sum_{i=1}^n{a_{ri}x_i}=b_r\] becomes
        \begin{alignat*}{3}
            &{}\sum_{i=1}^n{\left(\lambda a_{ri}x_i\right)}&{}=\lambda b_r\\
            \iff{}&\lambda\sum_{i=1}^n{\left(a_{ri}x_i\right)}&{}=\lambda b_r\\
            \iff{}&\sum_{i=1}^n{a_{ri}x_i}&=b_r\\
        \end{alignat*}
        when you apply the ERO and, therefore, does not change the solution set.
        
        \item Consider $A_{rs}(\lambda)$, the ERO that adds one row times by a factor to another row. For rows $r$ and $s$ we see that any solutions to
        \begin{align*}
            \sum_{i=1}^n{a_{ri}x_i}&=b_r,\\
            \sum_{i=1}^n{a_{si}x_i}&=b_s\\
        \end{align*}
        are clearly solutions to
        \begin{align*}
            \sum_{i=1}^n{a_{ri}x_i}&=b_r,\\
            \sum_{i=1}^n{(a_{si}+\lambda a_{ri})x_i}&=b_s+\lambda b_r.\\
        \end{align*}
    \end{enumerate}
\end{proof}

\begin{example}
    Find the intersection of the planes defined by
    \[
        \linsys{
            x_1-2x_2+x_3=3,
            x_2+2x_3=1
        }
        .
    \]
    Matrix form:
    \begin{align*}
        \begin{pmatrix}[ccc|c]
            1&-2&1&3\\
            0&1&2&1\\
        \end{pmatrix}
        &\eroarrow{A_{21}(2)}{}
        \begin{pmatrix}[ccc|c]
            1&0&5&5\\
            0&1&2&1\\
        \end{pmatrix}
        .
    \end{align*}
    Free variable is $x_3=\lambda$, this gives the solution set
    \[
        \left\{
            \begin{pmatrix}
                x_1\\x_2\\x_3\\
            \end{pmatrix}
            \in\mathbb R^3\middlest
            \begin{pmatrix}
                x_1\\x_2\\x_3\\
            \end{pmatrix}
            =
            \begin{pmatrix}
                5-5\lambda\\
                1-2\lambda\\
                \lambda\\
            \end{pmatrix}
        \right\}
        .
    \]
\end{example}

\begin{example}
    Find the unique solution to the following system of linear equations.
    \[
        \linsys{
            x_1+2x_2+3x_3=1,
            2x_1+3x_2+2x_3=1,
            3x_1+2x_2+x_3=1
        }
    \]
    \begin{align*}
        \begin{pmatrix}[ccc|c]
            1&2&3&1\\
            2&3&2&1\\
            3&2&1&1\\
        \end{pmatrix}
        &\eroarrow{A_{12}(-2)}{A_{13}(-3)}
        \begin{pmatrix}[ccc|c]
            1&2&3&1\\
            0&-1&-4&-1\\
            0&-4&-8&-2\\
        \end{pmatrix}
        \\
        &\eroarrow{M_2(-1)}{}
        \begin{pmatrix}[ccc|c]
            1&2&3&1\\
            0&1&4&1\\
            0&-4&-8&-2\\
        \end{pmatrix}
        \\
        &\eroarrow{A_{21}(-2)}{A_{23}(4)}
        \begin{pmatrix}[ccc|c]
            1&0&-5&-1\\
            0&1&4&1\\
            0&0&8&2\\
        \end{pmatrix}
        \\
        &\eroarrow{M_{3}(\sfrac18)}{}
        \begin{pmatrix}[ccc|c]
            1&0&-5&-1\\
            0&1&4&1\\
            0&0&1&\sfrac14\\
        \end{pmatrix}
        \\
        &\eroarrow{A_{31}(5)}{A_{32}(-4)}
        \begin{pmatrix}[ccc|c]
            1&0&0&\sfrac14\\
            0&1&0&0\\
            0&0&1&\sfrac14\\
        \end{pmatrix}
    \end{align*}
    The solution to this system is 
    \[
        \bm x=
        \begin{pmatrix}
            \sfrac14\\0\\\sfrac14
        \end{pmatrix}
        .
    \]
\end{example}

\begin{example}
    Solve the system
    \[
        \begin{pmatrix}[ccc|c]
            1&2&3&1\\
            2&2&2&1\\
            3&2&1&1\\
        \end{pmatrix}
        .
    \]
    \begin{align*}
        \begin{pmatrix}[ccc|c]
            1&2&3&1\\
            2&2&2&1\\
            3&2&1&1\\
        \end{pmatrix}
        &\eroarrow{A_{12}(-2)}{A_{13}(-3)}
        \begin{pmatrix}[ccc|c]
            1&2&3&1\\
            0&-2&-4&-1\\
            0&-4&-8&-2\\
        \end{pmatrix}
        \\
        &\eroarrow{M_{2}(-\sfrac12)}{}
        \begin{pmatrix}[ccc|c]
            1&2&3&1\\
            0&1&2&\sfrac12\\
            0&-4&-8&-2\\
        \end{pmatrix}
        \\
        &\eroarrow{A_{21}(-2)}{A_{23}(4)}
        \begin{pmatrix}[ccc|c]
            1&0&-1&0\\
            0&1&2&\sfrac12\\
            0&0&0&0\\
        \end{pmatrix}
        \\
    \end{align*}
    Solution set: 
    \[
        \left\{
            \begin{pmatrix}
                \lambda\\
                \sfrac12-2\lambda\\
                \lambda\\
            \end{pmatrix}
            \in\mathbb R^3\middlest\lambda\in\mathbb R
        \right\}
        .
    \]
\end{example}

\begin{example}
    Solve the system
    \[
        \linsys{
            x_1+2x_2+3x_3=1,
            2x_1+2x_2+2x_3=2,
            3x_1+2x_2+x_3=1
        }
        .
    \]
    \begin{align*}
        \begin{pmatrix}[ccc|c]
            1&2&3&1\\
            2&2&2&2\\
            3&2&1&1\\
        \end{pmatrix}
        &\eroarrow{A_{12}(-2)}{A_{13}(-3)}
        \begin{pmatrix}[ccc|c]
            1&2&3&1\\
            0&-2&-4&0\\
            0&-4&-8&-2\\
        \end{pmatrix}
        \\
        &\eroarrow{A_{21}}{A_{23}(2)}
        \begin{pmatrix}[ccc|c]
            1&0&-1&1\\
            0&-2&-4&0\\
            0&0&0&-2\\
        \end{pmatrix}
        \\
        &\eroarrow{M_{2}(-\sfrac12)}{}
        \begin{pmatrix}[ccc|c]
            1&0&-1&1\\
            0&1&2&0\\
            0&0&0&-2\\
        \end{pmatrix}
        \\
    \end{align*}
    However, there are no solutions to $(0)x_1+(0)x_2+(0)x_3=-2$, therefore, the system is \textbf{inconsistent}. In other words, the solution set is empty.
\end{example}

\begin{example}
    Give the solution set to the following linear system 
    \[
        \linsys{
            x_1+3x_2+2x_3=a,
            2x_1+x_2-x_3=b,
            -x_1+x_2+2x_3=c
        }
    \]
    where $a,b,c\in\mathbb R$.
    
    \begin{align*}
        \begin{pmatrix}[ccc|c]
            1&3&2&a\\
            2&1&-1&b\\
            -1&1&2&c\\
        \end{pmatrix}
        &\eroarrow{A_{12}(-2)}{A_{13}}
        \begin{pmatrix}[ccc|c]
            1&3&2&a\\
            0&-5&-5&b-2a\\
            0&4&4&a+c\\
        \end{pmatrix}
        \\
        &\eroarrow{M_5(-\sfrac15)}{}
        \begin{pmatrix}[ccc|c]
            1&3&2&a\\
            0&1&1&-\sfrac15(b-2a)\\
            0&4&4&a+c\\
        \end{pmatrix}
        \\
        &\eroarrow{A_{21}(-3)}{A_{23}(-4)}
        \begin{pmatrix}[ccc|c]
            1&0&-1&a+\sfrac35(b-2a)\\
            0&1&1&-\sfrac15(b-2a)\\
            0&0&0&a+c+\sfrac45(b-2a)\\
        \end{pmatrix}
        \\
    \end{align*}
    Here $x_3$ is a free variable so we set $x_3=\lambda$. 
    \begin{align*}
        x_1-\lambda&=a+\frac35(b-2a)\iff\\
        x_1&=-\frac15a+\frac35b+\lambda\\
        x_2+\lambda&=-\frac15(b-2a)\iff\\
        x_2&=-\frac15b+\frac25a-\lambda\\
    \end{align*}
    Solution set: 
    \[
        \left\{
            \begin{pmatrix}
                (-\sfrac15)a+(\sfrac35)b\\
                (-\sfrac15)b+(\sfrac25)a\\
                0\\
            \end{pmatrix}
            +\lambda
            \begin{pmatrix}
                1\\-1\\1
            \end{pmatrix}
            \middlest \lambda\in\mathbb R
        \right\}
        .
    \]
\end{example}
