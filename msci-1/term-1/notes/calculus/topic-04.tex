\chapter{Integration}

This chapter is lacking many topics, due to the assumed knowledge from previous studies, namely:
\begin{enumerate}
    \item standard integrals;
    \item integration by parts;
    \item integration by substitution; and
    \item integration by rational functions using partial fractions.
\end{enumerate}

\section{Indefinite integrals}

\begin{definition}
    A function $f(x)$ has an indefinite integral (anti-derivative) \[F(x)=\int{f(x)\,dx}\] in $(a,b)$ if \[F'(x)=f(x)\;\forall\;x\in(a,b).\]
\end{definition}

\begin{remark}
    $F(x)$ is unique up to the addition of a constant.
\end{remark}

\begin{definition}
    $f(x)$ is integrable in $(a,b)$ if the indefinite integral $F(x)$ is continuous in $[a,b]$.
\end{definition}

\begin{example}
    $f(x)=\frac1{x^2}$ is not integrable in $(0,1)$ because \[F(x)=\int\dfrac1{x^2}\,dx=\dfrac{-1}{x^2}+c\] is not continuous on $[0,1]$ (as it is not continuous at $x=0$).
\end{example}

\section{Definite integral}

Following are some (fairly obvious) properties for definite integrals. Let $f(x)$ and $g(x)$ both be functions that are integrable on $(a,b)$.

\begin{enumerate}
    \item $\int_a^b{\lambda f(x)+\mu g(x)\,dx}=\lambda\int^b_af(x)\,dx+\mu\int^b_ag(x)\,dx$;
    \item if $c\in(a,b)$ then $\int^b_a{f(x)\,dx}=\int^c_a{f(x)\,dx}+\int^b_c{f(x)\,dx}$;
    \item if $f(x)\geq g(x)\;\forall\;x\in[a,b]$ then $\int^b_a{f(x)\,dx}\geq\int^b_a{g(x)\,dx}$; and
    \item if $m\in f(x)\leq M\;\forall\; x\in[a,b]$ then $m(b-a)\in\int^b_a{f(x)\,dx}\leq M(b-a)$.
\end{enumerate}

\section{The fundamental theorem of calculus}

\begin{theorem}[Fundamental theorem of calculus]
    If $f(x)$ is continuous on $[a,b]$ then $F(x)=\int^x_0{f(t)\,dt}$ is differentiable on $(a,b)$ with $F'(x)=f(x)$, that is, \[F'(x)=\dfrac{d}{dx}\left(\int^x_0f(t)\,dt\right)=f(x).\] Furthermore, if $\tilde F(x)$ is any indefinite integral of $f(x)$ then \[\int^b_a{f(x)\,dx}=\tilde F(b)-\tilde F(a).\]
\end{theorem}

\begin{example}
    Following are examples using the above theorem.
    \begin{enumerate}
        \item \[\dfrac d{dx}\int^x_0{e^t\sin{t}\,dt=e^x\sin{x}};\]
        \item \[\dfrac d{dx}\int^{x^2}_3{e^t\sin{t}}=(e^{x^2}\sin{x^2})(2x);\text{ and}\]
        \item \[\dfrac d{dx}\int^x_\pi t\sin{t}\,dt=t\sin{t}.\]
    \end{enumerate}
\end{example}

\section{Logs and exponentials revisited}

Following are important results that can be used regarding limits:

\begin{enumerate}
    \item powers beat logs, \[\lim_{x\to\infty}\left(\dfrac{\log{x}}{x^n}\right);\]
    \item exponentials beat powers, \[\lim_{x\to\infty}\left(\dfrac{x^n}{e^x}\right);\text{ and}\]
    \item \[\lim_{x\to\infty}\left(1+\dfrac ax\right)^x=e^a.\]
\end{enumerate}

\begin{lemma}
    For $x\neq 0$, \[e^x\geq 1+x.\]
\end{lemma}

\begin{proof}
    Let $f(x)=e^x-(1+x)$ where $f(0)=0$. As $f'(x)=e^x-1\geq0$, $f(x)$ is increasing in $[0,\infty)$. That is $f(x)\geq0\;\forall\;x\geq0$ and thus the lemma holds.
\end{proof}

\begin{lemma}
    For $x\geq0$ and $n\in\mathbb N$ \[e^x\geq\sum_{j=0}^n{\dfrac{x^j}{j}}.\]
\end{lemma}

% not proved but can be done with induction with reasoning similar to that of the lemma above.

% Here we will prove the important results stated earlier.

% \begin{enumerate}
%     \item Here we can use a substitution of $x=e^y$ to get
%     \[\lim_{x\to\infty}{\left(\dfrac{\log{x}}{x^n}\right)}=\lim_{y\to\infty}{\left(\dfrac y{e^{ny}}\right)}\]
%     and using the above lemmas \[0\leq\dfrac y{e^{ny}}\leq\dfrac y{1+ny+\frac12(ny)^2}\leq\dfrac y{\frac12(ny)^2}\]
% \end{enumerate}

\section{Integration using a recurrence relation}

\begin{example}
    Calculate \[\int_0^1x^2e^x\,dx.\]
    
    Let \[I_n=\int_0^1x^ne^x\,dx\] for integer $n\geq0$ and then
    \begin{align*}
        I_{n+1}&=\int_0^1x^{n+1}e^x\,dx\\
        &=(x^{n+1}e^x)\Big|^1_0-\int_0^1(n+1)x^{n}e^x\,dx\\
        &=e-(n+1)I_n.
    \end{align*}
    Therefore,
    \begin{align*}
        I_0&=\int_0^1e^x\,dx=e-1\\
        I_1&=e-I_0=1\\
        I_2&=e-2I_1=e-2,
    \end{align*}
    and \[\int_0^1x^2e^x\,dx=e-2.\]
\end{example}

\begin{example}
    Calculate \[\int\tan^4{(x)}\,dx.\]
    
    Let \[F_n(x)=\int\tan^n{x}\] and note that
    \begin{align*}
        \int\tan^n{x}\sec^2{x}\,dx&=\frac1{n+1}\tan^{n+1}\\
        &=\int\tan^n{x}(1+\tan^2{x})\,dx\\
        &=F_n(x)+F_{n+2}(x),
    \end{align*}
    and so \[F_{n+2}(x)=\frac1{n+1}\tan^{n+1}(x)-F_n(x).\] Therefore,
    \begin{align*}
        F_0(x)&=\int\,dx=x\\
        F_2(x)&=\tan{x} - x\\
        F_4(x)&=\frac13\tan^3{x}-\tan{x}+x+c.
    \end{align*}
\end{example}

\section{Definite integrals using odd and even functions}

The \textbf{oddness} or \textbf{eveness} of a function $f(x)$ can be useful for integrals of the form \[\int^a_{-a}f(x)\,dx.\]

\begin{theorem}
    For any odd function $f_{\text{odd}}(x)$ \[ \int^{a}_{-a} f_{\text{odd}}(x) \, dx = 0. \]
\end{theorem}

\begin{proof}
    \begin{align*}
        \int^{a}_{-a} f_{\text{odd}}(x) \, dx &= \int^{a}_{0} f_{\text{odd}}(x) \, dx + \int^{0}_{-a} f_{\text{odd}}(x) \, dx \\
        &=\int^{a}_{0}f_{\text{odd}}(x)\,dx-\int^{0}_{a}f_{\text{odd}}(-x)\,dx\\
        &=\int^{a}_{0}f_{\text{odd}}(x)\,dx+\int^{0}_{a}f_{\text{odd}}(x)\,dx\\
        &=\int^{a}_{0}f_{\text{odd}}(x)\,dx-\int^{a}_{0}f_{\text{odd}}(x)\,dx\\
        &=0
    \end{align*}
\end{proof}

\begin{theorem}
    For an even function $f_{\text{even}}(x)$ \[\int^a_{-a}f_{\text{even}}(x)\,dx=2\int^a_0f_{\text{even}}(x)\,dx\]
\end{theorem}

\begin{proof}
    \begin{align*}
        \int^a_{-a}f_{\text{even}}(x)\,dx&=\int^a_0f_{\text{even}}(x)\,dx+\int^0_{-a}f_{\text{even}}(x)\,dx\\
        &=\int^a_0f_{\text{even}}(x)\,dx-\int^0_{a}f_{\text{even}}(-x)\,dx\\
        &=\int^a_0f_{\text{even}}(x)\,dx+\int^a_0f_{\text{even}}(-x)\,dx\\
        &=2\int^a_0f_{\text{even}}(x)\,dx
    \end{align*}
\end{proof}

\begin{theorem}
    For any function $f(x)=f_{\text{even}}(x)+f_{\text{odd}}(x)$ \[\int^a_{-a}f(x)\,dx=2\int^a_0f_{\text{even}}(x)\,dx.\]
\end{theorem}

\begin{example}
    Calculate \[\int_{-1}^1\left(\dfrac{e^xx^4}{\cosh{x}}\right)\,dx.\]
    
    Let \[f(x)=\dfrac{e^xx^4}{\cosh{x}}\] and as $f_1(x)=\cosh{x}$ and $f_2(x)=x^4$ are both even functions then \[f_{\text{even}}(x)=\dfrac{x^4}{\cosh{x}}\left(\frac12(e^x+e^{-x})\right)=x^4,\] therefore, \[\int^1_{-1}f(x)\,dx=2\int^1_0x^4\,dx=\frac25x^5\Big|^1_0=\frac25.\]
\end{example}
