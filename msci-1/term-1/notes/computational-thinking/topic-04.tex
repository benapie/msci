\chapter{Types of problems}

In computer science we have the following central notions for solutions:
\begin{enumerate}
    \item computation (how its done);
    \item resources needed; and
    \item correctness.
\end{enumerate}

\begin{definition}[Abstraction]
    \textbf{Abstraction} is the process of removing unnecessary details from a situation in order to make it easier to solve.
\end{definition}

When we solve real problems with computers, we are actually solving abstractions that mirror reality. 

\begin{definition}[Decision problems]
    \textbf{Decision problems} consist of
    \begin{enumerate}
        \item a set of instances $I$;
        \item a set of yes-instances $Y \subset I$.
    \end{enumerate}
\end{definition}

\begin{definition}[Search problem]
    A \textbf{search problem} consists of a binary search relation $R \subset I \times J$ where $I$ is a set of instances and $J$ is a set of solutions.
\end{definition}

\begin{example}[Prime factoring]
    Prime factoring can be represented by the relation \[ R = \{(4, 2), (6, 2), (6, 3), (8, 2), (8, 4), (9, 3), (10, 2), (10, 5), \ldots\}. \]
\end{example}

\begin{definition}[Optimisation problem]
    A \textbf{optimisation problem} consists of:
    \begin{enumerate}
        \item a set of instances $I$;
        \item for each instance, a set of feasible solutions $f(x)$;
        \item a measure for every solution to an instance; and
        \item a goal being minimisation or maximisation.
    \end{enumerate}
\end{definition}
