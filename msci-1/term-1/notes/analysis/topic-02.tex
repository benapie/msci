\chapter{Numbers and inequalities}

\begin{definition}
    There are special sets that hold significant mathematical importance. Some special sets are:
    \begin{enumerate}
        \item $\mathbb N=\{1,2,3,4,\ldots\}$ the set of natural numbers, it is closed under addition and multiplication;
        \item $\mathbb Z=\{\ldots,-2,-1,0,1,2,\ldots\}$ the set of integers, it is closed under addition, subtraction, and multiplication;
        \item $\mathbb Q=\{\frac pq\mid p\in\mathbb Z,q\in\mathbb N\}$ the set of rational numbers, it is closed under addition, subtraction, multiplication, and division but it is not complete; 
        \item $\mathbb R$ is the set of real numbers (we will not define them), it is complete (the concept of completeness will be touched on later); and
        \item $\mathbb C=\{a+bi\mid a,b\in\mathbb R,i^2=-1\}$ is the set of complex numbers.
    \end{enumerate}
    It is important to note:
    \[\mathbb N\subset\mathbb Z\subset\mathbb Q\subset\mathbb R\subset\mathbb C\]
\end{definition}

$\mathbb R$ is an \textbf{ordered field}, and the axioms are as follows.

\begin{proposition}
    \textbf{Axioms of ordering for $\mathbb R$}:
    oet $a,b,c\in\mathbb R$, then:
    \begin{enumerate}
        \item trichotomy, either $a<b$, $a=b$, or $a>b$;
        \item addition law, $a<b\iff a+c<b+c$;
        \item multiplication law, if $c>0$ then  $a<b\iff ac<bc$; and
        \item transitivity, if $a<b$ and $b<c$ then $a<c$.
    \end{enumerate}
\end{proposition}

\begin{remark}
    The complex numbers $\mathbb C$ are not ordered. It does make sense to write $z_1<z_2$ for $z_1,z_2\in\mathbb C$.
\end{remark}

\begin{example}
    Following are examples of solving inequalities.
    \begin{enumerate}
        \item Find all $x\in\mathbb R$ satisfying \[5<3x+10\leq16.\]
        This means
        \[5<3x+10\qquad\text{and}\qquad 3x+10\leq16\]
        \begin{alignat*}{3}
            \iff-5&<3x&\leq6\\
            \iff-\frac53&<x&\leq2,
        \end{alignat*}
        therefore
        \[x\in\left(-\frac53,2\right].\]
        
        \item Solve \[\dfrac{2x+1}{x+3}>3.\]
        \begin{enumerate}
            \item Case 1, $x+3>0$:
            \begin{align*}
                \dfrac{2x+1}{x+3}&>3,\\
                2x+1&>3x+9,\\
                x&<-8.
            \end{align*}
            \item Case 2, $x+3<0$:
            \begin{align*}
                \dfrac{2x+1}{x+3}&>3,\\
                2x+1&>3x+9,\\
                x&>-8.\\
            \end{align*}
        \end{enumerate}
        Therefore, $x\in(-8,-3)$.
        
        \item Solve \[2x^2>x+6.\]
        \begin{align*}
            2x^2-x-6&>0,\\
            (2x+3)(x-2)&>0.
        \end{align*}
        Therefore
        \[x\in\mathbb R\setminus\left[-\frac32,2\right].\]
        
        \item Find all $x\in R$ so that \[-3(4-x)\leq12\]
        \begin{align*}
            -3(4-x)&\leq12,\\
            4-x&\geq-4,\\
            8&\geq x.
        \end{align*}
        
        \item Solve \[\dfrac{x+2}{3}<\dfrac{5-2x}{4}.\]
        \begin{align*}
            \dfrac{x+2}{3}<\dfrac{5-2x}{4}\\
            4x+8<15-6x\\
            10x<7\\
            x<\dfrac7{10}
        \end{align*}
        
        \item Solve $x^2-4x+3>0$.
        \begin{align*}
            (x-3)(x-1)>0\\
            (x>3\tand x>1)\tor(x<3\tand x<1)\\
            x>3\tor x<1
        \end{align*}
        
        \item Solve\[\dfrac{3}{x-2}\leq x.\]
        \begin{align*}
            \dfrac{3}{x-2}-x&\leq0,\\
            \dfrac{-x(x-2)+3}{x-2}&\leq0,\\
            \dfrac{x^2-2x-3}{x-2}&\geq0,\\
            \dfrac{(x-3)(x+1)}{(x-2)}&\geq0,\\
            (x\geq3)\tor(-1&\leq x<2).
        \end{align*}
    \end{enumerate}
\end{example}

\begin{definition}
    The absolute value of a complex number $z=x+iy$ is defined to be \[|z|=|x+iy|=\sqrt{x^2+y^2}.\] In particular, if $x$ is real then
    \[
        |x|=
        \begin{cases}
            x & x\geq0\\
            -x & x<0
        \end{cases}
        .
    \]
\end{definition}

Following are techniques that can be used when solving inequalities involving absolute value.

\begin{enumerate}
    \item Triangle inequality, if $z_1,z_2\in\mathbb C$ then \[|z_1+z_2|\leq|z_1|+|z_2|.\]
    \item A variant on (i) \[||z_1|-|z_2||\leq|z_1+z_2|.\]
    \item For all $z\in\mathbb C$, \[|-z|=|z|.\]
    \item The statement $|x|\leq c$ for $x\in\mathbb R$, with some fixed position $c$,
    \begin{align*}
        \iff&-c\leq x\leq c\\
        \iff&(x\leq c)\tand(x\geq c).
    \end{align*}
    Similar equivalences exist for $|x|\geq c$.
\end{enumerate}

\begin{example}
    Following are examples of solving inequalities with absolute values.
    \begin{enumerate}
        \item Solve \[|3x-4|\leq2.\]
        \begin{alignat*}{3}
            -2&\leq3x-4&\leq2\\
            2&\leq3x&\leq6\\
            \frac23&\leq x&\leq2
        \end{alignat*}
        
        \item Solve \[|2x+3|>5\]
        \begin{align*}
            (2x+3>5)&\tor(2x+3<-5)\\
            \iff (x>1)&\tor(x<-4)
        \end{align*}
        
        \item Solve \[|x+2|\leq|2x-1|.\]
        \begin{align*}
            \iff |x+2|^2&\leq|2x-1|^2\\
            \iff (x+2)^2&\leq(2x-1)^2\\
            \iff x^2+4x+4&\leq4x^2-4x+1\\
            \iff 3x^2-8x-3&\geq0\\
            \iff (3x+1)(x-3)&\geq0\\
            \implies(x\leq-\frac13)\tor(x\geq3)
        \end{align*}
    \end{enumerate}
\end{example}
