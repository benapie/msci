\chapter{Recursion}

\begin{definition}
    The factorial of a positive integer $n$, denoted $n!$, is the product of all integers from $1$ to $n$. More formally, \[ n! = \prod^{n}_{k = 1} k = n \cdot (n - 1) \cdot \ldots \cdot 2 \cdot 1. \]
\end{definition}

We can implement a function that calculates the factorial of a number either iteratively (Listing \ref{lst:fact_iter}) or recursively (Listing \ref{lst:fact_recu}).

\begin{lstlisting}[float,
                   language = Python,
                   caption = {A function which calculates the factorial using an iterative method in Python.},
                   label = {lst:fact_iter}]
def factorial(n):
    total = 1
    for i in range(1, n):
        total = total * (i+1)
    return total
\end{lstlisting}

\begin{lstlisting}[float,
                   language = Python,
                   caption = {A function which calculates the factorial using a recursive method in Python.},
                   label = {lst:fact_recu}]
def factorial(n):
    if n == 0:
        return 1
    return n * factorial(n-1)
\end{lstlisting}

\begin{definition}
    We can define an algorithm as \textbf{recursive} if it satisfies the following conditions:
    \begin{enumerate}
        \item it must have a base case;
        \item it must change state and move towards the base case; and
        \item it must call itself.
    \end{enumerate}
\end{definition}

\begin{definition}
    We can use \textbf{backtracking} to solve problems that have many possible solutions, but too many to exhaustively test. In essence, it involves building up a solution one step at a time and backtracking when unable to continue.
\end{definition}

Listing \ref{alg:backtracking_generic} shows a generic algorithm for using backtracking to find the solution to a problem.

\begin{algorithm}
    \caption{Generic algorithm for backtracking recursive technique.}
    \label{alg:backtracking_generic}
    \begin{algorithmic}
        \Procedure{ExtendSolution}{current solution}
            \If{current solution is valid}
                \If{current solution is complete}
                    \State \Return current solution
                \Else
                    \For{each extension of the current solution}
                        ExtendSolution(extension)
                    \EndFor
                \EndIf
            \EndIf
        \EndProcedure
    \end{algorithmic}
\end{algorithm}

\begin{example}
    A \textbf{knight's tour} is a sequence of moves of a knight on a chessboard such that the knight visits every square only once. We can calculate a knights tour on any size chessboard using a recursive backtracking technique shown in \ref{alg:knights_tour}, this follows the generic algorithm specified above.
\end{example}

\begin{algorithm}
    \caption{Knight's tour algorithm.}
    \label{alg:knights_tour}
    \begin{algorithmic}[H]
        \Procedure{ExtendSolution}{current solution}
            \If{new move is to unvisited square on board}
                \If{every square has been visited}
                    \State \Return current solution
                \Else
                    \For{each possible move}
                        \State ExtendSolution(with new move)
                    \EndFor
                \EndIf
            \EndIf
        \EndProcedure
    \end{algorithmic}
\end{algorithm}

\begin{example}
    The \textbf{eight queens puzzle} is the problem of placing $8$ queens on a $8\times 8$ chessboard such that no two queens threaten each other (according to the rules of chess). Again, using the generic template we get the solution shown in Algorithm \ref{alg:queens_tour} to solve this problem.
\end{example}

\begin{algorithm}
    \caption{Knight's tour algorithm.}
    \label{alg:queens_tour}
    \begin{algorithmic}[H]
        \Procedure{ExtendSolution}{current solution}
            \If{new queen is not under attack}
                \If{8 queens are on the board}
                    \State \Return current solution
                \Else
                    \For{each of eight squares in the next row}
                        \State ExtendSolution(with extra queen)
                    \EndFor
                \EndIf
            \EndIf
        \EndProcedure
    \end{algorithmic}
\end{algorithm}
