\chapter{Introduction}

\begin{definition}[Database]
    A \textbf{database} is a collection of logically related data, designed to meet the needs of an organisation.
\end{definition}

A database is a single repository of data, which may be shared by many users. All data is integrated with minimum amount of duplication and in larger databases we may have a data dictionary to store metadata about the data.

\begin{definition}[Database management system]
    A \textbf{database management system} (DBMS) is a system software that allows users to define, create, maintain, and control the access to the database. DBMS have two basic features:
    
    \begin{enumerate}
        \item data definition language (DDL), this allows users to define the database (specify data types, the structure, and constraints on the data);
        
        \item data manipulation language (DML), this allows users to insert, update, delete, and retrieve the data from the database.
    \end{enumerate}
\end{definition}

A DBMS offers \textbf{controlled access} to the database, such that

\begin{enumerate}
    \item the system is secure by preventing any unauthorised access;
    
    \item allows concurrency of use such that multiple users can access the database at once; and
    
    \item recovery control, in the event of data loss are malicious attack data can be rolled back.
\end{enumerate}

\begin{definition}[Database application program]
    A \textbf{database application program} is a computer program that interacts with the user and the DBMS.
\end{definition}

\begin{example}
    Structured query language (SQL) is a common DDL and DML that is used by database application program to communicate with the DBMS.
\end{example}

\begin{definition}[Components of a DBMS environment]
    Following are components you may find in a DBMS environment.
    
    \begin{enumerate}
        \item Hardware, can range from a PC to a network of computers.
        
        \item Software, database management system, operating system, network software, and also the application programs. 
        
        \item Data, the data used by the organisation and an abstract description of this data called the \textbf{schema} of the database.
        
        \item Procedures, documented instructions on how to use and run the system.
        
        \item People, the people involved with the system in any way.
    \end{enumerate}
\end{definition}

\begin{definition}[Roles in a DBMS environment]
    There are a number of roles that people may take in database management system environment:
    \begin{enumerate}
        \item database designers, which can be split into
        \begin{enumerate}
            \item logical designers, they consider what data to store;
            \item physical designers, they consider how the data is stored;
        \end{enumerate}
        \item database administrator, they maintain the database and overview the security and integrity of the database;
        \item application developers; and
        \item end users.
    \end{enumerate}
\end{definition}

Advantages of databases:
\begin{multicols}{2}
    \begin{enumerate}
        \item allows the sharing of data;
        \item control of redundancy;
        \item data consistency;
        \item data security and integrity;
        \item faster development of new applications;
        \item better data accessibility;
        \item economy of scale;
        \item control of concurrency; and
        \item better backup and recovery procedures.
    \end{enumerate}
\end{multicols}

Disadvantages of databases:
\begin{multicols}{2}
    \begin{enumerate}
        \item complexity and size of the system;
        \item higher cost of implementation of the DBMS;
        \item additional hardware cost;
        \item slower processing of some applications; and
        \item higher impact of a failure.
    \end{enumerate}
\end{multicols}


