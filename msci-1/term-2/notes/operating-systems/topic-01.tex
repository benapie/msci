\chapter{Introduction}

\begin{definition}[Operating systems] 
    An \textbf{operating system} is a program that acts as an intermediary between a user and the hardware.
\end{definition}

The above definition is adequate; however, it does not encompass all the tasks that operating systems perform. An operating system has the following goals:

\begin{enumerate}
    \item execute user programs;
    
    \item make solving user problems easier;
    
    \item make the computer system convenient to use; and
    
    \item use the computer's resources fairly and efficiently.
\end{enumerate}

More specifically, an operating system is

\begin{enumerate}
    \item a resource allocator, responsible for the management of the computer system's resources;
    
    \item a control program, controls the execution of user programs and the operation of input and output (I/O) devices; and
    
    \item a kernel, the one program that is running all the time.
\end{enumerate}

\begin{definition}[Process]
    A \textbf{process} is a unit of execution; an abstraction that is used to support the discussion and study of operating systems. A process includes
    
    \begin{enumerate}
        \item machine code;
        
        \item current activity (register contents, such as the program counter);
        
        \item data stack, temporary data (such as local variables);
        
        \item data section, data that is accessible for the entire runtime of the process (such as global variables) and is stored on RAM; and
        
        \item heap, memory allocated during runtime.
    \end{enumerate}
\end{definition}

To execute processes we must allocate \textbf{resources}, such as CPU time, memory, and access to I/O devices. The operating system is responsible for \textbf{process management}, which consists of

\begin{enumerate}
    \item process creation and deletion;
    
    \item process holding and resuming; and
    
    \item mechanisms for process synchronisation (such as priority and scheduling).
\end{enumerate}

\begin{definition}[Process control block]
    A \textbf{process control block} (PCB) is a data structure containing information needed to manage the scheduling of a particular process. 
\end{definition}

Information about processes is represented by a process control block (PCB). Information stored about a process may include

\begin{enumerate}
    \item an identifier;
    
    \item state of the process;
    
    \item CPU utilisation;
    
    \item CPU scheduling data (such as priority);
    
    \item memory usage; and
    
    \item miscellaneous data (such as owner and description).
\end{enumerate}

\begin{definition}[Process state] 
    The \textbf{state} of a process is the status of the execution. The state of a process may be
    
    \begin{enumerate}
        \item new, where the process is being created and is under the process of being ready for execution;
        
        \item ready, where process is ready for execution and is waiting in the scheduler;
        
        \item waiting, where the scheduler has allocated it resources and is waiting to be executed;
        
        \item running, where the process is being executed;
        
        \item terminated, where the execution of the process has either finished or has been terminated early.
    \end{enumerate}
\end{definition}

A new process, as a \textbf{parent process}, can create \textbf{child processes} which can, in turn, create more processes to form a \textbf{process tree}. There are three possibilities with resource sharing for a given process tree: 

\begin{enumerate}
    \item the parent and child processes share all resources;
    \item the parent and child processes share some resources; and
    \item the parent and child processes share nothing.
\end{enumerate}

Parent and child processes may also be executed concurrently or sequentially.

On the execution of the last line of code of a process, the child process will ask the operating system to delete itself and will send its output to the parent. A parent process may terminate a child process if: it exceeds its allocated resources; it is no longed required; or the parent itself is terminating.

\begin{definition}[Kernel] 
    The \textbf{kernel} consists of
    
    \begin{enumerate}
        \item a first level interrupt handler (FLIH);
        \item a dispatcher, this handles the changing of processes; and
        \item some intra-system communication, via buses. % todo
    \end{enumerate}
\end{definition}

\begin{definition}[Interrupt]
    An \textbf{interrupt} is a signal from hardware or software that will cause a change in process.
\end{definition}

\begin{definition} 
    An \textbf{interrupt routine} is a routine that is executed on an interrupt.
\end{definition}

The function of the FLIH is to determine the source of an interrupt and to initiate the serving of the interrupt.

Some instructions may only be accessed by the OS, this is a privileged instruction set. 

\begin{definition}[Dual mode operation]
    Dual mode operation aims to distinguish between OS code and user code; there are two modes: \textbf{kernel mode} and \textbf{user mode}.
\end{definition}

User processes may call on the kernel mode when the it requires a privileged instruction. % todo

The dispatches assigns resources for processes. It is initiated when a process cannot continue or when the CPU time can be better used elsewhere (for example: an interrupt, system call, error, etc.)

