\usepackage{mathtools}
\mathtoolsset{centercolon}
\usepackage{amssymb}

\newcommand{\sltwo}{\ensuremath{\mathfrak{sl}_{2, \mathbb C}}}
\newcommand{\slthree}{\ensuremath{\mathfrak{sl}_{3, \mathbb C}}}
\newcommand{\gl}[1][]{\ensuremath{\mathfrak{gl}_{#1}}}
\renewcommand{\sl}[1][]{\ensuremath{\mathfrak{sl}_{#1}}}
\newcommand{\su}[1][]{\ensuremath{\mathfrak{su}_{#1}}}

\usepackage{parskip}

\DeclareMathOperator{\GL}{GL}
\DeclareMathOperator{\Lie}{Lie}
\DeclareMathOperator{\SL}{SL}
\DeclareMathOperator{\tr}{tr}
\DeclareMathOperator{\ad}{ad}
\DeclareMathOperator{\id}{id}

\newcommand{\C}{\mathbb C}
\newcommand{\R}{\mathbb R}
\newcommand{\N}{\mathbb N}
\newcommand{\Z}{\mathbb Z}
\newcommand{\K}{\mathbb K}

\usepackage{amsthm}
\theoremstyle{plain}
\newtheorem{theorem}{Theorem}
\numberwithin{theorem}{section}
\newtheorem{lemma}[theorem]{Lemma}
\newtheorem{corollary}[theorem]{Corollary}
\newtheorem{proposition}[theorem]{Proposition}

\theoremstyle{definition}
\newtheorem{definition}[theorem]{Definition}
\newtheorem{example}[theorem]{Example}
\newtheorem{problem}[theorem]{Problem}
\newtheorem{algorithm}[theorem]{Algorithm}

\DeclareMathOperator{\Sym}{Sym}
\DeclareMathOperator{\SO}{SO}
\renewcommand{\O}{\operatorname O}
\DeclareMathOperator{\U}{U}
\DeclareMathOperator{\SU}{SU}

\usepackage{bm}
\usepackage{tikz-cd}

\DeclareMathOperator{\Ad}{Ad}
% \DeclareMathOperator{\ad}{ad}

\usepackage{tikz}
\tikzset{weight/.style = {draw, circle, fill, inner sep=1pt}}
\tikzset{mult1/.style = {draw, circle, inner sep=2.5pt}}
\tikzset{mult2/.style = {draw, circle, inner sep=4pt}}
\tikzset{mult3/.style = {draw, circle, inner sep=5.5pt}}
\tikzset{node distance={7mm}}
\usetikzlibrary{positioning}