\section{Singular homology}

\subsection{Definition}

We now move to generalise the simplicial homology to a much more powerful version: singular homology.

\begin{definition}[Chain complex]
	A \emph{chain complex} is a sequence of abelian groups $\{C_i\}_{i \in \Z}$, and homomorphism $\partial_i: C_i \to C_{i-1}$ such that $\partial_i \circ \partial_{i-1} = 0$, called the \emph{boundary maps}.
\end{definition}

Note we may denote a chain complex by the pair $(C_*, \partial_*)$, or just $C_*$. We mainly consider non-negative chain complexes, that is $C_i = 0$ for all $i < 0$. With just a chain complex, we can define its homology.

\begin{definition}[Homology]
	The \emph{homology} $H_*(C)$ of a chain complex $C = (C_*, \partial_*)$ is
	\[
		H_i(C) = \frac{\ker \partial_i}{\im \partial_{i+1}}.
	\]
\end{definition}

Some language:
\begin{enumerate}
	\item An element $c \in C_n$ is a \emph{$n$-chain}.
	\item If $\partial_n(c) = 0$, then $c$ is a \emph{$n$-cycle}.
	\item If $c \in \im \partial_{i+1}$, then $c$ is a \emph{$n$-boundary}.
\end{enumerate}
Informally, we may that $H_i(C)$ is the $n$-cycles modulo boundaries.

Now we are ready to introduce singular homology.

\begin{definition}[Singular homology]
	Let $X$ be a topological space. A \emph{singular $n$-simplex} of $X$ is a continuous map $\sigma: \Delta^n \to X$. For $n \in \Z_{\geq 0}$, we define the \emph{singular $n$-chains} $C_n(X)$ as the free abelian groups generated by the singular $n$-simplices. Let $\sigma \in C_n$ be an $n$-chain. We define the boundary maps as
	\[
		\partial(\sigma) = \sum_{j=0}^n (-1)^j \sigma \circ \iota_j
	\]
	where $\iota_j: \R^n \to \R^{n+1}$ is the inclusion map
	\[
		\begin{cases}
			v_i \mapsto v_i     & i < j,   \\
			v_i \mapsto v_{i+1} & i \geq j
		\end{cases}
	\]
	which we may just denote $[v_1, \ldots, \hat{v}_i, \ldots, v_n]$ as before.
	We extend $\partial$ linearly as we did before.
\end{definition}

This is quite a small a definition, but note here that $C_n$ (when non-trivial) is an uncountable set. A first sight, singular theory seems harder to compute, but it enjoys some formal properties.

\begin{lemma}
	$(C_*, \partial)$ is a chain complex.
\end{lemma}

\begin{proof}
	It is enough to observe that $\partial^2 = 0$.
\end{proof}

\begin{definition}
	Let $X$ be a topological space. The $n$th homology of $X$ is
	\[ H_n(X) = H_n(C_*(X)). \]
\end{definition}

We may denote the $n$-cycles of $X$ as $Z_n(X)$ and similarly the $n$-boundaries of $X$ as $B_n(X)$, thus $H_n(X) = Z_n(X) / B_n(X)$. It may seem that we do not have many tools to work with singular homology, but maybe we can something with a trivial space.

\begin{example}
	Let $X$ be the empty space. Then $C_n(X) = \{0\}$ for all $n \in \Z_{n \geq 0}$. Thus $H_n(X) = 0$ for all $n \in \Z_{\geq 0}$.
\end{example}

\begin{example}
	Let $X$ be the one-point space. Let us reason about the singular homology. First, lets consider $C_0(X)$: the free abelian group generated by the singular $0$-simplices. But there is only 1 singular $0$-simplex for $X$: that is, $\sigma: \Delta^0 = \{1\} \to \{\text{pt}\}$. Thus $C_0(X) = \Z \langle \sigma \rangle \cong \Z$. Similarly, for $n \geq 0$, $C_n(X) \cong \Z$ as there is only one $n$-simplex (a map from $\Delta^n$ to a $\{\text{pt}\}$). So we understand $C_n(X)$, what does our boundary maps look like? Let $n \in \Z_{\geq 0}$ and $\sigma_n$ be the $n$-simplex. The domain of $\sigma_n$ is $\Delta^n$, and $\partial_n$ takes $\sigma_n$ to the sum of the function defined over the $n+1$ faces $\Delta^{n+1}$ with the inclusion map. But, there is only one function that this may be, so we get
	\[
		\partial(\sigma_n) = \sum_{j=0}^n (-1)^j \sigma_{n-1} \cong
		\begin{cases}
			\id & 2 \mid n,   \\
			0   & \text{else}
		\end{cases}
	\]
	(extending linearly). Thus, for $n > 0$.
	\[
		Z_n(X) = B_n(X) =
		\begin{cases}
			0  & 2 \mid n,    \\
			\Z & \text{else}.
		\end{cases}
	\]
	We also have $Z_0(X) = \Z$ and $B_0(X) = 0$ as $\partial_1 = \partial_0 = 0$.
	So
	\[
		H_n(X) =
		\begin{cases}
			\Z & n = 0,            \\
			0  & \text{otherwise}.
		\end{cases}
	\]
\end{example}

\begin{proposition}
	Let $X$ be a space with $n$ path components. Then $H_0(X) = \Z^n$.
\end{proposition}

\begin{proof}
	The $0$-cycles of $X$ correspond to points in $X$. Thus, let $c, c' \in Z_0(X)$. $c$ and $c'$ are \emph{homologous} (that is, lie in the same homology class) if $c - c' \in B_0(X)$. That is, if there is a path $\sigma: \Delta^1 \to X$ such that $\sigma(0,1) = c$ and $\sigma(1,0) = c'$. Let $X_1, \ldots, X_n$ be a labelling of the path components of $X$ and let $x_i$ be a point in $X_i$. We then consider a homomorphism $\Z^n \to H_0(X)$ sending $e_i \mapsto [\Delta^0 \to \{x_i\} \to X]$. We claim such a map is surjective since every $0$-simplex in a $0$-cycle must be homologous to one of the $x_i$. We observe the injectivity by the fact that there exists no paths from $x_i$ to $x_j$, $i \neq j$.
\end{proof}

We will (for now) take for granted the next statement, but we will return to it.

\begin{theorem}
	Let $X$ be a $\Delta$-complex. Then $H_n^{\text{simp}}(X) \cong H_n(X)$ for all $n \in \Z$.
\end{theorem}

Succeeding this is techniques for calculating singular homology, for now we present some homologies of spaces for intuition.

\begin{example}
	\begin{itemize}
		\item
		      $
			      H_i(\R^n) =
			      \begin{cases}
				      \Z & i = 0,       \\
				      0  & \text{else}.
			      \end{cases}
		      $

		\item
		      $
			      H_i(S^n) =
			      \begin{cases}
				      \Z & i \in \{0, n\}, \\
				      0  & \text{else}.
			      \end{cases}
		      $

		\item
		      $
			      H_i(S^n \times S^n) =
			      \begin{cases}
				      \Z   & i \in \{0, 2n\}, \\
				      \Z^2 & i = n,           \\
				      0    & \text{else}.
			      \end{cases}
		      $

		\item
		      $
			      H_i(S^n \times S^m) =
			      \begin{cases}
				      \Z & i \in \{0, n, m, n+m\}, \\
				      0  & \text{else}.
			      \end{cases}
		      $
	\end{itemize}
\end{example}

\subsection{Chain maps}

One of the main advantages of defining homology of $X$ by the (usually infinite rank) abelian groups generated by all possible continuous maps of an $n$-simplex into $X$ is that it is easy to prove that it behaves well with respect to maps between spaces.

\begin{definition}[Chain map]
	A \emph{chain map} $F: C_* \to D_*$ between chain complexes $C_*$ and $D_*$ is a collection of homomorphisms $F_n: C_n \to D_n$ such that
	\[
		\partial_{n+1}^D \circ F_{n+1} = F_n \circ \partial_{n+1}^C.
	\]
\end{definition}

The above definition can be understand pictorially, $F_n$ is a chain map if the following diagram commutes.

\begin{center}
	% https://tikzcd.yichuanshen.de/#N4Igdg9gJgpgziAXAbVABwnAlgFyxMJZABgBpiBdUkANwEMAbAVxiRAB12GoIcEBfUuky58hFGQCMVWoxZtO3XgKEgM2PASKTyM+s1aIQAYQD6wMAGpJ-EIOEaxRAEy7q++UbOF7akZvFkAGY3WQMFLh4+O1V1US0UHWl3OUMQABFzKxsYh3jA12SwzwzTH1j-JxQQoo80xSiBGRgoAHN4IlAAMwAnCABbJDIQHAgkZ19egfHqUaQQ4vr2NDoevEYs634APWNckCnBxAW5xAAWSb6jnRGxxABWS+mH2buANhTwo04VtawNixbbbpfaHGa3JD3T4lABimxyTyOJ3e0LScPK3SuSA+EMQAHZ+BR+EA
	\begin{tikzcd}
		\ldots \arrow[r] & C_{n+1} \arrow[r, "\partial_{n+1}^C"] \arrow[d, "F_{n+1}"] & C_n \arrow[r] \arrow[d, "F_n"] & \ldots \\
		\ldots \arrow[r] & D_{n+1} \arrow[r, "\partial_{n+1}^D"]                      & D_n \arrow[r]                  & \ldots
	\end{tikzcd}
\end{center}

\begin{lemma}
	A chain map $F: C_* \to D_*$ induces a map on homology
	\begin{align*}
		F_* : H_n(C_*) & \to H_n(D_*),   \\
		[c]            & \mapsto [F(c)],
	\end{align*}
	for every $n \in \Z_{\geq 0}$.
\end{lemma}

% to ask, why does it have to be a homomorphism

\begin{proof}
	We recall that $H_n(C_*) = \ker \partial_n^C/\im_{n+1}^C$. We have to show that this map is well defined (cycles are mapped to cycles, boundaries are mapped to boundaries) and that $F_*$ is a homomorphism.

	First, we will show that if $c$ is an $n$-cycle in $C$, $F_n(c)$ is a $n$-cycle in $D$. Indeed,
	\[\partial^D_n(F_{n}(c))=F_{n-1}(\partial^C_n(c))=F_{n-1}(0)=0.\]

	Now let $d$ be a $n$-boundary. Then there is $e \in C_{n+1}$ such that $\partial_{n+1}^C(e) = d$. Then
	\begin{align*}
		[F_n(c+d)] & =[F_n(c)+F_n(d)]                       \\
		           & =[F_n(c)+F_n(\partial_{n+1}^C(e))]     \\
		           & =[F_n(c)+\partial_{n+1}^D(F_{n+1}(e))] \\
		           & = [F_n(c)].
	\end{align*}

	We have left to show that $F_*$ is a homomorphism. Let $c_1$ and $c_2$ be $n$-chains in $C$. Then
	\begin{align*}
		F_*([c_1] + [c_2])
		 & = F_*([c_1 + c_2])        \\
		 & = [F_n(c_1 + c_2)]        \\
		 & = [F_n(c_1) + F_n(c_2)]   \\
		 & = [F_n(c_1)] +[F_n(c_2)]  \\
		 & = F_*([c_1]) + F_*([c_2])
	\end{align*}
	as required.
\end{proof}

\begin{proposition}
	Let $f: X \to Y$ be a continuous map between topological spaces. Then $f$ induces a chain map $f_*: C_*(X) \to C_*(Y)$ defined by sending each singular simplex $i: \Delta^n \to X$ to $f \circ i: \Delta^n \to Y$.
\end{proposition}

\begin{proof}
	Let $\sigma: \Delta^n \to X$ be a singular $n+1$-simplex. Then
	\begin{align*}
		(f_n \circ \partial_{n+1}^X)(\sigma)
		 & = f_n\left(
		\sum_{j=0}^{n+1} (-1)^j \sigma \circ \iota_j
		\right)                                                           \\
		 & = \sum_{j=0}^{n+1} (-1)^j f_n(\sigma \circ \iota_j)            \\
		 & = \sum_{j=0}^{n+1} (-1)^j f \circ \sigma \circ \iota_j         \\
		 & = \sum_{j=0}^{n+1} (-1)^j (f_{n+1} \circ \sigma) \circ \iota_j \\
		 & = (\partial_{n+1}^Y \circ f_{n+1})(\sigma)
	\end{align*}
	as required.
\end{proof}

\begin{corollary}
	Let $f: X \to Y$ be a continuous map between topological spaces. Then $f$ induces a homomorphism $f_*: H_n(X) \to H_n(Y)$ for every $n \in \Z_{\geq 0}$.
\end{corollary}

\begin{proposition}
	If $f = \id: X \to X$ then $f_* = \id: H_n(X) \to H_n(X)$ for every $n \in \Z_{\geq 0}$. Let $f: Y \to Z$ and $g: X \to Y$ be continuous maps. Then $f_* \circ g_* = (f \circ g)_*: H_n(X) \to H_n(Z)$ for every $n \in \Z_{\geq 0}$.
\end{proposition}

\begin{proof}
	Let $\sigma: \Delta^n \to X$ be a singular $n$-simplex. Then
	\[f_n([\sigma])=[f_n(\sigma)]=[f\circ\sigma]=[\sigma]\]
	and similarly
	\begin{align*}
		(f_* \circ g_*)([\sigma])
		 & = [(f_* \circ g_*)(\sigma)] \\
		 & = [f_*(g\circ\sigma)]       \\
		 & = [f\circ g \circ\sigma]    \\
		 & = [(f\circ g)_*(\sigma)]    \\
		 & = (f\circ g)_*(\sigma)
	\end{align*}
	as expected.
\end{proof}

\begin{proposition}
	Let $f: X \to Y$ be a homeomorphism of topological spaces. Then the induced map $f_*: H_n(X) \xrightarrow{\cong} H_n(Y)$ is an isomorphism for every $n \in \Z_{\geq 0}$.
\end{proposition}

\begin{proof}
	As $f$ is a homeomorphism, it has a continuous inverse $f^{-1}$. By the previous proposition, $f_* \circ (f^{-1})_* = (f \circ f^{-1})_* = (\id)_* = \id$ and similarly $(f^{-1})_* \circ f_* = \id$, thus $(f^{-1})_*$ is a left and right inverse and so $f_*$ is an isomorphism.
\end{proof}

Thus homology can be used to prove that a pair of topological spaces are not homeomorphic. Specifically, for spaces $X$ and $Y$, if $H_n(X) \not\cong H_n(Y)$, then $X$ and $Y$ are not homeomorphic (as one may expect).

\subsection{Exact sequences}

Exact sequences is a tool that we will use to compute singular homology.

\begin{definition}[Exact sequence]
	A sequence of abelian groups and homomorphisms $A \xrightarrow f B \xrightarrow g C$ is said to be \emph{exact at $B$} if $\im f = \ker g$. A sequence
	\[\ldots \to A_{i+1} \to A_i \to A_{i-1} \to \ldots\]
	is \emph{exact} if $A_{i+1} \to A_i \to A_{i-1}$ is exact at $A_i$ for every $i$.
\end{definition}

Consider the chain complex
\[
	\ldots \to C_{n+1} \xrightarrow{\partial_{n+1}} C_n \xrightarrow{\partial_{n}} C_{n-1} \to \ldots.
\]
We have that $\partial_n \circ \partial_{n+1} = 0$, thus $\im \partial_{n+1} \subset \ker \partial_n$. If $C_*$ is exact, then $H_n(C_*) = 0$ for every $n$.

\begin{definition}[Short exact sequence]
	A \emph{short exact sequence} is a five-term exact sequence
	\[0 \to A \xrightarrow f B \xrightarrow g C \to 0.\]
\end{definition}

Consider a short exact sequence as above. As this sequence is exact at $A$, $\ker f = \im 0 = 0$. Similarly, as it is exact at $C$, $\im g = \ker 0 = C$. Thus, we conclude that $f$ is injective and $g$ is surjective. As the sequence is exact at $B$, $\im f = \ker g$. Using this alongside the first isomorphism theorem for groups, we get
\[C \cong B/\ker g = B / \im f \cong B / f(A).\]

\begin{definition}[Short exact sequence of chain complexes]
	A \emph{short exact sequence of chain complexes} is a short exact sequence of chain complexes and chain maps
	\[0 \to C_* \xrightarrow f D_* \xrightarrow g E_* \to 0\]
	with $0 \to C_n \to D_n \to E_n \to 0$ a short exact sequence of abelian groups for every $n$.
\end{definition}

\begin{theorem}
	A short exact sequence of chain complexes $0 \to C_* \xrightarrow f D_* \xrightarrow g E_* \to 0$ determines a long exact sequence in homology groups
	\[\ldots \to H_{n+1}(E) \xrightarrow{\delta} H_n(C) \xrightarrow{f_*} H_n(D) \xrightarrow{g_*} H_n(E) \xrightarrow{\delta} \ldots\]
	for some $\delta$.
\end{theorem}

The following diagram above will prove useful for reference in the proof of this theorem.
\begin{center}
	% https://tikzcd.yichuanshen.de/#N4Igdg9gJgpgziAXAbVABwnAlgFyxMJZABgBoBGAXVJADcBDAGwFcYkQAdDxqCHBAL6l0mXPkIoyAJmp0mrdlx59BwkBmx4CRMgGZZDFm0SduvfiCEjN4ouQoH5xkAGEA+sDABqcgMtqNMW0UKQcaQwUTd0IrdVEtCWRdMLkjdndPAFpff2sgxPsZcKd2ABEPbxzYwIS7Un1itJMAUQqfP2r42xDSItTIkHKYgK7g5FCG-udW4bzalGS+iOdyrKqRmzHkyeX2VrWOjfyiABYU3ZMlc1U57uQzpZLLsxVcuM3Es52n02ULTo+dWIjiaIGIbxqd1CwMaA3BAOOC1IMKm7HhR3myHsJxBcIho0SoRxsOc6NuW1IxNRJnhshgUAA5vAiKAAGYAJwgAFskGQQDgIEhdLEOdykPZ+YLEAA2EWcnmIZKSpBnammND0dl4JhtXwAPRcb1FCulNAFSAAHCTFBwNVqsDrPO09aUjfKVWapQBWa3PO3axgVAQGt1ixBW5WIch8i7qzUBoMu0MKn2R8hSOVh6Oe8XCtTGpChSMAdkzCuLOcQAE5fXH7Y7KsHmsmkDW0xLY1x-Q7A54my2oxLzVGTmWhZXTWrWbrDmz3YhVcOI7Hp7MQAXEKnh9mpxVsrP1-PJ8OK2qGTOBxHh23Y+e1xud9uOz9zwcB+Qt1K83Os8epaP83nchT2HL0x0QECpXIC1wLbbcq3A8ghylKRiAESgBCAA
	\begin{tikzcd}
		& 0 \arrow[d]                                                & 0 \arrow[d]                                      & 0 \arrow[d]                            &        \\
		\ldots \arrow[r] & C_{n+1} \arrow[r, "\partial_{n+1}^C"] \arrow[d, "f_{n+1}"] & C_n \arrow[r, "\partial_{n}^C"] \arrow[d, "f_n"] & C_{n-1} \arrow[r] \arrow[d, "f_{n-1}"] & \ldots \\
		\ldots \arrow[r] & D_{n+1} \arrow[r, "\partial_{n+1}^D"] \arrow[d, "g_{n+1}"] & D_n \arrow[r, "\partial_{n}^D"] \arrow[d, "g_n"] & D_{n-1} \arrow[r] \arrow[d, "g_{n-1}"] & \ldots \\
		\ldots \arrow[r] & E_{n+1} \arrow[r, "\partial_{n+1}^E"] \arrow[d]            & E_n \arrow[r, "\partial_{n}^E"] \arrow[d]        & E_{n-1} \arrow[r] \arrow[d]            & \ldots \\
		& 0                                                          & 0                                                & 0                                      &
	\end{tikzcd}
\end{center}

\begin{proof}
	We will begin by defining $\delta$, which is effectively done by diagram chasing (a seemingly important skill in algebraic topology!). $\delta$ is infact a collection of maps, so we consider $\delta_n: H_{n}(E) \to H_{n-1}(C)$. We now let $[e] \in H_n(E)$. So $e \in E_n$ with $\partial_n^E(e) = 0$. We have that $g$ is surjective, thus there is $d \in D_n$ such that $g_n(d) = e$. Thus
	\[g_{n-1}(\partial_n^D(d)) = \partial_n^E(g_n(d)) = \partial_n^E(e) = 0, \]
	thus $\partial_n^D(d) \in \ker g_{n-1} = \im f_{n-1}$. Thus there is $c \in C_{n-1}$ such that $f_{n-1}(c) = \partial_n^D(d)$. Thus define $\delta_n([e]) = [c]$. We now check that this actually defines a homology class (that is, $c$ is a $n$-cycle in $C_{n-1}$), but indeed
	\[ f_{n-2}(\partial_{n-1}^C(c)) = \partial_{n-1}^D(f_{n-1}(c)) = \partial_{n-1}^D(\partial_n^D(d)) = 0 \]
	and as $f$ is injective, $\partial_{n-1}^C(c) = 0$. We now show that this map is well-defined. First, we show $\delta_n$ is independent on the choice of $d$, so let $d' \in D_n$ be another element such that $g_n(d') = e$. Then $g_n(d-d')=0$, and so $d - d' \in \ker g_n = \im f_n$. So there is $x \in C_n$ such that $d - d' = f_n(x)$. Thus
	\[f_{n-1}(\partial_n^C(x)) = \partial_n^D(f_n(x)) = \partial_n^D(d - d') = \partial_n^D(d) - \partial_n^D(d').\]
	Now let $c' \in C_{n-1}$ such that $f_{n-1}(c') = \partial_n^D(d')$. Then
	\[f_{n-1}(c - c') = \partial_n^D(d) - \partial_n^D(d') = f_{n-1}(\partial_n^C(x))\]
	and as $f_{n-1}$ is injective, we have $c - c' = \partial_n^C(x)$; that is, $c - c'$ is a $(n-1)$-boundary, and so belong to the same homology class. More precisely,
	\[[c] = [c' + \partial_n^D(x)] = [c'].\]
	Now we show that $\delta_n$ is independent on changing $e$ to $e + \partial_{n+1}^E(y)$ for some $y \in E_{n+1}$. We first note that, as $g_{n+1}$ is surjective, there is $w \in D_{n+1}$ such that $g_{n+1}(w) = y$. Thus
	\[g_{n}(\partial_{n+1}^D(w))=\partial_{n+1}^E(g_{n+1}(w))=\partial_{n+1}^E(y)\]
	and so
	\[g_n(d+\partial_{n+1}^D(w))=g_n(d)+g_n(\partial_{n+1}^D(w))=e+\partial_{n+1}^E(y).\]
	Thus $d + \partial_{n+1}^D(w)$ is the respective $d$ pick for $e + \partial_{n+1}^E(y)$ (as opposed to just $e$). But observe
	\[\partial_n^D(d+\partial_{n+1}^D(w))=\partial_n^D(d)+\partial_n^D(\partial_{n+1}^D(w))=\partial_n^D(d)\]
	and so $\delta(e + \partial_{n+1}^E(y))=\delta(e)$. We conclude that $\delta: H_n(E) \to H_{n-1}(C)$ is well-defined.

	We now have to show that exactness of the sequence; that is, proving that the long sequence is exact at $H_n(D)$, $H_n(E)$, and $H_n(C)$.

	First, we show that the sequence is exact at $H_n(D)$; that is, $\im f_* = \ker g_*$. We first show that $\im f_* \subset \ker g_*$. Let $[d] \in \im f_* \subset H_n(D)$, so there is $c \in C_n$ and $x \in D_{n+1}$ such that $d + \partial_{n+1}^D(x) = f_n(c)$. Thus
	\begin{align*}
		g_*([d]) & =[g_n(d)]=[g_n(d) + \partial_{n+1}^E(g_{n+1}(x))]=[g_n(d)+g_n(\partial_{n+1}^D(x))] \\
		         & =[g_n(d+\partial_{n+1}^D(x))]=[g_n(f_n(c))]=[0].
	\end{align*}
	Now we show that $\ker g_* \subset \im f_*$. Let $[d] \in \ker g_* \subset H_n(D)$. That is, there is $x \in E_{n+1}$ such that $g_n(d) = \partial_{n+1}^E(x)$. As $g$ is surjective, there is $d' \in D_{n+1}$ such that $g_{n+1}(d') = x$. Thus
	\[g_n(d)=\partial_{n+1}^E(g_{n+1}(d'))=g_n(\partial_{n+1}^D(d'))\]
	and so
	\[g_n(d - \partial_{n+1}^D(d'))=g_n(d)-g_n(\partial_{n+1}^D(d'))=0,\]
	therefore $d - \partial_{n+1}^D(d') \in \ker g_n = \im f_n$ and so $d - \partial_{n+1}^D(d') = f(c)$ for some $c \in C_n$. Thus
	\[[d]=[f(c)+\partial_{n+1}^D(d')]=[f(c)]=f_*([c])\in\im f_*\]
	as required.

	Next, we show that the sequence is exact at $H_n(E)$. We start with $\im g_* \subset \ker \delta$. Let $[e] \in \im g_*$; that is, there is $x \in E_{n+1}$ and $d \in D_n$ such that $e + \partial_{n+1}^E(x) = g_n(d)$ with $\partial_n^D(d) = 0$. As $\delta$ is well-defined, there is a unique $c \in C_{n-1}$ such that $f(c) = \partial_n^D(d) = 0$ (as defined above), and as $f$ is injective $c = 0$. Thus $\delta_n([e]) = c = 0$; that is, $[e] \in \ker\delta$. Now we show that $\ker\delta \subset \im g_*$. Let $[e] \in \ker\delta \subset H_n(E)$. Let $c \in C_{n-1}$ and $d \in D_{n}$ such that $\delta([e]) = [c]$, $f_{n-1}(c)=\partial_{n}^D(d)$, and $g_n(d) = e$ (as above). Then there is $x \in C_n$ such that $c = \partial_n^C(x)$. One may see that $[e] = [g_n(d)]=g_*([d])$ and conclude, but $\delta$ is defined on homology classes, that is cycles modulo boundaries, and we cannot say that $d$ is a boundary. Note that
	\begin{align*}
		\partial_n^D(d - f_n(x)) & = \partial_n^D(d) - \partial_n^D(f_n(x)) = \partial_n^D(d) - f_{n-1}(\partial_n^C(x)) \\
		                         & =\partial_n^D(d)-f_{n-1}(c)=0.
	\end{align*}
	Thus, we have
	\[[e] = [g_n(d)] = [g_n(d-f_n(x))]=g_*([d-f_n(x)]).\]

	Finally, we show that the sequence is exact at $H_n(C)$, starting with showing $\im\delta \subset \ker f_*$. Let $[c] \in \im\delta \in H_{n-1}(C)$ and $d \in D_n$ and $e \in E_n$ as in the definition of $\delta$. Then
	\[ f_*([c])=[f_{n-1}(c)]=[\partial_n^D(d)]=0 \]
	as needed. Now $\ker f_* \subset \im\delta$: let $[c] \in \ker f_* \subset H_{n-1}(C)$. That is, there is $x \in D_n$ such that $f_{n-1}(c)=\partial_n^D(x)$. Observe that $\partial_n^E(g_n(x))=g_{n-1}(\partial_n^D(x)) = g_{n-1}(f_{n-1}(c)) = 0$. Thus $g(x)$ is a $n$-cycle, and infact $\delta([g_n(x)])=[c]$ as $\delta$ is well-defined.
	\[[c]=\delta([g_n(x)])\in\im\delta\]
	and we are done.
\end{proof}

\subsection{Homotopy equivalence}

Homotopy theory studies objects which may be \emph{continuously deformed} into each other, this deformation is precisely what a \emph{homotopy} is. As always with topology, it is good to have a good intuition about what is going on.

\begin{definition}[Homotopy]
	Let $f, g: X \to Y$ be continuous maps between spaces. A \emph{homotopy} from $f$ to $g$ is a map $h: X \times I \to Y$ with $h_(x, 0) = f(x)$ and $h(x,1)=g(x)$. If such a homotopy exists, we may write $f \sim_h g$ or $f \sim g$.
\end{definition}

\begin{example}
	Two maps $f, g: \{\text{pt}\} \to X$ are homotopic if and only if $f(\text{pt})$ and $g(\text{pt})$ lay on the same path component of $X$. We see that any path from the points would define a valid homotopy.
\end{example}

\begin{example}
	For any space $X$ and $n \in \N$, any two maps $f, g:X \to \R^n$ are homotopic with the straight line homotopy, define as
	\begin{align*}
		h: X \times I & \to \R^n                   \\
		(x,t)         & \mapsto (1-t)f(x) + tg(x).
	\end{align*}
\end{example}

\begin{lemma}
	Let $f, f': X \to Y$ and $g, g': Y \to Z$ such that $f \sim f'$ and $g \sim g'$. Then $g \circ f \sim g' \circ f'$.
\end{lemma}

Homotopy determines an equivalence relation on maps between spaces, and the above lemma shows transitivity (symmetry and reflexivity are clear from the definition). We can also use homotopy to derive an equivalent relation on spaces.

\begin{definition}[Homotopy equivalent]
	A map $f: X \to Y$ is a \emph{homotopy equivalence} if there exists a map $g: Y \to X$ such that $f \circ g \sim \id_Y$ and $g \circ f \sim \id_X$. $g$ may be called the \emph[homotopy inverse] of $f$, and if such functions exist between spaces $X$ and $Y$, we say that they are \emph{homotopy equivalent}, denoted $X \simeq Y$.
\end{definition}

\begin{lemma}
	Let $g, h: Y \to X$ be homotopy inverses of some map $f: X \to Y$. Then $g \sim h$.
\end{lemma}

\begin{definition}[Contractible]
	A space $X$ is \emph{contactible} if $X \simeq \{\text{pt}\}$.
\end{definition}

\begin{example}
	$\R^n$ is contractible. Let $f: \R^n \to \{\text{pt}\}$ be defined the only way it can, and $g: \{\text{pt}\} \to \R^n$ with $\text{pt} \to \bm 0$. Then $f \circ g = \id_\text{pt}$ and $g \circ f = 0$. Homotopy is an equivalence relation, thus by reflexivity $f \circ g \simeq \id_{\text{pt}}$. For $g \circ f: \R^n \to \R^n$, we define the homotopy
	\begin{align*}
		h: \R^n \times I & \to \R^n,       \\
		(\bm x, t)       & \mapsto t\bm x.
	\end{align*}
\end{example}

\begin{example}
	$D^n$ is contractible, and this can be shown in the same way as in (i).
\end{example}

\begin{example}
	For all $n \in \N$, $\R^n \setminus \{\bm 0\} \simeq S^{n-1}$. Intuitively, this can be obtained with the inclusion map from $S^{n-1}$ to $\R^n\setminus\{0\}$ and the map $\bm x \mapsto \frac{\bm x}{\lVert \bm x \rVert}$ from $\R^n \setminus\{0\}$ to $S^{n-1}$. The required homotopies can be easily constructed.
\end{example}

\begin{theorem}
	Let $f: X \to Y$ be a homotopy equivalence. Then $f_*: H_n(X) \to H_n(Y)$ is an isomorphim for all $n \in \Z_{\geq 0}$.
\end{theorem}

So, if spaces $X$ and $Y$ have differing homology, then they are not homotopy equivalent. Using the examples above, we see that
\[ H_k(D^n) \cong H_k(\R^n) \cong H_k(\{\text{pt}\}), \qquad H_k(\R^n \setminus\{\text{pt}\}) \cong H_k(S^{n-1}) \]
for every $n \in \N$ and $k \in \Z_{\geq 0}$. We have the following similar result for fundamental groups.

\begin{theorem}
	Let $f: (X,x) \to (Y, y)$ be a based homotopy equivalence with $f(x) = y$. Then $f_*: \pi_1(X,x) \to \pi_1(Y, y)$ is an isomorphism.
\end{theorem}

\subsubsection{Mapping cylinders and mapping cones}

We now introduce some important homotopy equivalent spaces.

\begin{definition}[Mapping cylinder]
	Let $f: X \to Y$ be a map. The \emph{mapping cylinder} of $f$ is
	\[M_f = ((I\times X)\sqcup Y)/{\sim}\]
	where $\sim$ is the equivalence relation generated by $(0,x) \sim f(x)$ for each $x \in X$, and $\sqcup$ denotes the disjoint union.
\end{definition}

\begin{lemma}
	For every map $f: X \to Y$, $M_f \simeq Y$.
\end{lemma}

\begin{definition}[Cone]
	Let $f: X \to Y$ be a map. The \emph{cone} on a map $f$ is
	\[\operatorname{Cone}(f) = C_f = M_f/(X \times \{0\}).\]
	One may also view this as the mapping cylinder where the equivalence relation $\sim$ also has $(x,, 0) \simeq (x', 0)$ for all $x \in X$.
\end{definition}

\begin{lemma}
	For any space, the cone on the identity map is contractible.
\end{lemma}

\subsubsection{Retracts}

\begin{definition}[Retract]
	Let $A$ be a subspace of some space $X$. Then a map $r: X \to A$ is a \emph{retraction} if $\restr rA = \id_A$.
\end{definition}



\begin{definition}[Deformation retract]
	A \emph{deformation retraction} of a space $X$ onto a subspace $A$ is a homotopy $H: X \times I \to X$ between a retraction of $X$ onto $A$ and the identity map on $X$ with $\restr{H}{A \times I} = \id_A$.
\end{definition}

A retraction is just a map from a space to a subspace that preserves the positions of all points in that subspace, while a deformation retraction is a mapping that captures the concept of continuously shrinking a space into a subspace.

It is clear that a deformation retract defines a retract, but it is not always true that a retract is homotopic to the identity on the space.

\begin{remark}
	A deformation retract may be defined without the need of leaving points in $A$ fixed throughout the homotopy, and to call the above a \emph{strong deformation retract}.
\end{remark}

\subsection{Chain homotopy}

Chain homotopies are maps between chains that act in a particular nice way; modelling the behaviour of homotopies.

\begin{definition}[Chain homotopy]
	Two chain maps $f,g: C_* \to D_*$ are said to be \emph{chain homotopic} if there exists a homomorphism $P_n: C_n \to D_{n+1}$ for each $n \in \Z$ such that
	\[f_n - g_n = \partial^D_{n+1}\circ P_n + P_{n-1} \partial^C_n.\]
	If $f$ and $g$ are chain homotopic, we write $f \sim g$.
\end{definition}

Note that in the following diagrams, we make no statement of commutation.

\begin{center}
	% https://tikzcd.yichuanshen.de/#N4Igdg9gJgpgziAXAbVABwnAlgFyxMJZARgBoAGAXVJADcBDAGwFcYkQBhAfWDAGpiAXxCDS6TLnyEUZAEzU6TVuwAiPfkJFiQGbHgJEAzBQUMWbRJy6FR4vVKIBWEzTPLL3XgFpNtnRP1pZGN5VyULEDUbbV1JAxRnUMVzVXUfYT9YwKJyF2T3EAAdQsYoCBwETICHFFyktwji0vLKmOr45AA2PIb2JrKKrTs4oO768L6SgcqFGCgAc3giUAAzACcIAFskbpAcCCRyP3Wtw5p9pFkwlMtitHo1vCYAPU8NDO0T7cQrvYPEYz5RqFe6PLAvbjRVYbb4AdnO-yEnxhSDIfyQABZrgU7g8noxnmpeAIPtDTogsejEM4gZNQfjCdYhiAvpcEajsREAApMmiMegAIxgjC57WkICwYGwsGZrIB7IpnPYPO8mj5guFovs8QlUqwMuOKMQuSpsMN5IAHArHObvoCLogLbakDSHQBOZ3GhVo3qWFbqEkAAi8gfmAbVIH5QpFYvYkulbE9vwdlN9LOswdDvMjGpj2vF8f1ieR5Pt-xpaf9qsEmbD1ZA6ujWpGcb1BsogiAA
	\begin{tikzcd}
		\ldots \arrow[r] & C_{n+1} \arrow[rr, "\partial^C_{n+1}"] \arrow[ldd] \arrow[dd, "f_{n+1} - g_{n+1}" description] &  & C_n \arrow[rr, "\partial^C_n"] \arrow[lldd, "P_n" description] \arrow[dd, "f_n - g_n" description] &  & C_{n-1} \arrow[lldd, "P_{n-1}" description] \arrow[r] \arrow[dd, "f_{n-1} - g_{n-1}" description] & \ldots \arrow[ldd] \\
		&                                                                                                &  &                                                                                                    &  &                                                                                                   &                    \\
		\ldots \arrow[r] & D_{n+1} \arrow[rr, "\partial^D_{n+1}"]                                                         &  & D_n \arrow[rr, "\partial^D_n"]                                                                     &  & D_{n-1} \arrow[r]                                                                                 & \ldots
	\end{tikzcd}
\end{center}

\begin{proposition}
	If chain maps $f \sim g: C_* \to D_*$, then $f_* = g_*: H_n(C_*) \to H_n(D_*)$ for every $n \in \Z_{\geq 0}$.
\end{proposition}

\begin{proof}
	Let $c$ be a cycle of $C_n$. Then $\partial c = 0$. Thus
	\[
		f_*([c])
		= [f(c)]
		= [g(c) + \partial Pc - P\partial c]
		= [g(c) + P(0)]
		= [g(c)]
		= g_*([c])
	\]
	and so $f_* = g_*$.
\end{proof}

\begin{theorem}
	If two maps between spaces are homotopic, then their induced maps on the chain groups are chain homotopic.
\end{theorem}

\begin{proof}
	% todo
\end{proof}

\begin{corollary}
	If two maps between spaces are homotopic, then their induced maps on the homology classes are chain homotopic.
\end{corollary}

A \emph{chain homotopy equivalence} between two chain complexes is defined as one may expect, and similarly $C_*$ is said to be \emph{chain contractible} if $C_* \simeq 0$.

\begin{lemma}
	If two spaces are homotopy equivalent, then their chain groups are chain homotopy equivalent.
\end{lemma}

\begin{proof}
	This is an immediate consequence of the above theorem: let $f$ and $g$ be witnesses to the homotopy equivalence. Then $f_*$ and $g_*$ witness the chain homotopy equivalence between the corresponding chain groups.
\end{proof}

\begin{lemma}
	If two chain groups and chain homotopy equivalent, then their homology groups are isomorphic.
\end{lemma}

\begin{proof}
	If $f_*$ and $g_*$ are witnesses to the chain homotopy equivalences then $f_* \circ g_*, g_* \circ f_* \sim \id_*$. But by an earlier proposition, these maps must be equal to the identity map. Hence they are inverses of each other and we have established the isomorphism.
\end{proof}

\begin{corollary}
	If two spaces are homotopy equivalent, then their homology groups are isomorphic.
\end{corollary}

\subsection{Mayer-Vietoris sequence}

We have already seen that a short exact sequence of chain complexes induces a long exact sequence in homology, so now we need a choice of short exact sequence on a topological space.

\begin{theorem}
	Let $X$ be a space and $U, V \subset X$ such that $X = \mathring U \cup \mathring V$ and
	\[
		C_*^{\{U, V\}}(X) = \left\{
		\sum_i n_i \sigma_i:
		\sigma_i(\Delta^n) \subset U \;\text{or}\;\; \sigma_i(\Delta^n) \subset V
		\right\}.
	\]
	Then $C_*^{\{U, V\}}(X) \xhookrightarrow{} C_*(X)$ is a chain homotopy equivalence.
\end{theorem}

\begin{theorem}[Mayer-Vietoris]
	Let $X$ be a space and $U, V \subset X$ such that $X = \mathring U \cup \mathring V$ and let $\mathcal U = \{U, V\}$. Then there is a short exact sequence of chain complexes
	\[
		0 \to C_*(U \cap V)
		\xrightarrow{\varphi} C_*(U) \oplus C_*(V)
		\xrightarrow{\psi} C_*^{\{U, V\}}(X)
		\to 0
	\]
	where $\varphi(x) = (x, -x)$ and $\psi(u, v) = u + v$. Such a sequence induces the long exact sequence in homology:
	\[
		\ldots \to H_{n+1}(X) \xrightarrow{\delta} H_n(U \cap V)
		\xrightarrow{\varphi_*} H_n(U) \oplus H_n(V)
		\xrightarrow{\psi_*} H_n(X)
		\to \ldots
	\]
\end{theorem}

\begin{proof}
	We have exactness at $C_*(U \cap V)$ as a chain of $U \cap V$ that is the zero chain in $U$ must be the zero chain in $U \cap V$. Let $x$ be a chain of $U \cap V$. Then $(\psi \circ \varphi)(x) = \psi(x, -x) = 0$, thus $\im \varphi \subset \ker\psi$. Now let $u, v \in \ker\psi$. Then $u = -v$, and so $\varphi(u) = (u, -u) = (u,v) \in\im\varphi$. Thus the sequence is exact at $C_*(U) \oplus C_*(V)$. Now, finally we observe that $\im\psi = C_*^{\{U, V\}}$ by definition, and so the sequence is exact. We have already seen how a long exact sequence in homology can be induced from a short exact sequence, and the last piece we need is to see that $H_n^{\{U, V\}}(X) \cong H_n(X)$ by the earlier theorem.
\end{proof}

\subsection{Examples}

\begin{example}
	 Consider $S^1$ as the unit circle in $\C$ parametrised via $f: [0, 1] \to S^1$, $t \mapsto e^{2\pi i t}$. We let $U = f([0,1/2) \cup (1/2, 1])$ and $V = f((0,1))$. $U$ and $V$ meet the stipulation of the Mayer-Vietoris sequence, we also note that $U$ and $V$ are contractible and thus
	\[
		H_k(U) \cong H_k(V) \cong \begin{cases}
			\Z & k=0,         \\
			0  & \text{else}.
		\end{cases}
	\]
	We observe that $U \cap V$ is disconnected, with two components $f(0,1/2)$ and $f(1/2,1)$, both of which are contractible. Thus $U \cap V \simeq \{\text{pt}\} \sqcup \{\text{pt}\}$ and hence
	\[
		H_k(U \cap V) \cong \begin{cases}
			\Z^2 & k=0,         \\
			0    & \text{else}.
		\end{cases}
	\]
	The Mayer-Vietoris sequence yields
	\[0 \to H_1(S^1) \to H_0(U \cap V) \to H_0(U) \oplus H_0(V) \to H_0(S^1) \to 0\]
	which is isomorphic to
	\[0 \to H_1(S^1) \to \Z^2 \xrightarrow{A} \Z^2 \to H_0(S^1) \to 0.\]
	We wish to understand the map $A$, which records how the inclusion induced maps send the connected components of $U \cap V$ to the connected components of $U$ and $V$. $U$ and $V$ are both connected, thus $\iota_U: H_0(U \cap V) \to H_0(U)$ must send $(a,b) \mapsto a + b$ and similarly $\iota_V: H_0(U \cap V) \to H_0(V)$ must do the same. $A$ is recording the map $(\iota_U, -\iota_V)$, thus (with the canonical bases) we represent $A$ with the matrix
	\[\begin{pmatrix} 1 & 1 \\ -1 & -1 \end{pmatrix}.\] It is clear to see that $H_1(S^1) \cong \ker A \cong \Z$ and similarly $H_0(S^1) \cong \coker A \cong \Z$.
	\end{example}

\begin{example}
	\item Consider $S^n \subset \R^{n+1}$ as the unit sphere and take $U = S^n \setminus \{1,0,\ldots,0\}$ and $V = S^n \setminus \{(-1,0,\ldots,0)\}$, clearly these set us up for Mayer-Vietoris and we also see that $U \cong V \cong \R^n \simeq \{pt\}$. Also observe that $U \cap V \cong \R^n \setminus \{0\} \simeq S^{n-1}$. Looking at the Mayer-Vietoris sequence we get
	\[\ldots \to 0 \to H_k(S^n) \to H_{k-1}(S^{n-1}) \to 0 \to \ldots\]
	for $k > 1$ and so $H_k(S^n) \cong H_{k-1}(S^{n-1})$. By induction, we find that
	\[H_k(S^n) \cong \begin{cases}
			\Z & k \in \{0, n\}, \\
			0  & else
		\end{cases}\]
	although this result is not complete without a bit of sniffing around the end of the Mayer-Vietoris sequence.
\end{example}