\section{Variations of homology}

\subsection{Reduced homology}

Reduced homology is a minor modification to homology theory, and it allows us to make much cleaner statements. It is perhaps a more intuition form, in which a point has trivial reduced homology groups.

\begin{definition}[Augmented chain complex]
  Define the \emph{augmented chain complex} of a space $X$, denoted $\tilde C_*(X)$, to be
  \[\ldots \to C_n(X) \xrightarrow[]{\partial_n} C_{n-1}(X) \to \ldots \to C_{1}(X) \xrightarrow[]{\partial_1} C_0(X) \xrightarrow[]{\varepsilon} \Z \to 0\]
  where $\varepsilon: C_0(X) \to \Z$, $\sum_i\sigma_i \mapsto \sum_in_i$.
\end{definition}

\begin{definition}[Reduced homology]
  The \emph{reduced homology} of a space is $\tilde H_n(X) = H_n(\tilde C_*(X))$ for each $n \in \Z_{\geq 0}$.
\end{definition}

We now just confirm that this is indeed a minor modification, and reduced homology theory acts much in the same way as regular homology theory.

\begin{proposition}
  \begin{enumerate}
    \item A map $f: X \to Y$ between spaces induces a chain map between augmented chain groups, and thus homomorphisms bet\-ween reduced homology classes.
    \item If two maps between spaces are homologous, then so are their induced chain maps between augmented chain groups, and thus the homomorphisms between reduced homology classes are equal.
    \item $\tilde H_k(\varnothing) = \begin{cases}
              \Z & k = -1, \\
              0  & else.
            \end{cases}$
    \item If $X$ is non-empty, then
          \[H_k(X) = \begin{cases}
              \tilde H_k(X)           & k > 0, \\
              \tilde H_0(X) \oplus \Z & k = 0.
            \end{cases}\]
  \end{enumerate}
\end{proposition}

\subsection{Tensor products}

\begin{definition}[Tensor product]
  Let $A$ and $B$ be abelian group. The \emph{tensor product} is a quotient of the free abelian group generated by symbols of the form $a_i \otimes b_i$, with $a_i \in A$ and $b_i \in B$
  \[A \otimes B = \left\{
    \sum_i a_i \otimes b_i
    \right\}/{\sim}\]
  where the relation $\sim$ is generated by $(a + a') \otimes b = a \otimes b + a' \otimes b$ and $a \otimes (b + b') = a \otimes b + a \otimes b'$.
\end{definition}

We observe that $a \otimes 0 = 0 \otimes a = 0$. We also have the following properties.

\begin{enumerate}
  \item $A \otimes B \cong B \otimes A$.
  \item $A \otimes (B \otimes C) \cong (A \otimes B) \otimes C$.
  \item $\Z \otimes A \cong A$.
  \item $\Z/k \otimes A \cong A/kA$.
  \item $\Z/n \otimes \Z/m \cong (\Z/n)/(m\Z/n) \cong \Z/\gcd(m,n)$.
  \item If $A$ is a finitely generated abelian group and $A \cong \Z^r \oplus TA$, then $\Q \otimes A \cong \Q^r$.
\end{enumerate}

\subsection{Homology with coefficients}

Let $A$, $B$, and $C$ be abelian groups. For a homomorphism $f: A \to B$, we observe that there is an inuced homomorphism $f \otimes \id: A \otimes C \to B \otimes C$ such that $\sum_i a_i \otimes c_i \mapsto \sum_i f(a_i) \otimes c_i$. We simialrly have a map $\id \otimes f$.

Now let $C = (C_*, \partial_*)$ be a chain complex of abelian groups, and let $G$ be some abelian group. Then we construct the chain complex
\[
  \ldots \to C_n \otimes G \xrightarrow[]{\partial_n \otimes \id} C_{n-1} \otimes G \to \ldots \to 0
\]
which is easy to check that it is a chain complex.

\begin{definition}[Homology with coefficients]
  The $n$th homology of a chain complex $C_*$ with coefficients in an abelian group $G$ is $H_n(C_*; G) = H_n(C_* \otimes G)$.
\end{definition}

Note that $H_n(C_* \otimes G)$ may not equal $H_n(C_*) \otimes G$, unless $G$ is a field.

\begin{definition}
  For a space $X$, we define $H_n(X;G) = H_n(C_*(X); G)$.
\end{definition}

The homology theory we have developed coincides with homology with coefficients in $\Z$.

\begin{lemma}
  Let $X$ be a space and $n \in \Z_{\geq 0}$. The map $C_n(X) \to C_n(X) \otimes \Z$ such that $\sigma \mapsto \sigma \otimes 1$ induces a chain isomorphism and therefore an isomorphism on homology; that is, $H_n(X) \cong H_n(X; \Z)$.
\end{lemma}

As one may expect, homology with coefficients is invariant under homotopy equivalence, and the Mayer-Vietoris sequence is exact. Commonly used coefficients are $\Q$, $\Z/m$, and $\R$.

\begin{example}
  For an abelian group $A$, $H_k(S^n;A) = \begin{cases}
      A & k \in \{0,n\}, \\
      0 & \text{else}.
    \end{cases}$
  \end{example}
\begin{example}
  We now look at the homology groups for $\mathbb{RP}^2$ and see how they change over commonly chosen coefficients. We can compute the homology of $\mathbb{RP}^2$ by considering the union of a Mobius band and a $2$-disk, where the intersection is homotopy equivalent to the common $S^1$ boundary.
  \begin{align*}
    H_k(\mathbb{RP}^2; \Z)   & = \begin{cases}
                                   \Z   & k = 0,       \\
                                   \Z/2 & k = 1,       \\
                                   0    & \text{else}.
                                 \end{cases}     \\
    H_k(\mathbb{RP}^2; \Q)   & = \begin{cases}
                                   \Q & k = 0,       \\
                                   0  & \text{else}.
                                 \end{cases}       \\
    H_k(\mathbb{RP}^2; \Z/2) & = \begin{cases}
                                   \Z/2 & k \in \{0,1,2\}, \\
                                   0    & \text{else}.
                                 \end{cases} \\
    H_k(\mathbb{RP}^2; \Z/3) & = \begin{cases}
                                   \Z/3 & k = 0,       \\
                                   0    & \text{else}.
                                 \end{cases}     \\
  \end{align*}
\end{example}

\subsection{Relative homology}

We eventually move to cellular homology, but to do this we need the tool of \emph{relative homology}.

\begin{definition}[Relative homology]
  For a space $X$ and subspace $A \subset X$, define
  \[C_n(X,A) = \frac{C_n(X)}{C_n(A)}. \]
  The boundary map on $C(X)$ induces a well-defined boundary map on $C(X,A)$ (as you may expect). We then define
  \[H_n(X,A) = H_n(C(X,A)).\]
\end{definition}

This is another generalisation (as with coefficients) as $H_n(X, \varnothing) \cong H_n(X)$ for every $n \in \Z_{\geq 0}$. Interestingly, let $X \neq \varnothing$ and let $x_0 \in X$. Then $H_n(X, \{x_0\}) \cong \tilde H_n(X)$ for every $n \in \Z_{\geq 0}$. The map $\sum_i n_i \sigma_i \mapsto \sum_i n_i \sigma_i - (\sum_i n_i)[x_0]$ induces the isomorphism on the zeroth homology group.

The relative homology $H_n(X,A)$ may be used to help us understand the homology of $A$ or $X$.

\begin{theorem}
  Let $X$ be a space and $A \subset X$ a subspace.
  \begin{enumerate}
    \item There is a short exact sequence of chain complexes
          \[ 0 \to C_*(A) \xrightarrow{i_*} C_*(X) \xrightarrow{q} C_*(X,A) \to 0\]
          with associated long exact sequence in homology
          \[\ldots \to H_{n+1}(X,A) \xrightarrow\delta H_n(A) \xrightarrow{i_*} H_n(X) \xrightarrow{q_*} H_n(X,A) \to \ldots\]

    \item If $f: (X,A) \to (Y,B)$ is a map of pairs (that is, $f(A) \subset B$), then there is an induced map $f_*: H_n(X,A) \to H_n(Y, B)$ for all $n \in \Z_{\geq 0}$ such that the diagram
    
          \adjustbox{scale=0.85,center}{%
            % https://tikzcd.yichuanshen.de/#N4Igdg9gJgpgziAXAbVABwnAlgFyxMJZABgBpiBdUkANwEMAbAVxiRAB12GoIcEBfUuky58hFGQCMVWoxZtO3XgKEgM2PASKTyM+s1aIQACQD6wMAGpJ-ABQANUgAIAggEoQg4RrHbS06n15IzMLazsATWd3T1V1US0UACZdQLlDE1MwWxivNRFNcWQAZlTZAzYzbPsPPPjCogAWMqCMqodSXLiC32T-PXTKrNsAIVrun0SS-rSKkOGI8e8EouaA8uDM7Kix2OWGlABWFsGjRR4+PfzJouP11oUuC4EZGCgAc3giUAAzACcIABbJApEA4CBIZobDKcWAMHB0K7-IGQ6jgpDHaFsLCmABUSIBwMQmPRiAAbLNNgBHPEElGIAAcaIhiAAnJSMjT8XlkUSAOzMpBMrFGHHc1S8pClMEsgUijjsOEIulEqGkuUPIw-Wk8wkgwWIaWakDa8W-PXEg3C42mlVICkypDs+W23X0h2kyTEN1E52emwSi06R2Gn1IMghpL8Cj8IA
            \begin{tikzcd}
              \ldots \arrow[r] & {H_{n+1}(X, A)} \arrow[r, "\delta"] \arrow[d, "f_*"] & H_n(A) \arrow[r, "i_*"] \arrow[d, "f_*"] & H_n(X) \arrow[r, "q_*"] \arrow[d, "f_*"] & {H_n(X,A)} \arrow[d, "f_*"] \arrow[r] & \ldots \\
              \ldots \arrow[r] & {H_{n+1}(Y, A)} \arrow[r, "\delta"]                  & H_n(B) \arrow[r, "i_*"]                  & H_n(Y) \arrow[r, "q_*"]                  & {H_n(Y,B)} \arrow[r]                  & \ldots
            \end{tikzcd}
          }

          commutes.
  \end{enumerate}
\end{theorem}

The property of a map of pairs outlined in (ii) in the above theorem is known as \emph{naturality}, and this property may become useful in later reasoning. 

\begin{example}
  Let $A \subset X$ and suppose that the inclusion $A \xhookrightarrow{} X$ is a homotopy equivalence. Then every $H_n(X,A)$ is trivial. 
\end{example}
\begin{example}
  Let $a, b \in \R^2$ be distinct. First, we see that $H_n(\R^2, \{a\})$ is trivial for every $n$. In contrast, $H_1(\R^2, \{a,b\}) \cong \Z$ but $H_k(\R^2, \{a,b\})$ is trivial for $k\neq 1$. Interesting, in both cases the zeroth homology group vanishes. 
\end{example}

\begin{example}
  $H_k(D^n, S^{n-1})$ is trivial unless $k = n$, in which case it is isomorphic to $\Z$. 
\end{example}

\subsection{Excision}

Relative homology has a useful trick in we can remove a subsets from a pair without changing their relative homology. 

\begin{theorem}
  Let $Z \subset A \subset X$ be subsets such $\overline Z \subset \mathring A$. Then the inclusion $(X \setminus Z, A \setminus Z) \xhookrightarrow{} (X,A)$ induces an isomorphism $H_n(X \setminus Z, A \setminus Z) \to H_n(X,A)$ for every $n \in \Z_{\geq 0}$.
\end{theorem}

This is also equivalent to the following theorem.

\begin{theorem}
  Let $U, V \subset X$ be subsets of $X$ such that $\mathring U \cup \mathring V = X$. Then the inclusion $(V, U \cap V) \xhookrightarrow{} (X,U)$ induces an isomorphism $H_n(V, U \cap V) \to H_n(X, U)$ for every $n \in \Z_{\geq 0}$. 
\end{theorem}

The first theorem is the on to use, but we will prove the second. 

\begin{proof}
  % proof of equivalence of the theorems and the proof of 2.
\end{proof}

We recall that a manifold is a second countable Hausdourff space that is locally homeomorphic to Euclidean space. By second countable, we mean the base (of open sets) is countable. By locally homeomorphic, we mean that every point has a neighbourhood homeomorphic to the open Euclidean $n$-ball. A surface is a 2-dimensional manifold, and a closed surface is a surface that is compact and without boundary. 

\begin{example}
  Let $\Sigma$ be a closed surface  and let $p \in \Sigma$. By excision, taking $X = \Sigma$ and $A = \Sigma \setminus \{p\}$, and $Z = \Sigma \setminus U$ (where $U$ is a neighbourhood of $p$ homeomorphic to $\R^2$) we get that $H_2(\Sigma, \Sigma \setminus \{p\}) \cong H_2(U, U \setminus \{p\}) \cong H_2(\R^2, \R^2 \setminus \{0\})$. Then, using our tools from relative homology we have
  \[ \ldots \to H_2(\R^2) \to H_2(\R^2, \R^2 \setminus \{0\}) \to H_1(\R^2 \setminus \{0\}) \to H_1(\R^2) \to \ldots \] 
  which is equivalent to
  \[ 0 \to H_2(\R^2, \R^2 \setminus \{0\}) \to H_1(\R^2 \setminus \{0\}) \to 0. \]
  So $H_2(\R^2, \R^2 \setminus \{0\}) \cong H_1(S^1) \cong \Z$. This is called the \emph{local homology} of the surface at $p$. 
\end{example}