\usepackage[utf8]{inputenc}

\usepackage{parskip}

\usepackage{amssymb}
\usepackage{amsmath}
\usepackage{amsfonts}
\usepackage{mathtools}

\usepackage{todonotes}

\usepackage{csquotes}

\usepackage{algpseudocode}
\usepackage{algorithm}

\DeclareMathOperator{\AQ}{AQ}
\DeclareMathOperator{\DAQ}{\Delta AQ}
\DeclareMathOperator{\Q}{Q}
\DeclareMathOperator{\HE}{HE}
\DeclareMathOperator*{\argmax}{arg\,max}
\DeclareMathOperator{\Ep}{Ep}

\usepackage{braket}

\addtolength{\oddsidemargin}{-.875in}
\addtolength{\evensidemargin}{-.875in}
\addtolength{\textwidth}{1.75in}
\addtolength{\topmargin}{-.875in}
\addtolength{\textheight}{1.75in}

\usepackage[backend=biber]{biblatex}
\addbibresource{ref.bib}

\title{Natural Computing Part A}
\author{Ben Napier}
\date{March 2022}

\begin{document}


\begin{questions}

  \section{Problem sheet 1}
  \question
  \begin{parts}
    \part \hspace{0em}
    \begin{solution}
      $\ker f = \{0\}$, $\im f = 2\Z$, $\coker f = \Z/(2\Z) \cong \Z/2$.
    \end{solution}

    \part\hspace{0em}
    \begin{solution}
      $\ker f = \{\overline 0\}$, $\im f = (2\Z)/2r \cong \Z/r$, $\coker f = (\Z/2r)/((2\Z)/2r) \cong \Z/2$
    \end{solution}

    \part\hspace{0em}
    \begin{solution}
      $\ker f = \{(0, \overline 0)\}$, $\im f = (2\Z) \oplus (2\Z)/4 \cong (2\Z) \oplus \Z/2$, $\coker f = (\Z \oplus \Z/4)/(2\Z \oplus (2\Z)/4)$.
    \end{solution}

    \part\hspace{0em}
    \begin{solution}
      $\ker f = \{(0,0)\}$, $\im f = \langle (-1, 1) \rangle \cong \Z$, $\coker f \cong \Z^2 / \Z \cong \Z$.
    \end{solution}

    \part\hspace{0em}
    \begin{solution}
      $\ker f = \{(0,0)\}$,
      $\im f = \Z\langle (1,1), (1, -1) \rangle$,
      $\coker f = \Z^2/(\Z\langle (2,0), (0,2) \rangle) \cong \Z/2$.
    \end{solution}

    \part \hspace{0em}
    \begin{solution}
      Same as above, direct sum and direct product coincide for finite number of sets.
    \end{solution}
  \end{parts}

  \question
  \begin{parts}
    \setcounter{partno}{4}
    \part \hspace{0em}
    \begin{solution}

      \begin{center}
        % https://tikzcd.yichuanshen.de/#N4Igdg9gJgpgziAXAbVABwnAlgFyxMJZABgBpiBdUkANwEMAbAVxiRAGEB9ARgAoAdfgHE6AW1F0AlCAC+pdJlz5CKAEzkqtRizZdiA4WInS5C7HgJEy3TfWatEIYrPkgM55UXU3qdnY8EJHAALACNQgAIALQA9QhlNGCgAc3giUAAzACcIUSQyEBwIJG5fbQcQQVgGHDoXTJy8xAKipHUtezZBAGMCZPqQbNyS6lbEAGYyzoD+XrB+00HGttHiian-Sv5QmGSsMGA0IKysAA8ZYkFBS6vBGigIHDhbm5gwKEPjs5kQajhgrAZHD5ag7d4gkAMOg7BgABUUFhUIBOyWCwISMiAA
        \begin{tikzcd}
          C_1(\Gamma) \arrow[rr, "\delta"] \arrow[d, "\cong"]          &  & C_0(\Gamma) \arrow[d, "\cong"] \\
          0 \arrow[rr, "\begin{pmatrix}0\\0\\\vdots\\0\end{pmatrix}"'] &  & \mathbb Z^n
        \end{tikzcd}
      \end{center}
      $\ker \delta = \{0\}$, $\im \delta = \{0\}$.
      Thus $H_1(\Gamma) = 0$ and $H_0(\Gamma) = \Z^n/\{0\} = \Z^n$.
    \end{solution}
    
    \part \hspace{0em}
    \begin{solution}
      \begin{center}
        \begin{tikzcd}
          C_1(\Gamma) \arrow[rr, "\delta"] \arrow[d, "\cong"]          &  & C_0(\Gamma) \arrow[d, "\cong"] \\
          0 \arrow[rr] &  & \mathbb Z^n
        \end{tikzcd}
      \end{center}
      With $\delta$ given by the matrix
      \[
        \begin{pmatrix}
          -1 & -1 & \ldots & -1 \\
          1  & 1  & \ldots & 1
        \end{pmatrix}.
      \]
      To calculate the kernel of $\delta$, we solve
      \[
        \begin{pmatrix}
          -1 & -1 & \ldots & -1 \\
          1  & 1  & \ldots & 1
        \end{pmatrix}
        \begin{pmatrix}
          a_1 \\ a_2 \\ \vdots \\ a_n
        \end{pmatrix}
        = 0,
      \]
      which gives us $a_1 = -(a_2 + a_3 + \ldots a_n)$ for arbitrary $a_i$, $i > 1$.
      Thus $\ker \delta \cong \Z^{n-1} \cong H_1(\Gamma)$.
      Now $\im \delta = \langle (-1, 1) \rangle \cong \Z$, thus
      $H_0(\Gamma) \cong \Z^2/\Z \cong \Z$.
    \end{solution}
  \end{parts}


  \section{Problem sheet 2}
  \setcounter{question}{1}
  \question \hspace{0em}
  \begin{solution}
    First we prove that $C_*$ is a chain complex. 
    Clearly each group is abelian, now we prove that $\alpha$ and $\beta$ are homomorphisms.
    Let $a,b,c,d \in \Z$.
    \begin{align*}
      \alpha((a,b) + (c,d))
      &= \alpha(a + c, b + d) \\
      &= (a + c - b - d, a + c - b - d, a + c - b - d) \\
      &= (a - b, a - b, a - b) + (c - d, c - d, c - d) \\
      &= \alpha(a,b) + \alpha(c,d)
    \end{align*}
    thus $\alpha$ is a homomorphism.
    Now let $a,b,c,d,e,f \in \Z$.
    \begin{align*}
      \beta(a,b,c) + (d,e,f))
      &= \beta(a + d, b + e, c + f) \\
      &= (c + f - a - d, 0) \\
      &= (c - a, 0) + (f - d, 0) \\
      &= \beta(a,b,c) + \beta(d,e,f)
    \end{align*}
    so $\beta$ is a homomorphism.
    Now,
    \begin{align*}
      (\beta \circ \alpha)(a,b) 
      &= \beta(\alpha(a,b)) \\
      &= \beta(a-b, a-b, a-b) \\
      &= (a-b - (a-b), 0) \\
      &= 0;
    \end{align*}
    thus, $(C_*, (\alpha, \beta))$ is a chain complex.

    Now we prove the same for $D_*$: all groups are clearly abelian.
    Let $a,b,c,d \in \Z$.
    Then
    \begin{align*}
      \gamma((a,b) + (c,d))
      &= \gamma(a+c, b+d) \\
      &= 2(a + c -b - d) \\
      &= 2(a - b) + 2(c - d) \\
      &= \gamma(a,b) + \gamma(c,d)
    \end{align*}
    thus $\gamma$ is a homomorphism.
    Let $a,b \in \Z$.
    \begin{align*}
      \delta(x + y)
      &= 0 \\
      &= \delta(x) + \delta(y),
    \end{align*}
    so $\delta$ is a homomorphism.
    Clearly $(\gamma \circ \delta) = 0$, thus $D_*$ is a chain complex.

    Now we prove that $f_i$ defines a chain map.
    Let $a,b \in \Z$.
    \begin{align*}
      (\gamma \circ f_2)(a,b)
      &= \gamma(2a, 2b) \\
      &= 4a - 4b \\
      (f_1 \circ \alpha)(a,b)
      &= f_1(a-b, a-b, a-b) \\
      &= 4a - 4b.
    \end{align*}
    Thus $\gamma \circ f_2 = f_1 \circ \alpha$.
    Now let $a,b,c \in \Z$.
    Then
    \begin{align*}
      (\delta \circ f_1)(a,b,c) 
      &= \delta(2a + 2b) \\
      &= 0 \\
      (f_0 \circ \beta)(a,b,c)
      &= f_0(c-a, 0) \\
      &= 0
    \end{align*}
    so $\delta \circ f_1 = f_0 \circ \beta$.
    So $f_i$ defines a chain map.

    Now for the homology groups.

    $C_*$:
    \begin{align*}
      \ker(\delta_0^C) &= \Z \oplus \Z \\
      \ker(\delta_0^C) &= \Z \\
      \ker(\delta_0^C) &= \Z \\
      \im(\delta_1^C) &= \Z \\
      \im(\delta_2^C) &= \Z \\
    \end{align*}
    So 
    \[
      H_0(C) = \Z, \qquad H_1(C) = 0, \qquad H_2(C) = \Z.
    \]

    $D_*$:
    \begin{align*}
      \ker(\delta_0^D) &= \Z \\
      \ker(\delta_0^D) &= \Z \\
      \ker(\delta_0^D) &= \Z \\
      \im(\delta_1^D) &= 0 \\
      \im(\delta_2^D) &= 2\Z \\
    \end{align*}
    
    So 
    \[
      H_0(D) = \Z, \qquad H_1(D) = \Z/2, \qquad H_2(D) = \Z.
    \]
  \end{solution}

  \setcounter{question}{3}
  \question \hspace{0em}
  \begin{solution}
    We note that $\overline{\sigma} \circ \iota_0 = \sigma \circ \iota_1$ and
    $\overline{\sigma} \circ \iota_1 = \sigma \circ \iota_0$, so
    \begin{align*}
      \delta(\sigma + \overline{\sigma}) 
      &= (\sigma + \overline{\sigma})\circ \iota_0 -(\sigma + \overline{\sigma})\circ \iota_1 \\
      &= 0,
    \end{align*}
    thus indeed $\sigma + \overline{\sigma}$ is a cycle.
  \end{solution}
\end{questions}

\end{document}