\usepackage[utf8]{inputenc}

\usepackage{parskip}

\usepackage{amssymb}
\usepackage{amsmath}
\usepackage{amsfonts}
\usepackage{mathtools}

\usepackage{todonotes}

\usepackage{csquotes}

\usepackage{algpseudocode}
\usepackage{algorithm}

\DeclareMathOperator{\AQ}{AQ}
\DeclareMathOperator{\DAQ}{\Delta AQ}
\DeclareMathOperator{\Q}{Q}
\DeclareMathOperator{\HE}{HE}
\DeclareMathOperator*{\argmax}{arg\,max}
\DeclareMathOperator{\Ep}{Ep}

\usepackage{braket}

\addtolength{\oddsidemargin}{-.875in}
\addtolength{\evensidemargin}{-.875in}
\addtolength{\textwidth}{1.75in}
\addtolength{\topmargin}{-.875in}
\addtolength{\textheight}{1.75in}

\usepackage[backend=biber]{biblatex}
\addbibresource{ref.bib}

\title{Natural Computing Part A}
\author{Ben Napier}
\date{March 2022}

\begin{document}

\begin{center}
  \textbf{Assignment 2} \\
  \textsc{Algebraic Topology IV, \\ Michaelmas Term, Week 5} \\
  \textsc{Ben Napier}
  \vspace{1em}
\end{center}

\begin{questions}
  \question
  \begin{parts}
    \part If $0 \to A \to B \to \Z \to 0$ is an exact sequence of abelian groups, prove that $B \cong A \oplus \Z$.
    \begin{solution}
      As this is an exact sequence, we have that $\ker f = 0$, $\im f = \ker g$, and $\im g = \Z$. Now, as $f$ is a homomorphism and $\ker f = 0$, then $f$ must be injective. Thus $f(A) \cong A$ by the isomorphism given by the corestriction of $f$ to $f(A)$ (denoted $\corestr fA$). By the first isomorphism theorem, we have $\im g \cong B / \ker g$. That is,
      \[ \Z \cong B / \ker g \cong B / \im f \cong B / f(A) \cong B/A. \]
      Thus we conclude $B \cong A \oplus \Z$.
    \end{solution}

    \part If $0 \to \Z \to G \to \Z \to \Z \to \Z \to 0$ is an exact sequence of abelian groups, identify the isomorphism type of G.
    \begin{solution}
      \[ 0 \xrightarrow{0_1} \Z \xrightarrow{f} G \xrightarrow{g} \Z \xrightarrow{h} \Z \xrightarrow{\varphi} \Z \xrightarrow{0_2} 0. \]
      As this is an exact sequence, we can work from right to left on the maps using the fact that the image of a map is equal to the kernel of the map next along the chain. Thus we get $\im \varphi = \ker 0_2 = \Z$, $\im h = \ker \varphi$, $\im g = \ker h$, $\im f = \ker g$, and $\ker f = \im 0_1 = 0$. We can now apply the first isomorphism theorem. Firstly, $\Z \cong \im \varphi \cong \Z/\ker\varphi$, thus $\im h = \ker\varphi \cong 0$. Next, $0 \cong \im h \cong \Z / \ker h$, thus $\im g = \ker h \cong \Z$. Now from the other side, $\ker g = \im f \cong \Z/\ker f \cong \Z$ (as $\ker f = 0$). Finally, $\im G \cong G / \ker g$, thus we conclude
      \[ \Z \cong G / \Z \implies G \cong \Z \oplus \Z. \]
    \end{solution}
  \end{parts}

  \question Prove that a convex subset $X$ of $\R^n$ is homotopy equivalent to the one point space.
  \begin{solution}
    Let $X \subset \R^n$ be convex. Define $f: X \to \{p_0\}$ ($p_0 \in X$) by $x \mapsto p_0$ for all $x \in X$ and define $g: \{p_0\} \to X$ by $p_0 \mapsto p_0$. Now $f \circ g = \id_{\{p_0\}}$, which is indeed homotopic to itself. For $g \circ f$, we define the homotopy $H: X \times I \to X$ by $H(x, t) = tp_0 + (1-t)x$. We now confirm this defines a homotopy. Indeed,
    \begin{align*}
      H(x, 0) & = x = {\id}_{X}(x),     \\
      H(x, 1) & = p_0 = (g \circ f)(x).
    \end{align*}
    As $X$ is convex, we claim that $H(x, t) \in X$ for all $x \in X$ and $t \in I$. Observe that for a fixed $x \in X$, $H(x, t)$ describes the line segment from $x$ to $p_0$, which is contained within $X$ by definition.
  \end{solution}

  \question Group the \textsf{SANS SERIF CAPITAL} letters of the alphabet according to their homeomorphism classes, and according to their homotopy equivalence classes. You do not need to give proofs.
  \begin{solution}
    Homotopy equivalence classes:
    \[
      \mathsf{
        \{A, R, O, Q, D, P\}, \quad
        \{B\}, \quad
        \{H,I,J,K,L,M,N,S,T,U,V,W,X,Y,Z,C,E,F,G\}.
      }
    \]
    Homeomorphism classes:
    \[
      \mathsf{
        \{D,O\},
        \{Q\},
        \{A,R\},
        \{P\},
        \{B\},
        \{H,K\},
        \{J,L,M,N,S,U,V,W,G,Z,C,I\},
        \{E,F,Y,T\},
        \{X\}.
      }
    \]
  \end{solution}
\end{questions}

\end{document}