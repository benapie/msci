\section{}

\question Suppose that $V$ is a representation of $G$ and that $W_1$ and $W_2$ are irreducible subrepresentations. Show that either $W_1 = W_2$ or $W_1 \cap W_2 = \{0\}$.
\begin{solution}
    Let $\bm v \in W_1 \cap W_2$. Then, as $W_1$ and $W_2$ are subrepresentations we have $\rho(g)\bm v \in W_1$ and $\rho(g)\bm v \in W_2$ for all $g \in G$.Thus $\rho(g)\bm v \in W_1 \cap W_2$ for all $g \in G$; thus, $W_1 \cap W_2$ is a subrepresentation of $W_1$ and $W_2$. As $W_1$ and $W_2$ are irreducible, then either $W_1 \cap W_2 = \{0\}$ or $W_1 = W_2 = W_1 \cap W_2$.
\end{solution}

\question Consider $G = S_n$ with its permutation representation action on $\C^n$ characterized by
\[ \pi(g)e_i = e_{g(i)}, \]
and let $V \subset \C^n$ be the subspace
\[ \left\{(a_1, \ldots, a_n) \in \C^n: \sum_{i=1}^n a_i = 0\right\}. \]
Show that $V$ is irreducible.
\begin{solution}
    Let $W \subset V$ be a non-zero representation, and let $\bm a = (a_1, \ldots, a_n) \in W$. Without loss of generality, we may assume $a_1 \neq a_2$ (if this is not the case, then we can apply a permutation such that it is, since membership in $V$ implies that for all non-zero elements, the entries cannot be equal). Thus
    \[ \bm a - \pi((1\,2))\bm a = (a_1 - a_2, a_2 - a_1, 0, \ldots, 0) \in W. \]
    As $a_1 \neq a_2$, we divide through by $a_1 - a_2$ to get
    \[ \bm e_1 - \bm e_2 = (1, -1, 0, \ldots, 0) \in W. \]
    We see that $\bm e_1 - \bm e_i \in W$ by applying the permutation $(2\,i)$ to $\bm e_1 - \bm e_2$ for all $i \in \{2, \ldots, n\}$. We see that $\{\bm e_1 - \bm e_i\}_{i \in \{2,\ldots, n\}}$ are linearly independent, thus $\dim W \geq n - 1 = \dim V$; thus $W = V$. It is also not hard to see that $\{\bm e_1 - \bm e_i\}_{i \in \{2,\ldots, n\}}$ is a basis of $V$ by inspection.
\end{solution}

\question Let $(\pi_V, V)$ and $(\pi_W, W)$ be two representations of a finite group $G$.
\begin{parts}
    \part Show that if $T \in \Hom_G(V,W)$ is a $G$-homomorphism and an isomorphism of vector spaces, then $T^{-1}$ is also a $G$-homomorphism.
    \begin{solution}
        As $T$ is an isomorphism, $T^{-1}$ is well defined. Let $w \in W$ and $v \in V$ such that $T(v) = w$. Using that $T$ is a $G$-homomorphism, we get
        \begin{align*}
            T\pi_V v        & = \pi_W T v,     \\
            T\pi_V T^{-1} w & = \pi_W w,       \\
            \pi_V T^{-1}w   & = T^{-1}\pi_W w,
        \end{align*}
        as required.
    \end{solution}

    \part Assume $\dim V = \dim W = n$ and identify $V$ and $W$ with $\C^n$ by choosing bases. Show that $V \cong W$ as representations of $G$ if and only if there exists a $T \in \GL_n(\C)$ such that
    \[ T\pi_V(g)T^{-1} = \pi_W(g) \]
    for all $g \in G$.
    \begin{solution}
        $T$ is a $G$-homomorphism if and only if the matrix of $T$ is invertible.
    \end{solution}
\end{parts}

\question Prove the following. Let $(\pi, V)$ be an irreducible representation of a finite group $G$ and $Z = Z(G)$ be the center of $G$. Then $Z$ acts on $V$ as a character. That is, there exists a homomorphism $\chi: Z \to \C^\times$ such that
\[ \pi(z)v = \chi(z)v \]
for all $v \in V$.
\begin{solution}
    $\pi(z)$ is a $G$-homomorphism for all $z \in Z$, as it commutes with $\rho(g)$ for all $g \in G$. Thus, by Schur's Lemma, $\rho(z)$ acts by a non-zero scalar, $\chi(z)$, on $V$. As $\pi$ is a homomorphism, $\chi$ must be too.
\end{solution}

\question Find a two-dimensional irreducible representation of $C_n$ over $\R$ (for $n \geq 3$), and prove that it is irreducible. Why does this mean that Schur's lemma doesn't hold with real coefficients?
\begin{solution}
    We can consider $\rho: C_n \to \GL_2(\R)$ the \emph{rotation} representation, given by
    \[
        \rho(g) =
        \begin{pmatrix}
            \cos(2\pi/n)  & \sin(2\pi/n) \\
            -\sin(2\pi/n) & \cos(2\pi/n) \\
        \end{pmatrix}
    \]
    where $g$ is the generator of $C_n$. Since $\rho(g^k) = \rho(g)^k$, this defines a representation. Let $W \subset \R^2$ be a non-zero proper subrepresentation of this representation. Then $W$ must be a one-dimensional representation which is invariant under $\rho(g)$, but it is clear that this cannot happen. Thus the rotational representation is irreducible.

    Schur's Lemma does not hold here, as if it did, every irreducible representation of an abelian group would be one-dimensional.
\end{solution}

\question Show that, if $V$ is a vector space, $W \subset V$ is a subspace, and $\pi: V \to W$ is a projection, then
\begin{parts}
    \part $V = W \oplus \ker\pi$, and
    \begin{solution}
        Note that for all $v \in V$, $\pi^2(v) = \pi(v)$. Thus
        \[ \pi(v - \pi(v)) = 0 \]
        and so $v = \pi(v) + (v - \pi(v))$. We have left to show that $W \cap \ker\pi = \{0\}$, but this is immediate.
    \end{solution}

    \part $\tr\pi = \dim W$.
    \begin{solution}
        We consider $\pi: V \to V$. We can write $\pi$ as the block matrix
        \[
            \left(
            \begin{array}{c|c}
                    I_W & * \\ \hline
                    0   & 0 \\
                \end{array}
            \right).
        \]
        From this, the result is clear.
    \end{solution}
\end{parts}

% \question Let $V$ be the permutation representation of $D_5$ on the set of vertices of the regular pentagon. Write $V$ as a direct sum of irreducible subrepresentations.
% \begin{solution}
%     Let $e_1, \ldots, e_5$ be the basis vectors of $V$ corresponding to the vertices labelled $1, \ldots, 5$ in anticlockwise order. Then $\rho(r) e_i = e_{i+1}$ with addition modulo 5. We take $s$ as the reflection in the axis through vertex $1$, so $\rho(s)e_1 = e_1$, $\rho(s)e_2 = e_5$, $\rho(s)e_3 = e_4$, $\rho(s)e_4 = e_3$, and $\rho(s)e_5 = e_1$.
% \end{solution}

\question Suppose that $G$ is a group and $V = \C[G]$, and that $\chi: G \to \C^\times$ is a one-dimensional character. Show that
\[ v_X = \sum_{g \in G} \chi^{-1}(g)[g] \]
spans a one-dimensional subrepresentation on which $G$ acts via $\chi$. 
\begin{solution}
    We have to show that $gv_x = \chi(g)v_x$ for all $g \in G$. 
    \begin{align*}
        gv_X
        &= g \sum_{h \in G} \chi^{-1}(h)[h] \\
        &= \sum_{h \in G} \chi^{-1}(h)[gh] \\
        &= \sum_{g' \in G} \chi^{-1}(g^{-1}g')[g'] \\
        &= \chi^{-1}(g^{-1}) \sum_{g' \in G} \chi^{-1}(g')[g'] \\
        &= \chi(g) v_X
    \end{align*}
    as required. 
\end{solution}

\question Decompose the group ring $\C[S_3]$ as the direct sum of irreducible representations of $S_3$. That is, find explicit irreducible subrepresentations of $\C[S_3]$ such that it is the direct sum of those subrepresentations. 
\begin{solution}
    We have
    \[ \C[G] \cong \bigoplus_{\rho \in \Irr(G)} \rho^{\dim \rho} \]
    and so 
    \[ \C[S_3] \cong \triv \oplus \varepsilon \oplus \rho \oplus \rho \]
    where $\rho$ is the regular representation (that is, the permutation representation for the action of $G$ on itself). We have left to find the subspaces $V_0, V_1, V_2, V_3 \subset \C[S_3]$ such that $V_0$ is one-dimensional and $S_3$ acts trivially on it, $V_1$ is one-dimensional and $S_3$ acts as the sign representation on it, and $V_2$ and $V_3$ are two-dimensional and distinct, each isomorphic to $\rho$.

    We first note that
    \[ \alpha = \sum_{h \in G} [h] \]
    is preserved under multiplication by $g$, thus we take $V_0 = \langle \alpha \rangle$. Next consider
    \[ \beta = \sum_{h \in G} \varepsilon(h) [h]. \]
    Observe
    \begin{align*}
        g\beta &= g\sum_{h \in G} \varepsilon(h) [h]\\
        &= \sum_{h \in G} \varepsilon(h) [gh] \\
        &= \sum_{g' \in G} \varepsilon(g^{-1}g') [g'] \\
        &= \varepsilon(g^{-1}) \sum_{g' \in G} \varepsilon(g') [g'] \\
        &= \varepsilon(g) \beta
    \end{align*}
    and so $V_1 = \langle \beta \rangle$. By examining the eigenvectors of $(1\,2\,3)$, we get our last two representations. 
\end{solution}

