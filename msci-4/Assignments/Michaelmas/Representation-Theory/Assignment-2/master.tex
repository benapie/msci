\documentclass[a4paper, answers]{exam}

% tikzcd.yichuanshen.de/ tikcd diagrams
%======================%
%   Standard packages  %
%======================%
\usepackage[utf8]{inputenc}
\usepackage[T1]{fontenc}
\usepackage{lmodern}
\usepackage[UKenglish]{babel}
\usepackage{enumitem}
\usepackage{tasks}
\usepackage{graphicx}
\setlist[enumerate,1]{
  label={(\roman*)}
}
\usepackage{parskip}
\usepackage{hyperref}

%======================%
%        Maths         %
%======================%
\usepackage{amsfonts, mathtools, amsthm, amssymb}
\usepackage{xfrac}
\usepackage{bm}
\newcommand\N{\ensuremath{\mathbb{N}}}
\newcommand\R{\ensuremath{\mathbb{R}}}
\newcommand\Z{\ensuremath{\mathbb{Z}}}
\newcommand\Q{\ensuremath{\mathbb{Q}}}
\newcommand\C{\ensuremath{\mathbb{C}}}
\newcommand\F{\ensuremath{\mathbb{F}}}
\newcommand{\abs}[1]{\ensuremath{\left\lvert #1 \right\rvert}}
\newcommand\given[1][]{\:#1\vert\:}
\newcommand\restr[2]{{% we make the whole thing an ordinary symbol
  \left.\kern-\nulldelimiterspace % automatically resize the bar with \right
  #1 % the function
  \vphantom{\big|} % pretend it's a little taller at normal size
  \right|_{#2} % this is the delimiter
}}

\newcommand\corestr[2]{{% we make the whole thing an ordinary symbol
  \left.\kern-\nulldelimiterspace % automatically resize the bar with \right
  #1 % the function
  \vphantom{\big|} % pretend it's a little taller at normal size
  \right|^{#2} % this is the delimiter
}}
\usepackage{siunitx}

\usepackage{afterpage}

\usepackage{tikz-cd}
\usepackage{adjustbox}
\DeclareMathOperator{\norm}{N}
\DeclareMathOperator{\trace}{Tr}
\DeclareMathOperator*{\argmax}{arg\,max}
\DeclareMathOperator*{\argmin}{arg\,min}
\DeclareMathOperator*{\esssup}{ess\,sup}
\DeclareMathOperator*{\SL}{SL}
\DeclareMathOperator*{\GL}{GL}
\DeclareMathOperator*{\SO}{SO}
\DeclareMathOperator*{\aut}{Aut}
\DeclareMathOperator*{\id}{id}
\DeclareMathOperator*{\coker}{coker}
\DeclareMathOperator*{\im}{im}



%======================%
%       CompSci        %
%======================%
\usepackage{forest}
\usepackage{textgreek}
\usepackage{algpseudocode}

%======================%
%    Pretty tables     %
%======================%
\usepackage{booktabs}
\usepackage{caption}

\begin{document}

\begin{center}
  \textbf{\textsc{Representation Theory IV, Michaelmas Term, Assignment 2}} \\
  \textsc{Ben Napier}
  \vspace{1em}
\end{center}

\begin{questions}
  \question In the group ring $\C[S_3]$, let
  \[ \alpha = [e] + [(12)] + [(23)] + [(31)] + [(123)] + [(132)] \]
  and let
  \[ \beta = 2[e] - [(123)] - [(132)]. \]
  \begin{parts}
    \part Find $\alpha^2$, $\beta^2$, and $\alpha\beta$.
    \begin{solution}
      Let $\sigma \in S_3$. Then
      \[ \alpha[\sigma] = [\sigma] + [(12)\sigma] + [(23)\sigma] + [(31)\sigma] + [(123)\sigma] + [(132)\sigma] = \alpha. \]
      Thus
      \[ \alpha^2 = \sum_{\sigma \in S^3} \alpha[\sigma] = 6\alpha. \]
      Similarly,
      \[ \beta[\sigma] = 2[\sigma] - [e] - [\sigma^2] \]
      thus
      \[ \beta^2 = 2([e] + \beta). \]
      It is clear to see that $\alpha\beta = 0$.
    \end{solution}

    \part For $(x,y,z) \in \C^3$ (with the permutation representation), compute $\alpha(x,y,z)$ and $\beta(x,y,z)$.
    \begin{solution}
      \begin{align*}
        \alpha(x,y,z)
         & = \left(\sum_{\sigma \in S_3} [\sigma]\right)(x,y,z) \\
         & = \sum_{\sigma \in S_3} \rho(\sigma)(x,y,z)          \\
         & = 2(x+y+z)(1,1,1).                                   \\
        \beta(x,y,z)
         & = 2\rho(e)(x,y,z)
        - \rho((123))(x,y,z)
        - \rho((132))(x,y,z)                                    \\
         & = (2x-z-y, 2y-x-z, 2z-y-x).
      \end{align*}
    \end{solution}
  \end{parts}

  \question Find the character tables of the following groups.
  \begin{parts}
    \part $C_4$
    \begin{solution}
      We let $w = e^{\pi i / 2}$.
      \begin{center}
        \begin{tabular}{ccccc}
          \toprule
          Class         & $e$ & $g$   & $g^2$ & $g^3$ \\
          \midrule
          $\mathfrak 1$ & $1$ & $1$   & $1$   & $1$   \\
          $\chi$        & $1$ & $w$   & $w^2$ & $w^3$ \\
          $\chi^2$      & $1$ & $w^2$ & $w$   & $w^3$ \\
          $\chi^3$      & $1$ & $w^3$ & $w^2$ & $w$   \\
          \bottomrule
        \end{tabular}
      \end{center}
    \end{solution}
    \part $D_5$.
    \begin{solution}
      We have the conjugacy classes
      \[ \{e\}, \{r, r^4\}, \{r^2, r^3\}, \{s, rs, r^2s, r^3s, r^4s\} \]
      which can be obtained easily.
      We recall the following representations of $D_5$.
      \begin{center}
        \begin{tabular}{ccc}
          \toprule
          Label         & $\rho(r)$                                                            & $\rho(s)$                                      \\
          \midrule
          $\mathfrak 1$ & $1$                                                                  & $1$                                            \\
          $\epsilon$    & $1$                                                                  & $-1$                                           \\
          $\rho_1$      & $\begin{pmatrix}e^{2\pi i/5} & 0 \\ 0 & e^{-2\pi i/5} \end{pmatrix}$ & $\begin{pmatrix} 0 & 1 \\ 1 & 0 \end{pmatrix}$ \\
          $\rho_2$      & $\begin{pmatrix}e^{4\pi i/5} & 0 \\ 0 & e^{-4\pi i/5} \end{pmatrix}$ & $\begin{pmatrix} 0 & 1 \\ 1 & 0 \end{pmatrix}$ \\
          \bottomrule
        \end{tabular}
      \end{center}
      Now we conclude, let $w = e^{2\pi i / 5}$.
      \begin{center}
        \begin{tabular}{ccccc}
          \toprule
          Class         & $\{e\}$ & $\{r, r^4\}$ & $\{r^2, r^3\}$ & $\{s, rs, r^2s, r^3s, r^4s\}$ \\
          \midrule
          $\mathfrak 1$ & $1$     & $1$          & $1$            & $1$                           \\
          $\chi$        & $1$     & $1$          & $1$            & $-1$                          \\
          $\chi^2$      & $2$     & $w + w^4$    & $w^2 + w^3$    & $0$                           \\
          $\chi^3$      & $2$     & $w^2 + w^3$  & $w + w^4$      & $0$                           \\
          \bottomrule
        \end{tabular}
      \end{center}
    \end{solution}
  \end{parts}

  \question
  Let $G$ be a finite group acting on a finite set $X$. Let $\chi$ be the character of the permutation representation. Prove that
  \[ \chi(g) = \lvert \{x \in X: gx = x \}\rvert. \] 
  Find the character of the regular representation. 
  \begin{solution}
    Let $g \in G$. We consider the matrix of $\rho(g)$, denoted $(m_{ij})$ from fixing the basis $(e_{x_1}, \ldots, e_{x_n})$. We notice that $m_{ii} = 1$ iff $e_{x_i}$ is taken to $e_{x_i}$ through $\rho$. That is, $gx_i = x$. Thus we conclude  
  \[ \chi(g) = \lvert \{x \in X: gx = x \}\rvert. \] 
    The regular representation associates each member of $G$ with a linear map that permutes the basis according to the table of $G$. So for all $g \neq e$, $\chi(g) = 0$. For $g = e$, we see that $gx = x$ holds for all $x$, thus $\chi(g) = \lvert G \rvert$.
  \end{solution}
\end{questions}

\end{document}
