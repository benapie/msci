\documentclass[a4paper, answers]{exam}

% tikzcd.yichuanshen.de/ tikcd diagrams
%======================%
%   Standard packages  %
%======================%
\usepackage[utf8]{inputenc}
\usepackage[T1]{fontenc}
\usepackage{lmodern}
\usepackage[UKenglish]{babel}
\usepackage{enumitem}
\usepackage{tasks}
\usepackage{graphicx}
\setlist[enumerate,1]{
  label={(\roman*)}
}
\usepackage{parskip}
\usepackage{hyperref}

%======================%
%        Maths         %
%======================%
\usepackage{amsfonts, mathtools, amsthm, amssymb}
\usepackage{xfrac}
\usepackage{bm}
\newcommand\N{\ensuremath{\mathbb{N}}}
\newcommand\R{\ensuremath{\mathbb{R}}}
\newcommand\Z{\ensuremath{\mathbb{Z}}}
\newcommand\Q{\ensuremath{\mathbb{Q}}}
\newcommand\C{\ensuremath{\mathbb{C}}}
\newcommand\F{\ensuremath{\mathbb{F}}}
\newcommand{\abs}[1]{\ensuremath{\left\lvert #1 \right\rvert}}
\newcommand\given[1][]{\:#1\vert\:}
\newcommand\restr[2]{{% we make the whole thing an ordinary symbol
  \left.\kern-\nulldelimiterspace % automatically resize the bar with \right
  #1 % the function
  \vphantom{\big|} % pretend it's a little taller at normal size
  \right|_{#2} % this is the delimiter
}}

\newcommand\corestr[2]{{% we make the whole thing an ordinary symbol
  \left.\kern-\nulldelimiterspace % automatically resize the bar with \right
  #1 % the function
  \vphantom{\big|} % pretend it's a little taller at normal size
  \right|^{#2} % this is the delimiter
}}
\usepackage{siunitx}

\usepackage{afterpage}

\usepackage{tikz-cd}
\usepackage{adjustbox}
\DeclareMathOperator{\norm}{N}
\DeclareMathOperator{\trace}{Tr}
\DeclareMathOperator*{\argmax}{arg\,max}
\DeclareMathOperator*{\argmin}{arg\,min}
\DeclareMathOperator*{\esssup}{ess\,sup}
\DeclareMathOperator*{\SL}{SL}
\DeclareMathOperator*{\GL}{GL}
\DeclareMathOperator*{\SO}{SO}
\DeclareMathOperator*{\aut}{Aut}
\DeclareMathOperator*{\id}{id}
\DeclareMathOperator*{\coker}{coker}
\DeclareMathOperator*{\im}{im}



%======================%
%       CompSci        %
%======================%
\usepackage{forest}
\usepackage{textgreek}
\usepackage{algpseudocode}

%======================%
%    Pretty tables     %
%======================%
\usepackage{booktabs}
\usepackage{caption}

% \addbibresource{ref.bib}

\begin{document}

\begin{center}
  \textbf{Graph homology to motivate the generalisation of the Tutte polynomial to simplicial complexes} \\
  \textsc{Project IV: Computational Topology, \\ Michaelmas Term, Week 5} \\
  \textsc{Ben Napier}
  \vspace{1em}
\end{center}

\begin{theorem}
  Let $G = (V,E)$ be a directed graph. Then
  \[ H_0(G) \cong \Z^{\lvert C \rvert}, \qquad H_1(G) \cong \Z^{\lvert E \rvert - \lvert V \rvert + \lvert C \rvert} \]
  where $C \subset G$ denotes the set of components of $G$.
\end{theorem}

\begin{proof}
  Recall that $H_0(G) = \ker \partial_0 / \im \partial_1$. Now we consider a component of $G$. We see that any vertex can be obtained by another vertex by adding elements of $\im \partial_1$. Thus all vertices belong to the same equivalence class. It is clear that we cannot do the same for two vertices within different components, thus $H_0(G) \cong \Z^{\lvert C \rvert}$. Now we let $T \subset E'$ be a spanning tree of a component $G' = (V', E')$. We observe that if we add another edge (not in $T$) to $T$, we get a cycle. Thus each edge in $E' \setminus T$ corresponds to a cycle in $G'$. $\lvert E' \setminus T \rvert = \lvert E' \rvert - (\lvert V' \rvert - 1)$. Considering every component, we get $H_1(G) \cong \Z^{\lvert E \rvert - \lvert V \rvert + \lvert C \rvert}$.
\end{proof}

\end{document}
