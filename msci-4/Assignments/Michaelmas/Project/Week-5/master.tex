\usepackage[utf8]{inputenc}

\usepackage{parskip}

\usepackage{amssymb}
\usepackage{amsmath}
\usepackage{amsfonts}
\usepackage{mathtools}

\usepackage{todonotes}

\usepackage{csquotes}

\usepackage{algpseudocode}
\usepackage{algorithm}

\DeclareMathOperator{\AQ}{AQ}
\DeclareMathOperator{\DAQ}{\Delta AQ}
\DeclareMathOperator{\Q}{Q}
\DeclareMathOperator{\HE}{HE}
\DeclareMathOperator*{\argmax}{arg\,max}
\DeclareMathOperator{\Ep}{Ep}

\usepackage{braket}

\addtolength{\oddsidemargin}{-.875in}
\addtolength{\evensidemargin}{-.875in}
\addtolength{\textwidth}{1.75in}
\addtolength{\topmargin}{-.875in}
\addtolength{\textheight}{1.75in}

\usepackage[backend=biber]{biblatex}
\addbibresource{ref.bib}

\title{Natural Computing Part A}
\author{Ben Napier}
\date{March 2022}

% \addbibresource{ref.bib}

\begin{document}

\begin{center}
  \textbf{Graph homology to motivate the generalisation of the Tutte polynomial to simplicial complexes} \\
  \textsc{Project IV: Computational Topology, \\ Michaelmas Term, Week 5} \\
  \textsc{Ben Napier}
  \vspace{1em}
\end{center}

\begin{theorem}
  Let $G = (V,E)$ be a directed graph. Then
  \[ H_0(G) \cong \Z^{\lvert C \rvert}, \qquad H_1(G) \cong \Z^{\lvert E \rvert - \lvert V \rvert + \lvert C \rvert} \]
  where $C \subset G$ denotes the set of components of $G$.
\end{theorem}

\begin{proof}
  Recall that $H_0(G) = \ker \partial_0 / \im \partial_1$. Now we consider a component of $G$. We see that any vertex can be obtained by another vertex by adding elements of $\im \partial_1$. Thus all vertices belong to the same equivalence class. It is clear that we cannot do the same for two vertices within different components, thus $H_0(G) \cong \Z^{\lvert C \rvert}$. Now we let $T \subset E'$ be a spanning tree of a component $G' = (V', E')$. We observe that if we add another edge (not in $T$) to $T$, we get a cycle. Thus each edge in $E' \setminus T$ corresponds to a cycle in $G'$. $\lvert E' \setminus T \rvert = \lvert E' \rvert - (\lvert V' \rvert - 1)$. Considering every component, we get $H_1(G) \cong \Z^{\lvert E \rvert - \lvert V \rvert + \lvert C \rvert}$.
\end{proof}

\end{document}
