\subsection*{Invariants of simplicial complexes}

In my study of computer science, I was introduced to the concept of graph properties and wondered if there exists generalisation of some common invariants to simplicial complexes.

We first introduce the \emph{chromatic polynomial}.

\begin{definition}
  Let $G = (V,E)$ be a graph. The \emph{chromatic polynomial} of $G$, $\chi_G \in \Z[\lambda]$, counts the number of its proper $\lambda$-colourings.
\end{definition}

We take for granted the fact that such a polynomial exists and is unique. Chromatic polynomials do not uniquely identify graphs, but it clearly encodes some information on the number of colourings. 

How may we generalise the chromatic polynomial to simplicial complexes? Well, we may define it only on its 1-skeleton, but then we lose all the higher dimensional information and we haven't really generalised anything. May we abstract the notion of a colouring to a higher-dimensional analogue? We now introduce a lemma-generalisation of the chromatic polynomial on graphs. 

\begin{definition}[Tutte polynomial]
  Let $G = (V,E)$ be a graph. For $A \subset E$, denote $k(A)$ as the number of connected components on the subgraph $(V,A)$. We define the \emph{Tutte polynomial} of $G$, $T_G \in \Z[x, y]$ as
  \[
    T_G(x,y) = \sum_{A \subset E} (x-1)^{k(A) - k(E)} (y-1)^{k(A) + \lvert A \rvert - \lvert V \rvert}.
  \]
\end{definition}

This polynomial can shown to specialise to the chromatic polynomial as follows:
\[
  \chi_G(k) = (-1)^{\lvert V \rvert - k(G)} \lambda^{k(G)} T_G(1 - \lambda, 0).
\]
The Tutte polynomial encodes much more information than the chromatic polynomial. For example, for a graph $G = (V,E)$, we have the following points:
\begin{enumerate}
  \item $T_G(1,1)$ counts the number of spanning forests;
  \item $T_G(2,2) = 2^{\lvert E \rvert}$;
  \item $T_G(2,0)$ counts the number of acyclic orientations; and
  \item $T_G(1,2)$ counts the number of spanning subgraphs. 
\end{enumerate}

Now how can we generalise the Tutte polynomial to simplicial complexes? Well, note that in our definition, the Tutte polynomial is a summation over each spanning subgraph, which is generalised with the notion of a \emph{subcomplex} (details to follow). We also claim that the exponents of each term may be reframed using \emph{homology}, something that we may also generalise to simplicial complexes. 

Firstly, for a graph $G = (V, E)$, $\lvert H_1(G) \rvert = k(E) + \lvert E \rvert - \lvert V \rvert$. Similarly, $\lvert H_0(G) \rvert = k(E)$. Thus, we may define the Tutte polynomial on a graph as
\[
  T_G(x,y) = \sum_{G' \subset G} (x-1)^{\lvert H_0(G') \rvert - \lvert H_0(G) \rvert} (y-1)^{\lvert H_1(G') \rvert}
\]
where the summation is taken over the \emph{spanning} subgraphs of $G = (V,E)$ (that is, $G'$ is of the form $(V, A)$, $A \subset E$).

It may be clear how we may generalise, but we lack an analogue for spanning subgraph. $L \subset K^{(n)}$ is a spanning $n$-dimensional subcomplex of a simplicial complex $K$ of dimension $\geq n$ if the $(n-1)$-skeleton of $L$ and $K$ coincide. 

For a simplicial complex $K$, we define the $n$-Tutte polynomial as
\[ 
  T_{K, n}(x,y) = \sum_{L \subset K^{(n)}} (x-1)^{\lvert H_{n-1}(L) \rvert - \lvert H_{n-1}(K) \rvert} (y-1)^{\lvert H_n(L) \rvert}
\]
and indeed this generalises as
\[
  T_{G,1}(x,y) = T_G(x,y).
\]
Thus, we may define the $n$-chromatic polynomial of a simplicial complex $K$ as
\[
  \chi_{K, n}(k) = (-1)^{\lvert C_{n-1}(K) \rvert - \lvert H_{n-1}(K) \rvert}
    \lambda^{\lvert H_{n-1}(K) \rvert} T_{K, n}(1 - \lambda, 0)
\]
where $C_{i}(K)$ denotes the $i$th chain group $K$.