\documentclass[a4paper, answers]{exam}

% tikzcd.yichuanshen.de/ tikcd diagrams
%======================%
%   Standard packages  %
%======================%
\usepackage[utf8]{inputenc}
\usepackage[T1]{fontenc}
\usepackage{lmodern}
\usepackage[UKenglish]{babel}
\usepackage{enumitem}
\usepackage{tasks}
\usepackage{graphicx}
\setlist[enumerate,1]{
  label={(\roman*)}
}
\usepackage{parskip}
\usepackage{hyperref}

%======================%
%        Maths         %
%======================%
\usepackage{amsfonts, mathtools, amsthm, amssymb}
\usepackage{xfrac}
\usepackage{bm}
\newcommand\N{\ensuremath{\mathbb{N}}}
\newcommand\R{\ensuremath{\mathbb{R}}}
\newcommand\Z{\ensuremath{\mathbb{Z}}}
\newcommand\Q{\ensuremath{\mathbb{Q}}}
\newcommand\C{\ensuremath{\mathbb{C}}}
\newcommand\F{\ensuremath{\mathbb{F}}}
\newcommand{\abs}[1]{\ensuremath{\left\lvert #1 \right\rvert}}
\newcommand\given[1][]{\:#1\vert\:}
\newcommand\restr[2]{{% we make the whole thing an ordinary symbol
  \left.\kern-\nulldelimiterspace % automatically resize the bar with \right
  #1 % the function
  \vphantom{\big|} % pretend it's a little taller at normal size
  \right|_{#2} % this is the delimiter
}}

\newcommand\corestr[2]{{% we make the whole thing an ordinary symbol
  \left.\kern-\nulldelimiterspace % automatically resize the bar with \right
  #1 % the function
  \vphantom{\big|} % pretend it's a little taller at normal size
  \right|^{#2} % this is the delimiter
}}
\usepackage{siunitx}

\usepackage{afterpage}

\usepackage{tikz-cd}
\usepackage{adjustbox}
\DeclareMathOperator{\norm}{N}
\DeclareMathOperator{\trace}{Tr}
\DeclareMathOperator*{\argmax}{arg\,max}
\DeclareMathOperator*{\argmin}{arg\,min}
\DeclareMathOperator*{\esssup}{ess\,sup}
\DeclareMathOperator*{\SL}{SL}
\DeclareMathOperator*{\GL}{GL}
\DeclareMathOperator*{\SO}{SO}
\DeclareMathOperator*{\aut}{Aut}
\DeclareMathOperator*{\id}{id}
\DeclareMathOperator*{\coker}{coker}
\DeclareMathOperator*{\im}{im}



%======================%
%       CompSci        %
%======================%
\usepackage{forest}
\usepackage{textgreek}
\usepackage{algpseudocode}

%======================%
%    Pretty tables     %
%======================%
\usepackage{booktabs}
\usepackage{caption}

\addbibresource{ref.bib}

\begin{document}

\begin{center}
  \textbf{On the Tutte polynomial} \\
  \textsc{Project IV: Computational Topology, \\ Michaelmas Term, Week 4} \\
  \textsc{Ben Napier}
  \vspace{1em}
\end{center}

\begin{definition}[Tutte polynomial]
  Let $G = (V,E)$ be an undirected graph. For $A \subset E$, denote $k(A)$ as the number of connected components on the subgraph induced by $A$. We define the \emph{Tutte polynomial} as
  \begin{equation}
    \label{eq:tutte}
    T_G(x,y) = \sum_{A \subset E} (x-1)^{k(A) - k(E)} (y-1)^{k(A) + \lvert A \rvert - \lvert V \rvert}.
  \end{equation}
\end{definition}

This polynomial has some useful properties, namely it is indeed a graph property (preserved under isomorphisms), multiplicative over disjoint unions, and the Tutte polynomial of a dual graph $G^*$ of a undirected planar graph $G$ is given by
\[
  T_{G^*}(x,y) = T_G(y,x).
\]

We introduce another invariant of graphs.

\begin{definition}[Chromatic polynomial]
  Let $G = (V,E)$ be a directed graph. We define the \emph{chromatic polynomial} of $G$, $\chi_G$, as the number of proper $k$-colourings that exist on $G$.
\end{definition}

The Tutte polynomial is infact a generalisation of the chromatic polynomial. Indeed, for a undirected graph $G = (V,E)$ the Tutte polynomial specialises at $y = 0$:
\begin{equation}
  \chi_G(\lambda) = (-1)^{\lvert V \rvert - k(G)} \lambda^{k(G)} T_G(1 - \lambda, 0).
\end{equation}

We see that if $\chi_G(\lambda)(3) \geq 1$, then there is a three colouring on $G$. Thus determining the Tutte polynomial is $\mathsf{NP}$-hard. Infact, we have an upper bound on our complexity: it has been shown that it is $\mathsf{\#P}$-complete \cite{annan1995complexities} (while some of the coefficients can be computed in polynomial time).

Moving forward, we can generalised the notion of a Tutte polynomial to simplicial complexes, as outlined by \textcite{krushkal2014polynomial}. Within the definition of the Tutte polynomial (\ref{eq:tutte}), we sum over all spanning subgraphs of $G$. Using more \emph{higher-dimensional friendly language}, the sum over all subcomplexes of dimension 1 such that the entire 0-skeleton is included in each complex. Translating this to a simplicial complex, we may say that $L \subset K$ is a spanning $n$-dimensional subcomplex of a simplicial complex $K$ of dimension $\geq n$ if their $(n-1)$-skeletons coincide. We define
\begin{equation}
  T_K(x,y) = \sum_{L \subset K^{(n)}} x^{\lvert H_{n-1}(L) \rvert - \lvert H_{n-1}(K) \rvert} y^{\lvert H_n(L) \rvert}
\end{equation}
where the summation is taken over the \emph{spanning} $n$-subcomplexes $L$ of $K$.
We can see that (1) and (3) are equivalent for graphs by observing that $n(H) = k(A) + \lvert A \rvert - \lvert V \rvert$ (where $H = (V,A)$) defines the nullity of $H$: the rank of $H_1(H)$.

\printbibliography

\end{document}
