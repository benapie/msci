\clearpage 
\question[15] Show that the following two circuits are functionally equivalent.

\begin{center}
    \begin{quantikz}
        \lstick{} & \gate{H} & \ctrl{1} & \gate{H} & \qw \\
        \lstick{} & \gate{H} & \targ{} & \gate{H} & \qw
    \end{quantikz}
    \hspace{2em}
    \begin{quantikz}
        \lstick{} &  \targ{} & \qw \\
        \lstick{} & \ctrl{-1} & \qw
    \end{quantikz}
\end{center}

\begin{solution}
    An equivalent statement of this question is
    \[ \text{CNOT} = (H \otimes H) \; \overline{\text{CNOT}} \; (H \otimes H). \]
    We first look at some truth tables for CNOT gates. Note $\overline{\text{CNOT}}$ is the CNOT gate with the input source and target qubits swapped.
    \begin{center}
        \begin{tabular}{cccccc}
            \toprule
            \multicolumn{2}{c}{Input} & \multicolumn{2}{c}{CNOT} & \multicolumn{2}{c}{$\overline{\text{CNOT}}$}                                  \\
            \midrule
            $A$                       & $B$                      & $A$                                          & $B$      & $A$      & $B$      \\
            \midrule
            $\ket 0$                  & $\ket 0$                 & $\ket 0$                                     & $\ket 0$ & $\ket 0$ & $\ket 0$ \\
            $\ket 0$                  & $\ket 1$                 & $\ket 0$                                     & $\ket 1$ & $\ket 1$ & $\ket 1$ \\
            $\ket 1$                  & $\ket 0$                 & $\ket 1$                                     & $\ket 0$ & $\ket 1$ & $\ket 0$ \\
            $\ket 1$                  & $\ket 1$                 & $\ket 1$                                     & $\ket 1$ & $\ket 0$ & $\ket 1$ \\
            \bottomrule
        \end{tabular}
    \end{center}
    Pick the basis $\ket{00},\ket{01},\ket{10},\ket{11}$. Then we have the matrices
    \[
        \text{CNOT} =
        \begin{pmatrix}
            1 & 0 & 0 & 0 \\
            0 & 1 & 0 & 0 \\
            0 & 0 & 0 & 1 \\
            0 & 0 & 1 & 0 \\
        \end{pmatrix},
        \qquad
        \overline{\text{CNOT}} =
        \begin{pmatrix}
            1 & 0 & 0 & 0 \\
            0 & 0 & 0 & 1 \\
            0 & 0 & 1 & 0 \\
            0 & 1 & 0 & 0 \\
        \end{pmatrix}.
    \]
    For $H$, we have
    \[
        H = \frac{1}{\sqrt 2}
        \begin{pmatrix}
            1 & 1  \\
            1 & -1 \\
        \end{pmatrix}.
    \]
    Thus
    \[
        H \otimes H = \frac{1}{2}
        \begin{pmatrix}
            1 & 1 & 1 & 1 \\
            1 & -1 & 1 & -1 \\
            1 & 1 & -1 & -1 \\
            1 & -1 & -1 & 1 \\
        \end{pmatrix}.
    \]
    We then conclude,
    \begin{align*}
        (H \otimes H) \; \overline{\text{CNOT}} \; (H \otimes H)
        &= \frac14
        \begin{pmatrix}
            1 & 1 & 1 & 1 \\
            1 & -1 & 1 & -1 \\
            1 & 1 & -1 & -1 \\
            1 & -1 & -1 & 1 \\
        \end{pmatrix}
        \begin{pmatrix}
            1 & 0 & 0 & 0 \\
            0 & 0 & 0 & 1 \\
            0 & 0 & 1 & 0 \\
            0 & 1 & 0 & 0 \\
        \end{pmatrix}
        \begin{pmatrix}
            1 & 1 & 1 & 1 \\
            1 & -1 & 1 & -1 \\
            1 & 1 & -1 & -1 \\
            1 & -1 & -1 & 1 \\
        \end{pmatrix}
        \\
        &= \frac14
        \begin{pmatrix}
            1 & 1 & 1 & 1 \\
            1 & -1 & 1 & -1 \\
            1 & 1 & -1 & -1 \\
            1 & -1 & -1 & 1 \\
        \end{pmatrix}
        \begin{pmatrix}
            1 & 1 & 1 & 1 \\
            1 & -1 & -1 & 1 \\
            1 & 1 & -1 & -1 \\
            1 & -1 & 1 & -1 \\
        \end{pmatrix}
        \\
        &= 
        \begin{pmatrix}
            1 & 0 & 0 & 0 \\
            0 & 1 & 0 & 0 \\
            0 & 0 & 0 & 1 \\
            0 & 0 & 1 & 0 \\
        \end{pmatrix}
        \\
        &= \text{CNOT}.
    \end{align*}
\end{solution}