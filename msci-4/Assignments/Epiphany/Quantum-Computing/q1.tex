\clearpage
\question
\begin{parts}
    \part[10] Is it possible to output $n+1$ copies of an unknown qubit $\ket{\varphi}$ given $n$ copies of $\ket{\varphi}$ together with $\ket{0}$ for some $n$? That is, is it possible to quantumly implement the map
    \[ \ket{\varphi}^{\otimes n} \otimes \ket{0} \mapsto \ket{\varphi}^{\otimes(n+1)} \]
    for some $n$, or is it impossible for all $n$? 
\begin{solution}
    Suppose a unitary map $U$ exists such that
    \[
        U: \ket{\varphi}^{\otimes n} \otimes \ket{0} \mapsto \ket{\varphi}^{\otimes(n+1)}.
    \]
    Note that as $U$ is unitary, $UU^\dag = U^\dag U = I$.
    Let
    $
        \ket\varphi =
        \begin{pmatrix}
            \alpha_\varphi \\ \beta_\varphi
        \end{pmatrix}
    $
    and
    $
        \ket\psi =
        \begin{pmatrix}
            \alpha_\psi \\ \beta_\psi
        \end{pmatrix}
    $
    be arbitrary two qubits. Then
    \begin{align*}
        \braket{\varphi|\psi}^n
         & =
        \left(
        \begin{pmatrix}
                \overline\alpha_\varphi & \overline\beta_\varphi
            \end{pmatrix}
        \begin{pmatrix}
                \alpha_\psi \\ \beta_\psi
            \end{pmatrix}
        \right)^n
        \\
         & =
        \begin{pmatrix}
            \overline\alpha_\varphi & \overline\beta_\varphi
        \end{pmatrix}^{\otimes n}
        \begin{pmatrix}
            \alpha_\psi \\ \beta_\psi
        \end{pmatrix}^{\otimes n}
        \\
         & =
        \left(
        \begin{pmatrix}
            \overline\alpha_\varphi & \overline\beta_\varphi
        \end{pmatrix}^{\otimes n}
        \begin{pmatrix}
            \alpha_\psi \\ \beta_\psi
        \end{pmatrix}^{\otimes n}
        \right)
        \braket{0|0}
        \\
         & =
        \left(
        \begin{pmatrix}
            \overline\alpha_\varphi & \overline\beta_\varphi
        \end{pmatrix}^{\otimes n}
        \begin{pmatrix}
            \alpha_\psi \\ \beta_\psi
        \end{pmatrix}^{\otimes n}
        \right)
        \left(
        \begin{pmatrix}
                1 & 0
            \end{pmatrix}
        \begin{pmatrix}
                1 \\ 0
            \end{pmatrix}
        \right)
        \\
         & =
        \left(
        \begin{pmatrix}
            \overline\alpha_\varphi & \overline\beta_\varphi
        \end{pmatrix}^{\otimes n}
        \otimes
        \begin{pmatrix}
            1 & 0
        \end{pmatrix}
        \right)
        \left(
        \begin{pmatrix}
            \alpha_\psi \\ \beta_\psi
        \end{pmatrix}^{\otimes n}
        \otimes
        \begin{pmatrix}
            1 \\ 0
        \end{pmatrix}
        \right)
        \\
         & =
        \left(
        \begin{pmatrix}
            \overline\alpha_\varphi & \overline\beta_\varphi
        \end{pmatrix}^{\otimes n}
        \otimes
        \begin{pmatrix}
            1 & 0
        \end{pmatrix}
        \right)
        U^\dag U
        \left(
        \begin{pmatrix}
            \alpha_\psi \\ \beta_\psi
        \end{pmatrix}^{\otimes n}
        \otimes
        \begin{pmatrix}
            1 \\ 0
        \end{pmatrix}
        \right)
        \\
         & =
        \left(
        U
        \left(
            \begin{pmatrix}
                \alpha_\psi \\ \beta_\psi
            \end{pmatrix}^{\otimes n}
            \otimes
            \begin{pmatrix}
                1 \\ 0
            \end{pmatrix}
            \right)
        \right)^\dag
        U
        \left(
        \begin{pmatrix}
            \alpha_\psi \\ \beta_\psi
        \end{pmatrix}^{\otimes n}
        \otimes
        \begin{pmatrix}
            1 \\ 0
        \end{pmatrix}
        \right)
        \\
         & =
        \left(
        \begin{pmatrix}
            \overline\alpha_\varphi & \overline\beta_\varphi
        \end{pmatrix}^{\otimes n}
        \otimes
        \begin{pmatrix}
            \overline\alpha_\varphi & \overline\beta_\varphi
        \end{pmatrix}
        \right)
        \left(
        \begin{pmatrix}
            \alpha_\psi \\ \beta_\psi
        \end{pmatrix}^{\otimes n}
        \otimes
        \begin{pmatrix}
            \alpha_\psi \\ \beta_\psi
        \end{pmatrix}
        \right)
        \\
         & =
        \begin{pmatrix}
            \overline\alpha_\varphi & \overline\beta_\varphi
        \end{pmatrix}^{\otimes(n+1)}
        \begin{pmatrix}
            \alpha_\psi \\ \beta_\psi
        \end{pmatrix}^{\otimes(n+1)}
        \\
         & =
        \left(
        \begin{pmatrix}
            \overline\alpha_\varphi & \overline\beta_\varphi
        \end{pmatrix}
        \begin{pmatrix}
            \alpha_\psi \\ \beta_\psi
        \end{pmatrix}
        \right)^{n+1}
        \\
         & =
        \braket{\varphi|\psi}^{n+1}.
    \end{align*}
    That is, $\braket{\varphi|\psi}^n(\braket{\varphi|\psi} - 1) = 0$. Thus $\braket{\varphi|\psi} \in \{0,1\}$, which does not necessarily hold for arbitrary qubits. \qed
\end{solution}
    \part[10] Can you generalize/strengthen your result?
    \begin{solution}
        We strengthen this assertion to the following. 
    
        There is does not exist a unitary map
        \[
            U:
            \ket{\varphi}^{\otimes n} \otimes \ket{\rho}^{\otimes m}
            \mapsto \ket{\varphi}^{\otimes n + m}
        \]
        for every qubit $\ket{\varphi}$, where $\ket\rho = \alpha_\rho\ket0 + \beta_\rho\ket 1$ is some fixed qubit and $n, m \in \mathbb N$.
    
        The proof follows a similar lines to the proof of the original assertion. Let
        $
            \ket\varphi =
            \begin{pmatrix}
                \alpha_\varphi \\ \beta_\varphi
            \end{pmatrix}
        $
        and
        $
            \ket\psi =
            \begin{pmatrix}
                \alpha_\psi \\ \beta_\psi
            \end{pmatrix}
        $
        be arbitrary two qubits. Then
        \begin{align*}
            \braket{\varphi|\psi}^n
             & =
            \left(
            \begin{pmatrix}
                    \overline\alpha_\varphi & \overline\beta_\varphi
                \end{pmatrix}
            \begin{pmatrix}
                    \alpha_\psi \\ \beta_\psi
                \end{pmatrix}
            \right)^n
            \\
             & =
            \begin{pmatrix}
                \overline\alpha_\varphi & \overline\beta_\varphi
            \end{pmatrix}^{\otimes n}
            \begin{pmatrix}
                \alpha_\psi \\ \beta_\psi
            \end{pmatrix}^{\otimes n}
            \\
             & =
            \left(
            \begin{pmatrix}
                \overline\alpha_\varphi & \overline\beta_\varphi
            \end{pmatrix}^{\otimes n}
            \begin{pmatrix}
                \alpha_\psi \\ \beta_\psi
            \end{pmatrix}^{\otimes n}
            \right)
            \braket{\rho|\rho}^m
            \\
             & =
            \left(
            \begin{pmatrix}
                \overline\alpha_\varphi & \overline\beta_\varphi
            \end{pmatrix}^{\otimes n}
            \begin{pmatrix}
                \alpha_\psi \\ \beta_\psi
            \end{pmatrix}^{\otimes n}
            \right)
            \left(
            \begin{pmatrix}
                    \overline\alpha_\rho & \overline\beta_\rho
                \end{pmatrix}
            \begin{pmatrix}
                    \alpha_\rho \\ \beta_\rho
                \end{pmatrix}
            \right)^m
            \\
             & =
            \left(
            \begin{pmatrix}
                \overline\alpha_\varphi & \overline\beta_\varphi
            \end{pmatrix}^{\otimes n}
            \otimes
            \begin{pmatrix}
                \overline\alpha_\rho & \overline\beta_\rho
            \end{pmatrix}^{\otimes m}
            \right)
            \left(
            \begin{pmatrix}
                \alpha_\psi \\ \beta_\psi
            \end{pmatrix}^{\otimes n}
            \otimes
            \begin{pmatrix}
                \alpha_\rho \\ \beta_\rho
            \end{pmatrix}^{\otimes m}
            \right)
            \\
             & =
            \left(
            \begin{pmatrix}
                \overline\alpha_\varphi & \overline\beta_\varphi
            \end{pmatrix}^{\otimes n}
            \otimes
            \begin{pmatrix}
                \overline\alpha_\rho & \overline\beta_\rho
            \end{pmatrix}^{\otimes m}
            \right)
            U^\dag U
            \left(
            \begin{pmatrix}
                \alpha_\psi \\ \beta_\psi
            \end{pmatrix}^{\otimes n}
            \otimes
            \begin{pmatrix}
                \alpha_\rho \\ \beta_\rho
            \end{pmatrix}^{\otimes m}
            \right)
            \\
             & =
            \left(
            U
            \left(
                \begin{pmatrix}
                    \alpha_\psi \\ \beta_\psi
                \end{pmatrix}^{\otimes n}
                \otimes
                \begin{pmatrix}
                    \alpha_\rho \\ \beta_\rho
                \end{pmatrix}^{\otimes m}
                \right)
            \right)^\dag
            U
            \left(
            \begin{pmatrix}
                \alpha_\psi \\ \beta_\psi
            \end{pmatrix}^{\otimes n}
            \otimes
            \begin{pmatrix}
                \alpha_\rho \\ \beta_\rho
            \end{pmatrix}^{\otimes m}
            \right)
            \\
             & =
            \left(
            \begin{pmatrix}
                \overline\alpha_\varphi & \overline\beta_\varphi
            \end{pmatrix}^{\otimes n}
            \otimes
            \begin{pmatrix}
                \overline\alpha_\varphi & \overline\beta_\varphi
            \end{pmatrix}^{\otimes m}
            \right)
            \left(
            \begin{pmatrix}
                \alpha_\psi \\ \beta_\psi
            \end{pmatrix}^{\otimes n}
            \otimes
            \begin{pmatrix}
                \alpha_\rho \\ \beta_\rho
            \end{pmatrix}^{\otimes m}
            \right)
            \\
             & =
            \begin{pmatrix}
                \overline\alpha_\varphi & \overline\beta_\varphi
            \end{pmatrix}^{\otimes(n+m)}
            \begin{pmatrix}
                \alpha_\psi \\ \beta_\psi
            \end{pmatrix}^{\otimes(n+m)}
            \\
             & =
            \left(
            \begin{pmatrix}
                \overline\alpha_\varphi & \overline\beta_\varphi
            \end{pmatrix}
            \begin{pmatrix}
                \alpha_\psi \\ \beta_\psi
            \end{pmatrix}
            \right)^{n+m}
            \\
             & =
            \braket{\varphi|\psi}^{n+m}.
        \end{align*}
        Thus $\braket{\varphi|\psi}^n(\braket{\varphi|\psi}^m - 1) = 0$. Thus $\braket{\varphi|\psi}$ is either $0$ or an $m$th root of unity, neither of which is necessarily true for arbitrary qubits. \qed
    \end{solution}
\end{parts}

% \begin{solution}
%     Pick the basis $\ket 0, \ket 1$ and let $\ket\varphi = \alpha\ket 0 + \beta\ket 1$ where $\alpha, \beta \in \mathbb C$ and $\lvert \alpha \rvert^2 + \lvert \beta \rvert^2 = 1$. We are looking to construct a map
%     \[
%         U: \ket\varphi^{\otimes n} \otimes \ket0 \mapsto \ket\varphi^{\otimes(n+1)}.
%     \]
%     It is a simple exercise to show that
%     \begin{equation}
%         \label{eq:n-tensor-product}
%         \ket\varphi^{\otimes n} = \sum_{i \in \{0,\ldots, n\}} \sum_{b \in B_i^n} \alpha^{n-i} \beta^i b.
%     \end{equation}
%     where $B_i^n = \{\ket{x_1x_2\ldots x_n}: \text{$x_j \in \{0,1\}$ and only $i$ of the $x_j$'s are 1}\}$. 
%     Note that, under $U$, we have
%     \begin{align*}
%         \ket{00\ldots00} &\mapsto \ket{00\ldots00}, \\
%         \ket{11\ldots10} &\mapsto \ket{11\ldots11}.
%     \end{align*}
%     Using Equation \ref{eq:n-tensor-product}, the coefficient of $\ket{00\ldots00}$ in $\ket{\varphi}^{\otimes n} \otimes \ket0$ is $\alpha^n$, and the coefficient of $\ket{11\ldots10}$ is $\beta^n$.

%     Conversely, the coefficient of $\ket{00\ldots00}$ in $\ket{\varphi}^{\otimes (n+1)}$ is $\alpha^{n+1}$ and the coefficient of $\ket{11\ldots11}$ is 
% \end{solution}