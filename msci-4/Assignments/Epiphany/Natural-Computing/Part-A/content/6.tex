\section{Ant-Q in comparison to other heuristic algorithms}

Ant-Q was compared against four different heuristic algorithms, which we briefly describe below. Note these comparisons were done on the TSP, not ATSP. We encourage the reader to ponder why some of the following methods do not apply for ATSP (for EN and SOM) or how they may be adapted (for SA and FI); economy prohibits further discussion on this here.

\begin{description}
    \item[Elastic net (EN)] This algorithm \cite{durbin1987analogue} starts with $m$ points with $m \gg n$ lying on a circular ring located at the centre of the cities  (this ring can be arbitrarily constructed) embedded on $\mathbb R^2$. The rubber band is then gradually elongated until it passes sufficiently near each city define a tour, by two forces: one for minimising the length of the ring (corresponding to the tour length) and one for minimising the distance between cities and points on the ring.
    \item[Simulated annealing (SA)] This algorithm \cite{van1987simulated}, like Ant-Q, takes inspiration from real-world phenomena. It is an iterative process in which there is a running variable, which we will call \emph{entropy}, which is decreased every step. We first pick an initial tour (this can be found using any method). Then, every step we will randomly generate a set of random transformations to the tour, and each perturbation is assigned a \emph{cost difference} (how much better the transformation makes the tour). If the cost difference is positive, the transformation is applied. Otherwise, it stochastically decided whether or not to apply the transformation depending on the entropy (more entropy leads to picking unfavourable transformations more) and the cost difference. If after a step no perturbations are observed, we output the tour. 
    \item[Self-organising map (SOM)] This algorithm \cite{kohonen1990self} is similar to the EN algorithm. We pick $m$ \emph{neurons} and form them as a ring on $\mathbb R^2$ with the cities embedded. We iteratively select a random city and move the closest neuron closer to that city (the amount moved is dictated by the \emph{learning rate}). We also move the neighbours (identified using a radius around the neuron) of the closest neuron. The radius at which neighbours are identified is gradually reduced with each iteration, as well as the learning rate. 
    \item[Farthest insertion (FI)] This algorithm \cite{nicholson1967sequential} starts by picking the two cities farthest apart, say $s_1$ and $s_2$. We then pick $s_3$ as the city that is farthest away from the edge $(s_1, s_2)$. We then form the mini-tour $T = (s_1, s_2, s_3, s_1)$. Subsequent cities are added as follows: pick the point $s_k$ farthest from any point that has been incorporated into $T$ and note the edge $(s_i, s_j)$ in $T$ which is closest to $s_k$. We replace the edge $(s_i, s_j)$ in $T$ with $(s_i, s_k, s_j)$. This is repeated until all edges are incorporated. 
\end{description}

FI improved with local optimisation (FI2 and FI3), SA improved with local optimisation (SA3), and an improved version of SOM (SOM+) was also compared against Ant-Q (algorithms introduced in \cite{lin1965computer}) and can be seen in Table 5 of \cite{gambardella1995ant}. It is also noted that the local optimisation heuristics were also applied to Ant-Q and found no benefit (that is, solutions from Ant-Q are locally optimal with respect to the 2-opt and 3-opt heuristics). Each algorithm (and improvements) were executed on a set of 5 50-city problems. Ant-Q was performed with iteration-best delayed reinforcement and pseudo-random-proportional action choice. Both Ant-Q and SA3 performed equally the best, finding the best runs in 4 of the 5 cities.

It is outlined that, although the experiment is promising, it does not pose a feasible solution to large TSP problems. Ant-Q's iteration complexity is $O(mn^2)$, and there exists better heuristic methods to tackle large problems. Instead, Ant-Q's strengths lie with ATSP problems. ATSP is a much harder problem than TSP, but Ant-Q's iteration complexity does not increase in this case. Ant-Q was compared against some exact algorithms for ry48p and 43X2 (both ATSP problems) and was found to get the optimal route as well as having mean route length close to this optimal. It even managed to compute the solution for 43X2 when the exact algorithm FT-92 was unable after 32 hours of running time. 