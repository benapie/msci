\documentclass[a4paper, 11pt]{report}

\usepackage{graphicx}
\usepackage{amsmath, amssymb}
\usepackage{bm}
\usepackage{siunitx}
\usepackage{hyperref}
\usepackage{booktabs}

%%%
%%% Maths shortcuts
%%%
\newcommand{\N}{\mathbb N}
\newcommand{\Z}{\mathbb Z}
\newcommand{\R}{\mathbb R}
\newcommand{\pt}{\text{pt}}
\DeclareMathOperator{\id}{id}
\DeclareMathOperator{\im}{im}
\DeclareMathOperator{\coker}{coker}

%%%
%%% Tikz
%%%
\usepackage{tikz}
  \tikzstyle{vertex}=[circle,fill=black!25,minimum size=12pt,inner sep=2pt]
\usepackage{tikz-cd}

%%%
%%% Title page
%%%
\usepackage{titling}
\pretitle{%
  \vspace{-4em}
  \begin{center}
    \huge
    \includegraphics[width=8em]{images/logo}\\[\bigskipamount]
    \vspace{1em}
    }

    \posttitle{
  \end{center}
  \vspace{1em}
}

%%%
%%% Referencing
%%%
\usepackage[sorting=none, backend=biber]{biblatex}
\addbibresource{ref.bib}


%%%
%%% Theorems
%%%
\usepackage{amsthm}

\theoremstyle{plain}
\newtheorem{theorem}{Theorem}[chapter]
\newtheorem{lemma}[theorem]{Lemma}
\newtheorem{corollary}[theorem]{Corollary}
\newtheorem{proposition}[theorem]{Proposition}

\theoremstyle{definition}
\newtheorem{definition}[theorem]{Definition}
\newtheorem{example}[theorem]{Example}
\newtheorem{remark}[theorem]{Remark}

\newtheoremstyle{break}{\topsep}{\topsep}{}{}{\bfseries}{}
{\newline}{\thmname{#1}\thmnumber{ #2}\thmnote{ ({\normalfont\textsc{#3}})}.}
\theoremstyle{break}
\newtheorem{problem}[theorem]{Problem}