\chapter{Conclusions and further work}

In Chapters \ref{cha:background} and \ref{cha:computational-complexity}, we built up the required theory to understand the computation of persistent homology. We then moved to understand the problems and corresponding algorithms for both: the construction of Vietoris-Rips filtrations (Section \ref{sec:vietoris-rips-construction}; and the computation of the persistent barcodes (Section \ref{sec:computing-persistent-homology}), presenting empirical evidence to the real-world run time of such algorithms. We first outline the main findings from our experiments.

We first introduced the construction of Vietoris-Rips filtrations, where we concluded that the methods provided by sklearn \cite{scikit-learn} gave exceptionally increased performance over the brute-force algorithm. For the expansion method, we found that the incremental algorithm outperformed the similar inductive algorithm, which both greatly improved on the brute-force algorithm.

We then moved to the persistent homology algorithm, where we introduced the standard algorithm \cite{edelsbrunner2000topological} and posed three \emph{speed-ups} that allowed, what is effectively the same algorithm at the core, to run significantly faster. We first showed the benefit of \emph{sparse matrix representation}, allowing us to avoid storing redundant elements in sparse matrices. We then moved to show a \emph{clearing} technique \cite{chen2011persistent} that allowed significantly less column operations to be performed. We then concluded on the use of the coboundary matrix for reduction \cite{de2011dualities}, which provided more opportunities in the use of the clearing technique. 

We then moved onto potential applications of persistent homology in Chapter \ref{cha:applications}, showing how our framework can be applicable to many different problems. Note that one of the main challenges found in persistent homology is the interpretation of persistent barcodes. Uses of persistent homology on point-cloud data is vast \cite{otter2017roadmap}, thus we focussed on more novel techniques. We showed how Betti numbers can give a measure on the complexity of a graph, both in the context of protein-protein interaction graphs and for control-flow graphs. We additionally posed the use of power filtration of a contact graph to determine vaccination strategies to reduce the spread of infectious disease. 

We conclude our work here, not without a highlight of potential topics for further research, as well as some further study for engrossed readers.

The following is some topics posed as further work.

\begin{description}
    \item[HNSW] Hierarchical navigable small-world networks (HNSW) \cite{malkov2018efficient} are positioned as the current state-of-the-art algorithm, for \textsc{$\varepsilon$-NNs}, but application of this technique to the skeleton method in the Vietoris-Rips construction step in the pipeline of persistent homology is yet to be seen. 
    \item[Theoretical complexity bounds for expansion methods] Our understanding on the inherent \emph{difficulty} of the expansion methods seem much more scarce, and more research into the underlying complexity theory would be beneficial.
    \item[Persistent homology with coefficients in a non-divisible group] Substantial research has been conducted studying efficient methods of calculating the Smith normal form of an integer matrix \cite{havas1998extended}, but applying such a method to persistent homology as well as existing speed-ups we have outlined is yet to be seen. We regret the omission of such content; however, the required context delves deep into number theory which we could respectfully fit here. We do note a need for a rework of theoretical foundations needed for such an undertaking, as our succinct decomposition of persistence modules relies on coefficients in a field. Additional considerations must also be taken by the use; that is, does torsional information induced by such coefficients give any useful insight.
    \item[Alternative sparse matrix representations] In our exposition, we outlined one of many techniques for storing our boundary matrix. Many alternatives exist, and the use of different structures may lead to input-dependent benefits. We pose not only the use of sorted tree-like data structures, but also the use of more unique data structures such as \emph{simplex trees}, introduced by \cite{boissonnat2014simplex}.
    \item[Persistent homology as an analogue to cyclomatic complexity] \textcite{huntsman2020path} posed the use of \emph{path homology} as a stronger analogue of cyclomatic complexity. In a similar vein, can we derive useful measures of program complexity from persistent homology (such as using the power filtration)? Here, we have simply posed the persistent homology algorithm as a fast method of computing cyclomatic complexity. 
    \item[Relationship between persistent homology and other complexity measures on graphs] A related proposal to the previous, are there any relationships between graph (or complex) complexity measures derived from persistent homology and other, more established graph (or complex) complexity measures (such as clustering coefficients, assortativity, etc.).
    \item[Novel filtrations as inputs to the persistent homology algorithms] The use of the persistent homology algorithms lies, almost exclusively, in the choice of filtration we build from our data. For example, the we have shown the \emph{power filtration} to give more insight into topological features of graphs, without needing any more information \cite{benzekry2015design}. To allow persistent homology to find applications in more fields, rigorous investigation is needed on complexes that may been constructed that allow us to interpret persistent homology in a useful way. 
\end{description}

The following is some further material, for interested readers.

\begin{description}
    \item[Spectral sequences] \emph{Spectral sequences} and \emph{persistent homology} are both tools within algebraic topology which are defined on a filtration, and can be applied to the study of the topological features of the underlying spaces. They are both deeply related, but spectral sequences require a much more rigorous familiarity with algebraic topology \cite{romero2013spectral}. Note that problems we outlined with non-divisible groups, such as $\mathbb Z$, are still present. 
    \item[Stability] We introduced persistent homology as a tool robust against noise. The \emph{bottleneck distance} is a natural distance function defined on the space of persistent diagrams, and the use of this function can be used to prove the robustness of persistent homology \cite{cohen2007stability}.
    \item[Cohomology and duality] For those interested in more theoretical study, a substantial amount of research has been conducted into the relations between homology and cohomology, providing particularly succinct results \cite{maunder1996algebraic, Hatcher:478079}.
\end{description}