\chapter{Some foundations of topology}
\label{app:topology}

First, we cover some elementary group theory.

\begin{definition}[Free abelian group]
  A \emph{free abelian group} on generators $\{e_\alpha\}_{\alpha \in \mathcal A}$ is the group of formal sums
  \[ \Z\langle\{e_\alpha\}_{\alpha \in \mathcal A}\rangle =
    \left\{\sum_{\alpha\in\mathcal A} n_\alpha e_\alpha: n_\alpha \in \Z, \;\text{finitely many $n_\alpha \neq 0$}\right\}. \]
\end{definition}

If $\mathcal A$ has size $k$, then $\Z\langle \{e_\alpha\}_{\alpha \in \mathcal A}\rangle \cong \Z^k$.

In general, we have a notion of direct product of groups; however, for free abelian groups we have the restriction that formal sums must have finitely many non-zero terms. Thus the direct product of infinite families of free abelian groups may not be free abelian. From this, we introduce the notion of \emph{direct sum}.

\begin{definition}[Direct sum]
  Let $\{A_i\}_{i \in I}$ be a family of abelian groups. Then the \emph{direct sum} is
  \[\bigoplus_{i \in I} A_i = \left\{ (a_i)_{i \in I} \in \prod_{i \in I} A_i: \text{finitely many $a_i \neq 0$} \right\}.\]
\end{definition}

We now recall some elementary topology.

\begin{definition}[Topological space]
  A \emph{topological space} (or just \emph{space}) is an ordered pair $(X, \tau)$ where $X$ is a set and $\tau \subset \mathcal P(X)$ such that
  \begin{enumerate}
    \item the empty set and $X$ both belong to $\tau$;
    \item any arbitrary union of members of $\tau$ belongs to $\tau$; and
    \item any finite intersection of members of $\tau$ belongs to $\tau$.
  \end{enumerate}
  $\tau$ is called a \emph{topology} on $X$, the elements of $\tau$ are called the \emph{open sets} of $X$.
\end{definition}

Note we may just refer to a space $(X, \tau)$ as $X$. We may also define a topology in terms of a basis: a basis $\mathcal B$ of a space $X$ is a collection of open sets such that every open set is the union of elements from $\mathcal B$.

\begin{definition}[Subspace topology]
  Given a topological space $(X, \tau)$ and a subspace $S \subset X$, the \emph{subspace topology} on $S$ is defined by \[\tau_S = \{S \cap U: U \in \tau\}.\]
\end{definition}

\begin{example}[Standard topology on $\R^n$]
  A topology may be derived from any metric, and when we refer to the \emph{standard topology} in $\R^n$ we will assume it is the one induced from the Euclidean metric: the distance from $\bm x$ to $\bm y$ is given by
  \[d(\bm x, \bm y) = \sqrt{\sum_{i=1}^n (x_i - y_i)^2}. \]
  The topology induced by this is generated by the basis of open balls.
\end{example}

\begin{example}[Standard subsets of $\R^n$]
  When considering subsets of $\R^n$, we assume the subspace topology of the standard topology.
  \begin{itemize}
    \item (Closed unit interval) $I = \{x \in \R: x \in [0,1]\}$.
    \item (Closed $n$-disc) $D^n = \{\bm x \in \R^n: \lVert \bm x \rVert \leq 1\}$.
    \item (Open $n$-disc) $E^n = \{\bm x \in \R^n: \lVert \bm x \rVert < 1\}$.
    \item ($n$-sphere) $S^n = \{\bm x \in \R^n: \lVert \bm x \rVert = 1\}$.
  \end{itemize}
\end{example}

\begin{definition}[Product topology]
  Let $X$ and $Y$ be spaces. The \emph{cartesian product} of $X$ and $Y$ is
  \[X \times Y = \{(x,y): x \in X, y \in Y\}\]
  and may be given the \emph{product topology} generated by the basis
  \[\mathcal B_{X \times Y} = \{U \times V: \text{$U$ is open in $X$ and $V$ is open in $Y$}\}.\]
\end{definition}

\begin{example}[Product constructions]
  The product construction gives us some more basic spaces.
  \begin{itemize}
    \item (Torus) $\mathbb T = S^1 \times S^1 = \{(x, y): x,y \in S^1\}$.
    \item (Cylinder) $S^1 \times I = \{(x, y): x \in S^1, y \in [0,1]\}$.
  \end{itemize}
\end{example}

Let us define our first morphism on spaces.

\begin{definition}[Continuous]
  Let $X$ and $Y$ be spaces. A function $f: X \to Y$ is called \emph{continuous} if for every open subset $U \subset Y$, $f^{-1}(U) \subset X$ is also open.
\end{definition}

We may refer to a continuous function as a \emph{map}.

\begin{definition}[Homeomorphism]
  Let $X$ and $Y$ be spaces. A map $h: X \to Y$ is called a \emph{homeomorphism} if $h$ is bijective and $h^{-1}$ is continuous.
\end{definition}

If a homoemorphism exists between two spaces, they are said to be homeomorphic and we write $X \cong Y$, and indeed being homeomorphic is an equivalence relation on the set of spaces. If $X$ and $Y$ are homeomorphic, we may imagine that we can continuously stretch and bend $X$ into $Y$ (if we think of the spaces as geometric objects).

\begin{definition}[Quotient topology]
  Let $X$ be a space and $\sim$ be some equivalence relation on $X$. Consider the canonical projection map $\pi: X \to X/{\sim}$, $x \mapsto [x]$ (this map is surjective, so it has an inverse). We equip the set $X/{\sim}$ with the topology defined by: $U \subset X/{\sim}$ is open if and only if $\pi^{-1}(U)$ is open in $X$, we call this the \emph{quotient topology}.
\end{definition}

We can also build some of our elementary spaces using a quotient construction.

\begin{example}
  We consider $S^1$. We can build this space (up to homeomorphisms) by gluing the two ends of an interval. That is, $S^1 \cong I/{\sim}$ where $0 \sim 1$. Intuitively, you would be inclined to believe; however, for this simple example we will be thorough. To be clear, $S^1 = \{(x,y) \in \R^2: \lVert x \rVert = 1\}$ and $I = \{x \in \R: x \in [0,1]\}$. We note that we can parametrise $S^1$ as $S^1 = \{(\cos t, \sin t) \in \R^2: t \in [0, 2\pi)\}$. From this, we define the homeomorphism $h: S^1 \to I/{\sim}$ by $(\cos t, \sin t) \mapsto \frac1{2\pi}t$. Checking that $h$ is continuous, $h$ is a bijection, and $h^{-1}$ is continuous is easy to do.
\end{example}

\begin{example}
  Consider $\mathbb T$. We construct the torus out of the square $I \times I$, identifying the top and bottom points as well as the left and right points. That is, $\mathbb T = (I \times I)/{\sim}$ where $(x,0) \sim (x, 1)$ and $(0,y) \sim (1,y)$ for all $x,y \in I$. This is indeed homeomorphic to the product construction.
\end{example}

\begin{example}
  We can use this quotient construction to create some variations of simple spaces. For example, we build the Klein bottle $\mathbb K$ much like the torus, but with one of the identifications running backwards. That is, $\mathbb K = (I \times I)/{\sim}$ where $(x,0) \sim (x,1)$ and $(0,y) \sim (1, 1-y)$ for all $x,y \in I$.
\end{example}

We now define another equivalence relation on maps.

\begin{definition}[Homotopic]
  Two maps $f, g: X \to Y$ are \emph{homotopic}, written $f \simeq g$, if there is a \emph{homotopy} $H$, a map $H: X \times I \to Y$ with
  \begin{align*}
    H(x,0) & = f(x), \\
    H(x,1) & = g(x)
  \end{align*}
  for all $x \in X$.
\end{definition}

Establishing this as an equivalence relation is elementary. Two maps being homotopic can be thought of as being able to continuously deform $f$ into $g$. This relation also induces another relation on spaces.

\begin{definition}[Homotopy equivalent]
  Two spaces $X$ and $Y$ are \emph{homotopy equivalent}, written $X \simeq Y$ if there are maps $\alpha: X \to Y$ and $\beta: Y \to X$ such that $\alpha\circ\beta \simeq \id_Y$ and $\beta\circ\alpha \simeq \id_X$. 
\end{definition}

\begin{example}
  We denote $\pt$ for the one point space. We will show that $D^2 \simeq \pt$. To show this we define $\alpha: D^2 \to \pt$ the only way we can, and $\beta: \pt \to D^2$ by $\beta(\pt) = \bm 0$. We have $\alpha\circ\beta = \id_\pt \simeq \id_\pt$ and $\beta\circ\alpha = \bm 0 \simeq \id_{D^2}$ by the homotopy $H(\bm x, t) = tx$. 
\end{example}

\begin{definition}[Contractible]
  We say a space $X$ is \emph{contractible} if it homotopy equivalent to $\pt$. 
\end{definition}
