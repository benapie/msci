\subsection{Expansion methods}

Methods presented here are introduced by \textcite{zomorodian2010fast}, as well as the complexity analysis. 

The expansion of a 1-skeleton effectively \emph{fills in} higher dimensional simplices whose faces are present, and assigns a weight to these simplices. That is, for a graph $G$, calculate the clique complex $\Cl(G)$ with a corresponding weight function $w: \Cl(G) \to \mathbb R_{\geq 0}$. 

We have seen that we can easily compute the weights on the simplices (Lemma \ref{lem:weight-definition-equivalent}), thus we focus on the construction of the clique complex.

\begin{problem}[CliqueComplex]
    Instance: let $G = (V,E)$ be a graph. \\
    Question: compute $\Cl(G)$. 
\end{problem}

We are now discarding all embedding information of $S$, and focusing solely on the topological features of the $1$-skeleton given by the skeleton method. 

We first introduce the \emph{inductive expansion algorithm}. We fix some arbitrary ordering on the vertices of our 1-skeleton, to give meaning to the \textsc{LowerNghbrs} function in Algorithm \ref{alg:inductive-algorithm}.

\begin{algorithm}
    \caption{The inductive expansion algorithm.}
    \label{alg:inductive-algorithm}
    \begin{algorithmic}
        \Function{LowerNghbrs}{graph $G$, vertex $u \in G$}
        \State \Return $\{v \in V(G): v < u, \{u,v\} \in E(G)\}$
        \EndFunction
        \Function{InductiveExpansion}{graph $G$, level $k \in \{2, 3, \ldots\}$}
        \State $K \gets V(G) \cup E(G)$
        \For{$i \gets 1$ to k}
        \For{each $i$-simplex $\tau \in K$}
        \State $N \gets \cap_{u \in \tau} \textsc{LowerNghbrs}(G, u)$
        \For{each $v \in N$}
        \State $K \gets K \cup \{\tau \cup \{u\}\}$
        \EndFor
        \EndFor
        \EndFor
        \EndFunction
    \end{algorithmic}
\end{algorithm}

We briefly share some intuition of Algorithm \ref{alg:inductive-algorithm} (a sketch of correctness): the skeleton of this code follows closely to the brute force method. For each $i$-simplex, we check if it forms the face of a $(i+1)$-simplex (using the fixed ordering to avoid double checking simplices). If it does, we add that $(i+1)$-simplex to the complex.

Analysing the complexity of expansion algorithms is tricky as we have an output that has exponential size (in the size of the input), but we can form a rough lower bound on the first level of expansion: from line segments to triangles. For each edge, we look at the (lower) neighbours of the endpoints to compute the set of shared (lower) neighbours. We then perform a constant time operation on each of the shared neighbours, thus we get the lower bound $\sum_{v \in V} \left(\deg v\right)^2$. This was was bounded by \textcite{de1998upper}:
\[
    \frac{4\lvert E \rvert^2}{\lvert V \rvert} \leq \sum_{v \in V} \left(\deg v\right)^2 \leq \lvert E \rvert \left(
    \frac{2\lvert E \rvert}{\lvert V \rvert - 1} + \lvert V \rvert - 2
    \right).
\]
Thus, the first level of expansion runs in time $O(n^3)$. We repeat the subsequent levels by a similar sum on the degrees on the triangles, tetrahedron, etc.; however, we typically fix the number of dimensions to expand into, so we take the overall time complexity to be $O(n^3)$. Further research is needed into the restriction of this bound.

We highlight a key inefficiency in the inductive algorithm (to motivate the incremental algorithm): when we compute the neighbours of a simplex, we repeat computations already done for its faces. Thus, instead of inducting on dimension, we can incrementally add vertices and construct cofaces for which the vertex is maximal. 

\begin{algorithm}
    \caption{The incremental expansion algorithm, where we induct on the dimension.}
    \label{alg:incremental-algorithm}
    \begin{algorithmic}
        \Function{AddCofaces}{graph $G$, level $k$, simplex $\tau$, $N$, $K$}
        \If{$\dim\tau \geq k$}
        \State \Return
        \EndIf
        \For{each $v \in N$}
        \State $\sigma \gets \tau \cup \{v\}$
        \State $M \gets N \cap \textsc{LowerNghbrs}(G, v)$
        \State $\textsc{AddCofaces}(G, k, \sigma, M, K)$
        \EndFor
        \EndFunction
        \Function{IncrementalExpansion}{$G$, $k$}
        \State $K \gets \varnothing$
        \For{each $u \in V(G)$}
        \State $N \gets \textsc{LowerNghbrs}(G, u)$
        \State $\textsc{AddCofaces}(G, k, \{u\}, N, K)$
        \EndFor
        \EndFunction
    \end{algorithmic}
\end{algorithm}

In Algorithm \ref{alg:incremental-algorithm}, we start with an empty complex. We than add all the cofaces of each vertex for which the vertex is maximal. The initial calls of \textsc{AddCofaces} in \textsc{IncrementalExpansion} will add all simplices (of the $k$-skeleton) for which $u$ is a maximal vertex. 

Further restriction of the worst-case running time (than the one presented for the inductive algorithm) is yet to be seen. 
