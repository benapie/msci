\chapter{Problems in the computation of persistent homology}
\label{cha:problems}

Here we apply the knowledge built up in Chapter \ref{cha:background} and Chapter \ref{cha:computational-complexity} to introduce some problems in the computation of persistent homology. As previously mentioned, we focus on two main steps in the pipeline:
\begin{enumerate}
    \item construction of a filtered simplicial complex (Section \ref{sec:vietoris-rips-construction}); and
    \item computation of the corresponding persistent barcodes (Section \ref{sec:computing-persistent-homology}).
\end{enumerate}
For (i), we focus on Vietoris-Rips filtrations, as it is the prevalent complex used for point-cloud data \cite{otter2017roadmap}. 

We conduct many experiments in this section. All experiments were computed on a 6-Core processor with \SI{3.60}{\giga\hertz} base clock speed, and \SI{16}{\giga\byte} RAM with clock speed \SI{3733}{\mega\hertz}. The programming language used is \emph{Python} \cite{10.5555/1593511}. Source code for this project can be found in Appendix \ref{app:source-code}.

\documentclass[a4paper, 11pt]{article}

\usepackage[utf8]{inputenc}

\usepackage{parskip}

\usepackage{amssymb}
\usepackage{amsmath}
\usepackage{amsfonts}
\usepackage{mathtools}

\usepackage{todonotes}

\usepackage{csquotes}

\usepackage{algpseudocode}
\usepackage{algorithm}

\DeclareMathOperator{\AQ}{AQ}
\DeclareMathOperator{\DAQ}{\Delta AQ}
\DeclareMathOperator{\Q}{Q}
\DeclareMathOperator{\HE}{HE}
\DeclareMathOperator*{\argmax}{arg\,max}
\DeclareMathOperator{\Ep}{Ep}

\usepackage{braket}

\addtolength{\oddsidemargin}{-.875in}
\addtolength{\evensidemargin}{-.875in}
\addtolength{\textwidth}{1.75in}
\addtolength{\topmargin}{-.875in}
\addtolength{\textheight}{1.75in}

\usepackage[backend=biber]{biblatex}
\addbibresource{ref.bib}

\title{Natural Computing Part A}
\author{Ben Napier}
\date{March 2022}

\begin{document}

\section{Lie groups}

In this section, when we refer the \emph{Lie groups} we actually mean \emph{linear Lie groups}.

\subsection{Definition}

\begin{definition}[Lie group]
    A \emph{Lie group} is a closed subgroup of $\GL_n(\C)$ for some $n$.
\end{definition}

We mean \emph{closed} in the topological sense, a more correct definition would include details on \emph{smooth manifolds}, which we do not focus on here.

We denote $\gl[n, \mathbb K]$ as the set of $n \times n$ matrices with entries in $\mathbb K$.

We now generalise the exponential map $\exp$ to matrices, using the familiar power series.

\begin{definition}[Exponential map]
    Let $X \in \gl[n, \mathbb K]$. Then define $\exp: \gl[n, \mathbb K] \to \gl[n, \mathbb K]$ by
    \[ \exp(X) = \sum_{i=0}^\infty \frac{X^i}{i!}. \]
\end{definition}

We will omit details on the convergence of $\exp$, but it is convergent for all matrices and can be proved using the Cauchy-Schwartz. In particular, by considering the entry-wise norm.

We have some properties of the exponential map, and comment that it behaves similarly to the normal $\exp$. For all $X, Y, g \in \gl[n, \mathbb K]$ where $g$ is invertible, and $s, t \in \mathbb K$, we have the following.

\begin{itemize}
    \item $\exp(0) = I$
    \item $\exp(X + Y) = \exp(X) \exp(Y)$
    \item $(\exp(X))^{-1} = \exp(-X)$
    \item $\exp(sX)\exp(tX) = \exp((s+t)X)$
    \item $g\exp X g^{-1} = \exp(gXg^{-1})$
\end{itemize}

\begin{proposition}
    $\exp: \gl[n, \C] \to \gl[n, \C]$ is differentiable at zero (the zero matrix), and its derivative at the origin is $I$.
\end{proposition}

\begin{corollary}
    $\exp: \gl[n, \C] \to \gl[n, \C]$ is a local diffeomorphism at zero.
\end{corollary}

By this, we mean that $\exp$ has an inverse near zero.

We note that $\exp: \gl[n, \C] \to \gl[n, \C]$ is \emph{not} injective. In particular, it coincides with our regular exponential map at $n = 1$, so $\exp(2\pi i k) = 1$ for all $k \in \Z$.

\begin{lemma}
    $\exp: \gl[n, \C] \to \GL_n(\C)$ is surjective.
\end{lemma}

We not that $\exp: \gl[n, \R] \to \GL_n(\R)$ is \emph{not} surjective. Again, for $n = 1$ we see that $\exp$ is strictly positive.

\begin{proposition}
    $\det\exp = \exp\tr$.
\end{proposition}

\begin{proof}
    Let $X \in \gl[n, \K]$. We can conjugate $X$ so that it is upper triangular. The result is then immediate.
\end{proof}

\subsection{One-parameter subgroups}

We have seen that $\exp((s+t)X) = \exp(sX)\exp(tX)$, thus for all $X \in \gl[n, \C]$ we can define a group homomorphism $f: \R \to \GL_n(\C)$ such that $t \mapsto \exp(tX)$ (here $\R$ is given standard addition $+$).

Similarly, if we consider $G = \SO(2)$ then we can define a group homomorphism $\R \to \SO(2)$ by  $t \mapsto \text{rotation by $t$}$. We note that
\[
    \begin{pmatrix}
        \cos t  & \sin t \\
        -\sin t & \cos t
    \end{pmatrix}
    = \exp
    \begin{pmatrix}
        0 & -t \\
        t & 0
    \end{pmatrix}.
\]

\begin{definition}[One-parameter subgroup]
    Let $G$ be a Lie group. A \emph{one-parameter subgroup} is a differentiable group homomorphism $\gamma: (\R, +) \to G$. The matrix $\gamma'(0)$ is the \emph{infinitesimal generator}.
\end{definition}

\begin{theorem}
    Let $f: \R \to \GL_n(\C)$ be a one-parameter subgroup with infinitesimal generator $X$. Then
    \[ f(t) = \exp(tX). \]
\end{theorem}

\subsection{Lie algebras}

\begin{definition}[Lie algebra of a Lie group]
    Let $G$ be a Lie group. Its \emph{Lie algebra} is
    \[ \mathfrak g = \left\{
        X \in \gl[n, \C]: \exp(\R X) \subset G
        \right\}. \]
\end{definition}

We can alternatively define $\mathfrak g$ as the set of infinitesimal generators of all one-parameter subgroups.

\begin{proposition}
    Let $G$ be a Lie group and $\mathfrak g$ its Lie algebra. Then
    \[ \mathfrak g = \left\{
        X \in \gl[n, \C]: \text{$X = \gamma'(0)$ for some map $\gamma:[-a, a] \to G$ where $a > 0$}
        \right\}. \]
\end{proposition}

We may denote the Lie algebra of a Lie group $G$ by $\Lie(G)$.

\begin{example}[Some Lie algebras]\hspace{0em}
    \begin{itemize}
        \item $\Lie(\GL_n(\K)) = \gl[n, \K]$
        \item $\Lie(\SL_n(\K)) = \sl[n, \K] = \{X \in \gl[n, \K]: \tr(X) = 0\}$
        \item $\Lie(\O(n)) = \mathfrak o_n = \Lie(\SO(n)) = \mathfrak{so}_n = \{X \in \gl[n, \R]: X + X^\intercal = 0\}$
        \item $\Lie(\U(n)) = \mathfrak{u}_n = \{X \in \gl[n, \C]: X + X^\dagger = 0\}$
        \item $\Lie(\SU(n)) = \mathfrak{su}_n = \{X \in \mathfrak u_n: \tr(X) = 0\}$
    \end{itemize}
\end{example}

\begin{proposition}
    Let $\mathfrak g$ be the Lie algebra of a Lie group $G$. Then
    \begin{enumerate}
        \item $\mathfrak g \subset \gl[n, \C]$ is a real vector space;
        \item if $X \in \mathfrak g$ and $g \in G$, then $g X g^{-1} \in \mathfrak g$; and
        \item if $X, Y \in \mathfrak g$, then
              \[ [X,Y] := XY - YX \in \mathfrak g. \]
    \end{enumerate}
\end{proposition}

We now define a Lie algebra separate from a Lie group.

\begin{definition}[Lie algebra]
    A \emph{Lie algebra} $\mathfrak g$ is an $\R$-vector space with a bilinear map (called the \emph{Lie bracket}) $[-,-]: \mathfrak g^2 \to \mathfrak g$ such that
    \begin{enumerate}
        \item for all $X, Y \in \mathfrak g$, $[X,Y] = -[Y,X]$; and
        \item the \emph{Jacobi identity} holds: for all $X, Y, Z \in \mathfrak g$
              \[ [X, [Y,Z]] + [Y, [Z, X]] + [Z, [X,Y]] = 0. \]
    \end{enumerate}
\end{definition}

A \emph{Lie subalgebra} $\mathfrak h \subset \mathfrak g$ of a Lie algebra $\mathfrak g$ is a subspace which is closed under the Lie bracket.

\begin{example}
    Consider $\mathfrak g = \R^3$ with $[\bm v, \bm w] = \bm v \times \bm w$. Then $\mathfrak g$ is a Lie algebra, and in fact $\mathfrak g \cong \mathfrak{so}_3$.
\end{example}

The center of a Lie group is an abelian subgroup of $\mathfrak g$.

\begin{definition}
    A Lie algebra $\mathfrak g$ is \emph{abelian} if $[X,Y] = 0$ for all $X, Y \in \mathfrak g$. The \emph{center} of $\mathfrak g$ is
    \[ Z(\mathfrak g) = \{Z \in \mathfrak g: \text{$[Z,X] = 0$ for all $X \in \mathfrak g$}\}. \]
\end{definition}

\begin{definition}[Complex Lie group]
    A complex Lie group is a closed subgroup of $\GL_n(\C)$ whose Lie algebra is a complex subspace of $\gl[n, \C]$.
\end{definition}

\subsection{Morphisms}

\begin{definition}
    A \emph{Lie group homomorphism} $\phi: G \to G'$ between two Lie groups is a continuous group homomorphism.
\end{definition}

As usual, a Lie group isomorphism is a homomorphism which is bijective with continuous inverse.

\begin{definition}
    A \emph{Lie algebra homomorphism} $\varphi: \mathfrak g \to \mathfrak h$ is an $\R$-linear map such that for all $X, Y \in \mathfrak g$,
    \[ \varphi([X,Y]) = [\varphi(X), \varphi(Y)]. \]
\end{definition}

A Lie algebra isomorphism is an invertible homomorphism.

\begin{definition}[Derivative]
    Let $\phi: G \to H$ be a Lie group homomorphism. Define the \emph{derivative} of $\phi$ as
    \begin{align*}
        D\phi: \mathfrak g & \to \mathfrak h,                           \\
        D\phi(X)           & = \frac{d}{dt} \phi(\exp(tX))\bigg|_{t=0}.
    \end{align*}
\end{definition}

\begin{theorem}
    Let $\phi: G \to H$ be a Lie group homomorphism. Then
    \begin{enumerate}
        \item the diagram
              \begin{center}
                  % https://tikzcd.yichuanshen.de/#N4Igdg9gJgpgziAXAbVABwnAlgFyxMJZABgBpiBdUkANwEMAbAVxiRAB12BbOnACwBmAJzoBrAAQBzEAF9S6TLnyEUARnJVajFm049+wseL6z5IDNjwEiZVZvrNWiEAHFTCy8qLq71BzucACVlNGChJeCJQYQguJDIQHAgkdS1HNgARTjQ+LHcQGLjEACZqJKQAZj9tJw52HLy5aKFY+LLkkur0504YAA80fMKU9squgLr+wZkKGSA
                  \begin{tikzcd}
                      \mathfrak g \arrow[r, "D\phi"] \arrow[d, "\exp"] & \mathfrak h \arrow[d, "\exp"] \\
                      G \arrow[r, "\phi"]                              & H
                  \end{tikzcd}
              \end{center}
              commutes;
        \item for $g \in G$ and $X \in \mathfrak g$, we have
              \[ D\phi(gXg^{-1}) = \phi(g)D\phi(X)\phi(g)^{-1}; \]
        \item $D\phi$ is a Lie group homomorphism.
    \end{enumerate}
\end{theorem}

\begin{definition}
    Let $\phi: G \to H$ be a Lie group homomorphism. Then $\phi$ is \emph{holomorphic} if $D\phi$ is $\C$-linear.
\end{definition}

\begin{example}
    For an non-example, $\det: \GL_2(\C) \to \GL_1(\C)$ is \emph{not} holomorphic. 
\end{example}

\subsection{Representations of Lie groups}

We omit the definition of a representation of a Lie group $(\rho, V)$. The only difference to our normal definition is that $\rho$ is a Lie group homomorphism. Similarly, a representation of a Lie algebra $(\sigma, W)$ is the same but $\sigma$ is a Lie algebra homomorphism. 

We highlight a key difference to our traditional representations: a representation $(\rho, V)$ of a Lie algebra $\mathfrak g$ need not satisfy $\rho(XY) = \rho(X)\rho(Y)$. In fact, it is not even certain that $XY \in \mathfrak g$. Our definition required only that $\rho$ is $\R$-linear and $\rho$ commutes with the Lie bracket $[-,-]$. 

Our notions of $G$-homomorphisms, isomorphisms, subrepresentations, and irreducibility still hold as normal for representations of Lie groups and Lie algebras. 

If $G$ is a complex Lie group, then a holomorphic representation of $G$ is a complex representation whose derivate is $\C$-linear.

\begin{theorem}
    Let $A$ be a Lie group or a Lie algebra.
    \begin{enumerate}
        \item If $V_1$ and $V_2$ are irreducible finite-dimensional representations of $A$, then
        \[
            \dim_A(V_1, V_2) =
            \begin{cases}
                1 & V_1 \cong V_2, \\
                0 & \text{else}.
            \end{cases}
        \]
        \item Any irreducible finite-dimensional representation of an abelian $A$ is $1$-dimensional.
        \item Let $(\rho, V)$. Then $\rho$ has a central character, defined on the center of $A$.
    \end{enumerate}
\end{theorem}

\begin{proposition}
    Let $(\rho, V)$ be a finite-dimensional representation of a Lie group $G$. 
    \begin{enumerate}
        \item If $W \subset V$ is invariant under $\rho(G)$, then it is invariant under $D\rho(\mathfrak g)$.
        \item If $D\rho$ is irreducible, then $\rho$ is irreducible.
        \item If $\rho$ is unitary, then $D\rho$ is skew-Hermitian.
        \item Let $(\rho', V')$ be another finite-dimensional representation of $G$. Then if $\rho \cong \rho'$, then $D\rho \cong D\rho'$.
    \end{enumerate}
    If $G$ is connected, the converse hold. 
\end{proposition}

Thus for connected Lie groups, we can test for irreducibility and isomorphisms at the level of Lie algebras.

\subsection{Standard constructions for representations of Lie groups}

Here we will present some standard constructions for representations of Lie groups. Derivatives are given and not proved, but this is not a difficult task (usually).

Let $G \subset \GL_n(\C)$ be a Lie group.
\begin{itemize}
    \item  We have the obvious action of $g \in G$ on $\mathbb C^n$:
    \[ \rho(g) = g, \qquad D\rho(X) = X. \]
    \item Let $(\rho, V)$ and $(\sigma, W)$ be two representations of $G$. The direct sum $\rho \oplus \sigma$ has derivative
    \[ D(\rho \oplus \sigma) = D\rho \oplus D\sigma. \]
    \item We have the determinant representation $\det: G \to \C$, with $D\det = \tr$. 
    \item For a representation $(\rho, V)$ of $G$, the dual representation $(\rho^*, V^*)$ is defined by
    \[ (\rho^(g)(\lambda))(v) = \lambda(\rho(g^{-1})(v)) \]
    for $\lambda \in V^*$. We have
    \[ D\rho^*(X)(\lambda)(v) = -\lambda(D\rho(X)v). \]
    \item For two representations $(\rho, V)$ and $(\sigma, W)$ of $G$, then the tensor product representation $(\rho \otimes \sigma, V \otimes W)$ is a representation where
    \[ (\rho \otimes \sigma)(g) = \rho(g) \otimes \sigma(g) \]
    and
    \[ D(\rho \otimes \sigma)(g) = D\rho(g) \otimes \id_W + \id_V \otimes D\sigma(g). \]
    \item We also have the symmetric powers and alternating powers, which we consider as quotients as the tensor product representation and thus we will omit here. 
\end{itemize}

\subsection{The adjoint representation}

Let $G$ be a Lie group and $\mathfrak g$ its Lie algebra. We have seen that $\mathfrak g$ is closed under conjugation by $G$; that is, for all $X \in \mathfrak g$ and $g \in G$, we have $gXg^{-1} \in \mathfrak g$. Thus we have an action on $\mathfrak g$ by conjugation, called the \emph{adjoint representation}.

\begin{definition}[Adjoint representation]
    Let $G$ be a Lie group and $\mathfrak g$ its Lie algebra. The \emph{adjoint representation} $(\Ad, \mathfrak g)$ of $G$ is defined by
    \begin{align*}
        \Ad: G &\to \GL(\mathfrak g), \\
        \Ad(X)g &= gXg^{-1}.
    \end{align*}
    We similarly have the \emph{adjoint representation} $(\ad, \mathfrak g)$ of $\mathfrak g$ where
    \begin{align*}
        \ad: \mathfrak g &\to \gl(\mathfrak g), \\
        \ad &= D\Ad.
    \end{align*}
\end{definition}

We may write $\Ad_g(X)$ instead of $\Ad(g)(X)$, a similarly $\ad_X(Y)$ instead of $\ad(X)(Y)$. 

We have seen that for Lie group homomorphisms, the following diagram commutes.
\begin{center}
    % https://tikzcd.yichuanshen.de/#N4Igdg9gJgpgziAXAbVABwnAlgFyxMJZABgBpiBdUkANwEMAbAVxiRAB12BbOnACwBmAJzoBrAAQBzEAF9S6TLnyEUARnJVajFm049+wseL6z5IDNjwEiZVZvrNWiEAHFTCy8qLq71BzucACVlNGChJeCJQYQguJDIQHAgkdS1HNgARTjQ+LHcQGLjEACZqJKQAZj9tJw52HLy5aKFY+LLkkur0504YAA80fMKU9squgLr+wZkKGSA
    \begin{tikzcd}
        \mathfrak g \arrow[r, "D\phi"] \arrow[d, "\exp"] & \mathfrak h \arrow[d, "\exp"] \\
        G \arrow[r, "\phi"]                              & H
    \end{tikzcd}
\end{center}
Thus 
\[ \Ad_{\exp{tX}} = \exp_{t\ad X}. \]

\begin{theorem}
    Let $G$ be a Lie group with Lie algebra $\mathfrak g$ and $X, Y \in \mathfrak g$.
    \begin{enumerate}
        \item $\ad_X(Y) = [X,Y] = XY - YX$
        \item $\ad$ is a Lie algebra homomorphism, so
        \[ \ad_{[X,Y]} = [\ad_X, \ad_Y] \]
        and the Jacobi identity holds. 
    \end{enumerate}
\end{theorem}

\begin{proposition}
    Let $G$ be a Lie group and $\mathfrak g$ its Lie algebra. If $G$ is abelian, so is $\mathfrak g$. If $G$ is connected, the converse holds. 
\end{proposition}

\subsection{Maschke's theorem}

The main corollary of Maschke's theorem was that we can decompose a representation into the directed sum of irreducible representations. But this does \emph{not} hold for infinite groups. 

For example, we consider $G = (\R, +)$ and a representation $(\rho, \C^2)$ given by
\[
    \rho(x) =
    \begin{pmatrix}
        1 & x \\
        0 & 1
    \end{pmatrix}.
\]
This is reducible, in particular, $\langle e_1 \rangle$ is invariant under $\rho(g)$ for all $g \in G$. But we claim this is not decomposable. Indeed, it can be shown that $e_1$ is the only eigenvector of $\rho(g)$ up to scalar.

\begin{theorem}
    Every finite-dimensional representation of a compact Lie group is decomposable.
\end{theorem}

Some compact Lie groups:
\begin{itemize}
    \item $U(n)$;
    \item $\SU(n)$; and
    \item $\SO(n)$.
\end{itemize}

Some non-compact Lie groups:
\begin{itemize}
    \item $\SL$; and 
    \item $\GL$.
\end{itemize}

\begin{theorem}
    Let $(p, V)$ be an irreducible finite-dimensional representation of $U(1)$ over $\C$. Then
    \begin{enumerate}
        \item $\dim V = 1$; and
        \item $\rho: U(1) \to \C^\times$ has form $p(z) = z^n$ for some $n \in \Z$. 
    \end{enumerate}
\end{theorem}

\section{$\sltwo$}

The aims of this section are as follows.
\begin{itemize}
    \item Classify the irreducible, finite-dimensional, $\mathbb C$-linear representations of $\sltwo$.
    \item Form methods for decomposing reducible representations of $\sltwo$.
\end{itemize}

We recap below.

\begin{itemize}
    \item Denote $\gl[n, \mathbb K]$ as the set of $n \times n$ matrices with entries in $\mathbb K$.
    \item A \emph{linear Lie group} is a closed (in the topological sense) subgroup of $\GL_n(\mathbb C)$ for some $n \in \mathbb N$.
    \item Let $X \in \gl[n, \mathbb C]$. Then
          \[ \exp(X) = \sum_{n=0}^\infty \frac{X^n}{n!}. \]
    \item The \emph{Lie algebra} $\mathfrak g$ of a linear Lie group $G$ is
          \[ \mathfrak g = \{X \in \gl[n, \mathbb C]: \exp(\mathbb RX) \subset G\}. \]
          We may consider the \emph{Lie functor}, $\Lie(-): \GL_2(\mathbb C) \to \gl[n, \mathbb C]$.
    \item $\Lie(\SL_n(\mathbb K)) = \sl[n, \mathbb K] = \{X \in \gl[n, \mathbb K]: \tr(X) = 0\}$.
\end{itemize}

The \emph{standard basis} for $\sltwo$ is
\[
    H =
    \begin{pmatrix}
        1 & 0 \\ 0 & -1
    \end{pmatrix}, \qquad
    X =
    \begin{pmatrix}
        0 & 1 \\ 0 & 0
    \end{pmatrix}, \qquad
    Y =
    \begin{pmatrix}
        0 & 0 \\ 1 & 0
    \end{pmatrix}.
\]

The idea here is to study representations $(\rho, V)$ of $\sltwo$ by looking at the eigenvectors and eigenvalues of $\rho(G)$.

\subsection{Weights}

\begin{definition}[Weight vector]
    Let $(\rho, V)$ be a $\C$-linear representation of $\sltwo$. Then a \emph{weight vector} in $V$ is an eigenvector of $\rho(H)$. The eigenvalue is called the \emph{weight}.
\end{definition}

\begin{example}
    \begin{enumerate}
        \item Consider the trivial representation $(\rho, \mathbb C)$ where $\rho(A) = 0$ for all $A \in \sltwo$. Pick basis $e \in \mathbb C^\times$ for $\mathbb C$. We have $\rho(H) = 0$ and so the (sole) weight vector is $e$, and its weight is $0$.
        \item Consider the standard representation $(\rho, \mathbb C^2)$, where $\rho(A) = A$ for all $A \in \sltwo$. So $\rho(H) = H$, so our weight vectors are $e_1, e_2$ with eigenvalues $1, -1$. Given that $e_1$ has weight $1$ and $e_2$ has weight $-1$, we may relabel $e_1 = e_1$ and $e_2 = e_{-1}$.
        \item Consider the representation $(\ad, \sltwo)$. We examine how $\ad$ acts on the standard basis:
              \begin{align*}
                  \ad_H(X) = [H,X] & = 2X,  \\
                  \ad_H(Y) = [H,Y] & = -2Y, \\
                  \ad_H(H) = [H,H] & = 0.
              \end{align*}
              Thus we have weight vectors $X$, $Y$, and $H$ with weights $2$, $-2$, and $0$ respectively.
        \item Consider the representation $(\rho, \C^2 \otimes \C^2)$ of $\sltwo$ which is the tensor of two standard representations. We pick the standard basis of $\C^2 \otimes \C^2$:
              \[ e_1 \otimes e_1, e_1 \otimes e_2, e_2 \otimes e_1, e_2 \otimes e_2. \]
              See that
              \[ (\rho)(H)(e_1 \otimes e_1) = \rho(H)e_1 \otimes e_1 + e_1 \otimes \rho(H) e_1 = 2e_1 \otimes e_1. \]
              In fact, we have the following lemma.
              \begin{lemma}
                  If $v$ is a weight vector of weight $\alpha$ and $w$ is a weight vector of weight $\beta$, then $v \otimes w$ is a weight vector of weight $\alpha + \beta$.
              \end{lemma}
              This can be seen by working through the above example. Thus our weights are $2$, $0$, $0$, and $-2$.
        \item Consider the standard representation $(\rho, \Sym^k(\C^2))$ of $\sltwo$. We pick the basis $\{e_1^ae_{-1}^{k-a}: 0 \leq a \leq k\}$. Then
              \begin{align*}
                  \rho(H)(e_1^ae_{-1}^{k-a})
                   & = \left(\rho(H)e_1^a\right)e_{-1}^{k-a} + e_1^a\left(\rho(H)e_{-1}^{k-a}\right) \\
                   & = ae_1^ae_{-1}^{k-a} - (k-a) e_1^ae_{-1}^{k-a}                                  \\
                   & = (2a-k) e_1^a e_{-1}^{k-a}.
              \end{align*}
              So $e_1^a e_{-1}^{k-a}$ is a weight vector with weight $2a - k$. Thus our weights are $\{-k, 2-k, 4-k, \ldots, k-4, k-2, k\}$.  For example, when $k = 5$ we get $\{-5, -3, -1, 1, 3, 5\}$ as our weights.
    \end{enumerate}
\end{example}

We now look at how $\rho(X)$ and $\rho(Y)$ act on weight vectors. For a representation $(\rho, V)$ of $\sltwo$, let
\[ V_\alpha = \{v \in V: \rho(H)v = \alpha v\} \]
be the \emph{eigenspace} for $\rho(H)$ with eigenvalue $\alpha \in \C$. We note that we have the decomposition
\[ V = \bigoplus_{\alpha} V_\alpha \]
where we direct sum over the weights of $V$.

\begin{proposition}
    Let $(\rho, V)$ be a $\C$-linear representation of $\sltwo$. Let $\alpha$ be a weight of $V$. Then
    \begin{align*}
        \rho(X)V_\alpha & \subset V_{\alpha + 2}, \\
        \rho(Y)V_\alpha & \subset V_{\alpha - 2}.
    \end{align*}
\end{proposition}

We view $\rho(X)$ as a \emph{raising operator}, and $\rho(Y)$ as a \emph{lowering operator}.

\begin{definition}[Highest weight vector]
    Let $(\rho, V)$ be a representation of $\sltwo$. A \emph{highest weight vector} $v \in V$ is a weight vector such that $\rho(X)v = 0$. The weight of $v$ is the \emph{highest weight}.
\end{definition}

\begin{example}
    \begin{enumerate}
        \item Consider the standard representation $(\rho, \Sym^5(\mathbb C^2))$ of $\sltwo$. Here $v_5 = e_1^5$ is the highest weight, it can easily be checked that $\rho(X)v_5 = 0$.
        \item Consider the tensor product of two standard representations $(\rho, \C^2 \otimes \C^2)$. We see that $e_1 \otimes e_1$ is a highest weight vector, but we also have another. We claim that $e_1 \otimes e_{-1} - e_{-1} \otimes e_1$ is a highest weight vector. Both $e_1 \otimes e_{-1}$ and $e_{-1} \otimes e_1$ have weight $0$, so $e_1 \otimes e_{-1} - e_{-1} \otimes e_1$ has weight $0$. But it can be checked that $\rho(X)$ kills $e_1 \otimes e_{-1} - e_{-1} \otimes e_1$.
    \end{enumerate}
\end{example}

\subsection{Classification of representations of $\sltwo$}

\begin{corollary}
    Any finite-dimensional representation of $\sltwo$ has a highest weight vector.
\end{corollary}

\begin{theorem}
    \begin{enumerate}
        \item For every $k \in \N_0$, there is a unique $\C$-linear and finite-dimensional representation (up to isomorphism) of $\sltwo$ such that it has a highest weight vector of weight $k$.
        \item Every finite-dimensional $\C$-linear irreducible representation of $\sltwo$ is isomorphic to one of the above representations.
    \end{enumerate}
\end{theorem}

\begin{proof}
    \begin{enumerate}
        \item We can just consider $e_1^k \in \Sym^k(\C^2)$.
        \item For this, we just show that $\Sym^k(\C^2)$ is irreducible. Let $W \subset \Sym^k(\C^2)$ be a subrepresentation. $W$ has a highest weight vector of the form $e_1^ae_{-1}^b$ with $a+b = k$, $a \geq 0$, and $b \leq k$. We also have $\rho(X)(e_1^ae_{-1}^b) = be_1^{a+1}e_{-1}^{b-1} \neq 0$ for $b > 0$. Thus $e_1^k$ is the unique highest weight vector in $\Sym^k(\C^2)$, so $e_1^k \in W$. We see that $W$ is invariant under $\rho(Y)$, and by repeated applications of $\rho(Y)$ we see that the entire basis of $\Sym^k(\C^2)$ is in $W$, and thus $W = \Sym^k(\C^2)$. Thus $\Sym^k(\C^2)$ is irreducible.
    \end{enumerate}
\end{proof}

\begin{lemma}
    Let $(\rho, V)$ be a $\C$-linear representation of $\sltwo$. If $v \in V$ is a highest weight vector with weight $k$. Then
    \[ XY^mv = m(k-m+1)Y^{m-1}v \]
    for all $m \in \N_0$.
\end{lemma}

\begin{lemma}
    If $(\rho, V)$ is a finite-dimensional irreducible representation of $\sltwo$ and $v \in V$ is a highest weight vector with weight $k \in \N_0$, then
    \[ V = \langle v, Yv, \ldots, Y^kv \rangle. \]
\end{lemma}

\begin{corollary}
    If $(\rho, V)$ is a irreducible representation of $\sltwo$, then $\rho(H)$ is diagonalisable.
\end{corollary}

\subsection{Decomposing $\sltwo$}

\begin{lemma}
    Every $A \in \sltwo$ can be written uniquely as $X + iY$ for $X, Y \in \su[n]$.
\end{lemma}

\begin{proof}
    We have $\su[n] = \{X \in \sl[n, \C]: X + X^\dagger = 0\}$. Write
    \[ A = \frac12(A - A^\dagger) - \frac i2(A + A^\dagger). \]
    Both components here are in $\su[n]$, and by arguing on the dimensions of $\sl[n, \C]$ and $\su[n]$ we see that this must be unique.
\end{proof}

\begin{lemma}
    There is a bijection between the $\C$-linear representations of $\sl[n, \C]$ and the complex representations of $\su[n]$.
\end{lemma}

\begin{proof}
    $\tilde\rho(X + iY) = \rho(X) + i\rho(Y)$.
\end{proof}

\begin{theorem}[Complete reducibility for $\sltwo$]
    Let $V$ be a finite-dimensional $\C$-linear representation of $\sltwo$. Then
    \[ V \cong \bigoplus_{i=1}^r V_i\]
    where each $V_i$ is a irreducible representation.
\end{theorem}

\begin{proof}
    By the previous lemma, it is enough to show that $V$ decomposes into irreducible representations of $\su[n]$. As $\SU(n)$ is simply connected, there is a representation $\hat\rho$ of $\SU(n)$ on $V$ whose derivative is $\rho$. $\su[n]$ is compact, so $\hat\rho$ decomposes (by Maschke's theorem). Finally, as $\su[n]$ is connected, so $\hat\rho$ decomposes on $\su[n]$.
\end{proof}

\subsection{Decomposing tensor products}

We recall that
\begin{align*}
    \{\text{weights of $V \otimes W$}\}
     & = \{\text{weights of $V$}\} + \{\text{weights of $W$}\},              \\
    \{\text{weights of $\Sym^k(V)$}\}
     & = \{\text{sum of unordered $k$-tuplets of weights of $V$}\},          \\
    \{\text{weights of $\Lambda^k(V)$}\}
     & = \{\text{sum of unordered $k$-tuplets of distinct weights of $V$}\}. \\
\end{align*}

For example, if a representation $(\rho, V)$ of $\sltwo$ has weights $\{-2,0,0,2\}$, then $\Lambda^2(V)$ has weights
$\{-2,-2,0,0,2,2\}$. Recall we are using multisets here. Similarly, $\Lambda^3(V) = \{-2,0,0,2\}$.

We now give a general method for decomposing tensor products into other representations.

Given the multisets of weights of a representation $(\rho, V)$ of $\sltwo$:
\begin{enumerate}
    \item let $k$ be the biggest weight;
    \item any weight vector $v \in V$ of weight $k$ must be a highest weight vector, thus
          \[ \langle v, Yv, \ldots, Y^kv \rangle \cong \Sym^k(\mathbb C^2); \]
    \item by complete reducibility
          \[ V \cong \Sym^k(\mathbb C^2) \oplus V' \]
          where $V'$ has the weights of $V$ with the weights $\{-k, -k+2, \ldots, k-2, k\}$ removed.
\end{enumerate}

\begin{theorem}
    A representation $(\rho, V)$ of $\sltwo$ is determined up to isomorphism by its weights.
\end{theorem}

This theorem allows us to apply the above technique without worry.

\begin{example}
    Let $V = \Sym^2(\mathbb C^2)$ where $\mathbb C^2$ is the standard representation. We will decompose $V \otimes V$ into irreducible representations and irreducible subrepresentations. First, we first the weights of $V \otimes V$.
    \begin{align*}
        \{\text{weights of $V \otimes V$}\}
         & = \{\text{weights of $V$}\} + \{\text{weights of $V$}\} \\
         & = \{-2, 0, 2\} + \{-2, 0, 2\}                           \\
         & = \{-4, -2, -2, 0, 0, 0, 2, 2, 4\}.
    \end{align*}
    We draw the following weight diagram.
    \begin{center}
        \begin{tikzpicture}
            \node[weight] (0) {};
            \node[mult1] at (0) {};
            \node[mult2] at (0) {};
            \node[mult3] at (0) {};
            \node[below of=0] {0};

            \node[weight] (-1) [left of=0] {};
            \node[mult1] at (-1) {};
            \node[mult2] at (-1) {};
            \node[below of=-1] {-2};

            \node[weight] (1) [right of=0] {};
            \node[mult1] at (1) {};
            \node[mult2] at (1) {};
            \node[below of=1] {-4};

            \node[weight] (-2) [left of=-1] {};
            \node[mult1] at (-2) {};
            \node[below of=-2] {2};

            \node[weight] (2) [right of=1] {};
            \node[mult1] at (2) {};
            \node[below of=2] {4};
        \end{tikzpicture}
    \end{center}
    So, using our method outlined above, we get the following.
    \begin{center}
        \begin{tikzpicture}
            \node[weight] (0.0) {};
            \node[weight] (0.-1) [left of=0.0] {};
            \node[weight] (0.1) [right of=0.0] {};
            \node[weight] (0.-2) [left of=0.-1] {};
            \node[weight] (0.2) [right of=0.1] {};
            \node [right=25mm] {$\Sym^4(\C^2)$};
            
            \node[weight] (A) [below of=0.0] {};
            \node[weight] (1.-1) [left of=A] {};
            \node[weight] (1.1) [right of=A] {};
            \node [right=24mm of A] {$\Sym^2(\C^2)$};

            \node[weight] (B) [below of=A] {};
            \node [right=24mm of B] {$\C$};
        \end{tikzpicture}
    \end{center}
    Now we decompose into subrepresentations. $V$ has basis $v_2 = e_1^2$, $v_0 = e_1e{-1}$, and $v_{-2} = e_{-1}^2$. We observe how $X$ and $Y$ acts on these.
    $X(v_2) = 0$, $X(v_0) = v_2$, and $X(v_{-2}) = 2v_0$. Similarly $Y(v_2) = (2v_0)$, $Y(v_0) = v_{-2}$, and $Y(v_{-2}) = 0$. From here, it is easy to piece the highest weight vectors by looking at how $X$ acts on various combinations (or known highest weight vectors).
\end{example}

\section{$\slthree$}

The aims of this section is similar to the previous: classify irreducible finite-dimensional $\C$-linear representations of $\slthree$ by the highest weights.

\begin{example}[Some representations of $\slthree$]
    \begin{enumerate}
        \item The standard representations on $\C^3$, with basis $e_1, e_2, e_3$.
        \item The dual standard representation on $(\C^3)^*$, with basis $e_1^*, e_2^*, e_3^*$ (we note that we did not use this representation in the previous section as dualling on $\sltwo$ reflects the weights about the origin).
        \item The adjoint representation $(\ad, \slthree)$.
        \item The tensor of symmetric powers $\Sym^a(\C^3) \otimes \Sym^b((\C^3)^*)$ (which is sadly not irreducible).
    \end{enumerate}
\end{example}

We proceed similar to before, but we need to redefine our notion of \emph{weights}.

\begin{definition}[Standard Cartan subalgebra]
    The \emph{standard Carton subalgebra} is the abelian subalgebra of $\slthree$
    \[
        \mathfrak h = \left\{
        \begin{pmatrix}
            h_1 & 0   & 0   \\
            0   & h_2 & 0   \\
            0   & 0   & h_3 \\
        \end{pmatrix}
        : h_1 + h_2 + h_3 = 0
        \right\}
        \subset \slthree.
    \]
\end{definition}

This is abelian as diagonal matrices commute.

\begin{definition}
    If $(\rho, V)$ is a $\C$-linear representation of $\slthree$, a \emph{weight vector} $v \in V$ is a simultaneous eigenvector of $\{\rho(H): H \in \mathfrak h\}$. The \emph{weight} $\alpha$ of $v$ is a linear map $\alpha: \mathfrak h \to \C$ such that $\rho(H)v = \alpha(H)v$. The \emph{weight space} of weight $\alpha$ is
    \[
        V_\alpha = \left\{
        v \in V: \text{$\rho(H) v = \alpha(H)V$ for all $H \in \mathfrak h$}
        \right\}.
    \]
\end{definition}

By simultaneous, we mean that it is an eigenvector regardless of the $H$ chosen.

We denote $E_{ij}$ for the matrix with a $1$ in entry $(i,j)$ and 0 elsewhere. We note that $E_{ij} \in \slthree$ if and only if $i \neq j$. We pick a basis of $\mathfrak h$ as the elements
\[
    H_{12} = E_{11} - E_{22} =
    \begin{pmatrix}
        1 & 0  & 0 \\
        0 & -1 & 0 \\
        0 & 0  & 0 \\
    \end{pmatrix}, \qquad
    H_{23} = E_{22} - E_{33} =
    \begin{pmatrix}
        0 & 0 & 0  \\
        0 & 1 & 0  \\
        0 & 0 & -1 \\
    \end{pmatrix},
\]
and we also define $H_{13} = H_{12} + H_{23}$. It will be enough to study the eigenvectors of $\rho(H_{12})$ and $\rho(H_{23})$.

\begin{example}
    \begin{enumerate}
        \item Let $(\rho, \C^3)$ be the standard representation with basis $e_1, e_2, e_3$. Then for $i \in \{1,2,3\}$ and $H \in \mathfrak h$, we have $\rho(H) e_i = L_i(H)e_i$ where $L_i(H) = h_i$ ($H = h_1E_{11} + h_2E_{22} + h_3E_{33}$). Thus we have the weight vectors being the $e_i$'s with respective weights being the $L_i$'s. Here $L_1, L_2, L_3$ span $\mathfrak h^*$, and there is one relation between them: $L_1 + L_2 + L_3 = 0$. Thus any element of $\mathfrak h^*$ can be written as $aL_1 - bL_3$ with $a, b \in \C$.
        \item Let $(\rho^*, (\C^3)^*)$ be the dual representation. Then $He_1^* = -h_ie_i$ (should be checked). Thus, the weights of the dual representation are $\{-L_1, -L_2, -L_3\}$.
        \item Consider the adjoint representation $(\ad, \mathfrak g)$ where $\mathfrak g = \slthree$. We see that
              \[ \ad_H(H') = [H, H'] = 0 \]
              for all $H, H' \in \mathfrak h$, thus $0$ is a weight of the adjoint representation. Thus,
              \[ \mathfrak g_0 := \text{$0$-weight space of $\mathfrak g$} = \mathfrak h \]
              (note we only proved that $\mathfrak h \subset \mathfrak g_0$, but this is indeed true). See that
              \[ [H, E_{ij}] = (h_i - h_j)E_{ij} \]
              for $H \in \mathfrak h$ and $i \neq j$. Thus $E_{ij} \in \slthree$ for $i \neq j$ is a weight vector with weight $h_i - h_j = L_i - L_j$.
    \end{enumerate}
\end{example}

\begin{definition}
    A \emph{root} of $\slthree$ is a non-zero weight of the adjoint representation. A \emph{root vector} is a weight vector of a root, and a \emph{root space} is the weight space of a root.
\end{definition}

We write
\[ \Phi = \{\pm(L_1 - L_2), \pm(L_2 - L_3), \pm(L_1 - L_3)\} \]
for the set of roots of $\slthree$. We call
\[ \Phi^+ = \{L_1 - L_2, L_2 - L_3, L_1 - L_3\} \]
the \emph{positive roots} and
\[ \Phi^+ = \{L_2 - L_1, L_3 - L_2, L_3 - L_1\} \]
the \emph{negative roots}. We write
\[ \Delta = \{L_1 - L_2, L_2 - L_3\}, \]
these are called the \emph{simple roots}. We may write $\alpha_{ij}$ for the root $L_i - L_j$.

Finally, we have the \emph{root space}, also called the \emph{Cartan decomposition}
\[ \mathfrak g = \mathfrak h \oplus \bigoplus_{\alpha \in \Phi} \mathfrak g_\alpha. \]

\subsection{Visualising weights}

\begin{theorem}
    Let $(\rho, V)$ be a finite-dimensional $\C$-linear representation of $\slthree$, then all its weights are elements of
    \[
        \Lambda_W = \left\{
        aL_1 - bL_3: a,b \in \Z
        \right\}
    \]
    called the \emph{weight lattice}.
\end{theorem}

\begin{proof}
    Let $\alpha = aL_1 - bL_3$ for $a,b \in \C$ be a weight of $V$. We have to prove that $a, b \in \Z$. We sketch the proof here. We consider the embedding
    \begin{align*}
        \sltwo & \xhookrightarrow{} \slthree \\
        H &\mapsto \left(
            \begin{array}{c|c}
                H & 0 \\ \hline
                0 & 0 \\
            \end{array}
        \right).
    \end{align*}
    By our $\sltwo$-theorem, all eigenvalues acting on $V$ are integers, thus $a \in \Z$. For $b \in \Z$, we use the similar embedding:
    \begin{align*}
        \sltwo & \xhookrightarrow{} \slthree \\
        H &\mapsto \left(
            \begin{array}{c|c}
                0 & 0 \\ \hline
                0 & H \\
            \end{array}
        \right).
    \end{align*}
\end{proof}

To visualise our weights: put $L_1$, $L_2$, and $L_3$ as vertices of an equilateral triangle. Then $\Lambda_W$ is the lattice generated by these. 

\begin{example}
    Weights for the standard representation on $\C^3$. 
    \begin{center}
        \begin{tikzpicture}
            \node[weight] at (1,0) {};
            \node[weight] at (2,0) {};
            \node[weight] at (3,0) {};
            \node[weight] at (4,0) {};
            \node[weight] at (5,0) {};
            \node[weight] at (6,0) {};

            \node[weight] at (1.5,1) {};
            \node[weight] at (2.5,1) {};
            \node[weight] at (3.5,1) {};
            \node[weight] at (4.5,1) {};
            \node[weight] at (5.5,1) {};
            \node[weight] at (6.5,1) {};

            \node[weight] at (1,2) {};
            \node[weight] at (2,2) {};
            \node[weight] at (3,2) {};
            \node[weight] at (4,2) {};
            \node[weight] at (5,2) {};
            \node[weight] at (6,2) {};

            \node[weight] at (1.5,3) {};
            \node[weight] at (2.5,3) {};
            \node[weight] at (3.5,3) {};
            \node[weight] at (4.5,3) {};
            \node[weight] at (5.5,3) {};
            \node[weight] at (6.5,3) {};

            \node[weight] at (1,4) {};
            \node[weight] at (2,4) {};
            \node[weight] at (3,4) {};
            \node[weight] at (4,4) {};
            \node[weight] at (5,4) {};
            \node[weight] at (6,4) {};

            \node[mult1] at (5,2) {};
            \node[mult1] at (3.5,3) {};
            \node[mult1] at (3.5,1) {};

            \node at (4.35,2) {$0$};
            \node at (5.35,2) {$L_1$};
            \node at (3.85,3) {$L_2$};
            \node at (3.85,1) {$L_3$};
        \end{tikzpicture}
    \end{center}
\end{example}

\begin{example}
    We now consider the weights of the adjoint representation.
    \begin{center}
        \footnotesize
        \begin{tikzpicture}
            \node[weight] at (-3,0) {};
            \node[weight] at (-2,0) {};
            \node[weight] at (-1,0) {};
            \node[weight] at (0,0) {};
            \node[weight] at (1,0) {};
            \node[weight] at (2,0) {};

            \node[weight] at (-2.5,1) {};
            \node[weight] at (-1.5,1) {};
            \node[weight] at (-0.5,1) {};
            \node[weight] at (0.5,1) {};
            \node[weight] at (1.5,1) {};
            \node[weight] at (2.5,1) {};

            \node[weight] at (-3,2) {};
            \node[weight] at (-2,2) {};
            \node[weight] at (-1,2) {};
            \node[weight] at (0, 2) {};
            \node[weight] at (1, 2) {};
            \node[weight] at (2, 2) {};

            \node[weight] at (-2.5,-1) {};
            \node[weight] at (-1.5,-1) {};
            \node[weight] at (-0.5,-1) {};
            \node[weight] at (0.5, -1) {};
            \node[weight] at (1.5, -1) {};
            \node[weight] at (2.5, -1) {};

            \node[weight] at (-3,-2) {};
            \node[weight] at (-2,-2) {};
            \node[weight] at (-1,-2) {};
            \node[weight] at (0, -2) {};
            \node[weight] at (1, -2) {};
            \node[weight] at (2, -2) {};


            \node[mult1] at (0,0) {};
            \node[mult2] at (0,0) {};
            \node at (0, 0) [xshift=10] {$0$};
            
            \node[mult1] at (1.5,1) {};
            \node at (1.5, 1) [xshift=12] {$\alpha_{13}$};

            \node[mult1] at (0,2) {};
            \node at (0, 2) [xshift=12] {$\alpha_{23}$};

            \node[mult1] at (-1.5,1) {};
            \node at (-1.5, 1) [xshift=12] {$\alpha_{21}$};

            \node[mult1] at (-1.5,-1) {};
            \node at (-1.5, -1) [xshift=12] {$\alpha_{31}$};

            \node[mult1] at (0,-2) {};
            \node at (0, -2) [xshift=12] {$\alpha_{32}$};

            \node[mult1] at (1.5,-1) {};
            \node at (1.5, -1) [xshift=12] {$\alpha_{12}$};
        \end{tikzpicture}
    \end{center}
\end{example}

\begin{example}
    Consider $\Sym^2(\C^3)$ where $\C^3$ is the standard representation. Our weight vectors are of the form $e_ie_j$ for $1 \leq i \leq j \leq 3$. We have
    \[ H(e_ie_j) = H(e_i)e_j + e_iH(e_j) = (L_i + L_j)(H) e_ie_j. \]
    Thus the weight of $e_ie_j$ is $L_i + L_j$. Considering every $i$ and $j$, we get
    \[ \text{weights} = \{2L_1, 2L_2, 2L_3, L_1 + L_2, L_2 + L_3, L_1 + L_3\}. \]
    Thus we draw our weights as follows. 
    \begin{center}
        \begin{tikzpicture}
            \node[weight] at (-3,0) {};
            \node[weight] at (-2,0) {};
            \node[weight] at (-1,0) {};
            \node[weight] at (0,0) {};
            \node[weight] at (1,0) {};
            \node[weight] at (2,0) {};

            \node[weight] at (-2.5,1) {};
            \node[weight] at (-1.5,1) {};
            \node[weight] at (-0.5,1) {};
            \node[weight] at (0.5,1) {};
            \node[weight] at (1.5,1) {};
            \node[weight] at (2.5,1) {};

            \node[weight] at (-3,2) {};
            \node[weight] at (-2,2) {};
            \node[weight] at (-1,2) {};
            \node[weight] at (0, 2) {};
            \node[weight] at (1, 2) {};
            \node[weight] at (2, 2) {};

            \node[weight] at (-2.5,-1) {};
            \node[weight] at (-1.5,-1) {};
            \node[weight] at (-0.5,-1) {};
            \node[weight] at (0.5, -1) {};
            \node[weight] at (1.5, -1) {};
            \node[weight] at (2.5, -1) {};

            \node[weight] at (-3,-2) {};
            \node[weight] at (-2,-2) {};
            \node[weight] at (-1,-2) {};
            \node[weight] at (0, -2) {};
            \node[weight] at (1, -2) {};
            \node[weight] at (2, -2) {};

            \node at (0, 0) [xshift=10] {$0$};
            \node at (1, 0) [xshift=9] {$L_1$};
            \node at (-0.5, 1) [xshift=9] {$L_2$};
            \node at (-0.5, -1) [xshift=9] {$L_3$};

            \node[mult1] at (2,0) {};
            \node[mult1] at (0.5,1) {};
            \node[mult1] at (-1,2) {};
            \node[mult1] at (-1,0) {};
            \node[mult1] at (-1,-2) {};
            \node[mult1] at (0.5,-1) {};
        \end{tikzpicture}
    \end{center}
\end{example}

We present a fundamental weight calculation, as we did with $\sltwo$. 

\begin{theorem}[Fundamental weight calculation]
    Let $(\rho, V)$ be a $\C$-linear representation of $\slthree = \mathfrak g$ and let $v \in V_\beta$ be a weight vector with weight $\beta \in \mathfrak h^*$. Let $\alpha \in \mathfrak h^*$ be a root and let $X_\alpha \in \mathfrak g_\alpha$ be a root vector. Then $\rho(X_\alpha)v = 0$. 
\end{theorem}

\begin{proof}
    Let $H \in \mathfrak h$. Then
    \begin{align*}
        H(X_\alpha(v)) &= ([H, X_\alpha] + X_\alpha H)v \\
        &= \alpha(H) X_\alpha v + X\alpha \beta(H) \\
        &= (\alpha + \beta)(H) (X_\alpha v). \qedhere
    \end{align*}
\end{proof}

\begin{definition}
    Let $(\rho, V)$ be a $\C$-linear representation of $\slthree$. Then a weight vector $v \in V$ is a \emph{highest weight vector} $\rho(X)v = 0$ for $X \in \{E_{12}, E_{13}, E_{23}\}$. The \emph{highest weight} of $v$ is the weight of $v$. 
\end{definition}

As $E_{13} = [E_{12}, E_{23}]$, we only need to check $E_{12}$ and $E_{23}$. 

\begin{proposition}
    If $(\rho, V)$ is a finite-dimensional representation, then a highest weight exists. 
\end{proposition}

\begin{proof}
    Define $l: \mathfrak h^* \to \C$ by $l(aL_1 - bL_3) = a + b$. Use this function on contradiction of having a weight vector of maximal $l$ value. 
\end{proof}

\subsection{Dominant weights}

Let $(\rho, V)$ be a representation of $\slthree$. If $v \in V$ is a highest weight vector of weight $aL_1 - bL_3$, then it is a highest weight vector for $V$ under the restrictions
\[
    \left(
        \begin{array}{c|c}
            \sltwo & 0 \\ \hline
            0 & 0 \\
        \end{array}
    \right), \qquad 
    \left(
        \begin{array}{c|c}
            0 & 0 \\ \hline
            0 & \sltwo \\
        \end{array}
    \right).
\]

Its weight for the top right restriction is $a$ and its weight for the bottom right copy is $b$. 

\begin{definition}[Dominant weight]
    A \emph{dominant weight} is an element of $\mathfrak h^*$ of the form $aL_1 - bL_3$ with $a, b \in \N_0$. 
\end{definition}

\begin{theorem}
    For each dominant weight $aL_1 - bL_3$ there is a unique (up to isomorphism) finite-dimensional $\C$-linear irreducible representation of $\slthree$ with highest weight vector that of the weight. 
\end{theorem}

We call such a representation $V^{(a,b)}$. 

\begin{example}\hspace{0em}
    \begin{itemize}
        \item $V^{(0,0)} = \C$ (trivial)
        \item $V^{(1,0)} = \C^3$ (standard)
        \item $V^{(0,1)} = (\C^3)^*$
        \item $V^{(1,1)} = (\ad, \slthree)$
        \item $V^{(2,0)} = \Sym^2(\C^3)$
    \end{itemize}    
\end{example}



\end{document}
\documentclass[a4paper, 11pt]{article}

\usepackage[utf8]{inputenc}

\usepackage{parskip}

\usepackage{amssymb}
\usepackage{amsmath}
\usepackage{amsfonts}
\usepackage{mathtools}

\usepackage{todonotes}

\usepackage{csquotes}

\usepackage{algpseudocode}
\usepackage{algorithm}

\DeclareMathOperator{\AQ}{AQ}
\DeclareMathOperator{\DAQ}{\Delta AQ}
\DeclareMathOperator{\Q}{Q}
\DeclareMathOperator{\HE}{HE}
\DeclareMathOperator*{\argmax}{arg\,max}
\DeclareMathOperator{\Ep}{Ep}

\usepackage{braket}

\addtolength{\oddsidemargin}{-.875in}
\addtolength{\evensidemargin}{-.875in}
\addtolength{\textwidth}{1.75in}
\addtolength{\topmargin}{-.875in}
\addtolength{\textheight}{1.75in}

\usepackage[backend=biber]{biblatex}
\addbibresource{ref.bib}

\title{Natural Computing Part A}
\author{Ben Napier}
\date{March 2022}

\begin{document}

\section{Lie groups}

In this section, when we refer the \emph{Lie groups} we actually mean \emph{linear Lie groups}.

\subsection{Definition}

\begin{definition}[Lie group]
    A \emph{Lie group} is a closed subgroup of $\GL_n(\C)$ for some $n$.
\end{definition}

We mean \emph{closed} in the topological sense, a more correct definition would include details on \emph{smooth manifolds}, which we do not focus on here.

We denote $\gl[n, \mathbb K]$ as the set of $n \times n$ matrices with entries in $\mathbb K$.

We now generalise the exponential map $\exp$ to matrices, using the familiar power series.

\begin{definition}[Exponential map]
    Let $X \in \gl[n, \mathbb K]$. Then define $\exp: \gl[n, \mathbb K] \to \gl[n, \mathbb K]$ by
    \[ \exp(X) = \sum_{i=0}^\infty \frac{X^i}{i!}. \]
\end{definition}

We will omit details on the convergence of $\exp$, but it is convergent for all matrices and can be proved using the Cauchy-Schwartz. In particular, by considering the entry-wise norm.

We have some properties of the exponential map, and comment that it behaves similarly to the normal $\exp$. For all $X, Y, g \in \gl[n, \mathbb K]$ where $g$ is invertible, and $s, t \in \mathbb K$, we have the following.

\begin{itemize}
    \item $\exp(0) = I$
    \item $\exp(X + Y) = \exp(X) \exp(Y)$
    \item $(\exp(X))^{-1} = \exp(-X)$
    \item $\exp(sX)\exp(tX) = \exp((s+t)X)$
    \item $g\exp X g^{-1} = \exp(gXg^{-1})$
\end{itemize}

\begin{proposition}
    $\exp: \gl[n, \C] \to \gl[n, \C]$ is differentiable at zero (the zero matrix), and its derivative at the origin is $I$.
\end{proposition}

\begin{corollary}
    $\exp: \gl[n, \C] \to \gl[n, \C]$ is a local diffeomorphism at zero.
\end{corollary}

By this, we mean that $\exp$ has an inverse near zero.

We note that $\exp: \gl[n, \C] \to \gl[n, \C]$ is \emph{not} injective. In particular, it coincides with our regular exponential map at $n = 1$, so $\exp(2\pi i k) = 1$ for all $k \in \Z$.

\begin{lemma}
    $\exp: \gl[n, \C] \to \GL_n(\C)$ is surjective.
\end{lemma}

We not that $\exp: \gl[n, \R] \to \GL_n(\R)$ is \emph{not} surjective. Again, for $n = 1$ we see that $\exp$ is strictly positive.

\begin{proposition}
    $\det\exp = \exp\tr$.
\end{proposition}

\begin{proof}
    Let $X \in \gl[n, \K]$. We can conjugate $X$ so that it is upper triangular. The result is then immediate.
\end{proof}

\subsection{One-parameter subgroups}

We have seen that $\exp((s+t)X) = \exp(sX)\exp(tX)$, thus for all $X \in \gl[n, \C]$ we can define a group homomorphism $f: \R \to \GL_n(\C)$ such that $t \mapsto \exp(tX)$ (here $\R$ is given standard addition $+$).

Similarly, if we consider $G = \SO(2)$ then we can define a group homomorphism $\R \to \SO(2)$ by  $t \mapsto \text{rotation by $t$}$. We note that
\[
    \begin{pmatrix}
        \cos t  & \sin t \\
        -\sin t & \cos t
    \end{pmatrix}
    = \exp
    \begin{pmatrix}
        0 & -t \\
        t & 0
    \end{pmatrix}.
\]

\begin{definition}[One-parameter subgroup]
    Let $G$ be a Lie group. A \emph{one-parameter subgroup} is a differentiable group homomorphism $\gamma: (\R, +) \to G$. The matrix $\gamma'(0)$ is the \emph{infinitesimal generator}.
\end{definition}

\begin{theorem}
    Let $f: \R \to \GL_n(\C)$ be a one-parameter subgroup with infinitesimal generator $X$. Then
    \[ f(t) = \exp(tX). \]
\end{theorem}

\subsection{Lie algebras}

\begin{definition}[Lie algebra of a Lie group]
    Let $G$ be a Lie group. Its \emph{Lie algebra} is
    \[ \mathfrak g = \left\{
        X \in \gl[n, \C]: \exp(\R X) \subset G
        \right\}. \]
\end{definition}

We can alternatively define $\mathfrak g$ as the set of infinitesimal generators of all one-parameter subgroups.

\begin{proposition}
    Let $G$ be a Lie group and $\mathfrak g$ its Lie algebra. Then
    \[ \mathfrak g = \left\{
        X \in \gl[n, \C]: \text{$X = \gamma'(0)$ for some map $\gamma:[-a, a] \to G$ where $a > 0$}
        \right\}. \]
\end{proposition}

We may denote the Lie algebra of a Lie group $G$ by $\Lie(G)$.

\begin{example}[Some Lie algebras]\hspace{0em}
    \begin{itemize}
        \item $\Lie(\GL_n(\K)) = \gl[n, \K]$
        \item $\Lie(\SL_n(\K)) = \sl[n, \K] = \{X \in \gl[n, \K]: \tr(X) = 0\}$
        \item $\Lie(\O(n)) = \mathfrak o_n = \Lie(\SO(n)) = \mathfrak{so}_n = \{X \in \gl[n, \R]: X + X^\intercal = 0\}$
        \item $\Lie(\U(n)) = \mathfrak{u}_n = \{X \in \gl[n, \C]: X + X^\dagger = 0\}$
        \item $\Lie(\SU(n)) = \mathfrak{su}_n = \{X \in \mathfrak u_n: \tr(X) = 0\}$
    \end{itemize}
\end{example}

\begin{proposition}
    Let $\mathfrak g$ be the Lie algebra of a Lie group $G$. Then
    \begin{enumerate}
        \item $\mathfrak g \subset \gl[n, \C]$ is a real vector space;
        \item if $X \in \mathfrak g$ and $g \in G$, then $g X g^{-1} \in \mathfrak g$; and
        \item if $X, Y \in \mathfrak g$, then
              \[ [X,Y] := XY - YX \in \mathfrak g. \]
    \end{enumerate}
\end{proposition}

We now define a Lie algebra separate from a Lie group.

\begin{definition}[Lie algebra]
    A \emph{Lie algebra} $\mathfrak g$ is an $\R$-vector space with a bilinear map (called the \emph{Lie bracket}) $[-,-]: \mathfrak g^2 \to \mathfrak g$ such that
    \begin{enumerate}
        \item for all $X, Y \in \mathfrak g$, $[X,Y] = -[Y,X]$; and
        \item the \emph{Jacobi identity} holds: for all $X, Y, Z \in \mathfrak g$
              \[ [X, [Y,Z]] + [Y, [Z, X]] + [Z, [X,Y]] = 0. \]
    \end{enumerate}
\end{definition}

A \emph{Lie subalgebra} $\mathfrak h \subset \mathfrak g$ of a Lie algebra $\mathfrak g$ is a subspace which is closed under the Lie bracket.

\begin{example}
    Consider $\mathfrak g = \R^3$ with $[\bm v, \bm w] = \bm v \times \bm w$. Then $\mathfrak g$ is a Lie algebra, and in fact $\mathfrak g \cong \mathfrak{so}_3$.
\end{example}

The center of a Lie group is an abelian subgroup of $\mathfrak g$.

\begin{definition}
    A Lie algebra $\mathfrak g$ is \emph{abelian} if $[X,Y] = 0$ for all $X, Y \in \mathfrak g$. The \emph{center} of $\mathfrak g$ is
    \[ Z(\mathfrak g) = \{Z \in \mathfrak g: \text{$[Z,X] = 0$ for all $X \in \mathfrak g$}\}. \]
\end{definition}

\begin{definition}[Complex Lie group]
    A complex Lie group is a closed subgroup of $\GL_n(\C)$ whose Lie algebra is a complex subspace of $\gl[n, \C]$.
\end{definition}

\subsection{Morphisms}

\begin{definition}
    A \emph{Lie group homomorphism} $\phi: G \to G'$ between two Lie groups is a continuous group homomorphism.
\end{definition}

As usual, a Lie group isomorphism is a homomorphism which is bijective with continuous inverse.

\begin{definition}
    A \emph{Lie algebra homomorphism} $\varphi: \mathfrak g \to \mathfrak h$ is an $\R$-linear map such that for all $X, Y \in \mathfrak g$,
    \[ \varphi([X,Y]) = [\varphi(X), \varphi(Y)]. \]
\end{definition}

A Lie algebra isomorphism is an invertible homomorphism.

\begin{definition}[Derivative]
    Let $\phi: G \to H$ be a Lie group homomorphism. Define the \emph{derivative} of $\phi$ as
    \begin{align*}
        D\phi: \mathfrak g & \to \mathfrak h,                           \\
        D\phi(X)           & = \frac{d}{dt} \phi(\exp(tX))\bigg|_{t=0}.
    \end{align*}
\end{definition}

\begin{theorem}
    Let $\phi: G \to H$ be a Lie group homomorphism. Then
    \begin{enumerate}
        \item the diagram
              \begin{center}
                  % https://tikzcd.yichuanshen.de/#N4Igdg9gJgpgziAXAbVABwnAlgFyxMJZABgBpiBdUkANwEMAbAVxiRAB12BbOnACwBmAJzoBrAAQBzEAF9S6TLnyEUARnJVajFm049+wseL6z5IDNjwEiZVZvrNWiEAHFTCy8qLq71BzucACVlNGChJeCJQYQguJDIQHAgkdS1HNgARTjQ+LHcQGLjEACZqJKQAZj9tJw52HLy5aKFY+LLkkur0504YAA80fMKU9squgLr+wZkKGSA
                  \begin{tikzcd}
                      \mathfrak g \arrow[r, "D\phi"] \arrow[d, "\exp"] & \mathfrak h \arrow[d, "\exp"] \\
                      G \arrow[r, "\phi"]                              & H
                  \end{tikzcd}
              \end{center}
              commutes;
        \item for $g \in G$ and $X \in \mathfrak g$, we have
              \[ D\phi(gXg^{-1}) = \phi(g)D\phi(X)\phi(g)^{-1}; \]
        \item $D\phi$ is a Lie group homomorphism.
    \end{enumerate}
\end{theorem}

\begin{definition}
    Let $\phi: G \to H$ be a Lie group homomorphism. Then $\phi$ is \emph{holomorphic} if $D\phi$ is $\C$-linear.
\end{definition}

\begin{example}
    For an non-example, $\det: \GL_2(\C) \to \GL_1(\C)$ is \emph{not} holomorphic. 
\end{example}

\subsection{Representations of Lie groups}

We omit the definition of a representation of a Lie group $(\rho, V)$. The only difference to our normal definition is that $\rho$ is a Lie group homomorphism. Similarly, a representation of a Lie algebra $(\sigma, W)$ is the same but $\sigma$ is a Lie algebra homomorphism. 

We highlight a key difference to our traditional representations: a representation $(\rho, V)$ of a Lie algebra $\mathfrak g$ need not satisfy $\rho(XY) = \rho(X)\rho(Y)$. In fact, it is not even certain that $XY \in \mathfrak g$. Our definition required only that $\rho$ is $\R$-linear and $\rho$ commutes with the Lie bracket $[-,-]$. 

Our notions of $G$-homomorphisms, isomorphisms, subrepresentations, and irreducibility still hold as normal for representations of Lie groups and Lie algebras. 

If $G$ is a complex Lie group, then a holomorphic representation of $G$ is a complex representation whose derivate is $\C$-linear.

\begin{theorem}
    Let $A$ be a Lie group or a Lie algebra.
    \begin{enumerate}
        \item If $V_1$ and $V_2$ are irreducible finite-dimensional representations of $A$, then
        \[
            \dim_A(V_1, V_2) =
            \begin{cases}
                1 & V_1 \cong V_2, \\
                0 & \text{else}.
            \end{cases}
        \]
        \item Any irreducible finite-dimensional representation of an abelian $A$ is $1$-dimensional.
        \item Let $(\rho, V)$. Then $\rho$ has a central character, defined on the center of $A$.
    \end{enumerate}
\end{theorem}

\begin{proposition}
    Let $(\rho, V)$ be a finite-dimensional representation of a Lie group $G$. 
    \begin{enumerate}
        \item If $W \subset V$ is invariant under $\rho(G)$, then it is invariant under $D\rho(\mathfrak g)$.
        \item If $D\rho$ is irreducible, then $\rho$ is irreducible.
        \item If $\rho$ is unitary, then $D\rho$ is skew-Hermitian.
        \item Let $(\rho', V')$ be another finite-dimensional representation of $G$. Then if $\rho \cong \rho'$, then $D\rho \cong D\rho'$.
    \end{enumerate}
    If $G$ is connected, the converse hold. 
\end{proposition}

Thus for connected Lie groups, we can test for irreducibility and isomorphisms at the level of Lie algebras.

\subsection{Standard constructions for representations of Lie groups}

Here we will present some standard constructions for representations of Lie groups. Derivatives are given and not proved, but this is not a difficult task (usually).

Let $G \subset \GL_n(\C)$ be a Lie group.
\begin{itemize}
    \item  We have the obvious action of $g \in G$ on $\mathbb C^n$:
    \[ \rho(g) = g, \qquad D\rho(X) = X. \]
    \item Let $(\rho, V)$ and $(\sigma, W)$ be two representations of $G$. The direct sum $\rho \oplus \sigma$ has derivative
    \[ D(\rho \oplus \sigma) = D\rho \oplus D\sigma. \]
    \item We have the determinant representation $\det: G \to \C$, with $D\det = \tr$. 
    \item For a representation $(\rho, V)$ of $G$, the dual representation $(\rho^*, V^*)$ is defined by
    \[ (\rho^(g)(\lambda))(v) = \lambda(\rho(g^{-1})(v)) \]
    for $\lambda \in V^*$. We have
    \[ D\rho^*(X)(\lambda)(v) = -\lambda(D\rho(X)v). \]
    \item For two representations $(\rho, V)$ and $(\sigma, W)$ of $G$, then the tensor product representation $(\rho \otimes \sigma, V \otimes W)$ is a representation where
    \[ (\rho \otimes \sigma)(g) = \rho(g) \otimes \sigma(g) \]
    and
    \[ D(\rho \otimes \sigma)(g) = D\rho(g) \otimes \id_W + \id_V \otimes D\sigma(g). \]
    \item We also have the symmetric powers and alternating powers, which we consider as quotients as the tensor product representation and thus we will omit here. 
\end{itemize}

\subsection{The adjoint representation}

Let $G$ be a Lie group and $\mathfrak g$ its Lie algebra. We have seen that $\mathfrak g$ is closed under conjugation by $G$; that is, for all $X \in \mathfrak g$ and $g \in G$, we have $gXg^{-1} \in \mathfrak g$. Thus we have an action on $\mathfrak g$ by conjugation, called the \emph{adjoint representation}.

\begin{definition}[Adjoint representation]
    Let $G$ be a Lie group and $\mathfrak g$ its Lie algebra. The \emph{adjoint representation} $(\Ad, \mathfrak g)$ of $G$ is defined by
    \begin{align*}
        \Ad: G &\to \GL(\mathfrak g), \\
        \Ad(X)g &= gXg^{-1}.
    \end{align*}
    We similarly have the \emph{adjoint representation} $(\ad, \mathfrak g)$ of $\mathfrak g$ where
    \begin{align*}
        \ad: \mathfrak g &\to \gl(\mathfrak g), \\
        \ad &= D\Ad.
    \end{align*}
\end{definition}

We may write $\Ad_g(X)$ instead of $\Ad(g)(X)$, a similarly $\ad_X(Y)$ instead of $\ad(X)(Y)$. 

We have seen that for Lie group homomorphisms, the following diagram commutes.
\begin{center}
    % https://tikzcd.yichuanshen.de/#N4Igdg9gJgpgziAXAbVABwnAlgFyxMJZABgBpiBdUkANwEMAbAVxiRAB12BbOnACwBmAJzoBrAAQBzEAF9S6TLnyEUARnJVajFm049+wseL6z5IDNjwEiZVZvrNWiEAHFTCy8qLq71BzucACVlNGChJeCJQYQguJDIQHAgkdS1HNgARTjQ+LHcQGLjEACZqJKQAZj9tJw52HLy5aKFY+LLkkur0504YAA80fMKU9squgLr+wZkKGSA
    \begin{tikzcd}
        \mathfrak g \arrow[r, "D\phi"] \arrow[d, "\exp"] & \mathfrak h \arrow[d, "\exp"] \\
        G \arrow[r, "\phi"]                              & H
    \end{tikzcd}
\end{center}
Thus 
\[ \Ad_{\exp{tX}} = \exp_{t\ad X}. \]

\begin{theorem}
    Let $G$ be a Lie group with Lie algebra $\mathfrak g$ and $X, Y \in \mathfrak g$.
    \begin{enumerate}
        \item $\ad_X(Y) = [X,Y] = XY - YX$
        \item $\ad$ is a Lie algebra homomorphism, so
        \[ \ad_{[X,Y]} = [\ad_X, \ad_Y] \]
        and the Jacobi identity holds. 
    \end{enumerate}
\end{theorem}

\begin{proposition}
    Let $G$ be a Lie group and $\mathfrak g$ its Lie algebra. If $G$ is abelian, so is $\mathfrak g$. If $G$ is connected, the converse holds. 
\end{proposition}

\subsection{Maschke's theorem}

The main corollary of Maschke's theorem was that we can decompose a representation into the directed sum of irreducible representations. But this does \emph{not} hold for infinite groups. 

For example, we consider $G = (\R, +)$ and a representation $(\rho, \C^2)$ given by
\[
    \rho(x) =
    \begin{pmatrix}
        1 & x \\
        0 & 1
    \end{pmatrix}.
\]
This is reducible, in particular, $\langle e_1 \rangle$ is invariant under $\rho(g)$ for all $g \in G$. But we claim this is not decomposable. Indeed, it can be shown that $e_1$ is the only eigenvector of $\rho(g)$ up to scalar.

\begin{theorem}
    Every finite-dimensional representation of a compact Lie group is decomposable.
\end{theorem}

Some compact Lie groups:
\begin{itemize}
    \item $U(n)$;
    \item $\SU(n)$; and
    \item $\SO(n)$.
\end{itemize}

Some non-compact Lie groups:
\begin{itemize}
    \item $\SL$; and 
    \item $\GL$.
\end{itemize}

\begin{theorem}
    Let $(p, V)$ be an irreducible finite-dimensional representation of $U(1)$ over $\C$. Then
    \begin{enumerate}
        \item $\dim V = 1$; and
        \item $\rho: U(1) \to \C^\times$ has form $p(z) = z^n$ for some $n \in \Z$. 
    \end{enumerate}
\end{theorem}

\section{$\sltwo$}

The aims of this section are as follows.
\begin{itemize}
    \item Classify the irreducible, finite-dimensional, $\mathbb C$-linear representations of $\sltwo$.
    \item Form methods for decomposing reducible representations of $\sltwo$.
\end{itemize}

We recap below.

\begin{itemize}
    \item Denote $\gl[n, \mathbb K]$ as the set of $n \times n$ matrices with entries in $\mathbb K$.
    \item A \emph{linear Lie group} is a closed (in the topological sense) subgroup of $\GL_n(\mathbb C)$ for some $n \in \mathbb N$.
    \item Let $X \in \gl[n, \mathbb C]$. Then
          \[ \exp(X) = \sum_{n=0}^\infty \frac{X^n}{n!}. \]
    \item The \emph{Lie algebra} $\mathfrak g$ of a linear Lie group $G$ is
          \[ \mathfrak g = \{X \in \gl[n, \mathbb C]: \exp(\mathbb RX) \subset G\}. \]
          We may consider the \emph{Lie functor}, $\Lie(-): \GL_2(\mathbb C) \to \gl[n, \mathbb C]$.
    \item $\Lie(\SL_n(\mathbb K)) = \sl[n, \mathbb K] = \{X \in \gl[n, \mathbb K]: \tr(X) = 0\}$.
\end{itemize}

The \emph{standard basis} for $\sltwo$ is
\[
    H =
    \begin{pmatrix}
        1 & 0 \\ 0 & -1
    \end{pmatrix}, \qquad
    X =
    \begin{pmatrix}
        0 & 1 \\ 0 & 0
    \end{pmatrix}, \qquad
    Y =
    \begin{pmatrix}
        0 & 0 \\ 1 & 0
    \end{pmatrix}.
\]

The idea here is to study representations $(\rho, V)$ of $\sltwo$ by looking at the eigenvectors and eigenvalues of $\rho(G)$.

\subsection{Weights}

\begin{definition}[Weight vector]
    Let $(\rho, V)$ be a $\C$-linear representation of $\sltwo$. Then a \emph{weight vector} in $V$ is an eigenvector of $\rho(H)$. The eigenvalue is called the \emph{weight}.
\end{definition}

\begin{example}
    \begin{enumerate}
        \item Consider the trivial representation $(\rho, \mathbb C)$ where $\rho(A) = 0$ for all $A \in \sltwo$. Pick basis $e \in \mathbb C^\times$ for $\mathbb C$. We have $\rho(H) = 0$ and so the (sole) weight vector is $e$, and its weight is $0$.
        \item Consider the standard representation $(\rho, \mathbb C^2)$, where $\rho(A) = A$ for all $A \in \sltwo$. So $\rho(H) = H$, so our weight vectors are $e_1, e_2$ with eigenvalues $1, -1$. Given that $e_1$ has weight $1$ and $e_2$ has weight $-1$, we may relabel $e_1 = e_1$ and $e_2 = e_{-1}$.
        \item Consider the representation $(\ad, \sltwo)$. We examine how $\ad$ acts on the standard basis:
              \begin{align*}
                  \ad_H(X) = [H,X] & = 2X,  \\
                  \ad_H(Y) = [H,Y] & = -2Y, \\
                  \ad_H(H) = [H,H] & = 0.
              \end{align*}
              Thus we have weight vectors $X$, $Y$, and $H$ with weights $2$, $-2$, and $0$ respectively.
        \item Consider the representation $(\rho, \C^2 \otimes \C^2)$ of $\sltwo$ which is the tensor of two standard representations. We pick the standard basis of $\C^2 \otimes \C^2$:
              \[ e_1 \otimes e_1, e_1 \otimes e_2, e_2 \otimes e_1, e_2 \otimes e_2. \]
              See that
              \[ (\rho)(H)(e_1 \otimes e_1) = \rho(H)e_1 \otimes e_1 + e_1 \otimes \rho(H) e_1 = 2e_1 \otimes e_1. \]
              In fact, we have the following lemma.
              \begin{lemma}
                  If $v$ is a weight vector of weight $\alpha$ and $w$ is a weight vector of weight $\beta$, then $v \otimes w$ is a weight vector of weight $\alpha + \beta$.
              \end{lemma}
              This can be seen by working through the above example. Thus our weights are $2$, $0$, $0$, and $-2$.
        \item Consider the standard representation $(\rho, \Sym^k(\C^2))$ of $\sltwo$. We pick the basis $\{e_1^ae_{-1}^{k-a}: 0 \leq a \leq k\}$. Then
              \begin{align*}
                  \rho(H)(e_1^ae_{-1}^{k-a})
                   & = \left(\rho(H)e_1^a\right)e_{-1}^{k-a} + e_1^a\left(\rho(H)e_{-1}^{k-a}\right) \\
                   & = ae_1^ae_{-1}^{k-a} - (k-a) e_1^ae_{-1}^{k-a}                                  \\
                   & = (2a-k) e_1^a e_{-1}^{k-a}.
              \end{align*}
              So $e_1^a e_{-1}^{k-a}$ is a weight vector with weight $2a - k$. Thus our weights are $\{-k, 2-k, 4-k, \ldots, k-4, k-2, k\}$.  For example, when $k = 5$ we get $\{-5, -3, -1, 1, 3, 5\}$ as our weights.
    \end{enumerate}
\end{example}

We now look at how $\rho(X)$ and $\rho(Y)$ act on weight vectors. For a representation $(\rho, V)$ of $\sltwo$, let
\[ V_\alpha = \{v \in V: \rho(H)v = \alpha v\} \]
be the \emph{eigenspace} for $\rho(H)$ with eigenvalue $\alpha \in \C$. We note that we have the decomposition
\[ V = \bigoplus_{\alpha} V_\alpha \]
where we direct sum over the weights of $V$.

\begin{proposition}
    Let $(\rho, V)$ be a $\C$-linear representation of $\sltwo$. Let $\alpha$ be a weight of $V$. Then
    \begin{align*}
        \rho(X)V_\alpha & \subset V_{\alpha + 2}, \\
        \rho(Y)V_\alpha & \subset V_{\alpha - 2}.
    \end{align*}
\end{proposition}

We view $\rho(X)$ as a \emph{raising operator}, and $\rho(Y)$ as a \emph{lowering operator}.

\begin{definition}[Highest weight vector]
    Let $(\rho, V)$ be a representation of $\sltwo$. A \emph{highest weight vector} $v \in V$ is a weight vector such that $\rho(X)v = 0$. The weight of $v$ is the \emph{highest weight}.
\end{definition}

\begin{example}
    \begin{enumerate}
        \item Consider the standard representation $(\rho, \Sym^5(\mathbb C^2))$ of $\sltwo$. Here $v_5 = e_1^5$ is the highest weight, it can easily be checked that $\rho(X)v_5 = 0$.
        \item Consider the tensor product of two standard representations $(\rho, \C^2 \otimes \C^2)$. We see that $e_1 \otimes e_1$ is a highest weight vector, but we also have another. We claim that $e_1 \otimes e_{-1} - e_{-1} \otimes e_1$ is a highest weight vector. Both $e_1 \otimes e_{-1}$ and $e_{-1} \otimes e_1$ have weight $0$, so $e_1 \otimes e_{-1} - e_{-1} \otimes e_1$ has weight $0$. But it can be checked that $\rho(X)$ kills $e_1 \otimes e_{-1} - e_{-1} \otimes e_1$.
    \end{enumerate}
\end{example}

\subsection{Classification of representations of $\sltwo$}

\begin{corollary}
    Any finite-dimensional representation of $\sltwo$ has a highest weight vector.
\end{corollary}

\begin{theorem}
    \begin{enumerate}
        \item For every $k \in \N_0$, there is a unique $\C$-linear and finite-dimensional representation (up to isomorphism) of $\sltwo$ such that it has a highest weight vector of weight $k$.
        \item Every finite-dimensional $\C$-linear irreducible representation of $\sltwo$ is isomorphic to one of the above representations.
    \end{enumerate}
\end{theorem}

\begin{proof}
    \begin{enumerate}
        \item We can just consider $e_1^k \in \Sym^k(\C^2)$.
        \item For this, we just show that $\Sym^k(\C^2)$ is irreducible. Let $W \subset \Sym^k(\C^2)$ be a subrepresentation. $W$ has a highest weight vector of the form $e_1^ae_{-1}^b$ with $a+b = k$, $a \geq 0$, and $b \leq k$. We also have $\rho(X)(e_1^ae_{-1}^b) = be_1^{a+1}e_{-1}^{b-1} \neq 0$ for $b > 0$. Thus $e_1^k$ is the unique highest weight vector in $\Sym^k(\C^2)$, so $e_1^k \in W$. We see that $W$ is invariant under $\rho(Y)$, and by repeated applications of $\rho(Y)$ we see that the entire basis of $\Sym^k(\C^2)$ is in $W$, and thus $W = \Sym^k(\C^2)$. Thus $\Sym^k(\C^2)$ is irreducible.
    \end{enumerate}
\end{proof}

\begin{lemma}
    Let $(\rho, V)$ be a $\C$-linear representation of $\sltwo$. If $v \in V$ is a highest weight vector with weight $k$. Then
    \[ XY^mv = m(k-m+1)Y^{m-1}v \]
    for all $m \in \N_0$.
\end{lemma}

\begin{lemma}
    If $(\rho, V)$ is a finite-dimensional irreducible representation of $\sltwo$ and $v \in V$ is a highest weight vector with weight $k \in \N_0$, then
    \[ V = \langle v, Yv, \ldots, Y^kv \rangle. \]
\end{lemma}

\begin{corollary}
    If $(\rho, V)$ is a irreducible representation of $\sltwo$, then $\rho(H)$ is diagonalisable.
\end{corollary}

\subsection{Decomposing $\sltwo$}

\begin{lemma}
    Every $A \in \sltwo$ can be written uniquely as $X + iY$ for $X, Y \in \su[n]$.
\end{lemma}

\begin{proof}
    We have $\su[n] = \{X \in \sl[n, \C]: X + X^\dagger = 0\}$. Write
    \[ A = \frac12(A - A^\dagger) - \frac i2(A + A^\dagger). \]
    Both components here are in $\su[n]$, and by arguing on the dimensions of $\sl[n, \C]$ and $\su[n]$ we see that this must be unique.
\end{proof}

\begin{lemma}
    There is a bijection between the $\C$-linear representations of $\sl[n, \C]$ and the complex representations of $\su[n]$.
\end{lemma}

\begin{proof}
    $\tilde\rho(X + iY) = \rho(X) + i\rho(Y)$.
\end{proof}

\begin{theorem}[Complete reducibility for $\sltwo$]
    Let $V$ be a finite-dimensional $\C$-linear representation of $\sltwo$. Then
    \[ V \cong \bigoplus_{i=1}^r V_i\]
    where each $V_i$ is a irreducible representation.
\end{theorem}

\begin{proof}
    By the previous lemma, it is enough to show that $V$ decomposes into irreducible representations of $\su[n]$. As $\SU(n)$ is simply connected, there is a representation $\hat\rho$ of $\SU(n)$ on $V$ whose derivative is $\rho$. $\su[n]$ is compact, so $\hat\rho$ decomposes (by Maschke's theorem). Finally, as $\su[n]$ is connected, so $\hat\rho$ decomposes on $\su[n]$.
\end{proof}

\subsection{Decomposing tensor products}

We recall that
\begin{align*}
    \{\text{weights of $V \otimes W$}\}
     & = \{\text{weights of $V$}\} + \{\text{weights of $W$}\},              \\
    \{\text{weights of $\Sym^k(V)$}\}
     & = \{\text{sum of unordered $k$-tuplets of weights of $V$}\},          \\
    \{\text{weights of $\Lambda^k(V)$}\}
     & = \{\text{sum of unordered $k$-tuplets of distinct weights of $V$}\}. \\
\end{align*}

For example, if a representation $(\rho, V)$ of $\sltwo$ has weights $\{-2,0,0,2\}$, then $\Lambda^2(V)$ has weights
$\{-2,-2,0,0,2,2\}$. Recall we are using multisets here. Similarly, $\Lambda^3(V) = \{-2,0,0,2\}$.

We now give a general method for decomposing tensor products into other representations.

Given the multisets of weights of a representation $(\rho, V)$ of $\sltwo$:
\begin{enumerate}
    \item let $k$ be the biggest weight;
    \item any weight vector $v \in V$ of weight $k$ must be a highest weight vector, thus
          \[ \langle v, Yv, \ldots, Y^kv \rangle \cong \Sym^k(\mathbb C^2); \]
    \item by complete reducibility
          \[ V \cong \Sym^k(\mathbb C^2) \oplus V' \]
          where $V'$ has the weights of $V$ with the weights $\{-k, -k+2, \ldots, k-2, k\}$ removed.
\end{enumerate}

\begin{theorem}
    A representation $(\rho, V)$ of $\sltwo$ is determined up to isomorphism by its weights.
\end{theorem}

This theorem allows us to apply the above technique without worry.

\begin{example}
    Let $V = \Sym^2(\mathbb C^2)$ where $\mathbb C^2$ is the standard representation. We will decompose $V \otimes V$ into irreducible representations and irreducible subrepresentations. First, we first the weights of $V \otimes V$.
    \begin{align*}
        \{\text{weights of $V \otimes V$}\}
         & = \{\text{weights of $V$}\} + \{\text{weights of $V$}\} \\
         & = \{-2, 0, 2\} + \{-2, 0, 2\}                           \\
         & = \{-4, -2, -2, 0, 0, 0, 2, 2, 4\}.
    \end{align*}
    We draw the following weight diagram.
    \begin{center}
        \begin{tikzpicture}
            \node[weight] (0) {};
            \node[mult1] at (0) {};
            \node[mult2] at (0) {};
            \node[mult3] at (0) {};
            \node[below of=0] {0};

            \node[weight] (-1) [left of=0] {};
            \node[mult1] at (-1) {};
            \node[mult2] at (-1) {};
            \node[below of=-1] {-2};

            \node[weight] (1) [right of=0] {};
            \node[mult1] at (1) {};
            \node[mult2] at (1) {};
            \node[below of=1] {-4};

            \node[weight] (-2) [left of=-1] {};
            \node[mult1] at (-2) {};
            \node[below of=-2] {2};

            \node[weight] (2) [right of=1] {};
            \node[mult1] at (2) {};
            \node[below of=2] {4};
        \end{tikzpicture}
    \end{center}
    So, using our method outlined above, we get the following.
    \begin{center}
        \begin{tikzpicture}
            \node[weight] (0.0) {};
            \node[weight] (0.-1) [left of=0.0] {};
            \node[weight] (0.1) [right of=0.0] {};
            \node[weight] (0.-2) [left of=0.-1] {};
            \node[weight] (0.2) [right of=0.1] {};
            \node [right=25mm] {$\Sym^4(\C^2)$};
            
            \node[weight] (A) [below of=0.0] {};
            \node[weight] (1.-1) [left of=A] {};
            \node[weight] (1.1) [right of=A] {};
            \node [right=24mm of A] {$\Sym^2(\C^2)$};

            \node[weight] (B) [below of=A] {};
            \node [right=24mm of B] {$\C$};
        \end{tikzpicture}
    \end{center}
    Now we decompose into subrepresentations. $V$ has basis $v_2 = e_1^2$, $v_0 = e_1e{-1}$, and $v_{-2} = e_{-1}^2$. We observe how $X$ and $Y$ acts on these.
    $X(v_2) = 0$, $X(v_0) = v_2$, and $X(v_{-2}) = 2v_0$. Similarly $Y(v_2) = (2v_0)$, $Y(v_0) = v_{-2}$, and $Y(v_{-2}) = 0$. From here, it is easy to piece the highest weight vectors by looking at how $X$ acts on various combinations (or known highest weight vectors).
\end{example}

\section{$\slthree$}

The aims of this section is similar to the previous: classify irreducible finite-dimensional $\C$-linear representations of $\slthree$ by the highest weights.

\begin{example}[Some representations of $\slthree$]
    \begin{enumerate}
        \item The standard representations on $\C^3$, with basis $e_1, e_2, e_3$.
        \item The dual standard representation on $(\C^3)^*$, with basis $e_1^*, e_2^*, e_3^*$ (we note that we did not use this representation in the previous section as dualling on $\sltwo$ reflects the weights about the origin).
        \item The adjoint representation $(\ad, \slthree)$.
        \item The tensor of symmetric powers $\Sym^a(\C^3) \otimes \Sym^b((\C^3)^*)$ (which is sadly not irreducible).
    \end{enumerate}
\end{example}

We proceed similar to before, but we need to redefine our notion of \emph{weights}.

\begin{definition}[Standard Cartan subalgebra]
    The \emph{standard Carton subalgebra} is the abelian subalgebra of $\slthree$
    \[
        \mathfrak h = \left\{
        \begin{pmatrix}
            h_1 & 0   & 0   \\
            0   & h_2 & 0   \\
            0   & 0   & h_3 \\
        \end{pmatrix}
        : h_1 + h_2 + h_3 = 0
        \right\}
        \subset \slthree.
    \]
\end{definition}

This is abelian as diagonal matrices commute.

\begin{definition}
    If $(\rho, V)$ is a $\C$-linear representation of $\slthree$, a \emph{weight vector} $v \in V$ is a simultaneous eigenvector of $\{\rho(H): H \in \mathfrak h\}$. The \emph{weight} $\alpha$ of $v$ is a linear map $\alpha: \mathfrak h \to \C$ such that $\rho(H)v = \alpha(H)v$. The \emph{weight space} of weight $\alpha$ is
    \[
        V_\alpha = \left\{
        v \in V: \text{$\rho(H) v = \alpha(H)V$ for all $H \in \mathfrak h$}
        \right\}.
    \]
\end{definition}

By simultaneous, we mean that it is an eigenvector regardless of the $H$ chosen.

We denote $E_{ij}$ for the matrix with a $1$ in entry $(i,j)$ and 0 elsewhere. We note that $E_{ij} \in \slthree$ if and only if $i \neq j$. We pick a basis of $\mathfrak h$ as the elements
\[
    H_{12} = E_{11} - E_{22} =
    \begin{pmatrix}
        1 & 0  & 0 \\
        0 & -1 & 0 \\
        0 & 0  & 0 \\
    \end{pmatrix}, \qquad
    H_{23} = E_{22} - E_{33} =
    \begin{pmatrix}
        0 & 0 & 0  \\
        0 & 1 & 0  \\
        0 & 0 & -1 \\
    \end{pmatrix},
\]
and we also define $H_{13} = H_{12} + H_{23}$. It will be enough to study the eigenvectors of $\rho(H_{12})$ and $\rho(H_{23})$.

\begin{example}
    \begin{enumerate}
        \item Let $(\rho, \C^3)$ be the standard representation with basis $e_1, e_2, e_3$. Then for $i \in \{1,2,3\}$ and $H \in \mathfrak h$, we have $\rho(H) e_i = L_i(H)e_i$ where $L_i(H) = h_i$ ($H = h_1E_{11} + h_2E_{22} + h_3E_{33}$). Thus we have the weight vectors being the $e_i$'s with respective weights being the $L_i$'s. Here $L_1, L_2, L_3$ span $\mathfrak h^*$, and there is one relation between them: $L_1 + L_2 + L_3 = 0$. Thus any element of $\mathfrak h^*$ can be written as $aL_1 - bL_3$ with $a, b \in \C$.
        \item Let $(\rho^*, (\C^3)^*)$ be the dual representation. Then $He_1^* = -h_ie_i$ (should be checked). Thus, the weights of the dual representation are $\{-L_1, -L_2, -L_3\}$.
        \item Consider the adjoint representation $(\ad, \mathfrak g)$ where $\mathfrak g = \slthree$. We see that
              \[ \ad_H(H') = [H, H'] = 0 \]
              for all $H, H' \in \mathfrak h$, thus $0$ is a weight of the adjoint representation. Thus,
              \[ \mathfrak g_0 := \text{$0$-weight space of $\mathfrak g$} = \mathfrak h \]
              (note we only proved that $\mathfrak h \subset \mathfrak g_0$, but this is indeed true). See that
              \[ [H, E_{ij}] = (h_i - h_j)E_{ij} \]
              for $H \in \mathfrak h$ and $i \neq j$. Thus $E_{ij} \in \slthree$ for $i \neq j$ is a weight vector with weight $h_i - h_j = L_i - L_j$.
    \end{enumerate}
\end{example}

\begin{definition}
    A \emph{root} of $\slthree$ is a non-zero weight of the adjoint representation. A \emph{root vector} is a weight vector of a root, and a \emph{root space} is the weight space of a root.
\end{definition}

We write
\[ \Phi = \{\pm(L_1 - L_2), \pm(L_2 - L_3), \pm(L_1 - L_3)\} \]
for the set of roots of $\slthree$. We call
\[ \Phi^+ = \{L_1 - L_2, L_2 - L_3, L_1 - L_3\} \]
the \emph{positive roots} and
\[ \Phi^+ = \{L_2 - L_1, L_3 - L_2, L_3 - L_1\} \]
the \emph{negative roots}. We write
\[ \Delta = \{L_1 - L_2, L_2 - L_3\}, \]
these are called the \emph{simple roots}. We may write $\alpha_{ij}$ for the root $L_i - L_j$.

Finally, we have the \emph{root space}, also called the \emph{Cartan decomposition}
\[ \mathfrak g = \mathfrak h \oplus \bigoplus_{\alpha \in \Phi} \mathfrak g_\alpha. \]

\subsection{Visualising weights}

\begin{theorem}
    Let $(\rho, V)$ be a finite-dimensional $\C$-linear representation of $\slthree$, then all its weights are elements of
    \[
        \Lambda_W = \left\{
        aL_1 - bL_3: a,b \in \Z
        \right\}
    \]
    called the \emph{weight lattice}.
\end{theorem}

\begin{proof}
    Let $\alpha = aL_1 - bL_3$ for $a,b \in \C$ be a weight of $V$. We have to prove that $a, b \in \Z$. We sketch the proof here. We consider the embedding
    \begin{align*}
        \sltwo & \xhookrightarrow{} \slthree \\
        H &\mapsto \left(
            \begin{array}{c|c}
                H & 0 \\ \hline
                0 & 0 \\
            \end{array}
        \right).
    \end{align*}
    By our $\sltwo$-theorem, all eigenvalues acting on $V$ are integers, thus $a \in \Z$. For $b \in \Z$, we use the similar embedding:
    \begin{align*}
        \sltwo & \xhookrightarrow{} \slthree \\
        H &\mapsto \left(
            \begin{array}{c|c}
                0 & 0 \\ \hline
                0 & H \\
            \end{array}
        \right).
    \end{align*}
\end{proof}

To visualise our weights: put $L_1$, $L_2$, and $L_3$ as vertices of an equilateral triangle. Then $\Lambda_W$ is the lattice generated by these. 

\begin{example}
    Weights for the standard representation on $\C^3$. 
    \begin{center}
        \begin{tikzpicture}
            \node[weight] at (1,0) {};
            \node[weight] at (2,0) {};
            \node[weight] at (3,0) {};
            \node[weight] at (4,0) {};
            \node[weight] at (5,0) {};
            \node[weight] at (6,0) {};

            \node[weight] at (1.5,1) {};
            \node[weight] at (2.5,1) {};
            \node[weight] at (3.5,1) {};
            \node[weight] at (4.5,1) {};
            \node[weight] at (5.5,1) {};
            \node[weight] at (6.5,1) {};

            \node[weight] at (1,2) {};
            \node[weight] at (2,2) {};
            \node[weight] at (3,2) {};
            \node[weight] at (4,2) {};
            \node[weight] at (5,2) {};
            \node[weight] at (6,2) {};

            \node[weight] at (1.5,3) {};
            \node[weight] at (2.5,3) {};
            \node[weight] at (3.5,3) {};
            \node[weight] at (4.5,3) {};
            \node[weight] at (5.5,3) {};
            \node[weight] at (6.5,3) {};

            \node[weight] at (1,4) {};
            \node[weight] at (2,4) {};
            \node[weight] at (3,4) {};
            \node[weight] at (4,4) {};
            \node[weight] at (5,4) {};
            \node[weight] at (6,4) {};

            \node[mult1] at (5,2) {};
            \node[mult1] at (3.5,3) {};
            \node[mult1] at (3.5,1) {};

            \node at (4.35,2) {$0$};
            \node at (5.35,2) {$L_1$};
            \node at (3.85,3) {$L_2$};
            \node at (3.85,1) {$L_3$};
        \end{tikzpicture}
    \end{center}
\end{example}

\begin{example}
    We now consider the weights of the adjoint representation.
    \begin{center}
        \footnotesize
        \begin{tikzpicture}
            \node[weight] at (-3,0) {};
            \node[weight] at (-2,0) {};
            \node[weight] at (-1,0) {};
            \node[weight] at (0,0) {};
            \node[weight] at (1,0) {};
            \node[weight] at (2,0) {};

            \node[weight] at (-2.5,1) {};
            \node[weight] at (-1.5,1) {};
            \node[weight] at (-0.5,1) {};
            \node[weight] at (0.5,1) {};
            \node[weight] at (1.5,1) {};
            \node[weight] at (2.5,1) {};

            \node[weight] at (-3,2) {};
            \node[weight] at (-2,2) {};
            \node[weight] at (-1,2) {};
            \node[weight] at (0, 2) {};
            \node[weight] at (1, 2) {};
            \node[weight] at (2, 2) {};

            \node[weight] at (-2.5,-1) {};
            \node[weight] at (-1.5,-1) {};
            \node[weight] at (-0.5,-1) {};
            \node[weight] at (0.5, -1) {};
            \node[weight] at (1.5, -1) {};
            \node[weight] at (2.5, -1) {};

            \node[weight] at (-3,-2) {};
            \node[weight] at (-2,-2) {};
            \node[weight] at (-1,-2) {};
            \node[weight] at (0, -2) {};
            \node[weight] at (1, -2) {};
            \node[weight] at (2, -2) {};


            \node[mult1] at (0,0) {};
            \node[mult2] at (0,0) {};
            \node at (0, 0) [xshift=10] {$0$};
            
            \node[mult1] at (1.5,1) {};
            \node at (1.5, 1) [xshift=12] {$\alpha_{13}$};

            \node[mult1] at (0,2) {};
            \node at (0, 2) [xshift=12] {$\alpha_{23}$};

            \node[mult1] at (-1.5,1) {};
            \node at (-1.5, 1) [xshift=12] {$\alpha_{21}$};

            \node[mult1] at (-1.5,-1) {};
            \node at (-1.5, -1) [xshift=12] {$\alpha_{31}$};

            \node[mult1] at (0,-2) {};
            \node at (0, -2) [xshift=12] {$\alpha_{32}$};

            \node[mult1] at (1.5,-1) {};
            \node at (1.5, -1) [xshift=12] {$\alpha_{12}$};
        \end{tikzpicture}
    \end{center}
\end{example}

\begin{example}
    Consider $\Sym^2(\C^3)$ where $\C^3$ is the standard representation. Our weight vectors are of the form $e_ie_j$ for $1 \leq i \leq j \leq 3$. We have
    \[ H(e_ie_j) = H(e_i)e_j + e_iH(e_j) = (L_i + L_j)(H) e_ie_j. \]
    Thus the weight of $e_ie_j$ is $L_i + L_j$. Considering every $i$ and $j$, we get
    \[ \text{weights} = \{2L_1, 2L_2, 2L_3, L_1 + L_2, L_2 + L_3, L_1 + L_3\}. \]
    Thus we draw our weights as follows. 
    \begin{center}
        \begin{tikzpicture}
            \node[weight] at (-3,0) {};
            \node[weight] at (-2,0) {};
            \node[weight] at (-1,0) {};
            \node[weight] at (0,0) {};
            \node[weight] at (1,0) {};
            \node[weight] at (2,0) {};

            \node[weight] at (-2.5,1) {};
            \node[weight] at (-1.5,1) {};
            \node[weight] at (-0.5,1) {};
            \node[weight] at (0.5,1) {};
            \node[weight] at (1.5,1) {};
            \node[weight] at (2.5,1) {};

            \node[weight] at (-3,2) {};
            \node[weight] at (-2,2) {};
            \node[weight] at (-1,2) {};
            \node[weight] at (0, 2) {};
            \node[weight] at (1, 2) {};
            \node[weight] at (2, 2) {};

            \node[weight] at (-2.5,-1) {};
            \node[weight] at (-1.5,-1) {};
            \node[weight] at (-0.5,-1) {};
            \node[weight] at (0.5, -1) {};
            \node[weight] at (1.5, -1) {};
            \node[weight] at (2.5, -1) {};

            \node[weight] at (-3,-2) {};
            \node[weight] at (-2,-2) {};
            \node[weight] at (-1,-2) {};
            \node[weight] at (0, -2) {};
            \node[weight] at (1, -2) {};
            \node[weight] at (2, -2) {};

            \node at (0, 0) [xshift=10] {$0$};
            \node at (1, 0) [xshift=9] {$L_1$};
            \node at (-0.5, 1) [xshift=9] {$L_2$};
            \node at (-0.5, -1) [xshift=9] {$L_3$};

            \node[mult1] at (2,0) {};
            \node[mult1] at (0.5,1) {};
            \node[mult1] at (-1,2) {};
            \node[mult1] at (-1,0) {};
            \node[mult1] at (-1,-2) {};
            \node[mult1] at (0.5,-1) {};
        \end{tikzpicture}
    \end{center}
\end{example}

We present a fundamental weight calculation, as we did with $\sltwo$. 

\begin{theorem}[Fundamental weight calculation]
    Let $(\rho, V)$ be a $\C$-linear representation of $\slthree = \mathfrak g$ and let $v \in V_\beta$ be a weight vector with weight $\beta \in \mathfrak h^*$. Let $\alpha \in \mathfrak h^*$ be a root and let $X_\alpha \in \mathfrak g_\alpha$ be a root vector. Then $\rho(X_\alpha)v = 0$. 
\end{theorem}

\begin{proof}
    Let $H \in \mathfrak h$. Then
    \begin{align*}
        H(X_\alpha(v)) &= ([H, X_\alpha] + X_\alpha H)v \\
        &= \alpha(H) X_\alpha v + X\alpha \beta(H) \\
        &= (\alpha + \beta)(H) (X_\alpha v). \qedhere
    \end{align*}
\end{proof}

\begin{definition}
    Let $(\rho, V)$ be a $\C$-linear representation of $\slthree$. Then a weight vector $v \in V$ is a \emph{highest weight vector} $\rho(X)v = 0$ for $X \in \{E_{12}, E_{13}, E_{23}\}$. The \emph{highest weight} of $v$ is the weight of $v$. 
\end{definition}

As $E_{13} = [E_{12}, E_{23}]$, we only need to check $E_{12}$ and $E_{23}$. 

\begin{proposition}
    If $(\rho, V)$ is a finite-dimensional representation, then a highest weight exists. 
\end{proposition}

\begin{proof}
    Define $l: \mathfrak h^* \to \C$ by $l(aL_1 - bL_3) = a + b$. Use this function on contradiction of having a weight vector of maximal $l$ value. 
\end{proof}

\subsection{Dominant weights}

Let $(\rho, V)$ be a representation of $\slthree$. If $v \in V$ is a highest weight vector of weight $aL_1 - bL_3$, then it is a highest weight vector for $V$ under the restrictions
\[
    \left(
        \begin{array}{c|c}
            \sltwo & 0 \\ \hline
            0 & 0 \\
        \end{array}
    \right), \qquad 
    \left(
        \begin{array}{c|c}
            0 & 0 \\ \hline
            0 & \sltwo \\
        \end{array}
    \right).
\]

Its weight for the top right restriction is $a$ and its weight for the bottom right copy is $b$. 

\begin{definition}[Dominant weight]
    A \emph{dominant weight} is an element of $\mathfrak h^*$ of the form $aL_1 - bL_3$ with $a, b \in \N_0$. 
\end{definition}

\begin{theorem}
    For each dominant weight $aL_1 - bL_3$ there is a unique (up to isomorphism) finite-dimensional $\C$-linear irreducible representation of $\slthree$ with highest weight vector that of the weight. 
\end{theorem}

We call such a representation $V^{(a,b)}$. 

\begin{example}\hspace{0em}
    \begin{itemize}
        \item $V^{(0,0)} = \C$ (trivial)
        \item $V^{(1,0)} = \C^3$ (standard)
        \item $V^{(0,1)} = (\C^3)^*$
        \item $V^{(1,1)} = (\ad, \slthree)$
        \item $V^{(2,0)} = \Sym^2(\C^3)$
    \end{itemize}    
\end{example}



\end{document}
