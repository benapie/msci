\documentclass[a4paper, 11pt]{article}

% tikzcd.yichuanshen.de/ tikcd diagrams
%======================%
%   Standard packages  %
%======================%
\usepackage[utf8]{inputenc}
\usepackage[T1]{fontenc}
\usepackage[UKenglish]{babel}
\usepackage{parskip}
\usepackage{hyperref}

%%%
%%% Referencing
%%%
\usepackage[sorting=none, backend=biber]{biblatex}
\addbibresource{ref.bib}


%======================%
%        Maths         %
%======================%
\usepackage{amsfonts, mathtools, amsthm, amssymb}
\newcommand\N{\ensuremath{\mathbb{N}}}
\newcommand\R{\ensuremath{\mathbb{R}}}
\newcommand\Z{\ensuremath{\mathbb{Z}}}
\newcommand\Q{\ensuremath{\mathbb{Q}}}
\newcommand\C{\ensuremath{\mathbb{C}}}
\newcommand\F{\ensuremath{\mathbb{F}}}
\newcommand{\encode}[1]{\ensuremath \langle #1 \rangle}

\usepackage{tikz-cd}
% \usepackage{adjustbox}
\DeclareMathOperator*{\id}{id}
\DeclareMathOperator*{\coker}{coker}
\DeclareMathOperator*{\im}{im}
\DeclareMathOperator*{\rank}{rank}
\DeclareMathOperator*{\nullity}{nullity}
\DeclareMathOperator*{\pers}{pers}
\DeclareMathOperator*{\ipers}{ipers}
\DeclareMathOperator*{\dgm}{dgm}
\DeclareMathOperator*{\codim}{codim}
\DeclareMathOperator*{\low}{low}

% Code
\usepackage{minted}
  \setminted[python]{
    frame=lines,
    framesep=2mm,
    baselinestretch=1.2,
    bgcolor=LightGray,
    fontsize=\footnotesize,
    linenos
  }
\usepackage{xcolor}
  \definecolor{LightGray}{gray}{0.9}

% pseudocode
\usepackage{algpseudocode, algorithm}


%======================%
%    Pretty tables     %
%======================%
\usepackage{booktabs}
\usepackage{caption}

%======================%
%    Theorems          %
%======================%
\theoremstyle{plain}
\newtheorem{theorem}{Theorem}
\newtheorem{lemma}{Lemma}

\theoremstyle{definition}
\newtheorem{definition}{Definition}
% \newtheorem{algorithm}{Algorithm}

\newtheoremstyle{break}{\topsep}{\topsep}{}{}{\bfseries}{}
{\newline}{\thmname{#1}\thmnumber{ #2}\thmnote{ ({\normalfont\textsc{#3}})}.}
\theoremstyle{break}
\newtheorem{problem}[theorem]{Problem}

\theoremstyle{remark}
\newtheorem*{remark}{Remark}