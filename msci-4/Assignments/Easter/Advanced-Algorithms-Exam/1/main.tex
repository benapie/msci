\question Let $G = (V,E)$ be a graph. Recall that a function $c: V \to \{1,\ldots, k\}$ is a $k$-colouring if $c(u) \neq c(v)$ for every pair of adjacent vertices $u$ and $v$. For $1 \leq i \leq k$, the set $C_i = \{i \in V: c(u) = i\}$ is called a \emph{colour set} of $c$. For a pair $i,j$ with $1 \leq i < j \leq k$, let $G[i,j]$ denote the subgraph of $G$ induced by the set $C_i \cup C_j$. 

We say that a $k$-colouring $c$ of $G$ is \emph{safe} if for every pair $i,j$ with $i \neq j$, the subgraph $G[i,j]$ has maximum degree at most $3$. The \emph{safety number} of $G$ is the smallest number $k$ such that $G$ has a safe $k$-colouring.

Let \textsc{Safety Number} be the problem of deciding if a given graph $G$ has safety number at most $k$ for some given integer $k$. If $k$ is fixed (so if $k$ is not part of the input), then we denote the problem as \textsc{$k$-Safety Number}.

\begin{parts}
    \part[5] Let $\Pi$ be some $\NP$-complete graph problem. Suppose that we can solve $\Pi$ in polynomial time for $P_6$-free graphs. Does this mean that we can solve $\Pi$ in polynomial time for $2P_3$-free graphs? Justify your answer.

\begin{solution}
    If we found that the class of $2P_3$-free graphs $\mathcal G_1$ are a subclass of the class of $P_6$-free graphs $\mathcal G_2$ (that is, $\mathcal G_1 \subset \mathcal G_2$), then we could solve $\Pi$ in polynomial restricted to $\mathcal G_1$. But we do not have this: $P_6$ is a graph that is $2P_3$-free, but it is certainly not $P_6$-free. Thus, the answer is no.
\end{solution}
    \part[8] For every integer $k \geq 1$, give a graph of safety number $k$ that is not a tree. For every integer $k \geq 1$, give also a tree of safety number $k$. Justify your answers.

\begin{solution}
    We denote $\Delta(G)$ as the maximum degree of a graph $G$.

    For a non-tree and $k = 1$, we claim that $2P_1$ has safety number 1. $2P_1$ is not a tree as it is disconnected, and we have the trivial $1$-colouring where both vertices are given the same colour. Then $\Delta(G) = 0 \leq 3$, so the maximum degree of any induced subgraph is $0$. Thus the colouring is safe. $2P_1$ is not empty, thus there is no $0$-colouring, proving our claim. The graph is drawn below.

    \begin{center}
        \begin{tikzpicture}
            \node[main, fill=red!50] (1) {};
            \node[main, fill=red!50] (2) [left of=1] {};
        \end{tikzpicture}
    \end{center}

    For a non-tree and $k = 2$, we consider $2P_2 = uv + wx$. As before, $2P_2$ is not a tree as it is disconnected, and we have a $2$-colouring $c: V(2P_2) \to \{1,2\}$ where $c(u)=c(w)=1$ and $c(v)=c(x)=2$. It is clear that this is a minimal colouring, the existence of edges prohibits a $1$-colouring. This colouring is safe as $\Delta(G) = 1 \leq 3$, and so any induced subgraph has maximum degree $1$. Thus $2P_2$ has safety number $2$. The graph is drawn below.

    \begin{center}
        \begin{tikzpicture}
            \node[main, fill=red!50] (1) {};
            \node[main, fill=blue!50] (2) [left of=1] {};
            \node[main, fill=red!50] (3) [left of=2] {};
            \node[main, fill=blue!50] (4) [left of=3] {};
            \draw (1) -- (2) (3) -- (4);
        \end{tikzpicture}
    \end{center}

    For a non-tree and $k \geq 3$, we may consider the complete graph $K_k$. We have $\chi(K_k) = k$, with each vertex having a distinct colour. For $x, y \in V(K_k)$, the induced subgraph $K_k[\{x,y\}]$ is $P_2$, thus $\Delta(K_k[\{x,y\}]) = 1$. Thus this colouring is safe, and so the safety number $k$. The graph is drawn below for $k = 3$.

    \begin{center}
        \begin{tikzpicture}
            \node[main, fill=red!50] (1) {};
            \node[main, fill=blue!50] (2) [above left of=1] {};
            \node[main, fill=green!50] (3) [below left of=2] {};
            \draw (1) -- (2) -- (3) -- (1);
        \end{tikzpicture}
    \end{center}

    For a tree and $k = 1$, we consider $P_1$ and the trivial colouring, which is a safe and minimal colouring. Thus the safety number of $P_1$ is $1$. 

    For a tree and $k \geq 2$, we consider the star graph $G = S_{3(k-1)}$ and let $v \in V(G)$ be the internal node. This is always $2$-colourable, where the internal node has one colour and the leaves all share another colour. For $k = 2$ this is safe, as $\Delta(S_{k-1}) = 3$ (and so any induced subgraph must have degree at most $3$). So we now consider $k \geq 3$. We first claim that a safe $k$-colouring exists for $G$. If we partition the leaves into $k-1$ sets of size $3$, then colouring each partition a distinct colour and the internal node a distinct colour, we have us a $k$-colouring ($k-1$ colour used for the leaves, $1$ used for the internal node). This is indeed safe, the subgraph induced from any two distinct partitions is $6P_1$, which has maximum degree $0$. Similarly, the subgraph induced from any partition and the internal node is the claw graph, which has maximum degree $3$. We finally claim that there is no safe $k-1$-colouring which is safe, and thus the safe $k$-colouring above must be minimal (and thus the safety number is $k$). Suppose we have a safe $(k-1)$-colouring $c: V(G) \to \{1, \ldots, k-1\}$. $c(v)$ must have a unique colour, as it is a dominating vertex. So we let $c(v) = 1$ (without loss of generality). We now claim that there is a colour set $C$ with size at least $4$. Indeed, we have used $1$ colour for the internal node, thus we have $k-2$ colours left to colour the remaining $3(k-1)$ leaves. Observe
    \[ \frac{3(k-1)}{k-2} \geq \frac{3(k-1)}{k-1} = 3, \]
    and so at least one of the colour sets has size $4$ or more. We now consider the subgraph induced by $C \cup \{v\}$, which is $K_{1, \lvert C \rvert}$ (as $C$ is a independent set and $v$ is a dominating vertex). But $\Delta(K_{1, \lvert C \rvert}) = \lvert C \rvert \geq 4$, so thus the colouring is not safe. Therefore, the safety number of $G$ is $k$.  
\end{solution}
    \part[5] Is the class of interval graphs closed under edge contraction?

\begin{solution}
    The usual definition of an interval graph is as the intersection graph of a set of intervals on the real line, but we restrict this to the intersection graph of a set of \emph{closed} intervals on the real line, as defined by Diner et al., 2015.

    Let $\{I_i\}_{i=1}^n$ be a finite set of closed intervals where $I_i = [a_i, b_i]$ for $a, b \in \mathbb R$ with $a \leq b$, and let $G = (V,E)$ be the corresponding intersection graph; that is,
    \begin{align*}
        V & = \{v_i\}_{i=1}^n,                           \\
        E & = \{v_iv_j: I_i \cap I_j \neq \varnothing\}.
    \end{align*}
    Pick $v_iv_j \in E$. We claim that $G' = G/v_iv_j$ is an interval graph; that is, there is a finite set of intervals $\mathcal J$ such that $G'$ is the interval graph of $\mathcal J$.

    We prove this by construction. For all $k \in \{1, \ldots, n\} \setminus \{i,j\}$, let $I_k \in \mathcal J$. Recall that $G/v_iv_j$ may produced from $G$ with the following operations:
    \begin{itemize}
        \item add a new vertex labelled $v_{i \sim j}$;
        \item add an edge between $v_{i \sim j}$ and every neighbour of $v_i$ and $v_j$; then
        \item delete $v_i$ and $v_j$.
    \end{itemize}
    Thus we add the final interval, $I_{i \sim j}$, defined by
    \[
        I_{i \sim j} = \left[\min\{a_i, a_j\}, \max\{b_i, b_j\}\right]
    \]
    and let $I_{i \sim j} \in \mathcal J$. Thus we have
    \[
        \mathcal J = \{I\}_{k \in \{1,\ldots,n\} \setminus \{i,j\}} \cup \{I_{i\sim j}\}.
    \]
    We claim that the interval graph of $\mathcal J$ is $G'$. Indeed, for all $k, l \in \{1,\ldots,n\} \setminus \{i,j\}$, $v_kv_l \in E(G')$ if and only if $I_k \cap I_l \neq \varnothing$ (that is, $v_kv_l \in E(G)$).  We have left to consider $i$ and $j$. But note, $I_i, I_j \subset I_{i \sim j}$ by definition. So if, for some $l \in \{1,\ldots, n\} \setminus \{i,j\}$, $I_l \cap I_{i \sim j}$ if and only if $I_l \cap I_i$ or $I_l \cap I_j$.

    Intuitively, we can consider a contraction of two connected intervals on a interval graph as taking the union of the interval, which is also an interval as the existence of an edge ensures that they are connected. 
\end{solution}
    \clearpage\part\hspace{0em}
\begin{solution}
   We construct $H \subset G$ as follows. Let $V_H = A \cup B$. For each vertex $u \in V$, flips a coin. If the coin reads tails, put $u$ in $A$. If the coin reads heads, put $u$ in $A$. Thus, for all $u \in V$, $\Pr[u \in A] = \tfrac12$ and $\Pr[u \in B] = \tfrac12$, and $A \cap B = \varnothing$. For each $uv \in E$, we let $uv \in E_H$ if and only if:
   \begin{enumerate}
       \item $u \in A$ and $v \in B$; or 
       \item $u \in B$ and $v \in A$.
   \end{enumerate}
   Then, for all $uv \in E$
   \[ \Pr[uv \in E_H] = 1 - \Pr[(u,v \in A)\lor(u,v \in B)]. \]
   Both events $[u,v \in A]$ and $[u,v \in B]$ are mutually exclusive (that is, they can't both happen), thus
   \[ \Pr[uv \in E_H] = 1 - (\Pr[u,v \in A] + \Pr[u,v \in B]). \]
   The events $[u \in A]$ and $[v \in A]$ are independent, thus we conclude 
   \begin{align*}
        \Pr[uv \in E_H] &= 1 - (\Pr[u \in A]\Pr[v \in A] + \Pr[u \in B]\Pr[v \in B]) \\
        &= \frac12.
   \end{align*}
   Enumerate the edges of $G$ as $E = \{e_1, \ldots e_m\}$ and for all $i \in \{1, \ldots, m\}$ let $X_i$ be the indicator function for edge $e_i$ being in $H$. We have
   \[
        \E[X_i] = 1\cdot \Pr[e_i \in H] + 0\cdot \Pr[e_i \not\in H] = \tfrac12.    
   \]
   Let $X = \sum_{i=1}^m X_i$; that is, $X$ is the random variable corresponding to the number of edges in $H$ in our construction. Then
   \begin{align*}
       \E[X] &= \E\left[\sum_{i=1}^m X_i\right] \\
       &= \sum_{i=1}^m \E[X_i] \\[5pt]
       &= \frac m2.
   \end{align*}
   As the expected number of edges in our random construction $H$ is $\tfrac m2$, there must be at least one $H$ with $\tfrac m2$ edges.
\end{solution}
    \part[15] Prove that \textsc{$3$-Safety Number} is polynomial-time solvable for every graph class of bounded treewidth.

\begin{solution}
    We will first describe the property of have a safe $3$-colouring in $\operatorname{MSO}_2$.

    Let $G = (V,E)$ be a graph. We first define some formulas to make our writing more economical.
    \begin{enumerate}
        \item For $u, v \in V$, $\adj(u,v)$ is true if and only if $u$ and $v$ are adjacent, so
              \[ \adj(u,v) = \lnot (u = v) \land \exists_{e \in E} (i(u,e) \land i(v,e)). \]
        \item Given $u \in V$ and $A \subset V \setminus \{u\}$, $\adj_0(u, A)$ is true if and only if $u$ is not adjacent to any vertex in $A$. Thus
              \[ \adj_0(u, A) = \forall_{w \in A}(\lnot \adj(u,w)). \]
        \item Given $u \in V$ and $A \subset V \setminus \{u\}$, $\adj_1(u, A)$ is true if and only if $u$ is adjacent to exactly one vertex in $A$. Thus
              \[ \adj_1(u, A) = \exists_{v_1 \in A}\forall_{w \in A}(\adj(u,w) \iff (w = v_1)). \]
              We similarly define $\adj_2(u, A)$ and $\adj_3(u,A)$ as true if and only if $u$ is adjacent to two and three vertices of $A$ respectively, so
              \begin{align*}
                  \adj_2(u, A) & = \exists_{v_1, v_2 \in A}((v_1 \neq v_2)                                            \\
                               & \qquad \land \forall_{w \in A}(\adj(u,w) \iff (w = v_1 \lor w = v_2))),              \\
                  \adj_3(u, A) & = \exists_{v_1, v_2, v_3 \in A}((v_1 \neq v_2 \land v_1 \neq v_3 \land v_2 \neq v_3) \\
                               & \qquad \land \forall_{w \in A}(\adj(u,w) \iff (w = v_1 \lor w = v_2 \lor w = v_3))).
              \end{align*}
        \item Given $u \in V$ and $A \subset V \setminus \{u\}$, $\adj_{\leq 3}(u, A)$ is true if and only if $u$ is adjacent to at most $3$ vertices in $A$, so combining the last point:
              \[ \adj_{\leq 3}(u, A) = \adj_0(u,A) \land \adj_1(u,A) \land \adj_2(u,A) \land \adj_3(u,A). \]
        \item Given $A \subset V(A)$, $\I(A)$ is true if and only if $A$ is an independent set in $G$, thus
              \[ \I(A) = \forall_{u, v \in A} (\lnot\adj(u, v)). \]
        \item Given $V_1, V_2, V_3 \subset V$, $P(V_1,V_2,V_3)$ is true if and only if $\{V_1, V_2, V_3\}$ forms a partition of $G$; that is,
              \begin{align*}
                  P(V_1, V_2, V_3) = \forall_{v \in V}( & (v \in V_1 \land v \not\in V_2 \land v \not\in V_3)   \\
                  \lor                                  & (v \not\in V_1 \land v \in V_2 \land v \not\in V_3)   \\
                  \lor                                  & (v \not\in V_1 \land v \not\in V_2 \land v \in V_3)). \\
              \end{align*}
    \end{enumerate}
    We conclude with the $\MSO$ formula for a graph having a safe $3$-colouring.
    \begin{align*}
        \exists_{V_1, V_2, V_3 \subset V} ( & I(V_1) \land I(V_2) \land I(V_3) \land P(V_1, V_2, V_3)                     \\
        & \land\forall_{v \in V_1} (\adj_{\leq 3}(v, V_2) \land \adj_{\leq 3}(v, V_3) \\
        & \land\forall_{v \in V_2} (\adj_{\leq 3}(v, V_1) \land \adj_{\leq 3}(v, V_3) \\
        & \land\forall_{v \in V_3} (\adj_{\leq 3}(v, V_1) \land \adj_{\leq 3}(v, V_2)).
    \end{align*}
    As the bound for this formula is constant (does not depend on the size of $G$), we may use Courcelle's theorem to conclude that \textsc{$3$-Safety Number} is polynomial-time solvable for every class of bounded treewidth. 
\end{solution}
\end{parts}