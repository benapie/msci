\documentclass[a4paper]{article}

\usepackage[utf8]{inputenc}

\usepackage{parskip}

\usepackage{amssymb}
\usepackage{amsmath}
\usepackage{amsfonts}
\usepackage{mathtools}

\usepackage{todonotes}

\usepackage{csquotes}

\usepackage{algpseudocode}
\usepackage{algorithm}

\DeclareMathOperator{\AQ}{AQ}
\DeclareMathOperator{\DAQ}{\Delta AQ}
\DeclareMathOperator{\Q}{Q}
\DeclareMathOperator{\HE}{HE}
\DeclareMathOperator*{\argmax}{arg\,max}
\DeclareMathOperator{\Ep}{Ep}

\usepackage{braket}

\addtolength{\oddsidemargin}{-.875in}
\addtolength{\evensidemargin}{-.875in}
\addtolength{\textwidth}{1.75in}
\addtolength{\topmargin}{-.875in}
\addtolength{\textheight}{1.75in}

\usepackage[backend=biber]{biblatex}
\addbibresource{ref.bib}

\title{Natural Computing Part A}
\author{Ben Napier}
\date{March 2022}

\begin{document}

\title{TOPOLOGY III \\ Spaces, Surfaces, and Simplicial complexes}
\maketitle

\begin{definition}[$n$-simplex]
	An \emph{n-simplex} $\sigma^n$ in $\mathbb R^n$
	where $N \geq n$ is a convex hull of $(n+1)$ points
	in general position in $\mathbb R^N$. We say that such a simplex is
	$n$-dimensional.
\end{definition}

A generalisation of triangles to $n$ dimensions.

\begin{definition}[Simplicial complex]
	A \emph{simplicial complex} $K \subset \mathbb R^N$ is a finite
	union of simplicies $\sigma^n$ ($n \leq N$) contained in $\mathbb N$
	such that if $\sigma_1, \sigma_2 \in K$, then $\sigma_1 \cap \sigma_2$
	is either empty, a single common face of them both, or one is a face of 
	the other.
\end{definition}

\begin{definition}[Triangulation]
	A \emph{triangulation} of a topological space $X$ is a simplicial
	complex $K$ with a homeomorphism $h:K \to X$.
\end{definition}

\begin{definition}[Closed surface]
	A topological space $S$ is a \emph{closed surface} if it is
	non-empty, Hausdorff, compact, connected, and it is locally Euclidean
	of dimension $2$.
\end{definition}

\begin{definition}[Open star]
	Let $x \in K$ for some finite simplicial complex $K$.
	The \emph{open star} of $x$ in $K$, denoted $\operatorname{St}(x,K)$,
	is the union of the interiors of all simplicies containing $x$
	together with $x$ itself.
\end{definition}

\begin{lemma}
	If a topological space $M$ is triangulated by a connected,
	finite simplicial cmplex $K$ of dimension $2$, then $M$ is a closed
	surface if and only if for each $x \in K$, $\operatorname{St}(x,K)$
	is homeomorphic to $E^2$ (the unit open ball).
\end{lemma}

\begin{corollary}
	If $X$ is triangulated by the connected finite $2$-dimensional
	simplicial complex $K$, then $X$ is a closed surface if and only if, for
	all vertices $v$ in $K$, we have $\operatorname{St}(v,K) \cong E^2$.
\end{corollary}

\begin{definition}[Orientability]
	For a surface $S$, we say that it is \emph{orientable}
	if it has an orientable triangulation; that is, for each pair
	of simplicies that meet on a common edge incude opposite directions
	on the common edge.
\end{definition}

Constructions on surfaces:
\begin{enumerate}
	\item Given two surfaces $S_1$ and $S_2$, their \emph{connected sum} 
	$S_1 \# S_2$ is obtained by removing the interiors of two small open discs,
	one from $S_1$ and one from $S_2$, and then identifying the two boundary
	circles formed. 
	If $S_1$ and $S_2$ are oriented, we select the resulting directions
	around the boundary circle such that $S_1 \# S_2$ inherits
	the orientation.
	
	\item Adding a \emph{handle} to a surface $S$ runs as follows.
	Suppose we remove the interiors of two open discs from $S$, then attach
	to the resulting boundary circles the ends of a cylinder
	$S^1 \times I$. Again, if $S$ is orientable we choose the direction
	of the boundary circle as to inherit the orientability property.

	\item Adding a cross-cap on a surface $S$ runs as follows.
	We remove a single open disc from $S$ and glue the resulting boundary circle to the boundary circle of a M\"obius band.
	This operation does \emph{not} preserve orientability,
	the result is non-orientable.
\end{enumerate}

\begin{definition}[Simplicial map]
	A map $f$ from a simplicial complex $K$ to another $L$ is
	called \emph{simplicial} if the image of the vertices of a simplex
	(of $K$) always spans a simplex (of $L$).
\end{definition}

\end{document}