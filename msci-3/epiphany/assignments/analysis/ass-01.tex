\documentclass[a4paper, answers]{exam}
\usepackage[utf8]{inputenc}

\usepackage{parskip}

\usepackage{amssymb}
\usepackage{amsmath}
\usepackage{amsfonts}
\usepackage{mathtools}

\usepackage{todonotes}

\usepackage{csquotes}

\usepackage{algpseudocode}
\usepackage{algorithm}

\DeclareMathOperator{\AQ}{AQ}
\DeclareMathOperator{\DAQ}{\Delta AQ}
\DeclareMathOperator{\Q}{Q}
\DeclareMathOperator{\HE}{HE}
\DeclareMathOperator*{\argmax}{arg\,max}
\DeclareMathOperator{\Ep}{Ep}

\usepackage{braket}

\addtolength{\oddsidemargin}{-.875in}
\addtolength{\evensidemargin}{-.875in}
\addtolength{\textwidth}{1.75in}
\addtolength{\topmargin}{-.875in}
\addtolength{\textheight}{1.75in}

\usepackage[backend=biber]{biblatex}
\addbibresource{ref.bib}

\title{Natural Computing Part A}
\author{Ben Napier}
\date{March 2022}

\begin{document}

\begin{center}
	\textbf{\textsc{Analysis III, Epiphany Term, Assignment 1}}
	\\
	\textsc{Ben Napier}
	\vspace{1em}
\end{center}

\begin{questions}

% /////////// %
% /         / %
\question Let $(X, \lVert \cdot \rVert)$ be a normed linear space.
\begin{parts}

%%%%%%%%%%%%%%%%%%%%%%%%%%%%%%%%%
\part Prove that if $(x_n)_n$ converges in $X$, then $(x_n)_n$
	  is a Cauchy sequence in $X$.
\begin{solution}
	Suppose $(x_n)$ converges to $x$.
	Let $\varepsilon > 0$, then there is $N \in \mathbb N$ such that
	for each $n \geq N$ we have 
	$\lVert x_n - x \rVert < \sfrac\varepsilon2$.
	Now let $m \geq N$. Then we have
	\[
		\lVert x_n - x_m \rVert \leq 
		\lVert x_n - x \rVert + \lVert x_m - x \rVert <
		\sfrac\varepsilon2 + \sfrac\varepsilon2 =
		\varepsilon.
	\]
	Thus $(x_n)$ is Cauchy.
\end{solution}

%%%%%%%%%%%%%%%%%%%%%%%%%%%%%%%%%
\part Let $(x_n)_n$ be a Cauchy sequence in $X$.
      Show that if $(x_n)_n$ has a convergent subsequence in $X$,
	  then $(x_n)_n$ converges in $X$.
\begin{solution}
	Let $(x_n)$ be Cauchy and suppose $(x_{n_k})$ is a convergent
	subsequence with limit $x$.
	Let $\varepsilon > 0$, then choose $N \in \mathbb N$ such that
	for all $k \geq N$ we have 
	$\lVert x_{n_k} - x \rVert < \sfrac{\varepsilon}2$.
	Choose $N' \in \mathbb N$ such that for all $n,m \geq N'$ we have
	$\lVert x_n - x_m \rVert < \sfrac{\varepsilon}2$.
	We fix $j$ such that $j > \max\{N, N'\}$, 
	then for $n \geq \max\{N, N'\}$ we have
	\[
		\lVert x_n - x \rVert \leq
		\lVert x_n - x_{n_j} \rVert + \lVert x_{n_j} - x \rVert <
		\sfrac{\varepsilon}2 + \sfrac{\varepsilon}2 =
		\varepsilon.
	\]
\end{solution}

%%%%%%%%%%%%%%%%%%%%%%%%%%%%%%%%%
\part Two norms $\lVert \cdot \rVert_1$ and $\lVert \cdot \rVert_2$
	  are \emph{equivalent} if there exists a constant $C > 0$ such that
	  $$\frac1C \lVert x \rVert_1 \leq \lVert x \rVert_2
	    \leq C \lVert x \rVert_1, \qquad x \in X.$$
	  If $\lVert \cdot \rVert_1$ and $\lVert \cdot \rVert_2$ are equivalent
	  norms on $X$, then prove that $(X, \lVert \cdot \rVert_1)$ is complete
	  if and only if $(X, \lVert \cdot \rVert_2)$.
\begin{solution}
	Consider the normed vector spaces $(X, \lVert \cdot \rVert_1)$ and $(X, \lVert \cdot \rVert_2)$ where $\lVert \cdot \rVert_1$ and $\lVert \cdot \rVert_2$ are equivalent. 
	Choose $C > 0$ such that
	$$\frac1C \lVert x \rVert_1 \leq \lVert x \rVert_2
	    \leq C \lVert x \rVert_1, \qquad x \in X.$$
	First suppose that $(X, \lVert \cdot \rVert_1)$ is complete.
	Now let $(x_n)$ be a Cauchy sequence in $(X, \lVert \cdot \rVert_2)$
	such that for all $\varepsilon > 0$ there is $N \in \mathbb N$
	such that for every $n,m \geq N$ we have $\lVert x_n - x_m \rVert < \varepsilon$.
	We claim that $(x_n)$ is also Cauchy in $(X, \lVert \cdot \rVert_1)$.
	Indeed,
	\[
		\lVert x_n - x_m \rVert_1 \leq
		C \lVert x_n - x_m \rVert_2 <
		C \varepsilon.
	\]
	Now, as $(X, \lVert \cdot \rVert_1)$ is complete, $(x_n)$ must converge to some limit in $X$, say $x$. 
	We claim that $(x_n)$ must also converge to this same limit with $\lVert \cdot \rVert_2$.
	We see,
	\[
		\lVert x_n - x \rVert_2 \leq
		C \lVert x_n - x \rVert_1 <
		C \varepsilon
	\]
	and thus $(X, \lVert \cdot \rVert_2)$ is complete.
	Now, the relation on norms defined above is clearly symmetric; thus,
	we are done.
\end{solution}
	
\end{parts}

% /////////// %
% /         / %
\question
\begin{parts}

%%%%%%%%%%%%%%%%%%%%%%%%%%%%%%%%%
\part Let $(X, \lVert \cdot \rVert)$ be a normed linear space.
      Prove that
	  $$\lVert x - y \rVert \geq \lVert x \rVert - \lVert y \rVert,
		\qquad \forall\; x,y \in X.$$
\begin{solution}
	\[
		\lVert x \rVert = 
		\lVert x - y + y \rVert \leq 
		\lVert x - y \rVert + \lVert y \rVert
	\]
	thus $\lVert x - y \rVert \geq \lVert x \rVert - \lVert y \rVert$.
\end{solution}

%%%%%%%%%%%%%%%%%%%%%%%%%%%%%%%%%
\part Let $f,g \in L^1(E)$. 
  	  Prove that, if 
	  $\lVert f + g \rVert_{L^1} = \lVert f \rVert_{L^1}
		+ \lVert g \rVert_{L^1}$, then
		$$\lVert af + bg \rVert_{L^1} = a \lVert f \rVert_{L^1}
			+ b \lVert g \rVert_{L^1}$$
	  for all $a \geq 0$ and $b \geq 0$.
	  Is this a special property of the $\lVert \cdot \rVert_{L_1}$
	  norm, or does it hold for any norm $\lVert \cdot \rVert$? 
	  Justify your response.
\begin{solution}
	For brevity we will write $\lVert \cdot \rVert = \lVert \cdot \rVert_{L^1}$.
	Let $f, g \in L^1(E)$ and assume $\lVert f + g \rVert = \lVert f \rVert + \lVert g \rVert$.
	First we assume that $a \geq b$.
	\begin{align*}
		\lVert af + bg \rVert
		&=    \lVert a(f + g) - (a - b) g \rVert \\
		&\geq \lVert a(f+g) \rVert - \lVert (a-b)g \rVert \\
		&= \lvert a \rvert \cdot \lVert f + g \rVert - \lvert a - b \rvert \cdot \lVert g \rVert \\
		&= a \left( \lVert f \rVert + \lVert g \rVert \right) - (a-b) \lVert g \rVert \\
		&= a \lVert f \rVert + b \lVert g \rVert.
	\end{align*}
	Now we consider the case where $b > a$. 
	\begin{align*}
		\lVert af + bg \rVert
		&=    \lVert b(f + g) - (b - a) f \rVert \\
		&\geq \lVert b(f+g) \rVert - \lVert (b - a)f \rVert \\
		&= \lvert b \rvert \cdot \lVert f + g \rVert - \lvert b - a \rvert \cdot \lVert f \rVert \\
		&= b \left( \lVert f \rVert + \lVert g \rVert \right) - (b-a) \lVert f \rVert \\
		&= a \lVert f \rVert + b \lVert g \rVert.
	\end{align*}
	$\lVert \cdot \rVert$ satisfies the triangle inequality, thus we have
	\[
		a \lVert f \rVert + b \lVert g \rVert
		\leq \lVert af + bg \rVert
		\leq \lVert af \rVert + \lVert bg \rVert 
		= a \lVert f \rVert + b \lVert g \rVert. 
	\]
	thus $a \lVert f \rVert + b \lVert g \rVert = \lVert af + bg \rVert$.
\end{solution}

\end{parts}

% /////////// %
% /         / %
\question Prove the following.
\begin{parts}

%%%%%%%%%%%%%%%%%%%%%%%%%%%%%%%%%
\part If $f,g \in L^\infty(E)$, then $f+g \in L^\infty(E)$ and 
$\lVert f + g \rVert_{L^\infty} \leq \lVert f \rVert_{L^\infty} + \lVert g \rVert_{L^\infty}$.
\begin{solution}
	Let $f, g \in L^\infty(E)$ and $M_f, M_g \geq 0$ essentially bound $f$ and $g$ respectively.
	Let $E_f, E_g \subset E$ with $m(E_f) = m(E_g) = 0$ such that $f(x) < M_f$ for all $x \in E \setminus E_f$ and $g(x) < M_g$ for all $x \in E \setminus E_g$.
	We observe that $m(E_f \cup E_g) = 0$ and for all $E \setminus (E_f \cup E_g)$ we have $(f + g)(x) = f(x) + g(x) \leq M_f + M_g$, thus $f + g$ is essentially bounded (by $M_f + M_g$) and thus $f + g \in L^\infty(E)$.

	Now, $\lvert f \rvert \leq \lVert f \rVert_{L^\infty}$ almost everywhere on $E$ as $\lVert f \rVert_{L^\infty}$ is an essential bound for $\lvert f \rvert$.
	Thus we have
	\[
		\lvert f + g \rvert 
		\leq \lvert f \rvert + \lvert g \rvert
		\leq \lVert f \rVert_{L^\infty} + \lVert g \rVert_{L^\infty}
	\]
	almost everywhere on $E$.
	Thus $\lVert f \rVert_{L^\infty} + \lVert g \rVert_{L^\infty}$ is an essential bound for $\lvert f + g \rvert$.
	But $\lVert f + g \rVert_{L^\infty}$ is the least essential bound for $\lvert f + g \rvert$, thus
	\[
		\lVert f + g \rVert_{L^\infty} 
		\leq \lVert f \rVert_{L^\infty} + \lVert g \rVert_{L^\infty}.
	\]
\end{solution}

%%%%%%%%%%%%%%%%%%%%%%%%%%%%%%%%%
\part If $f \in L^1(E)$ and $g \in L^\infty(E)$, then 
$\int_E \lvert fg \rvert \leq \lVert f \rVert_{L^1} \lvert g \rvert_{L^\infty}$.
\begin{solution}
	Observe $\lvert fg \rvert = \lvert f \rvert \cdot \lvert g \rvert \leq \lvert f \rvert \cdot \lVert g \rVert_{L^\infty}$.
	By the monotonicity of the Lebesgue integral, we have
	\begin{align*}
		\int_E \lvert fg \rvert
		&\leq \int_E \lvert f \rvert \cdot \lVert g \rVert_{L^\infty} \\
		&=    \left( \int_E \lvert f \rvert \right) \cdot \lVert g \rVert_{L^\infty} \\
		&=     \lVert f \rVert_{L^1} \cdot \lVert g \rVert_{L^\infty}.
	\end{align*}

\end{solution}

\end{parts}
\end{questions}
\end{document}
