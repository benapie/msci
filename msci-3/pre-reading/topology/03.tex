%! TEX root = ../master.tex

\subsection{Convergence and continuity}

The prime reason for the invention of topological spaces was to provide a
setting for studying convergence and continuity in a more general sense 
than in the context of metric spaces.

\begin{definition}[Convergence]
	Let $X$ be a topological space, $ 
		\{ x_i \}_{i=1}^{\infty} \subset X 
	$ be a sequence of points, and $x \in X$.
	We say that $x_i$ \textbf{converges} to $x$ if for every neighbourhood $U$
	of $x$ there exists $N \in \N$ such that for all $i \geq N$: $x_i \in U$.
	We denote this as $x_i \to x$ or $\lim_{i \to \infty} x_i = x$.
\end{definition}

\begin{problem}
	Show that in a metric space, the totopological definition of convergence is
	equivalent to the metric space definition.
\end{problem}

\begin{solution}
	Let $(X,d)$ be a metric space and let $(X,\mathcal T)$ be a topological space
	such that $d$ generates $\mathcal T$.
	Let $
		\left\{ x_i \right\}_{i=1}^\infty \subset X
	$ be a sequence of points and $x \in X$.
	First we assume $x_i \to x$ in the topological sense.
	Let $\varepsilon > 0$.
	$B_\varepsilon(x)$ is a neighbourhood of $x$;
	therefore, there exists $N \in \N$ such that for all $i \geq N$
	$x_i \in B_\varepsilon(x)$.
	That is,
	$d(x_i, x) < \varepsilon$.
	Hence topological convergence $\implies$ metric convergnce.
	Now we assume $x_i \to x$ in the metric sense.
	Let $U$ be a neighbourhood of $x$.
	As $U$ is open, it is the union of open balls.
	It is clearly true that topological convergence
	implies metric convergence.
\end{solution}

\begin{problem}
	Let $X$ be a discrete toplogical space.
	Show that the only convergent sequences are ones which are eventually
	constant.
\end{problem}

\begin{proof}
	Let $
		\left\{ x_i \right\}_{i=1}^\infty
	$ be a convergence sequence with limit $x$.
	Let $\varepsilon \in (0,1)$.
	As $x_i \to x$, there exists $N \in \N$ such that for all $i \geq N$
	we have $d(x,x_i) < \varepsilon$. So $d(x,x_i)=0 \implies x = x_i$ and hence
	the sequence is eventually constant.
\end{proof}

\begin{problem}
	Suppose $X$ is a topological space, $A \subset X$, and $\{x_i\}_{i=1}^\infty$
	is a sequence of points in $A$ that converges to $x \in X$.
	Show that $x \in \overline A$.
\end{problem}

\begin{solution}
	If $x \in A$ then clearly $x \in \overline A$. Now assume $x \not\in A$.
	Consider $x \in \ext A$. Let $U \subset \ext A$ be a neighbourhood of
	$x$. As $x_i \to x$, there exists some $N \in \N$ such that for all 
	$i \geq N$ we have $x_i \in U$. Except $x_i \in A$ for all $i \in \N$.
	Therefore $x \not\in \ext A$ and so $x \in \overline A$.
\end{solution}

\begin{definition}[Continuity]
	Let $X$ and $Y$ be topological spaces.
	A map $f: X \to Y$ is said to be \textbf{continuous} if for all $U$ open in 
	$Y$, $f^{-1}(U)$ is open in $X$.
\end{definition}

\begin{remark}
	A map is continous if its inverse preserves \emph{openness}.
\end{remark}

\begin{proposition}[]
	A map between topological spaces is continuous if and only if the preimage
	of every closed subset is closed.
\end{proposition}

\begin{proof}
	Let $X$ and $Y$ be topological spaces.
	\begin{description}
		\item[$\implies$] 
			Let $C \subset Y$ be closed.
			Then $O = Y \setminus C$ is open.
			So \[
				f^{-1}(C) 
				= f^{-1}(Y \setminus O) 
				= f^{-1}(Y) \setminus f^{-1}(O)
				= X \setminus f^{-1}(O)
			\]
			which is closed as $f^{-1}(O)$ is open due to the continuity of $f$.

		\item[$\impliedby$]
			Let $O \subset Y$ be open.
			Then $C = Y \setminus O$ is closed.
			So \[
				f^{-1}(O)
				= f^{-1}(Y \setminus C)
				= f^{-1}(Y) \setminus f^{-1}(C)
				= X \setminus f^{-1}(C)
			\]
			which is open as $f^{-1}(C)$ is closed.
	\end{description}
\end{proof}

\begin{proposition}[]
	\label{prop:function-continuous}
	Let $X$, $Y$, and $Z$ be topological spaces.
	\begin{enumerate}
		\item Every constant map is continuous.
		\item The identity map is continuous.
		\item If a map $f: X \to Y$ is continuous
			then for all open sets $O$ we have $\restr{f}{O}: O \to Y$ is
			continuous.
		\item If maps $f: X \to Y$ and $g: Y \to Z$ are continuous then
			their composition $g \circ f: X \to Z$ is continuous.
	\end{enumerate}
\end{proposition}

\begin{proof}
	\begin{enumerate}
		\item 
			Let $f: X \to Y$, $f(x) = a$.
			Let $U$ be open in $Y$.
			Then $f^{-1}(U) \in \{\varnothing, X\}$.
			Both possibilities are open in $X$.

		\item 
			Let $f: X \to X$, $f(x) = x$.
			Let $U$ be open in $X$.
			Then $f^{-1}(U) = U$ which is open.

		\item 
			Let $U$ be open in $X$ and $V$ be open in $Y$.
			Then \[
				\left(\restr{f}{U}\right)^{-1}(V) = f^{-1}(V) \cap U
			\]
			which is the intersection of two open sets, thus is open.
		
		\item
			Let $U \subset Z$ be open. $
				(g \circ f)^{-1}(U) = f^{-1}(g^{-1}(U))
			$ which is open as $f$ and $g$ are continuous.
	\end{enumerate}
\end{proof}

\begin{proposition}[]
	A map $f: X \to Y$ between two topological spaces is continuous if and only 
	if each point of $X$ has a neighbourhood on which the restriction of $f$ is
	continuous.
\end{proposition}

\begin{proof}
	\begin{description}
		\item[$\implies$] Immediate consequence.

		\item[$\impliedby$]
			Let $x \in X$, $U_x$ be the neighbourhood on which $\restr{f}{U_x}$ 
			is continuous, and $V$ be open in $Y$.
			Then
			\[
				\left( \restr f{U_x} \right)^{-1}(V)
				= \{ x \in U_x: f(x) \in V \}
				= f^{-1}(V) \cap U_x.
			\]
			and so $\left( \restr f{U_x} \right)^{-1}(V)$ is a neighbourhood of
			$x$ contained within $f^{-1}(V)$.
			Therefore $f^{-1}(V)$ is open.
	\end{description}
\end{proof}

\begin{definition}[Homeomorphism]
	Let $X$ and $Y$ be topological spaces.
	A \textbf{homeomorphism} from $X$ to $Y$ is a bijective map
	$\varphi: X \to Y$ such that both $\varphi$ and $\varphi^{-1}$
	are continuous.
	If such a bijection exists, we say $X$ and $Y$ are \textbf{homeomorphic} or
	\textbf{topologically equivalent} ($X \approx Y$).
\end{definition}

\begin{problem}
	Show that \emph{homeomorphic} is an equivalence relation on the class of all
	topological spaces.
\end{problem}

\begin{solution}
	Let $X$, $Y$, and $Z$ be topological spaces.
	By defining $\varphi: X \to X$, $\varphi(x) = x$ we see that all topological
	spaces are homeomorphic to theirself.
	Now assume that $X \approx Y$ with bijection $\varphi: X \to Y$.
	Now we let $\psi: Y \to X$, $\psi(y) = \varphi^{-1}(y)$.
	So $X \approx Y \implies Y \approx X$.
	Now assume that $X \approx Y$ with bijection $\varphi: X \to Y$
	and $Y \approx Z$ with bijection $\psi: Y \to Z$.
	Now we define $\theta: X \to Z$ as $\theta(x) = \psi(\varphi(x))$.
	Clearly $\theta$ is a bijection and itself and its inverse are continuous.
	Hence $X \approx Y, Y \approx Z \implies X \approx Z$.
\end{solution}

\begin{problem}
	\label{prob:homeo-preserves-openness}
	Let $(X, \mathcal T)$ and $(Y, \mathcal J)$ be topological spaces
	and let $f: X \to Y$ be a bijective map.
	Show that $f$ is a homeomorphism if and only if
	$f(\mathcal T) = \mathcal J$ (in the sense that $U$ is open in $X$ if
	and only if $f(U)$ is open in $Y$).
\end{problem}

\begin{solution}
	\hfill
	\begin{description}
		\item[$\implies$] 
			Let $U$ be open in $X$.
			Then $
				\left( f^{-1} \right)^{-1}(U) = f(U)
			$ is open in $Y$ as $f^{-1}$ is continuous.
			Now let $V \subset X$ such that $f(V)$ is open.
			Then $
			f^{-1}(f(V)) = V
			$ is open in $X$ as $f$ is continuous.

		\item[$\impliedby$]
			Let $U$ be open in $X$. Then $f(U)$ is open in $Y$.
			As $
				f(U) = \left( f^{-1} \right)^{-1}(U)
			$ and $U$ is open then $f^{-1}$ must be continuous.
			Simiarly $
				U = f^{-1}(f(U))
			$ and $f(U)$ is open so $f$ must be continuous.
	\end{description}
\end{solution}

\begin{problem}
	Suppose $f: X \to Y$ is a homeomorphism and $U$ is open in $X$.
	Show that $f(U)$ is open in $Y$ 
	and the restriction $\restr fU$ is a homeomorphism from $U$ to $f(U)$.
\end{problem}

\begin{solution}
	From Problem \ref{prob:homeo-preserves-openness} we know that $f(U)$ is open.
	As $f: X \to Y$ is a bijection so is $\restr fU: U \to f(U)$.
	We know from $(iii)$ of Proposition \ref{prop:function-continuous} that
	continuous function restricted to open sets are continuous.
	Therefore $\restr fU$ is continuous and so is $\left(\restr fU\right)^{-1}$.
\end{solution}

\begin{definition}[Fineness and coarseness]
	Let $\mathcal T$ and $\mathcal J$ be topologies on a set $X$. 
	\begin{enumerate}
		\item $\mathcal T$ is \textbf{finer} than $\mathcal J$ 
			if $\mathcal J \subset \mathcal T$.

		\item $\mathcal T$ is said to be \textbf{coarser} than $\mathcal J$
			if $\mathcal T \subset \mathcal J$.
	\end{enumerate}
\end{definition}

\begin{problem}
	Let $\mathcal T$ and $\mathcal J$ be topologies on some set $X$.
	Show that identity map of $X$ is 
	\begin{enumerate}
		\item continuous as a map from $(X, \mathcal T)$ to $(X, \mathcal J)$ if 
			and only if $\mathcal T$ is finer than $\mathcal J$; and
		\item a homeomorphism if and only if $\mathcal T = \mathcal J$.
	\end{enumerate}
\end{problem}

\begin{solution}
	Let $f$ be the identity map from $(X, \mathcal T)$ to $(X, \mathcal J)$.
	\begin{enumerate}
		\item
			Let $U \in \mathcal J$. 
			Then $f^{-1}(U) = U \in \mathcal T$ if and only if 
			$\mathcal J \subset \mathcal T$.
			
		\item 
			As $f$ is continuous, $U \in \mathcal J$ implies
			$f^{-1}(U) = U \in \mathcal T$.	
			As $f^{-1}$ is continuous, $U \in \mathcal T$ implies
			$\left(f^{-1}\right)^{-1}(U) = f(U) = U \in \mathcal J$.
			Hence $f$ being a homeomorphism implies that 
			$\mathcal T = \mathcal J$.
			Now if we assume $\mathcal T = \mathcal J$, we see from Problem
			\ref{prop:function-continuous} that $f$ must be continuous.
			Hence $f$ is a homeomorphism since $f$ is clearly a bijection and
			$f = f^{-1}$.
	\end{enumerate}
\end{solution}

\begin{example}[]
	\hfill
	\begin{enumerate}
		\item 
			Let $\bm x, \bm y \in \R^n$ and $\varepsilon, \delta > 0$. 
			Then \[
				B_{\varepsilon}(\bm x) \approx B_{\delta}(\bm y).
			\]
			The homeomorphism can easily be constructed via translations and
			dilations.
			This shows that \emph{size} is not a topological property
			(that is, a property that is unchanged by a homeomorphism).

		\item 
			Let $\B^n \subset \R^n$ be the unit ball and define a map
			$F: \B^n \to \R^n$ by
			\[
				F(\bm x) = \frac{\bm x}{1 - \abs{\bm x}}.
			\]
			It can be shown that the map
			\[
				G(\bm x) = \frac{\bm x}{1 + \abs{\bm x}} 
			\]
			is an inverse for $F$.
			$F$ is clearly bijective and as since $F$ and $F^{-1}$
			are both continuous, $F$ is a homeomorphism.
			Thus \emph{boundedness} is not a topological property.

		\item 
			Let $S^2$ be the unit sphere and $C$ be the cube of side length 2
			centered at the origin.
			The map $\varphi: C \to S^2$,
			\[
				\varphi(x,y,z) = \frac{(x,y,z)}{\sqrt{x^2+y^2+z^2}} 
			\]
			is a homeomorphism, with inverse
			\[
				\varphi^{-1}(x,y,z) = 
				\frac{(x,y,z)}{\max\{\abs x, \abs y, \abs z\}}.
			\]
			Thus \emph{having corners} is not a topological property.
	\end{enumerate}
\end{example}

\begin{problem}
	\label{prob:local-homeo-ex}
	Let $X$ be the half open interval $[0,1) \subset \R$ and 
	let $S^1$ be the unit circle in $\C$ 
	(both with Euclidean metric topologies).
	Define a map $a: X \to S^1$ by
	\[
		a(s) = e^{2\pi is} = \cos(2\pi s) + i \sin(2\pi s).
	\]
	Show that $a$ is continuous and bijective but it is not a homeomorphism.
\end{problem}

\begin{solution}
	$a$ is clearly continuous on $X$ since $\sin$ and $\cos$ are both continuous
	on $\R$.
	Now for $a^{-1}$, we will use the branch of $\log$ with a jump discontinuity
	at $R_{\frac{\pi}2}$.
	Then
	\[
		a^{-1}(x + iy) 
		= \frac1{2\pi i} \log(x+iy)
		= \frac1{2\pi}\arg(x+iy)
	\]
	but $\arg$ is not continuous on $S^1$;
	it has a jump discontinuity. 
	Hence $a^{-1}$ is not continuous and
	$a$ is not a homeomorphism.
	As we have an expression for $a^{-1}$, $a$ is bijective.
\end{solution}

\begin{definition}[Open and closed maps]
	Let $X$ and $Y$ be topological spaces, $U$ be open in $X$,
	and $V$ be closed in $X$.
	A map $f: X \to Y$ is an \textbf{open map} if $f(U)$ is open in $Y$.
	$f$ is a \textbf{closed map} if $f(V)$ is closed in $Y$.
\end{definition}

\begin{problem}
	Let $X$ and $Y$ be topological spaces.
	Let $f: X \to Y$ be a bijective continuous map.
	Show that the following statements are equivalent.
	\begin{enumerate}
		\item $f$ is a homomorphism.
		\item $f$ is open.
		\item $f$ is closed.
	\end{enumerate}
\end{problem}

\begin{proof}
	Let $U$ be open in $X$.
	\begin{description}
		\item[$(i) \implies (ii)$] 
			$f^{-1}$ is continuous, 
			so $\left(f^{-1}\right)^{-1}(U) = f(U)$ is open.
			Hence $f$ is open.

		\item[$(ii) \implies (iii)$]
			$X \setminus U$ is closed.
			$
				f(X \setminus U) 
				= f(X) \setminus f(U)
				= Y \setminus f(U)
			$ which is closed as $f(U)$ is open.

		\item[$(iii) \implies (i)$]
			$X \setminus U$ is closed in $X$.
			$
				f(X \setminus U)
				= f(X) \setminus f(U)
				= Y \setminus f(U)
			$ is closed;
			therefore $Y \setminus (Y \setminus f(U)) = f(U)$ is open.
			Then $\left(f^{-1}\right)^{-1}(U) = f(U)$ is open, and so
			$f^{-1}$ is continuous. 
	\end{description}
\end{proof}

\begin{proposition}[]
	Suppose $X$ and $Y$ are topological spaces
	and $f: X \to Y$ is any map.
	\begin{enumerate}
		\item
			$f$ is continuous if and only if
			$f(\overline A) \subset \overline{f(A)}$
			for all $A \subset X$.
		
		\item 
			$f$ is closed if and only if
			$f(\overline A) \supset \overline{f(A)}$
			for all $A \subset X$.

		\item
			$f$ is continuous if and only if
			$f^{-1}(\inte B) \subset \inte f^{-1}(B)$
			for all $B \subset Y$.
 
		\item
			$f$ is open if and only if
			$f^{-1}(\inte B) \supset \inte f^{-1}(B)$
			for all $B \subset Y$.
	\end{enumerate}
\end{proposition}

\begin{proof}
	\begin{enumerate}
		\item 
			\begin{description}
				\item[$\implies$]
					We have $
						A \subset f^{-1} \left( \overline{f(A)} \right)
					$. Then $
					\overline A \subset f^{-1} \left( \overline{f(A)} \right)
					$ as $\overline{f(A)}$ is closed. So $
						f(\overline A) \subset \overline{f(A)}
					$ as required.

				\item[$\impliedby$]
					Let $C$ be closed in $Y$.
					Then $
						f \left( \overline{f^{-1}(C)} \right)
						\subset \overline{f\left(f^{-1}(C)\right)}
						= \overline C
						= C
					$ hence $
						\overline{f^{-1}(C)} \subset f^{-1}(C)
					$ and so $f^{-1}(C)$ is closed.

			\end{description}
		\item
			\begin{description}
				\item[$\implies$]
					We have $
						A \subset f^{-1}(f(\overline A))
					$ and so $
						f(A) \subset f(\overline A)
					$ and as $f(\overline A)$ is closed $
						\overline{f(A)} \subset f(\overline A)
					$.

				\item[$\impliedby$]
					Let $C$ be closed in $X$.
					Then $
						f(C) = f(\overline C) \supset \overline{f(C)}
					$. Hence $f(C)$ is closed, and so $f$ is closed.
			\end{description}
	\end{enumerate}
	$(iii)$ follows a similar reasoning to $(i)$ and $(iv)$ follows a similar
	reasoning to $(ii)$ so it will be omitted.
\end{proof}

\begin{definition}[Local homeomorphism]
	Suppose $X$ and $Y$ are topological spaces
	and $f: X \to Y$ is a map.
	We say that $f$ is a \textbf{local homeomorphism} if
	for every point $x \in X$ there exists a neighbourhood
	$U \subset X$ such that $f(U)$ is open in $Y$
	and $\restr fU: U \to Y$ is a homeomorphism.
\end{definition}

\begin{example}[]
	Recall the map defined in Problem \ref{prob:local-homeo-ex}.
	This is an example of a map that is a local homeomorphism but not a
	homeomorphism.
\end{example}

\begin{proposition}[]
	\hfill
	\begin{enumerate}
		\item Every homeomorphism is a local homeomorphism.
		\item Every local homeomorphism is continuous and open.
		\item Every bijective local homeomorphism is a homeomorphism.
	\end{enumerate}
\end{proposition}

\begin{proof}
	%todo
\end{proof}

