%! TEX root = ../master.tex

\subsection{Bases and countability}

\begin{definition}[Basis]
	Let $X$ be a topological space.
	A collection $\mathcal B$ of subsets of $X$ is called a \textbf{basis} 
	for the topology of $X$ if
	\begin{enumerate}
		\item for all $U \in \mathcal B$, $U$ is open in $X$; and
		\item for all $U$ open in $X$ there exists 
			$\mathcal B' \subset \mathcal B$ such that
			$U = \bigcup_{V \in \mathcal B'} V$.
	\end{enumerate}
	We use the shorthand \emph{$\mathcal B$ is a basis for $X$} if the topology
	is understood.
\end{definition}

\begin{problem}
	Suppose $X$ is a topological space, 
	$\mathcal B$ is a basis for $X$, 
	and $U \subset X$.
	Show that $U$ is open in $X$ if and only if
	for all $p \in U$ there exists $B \in \mathcal B$
	such that $p \in B \subset U$.
\end{problem}

\begin{solution}
	\hfill
	\begin{description}
		\item[$\implies$] 
			Immediate from the definition of a basis.

		\item[$\impliedby$]
			Let $p$ be a point in $U$ and $B_p \in \mathcal B$ be such that
			$p \in B_p \subset U$.
			Now $U = \bigcup_{p \in U} B_p$ and as $B_p$ is open,
			$U$ is open.
	\end{description}
\end{solution}

\begin{remark}
	If a subset $U$ of a topological space $X$ with basis $\mathcal B $ satisfies
	\[
		\;\forall\; p \in U \;\exists\; B \in \mathcal B: p \in B \subset U
	\]
	we say that it satisfies the \textbf{basis criterion} with respect to
	$\mathcal B$.
\end{remark}

\begin{example}[Examples of bases]
	\hfill
	\begin{enumerate}
		\item 
			Let $M$ be a metric space.
			The collection of open balls in $M$ forms a basis for
			the metric topology.

		\item 
			Let $X$ be a set.
			The collection of singleton sets form a basis
			for the discrete toplogy.

		\item 
			Let $X$ be a set with the trivial topology.
			Then $\{X\}$ is a basis (and the only basis) for the topology.
	\end{enumerate}
\end{example}

\begin{problem}
	Show that the following collections are a basis for the Euclidean topology
	on $\R^n$.
	\begin{enumerate}
		\item 
			\[
				\mathcal B_1 = \left\{
					C_S(\bm x) : \bm x \in \R^n, \, S > 0 
				\right\}
			\]
			where $C_S(\bm x)$ is the open cube of side length $S$ centered at
			$\bm x$.

		\item 
			\[
				\mathcal B_2 = \left\{ 
					B_r(\bm x): r \in \Q, \, \bm x \in \Q^n 
				\right\}.
			\]
	\end{enumerate}
\end{problem}

