%! TEX root = ../master.tex

\subsection{Topologies}

\begin{definition}[Topology]
	Let $X$ be a set. 
	A \textbf{topology} on $X$ is a collection $\mathcal T$ of subsets of $X$
	satisfying the following properties:
	\begin{enumerate}
		\item $X, \varnothing \in \mathcal T$;
		\item $\mathcal T$ is closed under arbitrary union; and
		\item $\mathcal T$ is closed under finite intersection.
	\end{enumerate}
	A pair $(X, \mathcal T)$ is called a \textbf{topological space}. 
\end{definition}

\begin{remark}
	We may use the shorthand \emph{$X$ is a topological space}
	or \emph{$X$ is a space}.
	Elements of $X$ are typically called \textbf{points}, 
	and we call the sets that make up $\mathcal T$ the 
	\textbf{open subsets} of $X$. A
\end{remark}

\begin{definition}[Neighbourhood]
	If $X$ is a topological space and $p \in X$, a \textbf{neighbourhood} of $p$
	is an open subset of $X$ containing $p$.
\end{definition}

\begin{remark}
	In some contexts, a neighbourhood of $p$ is defined as a subset that
	\emph{contains } an open set containing $p$ but for what we are looking
	at the definition above is adequate.
\end{remark}

\begin{example}[Metric topology]
	Let $(M, d)$ be any metric space
	and let $\mathcal T$ be the collection of all open subsets of $M$
	(in the metric space sense).
	$\mathcal T$ is a topology on $M$, called the \textbf{metric topology},
	or the \textbf{topology generated by $d$}.
\end{example}

\begin{example}[Discrete topology]
	Let $X$ be a set and $\mathcal T = \mathcal P(X)$.
	$\mathcal T$ is called the \textbf{discrete topology} on $X$
	and $(X, \mathcal T$ is called the \textbf{discrete space}.
\end{example}

\begin{example}[Trivial topology]
	Let $Y$ be a set and let $\mathcal T = \{ Y, \emptyset \}$.
	This is called the \textbf{trivial topology} on $Y$.
\end{example}

\begin{definition}[Metrizable]
	We say that a topological space $(X, \mathcal T)$ is \textbf{metrizable} if
	there exists some metric $d$ on $X$ such that $\mathcal T$ is the topology
	generated by $d$.
\end{definition}

\begin{remark}
	If $X$ is a metrizable topological space, the metric that generated it is
	not uniquely determined; different metrics may generate the same topology.
\end{remark}

\begin{problem}
	\label{prob:metrics-generate-topologies}
	Suppose $M$ is a set and $d$ and $d'$ are two distinct metrics on $M$.
	Prove that $d$ and $d'$ generate the same topology on $M$ if and only if 
	for all $x \in M, r > 0$ there exists $r_1, r_2 \in \R$ such that
	\[
		B_{r_1}^{d'}(x) \subset B_r^d(x)
		\quad\text{and}\quad
		B_{r_2}^d(x) \subset B_r^{d'}(x).
	\]
\end{problem}

\begin{solution}
	\hfill
	\begin{description}
		\item[$\implies$] 
			Assume that $(M,d)$ and $(M,d')$ have the same metric topologies. 
			Let $x \in M, r > 0$. $B_r^d(x)$ is open in $(M,d)$; hence,
			it is also open in $(M,d')$. 
			So for all $y \in B_r^d(x)$ 
			there exists $r_1 \in \R$
			such that $B_{r_1}^{d'}(y) \subset B_r^d(x)$.
			Hence $B_{r_1}^{d'}(x) \subset B_r^d(x)$.
			Similarly we can show that there exists $r_2 \in \R$
			such that $B_{r_2}^d(x) \subset B_r^{d'}(x)$.

		\item[$\impliedby$] 
			Let $U \subset M$ be open in $(M,d)$.
			As $U$ is open, it is a union of open balls.
			Let $U = \bigcup_{x \in U} B_{\delta_x} (x)$.
			Now for all $x \in U$ there exists $r > 0$A such that
			$B_r^{d'}(x) \subset B_{\delta_x}^d(x)$.
			Hence for all $x \in U$, $B_{\delta_x}^d(x)$ is open in $(M,d')$.
			Hence $U$ is open in $(M,d')$.
			Similar reasoning can be used to show that for all $V$ open in
			$(M,d')$, $V$ is open in $(M,d)$.
			As $U$ is open in $(M,d)$ if and only if $U$ is open in $(M,d')$,
			we have shown that $d$ and $d'$ both generate the same topology.
	\end{description}
\end{solution}

\begin{problem}
	Let $(M,d)$ be a metric space, let $c$ be a positive real number, and
	define a new metric $d'$ on $M$ by $d'(x,y) = c\cdot d(x,y)$.
	Prove that $d$ and $d'$ generates the same topology on $M$.
\end{problem}

\begin{solution}
	Let $\mathcal T$ and $\mathcal T'$ be the topologies generated by
	$d$ and $d'$ respectively.
	Let $U \subset M$ be open in $(M,d)$. 
	Then for all $x \in U$ there exists $\epsilon > 0$ such that
	$B_\varepsilon^d(x) \subset U$.
	Now choose $\varepsilon_2 = c^{-1}\varepsilon_1$ 
	and observe that for all $x \in N$ we have
	$B_{\varepsilon_2}^{d'}(x) \subset U$
	since $B_{\varepsilon_2}^{d'}(x) = B_{\varepsilon_1}^d(x)$.
	Hence $U$ is open in $(M,d')$.
	We can use a similar reasoning to show that all open sets in $(M,d')$ are
	also open in $(M,d)$, and hence $d$ and $d'$ generate the same topologies.
\end{solution}

\begin{problem}
	Define a metric $d'$ on $\R^n$ by
	\[
		d'(\bm x, \bm y) = \max\{\abs{x_1 - y_1}, \ldots, \abs{x_n - y_n}\}.
	\]
	Show that the Euclidean metric and $d'$ generate the same topology on 
	$\R^n$.
\end{problem}

\begin{solution}
	Here we use a similar technique as the last problem, but we pick
	$\varepsilon' = \sqrt n \varepsilon$.
\end{solution}

\begin{problem}
	Let $X$ be any set and $d$ be the discrete metric on $X$.
	Show that $d$ generates the discrete toplogy.
\end{problem}

\begin{solution}
	Taking $\varepsilon \in (0,1)$, we see that all single point sets are open.
	As any union of open sets must also be open,
	all subsets are open.
\end{solution}

\begin{problem}
	\label{prob:me}
	Show that the Euclidean metric and the discrete metric generate the same 
	topology on $\Z$. 
\end{problem}

\begin{solution}
	Let $d$ be the discrete metric and $d'$ be the Euclidean metric.
	Let $x \in \Z$.
	First we consider $r \in (0,1)$. Pick $r_1, r_2 \in (0,1)$. Then
	\begin{align*}
		B_{r_1}^{d'}(x) &= \{x\} = B_r^d(x) \\
		B_{r_2}^d(x)    &= \{x\} = B_r^{d'}(x).
	\end{align*}
	Now consider $r \in [1, \infty)$ and pick $r_1$ and $r_2$ as before.
	Then
	\begin{align*}
		B_{r_1}^{d'}(x) &= \{x\} \subset \Z = B_r^d(x) \\
		B_{r_2}^{d}(x)  &= \{x\} \subset B_r^{d'}(x).
	\end{align*}
	By Problem \ref{prob:metrics-generate-topologies}, $d$ and $d'$
	generate the same topology on $\Z$. 
\end{solution}

\begin{problem}
	Suppose $X$ is a topological space and $Y$ is an open subset of $X$.
	Show that the collection of open subsets of $X$ that is contained in $Y$
	is a topology on $Y$.
\end{problem}

\begin{solution}
	Let $\mathcal T$ be the topology of $X$. The topology of $Y$
	is defined as
	\[
		\mathcal J = \left\{ U \in \mathcal T: U \subset Y \right\}.
	\]
	As $Y$ is open in $X$, $Y \in \mathcal T$ and so $Y \in \mathcal J$.
	We also have $\emptyset \in \mathcal J$.
	Now as $\mathcal T$ is closed under arbitrary union, so is $\mathcal J$.
	Similarly as $\mathcal T$ is closed under finite intersection, 
	so is $\mathcal J$.
	Therefore $\mathcal J$ defines a topology on $Y$.
\end{solution}

\begin{problem}
	Let $X$ be a set and suppose $ \{ T_\alpha \}_{\alpha \in A} $ is a
	collection of topologies on $X$.
	Show that the intersection $
		\mathcal T = \bigcap_{\alpha\in A} \mathcal T_\alpha
	$
	is a topology on $X$.
\end{problem}

\begin{solution}
	Clearly $\emptyset$ and $X$ are contained within $\mathcal T$.
	$U \in \mathcal T$ if and only if $U$ is open in every space.
	Now as each topology is closed under arbitrary union and finite
	intersection, $\mathcal T$ must also meet these requirements.
	Hence $\mathcal T$ is a topology on $X$.
\end{solution}
