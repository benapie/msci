%! TEX root = ../master.tex

\subsection{Closed subsets}

\begin{definition}[Closed subset]
	If $X$ is a topological space, $C \subset X$ is said to be \textbf{closed} if
	$X \setminus C$ is open.
\end{definition}

\begin{remark}
	From our definition of a topological space we see:
	$X$ and $\empty$ are closed subsets and
	the set of closed subsets has closure under finite union 
	and arbitrary intersection.
\end{remark}

\begin{example}[]
	The following are examples of closed subsets.
	\begin{enumerate}
		\item For any $a,b \in \R$, $[a,b]$ is a closed subset of $\R$.
			As is $[a, \infty]$ and $(-\infty, a]$.

		\item Every closed ball in a metric space is closed.

		\item Every subset of a discrete space is closed.
	\end{enumerate}
\end{example}

\begin{definition}
	\label{def:clos-int-ext-boun}
	Let $X$ be a topological space and $A \subset X$. 
	We define several subsets of $X$ as follows:
	\begin{enumerate}
	\item (\textbf{closure} of $A$)
			\[
				\overline A = \bigcap \left\{ 
					B \subset X : B \supset A, \;\text{$B$ is closed in $X$}
				\right\};
			\]

		\item (\textbf{interior} of $A$)
			\[
				\inte{A} = \bigcup \left\{ 
					C \subset X: C \subset A, \;\text{$C$ is open in $X$}
				\right\};
			\]

		\item (\textbf{exterior} of $A$)
			\[
				\ext{A} = X \setminus \overline A; \;\text{and}
			\]

		\item (\textbf{boundary} of $A$)
			\[
				\partial A = X \setminus \left( \inte{A} \cup \ext{A} \right).
			\]
	\end{enumerate}
\end{definition}

\begin{proposition}[]
	\label{prop:clos-int-ext-boun}
	Let $X$ be a topological space and let $A \subset X$ be any subset.
	\begin{enumerate}
		\item $x \in \inte A$ if and only if it has some neighbourhood contained 
			in $A$.

		\item $x \in \ext A$ if and only if it has some neighbourhood contained
			in $X \setminus A$.

		\item $x \in \partial A$ if and only if every neighbourhood of it
			contains a point in $A$ and $X \setminus A$.

		\item $x \in \overline A$ if and only if every neighbourhood of it
			contains a point in $A$.

		\item $\overline A = A \cup \partial A = \inte A \cup \partial A$.

		\item $\inte A$ and $\ext A$ are open in $X$,
			$\overline A$ and $\partial A$ are closed in $X$.
	\end{enumerate}
\end{proposition}

\begin{proof}
	Proving these statements is fairly trivial 
	and are often taken as definitions.
\end{proof}

\begin{proposition}[]
	If $X$ is a topological space and $A \subset X$ then the following are
	equivalent.
	\begin{enumerate}
		\item $A$ is open in $X$.
		\item $A = \inte A$.
		\item $A \cap \partial A = \varnothing$.
		\item Every point of $A$ has a neighbourhood contained in $A$.
	\end{enumerate}
\end{proposition}

\begin{proof}
	From the definition of an open set and a neighbourhood, clearly
	$(i) \iff (iv)$.
	By $(i)$ of Proposition \ref{prop:clos-int-ext-boun} clearly 
	$(iv) \iff (ii)$.
	Now assume $A \cap \partial A \neq \varnothing$. 
	Then clearly this would contradict $(iv)$ as by 
	$(iii)$ of Proposition \ref{prop:clos-int-ext-boun} 
	all neighbourhoods of a point in $\partial A$ must contain a point in $A$
	and $X \setminus A$. Therefore $(iv) \implies (iii)$.
	Similarly by assuming $(iv)$ is not true, then this contradicts $(iii)$.
	Therefore $(iv) \iff (iii)$.
\end{proof}

\begin{proposition}[]
	If $X$ is a topological space and $A \subset X$ then the following 
	are equivalent.
	\begin{enumerate}
		\item $A$ is closed in $X$.
		\item $A = \overline{A}$. 
		\item $A \cap \partial A = \partial A$.
		\item Every point in $X \setminus A$ has a neighbourhood contained
			inside $X \setminus A$.
	\end{enumerate}
\end{proposition}

\begin{proof}
	The proof for this proposition follows a similar reasoning to the proof for
	the previous proposition.
\end{proof}

\begin{definition}[]
	Given a topological space $X$ and a set $A \subset X$, we say that a point
	$p \in X$ is a \textbf{limit point} of $A$ if every neighbourhood of $p$
	contains a point of $A$ other than $p$. 
	A point $p \in A$ is called an \textbf{isolated point} of $A$ if $p$ has a
	neighbourhood $U \subset X$ such that $U \cap A = \{p\}$.
\end{definition}

\begin{remark}
	\hfill
	\begin{enumerate}
		\item All points of a set must either be a limit point of the set or an
			isolated point.
	
		\item A limit point of a set does not necessarily have to be contained 
			within that set (e.g. punctured balls).

		\item Limit poitns are also called \textbf{accumulation points} or
			\textbf{cluster points}.
	\end{enumerate}
\end{remark}

\begin{example}[]
	Consider $X = \R$ and $A = (0,1)$. 
	Then every point in $[0,1]$ is a limit point of $A$.
	If we let $B = \left\{ \frac1n \right\}_{n=1}^\infty$
	then the only limit point of $B$ is $0$, the rest of them are
	isolated points.
\end{example}

\begin{problem}
	Show that a subset of a topological space is closed if and only if
	it contains all of its limit points.
\end{problem}

\begin{solution}
	Let $X$ be a topological space and $A \subset X$ be closed.
	Then $X \setminus A$ is open so
	\begin{align*}
		\;\forall\; x \in X \setminus A \;\exists\; \varepsilon > 0
		&: B_\varepsilon(x) \subset X \setminus A \\
		\;\forall\; x \in X \setminus A \;\exists\; \varepsilon > 0 
		\;\forall\; y \in A
		&: y \not\in B_\varepsilon(x)
	\end{align*}
	and so there are no limit points in $X \setminus A$; hence,
	all limit points must exists within $A$.
\end{solution}

\begin{definition}[Dense]
	A subset $A$ of a topological space $X$ is said to be \textbf{dense}
	in $X$ if $\overline{A} = X$.
\end{definition}

\begin{problem}
	Show that a subset $A \subset X$ is dense if and only if
	every non-empty open subset of $X$ contains a point of $A$.
\end{problem}

\begin{solution}
	This is a direct result of $(iv)$ of 
	Proposition \ref{prop:clos-int-ext-boun}.
\end{solution}
