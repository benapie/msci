%! TEX root = ../master.tex
\subsection{Analysis}

This content comes from the first section in the second half of a two-course 
sequence called \emph{Introduction to Analysis} from \emph{MIT}.

\begin{definition}[]
	Let $\bm x = (x_1, \ldots, x_n) \in \R^n$.
	The \textbf{Euclidean norm} of $\bm x$ is
	\[
		\norm{\bm x} = \sqrt{x_1^2 + \ldots + x_n^2}
	\]
	and the \textbf{sup norm} is
	\[
		\abs x = \max_{i \in \{1,\ldots,n\}} \abs{x_i}.
	\]
\end{definition}

\begin{remark}
	From the norms defined above, for two points $\bm x, \bm y \in \R^2$ we get
	the Euclidean distance function $\norm{\bm x - \bm y}$
	and the sup distance function $\abs{\bm x - \bm y}$.
\end{remark}

\begin{proposition}[]
	A subset $U$ of $\R^n$ is open with the Euclidean distance function
	if and only if it is open with the sup distance function.
\end{proposition}

\begin{proof}
	Let $d_E$ be the Euclidean distance function and $d_{\sup}$ be the
	sup distance function.
	To prove this we must only show that there exists $a, b \in \R^n$
	such that for all $\bm x, \bm y \in \R^n$ we have $
		d_E(x,y) < a \, d_{\sup}(x,y)
	$ and $
		d_{\sup}(x,y) < b \, d_E(x,y)
	$.
\end{proof}

\begin{definition}[Continuity]
	Let $(X, d_X)$ and $(Y, d_Y)$ be two metric spaces
	and $f: X \to Y$ be a map.
	We say that $f$ is \textbf{continuous at $x_0$}
	if for all $\varepsilon > 0$ there exists $\delta > 0$ such that
	\[
		x \in B_{\delta}(x_0) 
		\implies f(x) \in B_{\varepsilon} \left(f(x_0)\right)
	\]
	(with the appropriate metrics used for the balls).
\end{definition}

\begin{theorem}[]
	Let $(X, d_X)$ and $(Y, d_Y)$ be two metric spaces,
	$f: X \to Y$ be a map,
	and $U$ be open in $Y$.
	$f$ is continuous if and only if $f^{-1}(U)$ is open in $X$.
\end{theorem}

\begin{proof}
	This is the \emph{definition} of continuous for topological spaces, and we 
	have already proved equivalence of topological continuity and metric 
	continuity previously.
\end{proof}

\begin{definition}[Limit point]
	Suppose that $X$ is a metric space and $A \subset X$.
	The point $x \in X$ is a \textbf{limit point} of $A$ if for all
	$\varepsilon > 0$ the ball $B_{\varepsilon}(x)$ is an infinite set.
\end{definition}

\begin{proposition}[]
	Suppose that $(X, d)$ is a metric space, $\varepsilon$ and $\delta$ are
	positive real numbers, and $x$ and $y$ are point in $X$.
	Then
	\[
		B_{\varepsilon}(x) \subset B_{\delta}(y)
		\implies \delta > d(x,y).
	\]
\end{proposition}

\begin{proof}
	\[
		x \in B_{\varepsilon}(x)
		\implies x \in B_{\delta}(y)
		\implies d(x,y) < \delta.
	\]
\end{proof}

\begin{proposition}[]
	Suppose $(X,d)$ is a metric space,
	$x$ is a point of $X$,
	$A_1 \subset B_{\varepsilon}$,
	and $A_2 \subset B_{\delta}(x)$.
	Then
	\[
		A_1 \cup A_2 \subset B_{\max\{\varepsilon, \delta\}}(x).
	\]
\end{proposition}

\begin{proof}
	Omitted.
\end{proof}

\begin{definition}[Bounded]
	Suppose $X$ is a metric space and $A \subset X$.
	We say that $A$ is \textbf{bounded} if there exists $\varepsilon > 0$
	such that for all $x, y \in A$ we have $y \in B_{\varepsilon}(x)$.
\end{definition}

\begin{definition}[Cover]
	Suppose $X$ is a metric space and $A \subset X$.
	The collection of subsets $\left\{ U_\alpha \right\}_{\alpha \in I}$
	on $X$ (where $I$ is a labelling set) is a \textbf{cover} of $A$ if
	\[
		A \subset \bigcup_{\alpha \in I} U_\alpha.
	\]
\end{definition}

\begin{definition}[Compact on $\R^n$]
	Consider the metric space $(\R^n, d)$.
	For any subset $A \subset \R^n$, $A$ is \textbf{compact} if
	it is closed and bounded.
\end{definition}

\begin{theorem}[Heine-Borel]
	Consider the metric space $(\R^n, d)$.
	Let $A \subset \R^n$ be compact and
	$\left\{ U_\alpha \right\}_{\alpha \in I}$ be an open cover of $A$
	(that is, $U_\alpha$ is open for all $\alpha \in I$).
	Then a finite number of $U_\alpha$'s cover $A$.
\end{theorem}

\begin{remark}
	The property that a finite number of $U_\alpha$'s cover $A$ is called the
	\textbf{Heine-Borel property} (H-B property).
	This is the property that allows us to extend the idea of being closed
	and bounded in Euclidean space to metric spaces and more generally,
	\emph{any} topological space.
\end{remark}

\begin{definition}[Compact]
	Let $X$ be a metric space and $A$ be a subset of $X$.
	$A$ is \textbf{compact} if each of its open covers has a finite subcover.
\end{definition}

\begin{theorem}[]
	Let $X$ and $Y$ be metric spaces and $f$ be a continuous map between them.
	If $A$ is compact in $X$, then $f(A)$ is compact in $Y$.
\end{theorem}

\begin{proof}
	Let $A$ be compact in $X$ 
	and let $\{U_\alpha\}_{\alpha \in I}$ be an open cover of $f(A)$. 
	As $f$ is continuous,
	$\{f^{-1}(U_\alpha)\}_{\alpha \in I}$ is an open cover of $A$
	and as $A$ is compact there exists a finite open subcover
	$\{f^{-1}(U_\alpha)\}_{\alpha \in J}$.
	Therefore $\{U_\alpha\}_{\alpha \in J}$ is a finite open subcover of $f(A)$
	and hence $f(A)$ is compact in $Y$.
\end{proof}

\begin{theorem}[]
	Suppose $X$ is a metric space, 
	$A$ is compact in $X$, 
	and $f: X \to \R$ is a continuous map.
	Then $f$ has a maximum point on $A$.
\end{theorem}

\begin{proof}
	$f(A)$ is compact and so it is closed and bounded.
	Therefore $a = \sup{f(A)}$ exists and is contained within $f(A)$.
\end{proof}

\begin{definition}[Uniform continuity]
	Suppose $(X, d)$ is a metric space,
	$A$ is a subset of $X$,
	and $f: X \to \R$ is a continuous mapping.
	We say that $f$ is \textbf{uniformly continuous} on $A$
	if for every $\varepsilon > 0$ there exists $\delta > 0$
	such that \[
		d(x, y) < \delta \implies \abs{f(x), f(y))} < \varepsilon
	\]
	for every $x, y \in A$.
\end{definition}

\begin{theorem}[]
	Let $X$ be a metric space,
	$A$ be a compact subset of $X$,
	and $f: X \to \R$ be a continuous map.
	Then $f$ is uniformly convergent on $A$.
\end{theorem}

\begin{definition}[]
	We say that a metric space is \textbf{connected} if it is impossible
	to write it as a disjoint union of two non-empty open sets.
\end{definition}

\begin{theorem}[]
	Let $X$ and $Y$ be metric spaces and $f$ be a continuous map between
	them.
	It follows that
	\[
		\text{$X$ is connected} \implies
		\text{$f(X)$ is connected}.
	\]
\end{theorem}

\begin{proof}
	Assume that $X$ is connected and $f(X)$ is not connected.
	Then $f(X) = U_1 \cup U_2$ for some disjoint, 
	non-empty, open sets $U_1$ and $U_2$.
	Hence $
		X = f^{-1} \left( U_1 \cup U_2 \right) = f^{-1}(U_1) \cup f^{-1}(U_2)
	$. This is the union of two disjoint, non-empty, open sets therefore
	$X$ is not connected; a contradiction.
\end{proof}


\begin{theorem}[Intermediate value theorem]
	Suppose $X$ is a connected metric space, 
	$f: X \to \R$ is a continuous map,
	$a, b \in f(X)$, and $r \in \R$ such that $a < r < b$.
	Then $r \in f(X)$.
\end{theorem}

\begin{proof}
	As $X$ is connected, $f(X)$ is connected.
	Now assume $r \not\in f(X)$.
	Then $
		X 
		= f^{-1}\left( (-\infty, r) \cup (r, \infty) \right)
		= f^{-1}((-\infty, r)) \cup f^{-1}((r, \infty))
	$ thus $X$ is not connected; a contradiction.
\end{proof}
