\subsection{Databases}

\begin{definition}
	A \emph{database} is a collection of logically related data,
	designed to meet the needs of an organisation.
\end{definition}

A database is a single repository of data which may be accessed by many users.
To facilitate multiple users to having access to a databse,
there exists \emph{database management systems} (DBMS).
We split such systems into two basic components:
\begin{enumerate}
	\item \emph{data definition language}, this allows the user to define
	the data to the stored; and
	\item \emph{data manipulation lanaguage}, this allows users to
	insert, update, delete, and retrieve data from the repository.
\end{enumerate}
A DBMS offers \emph{controlled access} to the databse, this allows the
implementation of recovery control, concurrency, and security.

\begin{example}
	\emph{Structured query language} (SQL) is an example of a data definition
	language and data manipulation language that allows users to communicate
	with a DBSM.
\end{example}

Components of a DBMS environment:
\begin{enumerate}
	\item \emph{hardware};
	\item \emph{software};
	\item \emph{data}, the \emph{schemaof the database};
	\item \emph{procedures}, documented instructions on how to use and run the
	system; and
	\item personnel.
\end{enumerate}

There are a number of roles that personnel may take in a DBSM:
\begin{enumerate}
	\item database designers, split into logical designers (what to store)
	and physical designers (how to store);
	\item database administrator, responsible for maintenance of the database
	and an overview of security and integrity;
	\item application developers; and
	\item end users.
\end{enumerate}