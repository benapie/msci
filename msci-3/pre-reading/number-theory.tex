%! TEX root = ../master.tex
\subsection{Prime numbers}

This section comes from \emph{
	Elementary Number Theory: Primes, Congruences, and Secrets
} by \emph{William Stein} (first chapter).

\begin{definition}[Divides]
	Suppose $a,b \in \Z$.
	We say that \textbf{$a$ divides $b$}, 
	denoted $a \mid b$,
	if there exists $c \in \Z$ such that $ac = b$.
	$a$ is called a \textbf{divisor} of $b$. 
\end{definition}

\begin{remark}
	The idea of divisors was introduced in Algebra II and simple examples
	will be omitted.
\end{remark}

\begin{definition}[Prime and composite]
	Let $n \in \Z$ and $n > 1$.
	$n$ is \textbf{prime} if the only positive divisors are $1$ and $n$. 
	If $n$ is not prime, it must be \textbf{composite}.
\end{definition}

\begin{remark}
	$1$ is not considered prime nor composite.
\end{remark}

\begin{theorem}[Fundamental theorem of arithmetic]
	Every natural number can be expressed as a product of primes unique up to 
	order.
\end{theorem}

\begin{remark}
	This theorem is ingrained in our intuition, but the proof requires a little
	more work.
\end{remark}

\begin{definition}[Greatest common divisor]
	Suppose $a, b \in \Z$.
	Then we define the \textbf{greatest common divisor} as \[
		\gcd(a,b) = \max \{ d \in \Z: d \mid a \;\text{and}\; d \mid b \}
	\] with the exception that $\gcd(0,0) = 0$.
\end{definition}

\begin{lemma}[]
	\label{lem:gcd-lem-1}
	Suppose $a, b \in \Z$. Then
	\[
		\gcd(a, b)
		= \gcd(b, a)
		= \gcd(\pm a, \pm b)
		= \gcd(a, b - a)
		= \gcd(a, b + a).
	\]
\end{lemma}

\begin{proof}
	Here we will only prove $\gcd(a, b) = \gcd(a, b - a)$;
	the proof for the others follows a similar reasoning.
	Let $d = \gcd(a, b)$.
	Then $d \mid a$ and $d \mid b$.
	Let $n, m \in \Z$ such that
	$dn = a$ and $dm = b$.
	Now $d(m - n) = b - a$
	and so $d \mid b - a$.
	Therefore
	\[
		\{ d \in \Z : d \mid a \;\text{and}\; d \mid b \}
		\subset \{ d \in \Z : d \mid a \;\text{and}\; d \mid b - a \}
	\]
	and hence $\gcd(a, b) \leq \gcd(a, b - a)$.
	Simiarly we see that \[
		\gcd(a, b - a)
		=    \gcd(-a, b - a)
		\leq \gcd(-a, b)
		=    \gcd(a,b).
	\]
\end{proof}

\begin{lemma}
	Suppose $a, b, n \in \Z$.
	Then
	\[
		\gcd(a,b) = \gcd(a, b - an).
	\]
\end{lemma}

\begin{proof}
	This can be shown by repeatively applying Lemma \ref{lem:gcd-lem-1}.
\end{proof}

\begin{proposition}[]
	\label{prop:quotient-remainder}
	Suppose $a$ and $b$ are integers where $b$ is non-zero.
	Then there exists unique integers $q$ and $r$ such that
	$0 \leq r \leq \abs b$ and $a = bq + r$.
\end{proposition}

\begin{proof}
	For simplicity, we will assume $a$ and $b$ are positive.
	Now let
	\[
		Q = \left\{ n \in \Z_{\geq 0} : a - bn \geq 0 \right\}.
	\]
	It is clear to see that $Q$ is non-empty and bounded.
	Let $q = \max Q$.
	Let $r = a - bq < b$.
	Here we have proved existence, but not uniqueness.
	Now let $q'$ and $r'$ satisfy the conditions.
	As $r' = a - bq' \geq 0$, we have $q' \in Q$.
	Therefore $q' \leq q$.
	We assume that $q' \neq q$, so $q' = q - m$ for $m \geq 1$.
	Now
	\[
		r' 
		= a - bq' 
		= a - b(q - m)
		= a - bq + bm
		= r + bm
		\geq b;
	\]
	a contradiction.
	Hence our assumption that $q' \neq q$ is false.
\end{proof}

\begin{algorithm}[Division algorithm]
	Suppose $a$ and $b$ are integers with $b$ non-zero.
	The \textbf{division algorithm} calculates integers $q$ and $r$
	such that $a = bq + r$ with $0 \leq r < \abs b$.
\end{algorithm}

\begin{remark}
	The steps of the division algorithm and using it to calculate the greatest 
	common divider was covered in Algebra II and will be omitted.
	The idea is to, with $a, b \in \Z$, apply Lemma \ref{lem:gcd-lem-1} to see $
		\gcd(a,b) = \gcd(b,r)
	$ with $q$ and $r$ coming from the application of Proposition 
	\ref{prop:quotient-remainder}.
\end{remark}

\begin{lemma}[]
	Let $a$, $b$, and $n$ be integers.
	Then
	\[
		\gcd(an, bn) = \abs n \cdot \gcd(a, b).
	\]
\end{lemma}

\begin{proof}
	For simplicity we will assume that $a$ and $b$ are positive.
	Now we will prove by induction using $a + b = 2$ as our base case.
	In this case, $a = b = 1$ and clearly $
		\gcd(n, n) = \abs n \cdot \gcd(1, 1)
	$
	so it holds for the base case.
	Now let $k > 2$ and we assume that our Lemma holds for all $a + b < k$.
	We assume without loss of generality that $a \geq b$,
	and let $a = q(b) + r$ with $0 \leq r < b$.
	Then $\gcd(a, b) = \gcd(b, r)$.
	Similarly $
		\gcd(an, bn) = \gcd(bn, rn)
	$. Now \[
		b + r = b + (a - bq) = a - b(q - 1) \leq a < a + b = k
	\] and so \[
		\gcd(an, bn) = \gcd(bn, rn) = \abs n \cdot \gcd(b, r).
	\]
	Thus, the as the Lemma holds for the base case $a + b = 2$ and
	for $a + b = k$ assuming it holds true for $a + b < k$: by induction, 
	the Lemma must hold for all $a + b > 1$ and thus $a, b > 0$.
\end{proof}

\begin{lemma}[Euclid's]
	Let $a$, $b$, and $n$ be integers such that $n$ divides $a$ and $b$.
	Then $n$ divides $\gcd(a, b)$.
\end{lemma}

\begin{proof}
	Let $m_1$ and $m_2$ be integers such that $nm_1 = a$ and $nm_2 = b$.
	Then
	\[
		\gcd(a,b) = \gcd(nm_1, nm_2) = \abs n \cdot \gcd(m_1, m_2).
	\]
\end{proof}

\begin{theorem}[]
	Let $p$ be a prime integer and $a$ and $b$ be positive integers.
	If $p$ divides $ab$, then $p$ divides $a$ or $p$ divides $b$.
\end{theorem}

\begin{proof}
	If $p$ divides $a$, we are done.
	Now assume $p$ does not divide $a$.
	Then $\gcd(p, a) = 1$ so
	$\gcd(pb, ab) = b \gcd(p, a) = b$.
	As $p \mid \gcd(pb, ab)$, then $p \mid b$.
\end{proof}

\begin{proposition}[]
	\label{prop:nat-primes}
	Every natural number $n > 1$ is a product of primes.
\end{proposition}

\begin{proof}
	Let $P(n)$ be the statement that $n$ is a product of primes.
	$P(2)$ holds as $2 = 2 \cdot 1$.
	Now assume that $P(k)$ holds.
	Now we consider whether $P(k+1)$ holds, that is, 
	is $k + 1$ a product of primes.
	If $k + 1$ is prime, then we are done.
	Assume otherwise, then $k + 1 = ab$ for some $a, b \in \N$
	with $2 \leq a, b \leq k$.
	By $P(k)$, $a$ and $b$ are a product of primes.
	And hence $n$ is a product of primes.
	Therefore all natural numbers arwe a product of primes.
\end{proof}

\begin{problem}
	Is there an algorithm that can factor an integer in polynomial time?
\end{problem}

\begin{remark}
	This is an open problem; we don't actually have an answer to this yet,
	at least for the Turing machine computation model.
	We do have an algorithm to factor an in polynomial time for quantum computers
	though.
\end{remark}

\begin{proof}[Proof for the fundamental theorem of arithmetic]
	We now have enough to solve this theorem.
	For $n = 1$ we have the empty product.
	Now lets look at $n > 1$.
	By Proposition \ref{prop:nat-primes}
	$n = p_1 p_2 \ldots p_d$.
	Now suppose
	$n = q_1 q_2 \ldots q_m$.
	Clearly
	\[
		p_1 \mid n = q_1 q_2 \ldots q_m,
	\]
	and by Euclid's lemma either $p_1 \mid q_1$ or $p_1 \mid q_2 \ldots q_m$.
	Now if $p_1 \mid q_1$ then $p_1 = q_1$ as they are prime.
	Now we repeatively apply Euclid's lemma to see that $p_1 = q_i$ for some
	$i \in \{1, \ldots, m\}$.
	Now we cancel $p_1$ and $q_i$ and repeat this argument to find that
	both expression are equal up to order.
\end{proof}

\begin{problem}
	Let $a$, $b$, $c$, and $n$ be integers with $a$ and $b$ coprime.
	Prove that
	\begin{enumerate}
		\item $(a \mid n \;\text{and}\; b \mid n) \implies ab \mid n$; and
		\item $a \mid bc \implies a \mid c$.
	\end{enumerate}
\end{problem}

\begin{solution}
	\hfill
	\begin{enumerate}
		\item
			Let $am_1 = n$ and $bm_2 = n$.
			As $a$ and $b$ are coprime, there exists integers $r_1$ and $r_2$
			such that $ar_1 + br_2 = 1$.
			Multiplying by $b$ we get $nar_1 + nbr_2 = n$.
			Then $bm_2ar_1 + am_1br_2 = n$
			and so $ab(m_2r_1 + m_1r_2) = n$
			hence $ab \mid n$.

		\item
			As $a \mid bc$, $a \mid b$ or $a \mid c$.
			As $a \not\mid b$, then $a \mid c$.
	\end{enumerate}
\end{solution}


\begin{problem}
	Let $a$, $b$, $c$, $d$, and $m$ be integers.
	Prove that
	\begin{enumerate}
		\item $(a \mid b \;\text{and}\; b \mid c) \implies a \mid c$;
		\item $(a \mid b \;\text{and}\; c \mid d) \implies ac \mid bd$;
		\item $m \neq 0 \implies (a \mid b \iff ma \mid mb)$; and
		\item $(a \mid d \;\text{and}\; d \neq 0) \implies \abs d \leq \abs a$.
	\end{enumerate}
\end{problem}

\begin{solution}
	\begin{enumerate}
		\item Omitted.
		\item 
			Let $an = b$ and $cm = d$.
			Then $(ac)(nm) = bd$ and so $ac \mid bd$.
		\item
			\begin{description}
				\item[$\implies$] 
					Let $an = b$.
					Then $(ma)(n) = mb$ and so $ma \mid mb$.
				\item[$\impliedby$]
					Let $man = mb$.
					Then $m(an - b) = 0$.
					As $m = \neq 0$, $an = b$.
					Hence $a \mid b$.
			\end{description}
		\item
			Let $an = d$.
			So $\abs a \abs n = \abs{an} = \abs d$.
			Therefore $\abs d \leq \abs a$.
	\end{enumerate}
\end{solution}

\begin{problem}
	Suppose $a$, $b$, and $n$ are positive integers.
	Prove that $a^n \mid b^n \implies a \mid b$.
\end{problem}

\begin{solution}
	Let $p_1, p_2, \ldots$ be the primes in order.
	By the fundamental theorem of arithmetic we can uniquely (up to order)
	express $a$ and $b$ as a product of these primes.
	Let
	\[
		a = p_1^{\alpha_1} p_2^{\alpha_2} \ldots p_r^{\alpha_r},
		\qquad b = p_1^{\beta_1} p_2^{\beta_2} \ldots p_r^{\beta_r}.
	\]
	Note that
	\[
		a^n = p_1^{n\alpha_1} p_2^{n\alpha_2} \ldots p_r^{n\alpha_r},
		\qquad b^n = p_1^{n\beta_1} p_2^{n\beta_2} \ldots p_r^{n\beta_r}
	\]
	and as $a^n \mid b^n$ we must have $n\alpha_i \leq n\beta_i$ for each $i$.
	So $\alpha_i \leq \beta_i$.
	Then clearly $a \mid b$.
\end{solution}

\begin{problem}
	Suppose $a$ and $k$ are positive integers and $p$ is prime.
	Prove that $p \mid a^k \implies p^k \mid a^k$.
\end{problem}

\begin{solution}
	We let $a = p_1^{\alpha_1} p_2^{\alpha_2} \ldots p_r^{\alpha_r}$.
	Now $a^k = p_1^{k\alpha_1} p_2^{k\alpha_2} \ldots p_r^{k\alpha_r}$.
	As $p \mid a^k$, we must have $p = p_i$ for some $i$.
	But note that $p_i^k$ is a factor too; hence
	$p^k \mid a^k$.
\end{solution}
