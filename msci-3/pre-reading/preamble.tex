%%%%%%%%%%%%%%%%%%%%%%%%%%%%%%%%%%%%%%%%%%%%%%%%%%
%%%%%%%%%%%%%%%%=%=%=%=%==%=%=%=%=%%%%%%%%%%%%%%%%
%%%%%%%%%=%=%=%=%=%============%=%=%=%=%=%%%%%%%%%
%%%%%%=%=%=%==========================%=%=%=%%%%%%
%%%/         2020 MICHAELMAS PREAMBLE         \%%%
%%%\                Ben Napier                /%%%
%%%%%%=%=%=%==========================%=%=%=%%%%%%
%%%%%%%%%=%=%=%=%=%============%=%=%=%=%=%%%%%%%%%
%%%%%%%%%%%%%%%%=%=%=%=%==%=%=%=%=%%%%%%%%%%%%%%%%
%%%%%%%%%%%%%%%%%%%%%%%%%%%%%%%%%%%%%%%%%%%%%%%%%%
% tikzcd.yichuanshen.de/ tikcd diagrams
%======================%
%   Standard packages  %
%======================%
\usepackage[utf8]{inputenc}
\usepackage[T1]{fontenc}
\usepackage{lmodern}
\usepackage[UKenglish]{babel}
\usepackage{enumitem}
\usepackage{tasks}
\usepackage{graphicx}
\setlist[enumerate,1]{
	label={(\roman*)}
}
\usepackage{parskip}

%======================%
%        Maths         %
%======================%
\usepackage{amsfonts, mathtools, amsthm, amssymb}
\usepackage{xfrac}
\usepackage{bm}
\newcommand\N{\ensuremath{\mathbb{N}}}
\newcommand\R{\ensuremath{\mathbb{R}}}
\newcommand\Z{\ensuremath{\mathbb{Z}}}
\newcommand\Q{\ensuremath{\mathbb{Q}}}
\newcommand\C{\ensuremath{\mathbb{C}}}
\newcommand\F{\ensuremath{\mathbb{F}}}
\newcommand\B{\ensuremath{\mathbb{B}}}
\newcommand{\abs}[1]{\ensuremath{\left\lvert #1 \right\rvert}}
\newcommand\given[1][]{\:#1\vert\:}
\newcommand\restr[2]{{% we make the whole thing an ordinary symbol
	\left.\kern-\nulldelimiterspace % automatically resize the bar with \right
	#1 % the function
	\vphantom{\big|} % pretend it's a little taller at normal size
	\right|_{#2} % this is the delimiter
}}
\usepackage{tikz-cd}
\DeclareMathOperator{\norm}{N}
\DeclareMathOperator{\trace}{Tr}
\DeclareMathOperator{\inte}{int}
\DeclareMathOperator{\ext}{ext}

%======================%
%       CompSci        %
%======================%
\usepackage{circuitikz}
%\usepackage{algorithm, algpseudocode}
%\usepackage{listings}
%\lstset{
%    language=Python,
%    basicstyle=\small\ttfamily,
%    numbers=left,
%    numberstyle=\tiny,
%    frame=tb,
%    columns=fullflexible,
%    showstringspaces=false,
%    breaklines=true,
%    postbreak=\mbox{{$\hookrightarrow$}\space},
%}
\usepackage{textgreek}\newcommand{\sectionbreak}{\clearpage}

%======================%
%    Module Specific   %
%======================%

%======================%
%       Drawing        %
%======================%
%\usepackage{forest}
%\usepackage{tikz, pgfplots, pgf}
%\pgfplotsset{compat=1.16}
%\usetikzlibrary{positioning}

%======================%
%    Pretty tables     %
%======================%
\usepackage{booktabs}
\usepackage{caption}
\captionsetup[table]{skip=10pt}

%======================%
%     Environments     %
%======================%
\usepackage{mdframed}
\usepackage{xcolor}
\mdfsetup{
	%skipabove=0em,
	skipbelow=4pt,
    startcode=\leavevmode,
	linewidth=0,
	needspace=7em,
	innerbottommargin=6pt
}

% definition
\newtheoremstyle{definition}
{1pt}{} % space above/below theorem
{} % body font
{} % indent space 
{\vspace{3pt}} % head font
{} % punctuation between head and bodyla
{\newline} % space after theorem head
{\color{teal!50!black}{\textbf{\thmname{#1}\thmnumber{ #2}}\thmnote{ (#3)}.}}
\theoremstyle{definition}
\newmdtheoremenv[
	backgroundcolor=teal!10,
	startcode=\leavevmode\renewcommand\em{\bfseries\color{teal!50!black}}
]{definition}{Definition}[section]

% problem
\newtheoremstyle{problem}
{1pt}{} % space above/below theorem
{} % body font
{} % indent space 
{\vspace{3pt}} % head font
{} % punctuation between head and bodyla
{\newline} % space after theorem head
{\color{red!50!black}{\textbf{\thmname{#1}\thmnumber{ #2}}\thmnote{ (#3)}.}}
\theoremstyle{problem}
\newmdtheoremenv[
	backgroundcolor=red!10,
	startcode=\leavevmode\renewcommand\em{\bfseries\color{red!50!black}}
]{problem}[definition]{Problem}

% example
\newtheoremstyle{example}
{1pt}{} % space above/below theorem
{} % body font
{} % indent space 
{\vspace{3pt}} % head font
{} % punctuation between head and bodyla
{\newline} % space after theorem head
{\color{orange!50!black}{\textbf{\thmname{#1}\thmnumber{ #2}}\thmnote{ (#3)}.}}
\theoremstyle{example}
\newmdtheoremenv[
	backgroundcolor=orange!10,
	startcode=\leavevmode\renewcommand\em{\bfseries\color{orange!50!black}}
]{example}[definition]{Example}

\newmdtheoremenv[
	backgroundcolor=orange!10,
	startcode=\leavevmode\renewcommand\em{\bfseries\color{orange!50!black}}
]{examplesx}[definition]{Examples}

\newenvironment{examples}[1][]{
\begin{examplesx}[#1]\leavevmode\vspace{-\baselineskip}\vspace{-0.3em}}
{\end{examplesx}}

% algorithm
\newtheoremstyle{algorithm}
{1pt}{} % space above/below theorem
{} % body font
{} % indent space 
{\vspace{3pt}} % head font
{} % punctuation between head and body
{\newline} % space after theorem head
{\color{cyan!50!black}{\textbf{\thmname{#1}\thmnumber{ #2}}\thmnote{ (#3)}.}}
\theoremstyle{algorithm}
\newmdtheoremenv[
	backgroundcolor=cyan!10,
	startcode=\leavevmode\renewcommand\em{\bfseries\color{cyan!50!black}}
]{algorithm}[definition]{Algorithm}

% theorems
\newtheoremstyle{theorems}
{1pt}{} % space above/below theorem
{\itshape} % body font
{} % indent space 
{\vspace{3pt}} % head font
{} % punctuation between head and bodyla
{\newline} % space after theorem head
{\color{violet!50!black}{\textbf{\thmname{#1}\thmnumber{ #2}}\thmnote{ (#3)}.}}
\theoremstyle{theorems}
\newmdtheoremenv[%s/textbf/emph/g
	backgroundcolor=violet!10,
	startcode=\leavevmode\renewcommand\em{\bfseries\color{violet!50!black}}
]{theorem}[definition]{Theorem}

\newmdtheoremenv[
	backgroundcolor=violet!10,
	startcode=\leavevmode\renewcommand\em{\bfseries\color{violet!50!black}}
]{proposition}[definition]{Proposition}


\newmdtheoremenv[
	backgroundcolor=violet!10,
	startcode=\leavevmode\renewcommand\em{\bfseries\color{violet!50!black}}
]{corollary}[definition]{Corollary}

\newmdtheoremenv[
	backgroundcolor=violet!10,
	startcode=\leavevmode\renewcommand\em{\bfseries\color{violet!50!black}}
]{lemma}[definition]{Lemma}

\surroundwithmdframed[
	backgroundcolor=black!7,
	topline=false,
	rightline=false,
	bottomline=false,
	linewidth=1pt,
	linecolor=black!60
]{proof}

% remark
\theoremstyle{remark}
\newtheorem*{remarkx}{Remark}
\newenvironment{remark}
    {\begin{remarkx}}
    {\renewcommand{\qedsymbol}{\footnotesize $\triangle$}\qed\end{remarkx}}

% solution
\newenvironment{solution}{\begin{proof}[Solution]}{\end{proof}}

%======================%
%    Page Formatting   %
%======================%
\usepackage{fancyhdr}
\usepackage{titling}
\usepackage{titlesec}
\pagestyle{fancy}
\fancyhf{}
\lhead{\textsc{\mytitle}}
\rfoot{\leftmark}
\rhead{\thepage}
\renewcommand{\footrulewidth}{0.5pt}
\date{2020-2021}
\title{\textsc\mytitle}
\author{Ben Napier}

%======================%
%    Lecture Command   %
%======================%
\newcommand{\lecture}[2]{\marginpar{Lecture #1 \\ On #2}}

%======================%
%   Header and Footer  %
%======================%
\usepackage{hyperref}
\hypersetup{
    colorlinks,
    citecolor=black,
    filecolor=black,
    linkcolor=black,
    urlcolor=black,
    pageanchor=false
}
