\documentclass[a4paper, answers]{exam}
\usepackage[utf8]{inputenc}

\usepackage{parskip}

\usepackage{amssymb}
\usepackage{amsmath}
\usepackage{amsfonts}
\usepackage{mathtools}

\usepackage{todonotes}

\usepackage{csquotes}

\usepackage{algpseudocode}
\usepackage{algorithm}

\DeclareMathOperator{\AQ}{AQ}
\DeclareMathOperator{\DAQ}{\Delta AQ}
\DeclareMathOperator{\Q}{Q}
\DeclareMathOperator{\HE}{HE}
\DeclareMathOperator*{\argmax}{arg\,max}
\DeclareMathOperator{\Ep}{Ep}

\usepackage{braket}

\addtolength{\oddsidemargin}{-.875in}
\addtolength{\evensidemargin}{-.875in}
\addtolength{\textwidth}{1.75in}
\addtolength{\topmargin}{-.875in}
\addtolength{\textheight}{1.75in}

\usepackage[backend=biber]{biblatex}
\addbibresource{ref.bib}

\title{Natural Computing Part A}
\author{Ben Napier}
\date{March 2022}

\begin{document}

\begin{center}
	\textbf{\textsc{Number Theory III, Week 4}} \\
	\textsc{Ben Napier}
	\vspace{1em}
\end{center}

\begin{questions}
	\question 
	Find all pairs of positive integers $a,b$ such that
	$a^2 + 2b^2 = 19 \times 43$.

	\question
	Find how many solutions $a,b \in \Z$ exist to the following equation
	\[
		a^2 + 2b^2 = 3^{14} \times 43^{10}.
	\]

	\question
	Find how many solutions $(a,b)$ there are with 
	(i) $a,b \in \Z$ and
	(ii) $a,b \in \N$
	to the following equations
	\[
		a^2 + b^2 = 27 \times 41 \times 43.
	\]

	\question
	Find all solutions $x, y \in \Z$ that satisfy $x^2 + 8 = y^3$.
	(Hint: Consider the element $\alpha = x + 2\sqrt{-2} \in \Z[\sqrt{-2}]$,
	and hsow that there is a $\beta \in \Z[\sqrt{-2}]$ such that
	$\beta^3 = \alpha$).

	\question
	Let $R$ be a commutative ring which is an integral domain.
	Define the set $F_R$ by
	\[
		F_r = \left\{
			\frac rs :
			r \in R,
			s \in R \setminus \left\{
				0
			\right\}
		\right\} / {\sim}
	\]
	where ${\sim}$ indicates the equivalence relation:
	$\frac{r_1}{s_1} \sim \frac{r_2}{s_2}$
	if $r_1s_2 = s_1r_2$.
	\begin{parts}
		\part
		Define the operations
		$
			\frac{r_1}{s_1}  + \frac{r_2}{s_2} 
			= \frac{r_1s_2 + s_1r_2}{s_1s_2} 
		$
		and
		$
			\frac{r_1}{s_1}  \cdot \frac{r_2}{s_2}
			= \frac{r_1r_2}{s_1s_2}
		$.
		Show that $F_R$ is a field with respect to these two operations
		and the map
		$r \mapsto \frac r1$
		defines an injection of $R$ in $F_R$ as a ring.
		We call $F_R$ the field of fractions of $R$.

		\part
		We say that $R$ is integrally closed in $F_R$ if for every
		$\alpha \in F_R$, which is a root of a monic polynomial with
		coefficients in $R$, is actually in $R$, i.e. $\alpha \in R$.
		Is $\Z[\sqrt{-3}]$ integrally closed in its field of fractions?

		\part
		Assume that $R = \mathcal O_K$, for some number field $K$.
		Show that $F_R = K$.
		Show further that $\mathcal O_K$ is integrally closed in $K$.
	\end{parts}

	\question
	Let $K = \Q(\sqrt d)$, with $d$ square free and assume that $\mathcal O_K$
	is a UFD.
	For an \emph{odd} prime integer $p$ show that:
	\begin{parts}
		\part 
		if there is no $x \in \Z$ such that $x^2 \equiv d \pmod p$,
		then $p$ is also prime in $\mathcal O_K$,

		\part 
		if there is an $x \in \Z$ such that $x^2 \equiv d \pmod p$, and $p$
		does not divide $d$ then $p = \pm \pi \overline \pi$, for some
		prime $\pi \in \mathcal O_K$ and
		$\pi/\overline\pi \not\in\mathcal O_K$,

		\part
		if $p$ divides $d$, then $p = u\pi^2$ for some prime
		$\pi \in \mathcal O_K$ and $u \in \mathcal O_K^\times$.
	\end{parts}
\end{questions}

\end{document}
