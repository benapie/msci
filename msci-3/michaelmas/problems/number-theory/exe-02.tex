\documentclass[a4paper, answers]{exam}
\usepackage[utf8]{inputenc}

\usepackage{parskip}

\usepackage{amssymb}
\usepackage{amsmath}
\usepackage{amsfonts}
\usepackage{mathtools}

\usepackage{todonotes}

\usepackage{csquotes}

\usepackage{algpseudocode}
\usepackage{algorithm}

\DeclareMathOperator{\AQ}{AQ}
\DeclareMathOperator{\DAQ}{\Delta AQ}
\DeclareMathOperator{\Q}{Q}
\DeclareMathOperator{\HE}{HE}
\DeclareMathOperator*{\argmax}{arg\,max}
\DeclareMathOperator{\Ep}{Ep}

\usepackage{braket}

\addtolength{\oddsidemargin}{-.875in}
\addtolength{\evensidemargin}{-.875in}
\addtolength{\textwidth}{1.75in}
\addtolength{\topmargin}{-.875in}
\addtolength{\textheight}{1.75in}

\usepackage[backend=biber]{biblatex}
\addbibresource{ref.bib}

\title{Natural Computing Part A}
\author{Ben Napier}
\date{March 2022}

\begin{document}

\begin{center}
	\textbf{\textsc{Number Theory III, Problem Sheet 2}} \\
	\textsc{Ben Napier}
	\vspace{1em}
\end{center}

\begin{questions}
	\question
	Let $L$ be an extension of a field $F$ with $[L:F] < \infty$.
	\begin{parts}
		\part 
		Let $K_1$, $K_2$ be fields with $F \subset K_1$ and $K_2 \subset L$.
		Define $K_1 K_2$ as the smallest field in $L$ which contains both $K_1$
		and $K_2$.
		Show that
		\[
			[K_1 K_2 : F] \leq [K_1 : F] [K_2 : F].
		\]

		\part
		With the notation as above, assume now that $\gcd(n,m) = 1$ where
		$m = [K_1 : F]$ and $n = [K_2: F]$.
		Show that $[K_1 K_2 : F] = [K_1 : F][K_2 : F]$.
	\end{parts}

	\question
	Let $\theta \in \C$ and set $K = \Q(\theta)$.
	For $a$, $b$, and $c$ in $\Q$, calculate 
	$\operatorname{Tr}_{K/\Q}(a + b\theta + c\theta^2)$ and
	$\operatorname{N}_{K/\Q}(a + b\theta + c\theta^2)$ where $\theta$ satisfies
	$\theta^3 + \theta^2 + 2 = 0$.
	
	\question
	\begin{parts}
		\part
		Let $d$ be a square free integer.
		Show that $\sqrt d \not\in \Q$.

		\part
		Let $K$ be a number field with $[K: \Q] = 2$.
		Show that there exists a square free integer $d \in \Z$ such that
		$K = \Q(\sqrt d)$.
	\end{parts}

	\question
	Let $\alpha \in \overline\Z$.
	\begin{parts}
		\part
		Show that, $\operatorname{N}_{L/\Q}(\alpha) \in \Z$ and
		$\operatorname{Tr}_{L/\Q}(\alpha) \in \Z$ for any number field $L$
		that contains $\alpha$.

		\part
		Let $\alpha \in K$, for some number field $K$.
		Show that $\alpha \in \mathcal O^\times_K$ if and only if
		$\operatorname{N}_{K/\Q}(\alpha) = \pm 1$.
	\end{parts}

	\question
	\begin{parts}
		\part
		Show that the folowing two definitions of the field of algebraic numbers
		$\overline\Q$ are equivalent: 1) the union of all algebraic numbers,
		and 2) the union of all number fields.

		\part
		Determine all the irreducible polynomials in $\overline\Q[x]$.
	\end{parts}

	\question
	(Hard) Let $F \subset L$ be a field extension with $[L:F] = d < \infty$.
	Show that if $L = F(\theta)$ for some $\theta \in L$ then there exists
	only finitely many different subfields $K$ of $L$ containing $F$.
\end{questions}

\end{document}
