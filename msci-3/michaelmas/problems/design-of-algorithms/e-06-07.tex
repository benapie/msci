\documentclass[a4paper, answers]{exam}
\usepackage[utf8]{inputenc}

\usepackage{parskip}

\usepackage{amssymb}
\usepackage{amsmath}
\usepackage{amsfonts}
\usepackage{mathtools}

\usepackage{todonotes}

\usepackage{csquotes}

\usepackage{algpseudocode}
\usepackage{algorithm}

\DeclareMathOperator{\AQ}{AQ}
\DeclareMathOperator{\DAQ}{\Delta AQ}
\DeclareMathOperator{\Q}{Q}
\DeclareMathOperator{\HE}{HE}
\DeclareMathOperator*{\argmax}{arg\,max}
\DeclareMathOperator{\Ep}{Ep}

\usepackage{braket}

\addtolength{\oddsidemargin}{-.875in}
\addtolength{\evensidemargin}{-.875in}
\addtolength{\textwidth}{1.75in}
\addtolength{\topmargin}{-.875in}
\addtolength{\textheight}{1.75in}

\usepackage[backend=biber]{biblatex}
\addbibresource{ref.bib}

\title{Natural Computing Part A}
\author{Ben Napier}
\date{March 2022}

\begin{document}

\begin{center}
	\textbf{\textsc{Design of Algorithms and Data Structures III, 
	Weeks 6 \& 7}} \\
	\textsc{Ben Napier}
	\vspace{1em}
\end{center}

\begin{questions}
	\question 
	Consider the binomial coefficient $\binom nk$ for $n,k \in \N$,
	$n \geq 2$, and $1 \leq k < n$.
	Describe an $O(n)$-processor PRAM algorithm to evaluate $\binom nk$
	in time $O(\log n)$.
	(Assume there is neither a built-in function for evaluating the binomial
	coefficient nor for factorials.)
	Briefly argue the correctness of your approach and explain why the claimed
	bounds on number of processors and running time hold.
	Which PRAM model is needed?

	\question 
	Describe a parallel algorithm for an $n$-processor
	CREW-PRAM algorithm that computes all prime numbers among $1, \ldots, n$
	in $O(\log n)$ steps.
	Reminder: a number $p$ is prime if and only if $p$ divides
	$(p-1)! + 1$.
	You do not need to write pseudo-code.

	\question
	Let $T = (V,E)$,
	$\left\lvert V \right\rvert = n$, be a directed, rooted, binary tree
	where all edges are directed towards the root.
	Show how every vertex can identify the root on a $O(n)$-processor
	CREW PRAM in time $O(1)$.
	Explain how the tree should initially be stored in the PRAM's memory.
	Show how to solve the problem on an EREW PRAM; how long would it take there?

	\question
	Formally prove that the last step of the tree routing algorithm does indeed
	produce a correct orientation of each edge toward the root.
\end{questions}

\end{document}
