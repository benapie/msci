\documentclass[a4paper, answers]{exam}
\documentclass[a4paper, answers]{exam}

% tikzcd.yichuanshen.de/ tikcd diagrams
%======================%
%   Standard packages  %
%======================%
\usepackage[utf8]{inputenc}
\usepackage[T1]{fontenc}
\usepackage{lmodern}
\usepackage[UKenglish]{babel}
\usepackage{enumitem}
\usepackage{tasks}
\usepackage{graphicx}
\setlist[enumerate,1]{
  label={(\roman*)}
}
\usepackage{parskip}
\usepackage{hyperref}

%======================%
%        Maths         %
%======================%
\usepackage{amsfonts, mathtools, amsthm, amssymb}
\usepackage{xfrac}
\usepackage{bm}
\newcommand\N{\ensuremath{\mathbb{N}}}
\newcommand\R{\ensuremath{\mathbb{R}}}
\newcommand\Z{\ensuremath{\mathbb{Z}}}
\newcommand\Q{\ensuremath{\mathbb{Q}}}
\newcommand\C{\ensuremath{\mathbb{C}}}
\newcommand\F{\ensuremath{\mathbb{F}}}
\newcommand{\abs}[1]{\ensuremath{\left\lvert #1 \right\rvert}}
\newcommand\given[1][]{\:#1\vert\:}
\newcommand\restr[2]{{% we make the whole thing an ordinary symbol
  \left.\kern-\nulldelimiterspace % automatically resize the bar with \right
  #1 % the function
  \vphantom{\big|} % pretend it's a little taller at normal size
  \right|_{#2} % this is the delimiter
}}

\newcommand\corestr[2]{{% we make the whole thing an ordinary symbol
  \left.\kern-\nulldelimiterspace % automatically resize the bar with \right
  #1 % the function
  \vphantom{\big|} % pretend it's a little taller at normal size
  \right|^{#2} % this is the delimiter
}}
\usepackage{siunitx}

\usepackage{afterpage}

\usepackage{tikz-cd}
\usepackage{adjustbox}
\DeclareMathOperator{\norm}{N}
\DeclareMathOperator{\trace}{Tr}
\DeclareMathOperator*{\argmax}{arg\,max}
\DeclareMathOperator*{\argmin}{arg\,min}
\DeclareMathOperator*{\esssup}{ess\,sup}
\DeclareMathOperator*{\SL}{SL}
\DeclareMathOperator*{\GL}{GL}
\DeclareMathOperator*{\SO}{SO}
\DeclareMathOperator*{\aut}{Aut}
\DeclareMathOperator*{\id}{id}
\DeclareMathOperator*{\coker}{coker}
\DeclareMathOperator*{\im}{im}



%======================%
%       CompSci        %
%======================%
\usepackage{forest}
\usepackage{textgreek}
\usepackage{algpseudocode}

%======================%
%    Pretty tables     %
%======================%
\usepackage{booktabs}
\usepackage{caption}

\begin{document}

\begin{center}
	\textbf{\textsc{Design of Algorithms and Data Structures, Week 3 Exercises}}
	\\
	\textsc{Ben Napier}
\end{center}

\vspace{1em}

\begin{questions}
	\question Consider the hash function
	\[
		h_a(x) = (ax \bmod p) \bmod n
	\]
	and the family
	\[
		H = \left\{ h_a: a \in \left\{ 1, 2, \ldots, p_1 \right\} \right\}.
	\]
	This family is not $2$-universal.
	Prove that $H$ is almost $2$-universal in the sense that for every
	$x,y \in \left\{ 0, 1, \ldots, p-1 \right\}$, if $h$ is chosen uniformly
	at random from $H$ then
	\[
		\operatorname{Pr}(h(x) = h(y)) \leq \frac2n.
	\]

	\question We examine a specific way in which $2$-universal hash functions
	differ from completely random hash functions.
	Let $S = \left\{ 0,1, \ldots, k \right\}$ and consider a hash function
	$h: \left\{ 0, 1, \ldots, p-1 \right\}$ for some prime $p \gg k$.
	Consider the values $h(0), h(1), \ldots, h(k)$.
	If $h$ is a completely random hash function, then the probability that $h(0)$
	is smaller than any of the otehr valeus is roughly $\sfrac1{(k+1)}$.
	(There may be a tie for the smallest value, so the probability that any
	$h(i)$ is the unique smallest is slightly less than $\sfrac1{(k+1)}$.)
	Now consider a hash function $h$ chosen from the family
	\[
		H = \left\{ h_{a,b}: a,b \in \left\{ 0, 1 \ldots, p-1 \right\} \right\},
		\qquad h_{a,b} = (ax + b) \bmod p.
	\]
	Estimate the probability that $h(0)$ is smaller than $h(1), \ldots, h(k)$ by
	randomly choosing $10000$ hash function from $H$ and computing $h(x)$ for all
	$x \in S$. 
	Run this experiment for $k = 32$ and $k = 128$, using primes 
	$p = 5\,023\,309$ and $p = 10\,570\,849$.
	Is your estimate close to $\sfrac1{(k+1)}$?
\end{questions}
\end{document}
