\documentclass[a4paper, answers]{exam}
\usepackage[utf8]{inputenc}

\usepackage{parskip}

\usepackage{amssymb}
\usepackage{amsmath}
\usepackage{amsfonts}
\usepackage{mathtools}

\usepackage{todonotes}

\usepackage{csquotes}

\usepackage{algpseudocode}
\usepackage{algorithm}

\DeclareMathOperator{\AQ}{AQ}
\DeclareMathOperator{\DAQ}{\Delta AQ}
\DeclareMathOperator{\Q}{Q}
\DeclareMathOperator{\HE}{HE}
\DeclareMathOperator*{\argmax}{arg\,max}
\DeclareMathOperator{\Ep}{Ep}

\usepackage{braket}

\addtolength{\oddsidemargin}{-.875in}
\addtolength{\evensidemargin}{-.875in}
\addtolength{\textwidth}{1.75in}
\addtolength{\topmargin}{-.875in}
\addtolength{\textheight}{1.75in}

\usepackage[backend=biber]{biblatex}
\addbibresource{ref.bib}

\title{Natural Computing Part A}
\author{Ben Napier}
\date{March 2022}

\begin{document}

\begin{center}
	\textbf{\textsc{Design of Algorithms and Data Structures, Week 3 Exercises}}
	\\
	\textsc{Ben Napier}
\end{center}

\vspace{1em}

\begin{questions}
	\question Consider the hash function
	\[
		h_a(x) = (ax \bmod p) \bmod n
	\]
	and the family
	\[
		H = \left\{ h_a: a \in \left\{ 1, 2, \ldots, p_1 \right\} \right\}.
	\]
	This family is not $2$-universal.
	Prove that $H$ is almost $2$-universal in the sense that for every
	$x,y \in \left\{ 0, 1, \ldots, p-1 \right\}$, if $h$ is chosen uniformly
	at random from $H$ then
	\[
		\operatorname{Pr}(h(x) = h(y)) \leq \frac2n.
	\]

	\question We examine a specific way in which $2$-universal hash functions
	differ from completely random hash functions.
	Let $S = \left\{ 0,1, \ldots, k \right\}$ and consider a hash function
	$h: \left\{ 0, 1, \ldots, p-1 \right\}$ for some prime $p \gg k$.
	Consider the values $h(0), h(1), \ldots, h(k)$.
	If $h$ is a completely random hash function, then the probability that $h(0)$
	is smaller than any of the otehr valeus is roughly $\sfrac1{(k+1)}$.
	(There may be a tie for the smallest value, so the probability that any
	$h(i)$ is the unique smallest is slightly less than $\sfrac1{(k+1)}$.)
	Now consider a hash function $h$ chosen from the family
	\[
		H = \left\{ h_{a,b}: a,b \in \left\{ 0, 1 \ldots, p-1 \right\} \right\},
		\qquad h_{a,b} = (ax + b) \bmod p.
	\]
	Estimate the probability that $h(0)$ is smaller than $h(1), \ldots, h(k)$ by
	randomly choosing $10000$ hash function from $H$ and computing $h(x)$ for all
	$x \in S$. 
	Run this experiment for $k = 32$ and $k = 128$, using primes 
	$p = 5\,023\,309$ and $p = 10\,570\,849$.
	Is your estimate close to $\sfrac1{(k+1)}$?
\end{questions}
\end{document}
