%! TEX root = master.tex

\section{Motivation}
\lecture{2}{16/10}

Recall the following problems.

\begin{problem}[Vertex cover]
	Let $G = (V,E)$ be a graph and $k \in \N$.
	Is there a set $W \subset V$ with $\left\lvert W \right\rvert \leq k$
	such that for every $(i,j) \in E$ we have
	\[
		\left\{ i,j \right\} \cap W \neq \emptyset?
	\]
\end{problem}

\begin{problem}[Independent set]
	Let $G = (V,E)$ be a graph and $k \in \N$.
	Is there a set $W \subset V$ with
	$\left\lvert W \right\rvert \geq k$
	such that for every $(i,j) \not\in E$.
\end{problem}

Let $G = (V,E)$ be a graph and $W \subset V$.
Then $W$ is a vertex cover if and only if $V \setminus W$
is an independent set.

Now both of the these problems are $\mathsf{NP}$-complete,
so from a complexity perspective they are equally hard.

Both of these problems can be solved via complete enumeration
in $O(n^k)$ time, and we can even solve
vertex cover in $O(2^k n)$ time.
Hence, for large graphs we can quickly check for small vertex covers.
However, there is no known algorithm for independet set that
runs in $n^{o(k)}$ time, which makes it infeasible
even with $k = 10$ and $n = 100$.
So, it looks like independent set is (probably) harder than vertex cover,
in some sense.

\begin{algorithm}[$O(2^kn)$ algorithm for vertex cover]
	We build a binary search tree of depth $k$ as follows.
	\begin{enumerate}
		\item Pick any edge $(x,y)$ and branch.

		\item Left branch puts $x$ in the vertex cover and removes
			it (and all edges) from $G$.

		\item Right branch puts $y$ in the vertex cover and removes
			it (and all edges) from $G$.

		\item Recurse by choosing any edge on each branch.

		\item Recurse by choosing any edge on
			each branch.
	\end{enumerate}
\end{algorithm}

\begin{remark}
	If $G$ has a vertex cover with at most
	$k$ elements, the algorithm must find it in one of the branches
	of the search tree.
	Moreover, the tree has $O(2^k)$ nodes,
	and $O(n)$ work is done at each node.
	Hence $O(2^k n)$ in total.
\end{remark}

\begin{definition}[Parametrized problem]
	Let $\Sigma$ be a finite alphabet.
	A \emph{parametrized problem} 
	is a language $L \subset \Sigma^\star \times \N$.
	The second component is called the \emph{parameter} of the problem.
\end{definition}

The idea with parametrized complexity is to measure the complexity of a problem
as a \emph{function} of the parameter(s) as well as the input size.

\begin{definition}[Fixed-parameter tractable]
	Suppose $L$ is a parametrised problem.
	We say that $L$ is \emph{fixed-parameter tractable} (FPT)
	if the question $(x,k) \in L?$ can be decided in running time
	$f(k) \cdot \left\lvert x \right\rvert^{O(1)}$
	where $f: \N \to \R$ is an arbitrary function.
	The corresponding complexity class if called $\mathsf{FPT}$.
\end{definition}

$\mathsf{FPT}$ is a robust complexity class,
further note that $f$ can be arbitrarily bad.

We can choose many different parameters for a problem;
for example, the size of subsets for vertex covers
and clique.
For $\mathsf{SAT}$, we can parametrise by
the number of variables or by the number of clauses.
We can also parametrise graph colouring by the number of colours
needed, and we can see that this problem cannot be in $\mathsf{FPT}$
as it is in $\mathsf{NP}$-complete for $k = 3$.

\begin{examples}[Problems in $\mathsf{NP}$-hard and $\mathsf{FPT}$]
	\begin{enumerate}
		\item Finding a vertex cover of size $k$;
		\item finding a path of length $k$;
		\item finding $k$ disjoint triangles;
		\item drawing a graph on a plane with $k$ edge crossings; and
		\item finding disjoint paths that connect $k$ pairs of vertices.
	\end{enumerate}
\end{examples}

We have many techniques we use to show that problems are $\mathsf{FPT}$.
We previously saw \emph{bounded search tree} which we used for our
$O(2^kn)$ algorithm for vertex cover.
We will now introduce \emph{kernelisation}.
As this is a developing field in Computer Science, there is no
clear consensus on how kernelisation should be formally defined.

\begin{definition}[Kernelisation]
	Let $\Sigma$ be a finite alphabet.
	A \emph{kernelisation} for a parametrised problem
	is an algorithm that takes an instance $(x,k) \in \Sigma^\star \times \N$
	and maps it in time polynomial in $\left\lvert x \right\rvert$ and $k$
	to an instance $(x', k') \in \Sigma^\star \times \N$
	such that
	\begin{enumerate}
		\item $(x, k) \in L \iff (x', k') \in L$;
		\item $k' < k$; and
		\item $\left\lvert x' \right\rvert \leq f(k)$ for some
			computable function $f: \N \to \R$.
	\end{enumerate}
	We call $(x', k')$ a \emph{kernel}.
\end{definition}

\begin{proposition}[]
	\label{prop:kernel-implies-fpt}
	Let $L$ be a parametrised decidable problem.
	If there exists a kernelisation for $L$, then $L$ is $\mathsf{FPT}$.
\end{proposition}

\begin{proof}
	Let $(x, k), (x', k') \in L$ where
	$(x', k')$ is the kernel of $(x,k)$
	calculate from some kernelisation.
	Note that
	\[
		\left\lvert (x', k') \right\rvert
		= \left\lvert x' \right\rvert + \left\lvert k' \right\rvert
		\leq \left\lvert f(k) \right\rvert + \left\lvert k' \right\rvert;
	\]
	hence, it does not depend on $n$ (only on $k$).
	$(x', k')$ can be solved by the algorithm that proves
	that the problem is deciable.
	$(x', k')$ was calculated in $\left\lvert x \right\rvert^{O(1)}$,
	and then solved in 
	$g(k) = \left\lvert f(k) + \left\lvert k' \right\rvert \right\rvert$
	and so $L \in \mathsf{FPT}$.
\end{proof}

\begin{algorithm}[Kernelisation for vertex cover]
	Let $G = (V,E)$ be a graph and $k \in \N$.
	We exhaustively apply the following rules.
	\begin{enumerate}
		\item If $v \in V$ is an isolated vertex, then 
			$(G, k) \to (G \setminus v, k)$.

		\item If $\deg(v) > k$, $(G,k) \to (G \setminus v, k - 1)$.
	\end{enumerate}
	If neither of the rules can be applied:
	\begin{enumerate}
		\item if $\left\lvert V \right\rvert > k(k+1)$
			then there is no vertex cover of size $k$;
		\item if $\left\lvert V \right\rvert \leq k(k+1)$,
			then we have a kernel.
	\end{enumerate}
\end{algorithm}

\begin{theorem}[]
	A parametrised problem is $\mathsf{FPT}$
	if and only if it admits kernelisation.
\end{theorem}

\begin{proof}
	Proposition \ref{prop:kernel-implies-fpt} shows us $\impliedby$,
	now we prove $\implies$.
	Assume the problem has an $f(k) \left\lvert x \right\rvert^c$ algorithm
	for some constant $c$.
	To kernelize the input, we
	run our algorithm for
	$\left\lvert x \right\rvert^{c+1}$
	steps.
	If the algorithm terminates, we return our yes or no instance.
	If it exceeds this bound, then our input is itself a kernel.
\end{proof}

\begin{remark}
	The size of our kernel is critical for running times;
	there is active research in improving kernel sizes of problems.
\end{remark}

\begin{definition}[Parametrized reduction]
	Suppose $X$ and $Y$ are parametrized problems
	and $(I, k) \in X$.
	We define a \emph{parametrized reduction} as a function
	$\phi: X \to Y$
	if it satisfies the following:
	\begin{enumerate}
		\item $(I,k) \in X \iff \phi(I,k) \in Y$;
		\item $\phi(I,k)$ can be computed in time
			$f(k) \left\lvert I \right\rvert^{O(1)}$
			for some $f: \N \to \R$; and
	\item $k' \leq g(k)$ for some $g: \N \to \R$
		and $(I', k') = \phi(I, k)$.
	\end{enumerate}
\end{definition}

\begin{remark}
	If we have a parametrized reduction between two parametrized problems
	$X$ and $Y$, then $Y \in \mathsf{FPT} \implies X \in \mathsf{FPT}$.
\end{remark}

\begin{example}[]
	There exists a parametrized redduction from the independent set problem
	to the clique problem; however, there does not exist a parametrized 
	reduction from the independent set problem to the vertex cover problem
	(with any choice of parameter).
\end{example}

\begin{problem}[Short acceptance]
	Suppose $M$ is a non-deterministic Turing machine,
	$x$ is an input string, and $k \in \N$.
	Is there a computation of $M$ that accepts $x$ after at most $k$ steps?
\end{problem}

\begin{definition}[$\mathsf{W[1]}$]
	We define the complexity class $\mathsf{W[1]}$
	to be the set of all parametrized problems
	such that there exists a parametrized reduction to 
	short acceptance.
\end{definition}


