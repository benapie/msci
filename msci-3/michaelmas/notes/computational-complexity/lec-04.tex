%! TEX root = master.tex

\section{Space complexity}
\lecture{4}{4/11}

Previously, we have defined complexity based on the time used by algorithms.
We now introduce similar analysis for the space used.

\begin{definition}[Space complexity]
	The \textbf{space complexity} of a Turing machine $T$
	is a function 
	\[
		\operatorname{Space}_T: \Sigma \to \N
	\]
	which denotes the distinct number of tape cells visited during the 
	computation of $x$ on $T$.
	If $T$ does not halt on $x$, $\operatorname{Space}_T(x)$ is undefined.
\end{definition}


We use asymptotic notation for space complexity as we did for time complexity.

\begin{definition}[]
	Let $f: \N \to \R$ and $\mathcal L$ be a decidable language.
	Then we say that the space complexity of $\mathcal L$ is in $O(f)$
	is there is a Turing machine $T$ such that $T$ decides $\mathcal L$
	and there is $n_0 \in \N$ and $c \in \R$ such that for every
	$x \in \Sigma$ with $\left\lvert x \right\rvert > n_0$ we have
	\[
		\operatorname{Space}_T(x) \leq cf(\left\lvert x \right\rvert).
	\]
\end{definition}

Now, like we did for time complexity, we can divide up the decidable languages into classes according to their space complexity.

\begin{definition}[Space complexity class]
	Let $f: \N \to \R$.
	The \textbf{space complexity class} $\mathsf{SPACE}(f)$ is defined
	to be the class of all languages with space complexity in $O(f)$.
\end{definition}

We sometimes call this $\operatorname{DSPACE}[f]$, for \emph{deterministic
space}.
As with time complexity, we obtain a \emph{robust} space complexity class by
by consider all \emph{polynomial} space complexity classes.

\begin{definition}[$\mathsf{PSPACE}$]
	\[
		\mathsf{PSPACE} = \bigcup_{k \in \N} \mathsf{SPACE}(n^k)
	\]
\end{definition}

\begin{theorem}[]
	Let $f: \N \to \R$. Then
	\[
		\mathsf{TIME}(f)
		\subset \mathsf{SPACE}(f)
		\subset \mathsf{TIME}(2^{O(f)}).
	\]
\end{theorem}

\begin{proof}
	The first inclusion is clear since visitng a cell requires at least
	one tiemstep.
	Now, for the second inclusion we define a graph $G$ for a given Turing 
	machine $T$ and input $x$.
	The vertices are all configurations of $T$, with at most
	$f(\left\lvert x \right\rvert)$ blank cells.
	Between any two vertices, there is an edge if there is a single
	transition linking the two configurations.
	Deciding whether $T$ accepts $x$ can be done in breadth first search
	in $2^{O(f(\left\lvert x \right\rvert))}$.
\end{proof}

Analysing the space complexity of an algorithm can be more useful than time as it can be reused.
For example, the satisfiability problem can be solved in linear space by
checking each assignment one at a time and reusing the space.
It follows that
\[
	\mathsf{NP} \subset \mathsf{PSPACE}
\]
as for any problem we can simply iterate through every possible value.

\begin{problem}[QSAT]
	Instance: a quantifier Boolean formula
	\[
		\Phi = (Q_1 x_1)(Q_2x_2) \ldots (Q_nx_n) \phi(x_1,x_2,\ldots,x_n)
	\]
	where each $x_i$ is a Boolean variable,
	$\phi$ is a Boolean formula in conjunctive normal form,
	and $Q_i$ are quantifiers. \\
	Question: is $\Phi$ logically valid?
\end{problem}

\begin{algorithm}[$\mathsf{PSPACE}$ algorithm for QSAT]
	Consider our input $\Phi$ as defined above.
	\begin{enumerate}
		\item If $\Phi$ has no quantifiers, it has only constants.
			Evaluate and accept if true.

		\item If $\Phi = \exists x\, \varphi$, then $\Phi$ is valid if
			$\varphi[x = \text{true}]$ or $\varphi[x = \text{false}]$
			are valid.
			We recursively call this algorithm on the two above and accept if
			one of them is accepted.

		\item We repeat a similar to last step but for 
			$\Phi = \forall x\,\varphi$ but we only accept is both are valid.
	\end{enumerate}
\end{algorithm}

\begin{proposition}[]
	The above algorithm uses $O(n)$ space.
\end{proposition}

\begin{proof}
	The depth of recursion is $n$, and we only store the value of
	one variable at each level.
	Thus, $O(n)$ space.
\end{proof}

\begin{theorem}[]
	QSAT is $\mathsf{PSPACE}$-complete.
\end{theorem}

\begin{proof}
	Suppose $L \in \mathsf{PSPACE}$ such that $L$ is decided by some
	Turing machine $T$ in $O(f(n))$ time.
	Construct a quantified conjunctive normal form formula $\varphi$ of size
	$O(f(\left\lvert x \right\rvert)^2)$ such that $\varphi$ is true
	if and only if $T$ accepts $x$.
	As in Cook-Levin, we build a conjuctive normal form formula
	$\phi_{T,x}$ of size $O(f(\left\lvert x \right\rvert))$ describing
	the arcs of $G_{T,x}$. 
	That is, for any two strings $C$ and $C'$,
	$\phi_{T,x}(C,C') = 1$ if and only if $(C,C')$ is an arc
	or $C = C'$.
	We then use $\phi_{T,x}$ to construct a formula $\varphi$ such that
	$\varphi(C,C')=1$ if and only if $C'$ can be reached from $C$ in $G_{T,x}$.
	Then $\varphi(C_{\text{start}}, C_{\text{acc}})$ is what we need.
	We inductively build the formula $\varphi_i$ such that
	$\varphi_i(C,C') = 1$ if and only if $C'$ can be reached from $C$
	in $G_{T,x}$ in at most $2^i$ steps.
	Note that $\varphi_0 = \phi_{T,x}$ and 
	$\varphi_{f(\left\lvert x \right\rvert)} = \varphi$.
	% todo?
\end{proof}

Now let us describe a game on a conjunctive normal form formula 
$\phi$ over $x_1, x_2, \ldots, x_n$ between two players: player 1 and player 2.
The two players take turns assigning trtuh values to the variables in order.
When all variables are instantiated, player 1 wins if the formula evaluates
true and player 2 wins otherwise.

\begin{problem}[Formula game]
	Instance: a conjunctive normal form formula $\phi$ over 
	$x_1, x_2, \ldots, x_n$. \\
	Question: is there a winning strategy in the game on $\phi$?
\end{problem}

\begin{theorem}[]
	Formula game is $\mathsf{PSPACE}$-complete.
\end{theorem}

\begin{proof}
	Formula game is the same problem as QSAT.
\end{proof}

\begin{problem}[Generalised geography]
	Instance: a game for 2 players consisting of a directed graph $G$,
	and a start node $s$.
	Players take turns to move along the edges from the current node.
	A player who cannot move without returning to a node already visited loses.
	\\
	Question: is there a winning strategy for the first player?
\end{problem}

Generalised geography can be shown to be $\mathsf{PSPACE}$-complete by
reduction from QSAT.
For any quantified Boolean formula, we can construct a game graph
such that the first player has a winning strategy if and only if the formula
is valid.
Many other $2$ player games are also $\mathsf{PSPACE}$-complete.

\begin{definition}[Alternating Turing machines]
	An \textbf{alternating Turing machine} is a non-deterministic Turing machine
	$N$ whose states $Q$ are split into parts: $Q_\text{and}$ and $Q_\text{or}$.
	An \textbf{eventually accept configuration} (EAC) on input $x$ is defined
	as
	\begin{enumerate}
		\item all accepting leaves in the computation are EAC;
		\item a configuration with state $Q_\text{and}$ is EAC if all
			successor configurations are  EAC; and
		\item a configuration with state $Q_\text{or}$ is EAC if there is a
			successor configuration which is EAC.
	\end{enumerate}
	We say that $N$ acceots $x$ if the initial state is EAC.
	$N$ decides a language $\mathcal L$ is for every $x \in \mathcal L$,
	$N$ accepts $x$ and for every $x \not\in \mathcal L$, $N$ rejects $x$.
\end{definition}

\begin{definition}[Alternating polynomial time]
	A language $\mathcal L$ belongs to the complexity class
	$\mathsf{AP}$, \textbf{alternating polynomial time},
	if and only if it can be decided by an alternating Turing machine
	in polynomial time.
\end{definition}

\begin{theorem}[]
	\hfill
	\vspace{-\baselineskip}
	\[
		\mathsf{AP} = \mathsf{PSPACE}.
	\]
\end{theorem}

