%! TEX root = master.tex
\section{Non-deterministic space}
\lecture{5}{20/11}

We can also define the space taken by a nondeterministic Turing machine.

\begin{definition}[Space complexity]
	The \emph{space complexity} of a nondeterministic Turing machine $T$
	is a function $\operatorname{NSpace}_T$ such that
	$\operatorname{NSpace}_T(x)$ is the number of cells visited in the shortest
	accepting path of $T(x)$, if one exists.
	Otherwise, it is the length of the longest branch.
	If not all paths of $T(x)$ halt, $\operatorname{NSpace}(x)$ is undefined.
\end{definition}

\begin{definition}[]
	For a function $f$,
	we say that the non-deterministic space complexity of a decidable
	language $\mathcal L$ is in $O(f)$ if there is a nondeterministic
	Turing machine $T$ which decides $\mathcal L$ and there is 
	constants $n_0 \in \N$ and $c > 0$ such that
	for all inputs $x$ with $\left\lvert x \right\rvert > n_0$,
	\[
		\operatorname{NSpace}_T(x) 
		\leq cf\left(\left\lvert x \right\rvert\right).
	\]
\end{definition}

\begin{definition}[$\mathsf{NSPACE}$]
	The nondeterministic space complexity class
	$\mathsf{NSPACE}(f)$ is defined to be the class of all languages with 
	nondeterministic space complexity in $O(f)$.
\end{definition}

\begin{theorem}[Savitch's theorem]
	For any $f \in \Omega(\log(n))$,
	\[
		\mathsf{NSPACE}\left(f(n)\right) 
		\subset \mathsf{SPACE}\left((f(n))^2\right).
	\]
\end{theorem}

We compare this result with our similar result for time complexity:
\[
	\mathsf{NTIME}\left(f\right) 
	\subset \bigcup_{c \geq 1} \mathsf{TIME}\left(c^f\right).
\]

\begin{definition}[$\mathsf{NPSPACE}$]
	$
		\mathsf{NPSPACE} = \bigcup_{k \in \N} \mathsf{NSPACE}\left(n^k\right)
	$.
\end{definition}

\begin{corollary}[]
	$
		\mathsf{PSPACE} = \mathsf{NPSPACE}
	$.
\end{corollary}

Thus, in polynomial space, having a nondeterministic Turing machine does
not add any more computational power.
\begin{proof}[Proof of Savitch's theorem]
	Let $\mathcal L \in \mathsf{NPSPACE}\left( 
		f 
	\right)$.
	Let $T$ be a nondeterminstic Turing machine that places $\mathcal L$
	in $\mathsf{NPSPACE}(f)$.
	Let $x$ be an input for $T$,
	and $G_{T,x}$ be a configuration graph of $T$ on $x$.
	Each configuration (vertex) of this graph has a most
	$
		f\left( 
			\left\lvert x \right\rvert 
		\right)
	$
	non-blank cells, thus each can be described in $c f\left( 
		\left\lvert x \right\rvert 
	\right)$ bits for some constant $c$.
	So $G_{T,x}$ must have less than $2^{cf\left( 
		\left\lvert x \right\rvert 
	\right)}$ vertices.
	Let $C$ and $C'$ be configurations of $T$ (that is, vertices of $G_{T,x}$).
	We define the functon $\psi_i$
	such that $\psi(C,C') = 1$ (is true) if and only if 
	the distance between $C$ and $C'$ (in $G_{T,x}$)
	is at most $2^i$, and $\psi(C,C') = 0$ otherwise.
	We claim that we can compute
	$
		\psi_{cf\left( 
			\left\lvert x \right\rvert 
		\right))}(C_\text{start}, C_\text{accept})
	$
	in
	$
	 O\left( 
		\left(
			f\left( 
				\left\lvert x \right\rvert 
			\right) 
		\right)^2
	 \right)
	$
	space, with the following divide-and-conquer algorithm.

	\textbf{Divide-and-conquer algorithm}
	Our input is a non-negative integer $j$ and two configurations
	$C$ and $C'$.
	\begin{enumerate}
		\item If $j = 0$, return 1 (true) if $C = C'$
			or $C'$ can be obtained in one step from $C$
			and return 0 (false) otherwise.
		\item For each configuration $C''$,
			if $\psi_{j-1}(C,C'') = 1$ and
			$\psi_{j-1}(C'',C') = 1$, then set 
			$\psi_{j}(C,C') = 1$.
		\item Return 0 (false).
	\end{enumerate}
	The depth of recursion is
	$
	O\left( 
		f\left( 
			\left\lvert x \right\rvert 
		\right) 
	\right)
	$
	and each recursive call required
	$
	O\left( 
		f\left( 
			\left\lvert x \right\rvert 
		\right) 
	\right)
	$
	space for the parameters.
\end{proof}

$\mathsf{PSPACE}$ is powerful, so we naturally consider more restircted
space complexity classes.
Linear space is enough to solve satisfiability, to get sublinear space
complexity we disregard the input and only consider working space.

\begin{definition}[$\mathsf{L}$ and $\mathsf{NL}$]
	\[
		\mathsf L = \mathsf{SPACE}(\log n), \qquad
		\mathsf{NL} = \mathsf{NSPACE}(\log n).
	\]
\end{definition}

Logarithmic space is sufficient to hold a constant number of pointers
into the input and a logarithmic number of boolean flags (counters).

\begin{examples}
	\begin{enumerate}
		\item Let $\mathcal L = \left\{
			0^n 1^n : n \in \Z_{\geq 0}
		\right\}$.
		We claim that $\mathcal L \in \mathsf L$.
		Indeed, observe the following algorithm.
		Initialise two counters for the number of $\mathtt 0$'s
		and the number of $\mathtt 1$'s.
		Scan through each of the characters, at each character
		execute the following.
		\begin{enumerate}
			\item If the current and next character reads $\mathtt 10$,
				reject.
			\item Incrememnt the corresponding counter.
		\end{enumerate}
		If the counters match, accept.
		Otherwise, reject.

		\item 
		We claim that determining palidromes in a problem that is in
		$\mathsf L$.
		Indeed, we will describe an algorithm that uses logarithmic
		space.
		Initialise two pointers, one at the start of a string and one at the
		end.
		If the value of both pointers match, move the starting pointer
		up the string and move the ending pointer down the string.
		Otherwise, reject.
		If the pointers end up pointing to the same address or switch addresses,
		accept.
	\end{enumerate}
\end{examples}

\begin{problem}[Planar graph isomorphism]
	Instance: undirected graphs $G = (V_G, E_G)$ and
		$H = (V_H, E_H)$. \\
	Question: are $G$ and $H$ isomorphic?
\end{problem}

\begin{theorem}[]
	Planar graph isomorphism is in $\mathsf L$.
\end{theorem}

We will give a nondeterministic logarthmic space algorithm
for solving reachability.

\begin{algorithm}[$\mathsf{NL}$ algorithm for reachability]
	We are given a finite directed graph $G = (V,E)$ and
	vertices $s,t \in V$.
	Declare a counter for the number of vertices in a graph,
	and initialise it to $\left\lvert V \right\rvert$.
	Further initialise a pointer to $s$.
	While the counter is non-zero, do the following steps.
	\begin{enumerate}
		\item
		If the current node is $t$, then return true.

		\item
		Nondeterministically choose every node that can be reached from the
		current node.

		\item 
		Update pointer and increment the counter.
	\end{enumerate}
	Return false.
\end{algorithm}

When we stated Savitch's theorem, we required $f \in \Omega(\log(n))$, not
$f \in \Omega(n)$.
That is, Savitch's theorem also applies in sublinear space.

\begin{corollary}[]
	$\mathsf{NL} \subset \mathsf L^2$.
\end{corollary}

Here, we are using the notation
\[
	\mathsf L^2 = \mathsf{SPACE}\left( 
		\left( 
			\log(n) 
		\right)^2
	\right).
\]

\begin{theorem}[]
	$\mathsf{NL} \subset \mathsf P$.
\end{theorem}

The following proof will follow the general outline,
more of a \emph{proof idea} then a rigorous proof.

\begin{proof}
	We build a configuration graph for our
	$\mathsf{NL}$ algorithm.
	We can then decide whether the accepting configuration
	can be reached by the starting configuration in linear time
	in the size of the graph ($2^{O(f)}$ vertices).
	We then redefine the configuration graph to avoid storing the input.
	We store only the contents of the working tape and the position of the head.
	The new configuration has size
	$2^{O(\log\left\lvert x \right\rvert)} = \left\lvert x \right\rvert^{O(1)}$.
\end{proof}
