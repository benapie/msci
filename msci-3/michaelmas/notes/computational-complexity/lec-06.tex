%! TEX root = master.tex
\section{Log space reductions}
\lecture{6}{20/11}

We found polynomial-time reductions useful in classifying problems in
$\mathsf{NP}$.
We define a similar concept for log space.

\begin{definition}[Log space transducer]
	A \emph{log space transducer} (LST) $M$ is a type of Turing machine
	with
	\begin{enumerate}
		\item a read-only \emph{input} tape;
		\item a read and write \emph{work} tape bounded at $O(\log n)$
			symbols; and
		\item a write \emph{output} tape.
	\end{enumerate}
	$M$ computes a function $f: \Sigma^\star \to \Sigma^\star$
	where $f(x)$ is the string remaining on the output tape after $M$ halts when
	it started with $x$ on the input tape.
	$f$ is said to be \emph{log space computable}.
\end{definition}

\begin{definition}[Log space reducible]
	We say that a language $\mathcal L_1$ is
	\emph{log space reducible} to a language $\mathcal L_2$,
	denotes $\mathcal L_1 \leq_L \mathcal L_2$ if there is a log space
	computable function $f$ such that $x \in \mathcal L_1$ if and only if
	$f(x) \in \mathcal L_2$.
\end{definition}

A Turing machine that uses $f(n)$ space
runs in $nO\left( 
	2^{f(n)} 
\right)$ time, so
\[
	\mathcal L_1 \leq_L \mathcal L_2 \implies
	\mathcal L_1 \leq \mathcal L_2.
\]

An open question is whether there exists a problem that is 
$\mathsf{NP}$-complete, but is not complete with respect to log reductions.

\begin{theorem}[]
	Reachability is $\mathsf{NL}$-complete.
\end{theorem}

This is easily proved by constructing a configuration graph (in log space).
The machine accepts $x$ if and only if $C_\text{accept}$ is reachable from
$C_\text{start}$.

\begin{corollary}[]
	$\mathsf L \subset \mathsf{NL} \subset \mathsf P$.
\end{corollary}

