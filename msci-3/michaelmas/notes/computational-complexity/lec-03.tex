%! TEX root = master.tex

\section{Cryptography and zero knowledge proofs}
\lecture{3}{23/10}

\begin{definition}[Symmetric cryptography]
	In \emph{symmetric-key encryption}, there is only one key used for
	both encryption of plaintext and decryption of ciphertext.
\end{definition}

In this generic encryption scheme, messages are encrypted and decrypted
with the same key, so both sender and recipient need the same key in order
to communicate.

\begin{definition}[Asymmetric cryptography]
	In \emph{asymmetric-key encryption}, there are a pair of keys:
	\textbf{public keys}, which may be disseminated widely and used to
	encrypt plaintext; and \textbf{private keys} which are only known
	to the owner and are used to decrypt ciphertext.
\end{definition}

\begin{example}[RSA]
	One common algorithm used for public key cryptography is \emph{RSA},
	in this system the recipient generates two primes $p, q \in \N$
	and computes $n = pq$ and $\phi = (p-1)(q-1)$.
	The recipient then randomly choses $e \in \N$ such that
	$e < \phi$ and $\gcd(e, \phi) = 1$.
	The public key is $(n,e)$.
	The recipient then calculates $d$ such that
	$de \equiv 1 \bmod \phi$.
	The private key is $(n,d)$.
	Now, the sender encrypts a message $m$ by calculating $c = m^e \bmod n$
	and transmits $c$.
	The recipient decrypts the message by finding $m = c^d \bmod n$.
\end{example}

Now let us examine how hard it would be to find the private key given the public
key.
To do this, we would have to find $p,q$ given $n = pq$.
That is, we are looking at the following decision problem.

\begin{problem}[Integer factorisation]
	Instance: $n, m \in \N$ with $m < n$. \\
	Question: is there $d \in \left\{
		1, \ldots, m
	\right\}$ such that $d \mid n$?
\end{problem}

There is no polynomial algorithm known for integer factorisation;
it is in $\mathsf{NP}$ and $\mathsf{coNP}$
but it is \emph{not} in $\mathsf{NP}$-complete
(unless $\mathsf{coNP} = \mathsf{NP}$).

Now we move onto \emph{authentication scheme}; that is,
a scheme in which we can verify that a user is who they say they are.
A generic authentication scheme would involve:
\begin{enumerate}
	\item a client sending a request to a server;
	\item the server sending a request for proof of identity back to the
		client;
	\item the client sending the proof of identity;
	\item the server verifying whether the proof sent is valid; then
	\item the server either responding to the request or rejecting.
\end{enumerate}

A glaring weakness in this scheme is that the client has to send the proof
of identity, so anyone who can get their hands on this data can use it to
falsely identify as the client.
We are interested in ways a client can identify themselves without
transmitting any information, a \emph{zero-knowledge proof} (ZKP).

\begin{definition}[Zero-knowledge proof]
	A \emph{zero-knowledge proof} is a method by which one party
	(the \emph{prover}) can prove to another party (the \emph{verifier})
	that they know a value $x$ without conveying any information apart
	from the fact that they know $x$.
	A zero-knowledge proof must satisfy three properties:
	\begin{enumerate}
		\item (\emph{completeness}) if the statement is true, the honest 
			verifier will be convinced of the fact by a honest prover;

		\item (\emph{soundness}) if the statement is false, no cheating
			prover can convince the honest verifier that it is true
			(except with some small probability); and

		\item (\emph{zero-knowledge}) if the statement is true, no verifier
			lerans anythin gother than the fact that the statement is true.
	\end{enumerate}
\end{definition}

We will use the following problem to illustrate the essence of a zero-knowledge
proof.

\begin{problem}[Graph isomorphism]
	Instance: undirected graphs $G = (V_G, E_G)$ and $H = (V_H, E_H)$. \\
	Question: does there exist a bijection $f: V_G \to V_H$ such that for every
	$(u,v) \in E_G$, $(f(u), f(v)) \in E_H$?
\end{problem}

\begin{algorithm}[ZKP for graph isomorphism]
	Let $G$ and $H$ as above
	and suppose that a prover would like to prove to a verifier
	that they know a bijection $f$ as described above.
	The prover first calculates a random permutation of $G$, say $A$,
	and sends it to our verifier.
	The verifier then chooses between $G$ and $H$ randomly,
	and sends $0$ if $G$ is chosen and $1$ otherwise.
	The prover then computers a permutation $\rho$ such that $H$ is
	obtained from either of the graphs the verifier had chosen.
	$\rho$ is then sent to the verifier, who then checks that the
	graph they chose can be obtained from $A$ using $\rho$.
\end{algorithm}

\begin{proof}
	The algorithm above is clearly complete.
	Now, suppose we have a cheating prover.
	This prover does not know an isomorphism between $G$ and $H$,
	but does know one between $A$ and one of $G$ or $H$.
	Thus, the prover must guess which graph the verifier will pick.
	For $n$ repetitions, this gives us a probability of $\sfrac{1}{2^n}$
	that the prover guesses correctly.
	Finally, no knowledge about the isomorphism is revealed from the provers
	transmission; hence, this describes a zero-knowledge proof.
\end{proof}

\begin{problem}[3-coloring]
	Instance: an undirected graph $G$. \\
	Question: is there a $k$-coloring of $G$?
\end{problem}

\begin{algorithm}[ZKP for 3-coloring]
	Both prover and verifier have a graph $G = (V,E)$.
	The prover has a 3-coloring $c$.
	First, the prover permutes the colors in $c$ before coloring $G$.
	Then they cover up the colored vertices.
	The verifier picks a random edge and the prover reveals the colours of
	that edge.
	If the verifier has a distinct colour then they accept, otherwise
	they reject.
\end{algorithm}

We can implement the above algorithm using \emph{bit commitment}:
\begin{enumerate}
	\item for every $x \in V$, the prover creates a public key $(n_x, e_x)$.
		and a private key $(n_x, d_x)$;

	\item the prover encoes all colours and sends them to the verifier;

	\item the verifier choses a random edge and sends it to the prover;
		
	\item the prover sends the private keys for the vertices of the edge; and

	\item the verifier checks that the keys decode the colours.
\end{enumerate}
