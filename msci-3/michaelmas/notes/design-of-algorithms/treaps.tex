\documentclass[a4paper]{article}
%%%%%%%%%%%%%%%%%%%%%%%%%%%%%%%%%%%%%%%%%%%%%%%%%%
%%%%%%%%%%%%%%%%=%=%=%=%==%=%=%=%=%%%%%%%%%%%%%%%%
%%%%%%%%%=%=%=%=%=%============%=%=%=%=%=%%%%%%%%%
%%%%%%=%=%=%==========================%=%=%=%%%%%%
%%%/         2020 MICHAELMAS PREAMBLE         \%%%
%%%\                Ben Napier                /%%%
%%%%%%=%=%=%==========================%=%=%=%%%%%%
%%%%%%%%%=%=%=%=%=%============%=%=%=%=%=%%%%%%%%%
%%%%%%%%%%%%%%%%=%=%=%=%==%=%=%=%=%%%%%%%%%%%%%%%%
%%%%%%%%%%%%%%%%%%%%%%%%%%%%%%%%%%%%%%%%%%%%%%%%%%
% tikzcd.yichuanshen.de/ tikcd diagrams
%======================%
%   Standard packages  %
%======================%
\usepackage[utf8]{inputenc}
\usepackage[T1]{fontenc}
\usepackage{lmodern}
\usepackage[UKenglish]{babel}
\usepackage{enumitem}
\usepackage{tasks}
\usepackage{graphicx}
\setlist[enumerate,1]{
	label={(\roman*)}
}
\usepackage{parskip}

%======================%
%        Maths         %
%======================%
\usepackage{amsfonts, mathtools, amsthm, amssymb}
\usepackage{xfrac}
\usepackage{bm}
\newcommand\N{\ensuremath{\mathbb{N}}}
\newcommand\R{\ensuremath{\mathbb{R}}}
\newcommand\Z{\ensuremath{\mathbb{Z}}}
\newcommand\Q{\ensuremath{\mathbb{Q}}}
\newcommand\C{\ensuremath{\mathbb{C}}}
\newcommand\F{\ensuremath{\mathbb{F}}}
\newcommand{\abs}[1]{\ensuremath{\left\lvert #1 \right\rvert}}
\newcommand\given[1][]{\:#1\vert\:}
\newcommand\restr[2]{{% we make the whole thing an ordinary symbol
	\left.\kern-\nulldelimiterspace % automatically resize the bar with \right
	#1 % the function
	\vphantom{\big|} % pretend it's a little taller at normal size
	\right|_{#2} % this is the delimiter
}}
\usepackage{tikz-cd}
\usepackage{adjustbox}
\DeclareMathOperator{\norm}{N}
\DeclareMathOperator{\trace}{Tr}
\DeclareMathOperator*{\argmax}{arg\,max}
\DeclareMathOperator*{\argmin}{arg\,min}

%======================%
%       CompSci        %
%======================%
\usepackage{forest}
\usepackage{textgreek}

%======================%
%    Module Specific   %
%======================%

%======================%
%       Drawing        %
%======================%
%\usepackage{forest}
%\usepackage{tikz, pgfplots, pgf}
%\pgfplotsset{compat=1.16}
%\usetikzlibrary{positioning}

%======================%
%     Environments     %
%======================%
\theoremstyle{definiton}
\newtheorem*{examplex}{Example}
\newenvironment{example}[1][]
    {\begin{examplex}[#1]}
    {\renewcommand{\qedsymbol}{\footnotesize $\triangle$}\qed\end{examplex}}

%======================%
%    Pretty tables     %
%======================%
\usepackage{booktabs}
\usepackage{caption}
\captionsetup[table]{skip=10pt}

%======================%
%    Page Formatting   %
%======================%
\usepackage{fancyhdr}
\usepackage{titling}
\usepackage{titlesec}
% \pagestyle{fancy}
\setlength{\droptitle}{-8em}
\renewcommand{\maketitlehooka}{\textsc}

\titleformat{\section}{\large\scshape}{}{0pt}{}
\titleformat{\paragraph}{\scshape}{}{0pt}{}
\renewcommand{\footrulewidth}{0.5pt}
\date{2020-2021}
\author{Ben Napier}

%======================%
%   Header and Footer  %
%======================%
\usepackage{hyperref}
\hypersetup{
    colorlinks,
    citecolor=black,
    filecolor=black,
    linkcolor=black,
    urlcolor=black,
    pageanchor=false
}
\begin{document}

\title{Treaps}
\maketitle

A \emph{treap} is a form of a binary search tree;
it is a hybrid of a \emph{binary search tree} and a \emph{heap}.

Recall the \emph{min-heap property} for a graph:
for any node $N$, if $P$ is a parent node of $N$, then the key of
$P$ is less than or equal to the key of $N$.
Now, at each node of a treap we store a \emph{key} and a \emph{priority}.
A treap is a binary search tree for the search keys
and a min-heap for the priorities.

Following is a treap, each node has a tuple $(k, p)$ where $k \in \Sigma$ for
$\Sigma = \{A,B,\ldots, Y,Z\}$ with alphabetical sorting and
$p \in \R$ where, for $q_1, q_2 \in \R$, $q_1$ is a higher priority than
$q_2$ if $q_1 < q_2$.

\begin{center}
	\begin{forest}
		[{$(M,1)$}
			[{$(H,2)$}
				[{$(G,7)$}
					[{$(A,9)$}]
				]
				[{$(I,4)$}
					[{$(L,8)$}]
				]
			]
			[{$(T,3)$}
				[{$(R,5)$}
					[{$(O,6)$}]
				]
			]
		]
	\end{forest}
\end{center}

Now we will move onto some analysis, and to make this easier will make
the assumption that all the key and priorities are distinct.

We claim: the structure of a treap is completely determined by the search
keys and the priorities. 
We can prove this by induction by examining both the heap property
and the binary search tree's properties.

We can construct a treap as follows (an alternative definition):
insert the nodes sequentially into an initially empty binary search tree,
in order of increasing priority
(using the usual binary search tree insertion algorithm).
The proof of this is also easy to prove (by induction).

Treap operations are as follows.
\begin{enumerate}
	\item Search: exactly as in a standard binary search tree.
	For a key $k$ of a node $v$ with depth $d$, 
	a \emph{successful search} takes
	$O(d)$ time. Let $v^+$ and $v^-$ be the respectively inorder successor 
	and predecessor of $v$ with respective depths $d^+$ and $d^-$.
	A \emph{unsuccessful search} takes either $O(d^+)$ or $O(d^-)$ time.
	\item Insert: first use the standard binary search tree algorithm,
	then we fix the heap property by performing rotations.
	Using the notation above, an insertion takes
	either $O(d^+)$ or $O(d^-)$ time.
	\item Delete: we rotate the node with the child of higher priority
	until it is a leaf.
	Once it is a leaf, we chop it off.
	As deletion is insertion in reverse, it takes the same time.
	\item Split: that is, split a treap $T$ into two treaps $T_<$ and
	$T_<$ along some pivot key $\pi$
	where all the nodes in $T_<$ have keys smaller than $\pi$
	and similarly $T_>$ have keys bigger than $\pi$.
	One way to do this: insert a node $v$ with $k_v = \pi$ and
	$p_v = -\infty$.
	Now $v$ is the root of the treap, and the left subtree and
	right subtree and the treaps we need.
	Again, splitting takes either $O(d^+)$ or $O(d^-)$ time.
	\item Merge: splitting in reverse, create a dummy root
	with the left subtree being $T_<$ and the right subtree
	being $T_>$. We then delete the dummy root as described above.
	As merging is splitting in reverse it takes the same time.
\end{enumerate}

With way we have constructed treaps thus far, 
the depth of a node in a treap is $\Theta(n)$ in the worst case.
This is bad.
We can reduce this by randomising the priorities in a list;
we only care about the keys in our binary search tree and having
fast operations.
This leads us to getting an expected logarithmic depth to our tree.

\emph{Randomised treap}: priorities are independently and
uniformly distributed continuous random variables.
Whenever we insert a new key into the treap, we generate a random real
number, for example, in $(0,1)$
(in practise we could just choose an integer in a large range,
we just want to avoid two elements having the same priority).

As we have shown that the time taken for the operations defined above
are proportional to the depth of the node, and that the expected
depth of a given node in a randomised treap is logarithmic (that is,
for $n$ nodes and a given node $v$ with depth $d$, $E(d) = O(\log n)$),
then we can perform every operation in $O(\log n)$ time.

\end{document}