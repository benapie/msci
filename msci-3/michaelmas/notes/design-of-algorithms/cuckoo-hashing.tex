\documentclass[a4paper]{article}
%%%%%%%%%%%%%%%%%%%%%%%%%%%%%%%%%%%%%%%%%%%%%%%%%%
%%%%%%%%%%%%%%%%=%=%=%=%==%=%=%=%=%%%%%%%%%%%%%%%%
%%%%%%%%%=%=%=%=%=%============%=%=%=%=%=%%%%%%%%%
%%%%%%=%=%=%==========================%=%=%=%%%%%%
%%%/         2020 MICHAELMAS PREAMBLE         \%%%
%%%\                Ben Napier                /%%%
%%%%%%=%=%=%==========================%=%=%=%%%%%%
%%%%%%%%%=%=%=%=%=%============%=%=%=%=%=%%%%%%%%%
%%%%%%%%%%%%%%%%=%=%=%=%==%=%=%=%=%%%%%%%%%%%%%%%%
%%%%%%%%%%%%%%%%%%%%%%%%%%%%%%%%%%%%%%%%%%%%%%%%%%
% tikzcd.yichuanshen.de/ tikcd diagrams
%======================%
%   Standard packages  %
%======================%
\usepackage[utf8]{inputenc}
\usepackage[T1]{fontenc}
\usepackage{lmodern}
\usepackage[UKenglish]{babel}
\usepackage{enumitem}
\usepackage{tasks}
\usepackage{graphicx}
\setlist[enumerate,1]{
	label={(\roman*)}
}
\usepackage{parskip}

%======================%
%        Maths         %
%======================%
\usepackage{amsfonts, mathtools, amsthm, amssymb}
\usepackage{xfrac}
\usepackage{bm}
\newcommand\N{\ensuremath{\mathbb{N}}}
\newcommand\R{\ensuremath{\mathbb{R}}}
\newcommand\Z{\ensuremath{\mathbb{Z}}}
\newcommand\Q{\ensuremath{\mathbb{Q}}}
\newcommand\C{\ensuremath{\mathbb{C}}}
\newcommand\F{\ensuremath{\mathbb{F}}}
\newcommand{\abs}[1]{\ensuremath{\left\lvert #1 \right\rvert}}
\newcommand\given[1][]{\:#1\vert\:}
\newcommand\restr[2]{{% we make the whole thing an ordinary symbol
	\left.\kern-\nulldelimiterspace % automatically resize the bar with \right
	#1 % the function
	\vphantom{\big|} % pretend it's a little taller at normal size
	\right|_{#2} % this is the delimiter
}}
\usepackage{tikz-cd}
\usepackage{adjustbox}
\DeclareMathOperator{\norm}{N}
\DeclareMathOperator{\trace}{Tr}
\DeclareMathOperator*{\argmax}{arg\,max}
\DeclareMathOperator*{\argmin}{arg\,min}

%======================%
%       CompSci        %
%======================%
\usepackage{forest}
\usepackage{textgreek}

%======================%
%    Module Specific   %
%======================%

%======================%
%       Drawing        %
%======================%
%\usepackage{forest}
%\usepackage{tikz, pgfplots, pgf}
%\pgfplotsset{compat=1.16}
%\usetikzlibrary{positioning}

%======================%
%     Environments     %
%======================%
\theoremstyle{definiton}
\newtheorem*{examplex}{Example}
\newenvironment{example}[1][]
    {\begin{examplex}[#1]}
    {\renewcommand{\qedsymbol}{\footnotesize $\triangle$}\qed\end{examplex}}

%======================%
%    Pretty tables     %
%======================%
\usepackage{booktabs}
\usepackage{caption}
\captionsetup[table]{skip=10pt}

%======================%
%    Page Formatting   %
%======================%
\usepackage{fancyhdr}
\usepackage{titling}
\usepackage{titlesec}
% \pagestyle{fancy}
\setlength{\droptitle}{-8em}
\renewcommand{\maketitlehooka}{\textsc}

\titleformat{\section}{\large\scshape}{}{0pt}{}
\titleformat{\paragraph}{\scshape}{}{0pt}{}
\renewcommand{\footrulewidth}{0.5pt}
\date{2020-2021}
\author{Ben Napier}

%======================%
%   Header and Footer  %
%======================%
\usepackage{hyperref}
\hypersetup{
    colorlinks,
    citecolor=black,
    filecolor=black,
    linkcolor=black,
    urlcolor=black,
    pageanchor=false
}
\begin{document}

\title{Cuckoo Hashing}
\maketitle

We may use linked lists to inefficiently implement a \emph{dictionary}.
A less inefficient implementation may use balanced search trees.
Here, we will introduce two implementations based on \emph{hashing}.
We will use the following assumptions:
\begin{enumerate}
	\item all stored items have the same size and can be compared in constant
	time;
	\item the set of items to store $S$ is finite (say, bounded by $n$); and
	\item we have access to two uniform hash functions
	$h_1, h_2: S \to \{1, \ldots r\}$ for some $r \geq n$.
\end{enumerate}

\section{Hashing with chaining}

Idea: determine the location an item is stored using a hash function.

\begin{center}
	\begin{tikzcd}[column sep=1.4em,row sep=1.4em]
		\fbox{1} \arrow[d] & \fbox{2} & \fbox{3} \arrow[d] & \fbox{4} \arrow[d] 
		& \fbox{5} & \fbox{6} \arrow[d] & \fbox{7} & \fbox{8} \arrow[d] \\
		\fbox A \arrow[d] & & \fbox P & \fbox J \arrow[d] & & \fbox Q & 
		& \fbox D \arrow[d] \\
		\fbox H & & & \fbox B \arrow[d] & & & & \fbox W \\
 		& & & \fbox M & & & &            
	\end{tikzcd}
\end{center}

We store our data in an array of size $r$.
At each data location $a \in \{1, \ldots, r\}$ in our array, 
we store a data structure holding the \emph{bucket}
$S_a = \{x \in S: h_1(x) = a\}$.
The efficiency of this data structure is not relevant (as collision aren't
likely), so we use a linked list.

From our assumptions of our hash functions, we see that any operation on an
item $x \in S$ depends on the time to complete the operation on $S_a$
(as a linked list).
Now, the expected number of items in a given bucket is $n \times \frac 1r$; thus, all our operations are constant.

\section{Cuckoo hashing}

In Cuckoo hashing, each item has two possible position that it may
be stored in, corresponding to the values of $h_1$ and $h_2$.
When inserting an item $x \in S$, if $h_1(x)$ is occupied, say by $y \in S$,
then $y$ is removed from position $h_1(x)$ and $x$ is inserted,
then $y$ is inserted into the the alternative position of $y$ (which may not be $h_1(y)$ if $y$ has already been moved) using the same insertion protocol. 
This is repeated until a vacant position is found, or a loop is discovered.
In the latter case, we rehash the data structure and reinsert.

Follow is what we call a \emph{cuckoo graph}:
it shows where values are stored and where their alternative positions are.

\begin{center}
	\begin{tikzcd}[column sep=1em,row sep=1em]
		\fbox A \arrow[rrr, bend left] & \fbox C \arrow[r, bend right] 
		& \fbox{$\cdot$} & \fbox B \arrow[rrrrr, bend right] 
		& \fbox H \arrow[rrr, bend right] & \fbox P \arrow[rrr, bend left] 
		& \fbox{$\cdot$} & \fbox W \arrow[lll, bend right] & \fbox{$\cdot$} 
		& \fbox{$\cdot$}
	\end{tikzcd}
\end{center}

It can be shown that the expected insertion time (in absence of a rehash)
is constant.
We can see that rehashing is expensive, but it is not an issue as
it occurs infrequently; that is, for $\Theta(n)$ elements, the
expected number of rehashes is $O(1)$.
We claim the time for a rehash is $O(n)$, so the expected time for all
rehashes is $O(n)$.
Thus, the \emph{amortised} cost of rehashing is constant.

Infact, we can show:
\begin{enumerate}
	\item look-up runs in $O(1)$ worst-case;
	\item deletion runs in $O(1)$ worst-case; and
	\item insertion runs in $O(1)$ expected (amortised).
\end{enumerate}

The applicablility of cuckoo hashing disappears for situations where
having to rehash at a random time becomes infeasible. 
We can make Cuckoo hashing more reliable by using using a part of memory
of size $m \in \N$.
If we obtain an element that cannot be added to the list (without causing
a rehash) then we put it in the stash.
This infact lowers the \emph{rehash rate} from $O\left( \sfrac1n \right)$
to $O\left( \sfrac1{n^{m+1}} \right)$.


\end{document}