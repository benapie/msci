\documentclass[a4paper]{article}
%%%%%%%%%%%%%%%%%%%%%%%%%%%%%%%%%%%%%%%%%%%%%%%%%%
%%%%%%%%%%%%%%%%=%=%=%=%==%=%=%=%=%%%%%%%%%%%%%%%%
%%%%%%%%%=%=%=%=%=%============%=%=%=%=%=%%%%%%%%%
%%%%%%=%=%=%==========================%=%=%=%%%%%%
%%%/         2020 MICHAELMAS PREAMBLE         \%%%
%%%\                Ben Napier                /%%%
%%%%%%=%=%=%==========================%=%=%=%%%%%%
%%%%%%%%%=%=%=%=%=%============%=%=%=%=%=%%%%%%%%%
%%%%%%%%%%%%%%%%=%=%=%=%==%=%=%=%=%%%%%%%%%%%%%%%%
%%%%%%%%%%%%%%%%%%%%%%%%%%%%%%%%%%%%%%%%%%%%%%%%%%
% tikzcd.yichuanshen.de/ tikcd diagrams
%======================%
%   Standard packages  %
%======================%
\usepackage[utf8]{inputenc}
\usepackage[T1]{fontenc}
\usepackage{lmodern}
\usepackage[UKenglish]{babel}
\usepackage{enumitem}
\usepackage{tasks}
\usepackage{graphicx}
\setlist[enumerate,1]{
	label={(\roman*)}
}
\usepackage{parskip}

%======================%
%        Maths         %
%======================%
\usepackage{amsfonts, mathtools, amsthm, amssymb}
\usepackage{xfrac}
\usepackage{bm}
\newcommand\N{\ensuremath{\mathbb{N}}}
\newcommand\R{\ensuremath{\mathbb{R}}}
\newcommand\Z{\ensuremath{\mathbb{Z}}}
\newcommand\Q{\ensuremath{\mathbb{Q}}}
\newcommand\C{\ensuremath{\mathbb{C}}}
\newcommand\F{\ensuremath{\mathbb{F}}}
\newcommand{\abs}[1]{\ensuremath{\left\lvert #1 \right\rvert}}
\newcommand\given[1][]{\:#1\vert\:}
\newcommand\restr[2]{{% we make the whole thing an ordinary symbol
	\left.\kern-\nulldelimiterspace % automatically resize the bar with \right
	#1 % the function
	\vphantom{\big|} % pretend it's a little taller at normal size
	\right|_{#2} % this is the delimiter
}}
\usepackage{tikz-cd}
\usepackage{adjustbox}
\DeclareMathOperator{\norm}{N}
\DeclareMathOperator{\trace}{Tr}
\DeclareMathOperator*{\argmax}{arg\,max}
\DeclareMathOperator*{\argmin}{arg\,min}

%======================%
%       CompSci        %
%======================%
\usepackage{forest}
\usepackage{textgreek}

%======================%
%    Module Specific   %
%======================%

%======================%
%       Drawing        %
%======================%
%\usepackage{forest}
%\usepackage{tikz, pgfplots, pgf}
%\pgfplotsset{compat=1.16}
%\usetikzlibrary{positioning}

%======================%
%     Environments     %
%======================%
\theoremstyle{definiton}
\newtheorem*{examplex}{Example}
\newenvironment{example}[1][]
    {\begin{examplex}[#1]}
    {\renewcommand{\qedsymbol}{\footnotesize $\triangle$}\qed\end{examplex}}

%======================%
%    Pretty tables     %
%======================%
\usepackage{booktabs}
\usepackage{caption}
\captionsetup[table]{skip=10pt}

%======================%
%    Page Formatting   %
%======================%
\usepackage{fancyhdr}
\usepackage{titling}
\usepackage{titlesec}
% \pagestyle{fancy}
\setlength{\droptitle}{-8em}
\renewcommand{\maketitlehooka}{\textsc}

\titleformat{\section}{\large\scshape}{}{0pt}{}
\titleformat{\paragraph}{\scshape}{}{0pt}{}
\renewcommand{\footrulewidth}{0.5pt}
\date{2020-2021}
\author{Ben Napier}

%======================%
%   Header and Footer  %
%======================%
\usepackage{hyperref}
\hypersetup{
    colorlinks,
    citecolor=black,
    filecolor=black,
    linkcolor=black,
    urlcolor=black,
    pageanchor=false
}
\begin{document}

\title{Universal hashing}
\maketitle

We have thus far assumed that our hash function are uniformly random,
but obtaining such a hash function in reality is not possible.
We introduce a weaker notion, a simple class of has functions that gives us 
guaranteed performance.

\section{Universal families}

Let $S$ be a set of items, and $n \in \N$ such that $\abs S \geq n$.
We say a collection of (hash) functions
$\mathcal H = \left\{ h_i \right\}_{i \in I}$
is \emph{$k$-universal} if for every distinct 
$x_1, \ldots, x_k \in S$
we have
\[
	\Pr_{h \in \mathcal H}
	\left( 
		h(x_1) = h(x_2) = \ldots = h(x_k)
	 \right) \leq \frac{1}{n^{k-1}}.
\]
We say that $\mathcal H$ is \emph{strongly $k$-universal} if for every
distinct $x_1, \ldots, x_k \in S$ and any 
$y_1 \ldots, y_k \in \left\{ 0, \ldots, n-1 \right\}$ we have
\[
	\Pr_{h \in \mathcal H} \left( 
		\bigcap_{i=1}^k \left( h(x_i) = y_i \right)
	 \right)
	 = \frac1{n^k}.
\]
Hash functions from a strongly $2$-universal family are
\emph{pairwise independent}.
The notion of strongly $k$-universal is a stronger notion than $k$-universal;
that is, if a family of (hash) functions is strongly $2$-universal,
then it is also $k$-universal.

\begin{example}
	Let $S = (x_1, \ldots, x_m)$ be a collection of items, and 
	$h: S \to \left\{ 0, \ldots, n-1 \right\}$ be a hash function from a
	$2$-universal family where $m \leq n \in \N$.
	Fix $x \in S$ and let 
	$X = \abs{\left\{ y \in S: h(y) = h(x) \right\}}$
	(the number of items in bucket $h(x)$).
	We can compute the bound
	\[\Pr\left( X \geq \frac{m^2}{n} \right) \leq \frac{1}{2}\]
	by using Markov's inequality.
	Let $Y$ be the maximum number of balls in $X$.
	Then the bound
	\[\Pr\left( X \geq \frac{m^2}{n} \right) \leq \frac{1}{2}\]
	concludes this example.
\end{example}

\begin{example}
	Let $n, m \in \N$, with $m \geq n$,
	$U = \left\{ 0, \ldots, m-1 \right\}$, and
	$V = \left\{ 0, \ldots, n-1 \right\}$.
	Let $p$ be some prime such that $p \geq m$.
	We define
	\begin{align*}
		h_{a,b}(x) &= \left( (ax + b) \bmod p \right) \bmod n \\
		\mathcal H &= \left\{ 
			h_{a,b}: 
			a \in \left\{ 0, \ldots, p-1 \right\},
			b \in \left\{ 0, \ldots, p-1 \right\}
		\right\}.
	\end{align*}
	It can be shown that $\mathcal H$ is $2$-universal.
\end{example}

\begin{example}[Non-example]
	Let $p \in \Z$ be prime, and $n \in \N$ such that $n \leq p$.
	Define
	\begin{align*}
		h_a &: \left\{ 0, \ldots, p-1 \right\} \to \left\{ 0, \ldots n-1 \right\} \\
		x &\mapsto \left( ax \bmod p \right) \bmod n
	\end{align*}
	and consider the family
	\[
		\mathcal H = \left\{ h_a : a \in \left\{ 1, \ldots, p-1 \right\} \right\}.
	\]
	It can be shown that $\mathcal H$ is \emph{not} $2$-universal.
\end{example}

\section{Perfect hashing}

\emph{Perfect hashing} is an efficient data structure for storing
static dictionaries.
Once data is stored, it is only used for search operations.

Motivation: suppose that a set $S$ of $m$ items is hashed
into a table of $n$ bins, using a hash function $h$ from a
$2$-universal family and utilising chain hashing (with linked lists).
It can be shown that
\[
	E\left( X \right) \leq \begin{cases}
		\frac mn & x \not\in S, \\
		1 + \frac{m-1}{n} & x \in S.
	\end{cases}
\]
If $m = n$, $E(X) = 1$.
But this does not give us an upper bound on worst-case performance.
Some bins can have more than $\sqrt n$ items.

An injective function $h: S \to K \subset \Z$ for a collection of items
$S$ is called a \emph{perfect hash function}.

The probability that a hash function 
$h: S \to \left\{ 0, \ldots, n-1 \right\}$ chosen uniformly at random from
a $2$-universal family is perfect is at least $\frac12$, 
given that $n \geq \abs S^2$ (by Markov's inequality).
This would require $\Omega(\abs S^2)$ buckets in order to get a
$50\%$ chance of getting a perfect hash function, rather wasteful.

We now introduce a two-level scheme: for a $2$-universal family $\mathcal H$,
we use $h \in \mathcal H$ to hash our collection $S$
into $m$ bins. 
For each bin with a collision, we use a different hash function
$g \in \mathcal H$ to construct a hash table with $k^2$ bins,
where $k$ is the number of elements in the bin.
This scheme can be shown to use $O(m)$ bins.

\end{document}