%! TEX root = master.tex
\lecture{6}{23/10}

\begin{proposition}[]
	Let
	$ \left\{ F_n : n \in \N \right\} $
	be a collection of non-empty nested closed sets, one of which is bounded.
	Then
	\[
		\bigcap_{n \in \N} F_n \neq \varnothing.
	\]
\end{proposition}

\begin{proof}
	By contradiction, assume that 
	$\bigcap_{n \in \N} F_n = \varnothing$.
	Then $\left( \bigcap_{n \in \N} F_N \right)^c = \R$
	and so $\bigcup_{n \in \N} F_n^c = \R$.
	Assume, without loss of generality, that $F_1$ is bounded.
	As $F_1$ is also closed, then
	\[
		F_1 \subset \bigcup_{i=1}^m F_{n_i}^c = F_{n_m}^c.
	\]
	and so $F_1 \cap F_{n_m} = \varnothing$;
	a contradiction to our nested property.
\end{proof}

\subsection{The Cantor set}

\begin{definition}[Accumulation point]
	A point $x \in \R$ is called an \emph{accumation point} 
	of a set $E \subset \R$ if $x$ is a point of closure
	of $E \setminus \left\{ x \right\}$.
	The set of accumation points of $E$ is denoted $E'$.
\end{definition}

\begin{proposition}[]
	Let $E \subset \R$.
	Then $E'$ is closed and $\overline E = E \cup E'$.
\end{proposition}

\begin{proof}
	Let $x \in \overline{E'}$ and $\varepsilon > 0$.
	Then there is $y \in E'$ such that
	$\left\lvert x - y \right\rvert < \frac12\varepsilon$.
	Now as $y$ is an accumulation point of $E$ there is
	$z \in E \setminus \left\{ y \right\}$ such that
	$\left\lvert y - z \right\rvert < \frac{1}{2} \varepsilon$.
	Thus
	\[
		\left\lvert x - z \right\rvert
		\leq \left\lvert x - y \right\rvert
		+ \left\lvert y - z \right\rvert
		< \varepsilon
		\implies x \in E'.
	\]

	Now we show that $\overline E = E \cup E'$.
	Let $x \in E'$.
	The for all $\delta > 0$ there is 
	$y \in E \setminus \left\{ x \right\} \subset E$
	such that
	$\left\lvert x - y \right\rvert < \delta$.
	This implies that $x \in \overline E$.
	Therefore $E' \subset \overline E$.
	So $E \cup E' \subset \overline E$,
	we have left to show that $\overline E \subset E \cup E'$.
	Take $x \in \overline E$.
	If $x \in E$, we are done.
	Now assume $x \not\in E$.
	Hence for every $\delta > 0$ there is 
	$y \in E \setminus \left\{ x \right\}$ such that
	$\left\lvert x - y \right\rvert < \delta$;
	the definition of an accumulation point!
\end{proof}

\begin{definition}[The Cantor set]
	The Cantor set $\mathcal C$ can be created by iteratively
	deleting the middle third from a set of line segments.
	One starts by deleting the middle third
	$\left( \frac13, \frac23 \right)$, leaving
	$\left[ 0, \frac13 \right] \cup \left[\frac23, 1\right]$.
	Then the open middle third of each of these remaining segments is deleted.
	This process is continued ad infinitum.
\end{definition}

\begin{theorem}[]
	The Cantor set $\mathcal C$ satisfies the following:
	\begin{enumerate}
		\item $\mathcal C$ is closed and non-empty;
		\item $C' = C$;
		\item there is a function $f$ from $\mathcal C$
			to $[0,1]$ that is surjective; and
		\item $\mathcal C$ may be covered by unions of intervals of 
			arbitrarily small total length.
	\end{enumerate}
\end{theorem}

\section{Continuous functions of $\R$}
\subsection{Continuity}

\begin{definition}[Continuous function]
	Suppose $E \subset \R$.
	We say that a function $f: E \to \R$ is \emph{continuous}
	at $e \in E$ if
	for every $\varepsilon > 0$ there is $\delta > 0$
	such that
	\[
		\left\lvert x - e \right\rvert < \delta
		\implies \left\lvert f(x) - f(e) \right\rvert < \varepsilon
	\]
	for every $x \in E$.
	$f$ is said to be \emph{continuous} if for every $e \in E$,
	$f$ is continuous at $e$.
\end{definition}

\begin{definition}[Limit of a function]
	Let $E \subset \R$, $f: E \to \R$, and $e \in E$.
	We write $\lim_{x \to e} f(x) = l$
	if for every $\varepsilon > 0$
	there is $\delta > 0$ such that
	\[
		\left\lvert x - e \right\rvert < \delta
		\implies \left\lvert f(x) - l \right\rvert < \varepsilon.
	\]
\end{definition}

\begin{proposition}[]
	Let $E \subset \R$, $f: E \to \R$ and $e \in E$.
	Then
	\[
		l = \lim_{x \to e} f(x)
	\]
	if and only if for every
	$ \left\{ a_n \right\}_{n \in \N} $
	with $a_n \to e$, we have $f(a_n) \to l$.
\end{proposition}

\begin{proof}
	$(\implies)$: let $\varepsilon > 0$. 
	Then there exists $\delta > 0$ such that
	$
		\left\lvert x - e \right\rvert < \delta
		\implies \left\lvert f(x) - l \right\rvert < \varepsilon
	$. Since $a_n \to e$, there is $N \in \N$ such that for every $n \geq N$,
	$\left\lvert a_n - e \right\rvert < \delta$.
	So $\left\lvert f(a_n) - l \right\rvert < \varepsilon$.
	
	$(\impliedby)$: let $\varepsilon > 0$.
	Then by definition there is $\delta > 0$
	and $N \in \N$ such that for every $n \geq N$
	with $\left\lvert a_n - e \right\rvert < \delta$,
	then $\left\lvert f(a_n) - l \right\rvert < \varepsilon$.
	Therefore, for all $x \in E$ with $\left\lvert x - e \right\rvert < \delta$
	we have $\left\lvert f(x) - l \right\rvert < \varepsilon$.
\end{proof}

A function $f: E \subset \R \to \R$ is continuous at $e \in E$
if $\lim_{x \to e} f(x) = f(e)$.
