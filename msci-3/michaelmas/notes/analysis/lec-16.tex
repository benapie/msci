%! TEX root = master.tex
\lecture{16}{28/11}

\begin{definition}[Lower and upper Lebesgue integral]
	Let $f: E \to \R$ be a bounded function with $m(E) < \infty$.
	We define the \emph{lower Lebesgue integral} as
	\[
		\sup \left\{
			\int_E \varphi: 
			\text{$\varphi$ simple},\,
			\varphi \leq f \;\text{on $E$}
		\right\}
	\]
	and the \emph{upper Lebesgue integral} as
	\[
		\inf \left\{
			\int_E \varphi:
			\text{$\varphi$ simple},\,
			f \leq \varphi \;\text{on $E$}
		\right\}.
	\]
\end{definition}

As $f$ is bounded and by the monotonicty of the Lebesgue integral of
simple functions, the lower and upper Lebesgue integral is well-defined.

\begin{definition}[Lebesgue integrable]
	Let $f: E \to \R$ be a bounded function with $m(E) < \infty$.
	We say that \emph{$f$ is Lebesgue integrable} if the upper and lower
	Lebesgue integrals are equal.
	This common value is called the \emph{Lebesgue integral}
	(or just \emph{the integral of $f$ over $E$}), denoted
	$\int_E f$.
\end{definition}

\begin{theorem}[]
	Let $f: E \to \infty$ be a bounded, measurable function and
	$m(E) < \infty$.
	Then $f$ is integrable over $E$.
\end{theorem}

\begin{proof}
	By the Simple Approximation Lemma, setting $\varepsilon = \frac1n$ for some
	$n \in \N$, there is simple functions $\varphi_n$ and $\psi_n$ such that
	$0 \leq \varphi_n \leq \psi_n$ and $0 \leq \psi_n - \varphi_n \leq \frac1n$
	on $E$.
	By the monotonicty of $\int_E \varphi$,
	\[
		0 
		\leq \int_E \varphi_n - \int_E \psi_n
		\leq \int_E (\varphi_n - \psi_n)
		\leq \int_E
		\frac1n m(E)
	\]
	and
	\begin{align*}
		0 &\leq
		\inf\left( 
			\left\{
				\int_E \psi: \psi \geq n
			\right\} 
		\right) - \sup\left( 
			\left\{
				\int_E \varphi: \varphi \leq f
			\right\} 
		\right) \\
		&\leq \int_E \psi_n - \int_E \varphi_n \\
		&\leq \frac1n m(E).
	\end{align*}
	Since $m(E) < \infty$,
	\[
		\inf\left\{
			\int_E \psi: \psi \geq f
		\right\}
		= \sup\left\{
			\int_E \varphi: \varphi \leq f
		\right\}
	\]
	and thus $f$ is integrable.
\end{proof}

\begin{theorem}[Linearity and monoticity of the integral of bounded,
	measurable functions on a finite domain]
	Let $f, g : E \to \R$ be bounded, measurable functions with $m(E) < \infty$.
	For every $\alpha, \beta \in \R$
	\[
		\int_E \alpha f + \beta g = \alpha \int_E f + \beta \int_E g
	\]
	and
	\[
		f \leq g \implies \int_E f \leq \int_E g.
	\]
\end{theorem}

\begin{proof}
	A linear combination of bounded and measurable functions is bounded and
	measurable,
	so $\int_E \alpha f + \beta g$ is well-defined and exists.
	First let $\alpha \neq 0 = \beta$.
	We claim that if $\psi$ is simple, $\alpha \psi$ is also simple.
	Now let $\alpha > 0$.
	Since the lower Lebesgue integral is equal to the upper Lebesgue integral,
	\[
		\int_E \alpha f
		= \inf_{\psi \geq \alpha f} \int_E \psi
		= \inf_{\sfrac{\psi}{\alpha} \geq f}\int_E\frac{\alpha \psi}{\alpha}
		= \alpha \inf_{\sfrac{\psi}{\alpha}} \int_E \frac{\psi}{\alpha}
		= \alpha \int_E f.
	\]
	Now if $\alpha < 0$,
	\[
		\int_E \alpha f
		= \inf_{\psi \geq \alpha f} \int_E \psi
		= \sup_{\sfrac{\psi}{\alpha} \leq f} \int_E \frac{\alpha \psi}{\alpha}
		= \alpha \sup_{\sfrac{\psi}{\alpha} \leq f} \frac{\psi}{\alpha}
		= \alpha \int_E f.
	\]
	Now it is enough to prove linearity in the case that $\alpha = \beta = 1$.
	Now let $\psi_1$ and $\psi_2$ be simple with $f \leq \psi_1$ and
	$g \leq \psi_2$.
	Then $\psi_1 + \psi_2$ is also simple.
	Observe that $f + g \leq \psi_1 + \psi_2$.
	Then
	\[
		\int_E f + g
		= \inf\left\{
			\int_E \psi: f + g \leq \psi
		\right\}
	\]
	and so
	\[
		\int_E f + g
		\leq \int_E(\psi_1 + \psi_2)
		= \int_E \psi_1 + \int_E \psi_2.
	\]
	So $\int_E f + g$ is a lower bound for
	$\int_E \psi_1 + \int_E \psi_2$.
	But $\int_E f + \int_E g$ is the largest lower bound for
	$\int_E \psi_1 + \int_E \psi_2$.
	So
	\[
		\int_E f + g \leq \int_E f + \int_E g.
	\]
	Let $\varphi_1$ and $\varphi_2$ be simple functions with
	$\varphi_1 \leq f$ and $\varphi_2 \leq g$.
	Then $\varphi_1 + \varphi_2 \leq f + g$ is simple.
	Observe that
	\[
		\int_E f + g
		= \sup\left\{
			\int_E \varphi: \varphi \leq f + g
		\right\}
		\geq \int_E \varphi_1 + \varphi_2
		= \int_E \varphi_1 + \int_E \varphi_2
	\]
	but $\int_E f + \int_E g$ is the smallest upper bound for
	$\int_E \varphi_1 + \int_E \varphi_2$.
	So $\int_E f + g \geq \int_E f + \int_E g$ and thus
	$\int_E f + g = \int_E f + \int_E g$.
	So now we have left to prove monotonicty.
	Let $h = g - f \geq 0$.
	Then
	\[
		\int_E g - \int_E f = \int_E g - f = \int_E h.
	\]
	Take $\psi = 0$ on $E$ an observe that $\psi \leq h$.
	Then
	\[
		0 = \int_E \psi \leq \int_E h
	\]
	and so $\int_E g - f \geq 0$.
	Hence $\int_E g - \int_E f \geq 0$.
\end{proof}

\begin{corollary}[]
	Let $f: E \to \R$ be a bounded, measurable function with $m(E) < \infty$.
	Then
	\[
		\left\lvert \int_E f \right\rvert 
		\leq \int_E \left\lvert f \right\rvert.
	\]
\end{corollary}

\begin{proof}
	$\left\lvert f \right\rvert$ is measurable and bounded.
	As $-\left\lvert f \right\rvert \leq f \leq \left\lvert f \right\rvert$
	and by the linearity and monotonicty on the Lebesgue integral
	\[
		-\int_E \left\lvert f \right\rvert
		\leq \int_E f
		\leq \int_E \left\lvert f \right\rvert
	\]
	and thus
	\[
		\left\lvert \int_E f \right\rvert 
		\leq \int_E \left\lvert f \right\rvert.
	\]
\end{proof}

\begin{theorem}[Bounded convergence theorem]
	Let
	$
		\left\{
			f_n: E \to \R
		\right\}
	$
	be a sequence of measurable functions with $m(E) < \infty$
	which is uniformly pointwise bounded
	(that is, there is $M \in \R$ such that 
	$\left\lvert f_n \right\rvert \leq M$ for each $n$).
	Then if $f_n \to f$ pointiwse on $E$ then
	\[
		\lim_{n \to \infty} \int_E f_n = \int_E f.
	\]
\end{theorem}

\begin{definition}[Vanish]
	A measurable function $f: E \to \R$ is said to
	\emph{vanish outside a set of finite measure}
	if there is $E_0 \subset E$ with $m(E_0) < \infty$ such that
	$f = 0$ on $E \setminus E_0$.
\end{definition}

It is convenient to say that a function that vanishes outside a set of
finite measure has \emph{finite support}.
We define the \emph{support} of a function $f: E \to \R$ as
\[
	\left\{
		x \in E: f(x) \neq 0
	\right\}.
\]
If $m(E) = \infty$, $f$ is bounded, and measurable on $E$
but has finite support then we can define its integral as
\[
	\int_E f = \int_{E_0} f
\]
where $m(E_0) < \infty$ and $f = 0$ on $E \setminus E_0$.

\begin{definition}[]
	Let $f: E \to \R$ be a non-negative measurable function.
	We define the \emph{integral of $f$ over E}
	by
	\[
		\int_E f
		= \sup\left\{
			\int_E h:
			\begin{array}{l}
				\text{$h$ bounded}, \\
				\text{$h$ measurable}, \\
				\text{$h$ of finite support}, \\
				0 \leq h \leq f
			\end{array}
		\right\}.
	\]
\end{definition}

