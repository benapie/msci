%! TEX root = master.tex
\lecture{19}{8/12}

\begin{corollary}[Additivity of domains of integration]
	Let $f$ be integrable over $E$.
	Assume $A$ and $B$ are disjoint measurable subsets of $E$.
	Then
	\[\int_{A \cup B} f = \int_A f + \int_B f.\]
\end{corollary}

\begin{proof}
	Observe $\abs{f \cdot \chi_A} \leq \abs f$
	and $\abs{f \cdot \chi_B} \leq \abs f$
	thus $\int_E \abs{f \cdot \chi_A} \leq \int_E \abs f < \infty$
	and $\int_E \abs{f \cdot \chi_B} \leq \int_E \abs f < \infty$.
	Therefore, $f \cdot \chi_A$ and $f \cdot \chi_B$ are integrable.
	Since $A$ and $B$ are disjoint,
	$f \cdot \chi_{A \cup B} = f \cdot \chi_A + f \cdot \chi_B$
	(convince yourself).
	So
	\[
		\int_E f \cdot \chi_{A \cup B}
		= \int_E \left(f \cdot \chi_A + f \cdot \chi_B\right)
		= \int_E f \cdot \chi_A + \int_E f \cdot \chi_B.
	\]
	Now convince yourself that if $C \subset E$, 
	then $\int_C f = \int_E f \cdot \chi_C$.
	But now we are done.
\end{proof}

\begin{theorem}[Lebesgue dominated convergence theorem]
	Let $\{f_n\}$ be a sequence of measurable functions over $E$.
	Suppose there is an integrable function $g$ on $E$
	which dominates $f_n$ on $E$
	(that is, $\abs{f_n} \leq g$).
	If $f_n \to f$ pointwise almost everywhere on $E$, then $f$ is integrable
	over $E$ and
	\[\lim_{n \to \infty} \int_E f_n = \int_E f.\]
\end{theorem}

\begin{proof}
	$\abs{f_n} \leq g$ for all $n$ and $\abs f \leq g$ almost everywhere on
	$E$.
	Thus
	\[
		\int_E \abs{f_n} \leq \int_E g < \infty, \qquad
		\int_E \abs f \leq \int_E g < \infty
	\]
	so $f_n$ and $f$ are integrable over $E$.
	So $f_n$ and $f$ are finite almost everywhere on $E$.
	By excising the set of measure zero for which $f_n$ and $f$
	are not finite, we get that they are finite on $E$.
	So $g - f$ and $g - f_n$ are well-defined,
	non-negative, and measurable.
	Now $f_n \to f$, so $g - f_n \to g - f$ (pointwise almost everywhere).
	By Fatou's Lemma,
	\[
		\int_E (g - f) \leq \liminf\int_E (g - f_n);
	\]
	but
	\[
		\int_E (g - f)
		= \int_E g - \int_E f
		\leq \int_E g - \limsup\int_E f_n
	\]
	and so $\limsup\int_E f_n \leq \int_E f$.
	Now consider $g + f_n \to g + f$, then similarly
	\[
		\int_E f \leq \liminf\int_E f_n. \qedhere
	\]
\end{proof}

\subsection{Convergence in measure}

\begin{definition}{Converge in measure}
	Let $\left\{ f_n \right\}$ be a sequence of measurable functions
	on $E$ and $f$ be a measurable function for which $f$ and $f_n$
	are finite almost everywhere on $E$.
	The sequence $\left\{ f_n \right\}$ is said to \emph{converge
	in measure} on $E$ to $f$ if for each $\eta > 0$,
	\[
		\lim_{n \to \infty} m\left( 
			\left\{ x \in E: \abs{f_n(x) - f(x)} > \eta \right\} 
		 \right) = 0.
	\]
\end{definition}

When we write $\left\{ f_n \right\} \to f$ in measure, we assume
$f_n$ and $f$ are finite almost everywhere. Observe:
if $f_n \to f$ uniformly then $f_n \to f$ in measure.
Indeed, if $f_n \to f$ uniformly for each $\eta > 0$ and $n \in \N$
large enough,
\[
	\left\{ x \in E: \abs{f_n(x) - f(x)} > \eta \right\} = \varnothing
\]
so $f_n \to f$ in measure.

\begin{proposition}
	Assume $E$ has finite measure.
	Let $\{f_n\}$ be a sequence of measurable functions on $E$ such that
	$f_n \to f$ pointwise almost everywhere on $E$,
	where $f$ is finite almost everywhere on $E$.
	Then $f_n \to f$ in measure.
\end{proposition}

\section{The Riemann Integral}

Let $f$ be a bounded real-valued functions defined on the closed, bounded
interval $[a,b]$.
Let $p = \left\{ x_0, x_1, \ldots, x_n \right\}$ be a partition of $[a,b]$
(that is, $a = x_0 < x_1 < \ldots < x_n = b$).
The \emph{upper and lower Darboux sums} are defined for $f$ with respect to
$p$ as
\begin{align*}
	L(f,p) &= \sum_{i=1}^n m_i\left( x_i - x_{i-1} \right) \\
	U(f,p) &= \sum_{i=1}^n M_i\left( x_i - x_{i-1} \right)
\end{align*}
where
\begin{align*}
	m_i(x_{i-1} - x_i) &= \inf\left\{ f(x): x \in (x_{i-1}, x_i) \right\} \\
	M_i(x_{i-1} - x_i) &= \sup\left\{ f(x): x \in (x_{i-1}, x_i) \right\}.
\end{align*}
We define the \emph{lower and upper Riemann integrals} of $f$ over $[a,b]$
by 
\begin{align*}
	\underline{I}_a^b(f) &= \sup\left\{ L(f, p): 
		\text{$p$ partition of $[a,b]$} \right\} \\
	\overline{I}_a^b(f) &= \inf\left\{ U(f,p): 
		\text{$p$ partition of of $[a,b]$} \right\}
\end{align*}

Observe that both of these are finite
$\underline I_a^b(f) \leq \overline I_a^b(f)$.
We say that $f$ is \emph{Riemann integrable} over $[a,b]$ if
$\underline I_a^b(f) = \overline I_a^b(f)$
and we denote the common value as $I_a^b(f)$ or 
$(R) \int_a^b f$.

\begin{definition}[Step function]
	A real-valued function $\psi$ is a \emph{step function} if there is a
	partition $p = \left\{ x_0, x_1, \-\ldots, x_m \right\}$ of $[a,b]$
	and $c_1, \ldots, c_n \in \R$ such that for 
	$i \in \left\{ 1, \ldots, n \right\}$ we have
	\[
		c_i \in (x_{i-1} x_i) \implies \psi(x) = c_i.
	\]
\end{definition}

Let $\psi$ be a step function, observe 
\[
	L(\psi, p) = \sum_{i=1}^n c_i(x_i - x_{i-1}) = U(\psi, p).
\]
Thus,
\[
	\underline I_a^b(\psi)
	= \overline I_a^b 
	= I_a^b
	= \sum_{i=1}^n c_i(x_i - x_{i-1}).
\]
So if $\psi$ is a step function, it is Riemann integrable.
So
\[
	\underline I_a^b(f)
	=\sup\left\{I_a^b(\varphi): \text{$\varphi$ is a step function},
	\text{$\varphi \leq f$ on $[a,b]$}\right\}
\]
and
\[
	\overline I_a^b(f) = 
	\inf\left\{ I_a^b(\varphi):
	\text{$\varphi$ is a step function},
	\text{$f \geq \varphi$ on $[a,b]$} \right\}.
\]

\begin{example}[Dirichlet's function]
	Define $f: [0,1] \to \R$,
	\[
		f(x) = \begin{cases}
			1 & x \in \Q, \\
			0 & x \in \R \setminus \Q. \\
		\end{cases}
	\]
	Let $p$ be any partition of $[0,1]$.
	By the density of $\Q$ and $\R \setminus \Q$,
	\[
		L(f,p) = 0, \qquad U(f,p) =1.
	\]
	Thus
	\[
		\underline I_0^1(f) = 0 < 1 = \overline I_0^1(f).
	\]
	Thus $f$ is not Riemann integrable.
	The set of rational are countable in $[0,1]$.
	Let $\left\{ q_k \right\}_{k=1}^\infty$ be an enumeration of the
	rational numbers in $[0,1]$.
	For $n \in \N$, define
	\[
		f_n(x) =
		\begin{cases}
			1 & \exists\; k \in \left\{ 1, \ldots, n \right\}: x = q_k, \\
			0 & \text{otherwise}. \\
		\end{cases}
	\]
	Each $f_n$ is a step function and thus Riemann integrable.
	Hence $\{f_n\}$ is an increasing sequence of Riemann integrable functions,
	$\abs{f_n} \leq 1$ on $[0,1]$ for all $n$, and $f_n \to f$ pointwise
	on $[0,1]$.
	But $f$ is not Riemann integrable.
\end{example}