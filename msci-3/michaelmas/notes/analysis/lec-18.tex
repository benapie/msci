%! TEX root = master.tex
\lecture{18}{6/12}

\begin{theorem}[The monotone convergence theorem]
	Let $\{f_n\}$ be an increasing sequence of non-negative
	measurable functions on $E$.
	If $\{f_n\} \to f$ pointwise almost everywhere on $E$, then
	\[
		\lim_{n \to \infty} \int_E f_n = \int_E f.
	\]
\end{theorem}

\begin{proof}
	By Fatou's Lemma,
	$\int_E f \leq \liminf \int_E f_n$.
	But for each $n$, $f_n \leq f$.
	So, by the monotonicity of integration for nonnegative measurable
	functions and as $\int_E f = \int_{E \setminus E_0} f$
	for $E_0 \subset E$, $m(E_0) = 0$, we have
	$\int_E f_n \leq \int_E f$.
	Thus $\limsup\int_E f_n \leq \int_E f$ and hence
	$\int_E f = \lim_{n \to \infty} \int_E f_n$.
\end{proof}

\begin{definition}
	A nonnegative measurable function on a measurable set $E$ is said to be
	\emph{integrable} over $E$ given
	\[
		\int_E f < \infty.
	\]
\end{definition}

\subsection{The Lebesgue integral on $R$}

Let $f$ be an extended real-valued function on$R$.
We have already defined $f^=$ and $f-$ by
\begin{align*}
	f^+(x) &= \max\{f(x), 0\} \\
	f^-(x) &= \max\{-f(x), 0\}.
\end{align*}
Both $f^+$ and $f^-$ are nonnegative functions on $E$.
Now observe that
\[
	f = f^+ - f^-, \qquad \abs f = f^+ + f^-.
\]
Observe that $f$ is measurable if and only if $f^+$ and $f^-$
are measurable.

\begin{proposition}
	Let $f$ be a measurable function on $E$.
	Then $f^+$ and $f^-$ are integrable over $E$ if and only if 
	$\abs{f^+}$ and $\abs{f^-}$ are integrable over $E$.
\end{proposition}

\begin{proof}
	$(\implies)$: assume $f^+$ and $f^-$ are integrable.
	So $\int_E f^+, \int_E f^- < \infty$.
	Thus
	\[\int_E \abs f = \int_E f^+ + f^- = \int_E f^+ + \int_E f^- < \infty.\]

	$(\impliedby)$: assume $\abs f$ is integrable, since
	$0 \leq f^+ \leq \abs f$, $0 \leq f^- \leq \abs f$ we have
	$\int_E f^= \leq \int_E \abs f < \infty$
	and
	$\int_E f^- \leq \int_E \abs f < \infty$.
\end{proof}

\begin{definition}[Integrable]
	A measurable function $f$ on $E$ is said to be
	\emph{integrable} over $E$ if $\abs f$ is integrable.
	When this is so, we define
	\[\int_E f = \int_E f^+ - \int_E f^-.\]
\end{definition}

Observe that if $f$ is nonnegative, $f = f^+$ and
$f^- = 0$, so $\int_E f = \int_E f^+$.

\begin{proposition}
	Let $f$ be integrable over $E$.
	Then $f$ is finite almost everywhere on $E$
	and $\int_E f = \int_{E \setminus E_0} f$
	if $E_0 \subset E$, $m(E_0) = 0$.
\end{proposition}

\begin{proof}
	% todo
\end{proof}

\begin{proposition}[The integral comparison test]
	Let $f$ be a measurable function on $E$.
	Suppose that there is a nonnegative function $g$ that is integrable over
	$E$ and dominates $f$ in the sense that $\abs f \leq g$ on $E$.
	Then $f$ is integrable over $E$ and
	$\abs{\int_E f} \leq \int_E \abs f$.
\end{proposition}

\begin{proof}
	Since $\abs f \leq g$, by the monotonicity of integration for nonnegative functions 
	$\int_E \abs f \leq \int_E g < \infty$
	and so $f$ is integrable. 
	Now
	\[
		\abs{\int_E f} 
		= \abs{\int_E f^+ - f^-}
		= \abs{\int_E f^+ - \int_E f^-}
		\leq \int_E f^+ + \int_E f^-
		= \int_E \abs f.
	\]
\end{proof}

Observe that for two integrable functions $f$ and $g$ over $E$,
the sum $f + g$ is \emph{not} properly defined on points in $E$ 
where $f$ and $g$ takes infinite values 
 of opposite sign.
However, by a previous proposition, if we define $A$ to be the points 
where $f$ and $g$ are finite then $m(E \setminus A) = 0$. 
Then we can consider $f + g$ on $A$ once we prove that 
$f + g$ is integrable on $A$, we define 
\[\int_E (f + g) = \int_A (f + g).\]

\begin{theorem}[Linearity and monotonicty of integration]
	Let $f, g$ be integrable over $E$. 
	Then for every $\alpha, \beta \in \R$,
	$\alpha f + \beta g$ is integrable over $E$ and 
	\[ \int_E \alpha f + \beta g 
	= \alpha \int_E f + \beta \int_E g \] 
	and
	\[f \leq g \implies \int_E f \leq \int_E g.\]
\end{theorem}

\begin{proof}
	% todo
\end{proof}