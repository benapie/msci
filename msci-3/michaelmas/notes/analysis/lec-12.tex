%! TEX root = master.tex
\section{Lebesgue measure}
\lecture{12}{19/11}

\begin{definition}[Lebesgue measure]
	The restriction of outer measure to the set of measurable sets is called the
	\emph{Lebesgue measure}.
	We denote it $m$.
\end{definition}

So for a measurable set $E$, we have
\[
	m(E) = m^\star(E).
\]

\begin{proposition}[]
	The Lebesgue measure is countably additive.
\end{proposition}

That is, for a collection
$
	\left\{
		E_k
	\right\}_{k=1}^\infty
$
of disjoint and measurable sets, then
$
	\bigcup_{k=1}^\infty E_k
$
is also measurable and
\[
	m\left( 
		\bigcup_{k=1}^\infty E_k 
	\right)
	= \sum^{\infty}_{k=1} m(E_k).
\]

\begin{proof}
	We already know that
	$
		\bigcup_{k=1}^\infty E_k
	$
	is measurable as the set of measurable sets is a $\sigma$-algebra.
	Now outer measure is subadditive, so we have left to prove
	\[
		m\left( 
			\bigcup_{k=1}^\infty E_k 
		\right) \geq \sum^{\infty}_{k=1} m(E_k). \tag{$\star$}
	\]
	Now observe that
	\[
		m\left( 
			\bigcup_{k=1}^n E_k 
		\right) = \sum^{n}_{k=1} (E_k)
	\]
	and
	$
		\bigcup_{k=1}^n E_k \subset \bigcup_{k=1}^\infty E_k 
	$.
	By the monotonicity of outer measure,
	\[
		m\left( 
			\bigcup_{k=1}^\infty E_k 
		\right) \geq \sum^{n}_{k=1} m(E_k)
	\]
	and as the LHS does not depend on $n$, we let $n \to \infty$ to get
	$(\star)$.
\end{proof}

\begin{theorem}[]
	The Lebesgue measure is translation invariant and countable additive.
\end{theorem}

\subsection{Continuity of measure}

\begin{definition}[]
	A countable collection of sets
	$
		\left\{
			E_k
		\right\}_{k=1}^\infty
	$
	is said to be \emph{ascending} provided for each $k \in \N$,
	$E_k \subset E_{k+1}$.
	Similarly, the collection is said to be \emph{descending} if 
	$E_{k+1} \subset E_k$.
\end{definition}

\begin{theorem}[]
	The Lebesgue measure possesses the following continuity properties.
	\begin{enumerate}
		\item
		If
		$
			\left\{
				A_k
			\right\}_{k=1}^\infty
		$
		is an ascending collection of measurable sets, then
		\[
			m\left( 
				\bigcup_{k=1}^\infty A_k
			\right) = \lim_{k \to \infty} \left( 
				m(A_k) 
			\right).
		\] 

		\item
		If
		$
			\left\{
				B_k
			\right\}_{k=1}^\infty
		$
		is a descending collection of measurable sets and $m(B_1) < \infty$,
		then
		\[
			m\left( 
				\bigcap_{k=1}^\infty B_k 
			\right) = \lim_{k\to\infty} \left( 
				m(B_k) 
			\right).
		\]
	\end{enumerate}
\end{theorem}

\begin{proof}
	$(i)$: we may assume that $m(A_k) < \infty$ for every $k \in \N$ as
	otherwise our statement holds trivially.
	We define $A_0 = \varnothing$ and $C_k = A_k \setminus A_{k-1}$.
	As 
	$
		\left\{
			A_k
		\right\}
	$ 
	is ascending,
	$
		\left\{
			C_k
		\right\}
	$
	are disjoint and $\bigcup A_k = \bigcup C_k$.
	Thus
	\[
		m\left( 
			\bigcup A_k 
		\right) = \sum m \left( 
			A_k \setminus A_{k-1}
		\right).
	\]
	Since
	$
		\left\{
			A_k
		\right\}
	$
	is ascending, by the excision property
	\begin{align*}
		\sum^{\infty}_{k=1} m\left( 
			A_k \setminus A_{k-1} 
		\right)
		&= \sum^{\infty}_{k=1} \left( 
			m(A_k) - m(A_{k-1}) 
		\right) \\
		&= \lim_{n\to\infty} \left( 
			\sum^{n}_{k=1} \left( 
				m(A_k) - m(A_{k-1}) 
			\right) 
		\right) \\
		&= \lim_{n\to\infty} \left( 
			m(A_n) - m(A_0) 
		\right) \\
		&= \lim_{n\to\infty} \left( 
			m(A_n) 
		\right).
	\end{align*}
	$(ii)$:
	define $D_k = B_1 \setminus B_k$ for each $k \in \N$.
	Since
	$
		\left\{
			B_k
		\right\}
	$
	is descending then
	$
		B_1 \setminus B_k \subset B_1 \setminus B_{k+1}
	$.
	So
	$
		\left\{
			D_k
		\right\}
	$
	is ascending.
	By $(i)$,
	\[
		m\left( 
			\bigcup_{k=1}^n B_k 
		\right) = \lim_{k \to \infty} m(B_k).
	\]
	Moreoever,
	\[
		\bigcup_{k=1}^\infty D_k
		= \bigcup_{k=1}^\infty (B_1 \setminus B_k)
		= B_1 \setminus \left( 
			\bigcap_{k=1}^\infty B_k
		\right).
	\]
	By the excision property (since $m(B_k) < \infty$),
	\[
		m\left( 
			B_1 \setminus \bigcap_{k=1}^\infty B_k
		\right) = \lim_{k \to \infty}(m(B_1) - m(B_k)).
	\]
	As $\bigcap_{k=1}^\infty B_k \subset B_1$, we have
	\begin{align*}
		m(B_1) - m\left( 
			\bigcap_{k=1}^\infty B_k 
		\right)
		&= m\left( 
			B_1 \setminus \bigcap_{k=1}^\infty B_k 
		\right) \\
		&= \lim_{k \to \infty} \left( 
			m(B_1) - m(B_k) 
		\right) \\
		&= m(B_1) - \lim_{k\to\infty} \left( 
			m(B_k) 
		\right). \qedhere
	\end{align*}
\end{proof}
