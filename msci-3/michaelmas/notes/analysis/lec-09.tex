%! TEX root = master.tex
\lecture{9}{11/11}

\begin{proposition}[]
	Outer measure is translation invariant.
\end{proposition}

\begin{proof}
	For any set $R \subset \R$ and real number $y \in \R$, 
	we will use the notation
	\[
		R + y = \left\{
			r + y: r \in R
		\right\}.
	\]
	Let $\left\{
		I_k
	\right\}_{k=1}^\infty$
	be a countable collection of sets and let $A \subset \R$.
	It is clear that $\{I_k\}$ covers $A$ if and only if
	$\{I_k + y\}$ covers $A + y$.
	Similarly, if $I_k$ is open then $I_k + y$ is also open
	with the same length.
\end{proof}

\begin{proposition}[]
	Outer measure is countably subadditive.
\end{proposition}

A \emph{subadditive} function is a function $f: A \to B$,
having a domain $A$ and an ordered codomain $B$ that are both closed
under addition, such that for every $x, y \in A$,
$f(x + y) \leq f(x) + f(y)$.
We call a function $g: \Sigma \to \R^\star$ (where $\R^\star$ denotes
the extended real numbers), having a $\sigma$-algebra domain $\Sigma$,
\emph{countably subadditive} if  we have
\[
	g\left( 
		\bigcup_{k=1}^\infty E_k
	\right) \leq
	\sum^{\infty}_{k=1} g(E_k)
\]
for any countable collection
$\left\{
	E_k
\right\}_{k=1}^\infty \subset \Sigma$.

Let $X$ be a set, then recall that a $\sigma$-alphabet 
$\Sigma \subset \mathcal P(X)$ is closed under complement and under
countable unions.

\begin{proof}
	Let $\left\{
		E_k
	\right\}_{k=1}^\infty$
	be a countable collection of sets.
	If one of these $E_k$ has infinite outer measure, then the statement
	holds trivially.
	Suppose $E_k$ has finite outer measure.
	Let $\varepsilon > 0$. Then for every $k \in \N$
	there is $\left\{
		I_{k,i}
	\right\}_{i=1}^\infty$ of open bounded intervals for which
	\[
		E_k \subset \bigcup_{i=1}^\infty I_{k,i}, \qquad
		\sum^{\infty}_{i=1} l(I_{k,i}) < m^\star(E_k) + \frac{\varepsilon}{2^k}.
	\]
	Observe that $\left\{
		I_{k,i}
	\right\}_{1 \leq k,i \leq \infty}$ is a countable collection of open
	bounded intervals that cover $\bigcup_{k=1}^\infty E_k$.
	Thus by definition
	\begin{align*}
		m^\star \left( 
			\bigcup_{k=1}^\infty E_k 
		\right)
		&\leq \sum_{1 \leq k, i \leq \infty} l(I_{k,i}) \\
		&= \sum_{k=1}^\infty \left( 
			\sum_{i=1}^\infty l(I_{k,i}) 
		\right) \\
		&< \sum_{k=1}^\infty \left( 
			m^\star(E_k) + \frac{\varepsilon}{2^k}  
		\right) \\
		&= \left( 
			\sum_{k=1}^\infty m^\star (E_k)
		\right) + \varepsilon.
	\end{align*}
	If $\left\{
		E_k
	\right\}_{k=1}^n$ is any finite collection of sets, then
	\[
		m^\star\left( 
			\bigcup_{k=1}^n E_k 
		\right) \leq \sum^{n}_{k=1} m^\star(E_k).
	\]
	The finite subadditivity property follows from the countable subadditivity
	property by taking $E_k = \varnothing$, recalling that 
	$m^\star(\varnothing) = 0$.
\end{proof}

\section{The $\sigma$-algebra of measurable sets on $\R$}

Outer measures have useful properties; however, they are not \emph{countable
additive} or even \emph{finitely additive}.

\begin{definition}[Measurable set]
	A set $E \subset \R$ is said to be \emph{measurable} if for every
	set $A$,
	\[
		m^\star(A) = m^\star(A \cap E) + m^\star(A \cap E^c).
	\]
\end{definition}

The usefulness of measurable sets becomes immediately apparent when we
see that if $A$ is measurable is disjoint from some set $B$, then
\[
	m^\star(A \cup B) = m^\star((A \cup B) \cap A)
		+ m^\star((A \cup B) \cap A^c)
	= m^\star(A) + m^\star(B).
\]
Observe that, since outer measure is subadditive, $A$ is measurable
if and only if $m^\star(A) \geq M^\star(A \cap E) + m^\star(A \cap E^c)$,
which holds if $m^\star(A) = \infty)$. Furthermore, see that $A$ is
measurable if and only if $A^c$ is measurable.
Clearly $\varnothing$ and $\R$ are measurable.
