%! TEX root = master.tex
\section{Measurable functions}
\lecture{13}{21/11}

\begin{definition}[Almost everywhere]
	A property $P$ is said to hold \emph{almost everywhere}
	(abbreviated \emph{a.e.})
	in a measurable set $E$ if there exists $E_0 \subset E$
	with $m(E_0) = 0$ such that $P$ holds in $E \setminus E_0$.
\end{definition}

\begin{proposition}[]
	\label{prop:measurable}
	Let $E \subset \R$ be measurable, $f: E \to \R$, and $c \in \R$.
	Then the following are equivalent.
	\begin{enumerate}
		\item
			$
				\left\{
					x \in E: f(x) > c
				\right\}
			$
			is measurable.

		\item
			$
				\left\{
					x \in E: f(x) \geq c
				\right\}
			$
			is measurable.

		\item
			$
				\left\{
					x \in E: f(x) < c
				\right\}
			$
			is measurable.

		\item
			$
				\left\{
					x \in E: f(x) \leq c
				\right\}
			$
			is measurable.
	\end{enumerate}
	Each of these properties implies that for each extended real number $c$,
	the set
	$
		\left\{
			x \in E: f(x) = c
		\right\}
	$
	is measurable.
\end{proposition}

\begin{proof}
	Since measurable sets form a $\sigma$-algebra,
	so $(i) \iff (iv)$ and $(ii) \iff (iii)$ as they are
	complementary.
	So now we prove $(i) \implies (ii)$.
	Observe
	\[
		\left\{
			x \in E: f(x) \geq c
		\right\}
		= \bigcap_{k=1}^\infty \left\{
			x \in E: f(x) > c - \frac1k
		\right\},
	\]
	and so we are done.
	Similarly, for $(ii) \implies (i)$ we see that
	\[
		\left\{
			x \in E: f(x) > c
		\right\}
		= \bigcup_{k=1}^\infty \left\{
			x \in E: f(x) \geq c + \frac1k
		\right\}
	\]
	and so we are done.
	For our last statement, observe that
	\[
		\left\{
			x \in E: f(x) = c
		\right\}
		=
		\left\{
			x \in E: f(x) \geq c
		\right\}
		\cap
		\left\{
			x \in E: f(x) \leq c
		\right\}
	\]
	and furthermore
	\[
		\left\{
			x \in E: f(x) = \infty
		\right\}
		=
		\bigcap_{k=1}^\infty
		\left\{
			x \in E: f(x) > k
		\right\}. \qedhere
	\]
\end{proof}

\begin{definition}[Lebesgue measurable]
	An extended real-valued function $f$ defined on $E$ is said to be
	\emph{Lebesgue measurable} (or just \emph{measurable})
	provided $E$ is measurable and it satisfies the four properties
	from Proposition \ref{prop:measurable}.
\end{definition}

\begin{proposition}[]
	Let $f$ be a function defined on a measurable set $E$.
	Then $f$ is measurable if and only if for every open set
	$O \subset f(E)$,
	$f^{-1}(O)$ is measurable.
\end{proposition}

\begin{proof}
	$(\impliedby)$:
	consider the open set $(c, \infty)$.
	Observe that
	$
		f^{-1}(c, \infty) =
		\left\{
			x \in E: f(x) > c
		\right\}
	$
	is measurable.
	Thus $f$ is measurable.
	$(\implies)$:
	let $O$ be open.
	Then there is a collection
	$
		\left\{
			I_k
		\right\}
	$
	of open bounded intervals such that
	$
		O = \bigcup_{k=1}^\infty I_k
	$.
	Observe that for every 
	$I_k = (a_k, b_k)$,
	$I_k = A_k \cap B_k$
	where $A_k = (a_k, \infty)$ and $B_k = (-\infty, b_k)$.
	Since $f$ is measurable, $f^{-1}(A_k)$ and $f^{-1}(B_k)$ are measurable
	(look at the definition of these sets).
	Then
	\[
		f^{-1}(O)
		= f^{-1}
		\left( 
			\bigcup_{k=1}^\infty (A_k \cap B_k) 
		\right)
		=
		\bigcup_{k=1}^\infty
		\left( 
			f^{-1}(A_k) \cap f^{-1}(B_k) 
		\right)
	\]
	is measurable.
\end{proof}

\begin{proposition}[]
	A continuous real-valued function with a measurable domain is measurable.
\end{proposition}

\begin{proof}
	Let $E \subset \R$ be measurable, $O \subset \R$ be open, and
	$f: E \to \R$ be continuous.
	As $f$ is continuous, 
	$f^{-1}(O) = E \cap U$ where $U$ is open.
	As $U$ is open (and thus a countable union of disjoint open intervals),
	it is measurable.
	Hence $f^{-1}(O) = E \cap U$ is measurable.
	Therefore $f$ is measurable.
\end{proof}

\begin{proposition}[]
	A monotone function that is defined on an interval is measurable.
\end{proposition}

\begin{proof}
	Let $E$ be an interval, hence it is measurable.
	Then
	$
		\left\{
			x \in E: f(x) > c
		\right\}
	$
	is either an interval, a single point set, or empty for all $c$, 
	thus it is measurable.
	Hence $f$ is measurable.
\end{proof}

\begin{proposition}[]
	Let $f$ be an extended real-valued function on a measurable set $E$.
	\begin{enumerate}
		\item If $f$ is measurable on $E$ and $f(x) = g(x)$ a.e. on $E$.
			Then $g$ is measurable on $E$.

		\item Let $D \subset E$ be a measurable subset.
			$f$ is measurable on $E$ if and only if
			$\restr fD$ and $\restr f{E \setminus D}$ are measurable.
	\end{enumerate}
\end{proposition}

\begin{proof}
	$(i)$: define 
	$
		A =
		\left\{
			x \in E: f(x) \neq g(x)
		\right\}.
	$
	Now let, as $f(x) = g(x)$ a.e. there is $E_0 \subset E$ with measure zero
	such that $f(x) = g(x)$ for every $x \in E \setminus E_0$.
	Observe that $A \subset E_0$ so
	\[
		m^\star(A) \leq m^\star(E_0) = 0,
	\]
	and so $A$ is measurable with $m(A) = 0$.
	Further, note
	\[
		\left\{
			x \in E: g(x) > c
		\right\}
		= \left\{
			x \in A: g(x) > c
		\right\} \cup \left( 
			\left\{
				x \in E: f(x) > c
			\right\}
			\cap 
			\left( 
				E \setminus A 
			\right)
		\right).
	\]
	Now
	$
		\left\{
			x \in A: g(x) > c
		\right\}
		\subset A
	$
	is measurable since it is a subset of a set of measure zero
	(from similar argument to before).
	Finally, as $E$ and $A$ are measurable, $E \setminus A$.
	Combining this with the fact that $f$ is measurable,
	the set
	\[
		\left\{
			x \in E : f(x) > c
		\right\}
		\cap (E \setminus A)
		=
		\left(
			\left\{
				x \in E: f(x) > c
			\right\}^c
			\cup (E \setminus A)^c
		\right)^c
	\]
	is measurable.
	So $g$ is measurable.
	$(ii)$: let $c \in \R$.
	Then
	\[
		\left\{
			x \in E: f(x) > c
		\right\}
		=
		\left\{
			x \in D: f(x) > c
		\right\}
		\cup 
		\left( 
			\left\{
				x \in E \setminus D: f(x) > c
			\right\} 
		\right).
	\]
	If $f$ is measurable on $E$,
	$
		\left\{
			x \in E: f(x) > c
		\right\}
	$
	is measurable.
	Hence
	\[
		\left\{
			x \in D : f(x) > c
		\right\}
		\cup
		\left\{
			x \in E \setminus D: f(x) > c
		\right\}
	\]
	is measurable.
	Hence $f$ is measurable on $D$ and $E \setminus D$.
\end{proof}

\begin{proposition}[]
	For a finite family
	$
		\left\{
			f_k
		\right\}_{k=1}^m
	$
	of measurable functions each with a domain $E$.
	Then
	$
		\min
		\left\{
			f_1, \ldots, f_n
		\right\}
	$
	and
	$
		\max
		\left\{
			f_1, \ldots, f_n
		\right\}
	$
	are measurable.
\end{proposition}

\begin{proof}
	Let $c \in \R$.
	Then observe
	\[
		\left\{
			x \in E: \max
			\left\{
				f_1, \ldots, f_n
			\right\}
			(x) > c
		\right\}
		=
		\bigcap_{k=1}^\infty
		\left\{
			x \in E:
			f_k(x) > c
		\right\}
	\]
	and so is measurable.
	A similar argument show that the minimum functino is also measurable.
\end{proof}

For a function $f$ defined on $E$, we have the associated functions
$\left\lvert f \right\rvert$, $f^+$, and $f^-$
defined by
\begin{align*}
	\left\lvert f \right\rvert(x)
	&= \max\left\{
		f(x)< -f(x)
	\right\}, \\
	f^+(x)
	&= \max\left\{
		f(x), 0
	\right\}, \\
	f^-(x)
	&= \max\left\{
		-f(x), 0
	\right\}.
\end{align*}
By the previous proposition, if $f$ is measurable so are the three
functions defined above.
