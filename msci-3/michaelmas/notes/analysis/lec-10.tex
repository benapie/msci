%! TEX root = master.tex
\lecture{10}{17/11}

\begin{proposition}[]
	Any set of outer measure zero is measurable.
	In particular, any countable set is measurable.
\end{proposition}

\begin{proof}
	Let $E \subset \R$ have outer measure zero.
	Let $A \subset \R$.
	Since $A \cap E \subset E$ and
	$A \cap E^c \subset A$, by the monotonicity of outer measure
	\[
		m^\star(A \cap E) \leq m^\star(E) = 0, \qquad
		m^\star(A \cap E^c) \leq m^\star(A).
	\]
	Thus
	\begin{align*}
		m^\star(A)
		&\geq m^\star(A \cap E^c) \\
		&= 0 + m^\star(A \cap E^c) \\
		&\geq m^\star(A \cap E) + m^\star(A \cap E^c)
	\end{align*}
	and therefore $E$ is measurable.
\end{proof}

\begin{proposition}[]
	The union of a finite collection of measurable sets is measurable.
\end{proposition}

\begin{proof}
	Let $E_1, E_2 \subset \R$ be measurable and $A \subset \R$.
	Then
	\begin{align*}
		m^\star(A)
		&= m^\star(A \cap E_1) + m^\star(A \cap E_1^c) \\
		&= m^\star(A \cap E_1) + m^\star((A \cap E_1^c) \cap E_2)
			+ m^\star((A \cap E_1^c) \cap E_2^c) \\
		&= m^\star(A \cap E_1) + m^\star((A \cap E_1^c) \cap E_2)
			+ m^\star(A \cap (E_1 \cup E_2)^c) \\
		&\geq m^\star(A \cap (E_1 \cup E_2)) + m^\star(A \cap (E_1 \cup E_2)^C)
	\end{align*}
	as $(A \cap E_1) \cup (A \cap E_1^c \cap E_2) = A \cap (E_1 \cup E_2$.
	Thus $E_1 \cup E_2$ is measurable.
\end{proof}

\begin{proposition}[]
	Let $A \subset \R$ and $
		\left\{
			E_k
		\right\}_{k=1}^n
		$ be a finite disjoint collection of measurable sets.
		Then
		\[
			m^\star\left( 
				A \cap \left( 
					\bigcup_{k=1}^n E_k 
				\right) 
			\right) = \sum^{n}_{k=1} m^\star(A \cap E_k).
		\]
		In particular,
		\[
			m^\star\left( 
				\bigcup_{k=1}^n E_k 
			\right) = \sum^{n}_{k=1} m^\star(E_k).
		\]
\end{proposition}

\begin{proof}
	This proof proceeds by induction on $n$.
	For $n = 1$, this is clearly true.
	Now assume it is true for $n - 1$.
	Since the collection $\left\{
		E_k
	\right\}_{k=1}^n$ is disjoint we have, \[
		A \cap \left( 
			\bigcup_{k=1}^n E_k 
		\right) \cap E_n = A \cap E_n
	\]
	and \[
		A \cap \left( 
			\bigcup_{k=1}^n E_k 
		\right) \cap E^c_n = A \cap \left( 
			\bigcup_{k=1}^{n-1} E_k 
		\right).
	\]
	By the measurability of $E_n$ and the induction assumption,
	\begin{align*}
		m^\star\left( 
			A \cap \left( 
				\bigcup_{k=1}^n E_k
			\right) 
		\right)
		&= m^\star(A \cap E_n) + m^\star\left( 
			A \cap \left( 
				\bigcup_{k=1}^{n-1} E_k 
			\right) 
		\right) \\
		&= m^\star(A \cap E_n) +
			\sum^{n-1}_{k=1} m^\star(A \cap E_k) \\
		&= \sum_{k=1}^n m^\star(A \cap E_n). \qedhere
	\end{align*}
\end{proof}

\begin{definition}[Algebra]
	A collection of subsets of $\R$ is called an \emph{algebra} if it contains
	$\R$ and is closed under complementation and finite unions.
\end{definition}

\begin{proposition}[]
	The collection of measurable sets form an algebra.
\end{proposition}

It is important to observe that if a set is the union of a countable collection
of measurable sets, then it is also the union of a countable disjoint collection
of measurable sets.
Indeed, let $
	\left\{
		A_k
	\right\}_{k=1}^\infty
$ be a countable collection of measurable sets.
We define $A_1' = A_1$ and for each $k \in \left\{
	2, 3, \ldots
\right\}$ define \[
	A_k' = A_k \setminus \bigcup_{i=1}^{k-1} A_i.
\]
As the set of measurable sets is an algebra, $
	\left\{
		A_k'
	\right\}_{k=1}^\infty
$ is a disjoint collection of measurable sets whose union is the same as that of
$
	\left\{
		A_k
	\right\}_{k=1}^\infty
$.

\begin{proposition}[]
	The union of a countable collection of measurable sets is measurable.
\end{proposition}

\begin{proof}
	Let $E \subset \R$ be the union of a countable collection of measurable
	sets.
	Then there is a disjoint collection $
		\left\{
			E_k
		\right\}_{k=1}^\infty
	$
	such that $E = \bigcup_{k=1}^\infty E_k$.
	Let $A \subset \R$ and $n \in \N$.
	Define $F_n = \bigcup_{k=1}^n E_k$.
	Since $F_n$ is measurable and $E^c \subset F_n^c$,
	\begin{align*}
		m^\star(A)
		&= m^\star(A \cap F_n) + m^\star(A \cap F_n^c) \\
		&\geq m^\star(A \cap F_n) + m^\star(A \cap E^c).
	\end{align*}
	Now
	\[
		m^\star(A \cap F_n)
		= \sum_{k=1}^n m^\star(A \cap E_k)
	\]
	and thus
	\[
		m^\star(A) \geq \sum^{n}_{k=1} \left( 
			m^\star(A \cap E_k)
		\right)
		+ m^\star(A \cap E^c).
	\]
	As the LHS of this expression is independent of $n$, we let 
	$n \to \infty$ to obtain
	\[
		m^\star(A) \geq \sum^{\infty}_{k=1} \left( 
			m^\star(A \cap E_k)
		\right)
		+ m^\star(A \cap E^c).
	\]
	By the subadditivty property of outer measure,
	\[
		m^\star(A) \geq m^\star(A \cap E) + m^\star(A \cap E^c)
	\]
	and hence $E$ is measurable.
\end{proof}

We observe that, by De Morgan's identities,
a collection of measurable set is also closed with respect to countable
intersections.

\begin{definition}[$\sigma$-algebra]
	A collection of subsets of $\R$ is called a
	\emph{$\sigma$-algebra} if it is closed under complementation and countable
	unions.
\end{definition}

\begin{proposition}[]
	The collection of measurable sets is a $\sigma$-algebra.
\end{proposition}
