%! TEX root = master.tex
\lecture{7}{26/10}

\begin{proposition}[]
	Let $f: \R \to \R$.
	Then $f$ is continuous if and only if
	for every open set $U \subset \R$,
	then $f^{-1}(U)$ is also open.
\end{proposition}

\begin{proof}
	$(\implies)$:
	let $U \subset \R$ be open and $x \in f^{-1}(U)$.
	Then $f(x) \in U$.
	As $U$ is open, there is $\varepsilon > 0$
	such that 
	$(f(x) - \varepsilon, f(x) + \varepsilon) \subset U$.
	By the continuity of $f$, there is $\delta > 0$
	such that
	$\left\lvert x - y \right\rvert < \delta \implies
	\left\lvert f(x) - f(y) \right\rvert < \varepsilon$
	for all $y \in \R$.
	Therefore for every $y$ satisfying
	$\left\lvert x - y \right\rvert < \delta$
	we have
	\[
		f(y) 
		\in \left( f(x) - \varepsilon, f(x) + \varepsilon \right)
		\subset U
	\]
	Therefore $f(y) \in U$,
	so $y \in f^{-1}(U)$.
	So we have shown that
	$(x - \delta, x + \delta) \subset f^{-1}(U)$.
	Thus $f^{-1}(U)$ is open.

	$(\impliedby)$:
	Let $x \in \R$ and fix $\varepsilon > 0$.
	Consider
	\[
		U = \left( f(x) - \varepsilon, f(x) + \varepsilon \right).
	\]
	$U$ is open, so $f^{-1}(U)$ is open.
	Now let $x \in f^{-1}(U)$.
	Then there is $\delta > 0$ such that
	$
		(x - \delta, x + \delta) \subset f^{-1}(U)
	$
	and so
	\[
		f\left( (x - \delta, d + \delta) \right)
		\subset U
		= (f(x) - \varepsilon, f(x) + \varepsilon).
	\]
	So if $\left\lvert x - y \right\rvert < \delta$
	then $\left\lvert f(x) - f(y) \right\rvert < \varepsilon$
	and thus $f$ is continuous.
\end{proof}

\begin{theorem}[]
	Let $F \subset \R$ be closed and bounded and let $f: F \to \R$
	be continuous.
	Then $f$ is bounded on $F$ and assumes its maximum and minimum on $F$.
	That is,
	there is $a, b \in F$ such that for every $x \in F$, 
	$f(a) \leq f(x) \leq f(b)$.
\end{theorem}

\begin{proof}
	First we prove that $f$ is bounded.
	Since $f$ is continuous on $F$,
	for every $x \in F$ there is an open interval $I_x$
	such that
	$
		\left\lvert f(y) - f(x) \right\rvert < 1
	$
	for every $y \in I_x \cap F$.
	Hence, $ \left\{ I_x: x \in F \right\} $
	is an open cover of $F$, so by the Heine-Borel theorem we can 
	find a finite subcover
	$ \left\{ I_{x_1}, \ldots, I_{x_n} \right\} $ 
	of $F$.
	Observe that for every $y \in F$,
	there is $i \in \left\{ 1, \ldots, n \right\}$
	such that $y \in I_{x_i}$.
	Then
	\[
		\left\lvert f(y) \right\rvert
		< 1 + \left\lvert f(x_i) \right\rvert
		< 1 + \max \left\{ f(x_1), \ldots, f(x_i) \right\}
	\]
	and so $f$ is bounded.
	
	We now prove that $f$ assumes its maximum on $F$.
	Let $m = \sup_{x \in F} f(x)$.
	Observe that $\left\lvert m \right\rvert < \infty$
	as $F$ is finite.
	Assume that there is no such $x \in F$ such that
	$f(x) = m$.
	So, for every $x \in F: f(x) < m$.
	By the continuity of $f$, there is an open interval
	$I_x$ such that for every $y \in I_x$
	\[
		f(y) < \frac12 \left( f(x) + m \right).
	\]
	By the Heine-Borel theorem, the open cover
	$ \left\{ I_x: x \in F \right\} $
	has a finite subcover
	$ \left\{ I_{x_1}, \ldots, I_{x_n} \right\} $.
	Let
	$
		a = \max\left\{
			\left\lvert f(x_1) \right\rvert,
			\ldots,
			\left\lvert f(x_n) \right\rvert
		\right\}.
	$
	Then for every $y \in F$ there is an open interval $I_{x_i}$
	containing $y$ such that
	\[
		f(y) < \frac12\left( f(x_i) + m \right) < \frac12(a + m).
	\]
	Thus $\frac12(a+m)$ is a bound for $f$ on $F$,
	but this is impossible since $\frac12(a+m) < m$;
	therefore, there is $x \in F$ such that $f(x) = m$.

	A similar property can be shown to hold for the minimum.
\end{proof}

\subsection{Uniform continuity}

\begin{definition}[Uniform continuity]
	A function $f: E \subset \R \to \R$
	is \emph{uniformly continuous}
	if for every $\varepsilon > 0$
	there is $\delta > 0$
	such that for every $x, y \in E$:
	\[
		\left\lvert x - y \right\rvert < \delta
		\implies \left\lvert f(x) - f(y) \right\rvert < \varepsilon.
	\]
\end{definition}

\begin{proposition}[]
	Let $F$ be closed and bounded, and $f: F \to \R$ be continuous.
	Then $f$ is uniformly continuous.
\end{proposition}

\begin{proof}
	Let $\varepsilon > 0$ and $x \in F$.
	Then there is $\delta = \delta(x)$ such that
	\[
		\left\lvert x - y \right\rvert < \frac{\delta}{2}
		\implies \left\lvert f(x) - f(y) \right\rvert 
		< \frac{\varepsilon}{2}.
	\]
	Now consider
	\[
		O_x = \left( 
			x - \frac{\delta(x)}{2},
			x + \frac{\delta(x)}{2}
		\right)
	\]
	and the open cover $\left\{
		O_x: x \in F
	\right\}$.
	By the Heine-Borel theorem, there is a finite subcover
	$\left\{
		O_{x_1}, \ldots, O_{x_n}
	\right\}$.
	Now let $\delta = \frac12 \min\left\{
		\delta(x_1), \ldots, \delta(x_n)
	\right\}$.
	Now take $y, z \in F$ with $\left\lvert y - z \right\rvert < \delta$.
	Observe that $y \in O_{x_i}$ for some $i \in \left\{
		1,\ldots,n
	\right\}$,
	so $\left\lvert y - x_i \right\rvert < \frac12 \delta(x_i)$.
	Hence
	\[
		\left\lvert z - x_i \right\rvert
		\leq \left\lvert z - y \right\rvert
		+ \left\lvert y - x_i \right\rvert
		\leq \delta + \frac{\delta(x_i)}{2} 
		< \delta(x_i)
	\]
	and as a consequence $
		\left\lvert f(x_i) - f(y) \right\rvert < \frac{\varepsilon}{2}
	$, $
	\left\lvert f(x_i) - f(z) \right\rvert < \frac{\varepsilon}{2}
	$, and $
	\left\lvert f(z) - f(y) \right\rvert < \varepsilon
	$. 
	Thus $f$ is uniformly convergent.
\end{proof}

\subsection{Convergence}

\begin{definition}[Pointwise convergence]
	Let $f_n: E \subset \R \to \R$ be a sequence of functions
	and $f: E \to \R$.
	We say that $\left\{
		f_n
	\right\}_{n \in \N}$
	\emph{converges pointwise} to $f$
	if for every $x \in E$ and $\varepsilon > 0$
	there is $N \in \N$
	such that for every $n \geq N$:
	\[
		\left\lvert f_n(x) - f(x) \right\rvert < \varepsilon.
	\]
\end{definition}


\begin{definition}[Uniform convergence]
	Let $f_n: E \subset \R \to \R$ be a sequence of functions
	and $f: E \to \R$.
	We say that $\left\{
		f_n
	\right\}_{n \in \N}$
	\emph{converges uniformly} to $f$
	if for every $\varepsilon > 0$
	there is $N \in \N$
	such that for every $x \in E$ and $n \geq N$:
	\[
		\left\lvert f_n(x) - f(x) \right\rvert < \varepsilon.
	\]
\end{definition}

\begin{definition}[]
	A real-valued function $f: E \subset \R \to \R$
	is said to be \emph{increasing} if
	$f(x) \leq f(x')$ for every $x, x' \in E$ with $x \leq x'$.
	We say that $f$ is \emph{decreasing}
	if $-f$ is increasing.
\end{definition}

