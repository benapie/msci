%! TEX root = master.tex
\lecture{14}{25/11}

\begin{theorem}[]
	Let $f$ and $g$ be measurable functions on $E$ that are finite
	almost everywhere on $E$.
	\begin{enumerate}
		\item (Linearity) For every $\alpha, \beta \in \R$,
			$\alpha f + \beta g$ is measurable on $E$.
		\item (Products) $fg$ is measurable on $E$.
	\end{enumerate}
\end{theorem}

\begin{proof}
	If $\alpha = 0$, then $\alpha f$ is clearly measurable.
	So we assume that $\alpha \neq 0$.
	Observe that for every $c \in \R$:
	\[
		\left\{
			x \in E: \alpha f(x) > c
		\right\}
		=
		\begin{cases}
			\left\{
				x \in E: f(x) > \frac{c}{\alpha}
			\right\}
			& \alpha > 0, \\
			\left\{
				x \in E: f(x) < \frac{c}{\alpha}
			\right\}
			& \alpha < 0.
		\end{cases}
	\]
	So if $f$ is measurable, clearly $\alpha f$ is too.
	For linearity, we have left to prove that the sum of two
	measurable functions is measurable.
	For $x \in E$, if $f(x) + g(x) < c$ then
	$f(x) < c - g(x)$.
	By the density of $\Q$ in $\R$,
	we can find a $q \in \Q$ such that
	\[
		f(x) < q < c - g(x).
	\]
	Hence
	\[
		\left\{
			x \in E: f(x) + g(x) < c
		\right\}
		= \bigcup_{q \in \Q}
		\left\{
			x \in E: g(x) < c - q
		\right\}
		\cap
		\left\{
			x \in E: f(x) < q
		\right\}
	\]
	and as $\Q$ is countable, $f(x) + g(x)$ is measurable.
	Now observe that
	\[
		fg = \frac12\left( 
			(f+g)^2 - f^2 - g^2 
		\right).
	\]
	So to prove that $fg$ is measurable, we must only show that
	$f^2$ is measurable.
	For $c > 0$
	\[
		\left\{
			x \in E: (f(x))^2 > c
		\right\}
		=
		\left\{
			x \in E: f(x) > \sqrt c
		\right\}
		\cup
		\left\{
			x \in E: f(x) < -\sqrt c
		\right\}
	\]
	and for $c < 0$
	\[
		\left\{
			x \in E: (f(x))^2 > c
		\right\}
		= E
	\]
	and thus $f^2$ is measurable.
\end{proof}

\begin{definition}[Function convergence]
	Let
	$
		\left\{
			f_n
		\right\}_{n \in \N}
	$
	be a sequence of functions where $f_n: E \to \R$ for each $n \in \N$
	for $E \subset \R$.
	Let $A \subset E$.
	\begin{enumerate}
		\item \emph{$f_n \to f$ pointwise on $A$} if
			\[
				\lim_{n \to \infty} f_n(x) = f(x)
			\]
			for every $x \in A$.

		\item \emph{$f_n \to f$ pointwise almost everywhere on $A$}
			if there is $B \subset R$ with $m(B) = 0$ such that
			$f_n \to f$ pointwise on $A \setminus B$; and

		\item \emph{$f_n \to f$ uniformly on $A$} if
			for every $\varepsilon > 0$ there is $N \in \N$
			such that
			\[
				\left\lvert f_n(x) - f(x) \right\rvert < \varepsilon
			\]
			for every $x \in A$ and $n \geq N$.
	\end{enumerate}
\end{definition}

\begin{proposition}[]
	Let
	$
		\left\{
			f_n
		\right\}_{n \in \N}
	$
	be a family of real-valued measurable functions with common domain $E$
	such that $f_n \to f$ pointwise almost everywhere on $E$.
	Then $f$ is measurable.
\end{proposition}

\begin{proof}
	Let $E_0 \subset E$ such that $m(E_0) = 0$
	and $f_n \to f$ pointwise on $E_0$.
	$f$ is measurable on $E$ if and only if
	$\restr f{E_0}$ and $\restr f{E \setminus E_0}$
	are measurable.
	Now $\left\{
		x \in E_0: f(x) < c
	\right\} \subset E_0$
	and $m(E_0) = 0$, so $\restr f{E_0}$ is measurable.
	Fix $c \in \R$. We must show that
	$
		\left\{
			x \in E \setminus E_0: f(x) < c
		\right\}
	$
	is measurable.
	Observe that for every $x \in E \setminus E_0$,
	$f_n(x) \to f(x)$ and so $f(x) < c$ if and only if there is
	$n, k \in \N$ such that $f_j(x) < f(x) - \frac1n < c - \frac1n$
	for every $j \geq k$.
	But for every $n, j \in \N$ (as $f_j$ is measurable)
	$
		\left\{
			x \in E: f_j(x) < c - \frac1n
		\right\}
	$
	is also measurable.
	So
	$
		\bigcap_{j=k}^\infty
		\left\{
			x \in E: f_j(x) < c - \frac1n
		\right\}
	$
	is measurable.
	Observe that
	\[
		\left\{
			x \in E: f(x) < c
		\right\}
		= \bigcup_{k \geq 1, n < \infty}
		\left( 
			\bigcap_{j=k}^\infty
			\left\{
				x \in E: f_j(x) < c - \frac1n
			\right\}
		\right)
	\]
	and so $f(x)$ is measurable.
\end{proof}

\begin{definition}[Charactersitic function]
	For $A \subset \R$, we define the
	\emph{charactersitic function of $A$} by
	\[
		\chi_A(x) =
		\begin{cases}
			1 & x \in A, \\
			0 & x \not\in A.
		\end{cases}
	\]
\end{definition}

Now, we can show that $\chi_A$ is measurable if and only if $A$ is measurable.
Indeed, for $c \leq 0$,
\[
	\left\{
		x \in \R: \chi_A(x) < c
	\right\}
	= \varnothing;
\]
for $c > 1$,
\[
	\left\{
		x \in \R: \chi_A(x) < c
	\right\}
	= \R;
\]
and for $c \in (0,1]$,
\[
	\left\{
		x \in \R: \chi_A(x) < c
	\right\}
	= \R \setminus A.
\]
This proves the existence of non-measurable functions due to the existence 
of non-measurable sets.

\begin{definition}[Simple function]
	Let $E \subset \R$ be measurable.
	Then $\varphi: E \to \R$ is \emph{simple} if it is measurable
	and $\varphi(E)$ is finite.
\end{definition}

Note that linear combinations of simple functions must also be simple.
Suppose $\varphi$ is simple on $E$, let $c_1, \ldots, c_n$
be the distinct values that $\varphi$ takes, and let
$
	E_k =
	\left\{
		x \in E: \varphi(x) < c_k
	\right\}.
$
Then
\[
	\varphi(x) = \sum_{k=1}^n c_k \chi_{E_k}(x).
\]
We call this the \emph{canonical representation} of a simple function $\varphi$.

\begin{lemma}[Simple approximation lemma]
	Let $f: E \to \R$ be measurable and bounded.
	Then for each $\varepsilon > 0$ there is simple functions
	$\varphi_{\varepsilon}, \psi_{\varepsilon}: E \to \R$
	such that
	\[
		\varphi_\varepsilon(x) \leq f(x) \leq \psi_\varepsilon(x),
		\quad
		0 \leq \psi_\varepsilon(x) - \varphi_\varepsilon(x) < \varepsilon
	\]
	for every $x \in E$.
\end{lemma}

\begin{proof}
	Let $(c, d)$ be an open and bounded interval such that
	$f(E) \subset (c,d)$.
	Let $c = y_0 < y_1 < \ldots < y_{n-1} < y_n = d$
	be a partition such that $y_k - y_{k-1} < \varepsilon$
	for every $k \in \left\{
		1, \ldots, n
	\right\}$.
	Define
	$I_k = [y_{k-1}, y_k)$ and $E_k = f^{-1}(I_k)$.
	Now $I_k$ is open and $f$ is measurable, so
	$E_k$ is measurable.
	So we define
	\begin{align*}
		\varphi_\varepsilon(x)
		&= \sum_{k=1}^n y_{k-1} \chi_{E_k}(x), \\
		\psi_\varepsilon(x)
		&= \sum_{k=1}^n y_k \chi_{E_k}(x).
	\end{align*}
	Now let $x \in E$.
	Since $f(E) \subset (c,d)$,
	there is a unique $k$ such that $f(x) \in I_k$.
	Thus
	\[
		\varphi_\varepsilon(x) = y_{k-1} \leq f(x) < y_k = \psi_\varepsilon(x)
	\]
	and $y_k - y_{k-1} < \varepsilon$ and thus meets our criteria.
\end{proof}
