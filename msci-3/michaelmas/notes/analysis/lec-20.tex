%! TEX root = master.tex
\lecture{20}{22/12}

Step functions take a finite number of values, and all intervals are
measurable thus all step functions are simple.
We can infer that the Riemann integral over a closed, bounded interval
of a step function agrees with the Lebesgue integral.

\begin{theorem}
	Let $f$ be a bounded function defined on $[a,b]$.
	If $f$ is Riemann integrable over $[a,b]$, then
	it is Lebesgue integrable over $[a,b]$ and the two integrals are equal.
\end{theorem}

\begin{proof}
	$f$ is Riemann integrable, so
	\[
		\sup\left\{ 
			I_a^b(\varphi): \text{$\varphi$ step}, 
			\varphi \leq f 
		\right\}
		= \inf\left\{ I_a^b(\varphi): 
		\text{$\varphi$ step}, 
		f \leq \varphi \right
		\}.
	\]
	To prove that $f$ is Lebesgue integral, we must show that 
	\[
		\sup\left\{ 
			\int_I \varphi:
			\text{$\varphi$ simple},
			\varphi \leq f
		 \right\}
		 = \inf\left\{ 
			 \int_I \varphi:
			 \text{$\varphi$ simple},
			 f \leq \varphi
		\right\}.
	\]
	We have already shown that $\text{step} \implies \text{simple}$
	and the integrals are equal.
	Thus, we have shown 
	$\text{Riemann integral} \implies \text{Lebesgue integral}$ and the
	integrals are equal.
\end{proof}

\begin{example}
	Let $E = \Q \cap [0,1]$.
	Observe $m(E) = 0$.
	Recall the Dirchilet function $f$ is the restriction of $\chi_E$
	to $[0,1]$.
	Thus, $f$ is integrable and
	\[
		\int_{[0,1]} f = \int_{[0,1]} 1 \cdot \chi_E
		= 1 \cdot m(E) = 0.
	\]
	We have alreqady shown that $f$ is not Riemann integrable.
\end{example}

\begin{lemma}
	Let $\left\{ \varphi_n \right\}$ and $\left\{ \psi_n \right\}$
	be sequences of integrable functions over $E$ such that
	$\left\{ \varphi_n \right\}$ is increasing and
	$\left\{ \psi_n \right\}$ is decreasing.
	Let $f$ be a function on $E$ such that
	\[
		\varphi_n \leq f \leq \psi_n \;\text{on}\; E
	\]
	for all $n$.
	Then if $\lim_{n \to \infty} \int_E \left( \psi_n  - \varphi_n \right) = 0$,
	then $\left\{ \varphi_n \right\} \to f$ pointwise almost everywhere on $E$,
	$\left\{ \psi_n \right\} \to f$ pointwise almost everywhere on $E$,
	$f$ is integrable over $E$,
	$\lim_{n \to \infty} \int_E \varphi_n = \int_E f$, and
	$\lim_{n \to \infty} \int_E \psi_n = \int_E f$.
\end{lemma}

\begin{proof}
	For $x \in E$, define $\varphi^\star(x) = \lim_{n \to \infty} \varphi_n(x)$
	and $\psi^\star(x) = \lim_{n \to \infty} \psi_n(x)$.
	Both are well-defined since 
	$\left\{ \varphi_n \right\}$ and $\left\{ \psi_n \right\}$
	are monoton and thus converge to an extended real number.
	They are both measurable as each is the pointwise limit of a sequence
	of measurable functions.
	Thus, we have the inequalities:
	\[
		\varphi_n \leq \varphi^\star \leq f \leq \psi^\star \leq
		\psi_n \tag{$\star$}
	\]
	on $E$ for all $n$.
	By monotonicity and linearity of the integral of nonnegative measurable
	functions:
	\[
		0 \leq \int_E (\psi^\star - \varphi^\star)
		\leq \int_E (\psi_n - \varphi_n)
	\]
	for all $n$ so
	\[
		0 \leq \int_E (\psi^\star - \varphi^\star)
		\leq \lim_{n \to \infty} \int_E (\psi_n - \varphi_n) = 0.
	\]
	By a previous proposition, $\psi^\star = \varphi^\star$ almost everywhere
	on $E$.
	But $\varphi^\star \leq f \leq \psi^\star$.
	Thus, $\left\{ \varphi_n \right\} \to f$,
	$\left\{ \psi_n \right\} \to f$ pointwise almost everywhere on $E$.
	So $f$ is measurable and as $0 \leq f - \varphi_1 \leq \psi_1 - \varphi_1$
	on $E$ and $\varphi_1$ and $\psi_1$ are integrable over $E$.
	We infer, from $(\star)$, that
	\[
		0 \leq \int_E \psi_n - \int_E f
		= \int_E (\psi_n - f)
		\leq \int_E (\psi_n - \varphi_n)
	\]
	and
	\[
		0 \leq \int_E f - \int_E \varphi_n
		= \int_E (f - \varphi_n)
		\leq \int_E (\psi_n - \varphi_n)
	\]
	and thus
	\[
		\lim_{n \to \infty} \int_E \varphi_n
		= \int_E f
		= \lim_{n \to \infty} \int_E \psi_n. \qedhere
	\]
\end{proof}

\begin{theorem}[Lebesgue]
	Let $f$ be a bounded function on $[a,b]$.
	Then $f$ is Riemann integrable over $[a,b]$ if and only if
	\[
		m\left( \left\{ 
			x \in [a,b]: \text{$f$ is not continuous at $x$} 
		\right\} \right) = 0.
	\]
\end{theorem}

\begin{proof}
	Suppose $f$ is Riemann integrable.
	By the equality of the upper and lower Riemann integrals over $[a,b]$
	that there are sequences of partitions $\left\{ p_n \right\}$
	and $\left\{ p_n' \right\}$ of $[a,b]$ for which
	\[\lim_{n \to \infty} \left( U(f, p_n) - L(f, p_n') \right) = 0\]
	where $U(f,p_n)$ and $L(f,p_n')$ are the upper and lower Darboux sums.
	Since, under refinement, lower Darboux sums increase and upper Darboux sums
	decrease. By replacing each $p_n$ with a common refinement of
	$p_1, \ldots, p_n, \allowbreak p_1', \ldots p_n'$, we may assume
	$p_{n+1}$ is a refinement of $p_n$ and $p_n = p_n'$.
	For each $n$, define $\varphi_n$
	to be the lower step function associated to $f$ with respect to $p_n$.
	That is, $\varphi_n$ agrees with $f$ at each partition point
	of $p_n$ and which on each open interval deteremined by $p_n$ has
	constant value equal to the infimum off on that inteveral.
	We define the upper step function $\psi_n$ in a similar manner.
	By the definition of the Darboux sums,
	\[
		L\left( f, p_n \right)
		= \int_a^b \varphi_n, \qquad
		U\left( f, p_n \right)
		= \int_a^b \psi_n
	\]
	for all $n$.
	Then $\left\{ \varphi_n \right\}$ and $\left\{ \psi_n \right\}$ are
	sequences of integrable functions such that for each index $n$,
	$\varphi_n \leq f \leq \psi_n$ on $E$.
	Moreoever, the sequences $\left\{ \varphi_n \right\}$ is increasing
	and $\left\{ \psi_n \right\}$ is decreasing since each $p_{n+1}$ is a
	refinement of $p_n$.
	Finally,
	\[
		\lim_{n \to \infty}
		\int_a^b \left( \psi_n - \varphi_n \right)
		= \lim_{n \to \infty}
		\left( U(f,p_n) - L(f,p_n) \right)
		= 0
	\]
	TODO
\end{proof}

\subsection{Improper Riemann integral}

\begin{definition}[Improper Riemann integral]
	Let $f: (a,b) \to \R$, $-\infty \leq a < b < \infty$ and assume $f$
	is bounded and Riemann integrable on closed bounded sub-intervals.
	Assume $\lim_{t \to a, t > a} I_t^b(f)$ exists in $\R$.
	Then this limit is defined to be the improper Riemann integral
	$I_a^b(f)$.
	We have a similar definition for $-\infty < a < b \leq \infty$.
\end{definition}

The improper Riemann integral of a function may exists without the function
being integrable (in the sense of Lebesgue).
For examples, $f(x) = \frac{\sin x}x$ on $[0, \infty]$.

\begin{proposition}
	If $f$ is integrable, then the improper Riemann integral is equal to the
	Lebesgue integral whenever the former exists.
\end{proposition}

\begin{proof}
	Suppose $f$ is Lebesgue integral and the Riemann integral
	$I_a^b(f)$ exists with improper lower limit $a$.
	If $a$ is finite, let $f_n = \chi_{\left[ a + \frac1n, b \right]}$.
	Then $f_n \to f$ on $[a,b]$ and $\abs{f_n} \leq \abs f$.
	So $I_a^b(f) = \lim_{n \to \infty} I_{a + \frac 1n}^b(f)
	= \lim_{n \to \infty} \int f_n = \int f$.
	Now let $a = -\infty$, $g_n = f \chi_{\left[ -n, b \right]}$.
	Then $g_n \to f$ on $[a,b]$ and $\abs{g_n} \leq \abs f$ so
	\[
		I_a^b(f) = \lim_{n \to \infty} I_{-n}^b (f)
		= \lim_{n \to \infty} \int g_n = \int f.
	\]
	The cases for improper upper limits are similar.
\end{proof}