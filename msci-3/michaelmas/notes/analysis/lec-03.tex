%! TEX root = master.tex

\subsection{$\limsup$ and $\liminf$}
\lecture{3}{12/10}

\begin{definition}[]
	Let $\left\{ x_n \right\}$ be a sequence of real numbers.
	We define
	\begin{align*}
		\limsup x_n 
		&= \inf_n\left\{ \sup_{k \geq n} \left\{ x_k \right\} \right\}, \\
		\liminf x_n 
		&= \sup_n\left\{ \inf_{k \geq n} \left\{ x_k \right\} \right\}.
	\end{align*}
\end{definition}

\begin{proposition}[]
	Let $\left\{ x_n \right\}$ be a sequence of real numbers.
	Then $\limsup x_n$ is the greatest cluster point of $\left\{ x_n \right\}$
	and $\liminf x_n$ is the least cluster point.
\end{proposition}

\begin{proof}
	By contradiction, we take $l_1$ to be a cluster point such that \[
		l_1 < l = \liminf x_n.
	\]
	By the previous lemma, there is a subsequence $\left\{ x_{n_k} \right\}$
	subseqeunce with limit $l_1$.
	That is, for all $\varepsilon > 0$ there is $N \in \N$ such that for all
	$k \geq N$ we have \[
		\left\lvert x_{n_k} - l_1 \right\rvert < \varepsilon.
	\]
	Now we choose $\varepsilon = \frac{l - l_1}{2}$. So \[
		x_{n_k} < l_1 + \varepsilon = l - \varepsilon.
	\]
	But $l = \liminf x_n = l - \varepsilon$; a contradiction.
	We can use a similar reasoning to obtain the proof for $\limsup$.
\end{proof}

\begin{proposition}[]
	Let $\left\{ x_n \right\}$ be a sequence of real numbers.
	An element $l = \limsup\left\{ x_n \right\}$ if and only if
	\begin{enumerate}
		\item for every $\varepsilon > 0$ there is $N \in \N$
			such that for all $n \geq N$ we have $x_n < l + \varepsilon$; and
		\item for every $\varepsilon > 0$ and $N \in \N$ there is $n \geq N$
			such that $x_n > l - \varepsilon$.
	\end{enumerate}
\end{proposition}

\begin{proof}
	We first prove $(i)$ by contradiction.
	Suppose there is $\varepsilon > 0$ such that for all $n \geq N$
	we have $x_n \geq l + \varepsilon$.
	Then \[
		l = \limsup\left\{ x_n \right\} 
		= \inf_n\left\{ \sup_{k\geq n} \left\{ x_k \right\} \right\}
		\geq l + \varepsilon.
	\]
	Similarly, for $(ii)$
	we assume that there exists $\varepsilon > 0$ and $N \in \N$
	such that for all $n > N$ we have $x_n \leq l - \varepsilon$.
	Now \[
		l = \limsup x_n
		= \inf_n \left\{ \sup_{k \geq n} \left\{ x_k \right\} \right\}
		\leq l - \varepsilon.
	\]
	Now we prove that $(i)$ and $(ii)$ imply our definition of $\limsup$.
	By $(i)$, $\left\{ x_n \right\}$ is bounded from above and by the
	completeness of $\R$, $\sup x_n$ exists and $\sup x_n \leq l + \varepsilon$.
	Moreover, from $(ii)$ the sequence $\left\{ x_n \right\}$ contains an element
	larger than $l - \varepsilon$.
	In particular,
	$\sup\left\{ x_n \right\} \geq l - \varepsilon$.
	Hence 
	\begin{align*}
		l - \varepsilon &\leq \sup x_n \leq l + \varepsilon \\ 
		l - \varepsilon &\leq \limsup x_n \leq l + \varepsilon
	\end{align*}
	and so for arbitrarily small $\varepsilon$ we have $l = \limsup x_n$.
\end{proof}

\begin{proposition}[]
	Let $\left\{ x_n \right\}$ be a sequence of real numbers.
	An element $l = \liminf x_n$ if and only if
	\begin{enumerate}
		\item for all $\varepsilon > 0$ there exists $N \in \N$ such that
			for all $n \geq N$ we have $x_n > l - \varepsilon$; and
		\item for all $\varepsilon > 0$ and $N \in N$ there exists
			$n \geq N$ such that $x_n < l + \varepsilon$.
	\end{enumerate}
\end{proposition}

\begin{solution}
	First we prove $(i)$.
	Assume, for a contradiction, that there is $\varepsilon > 0$ such that for
	all $n \geq N$, $x_n > l - \varepsilon$.
	Then \[
		l = \liminf x_n
		= \sup_n\left\{ \inf_{k\geq n}\left\{ x_k \right\} \right\}
		> l - \varepsilon.
	\]
	We can show similarly that $l < l + \varepsilon$
	by assuming the negation of $(ii)$.
	We have left to show that $(i)$ and $(ii)$ imply our definition of $\liminf$.
	By $(i)$, $\left\{ x_n \right\}$ is bounded and by the completeness of the
	reals $\inf x_n$ exists and $\inf x_n \geq l - \varepsilon$.
	By $(ii)$, there is some element of $\left\{ x_n \right\}$ that is smaller
	than $l + \varepsilon$. 
	So $\inf x_n \leq l + \varepsilon$.
	And so \[
		l - \varepsilon \leq \liminf x_n \leq l + \varepsilon
	\]
	which completes our proof.
\end{solution}

\begin{proposition}[]
	A sequence $\left\{ x_n \right\}$ of real numbers
	is convergent if and only if \[
		\liminf x_n = \limsup x_n.
	\]
\end{proposition}

\begin{proof}
	Let $x_n \to l$.
	So for all $\varepsilon > 0$
	there is $N \in \N$
	such that $\left\lvert x_n - l \right\rvert < \varepsilon$
	for all $n \geq N$.
	But $\left\lvert x_n - l \right\rvert < \varepsilon$ implies that
	$l - \varepsilon < x_n < l + \varepsilon$;
	hence, $\liminf x_n = l = \limsup x_n$ and we have proved
	the $\implies$ of this proposition.
	Now we prove the $\impliedby$.
	For all $\varepsilon > 0$ there is $N \in \N$ such that
	$x_n > l + \varepsilon$ for all $n \geq N$.
	Similarly, $x_n < l - \varepsilon$.
	Therefore $\left\lvert x_n - l \right\rvert < \varepsilon$.
\end{proof}

\begin{theorem}[Bolzano-Weierstrass]
	Every bounded sequence in $\R$ has a convergent subsequence.
\end{theorem}

\begin{proof}
	Let $\left\{ x_n \right\}$ be bounded by $M > 0$.
	Then $l = \limsup x_n \in [-M,M]$.
	We know that $l$ is a cluster point; hence,
	there is a subsequence of $\left\{ x_n \right\}$ that converges to $l$.
\end{proof}
