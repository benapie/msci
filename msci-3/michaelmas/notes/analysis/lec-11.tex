%! TEX root = master.tex
\lecture{11}{17/11}

\begin{lemma}[]
	Every interval is measurable.
\end{lemma}

\begin{proof}
	It is infact enough to show that $(a, \infty)$ is measurable for every
	$a \in \R$ since the set of measurable sets is a $\sigma$-algebra.
	Now let $A \subset \R$, $a \in \R$, $A_1 = A \cap (a, \infty)$, and
	$A_2 = A \cap (-\infty, a]$.
	We must show that $
		m^\star(A) \geq m^\star(A_1) + m^\star(A_2)
	$.
	If $m^\star(A) = \infty$ then we are done. Assume otherwise.
	Let $\varepsilon > 0$, then there is a countable collection of open
	intervals $\left\{
		I_n
	\right\}$ for which \[
		\sum l(I_n) \leq m^\star(A) + \varepsilon.
	\]
	Let 
	$I_n' = I_n \cap (a, \infty)$ 
	and
	$I_n'' = I_n \cap (-\infty, a]$.
	Then
	\[
		l(I_n) = l(I_n') + l(I_n'') = m^\star(I_n') + m^\star(I_n'').
	\]
	Since $A_1 \subset \bigcup I_n'$, we have
	\[
		m^\star(A_1) \leq m^\star\left( 
			\bigcup I_n' 
		\right)
		\leq \sum m^\star(I_n'),
	\]
	and
	\[
		m^\star(A_2) \leq m^\star\left( 
			\bigcup I_n'' 
		\right)
		\leq \sum m^\star(I_n'').
	\]
	Hence
	\begin{align*}
		m^\star(A_1) + m^\star(A_2)
		&\leq \sum \left( 
			m^\star(I_n') + m^\star(I_n'') 
		\right) \\
		&= \sum l(I_n) \\
		&\leq m^\star(A) + \varepsilon.
	\end{align*}
\end{proof}

Every open set is the disjoint union of a countable number of open intervals.
Hence, every open set is measurable.
Furthermore, every closed set is measurable.
The intersection of all $\sigma$-algebras in $\R$ that contain the open sets is a $\sigma$-algebra called the \emph{Borel $\sigma$-algebra}.
Members of this collection are called \emph{Borel sets}.
As the collection of measurable sets contains all open sets and is a
$\sigma$-algebra, every Borel set is measurable.

\begin{definition}[$G_\delta$ and $F_\sigma$ sets]
	A countable intersection of open sets is called a
	\emph{$G_\delta$ set}.
	A countable union of open sets is called a
	\emph{$F_\sigma$ set}.
\end{definition}

Clearly all $G_\delta$ sets and $F_\sigma$ sets are measurable.

\begin{theorem}[]
	The collection of measurable sets is a $\sigma$-algebra that contains
	the $\sigma$-algebra of Borel sets.
	Each interval, each open set, each closed sets, each $G_\delta$, and each
	$F_\sigma$ set is measurable.
\end{theorem}

\begin{proposition}[]
	The translation of a measurable set is measurable.
\end{proposition}

\begin{proof}
	Let $E$ be a measurable set,
	$A \subset \R$, and $y \in \R$.
	By the measurability of $E$ and the translation invariance of outer
	measure, we have
	\begin{align*}
		m^\star(A)
		&= m^\star(A - y) \\
		&= m^\star((A - y) \cap E)
			+ m^\star((A - y) \cap E^c) \\
		&= m^\star(A \cap (E + y)) + m^\star(A \cap (E^c + y)).
	\end{align*}
	Therefore, $E + y$ is measurable.
\end{proof}

\subsection{Regularity of measurable sets}

Measurable sets possess the following \emph{excision} property:
if $A$ is a measureable set of finite outer measure and 
$A \subset B \subset \R$, then
\[
	m^\star\left( 
		B \setminus A 
	\right) = m^\star(B) - m^\star(A).
\]
Indeed, by the measurability of $A$,
\begin{align*}
	m^\star(B) 
	&= m^\star(B \cap A) + m^\star(B \cap A^c) \\
	&= m^\star(A) + m^\star(B \setminus A).
\end{align*}

\begin{proposition}[]
	Let $E \subset \R$.
	The following statements are equivalent.
	\begin{enumerate}
		\item $E$ is measurable.
		\item For every $\varepsilon > 0$ there is an open set $G$ with
			$E \subset G$ and $m^\star(G \setminus E) < \varepsilon$.
		\item For every $\varepsilon > 0$ there is a closed set $F$ with
			$F \subset E$ and $m^\star(E \setminus F) < \varepsilon$.
		\item There is $G \subset G_\delta$ with $E \subset G$ and
			$m^\star(G \setminus E) = 0$; and
		\item There is $F \in F_\sigma$ with $F \subset E$ and
			$m^\star(E \setminus F) = 0$.
	\end{enumerate}
\end{proposition}

\begin{proof}
	$(i) \implies (ii)$: let $E$ be measurable with $m^\star(E) < \infty$.
	Given $\varepsilon > 0$, there is a countable collection $\left\{
		I_n
	\right\}$ such that $\sum l(I_n) \leq m^\star(E) + \varepsilon$.
	$O = \bigcup I_n$ is an open, $O \supset E$, and
	\begin{align*}
		m^\star(O \setminus E)
		&= m^\star\left( 
			\bigcup I_n 
		\right) - m^\star(E) \\
		&\leq \sum l(I_n) - m^\star(E) \\
		&\leq \varepsilon.
	\end{align*}
	Suppose now $m^\star(E) = \infty$.
	For each $n$, let $E_n = [-n,n] \cap E$.
	By a similar argument to before, there is $O_n$ such that
	$O_n \supset E_n$, $m^\star(O_n \setminus E_n) < \frac{\varepsilon}{2^n}$.
	Let $O = \bigcup O_n$.zealios v2 67g
	Then $O \supset E$ and
	\[
		m^\star(O \setminus E)
		= m^\star\left( 
			 \bigcup(O_n \setminus E_n)
		\right)
		< \varepsilon.
	\]

	$(ii) \implies (iv)$: for every $n \in \N$ there is an open set $O_n$
	such that $O_n \supset E$ and $m^\star(O_n \setminus E) < \frac1n$.
	Let $G = \cap O_n$.
	Then $G \supset E$ and 
	$m^\star(G \setminus E) \leq m^\star(O_n \setminus E) \leq \frac{1}{n}$.
	Thus $m^\star(G \setminus E) = 0$.

	$(iv) \implies (i)$: there is $G \in G_\delta$ such that $G \supset E$
	and $m(G \setminus E) = 0$.
	$G$ and $G \setminus E$ are measurable sets, so
	$E = G \setminus (G \setminus E)$ is measurable.

	$(i) \implies (iii)$: suppose $E$ is measurable, then $E^c$ is measurable.
	For every $\varepsilon > 0$ then there is an open $G$ such that
	$E^c \subset G$ and $m^\star(G \setminus E^c) < \varepsilon$.
	Let $F = G^c$, then $F$ is closed,
	$F \subset E$, and
	$m^\star(E \setminus F) = m^\star(E \setminus G^c) = m^\star(E \cap G)
	< \varepsilon$.

	$(iii) \implies (v)$: for every $n \in \N$ there is a closed set $F_n$
	such that $F_n \subset E$ and $m^\star(E \setminus F_n) < \frac1n$.
	Thus $m^\star(E \setminus F) = 0$.

	$(v) \implies (i)$: there is $F \in F_\sigma$ such that $F \subset E$
	and $m^\star(E \setminus F) = 0$.
	$F$ and $E \setminus F$ are measurable, so $E = F \cup (E \setminus F)$
	is a measurable set.
\end{proof}

