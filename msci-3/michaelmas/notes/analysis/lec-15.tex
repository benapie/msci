%! TEX root = master.tex
\lecture{15}{24/11}

\begin{theorem}[Simple approximation]
	An extended real-valued function $f$ on a measurable set $E$
	is measurable if and onl if there is a sequence of simple functions
	$
		\left\{
			\varphi_n
		\right\}
	$
	such that $\varphi_n \to f$ pointwise on $E$ and
	$\left\lvert \varphi_n \right\rvert \leq \left\lvert f \right\rvert$
	on $E$ for all $n \in \N$.
	If $f$ is non-negative, we may take
	$
		\left\{
			\varphi_n
		\right\}
	$
	to be increasing.
\end{theorem}

\begin{proof}
	$(\implies)$:
	as $f$ is the pointwise limit of a sequence of measurable functions,
	$f$ is measurable.
	$(\impliedby)$:
	assume $f$ is measurable and non-negative.
	The general case follows from expressing $f$ as the difference of two
	non-negative functions.
	Let $n \in \N$ and define
	\[
		E_n = \left\{
			x \in E: f(x) \leq n
		\right\}.
	\]
	Then $E_n$ is bounded and $\restr f{E_n}$ is a non-negative, bounded,
	and measurable function.
	By the Simple Approximation Lemma, with $\varepsilon = \frac1n$,
	we may select $\varphi_n$ and $\psi_n$ simple on $E_n$
	such that $\varphi_n(x) \leq f(x) \leq \psi_n(x)$
	and $0 \leq \psi_n(x) - \varphi_n(x) < \frac1n$.
	We extend $\varphi_n$ to all of $E$ by setting $\varphi_n(x) = n$
	if $f(x) > n$.
	$\varphi_n$ is a simple function defined on $E$ and
	$o \leq \varphi_n(x) \leq f(x)$.
	We claim $\varphi_n \to f$ pointwise on $E$.
	Indeed, if we let $x \in E$ we consider two cases.
	First, if $f(x)$ is finite, we choose $N \in \N$ such that
	$f(x) < N$.
	Then
	$0 \leq f(x) - \varphi_n(x) < \frac1n$
	for all $n \geq N$.
	Thus
	\[
		\lim_{n \to \infty} \varphi_n(x) = f(x).
	\]
	Now if we assume $f(x)$ is infinite, then $\varphi_n(x) = n$.
	So $\lim_{n \to \infty} \varphi_n(x) = f(x)$
	and clearly
	$
		\left\{
			\varphi_n
		\right\}
	$
	is increasing.
\end{proof}

\section{Lebesgue integral}

Recall our \emph{canonical representation} of a simple function $\varphi$,
\[
	\varphi(x) = \sum a_i \chi_{E_i}(x), \qquad
	E_i = \left\{
		x \in E: \varphi(x) < a_i
	\right\}.
\]

\begin{definition}[Interal of a simple function]
	For a simple function $\varphi: E \to \R$ with
	$\left\lvert E \right\rvert < \infty$, we define the
	\emph{the integral of $\varphi$ over $E$} by
	\[
		\int_E \varphi =
		\sum_{i=1}^n a_i m(E_i)
	\]
	where $\varphi$ has the canonical representation.
\end{definition}

\begin{lemma}[]
	Let
	$
		\left\{
			E_i
		\right\}_{i=1}^n
	$
	be a collection of disjoint measurable subsets of a set $E$
	with $m(E) < \infty$.
	For $i \in \left\{
		1, \ldots, n
	\right\}$
	we let $a_i \in \R$ be any real number.
	Then
	\[
		\varphi = \sum_{i=1}^n a_i \chi_{E_i} \;\text{on $E$}\; \implies
		\int_E \varphi = \sum_{i=1}^n a_i m(E_i)
	\]
\end{lemma}

\begin{proof}
	Let $\lambda_1, \ldots, \lambda_m$ be the unique values of
	$\varphi$.
	For each $K \in \left\{
		1, \ldots, m
	\right\}$,
	we define \[
		A_j = \left\{
			x \in E: \varphi(x) = \lambda_J
		\right\}.
	\]
	By definition,
	\[
		\int_E \varphi =
		\sum_{J=1}^m \lambda_J m(A_J).
	\]
	Let $I_J$ be the set of indices $i \in \left\{
		1, \ldots, n
	\right\}$
	for which $a_i = \lambda_J$.
	Then
	\[
		\left\{
			1 \ldots, n
		\right\}
		= \bigcup_{J=1}^m I_J
	\]
	and the union is disjoint.
	For each $j \in \left\{
		1, \ldots, m
	\right\}$,
	\[
		m(A_J) = \sum_{i \in I_J} m(E_i).
	\]
	Thus
	\begin{align*}
		\sum_{i=1}^n a_i m(E_i)
		&= \sum_{J=1}^n \left( 
			\sum_{i \in I_J} a_im(E_i) 
		\right) \\
		&= \sum_{J=1}^m \lambda_J \left( 
			\sum_{i\in I_j} m(E_I) 
		\right) \\
		&= \sum_{J=1}^m \lambda_J m(A_J) \\
		&= \int_E \varphi. \qedhere
	\end{align*}
\end{proof}

\begin{proposition}[Linearity and monotonicity of the integral of simple
	functions]
	Let $\varphi, \psi: E \to \R$ be simple functions,
	with $\left\lvert E \right\rvert < \infty$.
	For all $\alpha, \beta \in \R$,
	\[
		\int_E (\alpha \varphi + \beta \psi)
		= \alpha \int_E \varphi + \beta \int_E \psi
	\]
	and
	\[
		\varphi \leq \psi \implies \int_E \varphi \leq \int_E \psi.
	\]
\end{proposition}

\begin{proof}
	Now, as $\varphi$ and $\psi$ are simple,
	$\left\lvert \varphi(E) \right\rvert,
	\left\lvert \psi(E) \right\rvert < \infty$.
	So we can choose
	$
		\left\{
			E_i
		\right\}_{i=1}^n
	$
	be a disjoint collection of subsets of $E$ such that
	$\bigcup E_i = E$ and both $\varphi$ and $\psi$ are constant
	on $E_i$ fo reach $i \in \left\{
		1, \ldots, n
	\right\}$. 
	Let $a_i$ be the value of $\varphi$ on $E_i$ and $b_i$
	be the value of $\psi$ on $E_i$.
	By the previous Lemma,
	\begin{align*}
		\int_E \varphi &= \sum_{i=1}^n a_i m(E_i), \\
		\int_E \psi &= \sum_{i=1}^n b_i m(E_i).
	\end{align*}
	The simple function $\alpha\varphi + \beta\psi$ takes constant
	value $\alpha a_i + \beta b_i$ on $E_i$.
	Then
	\begin{align*}
		\int_E \alpha \varphi + \beta \psi
		&= \sum (\alpha a_i + \beta b_i) m(E_i) \\
		&= \alpha \sum a_i m(E_i) + \beta \sum b_i m(E_i) \\
		&= \alpha \int_E \varphi + \beta \int_E \psi.
	\end{align*}
	Now to prove monotonicty.
	Let $\eta = \psi - \varphi \geq 0$.
	Then
	\[
		\int_E \psi - \int_E \varphi
		= \int_E (\psi - \varphi)
		= \int_E \eta
		\geq 0. \qedhere
	\]
\end{proof}
