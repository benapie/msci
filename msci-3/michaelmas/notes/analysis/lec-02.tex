%! TEX root = master.tex

\lecture{2}{9/10}

\begin{proposition}[]
	The set $\Q$ is dense in $\R$.
	That is, if $x, y \in \R$
	with $x < y$ then there is $r \in \Q$ such that
	$r \in (x,y)$.
\end{proposition}

\begin{proof}
	Let $x, y \in \R$ with $x < y$
	and first suppose that $x \geq 0$.
	Then $y - x > 0$.
	Consider $\frac1{y-x} \in \R$.
	By the Archimedian principle there is $q \in \N$
	such that \[
		q > \frac{1}{y-x},
	\]
	and so $y - x > \frac1q$.
	Consider the set \[
		S = \left\{ k \in \Z: y < \frac{k}{q} \right\}.
	\]
	Clearly $S$ is non-empty and bounded by below.
	Take $p \in \Z$ to be the smallest element of $S$.
	Then $p - 1$ is not in $S$ and so \[
		\frac{p-1}{q} \leq y < \frac{p}{q}
	\]
	thus we can write \[
		x = y - (y - x) < \frac pq - \frac1q = \frac{p-1}{q}.
	\]
	Take $r = \frac{p-1}q$ and we see $x < r < y$.
	Now we suppose $x < 0$.
	By the archimedian property there is $n \in N$
	such that $n > -x$.
	In particular, $n + x > 0$.
	As $n + x < n + y$,
	from the previous argument there is $r \in \Q$ such that \[
		n + x < r < n + y
	\]
	and thus $x < r - n < y$.
\end{proof}

\begin{definition}[]
	We define the set of extended real numbers as 
	$\R \cup \left\{ -\infty, \infty \right\}$
	with $+, \cdot$ satisfying the following:
	\begin{enumerate}
		\item $a + \infty = \infty$; and
		\item $a \cdot \infty = \begin{cases}
				\infty  & a > 0 \\
				-\infty & a < 0 \\
		\end{cases}$.
	\end{enumerate}
\end{definition}


\subsection{Countability and uncountability}

\begin{definition}[]
	A set $A$ is countable if it is empty, 
	finite, 
	or there is a bijection
	$\varphi: \N \to A$.
	Otherwise, $A$ is uncountable.
\end{definition}

\begin{proposition}[]
	The set $\Q$ is countable.
\end{proposition}

\begin{proof}
	The map \[
		\phi: \N \to \left\{ 
			\frac11,
			\frac21,
			\frac12,
			\frac13,
			\frac22,
			\frac31,
			\ldots
		\right\}
	\]
	is a bijection from $\N$ to $\Q_{>0}$.
	We now the bijection $\varphi: \N \o \Q$ as follows:
	\begin{align*}
		\varphi(0) &= 0 \\
		\varphi(2n) &= \phi(n) \\
		\varphi(2n + 1) &= -\phi(n). \qedhere
	\end{align*}
\end{proof}

\begin{proposition}[]
	$\R$ is uncountable.
\end{proposition}

\begin{proof}
	It is sufficient to show that $[0,1]$ is uncountable.
	Suppose $\varphi: \N \to [0,1]$ is a bijection.
	Denote $\phi_n = \varphi(\{1, \ldots, n\})$.
	We may assume (without loss of generality)
	that $\varphi(1) = 1$.
	We are going to prove that for all $n \in \N$
	there exists an open interval $I_n = (y_n, z_n)$
	such that $I_n \cap \phi_n = \varnothing$ and the closure is strictly
	contained with $I_{n - 1}$. 
	The proof is by induction. So we start with the base case.
	For $n = 1$, we take $I_1 = (0,1)$.
	Now we assume that there exists an $I_n$ as stated.
	As $\Q$ is dense in $\R$,
	there exists $m > n$ such that $\varphi(m) \in I_n$
	and there exists $m' > m$ with $\varphi(m') \in (\varphi(m), z_n)$.
	We define $y_{n+1} = \varphi(m)$ and $z_{n+1} = \varphi(m')$,
	so $I_{n+1} = (y_{n+1}, z_{n+1})$. 
	By construction $I_{n+1} \cap \phi_{n+1} = \varnothing$
	and the closure $\overline I_{n+1}$ is strictly contained within $I_n$.

	Now we consider the set $A = \{y_n : n \in \N\}$.
	$A$ is clearly bounded from above, then by the completeness of $\R$
	there is $c = \sup A$ with $c > y_n$ for all $n \in \N$.
	Observe that for all $n, m \in \N$, $z_n > y_m$.
	So $z_n$ is an upper bound for $A$ for all $n \in \N$.
	Hence $y_n < c < z_n$ and so $c \in I_n$.
	Moreover, as $\varphi$ is a bijection, there is $n_0 \in \N$ such that
	$c \in \phi_{n_0}$;
	a contradiction since $I_n \cap \phi_n = \varnothing$.
\end{proof}

\section{Sequences in $\R$}

\begin{definition}[Sequence of real numbers]
	A sequence $\left\{ x_n \right\}$ of real numbers is a map
	\[
		x: \N \to \R, \qquad
		x: m \in \N \to x_m \in \R.
	\]
\end{definition}

\begin{definition}[Limit of a sequence]
	A real number $l$ is a \emph{limit of a sequence} $\left\{ x_n \right\}$ if
	for every $\varepsilon > 0$ there is $N \in \N$ such that for all $n \geq N$
	we have $\left\lvert x_n - l \right\rvert < \varepsilon$.
	We denote this $x_n \to l$.
\end{definition}

\begin{definition}[Limit of infinity]
	We say that $\lim_{n \to \infty} x_n = \infty$
	if for all $\Delta > 0$ there is $N \in \N$ such that
	for all $n \geq N$ we have $x_n > \Delta$.
\end{definition}

\begin{definition}[Cluster point]
	Let $\left\{ x_n \right\}$ be a sequence of real numbers.
	An element $l \in \R$ is a \emph{cluster point} of $x_n$
	if for every $\varepsilon > 0$ and $N \in N$
	there exists $n \in \N$ such that $n > N$ and
	$\left\lvert x_n - l \right\rvert < \varepsilon$.
\end{definition}

If $x_n \to l$, then $l$ is a cluster point.

\begin{lemma}[]
	If a sequence of real numbers has a cluster point, 
	then that point is a limit of some subsequence.
\end{lemma}

\begin{proof}
	Let $\left\{ x_n \right\}$ be a sequence of real numbers
	and let $\left\{ x_{n_k} \right\}$ be a subsequence.
	Let $\varepsilon = \frac1k$ and $N = k$, then by the definition of cluster
	points there exists $n_k \geq N$ with 
	$\left\lvert x_{n_k} - l \right\rvert < \varepsilon$
	and hence $l = \lim x_{n_k}$.
\end{proof}

