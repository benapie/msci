%! TEX root = master.tex
\lecture{17}{30/11}

\begin{proposition}[]
	Let $f: E \to \R$ be a non-negative function for some $E \subset \R$
	and $\lambda \geq 0$ be a non-negative real number.
	Then
	\[
		m\left( 
			\left\{
				x \in E: f(x) \geq \lambda y
			\right\} 
		\right)
		\leq
		\frac{1}{\lambda}
		\int_E f. \tag{$\star$}
	\]
\end{proposition}

\begin{proof}
	Let
	$
		E_\lambda =
		\left\{
			x \in E:
			f(x) \geq \lambda
		\right\}
	$.
	Suppose $m(E_\lambda) = \infty$,
	Let $n \in \N$. 
	Define
	$
		E_{\lambda,n} = E_\lambda \cap [-n,n]
	$
	and
	$
	\psi_n = \lambda \chi_{E_{\lambda,n}}
	$.
	$\psi_n$ is bounded and a function of finite support,
	so
	\[
		\lambda m \left( E_{\lambda, n} \right)
		= \int_{E_{\lambda,n}} \psi_n.
	\]
	Furthermore, $0 \leq \psi_n \leq f$
	for each $n$. 
	Therefore
	\[
		\infty 
		= \lambda m \left( E_{\lambda} \right)
		= \lim_{n \to \infty} \left( 
			\lambda m \left( E_{\lambda, n} \right) 
		\right)
		= \lim_{n \to \infty} \int_E \psi_n
		\leq f
	\]
	and so $\int_E f = \infty$; hence, $(\star)$ holds.
	Now suppose $m(E_\lambda) < \infty$.
	Let $h = \lambda \chi_{E_\lambda}$.
	$h$ is bounded and of finite support and
	$0 \leq h \leq \lambda < f$.
	So
	\[
		h m \left( E_\lambda \right) = \int_E h \leq \int_E f
	\]
	therefore
	\[
		m \left( 
			\left\{
				x \in E: f(x) > \lambda
			\right\} 
		\right)
		\leq
		\frac{1}{\lambda} \int_E f. \qedhere
	\]
\end{proof}

\begin{proposition}[]
	Let $f: E \to \R$ be a non-negative function on some $E \subset \R$.
	Then
	\[
		\int_E f = 0 \iff \text{$f = 0$ almost everywhere on $E$}.
	\]
\end{proposition}

\begin{proof}
	$(\implies)$: assume $\int_E f = 0$.
	Then
	\[
		m\left( 
			\left\{
				x \in E: f(x) \geq \frac1n
			\right\} 
		\right)
		\leq \frac1n \int_E f = 0,
	\]
	so
	\[
		\left\{
			x \in E: f(x) > 0
		\right\}
		\bigcup
		\left\{
			x \in E: f(x) \geq \frac1n
		\right\}
	\]
	and thus
	\[
		m \left( 
			\left\{
				x \in E: f(x) > 0
			\right\} 
		\right)
		= 0.
	\]
	Hence $f = 0$ almost everywhere on $E$.
	
	$(\impliedby)$: assume $f = 0$ almost everywhere on $E$.
	Let $\varphi$ be a simple function
	and $h$ be a bounded function of finite support such that
	\[
		0 \leq \varphi \leq h \leq f \quad\text{on $E$}.
	\]
	Since $f = 0$ almost everywhere on $E$,
	$\varphi = 0$ almost everywhere on $E$.
	Thus
	\[
		\int_E \varphi = 0
	\]
	and so
	\[
		\int_E h = 0
	\]
	and finally
	\[
		\int_E f = 0.
	\]
\end{proof}

\begin{theorem}
	[Linearity and monotonicty of the integral of measurable functions]
	Let $f, g: E \to \R$ be measurable functions on some
	measurable set $E \subset \R$.
	Then for every non-negative real numbers $\alpha, \beta > 0$
	we have
	\[
		\int_E \alpha f + \beta g
		= \alpha \int_E f + \beta \int_E g
	\]
	and
	\[
		f \leq g \;\text{on $E$}
		\implies \int_E f \leq \int_E g.
	\]
\end{theorem}

\begin{proof}
	For all $\alpha > 0$ and $h: E \to \R$ bounded and of finite support,
	\[
		0 \leq h \leq f \;\text{on} E
		\iff 0 \leq \alpha h \leq \alpha f.
	\]
	Now
	\[
		\int_E \alpha h = \alpha \int_E h
	\]
	and so we can see that
	\[
		\int_E \alpha f = \alpha \int_E f
	\]
	(argument needed here TODO) %TODO.
	It is thus enough to prove linearity for $\alpha = \beta = 1$.
	Let $h, k$ be bounded measurable functions of finite support with
	$0 \leq h \leq f$ and $0 \leq k \leq g$ on $E$.
	Thus $0 \leq h + k \leq f + g$ and so
	\[
		\int_E h + \int_E k
		= \int_E h + k
		\leq \int_E f + g
	\]
	and so $\int_E f + g$ is an upper bound for
	$\int_E h + \int_E k$.
	But $\int_E f + \int_E g$ is the least upper bound for
	$\int_E h + \int_E k$, so
	\[
		\int_E f + g \leq \int_E f + \int_E g.
	\]
	It remains to prove (for linearity)
	\[
		\int_E f + g \leq \int_E f + \int_E g.
	\]
	That is,
	for each bounded measurable function $l$ with finite support
	with $0 \leq l \leq f + g$ on $E$ we have
	\[
		\int_E l \leq \int_E f + \int_E g.
	\]
	We define $h, k$ on $E$ such that
	\[
		h = \min\{f,l\}, \qquad k = l - h.
	\]
	These are measurable, bounded, and of finite support.
	Let $x \in E$.
	If $l(x) \leq f(x)$, then $k(x) = l(x) - l(x) = 0 \leq g(x)$.
	If $l(x) > f(x)$, then $h(x) = f(x)$ and
	$k(x) = l(x) - f(x) \leq f(x) + g(x) - f(x) = g(x)$.
	So $k \leq g$ and $h \leq f$.
	By definition, we have $0 \leq h \leq f$
	and $0 \leq k \leq g$
	and so $l = h + k$ on $E$.
	Thus
	\[
		\int_E l
		= \int_E h + k
		= \int_E h + \int_E k
		\leq \int_E f + \int_E g.
	\]
	To prove monotonicty, it is enough to show if $h$ is a bounded measurable
	function of finite support such that $0 \leq h \leq f$, we have
	\[
		\int_E h \leq \int_E g.
	\]
	Let $h$ be a measurable bounded function of finite support
	with $0 \leq h \leq f \leq g$.
	By definition,
	\[
		\int_E h \leq \int_E g.
	\]
\end{proof}

\begin{theorem}[Additivity over domains of integrals]
	Let $f: E \to \R$ be a non-negative function on some measurable
	set $E \subset \R$.
	Let $A$ and $B$ be two disjoint measurable sets.
	Then
	\[
		\int_{A \cup B} f = \int_A f + \int_B f.
	\]
	If $E_0 \subset E$ such that $m(E_0) = 0$,
	then
	\[
		\int_E f = \int_{E \setminus E_0} f.
	\]
\end{theorem}

\begin{proof}
	todo %TODO
\end{proof}

\begin{lemma}[Fatou's Lemma]
	Let
	$
		\left\{
			E_n
		\right\}_{n=1}^\infty
	$
	be a collection of non-negative measurable functions
	with common domain $E$.
	If $f_n$ converges pointwise almost everywhere on $E$
	to some limit function $f:E \to \R$, then
	\[
		\int_E f \leq \liminf \int_E f_n. \tag{$\star$}
	\]
\end{lemma}

\begin{proof}
	As
	\[
		\int_E f = \int_{E \setminus E_0} f
	\]
	for all $E_0 \subset E$ such that $m(E_0) = 0$,
	we may assume $f_n \to f$ on all of $E$
	(effectively replacing $E$ with $E \setminus E_0$).
	To prove $(\star)$, we need to show that if $h$ is any
	bounded measurable function of finite support such that 
	$0 \leq h \leq f$, then
	\[
		\int_E h \leq \liminf \int_E f_n.
	\]
	Let $h$ be such a function.
	Now $h$ is bounded, so we let $M > 0$ such that 
	$\left\lvert h \right\rvert \leq M$.
	We define
	\[
		E_0 = \left\{
			x \in E: h(x) \neq 0
		\right\}
	\]
	and note $m(E_0) < \infty$.
	Let $n \in\ N$ and define
	$
		h_n = \min\{h, f_n\}
	$.
	$h_n$ is measurable and $0 \leq h_n \leq M$ on $E_0$
	and $h_n = 0$ on $E \setminus E_0$.
	Moreover, for every $x \in E$ we have $h(x) \leq f(x)$
	and as $f_n \to f$, $h_n \to h$.
	By the bounded convergence theorem (applying to $h_n$),
	\[
		\lim_{n \to \infty} \int_E h_n
		= \lim_{n \to \infty} \int_{E_0} h_n
		= \int_{E_0} h
		= \int_E h
	\]
	so $h_n \leq f_n$ on $E$.
	Therefore,
	\[
		\int_E h_n \leq \int_E f_n
	\]
	and so 
	\[
		\int_E h 
		= \lim_{n \to \infty} \int_E h_n
		= \liminf \int_E f_n. \qedhere
	\]
\end{proof}

