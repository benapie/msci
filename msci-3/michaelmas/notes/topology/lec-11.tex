%! TEX root = master.tex

\section{Interiors and closures, limit points, and product spaces}
\lecture{11}{?}

\begin{definition}[]
	Let $X$ be a topological space, $x \in X$, and $A \subset X$.
	\begin{enumerate}
		\item An \emph{open neighbourhood} of $x$ is an open set $N$ containing
			$x$.

		\item A point $x \in X$ is a \emph{limit point} of $A$ if every open
			neighbourhood $N$ of $x$ satisfies 
			$(N  \setminus \left\{ x \right\}) \cap A \neq \varnothing$.
	\end{enumerate}
\end{definition}

\begin{examples}
	\begin{enumerate}
		\item Let $X = \R$ with the standard topology and consider $0 \in X$.
			$\left( -\frac12, \frac12 \right)$ is an open neighbourhood of $0$.

		\item Let $X$ be a topological space, $U \subset X$ be open and 
			$p \in U$.
			Then $U$ is an open neighbourhood of $p$.

		\item Let 
			$
				A = 
				\left\{ 
					\frac1n : n \in \Z \setminus \left\{ 0 \right\} 
				\right\} \subset \R
			$.
			Clearly $0$ is a limit point of $A$.
			Now let $x \in \R \setminus \left\{ 0 \right\}$.
			Then there is $\varepsilon > 0$ such that
			\[
				(x - \varepsilon, x + \varepsilon) \cap A =
				\begin{cases}
					\varnothing & x \not\in A, \\
					\left\{ x \right\} & x \in A.
				\end{cases}
			\]
			Hence, $0$ is the only limit point of $A$.
	\end{enumerate}
\end{examples}

\begin{definition}[]
	Let $X$ be a topological space and $A \subset X$.
	The \emph{interior} of $A$, denoted $\mathring A$, is the largest open set
	contained in $A$.
	The \emph{closure} of $A$, denoted $\overline A$,
	is the smallest closed set containing $A$.
	More precisely
	\[
		\mathring A =
		\bigcup_{\text{open}\, U \subset A} U, \qquad
		\overline A=
		\bigcap_{\text{closed}\, C \supset A} C.
	\]
	We say that $A$ is \emph{dense} in $X$ if $\overline A = X$.
\end{definition}

Like we have with complements of open and closed sets, we also observe that
for a space $X$ and $A \subset X$ we have
$
	\overline{X \setminus A} = X \setminus \mathring A
$.

\begin{example}[]
	Let $X = \R$ with the standard topology and $\Q \subset X$.
	Note that no non-empty subset of $\R$ contains only rational numbers.
	Hence, $\mathring \Q = \varnothing$.
	Now 
	$
		\overline Q = 
		\R \setminus \left( \mathring{\R \setminus \Q} \right) = 
		\R \setminus \varnothing = \R
	$
	since $\mathring{\R \setminus \Q}$; hence, $\Q$ is dense in $\R$.
\end{example}

\begin{lemma}[]
	Let $X$ be a topological space and $A \subset X$.
	Then
	\[
		\overline A = A \cup \left\{ \text{limit points of $A$} \right\}.
	\]
\end{lemma}

\begin{proof}
	Suppose that $x \not\in \overline A$.
	Then $x \in X \setminus \overline A$, which is open.
	$
		(X \setminus \overline A) \cap A \subset 
		(X \setminus \overline A) \cap \overline A =
		\varnothing.
	$
	Hence $x$ is not a limit point of $A$, so $\text{RHS} \subset \text{LHS}$.
	Now we suppose that $x \not\in A$ and is also not a limit point of $A$.
	Then there is an open $U \subset X$ such that $x \in U$ and
	$U \cap A \in \left\{ \varnothing, \left\{ x \right\} \right\}$.
	As $x \not\in A$, we must have $U \cap A = \varnothing$.
	Then
	\[
		x \not\in X \setminus U \supset
		\overline A.
	\]
	Hence $x \not\in \overline A$.
	Therefore, we have $\text{LHS} = \text{RHS}$.
\end{proof}

