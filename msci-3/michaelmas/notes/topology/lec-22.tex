%! TEX root = master.tex
\lecture{22}{6/12}

\begin{definition}[Action of a group]
	An \emph{action of a group} $G$ on a topological space $X$ is a
	map $\cdot : G \times X \to X$ such that
	\begin{enumerate}
		\item $(hg) \cdot x = h \cdot (g \cdot x)$
			for every $h,g \in G$, $x \in X$;
		\item $1 \cdot x = x$ for every $x \in X$; and
		\item $g: G \to G$, $g(x) = g \cdot x$ is continuous
			for every $g \in G$.
	\end{enumerate}
\end{definition}

Now when $G$ is a topological group, we replace (iii) with the 
requirement that $\cdot: G \times X \to X$ is continuous.
Note that $X \xhookrightarrow{} G \times X \xrightarrow{\cdot}$ defines
$g$.

\begin{examples}
	\begin{enumerate}
		\item Trivial action:
		$G \times X \to X$, $(g,x) \mapsto g \cdot x = x$.

		\item Let $G = \operatorname{GL}_n(\R)$, $X = \R^n$.
		We define $(A, \bm v) \mapsto A \bm v$.
		This is continuous.
		Similarly, $\operatorname{O}(n)$ and $\operatorname{SL}(n)$
		act on $\R^n$ and similarly $S^{n-1}$.

		\item For a topological group $G$
		and $H \subset G$ subgroup, 
		then $H$ acts on $G$ by $h \cdot g = hg$.

		\item Let $N$ be a normal subgroup of a topological group
		$G$. Then $G$ acts on $N$ by
		$g \cdot n = gng^{-1}$.

		\item $S^3$ acts on $\R^4$ by quaternion multiplication.
	\end{enumerate}
\end{examples}

\begin{definition}[Orbit]
	If a group $G$ acts on a topological space $X$,
	we can define an equivalence relation $\sim$ on $X$ by saying
	$x \sim y$ if there is $g \in G$ with $gx = y$.
	An equivalence class is called an \emph{orbit}, denoted
	$Gx$.
	The corresponding quotient space is called the \emph{orbit space},
	denoted $X/G$.
	If $X/G$ is just a point, then the action is called \emph{transitive}.
\end{definition}

\begin{examples}
	\begin{enumerate}
		\item If $G$ acts trivially on $X$, every orbit is a point
		and $X/G = X$.

		\item The action of $\operatorname{O}(n)$ on $S^{n-1}$
		is transistive for $n \geq 1$.
		If $v_1 \in S^{n-1}$, complete to orthonormal basis
		$(v_1, \ldots, v_n)$.
		Then take $A \in \operatorname{O}(n)$,
		$A: e_i \mapsto v_i$ where
		$\left\{ e_1, \ldots, e_n \right\}$ is the standard basis.
		So $\operatorname{O}(n) e_1 = S^{n-1}$.
		$\operatorname{SO}(n)$ acts on $S^{n-1}$ transistively
		if and only if $n \geq 2$.
	\end{enumerate}
\end{examples}

\begin{theorem}
	Let $G$ be a connected topological group that acts on a topological space
	$X$ such that $X/G$ is connected.
	Then $X$ is connected.
\end{theorem}

\begin{proof}
	Let $x \in X$, $f_x: G \to X$, $g \mapsto g \cdot x$.
	This map is continuous.
	\begin{center}
		\begin{tikzcd}
			G \arrow[r, "{g \mapsto (g,x)}"] \arrow[rr, "f_x"', bend right] 
			& G \times X \arrow[r, "\cdot"] & X
		\end{tikzcd}
	\end{center}
	$\operatorname{im}(f_x) = Gx$, hence orbits are connected as continuous
	images of connected spaces are connected. 
	Suppose $X = U \cup V$ for disjoint open sets $U$ and $V$.
	THen $Gx = (U \cap Gx) \cup (V \cap Gx)$ partitions $Gx$ into two
	clopen sets.
	Hence either $Gx \subset U$ or $Gx \subset V$.
	Thus $U$ and $V$ are both unions of orbits.
	Let $\pi: X \to X/G$.
	We have $U = \pi^{-1}(A)$ and 
	$V = \pi^{-1}(B)$ for some $A, B \subset X/G$ with 
	$A \cap B = \varnothing$, $A \cup B = X/G$.
	$A$ and $B$ are both open by definition of quotient topology.
	Hence either $X/G = A$ or $X/G = B$.
	Thus $X = U$ or $X = V$.
\end{proof}