%! TEX root = master.tex
\lecture{13}{5/11}

\begin{theorem}[]
	Let $X$ be a set and $\mathcal B \subset \mathcal P(X)$
	such that
	\begin{enumerate}
		\item for all $x \in X$
			there is $B \in \mathcal B$
			such that $x \in B$; and

		\item if $x \in B_1 \cap B_2$ for some
			$B_1, B_2 \in \mathcal B$ then
			there is $B_3 \in \mathcal B$
			such that $x \in B_3 \subset B_1 \cap B_2$.
	\end{enumerate}
	Then there is a unique topology $\tau_{\mathcal B}$
	on $X$ of which $\mathcal B$ is a basis.
\end{theorem}

\begin{proof}
	Uniqueness:
	suppose $\mathcal B$ is a basis for $\tau_{\mathcal B}$.
	Then if $B_i \in \mathcal B$ with $i \in I$, we have
	\[
		\bigcup_{i\in I} B_i \in \tau_{\mathcal B}.
	\]
	Now consider
	\[
		\tau = \left\{
			\bigcup_{i \in I} B_i:
			B_i \in \mathcal B, \text{$I$ indexing set}
		\right\}.
	\]
	Note that $\tau \subset \tau_{\mathcal B}$.
	Also observe that if $U \in \tau_{\mathcal B}$ and $p \in U$ then
	there is $B_p \in \mathcal B$ such that
	$p \in B_p \subset U$.
	So
	\[
		U = \bigcup_{p \in U} \left\{ p \right\}
		\subset \bigcup_{p \in U} B_p \subset U.
	\]
	So $U \in \tau$ and so $\tau_{\mathcal B} \subset \tau$
	and so we have equality.

	Existence: define
	\[
		\tau_{\mathcal B} =
		\left\{
			\bigcup_{i \in I} B_i :
			B_i \in \mathcal B, I
		\right\}.
	\]
	Note that $\emptyset \in \tau_{\mathcal B}$.
	For each $p \in X$ there is $B_p \in \mathcal B$ such that
	$p \in B_p$.
	Hence
	\[
		x = \bigcup_{p \in X} B_p \in \tau_{\mathcal B}.
	\]
	Suppose $U_1, U_2 \in \tau$.
	Then there is labelling sets $I$ and $J$ such that
	\[
		U_1 = \bigcup_{i \in I} B_i, \qquad
		U_2 = \bigcup_{j \in J} B_j.
	\]
	Now
	\[
		U_1 \cap U_2 =
		\bigcup_{(i,j) \in I \times J} \left( B_i \cap B_j \right).
	\]
	For every $p_{ij} \in B_i \cap B_j$ there is $B_{p_{ij}} \in \mathcal B$
	such that $p_{ij} \in B_{p_{ij}} \subset B_i \cap B_j$.
	Therefore
	\[
		B_i \cap B_j = \bigcup_{p_{ij} \in B_i \cap B_j} B_{p_{ij}}.
	\]
	Hence
	\[
		U_1 \cap U_2 =
		\bigcup_{(i,j) \in I \times J}
		\left( 
			\bigcup_{p_{ij} \in B_i \cap B_j} B_{p_{ij}}
		\right)
		\in \tau_{\mathcal B}.\qedhere
	\]
\end{proof}

\begin{definition}[Cartesian product]
	Let $X$ and $Y$ be spaces.
	We define the \emph{cartesian product} of $X$ and $Y$
	to be
	\[
		X \times Y = \left\{
			(x,y) : x \in X, y \in Y
		\right\}
	\]
	and it is given the \emph{product topology}
	obtained through the previous theorem as
	\[
		\mathcal B_{X \times Y} = \left\{
			U \times V:
			\text{open $U \subset X$},
			\text{open $V \subset Y$}
		\right\}.
	\]
\end{definition}

\begin{examples}
	\begin{enumerate}
		\item 
		Let $X = Y = \R$ both equipped with standard topologies.
		Then $X \times Y = \R^2$ amd the product topology agrees
		with the standard topology.
		
		\item
		Consider $T^2 = S^1 \times S^1$,
		we call this space a \emph{torus}.
	\end{enumerate}
\end{examples}

\begin{definition}[]
	Let $X$ be a set and $\tau$ and $\tau'$ be topologies on $X$.
	If $\tau \subset \tau'$ we say that $\tau$ is \emph{smaller} than
	$\tau'$ or that $\tau'$ is \emph{bigger} than $\tau$.
\end{definition}

Terminology for the above definitions can often be confusing,
a lot of text may use the terms \emph{coarser} and
\emph{finer}.
