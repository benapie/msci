%! TEX root = master.tex


\begin{lemma}[]
	Let $X$ be a topological space.
	$X$ and $\varnothing$ are closed.
	Furthermore, arbitrary intersections and finite unions of closed sets are
	closed.
\end{lemma}

\begin{proof}
	$X = X \setminus \varnothing$ and $\varnothing = X \setminus X$.
	Now suppose $V_i$ is closed for $i \in I$.
	$X \setminus V_i$ is open.
	$X \setminus \bigcap_i V_i = \bigcup_i \left( X \setminus V_i \right)$;
	hence, arbitrary intersections are closed.
	Similarly
	$X \setminus \bigcup_{i = 1}^n V_i = \bigcap_{i=1}^n (X \setminus V_i)$
	and so finite unions are closed.
\end{proof}

\begin{definition}[]
	Suppose $X$ is a topological space such that for all distinct $x,y \in X$
	there exists open sets $U, V \subset X$ such that $x \in U$, $y \in V$, and
	$U \cap V = \varnothing$.
	We call $X$ a \emph{Hausdorff} space.
\end{definition}

Hausdorffness is an example of a \emph{separation axiom}.

\begin{lemma}[]
	Let $M$ be a metric space.
	Then $M$ is a Hausdorff space.
\end{lemma}

\begin{proof}
	Let $x, y \in M$ and $R = d(x,y) > 0$ where $d$ is the metric of $M$.
	Now let $z \in B_{\frac R2}(x) \cap B_{\frac R2}(y)$.
	Then
	\begin{align*}
		R
		&= d(x,y) \\
		&\leq d(x,z) + d(z,y) \\
		&< \frac R2 + \frac R2 \\
		&= R;
	\end{align*}
	hence, no such $z$ exists and 
	$B_{\frac R2}(x) \cap B_{\frac R2}(y) = \varnothing$.
\end{proof}

\begin{example}[]
	Consider the topological space $(X, \left\{ \varnothing, X \right\})$ where
	$X$ is a set with more than one element.
	Let $x, y \in X$ be distinct points. 
	Then the only open set such that $x \in U$ is $X$, and $y \in U$.
	Therefore, this is not a Hausdorff space.
\end{example}

\begin{example}[Furstenburg topology on $\Z$]
	Consider the topology on $\Z$ such that $U \subset \Z$ is open if for all
	$a \in \Z$ there is $0 \neq d \in \Z$ such that $a + d\Z \subset U$.
	Lets investigate this topology.
	Clearly $\varnothing$ and $\Z$ are open.
	Let $U_1$ and $U_2$ be open and $a \in U_1 \cap U_2$ with $d_i \in \Z$ such
	that $a + d_i\Z \subset U_i$ for $i \in \left\{ 1, 2 \right\}$.
	See that $a + (d_1d_2)\Z \subset U_1 \cap U_2$ and so a finite intersections
	of open sets are open.
	Now let $U_i$ be open for $i \in I$ for some collection $I$.
	Let $a \in \bigcup_i U_i$ and $d \in \Z$ such that 
	$a + d\Z \subset U_1$.
	Now $a + d\Z \subset U_1 \subset \bigcup_i U_i$; hence, $\bigcup_i U_i$ is 
	open.
	This is also Hausdorff, which we will prove.
	Suppose $a, b \in \Z$ are distinct integers.
	Let $U = a + 2(b-a) \Z \ni a$ and $V = b + 2(b-a)\Z \ni b$.
	If $z \in U \cap V$, we have $z = a + 2(b - a)m = b + 2(b-a)n$ and so
	$b - a = 2(b-a)(m-n)$ for some $m, n \in \Z$; a contradiction.
	Therefore $U \cap V = \varnothing$.
\end{example}

There is an interesting question of whether the Furstenburg topology on $\Z$
arises from a metric on $Z$.
