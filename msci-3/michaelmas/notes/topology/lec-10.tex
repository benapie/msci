%! TEX root = master.tex
\lecture{10}{?}

\begin{example}[Furstenberg's proof for the infitude of primes]
	Consider the space $(\Z, \tau)$ where $\Z \supset U \in \tau$ if and only if
	for all $a \in U$ there is non-zero $b \in \Z$ such that 
	$a\Z + b \subset U$.
	This is Furstenberg's topology, as introduced earlier.
	We may also phrase this as: ``\emph{the sets $a\Z + b$, $a \in \Z$, and 
	$0 \neq b \in \Z$ form a basis for the topology $\tau$}''.
	We have not yet covered bases of topologies yet, but it is upcoming.
	Now, observe that $b \subset \Z$ is open and since
	\[
		b\Z = \Z \setminus \bigcup_{i=1}^{b-1} \left( i + b\Z \right)
	\]
	it is also closed.
	Also, note that the complement of any non-empty finite set cannot be closed.
	Now every number (except $1$ and $-1$) is either prime or divisible by some
	prime.
	So 
	\[
		\Z \setminus \left\{ 1, -1 \right\} = 
		\bigcup_{\text{$p$ prime}} p \Z.
	\]
	If there were finitely many primes, then the RHS of the equation above would
	be the finite union of closed sets.
	But we have seen that the LHS of the equation above is \emph{not} closed. 
	\qed
\end{example}
