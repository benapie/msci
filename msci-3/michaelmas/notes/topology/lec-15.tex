%! TEX root = master.tex
\lecture{15}{11/11}

\section{Connectedness}

\begin{definition}[Clopen]
	If $X$ is a topological space then we say $A \subset X$ is \emph{clopen}
	if it is open and closed.
\end{definition}

For any space $X$, clearly $X$ and $\varnothing$ are clopen.

\begin{definition}[Connected]
	A topological space $X$ is called \emph{connected} if the only clopen
	subsets of $X$ is $X$ and $\varnothing$.
\end{definition}

\begin{problem}
	Show that $\R$ is connected.
\end{problem}

\begin{solution}
	Suppose $U \subset \R$ is clopen and $U \not\in \left\{
		\varnothing, \R
	\right\}$.
	Then $V = \R \setminus U$ is also clopen.
	Let $x \in U$ and $y \in V$.
	Without loss of generality, assume $x < y$.
	Define
	\[
		A = \left\{
			a \in [x, \infty):
			[x,a] \subset U
		\right\}.
	\]
	Clearly $x \in A$.
	If $a \in A$, then $a < y$.
	Let $s = \sup A$ and suppose $s \in U$.
	Then as $U$ is open there is $\varepsilon > 0$ such that
	$(s - \varepsilon, s + \varepsilon) \subset U$.
	Now $[x,s] \subset U$ and so $[x, s + \sfrac12 \varepsilon] \subset U$.
	Hence $s$ is not a supremum; and so $s \not\in U$.
	So $s \in V$.
	As $V$ is also open, there is $\varepsilon > 0$ such that
	$(s - \varepsilon, s + \varepsilon) \subset V$.
	So $(s - \varepsilon, s + \varepsilon) \not\subset U$,
	but then $s$ is also not a supremum.
	Hence no such $U$ and $V$ exist
	and so $\R$ and $\varnothing$ are the only clopen sets in $\R$.
	Therefore, $\R$ is connected.
\end{solution}

\begin{problem}
	Show that $\Q$ is not connected.
\end{problem}

\begin{solution}
	Observe that
	\[
		\Q = (\Q \cap (-\infty, \sqrt 2)) \cup (\Q \cap (\sqrt 2, \infty)).
	\]
	Now, both $\Q \cap (-\infty, \sqrt 2)$ and $\Q \cap (\sqrt 2, \infty)$
	are clopen and so $\Q$ is not connected.
\end{solution}

\begin{problem}
	Show that the connected subsets of $\R$ are the intervals.
\end{problem}

\begin{solution}
	Suppose $A \subset \R$ is connected.
	Let $x,y \in A$ and $z \in \R \setminus A$ with 
	$x < z < y$.
	Then
	\[
		A = \left( 
			(-\infty, z) \cap A 
		\right) \cup \left( 
			(z,\infty) \cap A 
		\right)
	\]
	which is the union of 2 clopen sets. 
	Hence $[x,y] \subset A$.
\end{solution}

\begin{problem}
	Consider $X = \left\{
		0,1
	\right\}$ with the discrete topology.
	Show that $X$ is not connected.
\end{problem}

\begin{solution}
	Observe that $\left\{
		0
	\right\}$ is clopen.
\end{solution}

\begin{problem}
	Show that the Furstenberg topology on $\Z$ is not connected.
\end{problem}

\begin{solution}
	Observe that $
		\Z = (2\Z) \cup (1 + 2\Z)
	$.
\end{solution}

\begin{theorem}[]
	The continuous image of a connected space is connected.
\end{theorem}

\begin{proof}
	Let $h: X \to Y$ be continuous and $X$ be a connected space.
	Suppose $A \subset h(X)$ is clopen, and so
	$A$ and $A' = h(X) \setminus A$ are open in $h(X)$.
	Let $U, U' \subset Y$ be open in $Y$ such that $A = U \cap h(X)$
	and $A' = U' \cap h(X)$.
	As $A \cup A' = h(X)$ we have
	\[
		X = h^{-1}(A \cup A')
		= h^{-1}(A) \cup h^{-1}(A')
		= h^{-1}(U) \cup h^{-1}(U').
	\]
	As $X$ is connected either $X = h^{-1}(U)$ or $X = h^{-1}(U')$.
	Hence $X = h^{-1}(A)$ or $X = h^{-1}(A')$.
	Thus $A = h(X)$ or $A' = h(X)$.
	That is, $A = \varnothing$.
	Hence the only clopen sets are $h(X)$ and $\varnothing$.
\end{proof}

\begin{proposition}[]
	Let $X$ be a topological space.
	Then the following are equivalent.
	\begin{enumerate}
		\item $X$ is connected.
		\item $X$ cannot be written as the disjoint union of two non-empty open
			sets.
		\item There is no continuous surjective function from $X$ to
			a discrete space with more than one point.
	\end{enumerate}
\end{proposition}

\begin{solution}
	$(i) \iff (ii)$: if $X$ is connected and
	$X = A \cup ( X \setminus A)$ where $A$ and $X \setminus A$ are open, then
	$A$ is clopen.
	$(i) \implies (iii)$: let $Y$ be a discrete space with more than one point.
	Observe that $Y$ is not connected. 
	If $h: X \to Y$ is surjective and continuous then $h(X) = Y$.
	So $X$ is unconnected.
	$(iii) \implies (ii)$: suppose $X = U \cup V$, where $U$ and $V$ are open
	disjoint and non-empty.
	Then define $f: X \to \left\{
		0,1
	\right\}$
	as
	\[
		f(p) =
		\begin{cases}
			0 & x \in U, \\
			1 & x \in V.
		\end{cases}
	\]
	Oberse that $f$ is surjective and continuous (by contrapositive).
\end{solution}

\begin{examples}[]
	\begin{enumerate}
		\item Consider $\det: \operatorname{GL}_n(\R) \to \R \setminus \left\{
			0
		\right\}$.
		$\det$ is continuous, surjective, and $\R \setminus \left\{
			0
		\right\}$ is unconnected.
		Therefore $\operatorname{GL}_n(\R)$ is unconnected.
		
		\item $O(n)$ is not connected by similar reasoning to $(i)$.

		\item $(0,1)$ is connected as we have the function
			\[
				\tan^{-1} \circ \left( 
					x \mapsto \frac{x + \sfrac{\pi}2}{\pi}  
				\right)
			\]
			which is surjective and continuous 
			(and of course $\R$ is connected).

		\item Let $X = (0,1)$ and $Y = (0,1]$.
			Observe that 
			\[
				X \setminus \left\{
					p
				\right\} =
				(0,p) \cup (p,1)
			\]
			is disconnected but $Y \setminus \left\{
				1
			\right\} = (0,1)$ hence $(0,1)$ and $(0,1]$
			are not homeomorphic.
	\end{enumerate}
\end{examples}
