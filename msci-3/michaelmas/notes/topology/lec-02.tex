%! TEX root = master.tex

\lecture{2}{5/10}
\begin{definition}[Open and closed sets]
	Let $X$ be a metric space. 
	A subset $U \subset X$ is called \emph{open} if
	for every $x \in U$ 
	there is $\varepsilon > 0$ 
	such that $B(x;\varepsilon) \subset U$.
	A subset $A \subset X$ is called \emph{closed} if $X \setminus A$ is open.
\end{definition}

Points in a open set can be thought of having \emph{elbow room}.

\begin{definition}[Neighbourhood]
	Suppose $X$ is a metric space, $N \subset X$, and $x \in N$.
	We call $N$ a \emph{neighbourhood} of $x$ if there exists an open set
	$V \subset X$ with $x \in V \subset N$.
\end{definition}

\begin{proposition}[]
	Open balls are open.
\end{proposition}

\begin{proof}
	Let $x \in X$ and $r > 0$. Consider $U = B(x;r)$.
	The idea of this proof is for each point $y \in U$
	to construct another ball centered on the point
	with radius $r - d(x,y)$.
	Through manipu:wqa
	lation of the triangle equality, it can be shown
	that this ball lies within our other ball and hence $U$ is open.
\end{proof}

\begin{proposition}[]
	Closed balls are closed.
\end{proposition}

\begin{proof}
	Let $x \in X$ and $r > 0$. Consider $A = D(x;r)$.
	Now consider $U = X \setminus A$.
	The idea of this proof is to show that for every $y \in U$
	the ball centered at $y$ with radius $d(x,y) - r$ is contained
	within $U$ and so $U$ is open, and so $A$ is closed.
\end{proof}
