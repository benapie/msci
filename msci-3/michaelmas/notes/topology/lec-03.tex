%! TEX root = master.tex

\lecture{3}{6/10}

\begin{definition}[Sequences]
	Let $(X,d)$ be a metric space and $a \in X$.
	A \emph{sequence} in $X$ is a function $a: \N \to X$.
	We write $(a_n)_{n \in \N}$ where $a_n = a(n)$.
	We say that a sequence $(a_n)_{n \in \N}$ converges to $a$ if for every
	$\varepsilon > 0$ there is $n_0 \in \N$ such that $d(a_n, a) < \varepsilon$
	for all $n \geq n_0$.
	We write $\lim_{n \to \infty} \left( a_n \right) = a$ or $a_n \to a$.
\end{definition}

\begin{proposition}[]
	\label{prop:continuous-metric-space}
	Suppose that $X$ and $Y$ are metric spaces, $a \in X$, and $f: X \to Y$.
	Then the following statements are equivalent.
	\begin{enumerate}
		\item 
			For every $\varepsilon > 0$ there is $\delta > 0$ such that
			\[
				d(a,x) < \delta \implies d(f(a), f(x)) < \varepsilon.
			\]

		\item 
			For all sequences $(a_n)_{n \in \N}$ such that $a_n \to a$ we have 
			$f(a_n) \to f(a)$.

		\item
			For all open subsets $U \subset Y$ such that $f(a) \in U$, 
			$f^{-1}(U)$ is a neighbourhood of $a$.
	\end{enumerate}
\end{proposition}

\begin{proof}
	$i \implies ii$: assume $i$.
	Let $(a_n)_{n \in \N}$ such that $a_n \to a$.
	Then there is $\delta > 0$ such that 
	$d(a,a_n) < \delta \implies d(f(a),f(a_n)) < \varepsilon$.
	But $a_n \to a$, so there is $n_0 \in \N$ such that for all $n \geq n_0$ we
	have $d(a, a_n) < \delta$ and hence $d(f(a), f(a_n)) < \varepsilon$ for all
	$n \geq n_0$. 
	Therefore, $f(a_n) \to f(a)$.
	$ii \implies iii$: assume $ii$ and, for a contradiction, assume that $iii$
	does not hold.
	Then there exists $U \subset Y$ open with $f(a) \in U$ such that $f^{-1}(U)$
	is not a neighbourhood of $a$.
	So $B\left( a; \frac1n \right)$ is not contained within $f^{-1}(U)$ for all
	$n \in \N$.
	Now let $a_n \in B(a; \frac1n) \setminus f^{-1}(U) \neq \varnothing$.
	Then $d(a,a_n) < \frac1n$ and so $a_n \to a$.
	As $U$ is open there is $\varepsilon > 0$ such that 
	$B(f(a); \varepsilon) \subset U$.
	Now $f(a_n) \not\in U$ so $d(f(a), f(a_n)) \geq \varepsilon$.
	Therefore $f(a_n) \not\to f(a)$; a contradiction.
	$iii \implies i$: let $\varepsilon > 0$.
	Then $B(f(a); \varepsilon)$ is open.
	So $f^{-1}(B(f(a); \varepsilon)$ is a neighbourhood of $a$.
	So there is $\delta > 0$ such that
	$B(a; \delta) \subset f^{-1}\left( B(f(a); \varepsilon) \right)$.
	That is,
	$d(a,x) < \delta \implies d(f(a),f(x)) < \varepsilon$.
\end{proof}

\begin{definition}[Continuous function]
	Let $f$ be a function that satisfies any of the statements in Proposition
	\ref{prop:continuous-metric-space}.
	Then $f$ is \emph{continuous} at $a$.
	If $f$ is continuous for all points $a \in X$, we say that $f$ is 
	\emph{continuous}.
\end{definition}

\begin{proposition}[]
	Let $X$ and $Y$ be metric spaces and $f: X \to Y$.
	$f$ is continuous if and only if $f^{-1}(U)$ is open for all open 
	$U \subset Y$.
\end{proposition}

\begin{proof}
	Suppose $f$ is continuous and $U \subset Y$ is open.
	If $a \in f^{-1}(U)$, then $f(a) \in U$ and so $f^{-1}(U)$ is a neighbourhood
	of $a$.
	Hence there is $\varepsilon > 0$ such that 
	$B(a; \varepsilon) \subset f^{-1}(U)$.
	Hence $f^{-1}(U)$ is open.
	Now we suppose $f^{-1}(U)$ is open for all open $U \subset Y$.
	If $a \in f^{-1}(U)$ then there exists $\varepsilon > 0$ such that
	$B(a; \varepsilon) \subset f^{-1}(U)$. 
	Hence $f^{-1}(U)$ is a neighbourhood of $a$ and hence $f$ is continuous at
	$a$.
\end{proof}

