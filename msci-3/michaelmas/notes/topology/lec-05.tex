%! TEX root = master.tex

\section{Topological spaces}
\lecture{5}{18/10}

\begin{definition}[Topology]
	Let $X$ be a set.
	A \emph{topology} on $X$ is a subset $\tau \subset \mathcal P(X)$ which 
	satisfies the following:
	\begin{enumerate}
		\item $X, \varnothing \in \tau$;
		\item for $i \in I$, 
			$U_i \in \tau \implies \bigcup_{i \in I} U_i \in \tau$; and
		\item $U_1, U_2 \in \tau \implies U_1 \cap U_2 \in \tau$.
	\end{enumerate}
	The elements of $\tau$ are called the \emph{open subsets} of $X$.
	If $A \subset X$ and $X \setminus A \in \tau$, we say that $A$ is 
	\emph{closed}.
\end{definition}

\begin{examples}
	\hfill
	\begin{enumerate}
		\item 
		Let $(M,d)$ be a metric space.
		Then the collection of open sets as defined in the last section defines 
		a topology.
		This is confirmed by the previous lemma.

		\item
		$(X, \mathcal P)$ defines a topological space, namely the 
		\emph{discrete} topology on $X$.

		\item
		$(X, \left\{ \varnothing, X \right\})$ 
		also defines a (boring) topological
		space, namely the \emph{indiscrete} or \textbf{trivial} topology 
		on $X$.

		\item
		Let $X = \Z_{\geq 0}$ and
		\[
			\tau = \left\{ \varnothing \right\} \cup
			\left\{ 
				U \subset X: \left\lvert X \setminus U \right\rvert < \infty 
			\right\}.
		\]
		We can see that $\varnothing, X \in \tau$.
		Simiarly, if $U_1, U_2 \in \tau$, then clearly $U_1 \cap U_2 \in \tau$
		as 
		$X \setminus (U_1 \cap U_2) = (X \setminus U_1) \cup (X \setminus U_2)$.
		Now let $U_i \in \tau$ with $i \in I$, $I$ being some arbitrary 
		labelling set.
		If all $U_i = \varnothing$, then $U_{i \in I} U_i \in \tau$.
		Now assume that there is $j \in I$ such that $U_j \neq \varnothing$.
		Then
		\[
			X \setminus \bigcup_i U_i 
			= \bigcap_i (X \setminus U_i) 
			\subset X \setminus U_j
		\]
		and hence is finite.
		Therefore, $\tau$ is a topology on $X$.

		\item 
		Let $(X, \tau)$ be a topological space and $A \subset X$.
		Then 
		\[ 
			\tau_A = \left\{ A \cap U: U \in \tau \right\}
		\]
		is a topology  on $A$ called the \emph{induced} topology or 
		\emph{subspace} topology.
	\end{enumerate}
\end{examples}

\begin{problem}
	Show that the metrics $d_p$ and $d_\infty$ for $p \in [1,\infty)$ both 
	induce the same topology on $\R^n$.
\end{problem}

\begin{solution}
	Recall
	\begin{align*}
		d_p(x,y)
			&= \sum_{i=1}^n \left( 
				\left\lvert x_i - y_i \right\rvert^p 
			\right)^{\frac1p} \\
		d_\infty(x,y)
			&= \max\left\{ 
				\left\lvert x_i - y_i \right\rvert: 
				i \in \left\{ 1, \ldots, n \right\}
			\right\}.
	\end{align*}
	Clearly we have $d_\infty(x,y) < d_p(x,y)$.
	Simiarly, $d_p(x,y) < \sfrac1{n^{\frac1p}} d_\infty(x,y)$ and so if a 
	subset is open with $d_p(x,y)$, it is also open with $d_\infty(x,y)$ and
	vice versa.
	Therefore, both metrics (regardless of choice of $p$ induce the same
	topology. 
\end{solution}
