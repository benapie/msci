%! TEX root = master.tex

\lecture{4}{?}

\begin{example}[]
	Any continuous function $f: \R \to \R$ that we saw from Calculus I is also
	continuous in the sense that we have just described.
\end{example}

\begin{example}[]
	Define $I: C[0,1] \to C[0,1]$ by
	\[
		I(f)(x) \int_0^x f(t) \,dt.`
	\]
	$I$ is continuous with the $\sup$ metric we defined earlier.
\end{example}

\begin{example}[]
	Consider $X$ with the discrete metric.
	Consider $f: X \to \R$ where $\R$ has the standard Euclidean metric.
	Then $f$ is continuous as if $U \subset X$ with $p \in U$ then
	\[
		p \in \left\{ p \right\} 
		= B\left(p; \frac12\right) \subset U
	\]
	and hence $U$ is open. 
	In other words, all subsets of $X$ are open and so all preimages of any
	subset of $\R$ is also open.
	On the other hand, $\operatorname{id}: (\R, d_2) \to (\R, d)$ where $d_2$ is
	the Euclidean metric and $d$ is the discrete metric on $\R$ is not
	continuous. 
	Why? Well 
	$\left\{ 0 \right\} = \operatorname{id}^{-1}{\left\{ 0 \right\}}$ 
	is not open in 
	$(\R, d_2$ but $\left\{ 0 \right\}$ is open in $(\R, d)$.
\end{example}

\begin{lemma}[]
	Suppose that $X$ is a metric space.
	\begin{enumerate}
		\item $X$ and $\varnothing$ are both open and closed.
		\item An arbitrary union of open sets is open.
		\item A finite intersection of open sets is open.
		\item A finite union of closed sets is closed.
		\item An arbitrary intersection of closed is closed.
	\end{enumerate}
\end{lemma}

\begin{proof}
	$i$: let $p \in X$.
	Then $B(p;1) \subset X$.
	So $X$ is open.
	$\varnothing$ is vacuously open.
	It is immediately observed that $U$ and $\varnothing$ are closed.
	
	$ii$: suppose $U_i$ is open for all $i \in I$, $I$ being an arbitrary
	labelling set.
	If $p \in \bigcup_{i \in I} U_i$, we have $p \in U_{i_0}$ for some
	$i_0 \in I$.
	Then there is $\varepsilon > 0$ such that 
	$p \in B(p; \varepsilon) \subset U_{i_0} \subset \bigcup_{i \in I} U_i$.
	Hence $\bigcup_{i \in I} U_i$ is open.
	
	$iii$: suppose $U_i$ for $i = 1, \ldots, n$.
	Let $p \in \bigcap_{i=1}^n U_i$.
	Then $p \in U_i$ for all $i$.
	So there is $\varepsilon_i > 0$ such that 
	$p \in B(p; \varepsilon_i) \subset U_i$ for all $i$.
	We set 
	$\varepsilon = \min\left\{ \varepsilon_1, \ldots, \varepsilon_n \right\}$.
	Then $p \in B(p; \varepsilon) \subset \bigcap_{i=1}^n U_i$.
	Hence $\bigcap_{i=1}^n U_i$ is open.

	$iv$: let $A_1, \ldots, A_n$ be closed.
	So $X \setminus A_i$ is open for all $i \in \left\{ 1,\ldots,n \right\}$.
	Then $\bigcup_{i} A_i$ is closed if and only if $X \setminus \bigcup_i A_i$ 
	is open.
	Note that
	$X \setminus \bigcup_i A_i = \bigcap_i \left( X \setminus A_i \right)$
	and by $iii$ we are done.

	$v$: let $A_i \subset X$ be closed, $i \in I$.
	So $X \setminus A_i$ is open.
	$\bigcap_i A_i$ is closed if and only if $X \setminus \bigcap_i A_i$ is open.
	Note that
	$X \setminus \bigcap_i A_i = \bigcup_i \left( X \setminus A_i \right)$
	and so we are done by $ii$.
\end{proof}

\begin{example}[]
	Consider $\R$ with the Euclidean metric and consider the set
	\[
		\left\{ 0 \right\}
		= \bigcap_{n=1}^\infty \left( -\frac1n, \frac1n \right).
	\]
	Clearly $\left\{ 0 \right\}$ is \emph{not} open in the Euclidean metric,
	thus this illustrates that an arbitrary intersection of open sets do not have
	to be open.
\end{example}
