%! TEX root = master.tex
\lecture{12}{?}

\begin{example}[]
	Consider $\Z$ with the Furstenberg's topology.
	Let 
	\[
		A = 
		\left\{ 
			p \in \Z: \text{$p$ is prime}
		\right\}
		\subset \Z.
	\]
	What is $\overline A$?
	Recall that $\overline A$ is the collection of limit points of $A$
	and the points of $A$.
	So what are the limit points of $A$?
	Suppose $a \in \Z \setminus \left( A \cup \left\{ -1,0,1 \right\} \right)$.
	Let $U = \left\{ a + 10na: n \in \Z \right\}$
	which is an open set and observe that $a \in U$.
	Note that
	\[
		U =
		\left\{ a + 10na: n \in \Z \right\} = 
		\left\{ \ldots, -9a, a, 11a, 21a, \ldots \right\}.
	\]
	We see that there are no primes in $U$, so $U \cap A = \varnothing$.
	Therefore, there are no limit points of $A$ in
	$\Z \setminus \left( A \cup \left\{ -1,0,1 \right\} \right)$.
	Now we see that
	\[
		0 \in \left\{ \ldots, -20, -10, 0, 10, 20 \ldots \right\}
	\]
	which is clearly open and disjoint from $A$, so $0$ cannot be a limit point.
	We have left to investigate $\pm 1$.
	Suppose $U$ is open and $1 \in U$.
	Then there is $d \neq 0$ such that
	\[
		1 \in \left\{ 1 + dn: n \in Z \right\}.
	\]
	But, by Dirichlet's theorem, the above set must contain primes
	(and likewise for $ \left\{ -1 + dn: n \in \Z \right\} $).
	Therefore, $1$ and $-1$ are limit points of $A$.
	Thus
	$
		\overline A = A \cup \left\{ -1, 1 \right\}.
	$
\end{example}

Dirichlet's theorem states that for every $a, d \in \N$ such that
$\gcd(a,d) = 1$, there are infinitely many primes of the form $a + nd$
where $n \in \N$.
In other words, there are infinitely many primes that are congruent to $a$
modulo $d$.

\begin{lemma}[]
	Let $M$ be a metric space, $A \subset M$, and $x$ be a limit point of $A$.
	Then there is $x_n \in A$ such that $x_n \to x$.
	Furthermore, if $x \in M \setminus A$ and there is $x_n \in A$ such that
	$x_n \to x$ then $x$ is a limit point of $A$.
\end{lemma}

\begin{proof}
	We first prove the initial statement.
	Obserse that for each $n \in \N$, $B\left( x; \frac1n \right) \cap A$ 
	contains a point not equal to $x$, since $x$ is a limit point.
	We let $x_n \in B(x; \frac1n) \cap A$ such that $x_n \neq x$.
	We see that $d(x,x_n) < \frac1n$, and so $x_n \to x$.

	Now we prove the second statement.
	Suppose $x \in M \setminus A$ and $x_n \in A$ such that $x_n \to x$.
	Let $U$ be an open neighbourhood of $x$.
	Then there is $\varepsilon > 0$ such that $B(x; \varepsilon) \subset U$.
	Now, as $x_n \to x$ there is some $N \in \N$ such that 
	$d(x,x_N) < \varepsilon$.
	Hence
	\[
		x_N \in B(x; \varepsilon) \cap A \subset U \cap A
	\]
	and so $x$ is a limit point of $A$.
\end{proof}

\begin{definition}[Basis]
	Let $X$ be a topological space.
	We define a \emph{basis} $\mathcal B$ to be a collection of open subsets
	such that every open set is a union of elements in $\mathcal B$.
\end{definition}

\begin{examples}[]
	\begin{enumerate}
		\item Let $M$ be a metric space.
			Then $\mathcal B = \left\{ B(x;\varepsilon: x \in M, \varepsilon > 0 \right\}$ is a valid basis.

		\item Let $X = \R^n$ with the standard topology.
			Then
			\[
				\mathcal B = \left\{ B\left( q; \frac1m \right): q \in \Q^n, m
				\in \N \right\}
			\]
			is a \emph{countable} basis for the topology.
			We will prove this.
			Let $U \subset \R^n$ be open.
			Consider $p \in U$.
			Then there is $\varepsilon > 0$ such that $B(p; \varepsilon) \subset
			U$.
			Now choose $m \in \N$ such that $\frac1m < \frac{\varepsilon}{2}$
			and choose $q \in \Q^n \cap B\left( p; \frac{1}{m} \right)$.
			Then $p \in B\left( q; \frac1m \right)$.
			Also, $B\left( q; \frac1m \right) \subset B(p; \varepsilon)$ as
			$\frac 1m < \frac{\varepsilon}{2}$ and $d(p,q) < \frac1m < 
			\frac{\varepsilon}{2}$.
			So if $x \in B\left( q; \frac1m \right)$ then
			\begin{align*}
				d(x,p)
				&\leq d(x,q) + d(q,p) \\
				&< \frac{\varepsilon}{2} + \frac{\varepsilon}{2} = \varepsilon.
			\end{align*}
			If we find such a $q_p$ and $m_p$ for each $p \in U$, then
			\[
				U =
				\bigcup_{p \in U} \left\{ p \right\} \subset
				\bigcup_{p \in U} B\left( q_p; \frac{1}{m_p}  \right) \subset U.
			\]
	\end{enumerate}
\end{examples}

\begin{theorem}[]
	\label{the:top-spaces-equiv}
	Let $X$ and $Y$ be topological spaces and $f: X \to Y$.
	The following statements are equivalent.
	\begin{enumerate}
		\item $f$ is continuous.
		\item $\mathcal B$ is a basis for $Y$ $\implies$ 
			for all $B \in \mathcal B$, $f^{-1}(B)$ is open.

		\item For all $A \subset X$, $f(\overline A) \subset \overline{f(A)}$.
		\item For all $B \subset Y$, 
			$\overline{f^{-1}(B)} \subset f^{-1}(\overline B)$.
		\item $B \subset Y$ is closed $\implies$ $f^{-1}(B)$ is closed.
	\end{enumerate}
\end{theorem}

\begin{proof}
	$(i) \implies (ii)$ is clear since the elements of $\mathcal B$ are open.
	$(ii) \implies (iii)$: note that, if $A \subset X$ then $f(A) \subset \overline{f(A)}$.
	So we will be done if we can show that for all $x \in \overline A \setminus A$ (that is, the limit points of $A$ not in $A$) $f(x) \in \overline{f(A)}$.
	Suppose $x \in \overline A \setminus A$, observe that $x$ is a limit point.
	Now suppose $f(x) \not\in \overline{f(A)}$.
	Then $x \in Y \setminus \overline{f(A)}$, which we can see is open.
	Let $U \in \mathcal B$ such that $f(x) \in U \subset Y \setminus \overline{f(A)}$.
	So $x \in f^{-1}(U)$, which is open by $(ii)$. 
	But observe that
	\[
		f^{-1}(U) 
		\subset f^{-1}\left(Y \setminus \overline{f(A)}\right)
		= X \setminus f^{-1}\left(\overline{f(A)}\right) 
		\subset X \setminus f^{-1}\left( f(A) \right)
		= X \setminus A
	\]
	and so $f^{-1}(U) \cap A = \varnothing$, hence $x$ is not a limit point;
	a contradiction. 
	$(iii) \implies (iv)$: consider $\overline{f^{-1}(B)}$.
	By $(iii)$
	\[
		f\left( \overline{f^{-1}(B)} \right)
		\subset \overline{f\left(f^{-1}(B)\right)}
		= \overline B
	\]
	and so $\overline{f^{-1}(B)} \subset f^{-1}(\overline B)$ as required.
	$(iv) \implies (v)$: let $B \subset Y$ be closed.
	So $\overline B = B$.
	Then
	\[
		f^{-1}(B) 
		\subset \overline{f^{-1}(B)} 
		\subset f^{-1}(\overline B) 
		= f^{-1}(B).
	\]
	So $f^{-1}(B) = \overline{f^{-1}(B)}$, and so $f^{-1}(B)$ is closed.
	Finally, $(v) \implies (i)$ has already been shown by a previous lemma.
\end{proof}
