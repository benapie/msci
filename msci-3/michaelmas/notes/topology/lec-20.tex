%! TEX root = master.tex
\section{Quotient spaces}
\lecture{20}{27/11}

\begin{definition}[Equivalence relation]
	A binary relation ${\sim}$ on $X$ is said to be an
	\emph{equivalence relation} if for every $a,b,c \in X$
	it satisfies
	\begin{enumerate}
		\item (reflexivity) $a \sim a$;
		\item (symmetry) $a \sim b \iff b \sim a$; and
		\item (transitivity) 
			$(a \sim b \;\text{and}\; b \sim c) \implies a \sim c$.
	\end{enumerate}
\end{definition}

\begin{definition}[Quotient space]
	Let $X$ be a set and $\sim$ be an equivalence relation on$X$.
	We define $X/{\sim}$ to be the set of equivalence classes for $X$
	on $\sim$.
	For $x \in X$,
	\[
		[x] = \left\{
			y \in X: y \sim x
		\right\}, \qquad
		X/{\sim} = \left\{
			[x] \subset X: x \in X
		\right\}.
	\]
\end{definition}

Note that there is a surjective map $\pi: X \to X/{\sim}: x \mapsto [x]$
which takes elements of $X$ to their equivalence class.

\begin{definition}[]
	Let $X$ be a topological space and $\sim$ be an equivalence relation on
	$X$.
	Define $\pi(x) = [x]$.
	We give $X/{\sim}$ the topolopy: $U \subset X/{\sim}$ is open
	if and only if $\pi^{-1}(U)$ is open.
\end{definition}

\begin{proposition}[]
	The above defines a topology.
\end{proposition}

\begin{proof}
	Let $U$ and $V$ be open in $X/{\sim}$.
	Then $\pi^{-1}(U \cap V) = \pi^{-1}(Y) \cap \pi^{-1}(V)$ is open.
	Let
	$
		\left\{
			U_n
		\right\}_{n=1}^\infty
		= \bigcup_{n=1}^\infty \pi^{-1}(U_n)
	$
	is open.
	Finally, $\pi(\varnothing) = \varnothing$
	and $\pi(X) = X$.
\end{proof}

\begin{proposition}[]
	The topology on $X/{\sim}$ is the largest topology such that
	$\pi$ is continuous.
\end{proposition}

\begin{proof}
	$\pi$ is continuous if and only if $U \subset X/{\sim}$ is open 
	implies that $\pi^{-1}(U)$ is open.
\end{proof}

\begin{lemma}[]
	Let $X$ and $Y$ be topological spaces, $A \subset Y$, and
	$\iota: A \xhookrightarrow{} Y$.
	Then $f: X \to A$ is continuous if and only if
	$\iota \circ f: X \to Y$ is continuous.
\end{lemma}

\begin{proof}
	$f$ is continuous $\iff$
	for each $U \subset A$ open, $f^{-1}(U)$ open $\iff$
	for every $V \subset Y$ open, $f^{-1}(V \cap A)$ open $\iff$
	for every $V \subset Y$ open, $f^{-1}(\iota^{-1}(V))$ open $\iff$
	$\iota \circ f$ is continuous.
\end{proof}

\begin{proposition}[]
	Let $X$ and $Y$ be topological spaces, $\sim$ be an equivalence relation
	on $X$, and $\pi(x) = [x]$.
	Then $f: X/{\sim} \to Y$ is continuous if and only if
	$f \circ \pi: X \to Y$ is continuous.
\end{proposition}

\begin{proof}
	$(\implies)$: clear.
	$(\impliedby)$:
	suppose $f \circ \pi$ is continunous.
	Let $U \subset Y$ be open.
	Then
	$
	(f \circ \pi)^{-1}(U) = \pi^{-1}(f^{-1}(U))$ is open.
	So $f^{-1}(U)$ is open in $X/{\sim}$.
\end{proof}

\begin{example}[]
	Consider $\R$ with the equivalence relation $x \sim y \iff x - y \in \Z$.
	We will show that $R/{\sim}$ is homeomorphic to $S^1$.
	We define $\exp: \R \to S^1$, $\exp(t) = e^{2\pi it}$.
	Suppose $s \in \R$ such that $t \sim s$.
	Then 
	\[
		\exp(t) 
		= e^{2\pi i t} 
		= e^{2\pi it} e^{2\pi i(s - t}
		= e^{2\pi is}
		= \exp(s).
	\]
	Hence we can define a map $\overline{\exp}: \R/{\sim} \to S^1$
	where $\exp = \overline\exp \circ \pi$,
	with $\pi(x) = [x]$.
	\begin{center}
		\begin{tikzcd}
			\mathbb R \arrow[d, "\pi"'] \arrow[r, "\exp"]   & S^1 \\
			\mathbb R/{\sim} \arrow[ru, "\overline{\exp}"'] &    
		\end{tikzcd}
	\end{center}
	Since $\exp$ and $\pi$ are continuous,
	so is $\overline\exp$.
	Further note that $\overline\exp$ is bijective, continuous, and
	$S^1$ is Hausdorff.
	Hence $\R/{\sim}$ is compact since
	$\R/{\sim} = \pi([-10,10])$.
	Hence $\overline\exp$ is a homeomoprhism.
\end{example}

\begin{example}[]
	Observe $S^{n-1} \subset D^n \subset \R^n$.
	We claim that $D^n/S^{n-1}$ is homeomorphic to $S^n$,
	which we will now prove.
	Take coordinates $(\varphi, t)$ in $D^n$ where $t \in [0,1]$,
	$\varphi S^{n-1}$, $t\varphi \in D^n$.
	Define
	$
		f: D^n \to S^n \subset \R^{n+1}
	$
	by
	\[
		f(t\varphi) = \left( 
			\cos(t\pi), \varphi \sin(t\pi) 
		\right) \in \R^{n+1}.
	\]
	$f$ is continuous, and observe that
	\[
		f(S^{n-1}) = \left\{
			(-1, \bm 0)
		\right\} \subset \R^{n+1}
	\]
	where $\bm 0 \in \R^n$.
	Then there is $\overline f: D^n/S^{n-1} \to S^n$, from the following
	diagram.
	\begin{center}
		\begin{tikzcd}
			D^n \arrow[r, "f"] \arrow[d] \arrow[d, "\pi"'] & S^n \\
			D^n/S^{n-1} \arrow[ru, "f"']                   &    
		\end{tikzcd}
	\end{center}
	Observe that $\overline f$ is a bijection, $\overline f$ is continuous,
	$S^n$ is Hausdorff, and $D^n/S^{n+1} = \pi(D^n)$ is compact.
	Hence $\overline f$ is a homeomorphism.
\end{example}

