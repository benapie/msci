%! TEX root = master.tex
\section{Topological groups}
\lecture{21}{6/12}

\begin{definition}[]
	A \emph{topological group} $G$ is a Hausdorff space which is also a 
	group, such that multiplication $\cdot: G \times G \to G$ and
	inversion $i: G \to G$ are continuous.
\end{definition}

\begin{examples}
	\begin{enumerate}
		\item $\R^n$ with addition is a topological group.
			$\R^n \times \R^n \xrightarrow{\cdot} \R$,
			$(\bm x, \bm y) \mapsto \bm x + \bm y$ 
			and
			$\R^n \xrightarrow{i} \R^n$, $\bm x \mapsto -\bm x$
			are both continuous.

		\item
			$\C \setminus \left\{0\right\}$ is a topological group
			under multiplication.
			Both $\cdot: (w,z) \mapsto wz$ and
			$i: z \mapsto z^{-1}$ are both continuous.
			\[
				S^1 =
				\left\{
					z \in \C: \left\lvert z \right\rvert = 1
				\right\}
				\subset \C \setminus \left\{0\right\}
			\]
			is a subgroup.

		\item $\operatorname{GL}_n(\R)$ is a topological group under
			matrix multiplication.
			Both
			$\cdot: (A,B) \mapsto AB$ and $i: A \mapsto A^{-1}$
			are continuous and we have the subgroups
			$\operatorname{O}(n)$ and
			$\operatorname{SL}(n)$.

		\item Any subgroup of a topological group is also a topological
			group.
			Let $H \subset G$ be a subgroup of a topological group $G$ witb
			operations $\cdot$ and $i$.
			Then
			$H \times H \xhookrightarrow{} G \times G \xrightarrow{\cdot} G$
			and
			$H \xhookrightarrow{} G \xrightarrow{i} G$
			are both continuous.

		\item Quaternions:
			$\mathbb H \cong \R^4 = \langle 1,i,j,k \rangle$,
			where $a + bi + cj + dk \in \mathbb H$, $a,b,c,d \in \R$,
			$ij = k$, $jk = i$, $ki = j$, and $i^2 = j^2 = k^2 = -1$.
			$\mathbb H \setminus \left\{0\right\}$ is a non-abelian
			multiplicative group.
			$S^3 \subset \mathbb H \setminus \left\{0\right\}$
			is a subgroup and hence a topological groups.
			So $S_1$ and $S_3$ are topological groups.

		\item If $G$ is a group, giving $G$ the discrete topology 
			makes it a topological group.
	\end{enumerate}
\end{examples}

Let $G$ be a topological group.
For $x \in G$, define $L_x: G \to G$ by $L_x(g) = xg$, called the
\emph{left translation by $x$}.
This is continuous and has an inverse,
namely $L_{x^{-1}}$.
In particular, $L_x$ is a homeomorphism.
Similarly, \emph{right translation} is a homeomorphism $R_x$.

\begin{proposition}[]
	Let $G$ be a topological group and $K$ the connected component of $G$ that
	contains the identity element.
	Then $K$ is a closed normal subgroup of $G$.
\end{proposition}

\begin{proof}
	$K$ is closed as it is a connected component.
	$L_x$ is a homeomorphism, hence it permutates the connected components
	of $G$.
	Suppose $x \in K$, then 
	$L_{x^{-1}}(K) \ni x^{-1} x = 1$. Hence $L_{x^{-1}}(K) = K$.
	Hence $K = L_{x^{-1}}(K) \ni x^{-1} \cdot 1 = x^{-1}$.
	Suppose $x, y \in K$.
	Then $x = L_x(1) \in L_x(K)$ so $L_x(K) = K$.
	Hence
	$xy = L_x(y) \in L_x(K) = K$.
	Finally,
	note that if $g \in G$, then $g \in L_g(K) = gK$, $g \in R_g(K) = Kg$.
	Hence $gK$ and $Kg$ are the same connected component; hence, $gK = Kg$.
\end{proof}

