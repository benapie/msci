%! TEX root = master.tex
\section{Compactness}
\lecture{18}{19/11}

\begin{definition}[Compact]
	Let $X$ be a topological space.
	A \emph{cover} of $X$ is a collection 
	$
		\left\{
			U_i
		\right\}_{i \in I}
	$
	of subsets of $X$ such that 
	$\bigcup_{i \in I} U_i = X$.
	The cover is called an \emph{open cover} if every $U_i$ is open.
	A \emph{subcover} of
	$
		\left\{
			U_i
		\right\}_{i \in I}
	$
	is a cover
	$
		\left\{
			U_j
		\right\}_{j \in J}
	$
	where $J \subset I$.
	$X$ is said to be \emph{compact} if and only if every open cover of $X$
	admits a finite subcover.
\end{definition}

\begin{examples}
	\begin{enumerate}
		\item If $X$ is a finite set, it is compact.
		\item We claim $\R$ is not compact.
			Consider the open cover
			$\left\{
				U_i
			\right\}$
			where
			$U_i = (i-1,i+1)$.
			Clearly this covers $\R$, but there is no finite
			subcover that can cover $\R$.
		\item Let $X$ be an infinite set with the topology
			\[
				\tau = \left\{
					U \subset X: 
					\text{$X \setminus U$ is finite}
				\right\} \cup \left\{
					0
				\right\}.
			\]
			We claim that $X$ is compact.
			Suppose
			$
				\left\{
					U_i
				\right\}_{i \in I}
			$ 
			is an open cover of $X$.
			Without loss of generality, we assume that each $U_i$ is
			non-zero.
			Now pick $j \in I$.
			Then
			$
				X \setminus U_j = \left\{
					p_1, \ldots, p_n
				\right\}
			$.
			For each
			$
				k \in \left\{
					1, \ldots, n
				\right\}
			$,
			there is 
			$i_k \in I$
			such that
			$p_k \in U_{i_k}$.
			Then
			\[
				X 
				= U \cup \left\{
					p_1, \ldots, p_n
				\right\}
				= U \cup U_{i_1} \cup U_{i_2} \cup \ldots \cup U_{i_n}.
			\]
		\item We claim the closed interval $[0,1]$ is compact.
			Let
			$
				\left\{
					U_i
				\right\}_{i \in I}
			$
			be an open cover of $[0,1]$.
			Without loss of generality, we assume that each $U_i$
			is an open interval.
			Let $A$ be the collection of points $a \in [0,1]$
			such that there is a finite subcover of $[0,a]$
			from the open cover above.
			Clearly $0 \in A$, so $A$ is non-empty.
			Thus we let $s = \sup A$.
			Suppose $s < 1$, thus
			$s \in U_j \supset (s - \varepsilon, s + \varepsilon)$ 
			for some $j \in J$ and $\varepsilon > 0$.
			As $U_j$ is an open interval, 
			$
				\left[
					0, s - \frac{\varepsilon}{2}
				\right]
			$
			can be covered by a finite number of $U_i$'s.
			Thus, with $U_j$, also covers
			$
				\left[
					0, s + \frac{\varepsilon}{2} 
				\right]
			$.
			Thus $s$ must not be the supremum. 
			Therefore, $\sup A = 1$.
	\end{enumerate}
\end{examples}

\begin{proposition}[]
	The continuous image of a compact set is compact.
\end{proposition}

\begin{proof}
	Let $X$ be a compact topological space, $Y$ be a topological space,
	and $f: X \to Y$ be continuous.
	Let
	$
		\left\{
			V_i
		\right\}_{i \in I}
	$
	be an open cover of $Y$. So
	\[
		f(X) \subset \bigcup_{i \in I} V_i.
	\]
	Then
	\[
		X
		= f^{-1}(f(X))
		\subset f^{-1} \left( 
			\bigcup_{i \in I} V_i 
		\right)
		= \bigcup_{i \in I} f^{-1}(V_i)
		\subset X.
	\]
	As $X$ is compact, there is 
	$i_1, \ldots, i_n \in I$
	such that
	\[
		X = \bigcup_{k=1}^n f^{-1}\left( 
			V_{i_k} 
		\right).
	\]
	So
	\begin{align*}
		f(X)
		&= f\left( 
			\bigcup_{k=1}^n f^{-1}(V_{i_k})
		\right) \\
		&= \bigcup_{k=1}^n f(f^{-1}(V_{i_k})) \\
		&\subset \bigcup_{k=1}^n V_{i_k}
	\end{align*}
	and so $f(X)$ is compact.
\end{proof}

\begin{proposition}[]
	A closed subset of a compact space is compact.
\end{proposition}

\begin{proof}
	Let $X$ be compact and $A \subset X$ be closed.
	Let 
	$
		\left\{
			U_i
		\right\}_{i \in I}
	$
	be an open cover of $A$.
	Then
	\[
		X = \left( 
			\bigcup_{i \in I} U_i 
		\right) \cup \left( 
			X \setminus A 
		\right)
	\]
	gives an open cover of $X$.
	As $X$ is compact, there is 
	$i_1, \ldots, i_n \in I$
	such that
	\[
		X = \bigcup_{k=1}^n U_{i_k} \cup (X \setminus A)
	\]
	is a finite open cover of $X$.
	So $A \subset \bigcup_{k=1}^n U_{i_k}$,
	thus $A$ is compact.
\end{proof}

\begin{theorem}[]
	In Hausdorff spaces, compact subsets are closed.
\end{theorem}

\begin{proof}
	Let $X$ be a Hausdorff space, and $A \subset X$ be compact.
	Let $p \in X \setminus A$ and $q \in A$.
	Then there is open sets $U_q$ and $V_q$ such that
	$p \in U_q$, $q \in V_q$, and $U_q \cap V_q = \varnothing$.
	Then
	$
		A \subset \bigcup_{q \in A} V_q.
	$
	As $A$ is compact,
	there is $q_1, \ldots, q_n \in A$ such that
	$
		A \subset \bigcup_{k=1}^n V_{q_k}
	$.
	Note that 
	$\bigcap_{k=1}^n U_{q_k} \ni p$ 
	is an open set and
	\[
		\left(
			\bigcap_{k=1}^n U_{q_k}
		\right) \cap A
		\subset \left( 
			\bigcap_{k=1}^n U_{q_k} 
		\right) \cap \bigcup_{k=1}^n \left( 
			V_{q_k} 
		\right)
		= \varnothing.
	\]
	So $\bigcup_{k=1}^n U_{q_k} \subset X \setminus A$;
	hence, $X \setminus A$ is open and so $A$ is closed.
\end{proof}

\begin{corollary}
	A bijective continuous function from a compact space to a
	Hausdorff space is a homeomorphism.
\end{corollary}

\begin{proof}
	Let $X$ be compact, $Y$ be Hausdorff, and $f: X \to Y$ be a bijective
	continuous function.
	It is enough to show that if $A \subset X$ is closed, then
	$
		f(A) = \left( 
			f^{-1} 
		\right)^{-1}(A)
	$
	is closed.
	$A$ is compact, so $f(A)$ is compact, so $f(A)$ is closed.
\end{proof}

\begin{example}[]
	Consider the Furstenbergy topology on $\Z$.
	We claim that $\Z$ is \emph{not} compact.
	Let
	$
		U(a, b) = \left\{
			a + bn: n \in \N
		\right\}
	$
	and consider the open cover
	\[
		\left\{
			U(p,1): \text{$p$ prime}
		\right\} \cup \left\{
			U(5,2)
		\right\}
	\]
	of $\Z$.
	If there is a finite subcovering of this cover,
	then there is finitely many primes;
	a contradiction.
\end{example}

