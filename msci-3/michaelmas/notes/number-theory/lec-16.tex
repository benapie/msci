%! TEX root = master.tex
\lecture{16}{28/11}

\begin{theorem}[]
	Any proper non-zero ideal in $R$ has a unique factorisation into prime
	ideals (up to reordering).
\end{theorem}

\begin{proof}
	We know that this factorisation exists,
	we have left to prove that it is unique.
	Let
	\[
		I 
		= \mathfrak p_1 \ldots \mathfrak p_n
		= \mathfrak q_1 \ldots \mathfrak q_m.
	\]
	We need to show that $n = m$ and $\mathfrak p_i = \mathfrak q_i$
	(up to reorderings).
	As $\mathfrak p_n$ is prime,
	it must divide one of $\mathfrak q_i$ (by the previous Lemma).
	Without loss of generality, we will assume that 
	$\mathfrak p_n \mid \mathfrak q_m$.
	But $\mathfrak q_m$ is prime, so it is irreducible and thus
	$\mathfrak p_n = \mathfrak q_m$.
	Both are invertible, so we have
	\[
		\mathfrak p_1 \ldots \mathfrak p_{n-1}
		= \mathfrak q_1 \ldots \mathfrak q_{m-1}.
	\]
	We repeat this process, assuming without loss of generality that
	$m \geq n$, to get
	\[
		R = \mathfrak q_1 \ldots \mathfrak q_{m-n}. \qedhere
	\]
\end{proof}

Rings in which every non-zero proper ideal unqiuely factorises into a product
of prime ideals (up to reordering) are known as \emph{Dedekind domains}.

\begin{example}[]
	Let $R = \Z[\sqrt{-6}]$.
	Observe
	\[
		(1 + 3\sqrt{-6})(1 - 3\sqrt{-6}) = 5 \cdot 11
	\]
	as elements of $R$.
	Now both of these are indeed factorisations in $R$ to irreducibles.
	So $R$ is \emph{not} a unique factorisation domain.
	In terms of ideals, this gives
	\[
		(1 + 3\sqrt{-6})_R (1 - 3\sqrt{-6})_R
		= (5)_R (11)_R. \tag{$\star$}
	\]
	Now, we define
	\[
		\mathfrak p_5 = (5, 1 + 3\sqrt{-6}), \qquad
		\overline{\mathfrak p_5} = (5, 1 - 3\sqrt{-6})
	\]
	and similarly,
	\[
		\mathfrak p_{11} = (11, 1 + 3\sqrt{-6}), \qquad
		\overline{\mathfrak p_{11}} = (11, 1 - 3\sqrt{-6}).
	\]
	Through expansion, we see that
	$\mathfrak p_5 \overline{\mathfrak p_5} = (5)_R$, and
	$\mathfrak p_{11} \overline{\mathfrak p_{11}} = (11)_R$.
	Furthermore,
	$\mathfrak p_5 \mathfrak p_{11} = (1 + 3\sqrt{-6})_R$ and
	$\overline{\mathfrak p_5} \overline{\mathfrak p_{11}}
	= (1 + 3\sqrt{-6})_R$.
	We can show that 
	$R/\mathfrak p_5 \cong R/\overline{\mathfrak p_5} \cong \mathbb F_5$
	and similarly 
	$R/\mathfrak p_{11} \cong R/\overline{\mathfrak p_{11}} \cong 
	\mathbb F_{11}$.
	So $\mathfrak p_5$ and $\mathfrak p_{11}$ are maximal and thus prime.
	Thus $(\star)$ becomes
	\[
		(\mathfrak p_5 \mathfrak p_{11})
		(\overline{\mathfrak p_5 \mathfrak p_{11}})
		= (55)_R
		= (\mathfrak p_5 \overline{\mathfrak p_5})
			(\mathfrak p_{11} \overline{\mathfrak p_{11}})
	\]
\end{example}

We can even show that the notions of principal ideal domains and
unqiue factorisation domains coincide in $R$.

\begin{lemma}[]
	Let $I$ be a non-zero ideal in $R$.
	Then there is an ideal $J$ such that $IJ$ is principal.
\end{lemma}

\begin{proof}
	Let $\alpha \in I$ be a non-zero element.
	Then $(\alpha)_R \subset I$.
	$I$ is invertible, so $I \mid (\alpha)_R$.
	Consider the factorisations
	\begin{align*}
		I &= \mathfrak p^{e_1}_1 \ldots \mathfrak p_n^{e_n} \\
		(\alpha)_R &= \mathfrak p_1^{r_1} \ldots \mathfrak p_n^{r_n}.
	\end{align*}
	$\mathfrak p_i$ are distinct prime ideals, $e_i, r_i \in \N$, and
	$0 \leq e_i \leq r_i$.
	Let $J = \mathfrak p_1^{r_1 - e_1} \ldots \mathfrak p_n^{r_n - e_n}$.
\end{proof}

\begin{theorem}[]
	If $R$ is a unique factorisation domain, then $R$ is a principal ideal 
	domain.
\end{theorem}

\begin{proof}
	It is enough to show that every prime ideal is principal.
	Let $\mathfrak p$ be a prime ideal.
	Then there is an ideal $\mathfrak q$ such that
	\[
		\mathfrak{pq} = (a)_R
	\]
	for some $a \in \mathfrak p$.
	Since we are in a unique factorisation domain, we can decompse
	$a = p_1 \ldots p_m$ into irreducible (and thus prime) elements.
	The principal ideals $(p_i)_R$ are prime since $p_i$ are prime elements.
	Then we have \[
		\mathfrak{pq} = (p_1)_R \ldots (p_m)_R.
	\]
	Since we are in a Dedekind domain, this decomposition is unique.
	As $\mathfrak p$ is unique, $\mathfrak p = p_i$ for somek $i$.
	Hence $\mathfrak p$ is principal.
\end{proof}

\section{The case of quadratic fields}

We let $K = \Q(\sqrt d)$ where $d$ is a square free integer
and $R = \mathcal O_K = \Z + \omega\Z$ where $\omega = \sqrt d$
if $d \equiv 2,3 \pmod 4$ and $\omega = \frac{1+\sqrt d}2$ if
$d \equiv 1 \pmod 4$.

In any quadratic field, the map $x + y\sqrt d \mapsto x - y\sqrt d$
is a field automoprhism.
In particular,
$\overline{\alpha\beta} = \overline\alpha \overline\beta$
and
$\overline{\alpha + \beta} = \overline\alpha + \overline\beta$.
It is easy to see that $N_{K/\Q}(\alpha) = \alpha \overline\alpha$
and $\operatorname{Tr}_{K/\Q}(\alpha) = \alpha + \overline{\alpha}$.
We also note
\[
	\Q = \left\{
		\alpha \in K: \overline{\alpha} = \alpha
	\right\}.
\]
Furthermore, if $\alpha \in R$, then $ \overline{\alpha} \in R$.
Indeed, if $\alpha \in R$, then
\[
	\alpha^2 + a_1 a\alpha + a_0 = 0
\]
for some $a_0,a_1 \in \Z$.
But since $\overline{a_1} = a_1$
and $ \overline{a_0} = a_0$, we must have
\[
	\overline{\alpha}^2 + a_1 \overline{\alpha} + a_0 = 0.
\]
The following are easy to check:
\begin{enumerate}
	\item for every ideal $I \subset R$, 
		$
			\overline I = \left\{
				\overline\alpha: \alpha \in I
			\right\} \subset R
		$ is also an ideal; and
	\item if $I = (a_1, \ldots, a_n)$, then
		$
			\overline I = (\overline{a_1}, \ldots, \overline{a_n})
		$.
\end{enumerate}
