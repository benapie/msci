%! TEX root = master.tex
\lecture{10}{12/11}

\begin{theorem}[]
	For a cyclotomic field $K$ we have
	$[K:\Q] = \phi(n)$ and $\mathcal O_K = \Z[\zeta]$.
\end{theorem}

Here $\phi$ denotes Euler's totient function.
It denotes the order of the group $\left( 
	\Z/n\Z 
\right)^\times$ and is determined by the following rules:
\begin{enumerate}
	\item $\phi(nm) = \phi(n) \phi(m)$;
	\item $\phi(p^r) = (p-1)p^{r-1}$; and
	\item $\phi(1) = 1$.
\end{enumerate}

\section{Euclidean domains, principal ideal domains, and unique factorisation
domains}

\begin{definition}[Integral domain]
	An \emph{integral domain} is a commutative ring $R$ such that for every
	$a, b \in R$,
	\[
		ab = 0 \implies a = 0 \;\text{or}\; b = 0.
	\]
\end{definition}

Observe that for any number field $K$,
$\mathcal O_K \subset K$ and is an integral domain.

\begin{definition}[Euclidean function]
	Let $R$ be an integral domain.
	A \emph{Euclidean function} (or \emph{norm}) for $R$ is a function
	$\varphi: R \setminus \left\{
		0
	\right\} \to \N \cup \left\{
		0
	\right\}$ such that
	\begin{enumerate}
		\item for every $a, b \in R \setminus \left\{
			0
		\right\}$ such that $b \mid a \implies \varphi(b) \leq \varphi(a)$; and
		
		\item for every $a \in R$ and $b \in R \setminus \left\{
			0
		\right\}$ there is $q, r \in R$ such that
		$a = bq + r$ with $r = 0$ or $\varphi(r) < \varphi(b)$.
	\end{enumerate}
\end{definition}

\begin{definition}[Euclidean domain]
	An integral domain for which a Euclidean function exists is called a
	\emph{Euclidean domain} (ED).
\end{definition}

\begin{examples}
	\begin{enumerate}
		\item $\Z$ is a Euclidean domain.
			Indeed, the map defined by $a \mapsto \left\lvert a \right\rvert$
			is a Euclidean functioun.

		\item $\Q[x]$ is also a Euclidean domain.
			Indeed, the map $f(x) \mapsto \deg{f(x)}$.
	\end{enumerate}
\end{examples}

\begin{lemma}[]
	Consider the Gaussian integers, $K = \Q(i)$.
	The ring of Gaussian integers $\mathcal O_K = \Z[i]$ is a
	Euclidean domain with Euclidean function
	\[
		\varphi(x) = \operatorname{N}_{K/\Q}(x).
	\]
\end{lemma}

\begin{proof}
	For a quadratic field $K = \Q(\sqrt d)$,
	we have seen that $\operatorname{N}_{K/\Q}(a + b\sqrt d) = a^2 - db^2$
	for $a,b \in \Q$.
	Here we take $d = -1$.
	We have to check that $\varphi$ is actually a Euclidean function.
	Let $r,s \in \Z[i] \setminus \left\{
		0
	\right\}$.
	Then as $r \mid s$, $s = ra$ for some $a \in \Z[i]$.
	Then
	\begin{align*}
		N(s)
		&= N(ra) \\
		&= N(r) N(a) \\
		&\geq N(r).
	\end{align*}
	So $(i)$ holds.
	Now for $(ii)$: consider $x, y \in \Z[i]$ with $y \neq 0$.
	Then $\frac xy = a' + b'i$, $a', b' \in \Q$.
	Let $a, b \in \Q$ such that
	\[
		\left\lvert a - a' \right\rvert \leq \frac12, \qquad
		\left\lvert b - b' \right\rvert < \frac12.
	\]
	Set $q = a + bi$.
	Then
	\[
		x = qy + ((a' - a) + (b' - b)i)y = x - qy.
	\]
	Furthermore
	\begin{align*}
		\operatorname{N}_{K/\Q}(((a'-a) + (b'-b)i)y)
		&= \operatorname{N}_{K/\Q}((a' - a) + (b' - b)i)
			\operatorname{N}_{K/\Q}(y) \\
		&= ((a'-a)^2 + (b'-b)^2)
			\operatorname{N}_{K/\Q}(y) \\
		&\leq \frac12 \operatorname{N}_{K/\Q}(y) \\
		&< \operatorname{N}_{K/\Q}(y)
	\end{align*}
	as $y \neq 0$ if and only if $N_{K/\Q}(y) \neq 0$.
\end{proof}
