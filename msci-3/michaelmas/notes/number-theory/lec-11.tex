%! TEX root = master.tex
\lecture{11}{13/11}

\begin{definition}[Principal ideal domain]
	An integral domain $R$ is called a \emph{principal ideal domain}
	if every ideal of $R$ is principal.
\end{definition}

Recall that an ideal is \emph{principal} if it generated by a single element.

\begin{proposition}[]
	Every Euclidean domain is a principal ideal domain.
\end{proposition}

\begin{proof}
	Assume $R$ is a Euclidean domain with Euclidean function $\varphi$.
	Let $I$ be an ideal of $R$.
	We can assume $I$ is proper ($I \neq R$) and non-zero
	(as the zero ideal is generated by $0$ and $R = (1)$).
	Let $0 \neq x \in I$ such that for every $y \in I$
	$\varphi(x) \leq \varphi(y)$.
	This is possible since $\varphi$ maps to $\N \setminus \left\{
		0
	\right\}$.
	We claim that $x$ generates $I$.
	Indeed, let $y \in I$.
	We claim that $x \mid y$.
	If we assume not, the we can write $y = qx + r$ for some $q \in R$,
	$r \in R \setminus \left\{
		0
	\right\}$ with $\varphi(r) < \varphi(x)$.
	But $r = y - qx \in I$;
	this contradicts the minimality of $x$.
	Hence $x \mid y$. That is, $y \in (x)_R$.
	As this holds for all $y \in I$, we have $I = (x)_R$.
\end{proof}

From this, we see that $\Z$ and $\Z[i]$ are principal ideal domains.
Observe that not all principal ideal domains are not Euclidean domains.
For example, $\Z\left[\frac{1 + \sqrt{-19}}{2}\right]$ is a
principal ideal domain but not a Euclidean domain.

Recall from Algebra II that for an integral domain $R$, an element
$x \in R \setminus R^\times$ is called a \emph{prime element} if for every
$a, b \in R$ we have
\[
	x \mid ab \implies (x \mid a \quad\text{or}\quad x \mid b).
\]
We also know that if $x$ is prime then it is also irreducible.
Indeed, assume that $x = ab$ is prime then $x \mid ab$.  
Without loss of generality assume $x \mid a$.  
Hence there is $r \in R$ such that $a = xr$.
But then $x = xrb$ or $x(1 - rb) = 0$.
Since $x$ is non-zero, we have that $rb = 1$ and so $b \in R^\times$;
a contradiction.

\begin{proposition}[]
	Let $R$ be a principal ideal domain.
	Then an element $x \in R$ is a prime element if and only if $x$ is
	irreducible.
\end{proposition}

\begin{proof}
	$(\implies)$: proof above.
	$(\impliedby)$: assume $x$ is irreducible and $x \mid ab$.
	So there is $r \in R$ such that $ab = xr$.
	We want to show that $x \mid a$ or $x \mid b$.
	Consider the ideal $I = (x,a)_R$.
	As $R$ is a principal ideal domain there is $c \in R$
	such that $I = (c)_R$.
	So $x = cd_1$ and $a = c d_2$ for some $d_1, d_2 \in R$.
	As $x$ is irreducible either $c \in R^\times$ or $d_1 \in R^\times$.
	If $d_1 \in R^\times$, then $a = x d_1^{-1} d_2$ and so $x \mid a$.
	If $c \in R^\times$ then $I = R$, so $1 \in I$.
	Hence there exists $\lambda, \mu \in R$ such that
	$1 = \lambda x + \mu a$.
	That is, $b = \lambda bx + \mu ab = (\lambda b + r \mu)x$,
	so $x \mid b$.
\end{proof}

\begin{definition}[Unique factorisation domain]
	An integral domain $R$ is called a \emph{unique factorisation domain}
	if for every $0 \neq x \in R \setminus R^\times$ we can write $x$ as a
	product of irreducible elements, and this is unique up to reordering and
	multiplication by units.
\end{definition}

\begin{proposition}[]
	Let $R$ be a unique factorisation domain.
	Then an element $x \in R$ is prime if and only if $x$ is irreducible.
\end{proposition}

\begin{proof}
	$(\implies)$: follows a similar reasoning to the proof for the last
	proposition.
	$(\impliedby)$: assume $x$ is irreducible and $x \mid ab$.
	Then there is $c \in R$ such that $ab = xc$.
	As we are in a unique factorisatinop domain we may decompose $a,b,c$
	as follows:
	\[
		a = p_1 \ldots p_m, \qquad
		b = q_1 \ldots a_n, \qquad
		c = r_1 \ldots r_l.
	\]
	Then
	\[
		p_1 \ldots p_m q \ldots q_n = x r_1 \ldots r_l.
	\]
	As any decomposition is unique, $x = u_i p_i$ or $x = u_j q_j$
	for some units $u_i, u_j$.
	That is, $x \mid a$ or $x \mid b$.
\end{proof}

Interstingly, every pricipal ideal domain is also a unique factorisation domain.
In fact, both concepts coincide when we are considering $\mathcal O_K$ for some
number field $K$.

\begin{theorem}[]
	Let $K$ be a number field.
	Then for every 
	$0 \neq \alpha \in \mathcal O_K \setminus \mathcal O_K^\times$
	can be written as a product of irreducible elements.
\end{theorem}

\begin{proof}
	In this proof we let $\operatorname{N} = \operatorname{N}_{K/\Q}$.
	If $x \in \mathcal O_K$, then $N(x) \in \Z$.
	Moreover, $x \in \mathcal O_K^\times$ if and only if $N(x) = \pm 1$.
	In particular, if $x = ab$ with $a, b \not\in \mathcal O_K^\times$ then
	\[
		\left\lvert N(x) \right\rvert
		> \left\lvert N(a) \right\rvert.
	\]
	Let $0 \neq \alpha \in \mathcal O_K \setminus \mathcal O_K^\times$.
	If $\alpha$ is irreducible then we are done.
	Otherwise, let $\alpha = a_1 b_1$ for some 
	$a_1, b_1 \not\in \mathcal O_K^\times$.
	Then $a_1 \mid \alpha$ and
	\[
		\left\lvert N(a_1) \right\rvert
		< \left\lvert N(\alpha) \right\rvert.
	\]
	Now we consider $a_1$ and repeat the same logic.
	We keep repeating to obtain $a_1, \ldots, a_n$ such that
	\[
		a_n \mid a_{n-1} \mid \ldots \mid a_1 \mid \alpha
	\]
	with $N(a_n) < N(a_{n-1}) < \ldots < N(a_1) < N(\alpha)$.
	As $N(\alpha) \in \Z$, $a_n$ is irreducible.
	Set $p_1 = a_n$.
	Since $a)n \mid \alpha$, then $\alpha = p_1 c_1$ for some 
	$c_1 \in \mathcal O_K$.
	We repeat the above fo4 $c_1$ instead of $a_1$ to obtain
	$c_1 = p_2 d_2$ for some irreducible $p_2$.
	So $\alpha = p_1 p_2 c_2$.
	We repeat this to obtain $\alpha = p_1 \ldots p_m$ where
	$p_i$ is irreducible.
	Th is will stop as $\left\lvert N(\alpha) \right\rvert \in \N$
	and $\left\lvert N(p_i) \right\rvert > 1$.
\end{proof}
