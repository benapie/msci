%! TEX root = master.tex
\lecture{12}{13/11}

The above theorem establishes that for e very number field $K$, $\mathcal O_K$
will always have irreducible elements.
There are rings that do not contain any irreducible elements, for example 
$\overline \Z$.
If $0 \neq \alpha \in \overline\Z \setminus \overline \Z^\times$ then
\[
	\alpha^n + a_{n-1} \alpha^{n-1} + \ldots + a_1 \alpha + a_0 = 0
\]
for $a_i \in \Z$.
Observe that $\beta = \sqrt{\alpha} \in \overline \Z$ since
\[
	\beta^{2n} + a_{n-1} \beta^{2(n-1)} + \ldots + a_1 \beta^2 + a_0 = 0.
\]
As $\alpha = \beta^2$ and since $\alpha \not\in \overline\Z^\times$ we also
have $\beta \not\in \overline\Z^\times$.
Hence $\alpha$ is not irreducible.

\begin{theorem}[]
	Let $K$ be a number field.
	Then the following are equivalent.
	\begin{enumerate}
		\item $\mathcal O_K$ is a unique factorisation domain.
		\item For every $x \in \mathcal O_K$, $x$ is irreducible if and only if
			$x$ is prime.
	\end{enumerate}
\end{theorem}

\begin{proof}
	$(\implies)$: we have shown.
	$(\impliedby)$: assume for every $x \in \mathcal O_K$ we have that
	$x$ is irreducible if and only if $x$ is prime.
	Let $0 \neq \alpha \in \mathcal O_K \setminus \mathcal O_K^\times$.
	Assuem $\alpha = p_1 \ldots p_n$ for $p_i$ irreducible.
	Further assume $\alpha = q_1 \ldots q_m$ for $q_i$ irreducible.
	Without loss of generality, we assume $n \leq m$ and $p_n \mid q_m$
	(by rearranging).
	Thus $q_m = u_n q_n$, $u_m \in \mathcal O_K^\times$ since $q_m$ is
	irreducible.
	So
	\[
		p_1 \ldots p_n = u_m q_1 \ldots q_{m-1} p_n.
	\]
	We repeat this to obtain
	\[
		1 = u_m \ldots u_{m-n} p_1 \ldots p_{m-n}
	\]
	but then $n = m$. 
	Hence decomposition to irreducibles is unique;
	hence $K$ is a unique factorisation domain.
\end{proof}

\begin{corollary}
	Let $K$ be a number field.
	If $\mathcal O_K$ is a principal ideal domain it is also a
	unique factorisation domain.
\end{corollary}

So $\Z[i]$ is a Euclidean domain, so it is principal ideal domain, and so 
it is a unique factorisation domain.
Note that not all $\mathcal O_K$ are unique factorisation domains.
For example, $\mathcal O_{\Q(\sqrt{-5})} = \Z[\sqrt{-5}]$.
Take $21 \in \Z[\sqrt{-5}]$.
We may write
\[
	3 \cdot 7 = 21 = (1 + 2\sqrt{-5})(1 - 2\sqrt{-5}).
\]
Now we claim that $3$, $7$, $1 + 2\sqrt{-5}$, and $1 - 2\sqrt{-5}$ are
irreducible.
Recall that $
	\operatorname{N}_{K/\Q}(a + b\sqrt{-5}) 
	= \operatorname{N}(a + b\sqrt{-5})
	= a^2 + 5b^2
$.
So $N(3) = 9 = N(x) N(y)$ then $N(x) = N(y) = 3$.
But $N(x) = a^2 + 5b^2$ for $a, b \in \Z$, so $3$ is irreducible.
A similar line of reasoning can be used to show that the other 3 elements
are also irreducible.
We also must check that these elements do not differ be units.
That is,
\[
	\frac{1 \pm \sqrt{-5}}{3}, \frac{1 \pm \sqrt{-5}}{7} 
	\not\in \mathcal O_K^\times.
\]
But neither one are even in $\mathcal O_K$, so we are done.
Hence $\mathcal O_K$ is not a unique factorisation domain.
