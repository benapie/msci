%! TEX root = master.tex
\lecture{15}{24/11}

\begin{proof}
	Let $\mathfrak p$ be a non-zero prime ideal
	and $\alpha \in \mathfrak p$ be a non-zero element.
	By the previous Lemma, $(\alpha)_R$ contains a product
	of one or more non-zero prime ideals.
	That is, there exists prime ideals $\mathfrak p_1, \ldots, \mathfrak p_t$
	such that
	\[
		(\alpha)_R \supset \mathfrak p_1 \ldots \mathfrak p_t. \tag{$\star$}
	\]
	We choose these prime ideals such that $t$ is as small as possible.
	Now, as $(\star)$, we must have that $\mathfrak p \supset \mathfrak p_i$
	for some
	$
		i \in \left\{
			1, \ldots, t
		\right\}
	$.
	Without loss of generality, we assume that 
	$\mathfrak p \supset \mathfrak p_1$.
	As both $\mathfrak p$ and $\mathfrak p_1$ are prime,
	they are maximal and thus $\mathfrak p = \mathfrak p_1$.
	That is, there is
	$\beta \in \mathfrak p_2 \ldots \mathfrak p_t$
	such that 
	$\beta \not\in (\alpha)_R$.
	In particular, $\beta \neq 0$.
	Then $\sfrac{\beta}{\alpha} \not\in R$,
	\[
		(\sfrac{\beta}{\alpha}) \mathfrak p
		= \alpha^{-1} \mathfrak p (\beta)_R
		\subset \alpha^{-1} \mathfrak p_1 (\mathfrak p_2 \ldots \mathfrak p_t)
		\subset \alpha^{-1}(\alpha)_R = R.
	\]
	We set $\gamma = \sfrac{\beta}{\alpha} \in K \setminus R$
	and $\mathfrak p' = (\gamma)_R + R$.
	This is a fractional ideal and we claim that it is the inverse 
	of $\mathfrak p$.
	Indeed
	\[
		\mathfrak p \mathfrak p' = (\gamma)_R \mathfrak p + \mathfrak p
	\]
	and hence $\mathfrak p \subset \mathfrak p \mathfrak p' \subset R$.
	But $\mathfrak p$ is prime and thus it is maximal.
	Thus we must have either $\mathfrak p = \mathfrak p \mathfrak p'$
	or $R = \mathfrak p \mathfrak p'$.
	But if $\mathfrak p = \mathfrak p \mathfrak p'$
	then $(\gamma)_R \mathfrak p + \mathfrak p = \mathfrak p$.
	Then $(\gamma)_R \mathfrak p \subset \mathfrak p$
	and thus $\gamma \mathfrak p \subset \mathfrak p$.
	But $\mathfrak p$ is known to be a finitely generated abelian group
	and so $\gamma \in R$.
	But we constructed $\gamma \in K \setminus R$; a contradiction.
	Thus we must have $\mathfrak p \mathfrak p' = R$.
\end{proof}	

The proof was non-constructive, but does tell us the form of the inverse
of a prime ideal, that is, it is of the form $(\gamma)_R + R$ for some
$\gamma \in K \setminus R$.

\begin{example}[]
	Let $R = \Z[\sqrt{-6}]$ and $K = \Q(\sqrt{-6})$.
	The ideal 
	$
		\mathfrak p = (5, 1 + 3\sqrt{-5})
	$
	is a maximal ideal in $R$ since the map
	$\phi: R \to \mathbb F_5$
	given by $\phi(a + b\sqrt{-6}) = a - 2b \pmod 5$
	is a surjective ring homomorphism with kernel equal to
	$\mathfrak p$. 
	Thus $\mathfrak p$ is also prime.
	We now claim that
	\[
		\mathfrak p^{-1}
		= \left( 
			\frac{1 - 3\sqrt{-6}}{5} 
		\right)_R + R.
	\]
	Note that
	\begin{align*}
		\mathfrak p \mathfrak p^{-1}
		&= (5, 1 + 3\sqrt{-6})_R \left( 
			\left( 
				\frac{1 - 3\sqrt{-6}}{5} 
			\right) + R
		\right)_R \\
		&= \frac{1-3\sqrt{-6}}{5} (5,1+3\sqrt{-6})_R
			+ (5, 1 + 3\sqrt{-6})_R
	\end{align*}
	and observe
	\begin{align*}
		\frac{1 - 3\sqrt{-6}}{5} \cdot 5 
			= 1 - 3\sqrt{-6} &\in \mathfrak p \mathfrak p^{-1} \\
		\left( 
			\frac{1 - 3\sqrt{-6}}{5}
		\right)
		\left( 
			1 + 3 \sqrt{-6} 
		\right)
			= 11 &\in \mathfrak p \mathfrak p^{-1} \\
		(1-3\sqrt{-6})(1 + 3\sqrt{-6}) 
			= 5 &\in \mathfrak p \mathfrak p^{-1} \\
		11 - 2 \cdot 5 = 1 &\in \mathfrak p \mathfrak p^{-1}.
	\end{align*}
	and so $\mathfrak p \mathfrak p^{-1} = R$, 
	since if $1$ is in any ideal we get to every element of the ring.
\end{example}

So we have that every non-zero prime ideal is invertible.
We want to extend this to all ideals; more specifically,
show that for every non-zero ideal $I \subset R$, we can write
$I$ as a product of prime ideals.

\begin{lemma}[]
	Let $I \subset R$ be an invertible ideal
	and $J \subset R$ be an ideal such that $I \supset J$.
	Then
	\begin{enumerate}
		\item $I^{-1} J \subset R$ and $I \mid J$; and
		\item $I^{-1} J \supset J$ and $I^{-1} J = J \iff I = R$.
	\end{enumerate}
\end{lemma}

\begin{proof}
	$(i)$: as $J \subset I$, then $I^{-1} J \subset I^{-1} I = R$.
	Thus $I^{-1}J$ is an integral ideal and so 
	$J = I I^{-1} J = I(I^{-1}J)$.
	So $I \mid J$.
	$(ii)$: as $R \supset I$, we times both sides by $I$ to get 
	$I^{-1} J \supset I^{-1} J I = J$.
	Now, for the $\iff$, clearly the $(\impliedby)$ comes out clearly.
	$(\implies)$: assume $I^{-1} J = J$.
	Thus for each $\alpha \in I^{-1}$, $\alpha J \subset J$.
	But $J$ is a finitely generated abelian group.
	Thus $\alpha$ algebraic integer (by a previous theorem) and
	so $\alpha \in R$.
	Hence $I^{-1} \subset R$.
	Thus 
	\[
		R = II^{-1} \subset IR = I \subset R. \qedhere
	\]
\end{proof}

\begin{theorem}[]
	Let $J$ be a non-zero ideal of $R$.
	Then
	\[
		J = \mathfrak p_1 \ldots \mathfrak p_n
	\]
	where each $\mathfrak p_i$ is prime and $n \in \N$.
\end{theorem}

\begin{proof}
	Suppose otherwise.
	That is, there is a non-zero ideal $J \subset R$ which is
	\emph{not} a product of prime ideals.
	Select the $J$ such that $N(J)$ is minimal
	(which we can do since $N$ maps to $\Z_{\geq 0}$).
	$J$ is clearly not a prime ideal itself.
	Hence $J$ is not maximal.
	Hence we can find an ideal $I \subset R$ inbetween $J$ and $R$.
	That is, $J \subset I \subset I$.
	Now, as $J \neq I$ we have $N(I) < N(J)$.
	By the minimality of the norm of $J$, $I$ must be expressible as a product
	of prime ideals.
	So $I$ must be invertible.
	This allows us to apply the Lemma we have just proved.
	That is, $I^{-1} J \supset J$
	(with $I^{-1}J \neq J$, since $I \neq R$).
	In particular, $N(I^{-1}J) < N(J)$.
	By the minimality of $N(J)$, $I^{-1}J$ must be a product of prime ideals.
	But then $J = II^{-1} J$.
	As we have shown that $I$ and $I^{-1}J$ are expressible as a product of
	prime ideals, $J$ must be too; a contradiction.
\end{proof}

\begin{corollary}[]
	All non-zero ideals in $R$ are invertible.
	Furthermore, $\mathcal J(R)$ is an abelian group with respect to 
	multiplication with neutral element $R$.
\end{corollary}

\begin{proof}
	The first point is clear as we have shown that every prime ideal
	is invertible and every ideal can be expressed as a product of prime ideals.
	For the second point, let $\mathfrak a \in \mathcal J(R)$.
	So $\mathfrak a = \lambda I$ for some $\lambda \in K^\times$ and
	non-zero ideal $I \subset R$.
	Then
	\[
		\mathfrak a^{-1} 
		= \lambda^{-1}I^{-1}
		= (\lambda^{-1})_R \cdot I^{-1}
		\in \mathcal J(R).
	\]
	Then
	\begin{align*}
		\mathfrak a \mathfrak a^{-1}
		&= (\lambda I) (\lambda^{-1} I^{-1}) \\
		&= \lambda \lambda^{-1} I I^{-1} \\
		&= IR \\
		&= R;
	\end{align*}
	thus, we have an inverse.
	We have already shown that for each $\mathfrak a, \mathfrak b \in
	\mathcal J(R)$,
	$\mathfrak a \mathfrak b = \mathfrak b \mathfrak a$.
	The closure, associativity, and identity properties are clear.
\end{proof}

\begin{definition}[Irreducible ideal]
	Let $\mathfrak q \subset R$ be an ideal.
	We say that $\mathfrak q$ is \emph{irreducible}
	if $\mathfrak q$ cannot be writen as $IJ$ where $I, J \subset R$
	are proper ideals.
\end{definition}

\begin{lemma}[]
	Let $\mathfrak p$ be a non-zero ideal in $R$.
	Then the following are equivalent.
	\begin{enumerate}
		\item $\mathfrak p$ is prime.
		\item $\mathfrak p$ is irreducible.
		\item $\mathfrak p \mid IJ \implies \mathfrak p \mid I
			\;\text{or}\; \mathfrak p \mid J$.
	\end{enumerate}
\end{lemma}

\begin{proof}
	%exercise
\end{proof}
