%! TEX root = master.tex

\lecture{3}{13/10}
\begin{definition}[Minimal polynomial]
	Let $L$ be a field extension of some field $F$
	and $\alpha \in L$ be algebraic over $L$.
	We define the
	\emph{minimal polynomial of $\alpha$ over $F$}
	as the monic polynomial $p(x) \in F[x]$ 
	of smallest degree
	such that $p(\alpha) = 0$.
	The degree of $p$ is called the \emph{degree} of $\alpha$ over $F$.
\end{definition}

Recall that a polynomial is \emph{monic}
if the leading term is equal to $1$.
Moreoever, a polynomial $f(x) \in F[x]$
is irreducible if $f$ is not constant and cannot
be written as the product of two non-constant polynomials.

\begin{proposition}[]
	Let $L$ be a field extension of some field $F$
	and let $p(x)$ be the minimal polynomial of some
	algebraic element $\alpha \in L$ over $F$.
	Then $p(x)$ is unique and irreducible.
	Moreoever, if $f(x) \in F[x]$ is monic and irreducible with 
	$f(\alpha) = 0$, then $f$ is the minimal polynomial of $\alpha$.
\end{proposition}

\begin{proof}
	First we will prove that $p(x)$ is unique.
	Assume that $p(x)$ and $q(x)$ 
	are both distinct minimal polynomials of $\alpha$ over $F$.
	Let $g(x) = p(x) - q(x)$.
	Then $\deg g < d$, since $p$ and $q$ are monic.
	Let $a$ be the leading term of $g$,
	then $a^{-1}g(x)$ is clearly a minimal polynomial since
	$g(\alpha) = 0$; a contradiction.
	Therefore, the minimal polynomial is unique.
	Now we prove that $p(x)$ is irreducible.
	Assume $p(x) = q(x) \cdot r(x)$ for some non-constant
	$q(x), r(x) \in F[x]$ with $\deg q, \deg r < \deg p$.
	Now
	\[
		0 = p(\alpha) = q(\alpha) \cdot r(\alpha) 
		\implies \left( q(\alpha) = 0 \;\text{or}\; r(\alpha) = 0 \right).
	\]
	In both scenarios, we can normalise $q$ or $r$ to get another candidate for
	our minimal polynomial that has a degree smaller than $p$; a contradiction.
	Finally, we prove our last statement.
	Let $f(x) \in F[x]$ monic and irreducible with $f(\alpha) = 0$.
	We are aiming to show that $f(x) = p(x)$, and it is enough to show
	that $\deg f = \deg p$ (from our first statement).
	Assume not, that is $\deg f > \deg p$.
	Recall, by the division, we can express $f$ as
	\[
		f(x) = d(x) \cdot p(x) + r(x)
	\]
	for $r(x), d(x) \in F[x]$ and $\deg r < \deg p$.
	Since $f(x)$ is irreducible, $r(x) \neq 0$.
	Then
	\[
		r(\alpha) = f(\alpha) - d(\alpha) \cdot p(\alpha) = 0
	\]
	with $\deg r < \deg p$; a contradiction.
	Hence $\deg f = \deg p$ and hence our result.
\end{proof}

\begin{example}[]
	Consider $i \in \C$.
	The polynomial $p(x) = x^2 + 1$ is the minimal polynomial of $i$
	over $\R$ and $\Q$.
	Clearly, $p(x) \in \R[x], \Q[x]$ and $p(i) = 0$
	and $p$ is of smallest degree as if $\deg p = 1$, then $p \in \C[x]$.
	However, $p$ is not irreducible over $\C$ as
	\[
		p(x) = (x+i)(x-i).
	\]
	The minimal polynomial for $i$ over $\C$ is simply $q(x) = x - i$.
\end{example}

\begin{example}[]
	Consider $\alpha = \sqrt[7]{5}$.
	We claim that the minimal polynomial of $\alpha$ over $\Q$ is
	$p(x) = x^7 - 5$.
	We will now prove this.
	$p$ is clearly monic and $p(\alpha) = 0$.
	Now we have left to show that $p$ is irreducible.
	By \emph{Eisenstein's criterion} with $p = 5$, we see that
	$p$ is irreducible.
	Therefore, $p$ is the minimal polynomial of $\alpha$ over $\Q$.
\end{example}

\begin{example}[]
	Let $\alpha = e^{\frac{2\pi i}{p}} \in \C$ where $p$ is a prime number.
	Clearly $\alpha$ is algebraic over $\Q$ as we can take $f(x) = x^p - 1$.
	\emph{However}, $f(x)$ is not the minimal polynomial of $\alpha$ over $\Q$
	as it is not irreducible:
	\[
		f(x) = x^p - 1 = (x - 1)(x^{p-1} + x^{p-2} + \ldots + 1).
	\]
	We now let $\phi(x) = x^{p-1} + x^{p-2} + \ldots + 1$.
	We claim that $\phi$ \emph{is} the minimal polynomial of $\alpha$ over $\Q$.
	It is clear to see that we must only show that it is irreducible,
	since $\phi(\alpha) = 0$ and it is monic.
	We cannot use \emph{Eisenstein's criterion} for $\phi(x)$...
	but we can use it for $\phi(x + 1)$.
	\begin{align*}
		\phi(x+1)
		&= (x+1)^{p-1} + (x+1)^{p-2} + \ldots + 1 \\
		&= \frac{(x+1)^p - 1}{(x + 1) - 1} \\
		&= x^{p-1} + \binom p1 x^{p-2} + \ldots + \binom p1.
	\end{align*}
	Now recall
	\[
		\binom pj = \frac{p!}{j!(p-j)!} , \qquad \binom p1 = p.
	\]
	We see that $p \mid \binom pj$, and $\phi(x+1)$ is irreducible
	and hence $\phi(x)$ is irreducible (easily shown via contradiction).
\end{example}
