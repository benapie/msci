%! TEX root = master.tex
\lecture{17}{2/12}

\begin{proposition}[]
	Let $M \subset R$ be an (additive) subgroup.
	Then $M$ is of the form
	\[
		M = n\Z + (a + m\omega)\Z \tag{$\star$}
	\]
	for $n, m \in \Z_{\geq 0}$.
\end{proposition}

\begin{proof}
	If $M = R$, then $R = \Z + w\Z$.
	Now consider any subgroup $M$ of $R$.
	Consider the group
	\[
		H = \left\{
			s \in \Z: r + s\omega \in M, r \in \Z
		\right\}.
	\]
	All subgroups of $\Z$ are of the form $m\Z$,
	hence there is $a \in \Z$ such that $a + m\omega \in M$.
	Consider $M \cap \Z \subset \Z$.
	It again is a subgroup of $\Z$.
	Now we prove $(\star)$.
	$\subset$ is clear, so we assume $r + s\omega \in M$.
	Since $s \in H$,
	we have
	\[
		s - ua = r + s\omega - (a + m\omega)
		\in M \cap \Z
	\]
	since both summands are in $M$ and $s - ua$ is in $\Z$.
	Hence $r - ua = vn$ for some $v \in \Z$.
	Hence we obtain
	\[
		r + sw
		= r - ua + u(a + n\omega)
		= vn + u(a + n\omega)
		\in n \Z + (a + m\omega)\Z.\qedhere
	\]
\end{proof}

All ideals in $R$ are also an additive subgroup.
In particular, every ideal $I$ in $R$ is a finitely generated
abelian group and so $I$ is finitely generated.
Further, each $I = (\alpha, \beta)_R$ for some
$\alpha, \beta \in R$.
We have actually shown that for each
\[
	I = n\Z + (a + m\omega)\Z
\]
we must have that $n,m \neq 0$.
Indeed, if $m = 0$ then $\omega I \not\in I$
and if $n = 0$, $I \cap \Z = 0$,
which we know is non-zero if $I$ is non-zero.
We see that $I$, as an abelian group, is also \emph{free}.
Since if
\[
	a_1 n + a_2 (a + m \omega) = b_1 n + b_2(a + m\omega)
\]
then $a_1 = b_1$ and $a_2 = b_2$.
Indeed, if we write 
\[
	(a_1 - b_1)n + (b_1 - b_2) (a + m \omega) = 0
\]
then
\[
	\left( (a_1 - a_2)n + (b_1 - b_2) \right)a
	+ (b_1 - b_2)(a _ m\omega) = 0.
\]
Since $\left\{
	1, \omega
\right\}$
is a basis of $K$ over $\Q$, $(b_1 - b_2) m \omega = 0$
and as $m \neq 0$, $b_1 = b_2$.
Moreover, $ \left( (a_1 - a_2)n + (b_1 - b_2)\right)a = 0$
and since $n \neq 0$, $a_1 = a_2$.
But we can go even further...

\begin{proposition}[]
	If an ableina group of the form
	\[
		M = n\Z + (a + m\omega)\Z \subset R
	\]
	is an ideal in $R$, them $m, n \neq 0$ and
	\[
		m \mid n, \qquad
		m \mid a \quad (a = mb), \qquad
		n \mid m\operatorname{N}(b + \omega).
	\]
\end{proposition}

\begin{proof}
	Since $M$ is an ideal, $c \in M \cap I$.
	Tuhs $c \omega \in M$ and so $c \in H$.
	This gives $n \Z \subset m\Z$ and so $m \mid n$.
	Furthermore, $\omega^2 = x + y\omega$ for some $x, y \in \Z$.
	Since $M$ is an ideal, $a + m\omega \in M$
	gives us
	\[
		(a + m\omega)\omega = mx + (a + my) \omega \in M,
	\]
	and so $a + my \in H$ and so $m \mid a$.
	We write $a = mb$ for some $b \in \Z$.
	Then
	\[
		a + m\omega = m(b + \omega) \in M.
	\]
	Since $(b + \overline\omega) \in R$, we have
	\[
		m(b + \omega)(b + \overline\omega) = M \cap \Z.
	\]
	That is, $n \mid m\operatorname{N}(b+\omega)$.  
\end{proof}
