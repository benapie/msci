%! TEX root = master.tex

\subsection{$p$-cyclotomic fields}
\lecture{9}{5/11}

\begin{definition}[Cyclotomic field]
	Let $p$ be prime and $\zeta = e^{\frac{2\pi i}p} \in \C$.
	The field $K = \Q(\zeta)$ is called the
	\emph{$p$th cyclotomic field}.
\end{definition}

Let $K = \Q(\zeta)$. 
We have already seen that
\[
	[K: \Q] = p - 1.
\]
Our aim now is to find $\mathcal O_K$.
We have also seen that the minimum polynomial of $\zeta$ is
\[
	\Phi(x) = x^{p-1} + x^{p-2} + \ldots + x + 1 \in \Z[x].
\]
Clearly, $\zeta \in \mathcal O_K$.
Infact, as $\mathcal O_K$ is a ring, $\zeta^i \in \mathcal O_K$ for
$i \in \left\{
	1, \ldots, p-1
\right\}$.
As we saw before:
\[
	\Phi(x) = \frac{x^p - 1}{x - 1},
\]
so we have $\Phi(\zeta^i) = 0$.
Therefore
\[
	\Phi(x) = \prod_{i=1}^{p-1} (x - \zeta)^i.
\]
Clearly $\Phi(1) = 1^{p-1} + \ldots + 1 = p$, so
\[
	p = \prod_{i=1}^{p-1} \left( 
		1 - \zeta^i
	\right)
\]

\begin{lemma}[]
	Let $p$ be prime, $\zeta = e^{\frac{2\pi i}{p}} \in \C$, and
	$K = \Q(\zeta)$.
	Then for every $i \in \left\{
		1, \ldots, p-1
	\right\}$ we have
	\[
		\frac{1 - \zeta^i}{1 - \zeta} \in \mathcal O_K^\times.
	\]
\end{lemma}

\begin{proof}
	Observe that
	\[
		\frac{1 - \zeta^i}{1 - \zeta}
		= 1 + \zeta + \ldots + \zeta^{i-1}
		\in \Z[\zeta].
	\]
	Now let $\varphi = \zeta^i$.
	Then there is $j \in \left\{
		1, \ldots, p-1
	\right\}$ such that $\varphi^j = \zeta$.
	Therefore
	\[
		\frac{1 - \zeta}{1 - \zeta^i} 
		= \frac{1 - \varphi^j}{1 - \varphi}
		= 1 + \varphi + \ldots + \varphi^j
		= 1 + \zeta^i + \ldots + \zeta^{ij}
		\in \Z[\zeta]
	\]
	and thus $\frac{1 - \zeta^i}{1-\zeta}$ is a unit.
\end{proof}

By the Lemma above, we may define $
	u_i = \frac{1 - \zeta^i}{1 - \zeta} \in \mathcal O^\star_K
$ and so we can write \[
	p = u_1 u_2 \cdots u_{p-1} \left( 
		1 - \zeta 
	\right)^{p-1}.
\]

\begin{lemma}[]
	\hfill
	\vspace{-\baselineskip}
	\[
		(1-\zeta)_{\mathcal O_K} \cap \Z = \left( 
			p 
		\right)_\Z.
	\]
\end{lemma}

Recall that $(1 - \zeta)_{\mathcal O_K}$ denotes the principal ideal in 
$\mathcal O_K$ generated by $1 - \zeta$.

\begin{proof}
	By a problem sheet, we know that $I \cap \Z$ is a non-trivial ideal
	of $\Z$ (since $I$ is non-trivial).
	We claim that $I$ is proper;
	that is, $1 - \zeta \not\in \mathcal O_K^\times$.
	If this is not the case then,
	as 
	\[
		p = u_1 u_2 \cdots u_{p-1} \left( 
			1 - \zeta 
		\right)^{p-1},
	\]
	we would have $p \in \mathcal O_K^times$.
	But $p \in \Z$ and thus $p \in \Z^\times$; a contradiction.
	So $I$ is non-trivial proper.
	In particular, $I \cap \Z$ is proper
	since otherwise $1 \in I \cap \Z$.
	Since $u(1 - \zeta)^{p-1} \in I$,
	$p \in I \cap \Z$.
	So the ideal must be generated by $p$ since it is the only ideal
	that contains $p$.
\end{proof}

\begin{theorem}[]
	Let $p$ be prime, $\zeta = e^{\frac{2\pi i}{p}}$,
	$K = \Q(\zeta)$, and $y \in I$.
	Then
	\[
		\operatorname{Tr}_{K/\Q}(y) 
		\in (1-\zeta)_{\mathcal O_K} \cap \Z = (p)_\Z.
	\]
\end{theorem}

\begin{theorem}[]
	Let $K = \Q(\zeta)$ be a cyclotomic field. Then
	\[
		\mathcal O_K = \Z[\zeta].
	\]
\end{theorem}

\begin{proof}
	Clearly $\Z[\zeta] \subset \mathcal O_K$.
	Now let $x \in \mathcal O_K$.
	We may write
	\[
		x = \sum_{i=0}^{p-2} a_i \zeta^i, \qquad a_i \in \Q.
	\]
	Our aim is to show that $a_i \in \Z$.
	We have 
	\[
		x \zeta = \sum_{i=0}^{p-2} a_i \zeta^{i+1}.
	\]
	By subtracting from our original expression, we get
	\[
		x(1 - \zeta) 
		= a_0 (1 - \zeta) + a_1 (\zeta - \zeta^2)
		+ \ldots
		+ a_{p-2}(\zeta^{p-2} - \zeta^{p-1}).
	\]
	But observe
	\[
		\operatorname{Tr}_{K/\Q}(x(1-\zeta))
		= a_0 \operatorname{Tr}_{K/\Q}(1 - \zeta)
		+ \ldots + a_{p-2} \operatorname{Tr}_{K/\Q}
			(\zeta^{p-2} - \zeta^{p-1})
	\]
	and since we can write $\operatorname{Tr}_{K/\Q}(\zeta^i) = -1$
	for all $i \in \left\{
		0, \ldots, p-1
	\right\}$ since they are the roots of the same equation.
	Thus
	\[
		\operatorname{Tr}_{K/\Q}(x(1-\zeta))
		= a_0 \operatorname{Tr}_{K/\Q}(1 - \zeta)
		= a_0((p-1)-(-1))
		= a_0p.
	\]
	We have that
	\[
		\operatorname{Tr}_{K/\Q}(x(1-\zeta)) \in (p)_\Z;
	\]
	hence, it must be the case that $a_0 \in \Z$.
	We use this same argument to inductively show this for all $a_i$.
\end{proof}
