%! TEX root = master.tex
\lecture{20}{9/12}

Our next question: given a prime $p \in \Z$, how does $(p)_R$ factor
in $R$?

\begin{theorem}
	Let $p \in \Z$ be an odd prime and let $K = \Q(\sqrt d)$
	with $d$ a square free integer.
	We set $R = \mathcal O_K$.
	\begin{enumerate}
		\item If $d \equiv 0 \pmod p$, then
		$(p)_R = \mathfrak p \overline{\mathfrak p}$ where 
		$\mathfrak p = (p, m - \sqrt d)_R$ (a prime ideal)
		and $\mathfrak p \neq \overline{\mathfrak p}$,
		we say that $(p)_\Z$ ramifies in $R$.

		\item If $d \equiv m^2 \not\equiv 0 \pmod p$,
		then $(p)_R = \mathfrak p \overline{\mathfrak p}$
		where $\mathfrak p = (p, m - \sqrt d)_R$ (a prime ideal)
		and $\mathfrak p \neq \overline{\mathfrak p}$,
		we say that $(p)_\Z$ splits in $R$.

		\item If there is no $m \in \Z$ such that
		$d \equiv m^2 \pmod p$, then $(p)_R$ is a prime ideal.
		We say that $(p)_\Z$ is inert in $R$.
	\end{enumerate}
\end{theorem}

\begin{proof}
	$(i)$: if $p \mid d$, then we set $\mathfrak p = (p, \sqrt d)_R$
	and then
	\[
		\mathfrak p^2
		= (p, \sqrt d)_R (p, \sqrt d)_R
		= (p^2, p\sqrt d, d)_R
		= (p)_R \left( p, \sqrt d, \frac dp \right)_R.
	\]
	But $\gcd\left( p, \frac dp \right) = 1$ since $d$ is square free.
	Hence,
	$xp + y\frac dp = 1$ for some $x, y \in R$.
	That is, $1 \in \left( p, 2d, \frac dp \right)_R$
	and so $\left( p, 2d, \frac dp \right)_R = R$.

	$(ii)$: if $d \equiv m^2 \not\equiv 0 \pmod p$,
	then if we set
	$\mathfrak p = (p, m - \sqrt d)_R$ we have
	$\mathfrak p \overline{\mathfrak p} = \left( \norm(\mathfrak p) \right)_R$,
	but
	\begin{align*}
		\norm(\mathfrak p)
		&= \gcd\left( 
			\norm{p},
			\norm(m - \sqrt d),
			\trace(p(m + \sqrt d))
		 \right) \\
		 &= \gcd(p^2, m^2 - d, 2mp) \\
		 &= p.
	\end{align*}
	In this case, we cannot have $\mathfrak p = \overline{\mathfrak p}$
	as otherwise $2m = (m + \sqrt d) + (m - \sqrt d) \in \mathfrak p$
	and hence $1 = \gcd(2m, p) \in \mathfrak p$; a contradiction as
	$\mathfrak p \neq R$.

	$(iii)$: assume $(p)_R$ is not prime and hence not maximal.
	Then it must be contained within a prime ideal with at most two generators
	$\alpha$ and $\beta$ not both of which are divisible by $p$.
	We can assume $p \nmid \alpha = \frac{r + s\sqrt d}2$.
	As $\norm(\mathfrak p) \mid \norm\left( (p)_R \right)$,
	we deduce that $\norm(\mathfrak p) = p$ since $\mathfrak p \neq R$ implies
	$\norm(\mathfrak p) \neq 1$
	and $\mathfrak p \neq (p)_R$ implies $\norm(\mathfrak p) = p^2$.
	But since
	\[
		\norm{\mathfrak p}
		= \gcd(\norm(\alpha), \norm(\beta), \trace(\alpha \overline\beta)).
	\]
	We have in particular, 
	$p \mid \alpha \overline\alpha = \frac{r^2 - s^2d}4$.
	So $r^2 \equiv s^2d \pmod{4p}$.
	In particular, $r^2 \equiv s^2d \pmod p$.
	Now $s$ must be invertible modulo $p$, otherwise$p \mid s$ and hence
	$p \mid r$ and $p \mid \alpha$.
	Therefore,
	$d \equiv \left( rs^{-1} \right)^2 \pmod p$, a square;
	a contradiction.
\end{proof}

\begin{theorem}
	We consider the case $p = 2$.
	\begin{enumerate}
		\item If $d \equiv 2 \pmod 4$, then $(2)_R = \mathfrak p^2$
		where $\mathfrak p = (2, \sqrt d)_R$.
		We say that $(2)_\Z$ ramifies in $R$.

		\item If $d \equiv 3 \pmod 4$, then $(2)_R = \mathfrak p^2$,
		where $\mathfrak p = (2, 1 + \sqrt d)_R$.
		Again, we say that $(2)_\Z$ ramifies in $R$.

		\item If $d \equiv 1 \pmod 8$, 
		then $(2)_R = \mathfrak p \overline{\mathfrak p} \neq \mathfrak p^2$
		where $\mathfrak p = \left( 2, \frac{1 + \sqrt d}2 \right)_R$.
		We say that $(2)_\Z$ splits in $R$.

		\item If $d \equiv 5 \pmod 8$, then $(2)_R$ is prime.
		We say that $(2)_\Z$ is inert in $R$.
	\end{enumerate}
\end{theorem}

\begin{proof}
	The proof is similar to above, so we roughly sketch the proof here.
	$(i)$: $\norm(\mathfrak p) = \gcd(4, 2 \sqrt d, d) = 2$,
	hence $\mathfrak p \overline{\mathfrak p} = (2)_R$ but clearly
	$\mathfrak p = \overline{\mathfrak p}$.
	
	$(ii)$: $\norm(\mathfrak p) 
	= \gcd(4, 2(1 + \sqrt d), (1 + d) + 2\sqrt d) = 2$, hence
	$\mathfrak p \overline{\mathfrak p} = (2)_R$ and 
	$\overline{\mathfrak p} = (2, 1 - \sqrt d)_R = (2, -1 - \sqrt d)
	= \mathfrak p$.

	$(iii)$: $\norm(\mathfrak p) = \gcd\left( 4, \frac{1 - d}4, 2 \right) = 2$
	hence $\mathfrak p \overline{\mathfrak p} = (2)_R$ and
	$\overline{\mathfrak p} \neq \mathfrak p$,
	otherwise
	$1 = \frac{1 + \sqrt d}2 + \frac{1 - \sqrt d}2 \in \mathfrak p$.

	$(iv)$: If $(2)_R$ is not prime, then $ \mathfrak p \supset (2)_R$ 
	some prime $\mathfrak p$ with two generators:
	$\mathfrak p = (\alpha, \beta)_R$, not both of which are divisible
	by $2$.
	We assume $2 \nmid \alpha = \frac{r + s\sqrt d}2$, say $r,s \in \Z$ and 
	$r \equiv s \pmod 2$.
	Arguing as in the previous theorem, we have that
	$2 \mid \norm(\mathfrak p)$ and by the Hurtwitz
	\[
		2 \mid \gcd(
			\alpha\overline\alpha,
			\alpha\overline\beta,
			\overline\alpha\beta,
			\beta\overline\beta
		).
	\]
	Hence, in particular
	\[
		2 \mid \alpha\overline\alpha = \frac{r^2 - ds^2}{4}.
	\]
	That is, $r^2 \equiv ds^2 \pmod 8$.
	If $r$ is odd, then
	$r^2 \equiv dr^2 = 5r^2 \pmod 8$ which contradicts
	$r^2 \equiv 1 \pmod 8$.
	If $r$ is even, then $s$ is also even.
	Thus the congruence reads
	\[(2r')^2 \equiv d(2s')^2 \pmod 8\]
	where $r' = \sfrac r2$ and $s' = \sfrac s2$.
	Since $d \equiv 1 \pmod 2$, we obtian that
	$(r')^2 \equiv (s')^2 \pmod 2$ and so
	$r' \equiv s' \pmod 2$.
	But then
	\[
		2 \mid 2 \cdot \frac{r' + s' \sqrt d}2
		= r' + s' \sqrt d
		= \alpha;
	\]
	a contradiction.
\end{proof}