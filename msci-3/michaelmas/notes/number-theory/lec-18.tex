%! TEX root = master.tex
\lecture{18}{2/12}

We showed $I \subset R$ has a particular form:
$I = n\Z + (a + b\omega)\Z$ with
\begin{enumerate}
	\item $n,m \neq 0$,
	\item $n \mid m$,
	\item $n \mid a$,
	\item $n \mid m \operatorname{N}(b + \omega)$ where $a = bm$, \emph{and}
	\item
		$\omega = \begin{cases}
			\sqrt d & d \equiv 2,3 \pmod 4, \\
			\frac{1 + \sqrt d}{2} & d \equiv 1 \pmod 4. \\
		\end{cases}$
\end{enumerate}

In particular, for any ideal $I \subset R$, there is $\alpha, \beta \in R$
such that $I = (\alpha, \beta)_R$ since we can always take
$\alpha = n$ and $\beta = a + b \omega$.

We may not need two generators though,
for example if $I = R$ then $\alpha = 1$ is enough,
we don't need $\beta$.

Our aim is to, given an ideal $I = (\alpha, \beta)_R \subset R$,
calcualted $I^{-1}$.

\begin{lemma}[Hurwitz]
	Let $\alpha, \beta \in R$ and $m \in \N$.
	If
	\[
		m \mid \operatorname{N}(\alpha), \qquad
		m \mid \operatorname{N}(\beta), \qquad
		m \mid \operatorname{Tr}(\alpha\overline\beta)
	\]
	then
	\[
		m \mid \alpha\overline\beta, \qquad
		m \mid \overline\alpha\beta
	\]
	in $R$.
\end{lemma}

\begin{proof}
	Set $\gamma = \frac{\alpha\overline\beta}{m} \in K$.
	Our aim is to show that $\gamma \in R$.
	Now 
	\[
		\overline\gamma 
		= \overline{\frac{\alpha\overline\beta}{m}}
		= \frac{\overline\alpha\beta}{m}
	\]
	so
	\[
		\operatorname{Tr}(\gamma)
		= \gamma + \overline\gamma
		= \frac{\alpha\overline\beta}{m}
			+ \frac{\overline\alpha\beta}{m}
		= \frac{\alpha\overline\beta + \overline\alpha\beta}{m}
		= \frac{\operatorname{Tr}(\alpha\overline\beta)}{m} 
	\]
	and
	\[
		\operatorname{N}(\gamma)
		= \gamma\overline\gamma
		= \left( \frac{\alpha\overline\beta}m \right)
			\left( \frac{\overline\alpha\beta}m \right)
		= \frac{\alpha\overline\alpha}{m} \frac{\beta\overline\beta}{m}
		= \frac{\operatorname{N}(\alpha)}{m} \cdot
			\frac{\operatorname{N}(\beta)}{m}.
	\]
	In particular,
	$m \mid \operatorname{N}(\alpha)$, $m \mid \operatorname{N}(\beta)$,
	and
	$m \mid \operatorname{N}(\alpha\overline\beta)$.
	We have $\operatorname{Tr}(\gamma), \operatorname{N}(\gamma) \in \Z$,
	hence $\gamma$ is the root of the following polynomial
	\[
		x^2 - \operatorname{Tr}(\gamma) x + \operatorname{N}(\gamma) = 0
	\]
	and we substitute to get
	\[
		x^2 - (\gamma + \overline\gamma) x + \gamma\overline\gamma = 0.
	\]
	\emph{But}, 
	$x^2 - (\gamma + \overline\gamma)x + \gamma\overline\gamma \in \Z[x]$.
	Thus $\gamma \in \overline\Z \cap K = R$.
	So $\frac{\alpha\overline\beta}m \in R$ and so 
	$m \mid \alpha\overline\beta$.
	Similarly $\frac{\overline \alpha \beta}m \in R$ and so
	$m \mid \overline\alpha \beta$.
\end{proof}

\begin{proposition}[]
	Let $I = (\alpha, \beta)_R$ for some $\alpha, \beta in R$.
	Then
	\[
		I \cdot \overline I 
		= (\alpha, \beta)_R (\overline\alpha, \overline\beta)_R
		= (f)_R,
	\]
	where $f = \operatorname{gcd}\left(
		\operatorname{N}(\alpha),
		\operatorname{N}(\beta),
		\operatorname{Tr}(\alpha\overline\beta)
	\right) \in \N$.
	In particular, $I^{-1} = (f^{-1})_R \cdot \overline I$.
\end{proposition}

\begin{proof}
	Now
	$
		\overline I = \left\{
			\overline x \in R: x \in I
		\right\}
	$
	so if $I = (\alpha, \beta)_R$,
	$\overline I = (\overline\alpha, \overline\beta)_R$.
	Observe
	\begin{align*}
		I \cdot \overline I
		&= (\alpha, \beta)_R (\overline\alpha, \overline\beta)_R \\
		&= (\alpha\overline\alpha, \beta\overline\beta, \alpha\overline\beta,
			\overline\alpha\beta)_R \\
		&= (\operatorname{N}(\alpha), \operatorname{N}(\beta),
			\alpha\overline\beta, \overline\alpha\beta)_R.
	\end{align*}
	We set 
	$f = \gcd(\operatorname{N}(\alpha),
	\operatorname{N}(\beta),
	\operatorname{Tr}(\alpha\overline\beta))$.
	By Hurwitz Lemma,
	\[
		\frac{\alpha\overline\beta}f, \frac{\overline\alpha\beta}f \in \Z.
	\]
	By the definition of $f$, $f \mid \operatorname{N}(\alpha)$,
	$f \mid \operatorname{N}(\beta)$, 
	and $f \mid \operatorname{Tr}(\alpha\overline\beta)$.
	So by the satement of Hurwitz, $f \mid \alpha\overline\beta$ and
	$f \mid \overline\alpha\beta$.
	So
	\[
		I \cdot \overline I
		= (\operatorname{N}(\alpha), \operatorname{N}(\beta),
			\alpha\overline\beta, \overline\alpha \beta)_R
		= (f)_R \left( \frac{\operatorname{N}(\alpha)}f,
			\frac{\operatorname{N}(\beta)}{f},
			\frac{\alpha\overline\beta}{f},
			\frac{\overline\alpha\beta}{f} 
		\right)_R
	\]
	with 
	$
		\frac{\operatorname{N}(\alpha)}{f},
		\frac{\operatorname{N}(\beta)}{f},
		\frac{\alpha\overline\beta}{f},
		\frac{\overline\alpha\beta}{f}
		\in R
	$.
	Now define
	\[
		J = \left( 
			\frac{\operatorname{N}(\alpha)}{f},
			\frac{\operatorname{N}(\beta)}{f},
			\frac{\alpha\overline\beta}{f},
			\frac{\overline\alpha\beta}{f}
		\right).
	\]
	We need tos how that $J = R$.
	To show this, we show that $1 \in J$.
	Since $f = \gcd(\operatorname{N}(\alpha), \operatorname{N}(\beta),
	\operatorname{Tr}(\alpha\overline\beta)$
	there is $m_1, m_2, m_3 \in \Z$ such that 
	\[
		f = m_1 \operatorname{N}(\alpha)
		+ m_2 \operatorname{N}(\beta)
		+ m_3 \operatorname{Tr}(\alpha\overline\beta)
	\]
	and hence
	\[
		1 
		= m_1 \frac{\operatorname{N}(\alpha)}{f} 
			+ m_2 \frac{\operatorname{N}(beta)}{f} 
			+ m_3 \frac{\operatorname{Tr}(\alpha\overline\beta}{f} 
	\]
	but
	\[
		J = \left( 
			\frac{\operatorname{N}(\alpha)}f,
			\frac{\operatorname{N}(\beta)}{f},
			\frac{\alpha\overline\beta}{f},
			\frac{\overline\alpha\beta}{f} 
		\right).
	\]
	In particular,
	\[
		\frac{\alpha\overline\beta + \overline\alpha\beta}{f} 
		= \frac{\operatorname{Tr}(\alpha\overline\beta)}{f}.
	\]
	Thus, $1 \in J$ and so $J = R$.
	Thus $I \cdot \overline I = (f)_R$.
	So
	\[
		I \cdot \left( \left( f^{-1} \right)_R \overline I \right) = R
	\]
	and therefore
	\[
		I^{-1} = \left( f^{-1} \right)_R (\overline\alpha,
		\overline\beta)_R.
	\]
\end{proof}

Note that above, we may have $\beta = 0$.
In this case, $I = (\alpha)_R$, 
$f = \left\lvert \operatorname{N}(\alpha) \right\rvert$.
So \[
	I \cdot \overline I
	= \left( \left\lvert \operatorname{N}(\alpha) \right\rvert \right)_R
	= \left( \operatorname{N}(\alpha) \right)_R
\]
thus
\[
	(\alpha)_R (\overline\alpha)_R
	= (\alpha\overline\alpha)_R
\]
and therefore
\[
	\left( (\alpha)_R \right)^{-1}
	= \left( \alpha^{-1} \right)_R.
\]

\begin{example}[]
	Let $K = \Q(\sqrt{-11})$ and observe that
	$R = \Z\left[\frac{1 + \sqrt{-11}}2\right]$
	since $-11 \equiv -3 \equiv 1 \pmod 4$.
	Define
	\[
		I = (5 + 7\sqrt{-11}, 13 - 10\sqrt{-11}).
	\]
	We set $\alpha = 5 + 7\sqrt{-11}$ and
	$\beta = 13 - 10\sqrt{-11}$
	and calculate
	\begin{align*}
		f
		&= \gcd(\operatorname{N}(\alpha),
		\operatorname{N}(\beta),
		\operatorname{Tr}(\alpha\overline\beta)) \\
		&= \gcd(25 + 11\cdot49,
		169 + 11\cdot100,
		-705-141\sqrt{-11} -705 + 141\sqrt{-11}) \\
		&= \gcd(564,1269,-1410) \\
		&= 141.
	\end{align*}
	Hence
	\[
		I^{-1} = \left( \frac1{141} \right)_R \cdot
		\left(5 - 7\sqrt{-11}, 13 + 10\sqrt{-11}\right).
	\]
\end{example}

Our next aim is to understand the norm of an ideal with respect to its
generators.

\begin{lemma}[]
	Let $I = n\Z + (a + m\omega) \Z$ be an ideal in $R$ for some
	$n,m \in \N$.
	Then
	\[
		N(I) = \left\lvert R/I \right\rvert = nm.
	\]
\end{lemma}

\begin{proof}
	Claim: the following are all the representations of the classes
	of $R$ modulo $I$:
	\[
		S = \left\{
			r + s\omega:
			r \in \left\{
				0, \ldots, n-1
			\right\},
			s \in \left\{
				0, \ldots, m-1
			\right\}
		\right\}.
	\]
	First, we show that every element of $R$ ($x + y\omega$
	for $x, y \in \Z$) is congruent modulo $I$ to some element in $S$.
	We write $y = mq + s$ for some $q \in \Z$ and
	$s \in \left\{
		0, \ldots, m-1
	\right\}$.
	Then
	\[
		x + y\omega - q(a + m\omega) 
		= x - qa + (y - q \omega) \omega
		= x' + s\omega
	\]
	where $x' = x - qa$.
	Then
	\[
		x' + s\omega \equiv x + y \omega \pmod I
	\]
	since $q(a + m\omega) \in I$.
	Similary, we write
	\[
		x' = nq' + r, \qquad q' \in \Z, r \in \left\{
			0, \ldots,n-1
		\right\}.
	\]
	So
	\[
		x' + s\omega = nq' + r + s\omega \equiv r + sw \pmod I
	\]
	since $nq' \in I$.
	So
	\[
		x + y\omega \equiv r + s\omega \pmod I.
	\]
	So every element has a representation in $S$.
	We now must show that no two elements in $S$ are congruent.
	Assume otherwise.
	Then then is $r, r' \in \left\{
		0, \ldots,n-1
	\right\}$
	and $s, s' \in \left\{
		0, \ldots, m-1
	\right\}$
	such that
	\[
		r + s\omega \equiv r' + s'\omega \pmod I.
	\]
	So $(r - r') + (s - s') \omega \in I$.
	Therefore $m \mid s - s'$ and $n \mid r - r'$,
	a contradiction.
\end{proof}


