%! TEX root = master.tex

\lecture{4}{?}

\begin{definition}[Fields generated by elements]
	Let $F \subset L$ be a field extension and let $\alpha \in L$. 
	We define $F(\alpha) \subset L$ to be the smallest field extension of $F$
	that contains $\alpha$ and $F$.
	This field is called the field generated by $\alpha$ over $F$.
\end{definition}

We note that
\[
	F(\alpha) = \bigcap K
\]
where the intersection above is taken over all fields $K$ such that
$F \subset K \subset L$, and $\alpha \in K$. That is, $F(\alpha)$ can be
understood as the intersection of all fields in $L$ that contains $F$
and $\alpha$.

In fact, we define $F(\alpha_1, \ldots, \alpha_n)$ to be the smallest field
extension of $F$ that contains $F$ and all $\alpha_1, \ldots, \alpha_n$.
We call it the field generated by $\alpha_1, \ldots, \alpha_n$ over $F$.

Moreover, we can show that 
$
	F(\alpha_1, \ldots, \alpha_n) 
		= F(\alpha_1, \ldots, \alpha_{n-1})(\alpha_n)
$
by arguing inductively, and hence it is sufficient to show that
$F(\alpha, \beta) = F(\alpha)(\beta)$.
The field $F(\alpha, \beta)$ contains $F$ and $\alpha$, and hence it 
contains
$F(\alpha)$. Since it also contains $\beta$, we see that
$F(\alpha)(\beta) \subset F(\alpha, \beta)$ by the minimality of the field
$F(\alpha)(\beta)$.
On the other hand, since the field $F(\alpha)(\beta)$ contains $F$, 
$\alpha$, and $\beta$ we have by the minimality of $F(\alpha, \beta)$ that
$F(\alpha, \beta) \subset F(\alpha)(\beta)$.

\emph{Warning}: that $F(\alpha)$ should \emph{not} be confused with the
notation $F[\alpha]$.
The latter is by definition a \emph{ring}, defined as the values of
polynomials in $F[x]$ on $\alpha$.
That is,
\[
	F[\alpha]
	= \left\{
		\sum_{i=0}^n a_i\alpha^i : a_i \in F
	\right\}
	= \left\{ 
		f(\alpha) : f \in F[x]
	\right\}
	\subset L.
\]
In general we have $F(\alpha) \neq F[\alpha]$; for example, if $F = \Q$ and
$L = \C$ and take $\alpha = \pi$ then indeed we see that
$\Q(\pi) \neq \Q[\pi]$.
Indeed $\pi^{-1} \in \Q(\pi)$ but $\pi^{-1} \not\in \Q[\pi]$.

\begin{lemma}[]
	Let $F \subset L$ be a field extension and let $\alpha \in L$ be algebraic
	over $F$. Then
	\[
		F[\alpha] = F(\alpha).
	\]
\end{lemma}

\begin{proof}
	Let $p(x) = x^d + a_{d-1}x^{d-1} + \ldots + a_1x + a_0$ be the minimal 
	polynomial of $\alpha$ over $F$.
	Then, as $p(\alpha) = 0$, we have
	\[
		\alpha^d
		= -(a_{d-1} \alpha^{d-1} + \ldots + a_1 \alpha + a_0).
	\]
	So we have
	\[
		F[\alpha]
		= \left\{
			\sum_{i=0}^{d-1} b_i\alpha^i : b_i \in F
		\right\}.
	\]
	Hence $F[\alpha]$ is a $d$-dimensional vector space over $F$.
	Consider any non-zero $\beta \in F[\alpha]$
	and consider the $F$-linear map
	\[
		F[\alpha] \to F[\alpha], \qquad v \mapsto \beta v.
	\]
	The kernal of this map is trivial; hence,
	the linear map is surjective (recall that
	the dimension of the kernal plus the dimension of the image
	of a linear map is equal to the dimension of the vector space)
	so there is
	$\gamma \in F[\alpha]$ such that $\beta\gamma = 1$.
	Hence, $\beta$ has an inverse.
	Thus, we have proved that $F[\alpha]$ is a field.
	We have left to show that $F[\alpha] = F(\alpha)$.
	Note that $F[\alpha] \subset F(\alpha)$ and as
	$F[\alpha]$ contains $\alpha$ and $F$,
	we have that $F[\alpha] = F(\alpha)$.
\end{proof}

We note that this proof actually implies that
\[
	[F(\alpha): F] = d,
\]
where $d$ is the degree of the algebraic element $\alpha$
over $F$.

\begin{theorem}[]
	Let $L \supset K \supset F$ be field extensions.
	Then
	\[
		[L:F] = [L:K] \cdot [K:F].
	\]
\end{theorem}

\begin{proof}
	We first assume all field extensions are finite.
	Let $ \left\{ \alpha_1, \ldots, \alpha_r \right\} $
	be a basis of $K$ over $F$ and
	$ \left\{ \beta_1, \ldots, \beta_s \right\} $
	be a basis of $L$ over $K$.
	So $r = [K:F]$ and $s = [L:K]$.
	We claim that
	\[
		\mathcal B = \left\{
			\alpha_j \beta_k:
			j \in \left\{ 1, \ldots, r \right\},
			k \in \left\{ 1, \ldots, s \right\}
		\right\}
	\]
	is a basis for $L$ over $F$.

	Let $\gamma \in L$, then
	\[
		\gamma = \sum_{k=1}^s \lambda_k \beta_k
	\]
	for some $\lambda_k \in K$, as $L$ is a $K$-vector space.
	Now each $\lambda_k$ can be written as
	\[
		\lambda_k = \sum_{j=1}^r \mu_{jk}\alpha_j
	\]
	for some $\mu_{jk} \in F$.
	Hence
	\[
		\gamma = \sum_{k=1}^s
		\sum_{j=1}^r \mu_{jk} \alpha_j \beta_k,
	\]
	therefore $\mathcal B$ spans $L$ over $F$.
	Now suppose $
		\gamma 
		= \sum_{k=1}^s \sum_{j=1}^r \left(
			\mu_{jk} \alpha_j
		\right) \beta_k
		= 0
	$. 
	As $\beta_k$ form a basis of $K$ over $F$, it is necessary that
	$\mu_{jk} \alpha_j = 0$ for every $k \in \left\{ 1, \ldots, s \right\}$.
	Similarly, $\alpha_j$ are a basis for $L$ over $K$; hence,
	it is necessary that $\mu_{jk} = 0$ for every
	$j \in \left\{ 1, \ldots, r \right\}$.
	Hence, we have established linear indepedence of $\mathcal B$
	and so $\mathcal B$ is indeed a basis of $L$ over $F$.
	As $\mathcal B$ is a basis, we see that
	\[
		[L : F] = rs = [L:K] \cdot [K:F].
	\]
	Now, assume that $[K:F]$ is infinite.
	Then there are infinitely many elements of $K$
	(and hence of $L$) that are linearly independent over $F$,
	and so $[L:F]$ is also infinite.
	Similarly, if $[L:K]$ is infinite, then there are infinitely many elements
	of $L$ that are linearly independent over $K$,
	and so also linearly independent over $F$,
	so $[L:F]$ is infinite.
	On the other hand, if we know that $[L:F]$ is infinite
	then at least one of the $[L:K]$ or $[K:F]$ must be infinite
	as if both were finite then (from the first step of the proof)
	we would see that $[L:F]$ would be finite.
\end{proof}

