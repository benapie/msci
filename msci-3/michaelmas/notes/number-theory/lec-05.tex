%! TEX root = master.tex
\lecture{5}{20/10}

\begin{problem}
	Show that
	\[
		[\Q(\sqrt 2, \sqrt 3) : \Q] = 4.
	\]
\end{problem}

\begin{solution}
	Let $L = \Q(\sqrt 2, \sqrt 3)$.
	Set $K = \Q(\sqrt 2)$.
	Assume that $x^2 - 3$ is reducible over $K$.
	Then $\sqrt 3 \in K$ and so $\sqrt 3 = a + b \sqrt 2$
	for some $a, b \in \Q$.
	If $a = 0$, then $\sqrt 3 = b \sqrt 2$.
	This implies that $\sqrt 6 = b2 \in \Q$;
	a contradiction.
	If $a \neq 0$, then $a^2 + b^2 + 2ab\sqrt 2 = 3$
	(by squaring).
	Assuming $b \neq 0$: $\sqrt 2 \in \Q$; a contradiction.
	Finally, assuming $b = 0$ we have $\sqrt 3 \in \Q$,
	again a contradiction.
	Hence $x^2 - 3$ is irreducible over $K$.
	So $[K(\sqrt 3): K] = 2$ since $\sqrt 2$ has degree $2$
	over $K$.
	On the other hand, we see that $[K : \Q] = 2$ since 
	the irreducible polynomial of $\sqrt 2$ over $\Q$ is $x^2 - 2$.
	As $L = K(\sqrt 3)$, we see that
	\[
		[L: \Q] = [L:K] \cdot [K:\Q] = 4. \qedhere
	\]
\end{solution}

\begin{definition}[Algebraic]
	A number $\alpha \in \C$ which is algebraic over $\Q$ is said to be
	an \emph{algebraic number} .
	A field with $F$ with 
	$\Q \subset F \subset \C$
	and
	$[F : \Q] < \infty$
	is said to be an \emph{algebraic number field}
	(or simply \emph{number field}).
\end{definition}

\begin{example}
	$\Q\left( \sqrt{-2} \right)$,
	$\Q\left( \sqrt m \right)$,
	and $\Q\left( \sqrt 2, \sqrt 3 \right)$
	are examples of algebraic number fields.
\end{example}

\begin{theorem}[Simple extension theorem]
	Let $K$ be an algebraic number field.
	Then there is $\theta \in K$ such that
	\[
		K = \Q(\theta).
	\]
\end{theorem}

\begin{example}[]
	With $L = \Q(\sqrt 2, \sqrt 3)$ we have
	$L = \Q(\sqrt 2 + \sqrt 3)$.
	We see that $\theta = \sqrt 2 + \sqrt 3$
	is indeed in $L$.
	Observe that $\theta^3 = 11\sqrt 2 + 9\sqrt 3$,
	so we can express
	$\sqrt 2 = \frac12 (\theta^3 - 9\theta) \in \Q(\theta)$
	and
	$\sqrt 3 = -\frac12 (\theta^3 - 11\theta) \in \Q(\theta)$.
	This would indeed imply that $L \subset \Q(\theta)$
	and clearly $\Q(\theta) \subset L$;
	hence, we have equality.
\end{example}

The sum, difference, product, and quotient (with non-zero denominator)
of two algebraic numbers is algebraic; hence, we can define the
following set.

\begin{definition}[The field of algebraic numbers]
	The set $\overline \Q$ is defined as the union of all algebraic numbers.
\end{definition}

\begin{problem}
	Show that $\overline\Q$ is a field.
\end{problem}

\begin{solution}
	Let $\alpha \in \C$ be algebraic and non-zero.
	We have $[\Q(\alpha) : \Q] < \infty$, and so
	$\Q(\alpha)$ is algebraic.
	Namely, $\alpha^{-1} \in \Q(\alpha)$ is algebraic.
	Further let $\beta \in \C$ also be algebraic.
	Then, by the tower theorem, $\Q(\alpha, \beta)$
	is a finite extension of $\Q$ and so 
	$\Q(\alpha, \beta)$ is algebraic.
	Now clearly $\alpha + \beta, \alpha\beta \in \Q(\alpha, \beta)$
	and so they are both algebraic.
\end{solution}

\begin{problem}
	Show that
	\[
		[\overline\Q: \Q] = \infty.
	\]
\end{problem}

\begin{solution}
	If it were not, then every algebraic number would have degree
	$d$ or less over $\Q$, but by Eisenstein's criterion we can construct
	irreducible polynomials over $\Q$ of degree larger than $d$;
	for example, $x^{d+1}-2$.
	From this, we conclude that $\sqrt[d+1]{2}$ has degree $d + 1$
	and it is algebraic, but cannot be contained with $\overline\Q$;
	a contradiction.
\end{solution}
