%! TEX root = master.tex
\lecture{19}{9/12}

\begin{proposition}
    Let \(I\) be an ideal in \(R\).
    Then \(I \cdot \overline I = \left(N(I)\right)_R\).
    In particular, if $I = (\alpha, \beta)_R$ then
    \[
        N(I) =
        \gcd\left(
            \operatorname{N}(\alpha),
            \operatorname{N}(\beta),
            \operatorname{Tr}(\alpha)
        \right).
    \]
\end{proposition}

\begin{proof}
    We have seen that if $I = n \Z + (a + m \omega) \Z$, 
    then $\operatorname{N}(I) = nm$.
    Also, if $a = mb$ we have seen that
    \begin{align*}
        I \cdot \overline I
        &= \left(n, m(b + \omega)\right)_R \cdot
            \left( n, m(b + \overline\omega) \right)_R \\
        &= (nm)_R \cdot 
            \left( nm (b + \omega) \right)_R \cdot
            \left( nm(b + \overline\omega) \right)_R \cdot
            \left( m^2 \operatorname{N}(b + \omega) \right)_R
    \end{align*}
    and we know that $m$ divides $n$ hence, $n = m c_1$ for some $c_1 \in \Z$.
    Moreover, $n$ divides $m \operatorname{N}(b + \omega)$ and so
    $m \operatorname{N}(b + \omega) = nc_2$.
    In particular,
    \begin{align*}
        \left( n^2, mn(b + \omega), mn(b + \overline\omega), 
        m^2 \operatorname{N}(b + \omega) \right)
        = (nm)_R (c_1, b + \omega, b + \overline\omega, c_2)_R.
    \end{align*}
    That is,
    \[
        (f)_R = (nm)_R (c_1, b + \omega, b + \overline\omega, c_2)_R 
    \]
    and hence $(c_1, b + \omega, b + \overline\omega, c_2)_R$
    must be principal: generated by an integer.
    But the only integers that divide $b + \omega$ are $1$ and $-1$.
    Hence, we conclude that $(c_1, b + \omega, b + \overline\omega, c_2)_R = R$.
    Hence, 
    \(I \cdot \overline I = \left( \operatorname{N}(I) \right)_R\).
    Now let us assume that \(I = (\alpha, \beta)_R\).
    We know that \(I \cdot \overline I = (f)_R\) where
    \(
        f = \gcd\left(\operatorname{N}(\alpha),
                      \operatorname{N}(\beta),
                      \operatorname{Tr}(\alpha \overline\beta)
                \right)
    \).
    But then we should have \((f)_R = \left( \operatorname{N}(I) \right)_R\)
    and hence \(f = u \operatorname{N}(I)\) for some \(u \in R^\times\).
    Since \(f, \norm(I) \in \Z_+\) we actually have \(u = 1\).
    That is, 
    \(
        \norm(I) 
        = f 
        = \gcd(\norm(\alpha), \norm(\beta), \trace(\alpha\overline\beta))
    \).
\end{proof}

Note that if an ideal \(I\) is principal, say \(I = (\alpha)_R\),
then \(\norm(I) = \abs{N(\alpha)}\).
Indeed,
\(
    \left( \norm(I) \right)_R
    = (\alpha)_R (\overline\alpha)_R
    = (\norm(\alpha))_R
\).
We also know that \(\norm(I) > 0\),
hence \(\norm(I) = \abs{\norm(\alpha)}\).

If \(I, J \subset R\) are ideals of \(R\),
then
\[\norm(IJ) = \norm(I) \cdot \norm(J).\]
That is, the norm is multiplicative.
Indeed,
\[
    \left( \norm(IJ) \right)_R
    = IJ \overline{IJ}
    = I \overline I J \overline J
    = \left( \norm(I) \right)_R \left( \norm(J) \right)_R
    = \left( \norm(I) \norm(J) \right)_R.
\]
That is, 
\[
    \frac{\norm(IJ)}{\norm(I)\norm(J)} \in R^\times \cap \Q
    = \Z^\times
    = \left\{ 1, -1 \right\}.
\]
Since the norm of an ideal is always a positive integer, we have
\(\norm(IJ) = \norm(I) \norm(J)\).

\begin{theorem}
    Let \(\mathfrak p \subset R\)
    be a non-zero prime ideal, then
    \begin{enumerate}
        \item there is a unique prime \(p \in \Z\) such that
        \(p \mid (p)_R\);

        \item either \(\norm(\mathfrak p) = p^2\) and \(\mathfrak p = (p)_R\)
        or \(\norm(\mathfrak p) = p\) and 
        \(\mathfrak p \overline{\mathfrak p} = (p)_R\).
    \end{enumerate}
\end{theorem}

\begin{proof}
    We note that if \(\mathfrak p\) is a prime ideal, so is
    \(\overline{\mathfrak p}\).
    Indeed, if \(ab \in \overline{\mathfrak p}\) then
    \(\overline a \overline b \in \mathfrak p\).
    Since this ideal is prime, then
    \(\overline a \in \mathfrak p\)
    or \(\overline b \in \mathfrak p\).
    That is,
    \(a \in \overline{\mathfrak p}\)
    or \(b \in \overline{\mathfrak p}\).

    $(i)$: we have 
    $\mathfrak p \mid \mathfrak p \overline{\mathfrak p} =
    \left( \norm(\mathfrak p) \right)_R$.
    But \(\norm(\mathfrak p) \in \Z\) and hence
    $N(\mathfrak p) = p_1 p_2 \ldots p_n$
    for $p_i \in \Z$ prime.
    But then 
    $\mathfrak p \mid (p_1)_R \cdot (p_2)_R \cdot \ldots \cdot (p_n)_R$.
    Since $\mathfrak p$ is a prime ideal, $\mathfrak p \mid (p_j)_R$ for some
    $j$.
    Take $p_i = p_j$.
    Let $q \in \Z$ prime, $q \neq p$, and $\mathfrak p \mid (q)_R$.
    That is, $\mathfrak \supset (q)_R$ and $q \in \mathfrak p$.
    Similarly, $p \in \mathfrak p$.
    Since $\gcd(p, q) = 1$, we have $px + qy = 1$ for some $x, y \in \Z$.
    That is, 
    $1 \in \mathfrak p$; a contradiction.
    Hence $p$ is unqiue.
    
    $(ii)$: since $\mathfrak p \subset (p)_R$, we have
    $\norm(\mathfrak p) \mid \norm\left( (p)_R \right)$.
    Equivalently,
    $\norm(\mathfrak p) \mid p^2$.
    Since $\mathfrak p \neq R$, we have
    $\norm(\mathfrak p) \neq 1$.
    That is, either $\norm(\mathfrak p) = p^2$ or
    $\norm(\mathfrak p) = p$.
    If $\norm(\mathfrak p) = p$, we must have $\mathfrak p = (p)_R$
    since $\mathfrak p \supset (p)_R$ and 
    $\norm(\mathfrak p) = \norm\left( (p)_R \right)$.
    If $\norm(\mathfrak p) = p^2$, then
    $\left( \norm(\mathfrak p) \right)_R = (p)_R$
    and hence $\mathfrak p \overline{\mathfrak p} = (p)_R$.
\end{proof}