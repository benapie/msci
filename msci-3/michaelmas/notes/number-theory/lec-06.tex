%! TEX root = master.tex
\lecture{6}{21/10}

\begin{definition}[Field norm and trace]
	Let $F$ be a field, $L$ a finite extension of $F$
	with $d = [L:F]$, and $\alpha \in L$.
	We consider the $F$-linear map
	\[
		\hat\alpha: L \to L, \qquad 
		x \mapsto \alpha x.
	\]
	Let $A \in M_d(F)$ be the corresponding to the above linear map
	respective to some basis.
	We define the \emph{norm} of $\alpha$ from $L$ to $F$ as
	$N_{L/F}(\alpha) = \det(A)$
	and we define the \emph{trace} of $\alpha$ from $L$ to $F$ as
	$\operatorname{Tr}_{L/F}(\alpha) = \operatorname{Tr}(A)$.
\end{definition}

Recall from Linear Algebra I that the characteristic polynomial
does not depend on the basis fixed.
We can also find the trace and the determinant in the coefficients
of $A$ as
$
	p_A(X) 
	= X^d 
		- \operatorname{Tr}(A) X^{d-1} 
		+ \ldots + (-1)^d \det(A)
$.

\begin{problem}
	Let $L = \Q(\sqrt m) \supset \Q = F$ for some square free integer
	$m$ and consider any element
	$\alpha = a + b\sqrt m \in L$.
	Compute $N_{L/K}(\alpha)$ and $\operatorname{Tr}_{L/K}(\alpha)$.
\end{problem}

\begin{solution}
	Fix basis $ \left\{ 1, \sqrt m \right\} $
	for $L$. 
	Observe that $\hat\alpha(1) = a + b\sqrt m$
	and $\hat\alpha(\sqrt m) = bm + a\sqrt m$, 
	so we construct our corresponding matrix
	\[
		A =
		\begin{pmatrix}
			a & bm \\
			b & a \\
		\end{pmatrix}
		.
	\]
	Thus $N_{L/K}(\alpha) = a^2 - b^2m$
	and $\operatorname{Tr}_{L/K}(\alpha) = 2a$.
\end{solution}

\begin{proposition}[Properties of $N_{L/K}$ and $\operatorname{Tr}_{L/K}$]
	Let $F$ be a field and $L$ a finite extension of $F$
	with $d = [L:F]$.
	Then
	\begin{enumerate}
		\item for all $\alpha \in L$,
			$N_{L/F}(\alpha) = 0 \iff \alpha = 0$;

		\item for all $\alpha, \beta \in L$,
			$N_{L/F}(\alpha\beta) = N_{L/F}(\alpha) N_{L/F}(\beta)$;

		\item for all $\alpha \in F$,
			$N_{L/F}(\alpha) = \alpha^d$ and
			$\operatorname{Tr}_{L/F}(\alpha) = d\alpha$;

		\item for all $\alpha, \beta \in L$ and $\lambda, \mu \in F$,
			$\operatorname{Tr}_{L/F}(\lambda\alpha + \mu\beta)
			= \lambda \operatorname{Tr}_{L/F}(\alpha)
			+ \mu \operatorname{Tr}(\beta)$; and

		\item consider the tower $F \subset K \subset L$, 
			then for $\alpha \in L$,
			\[
				N_{L/F}(\alpha) = N_{K/F} \left( N_{L/K}(\alpha) \right),
				\quad
				\operatorname{Tr}_{L/F}(\alpha)
				= \operatorname{Tr}_{K/F} \left( 
					\operatorname{Tr}_{L/K}(\alpha)
				\right).
			\]
	\end{enumerate}
\end{proposition}

\begin{proof}
	$(i)$: we see that $\hat\alpha$ is injective if and only if $\alpha$ is 
	non-zero, that is, $\det(A) \neq 0$.
	$(ii)$, $(iii)$, and $(iv)$ are clear from the definitions of the
	norm and trace.
	$(v)$ is the hardest proof to obtain and we will omit it.
\end{proof}

\section{Algebraic integers}

Recall that if $\alpha \in \overline\Q$, then there exists a
$f(x) \in \Q[x]$ such that $f(\alpha) = 0$.
After clearing the denominaors, we can obtain a $g(x) \in \Z[x]$
such that $g(\alpha) = 0$; however,
this polynomial does not have to be monic.

\begin{definition}[Algebraic integers]
	An algebraic number $\alpha$ is said to be an \emph{algebraic integer} 
	if there is $g(x) \in \Z[x]$ which is monic and $g(\alpha) = 0$.
	We denote the set of algebraic integers as $\overline\Z$.
\end{definition}

\begin{examples}
	\begin{enumerate}
		\item
		Clearly all integers are also algebraic integers, as for $\alpha \in \Z$
		we can construct $g(x) = x - \alpha$ which is monic and $g(\alpha) = 0$.

		\item
		Let $m \in \N$ and $D \in \Z$.
		Then $\sqrt[m]{D}$ is an algebraic integer, since it is the root of
		$x^m - D$.	

		\item
		$\frac{1 + \sqrt{-3}}{2}$ is a root of $x^2 - x + 1 \in \Z[x]$,
		and so it is an algebraic integer.
	\end{enumerate}
\end{examples}

Our aim is to show that $\overline\Z$ defines a ring,
which is not immediately obvious.
For example, it is not clear from a glance whether the element
$i + \frac{1 + \sqrt{-3}}{2}$ is a root of a monic polynomial 
with coefficients in $\Z$.

\begin{theorem}[]
	Let $\alpha$ be an algebraic number and $p(x) \in \Q[x]$
	be its minimal polynomial.
	Then the following are equivalent.
	\begin{enumerate}
		\item $\alpha$ is an algebraic integer.
		\item $p(x) \in \Z[x]$.
		\item $
			\Z[\alpha] 
			= \left\{ \sum_{i=0}^{d-1} a_i\alpha^i : a_i \in \Z \right\}
		$.
	\item There is a non-trivial finitely generated abelian group (under
		addition) $G \subset \C$ such that $\alpha G \subset G$.
	\end{enumerate}
\end{theorem}
