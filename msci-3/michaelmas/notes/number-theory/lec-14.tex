%! TEX root = master.tex
\lecture{14}{20/11}

\begin{corollary}
	Let $I$ be a non-zero ideal $I$ in $R$.
	Then $R / I$ is finite.
\end{corollary}

\begin{proof}
	We know that $I \cap \Z$ is non-zero.
	Thus there is a non-zero $m \in I \cap \Z$.
	We have $(m)_R \subset I \subset R$ and
	\[
		R = \Z \alpha_1 + \ldots + \Z \alpha_r
	\]
	for some $r \in \N$.
	But then $(m)_R = mR = \Z m \alpha_1 + \ldots + \Z m \alpha_n$.
	Hence
	\[
		R / (m)_R \cong \Z/m\Z \times \ldots \times \Z/m\Z,
	\]
	$r$-many copies of $\Z/m\Z$.
	So $R/(m)_R$ is finite.
	By the general theory of abelian groups
	\[
		[R: (m)_R] = [R:I][I:(m)_R]
	\]
	and hence the RHS is finite.
	Thus $[R:I] < \infty$ and so
	$\left\lvert R/I \right\rvert < \infty$.
\end{proof}

\begin{definition}[]
	Let $I$ be a non-zero ideal in $R$.
	Then $N(I) = \left\lvert R/I \right\rvert \in \N$.
\end{definition}

From the Corollary above, this is well-defined.
We will note important properties of the norm, which will allow us to measure
the \emph{size} of an ideal.
\begin{enumerate}
	\item $N(R) = 1$.
	\item For ideals $I$ and $J$ with $I \supset J$ we have
		$N(I) \leq N(J)$. If $I \supset J$ and $I \neq J$, then
		$N(I) \leq N(J)$.
\end{enumerate}
These properties can be easily proved using the definitions.

\begin{proposition}[]
	Every non-zero prime ideal in $R$ is maximal.
\end{proposition}

\begin{proof}
	Let $\mathfrak p$ be a prime ideal in $R$.
	We know $R/\mathfrak p$ is finite.
	As $\mathfrak p$ is prime, $R/\mathfrak p$ is an integral domain
	which is finite.
	We claim that it is also a field.
	Indeed, given a non-zero element $\alpha \in R/\mathfrak p$ we consider
	the maps $R / \mathfrak p \to R / \mathfrak p$, $y \mapsto \alpha y$.
	This map is injective as $\alpha y_1 = \alpha y_2$.
	Thus $\alpha(y_1 - y_2) = 0$, and so $y_1 = y_2$ since $R/\mathfrak p$
	is an integral domain and $\alpha \neq 0$.
	But $R/\mathfrak p$ is finite so the map is surjective.
	Therefore there is $\beta \in R/\mathfrak p$ such that $\alpha \beta = 1$.
	Thus $\alpha$ is invertible.
	So $R/\mathfrak p$ is a field.
\end{proof}

From Algebra II, we have seen that if an ideal is maximal it is also prime.
Hence, in $R$, the notions of maximality and primality coincide.
Now we move onto showing that every prime ideal is invertible.

\begin{lemma}[]
	Let $I \subset R$ be a non-zero ideal.
	Then $I$ contains a product of one or more non-zero prime ideals.
\end{lemma}

\begin{proof}
	Assume otherwise.
	Suppose $I$ is such that $N(I)$ is minimal.
	Clearly $I$ is not prime.
	Then there is $b, b' \in R/I$ such that $bb' \in I$.
	Put $J = I + (b) \supset I$ ($I \neq J$) and
	$J' = I + (b')$ ($J' \neq I$).
	Then 
	\[
		N(I) < N(J), N(J').
	\]
	Hence $J$ contains prime ideals $\mathfrak p_1, \ldots, \mathfrak p_n$
	and $J'$ contains prime ideals $\mathfrak p_1', \ldots, \mathfrak p_m'$.
	Then 
	\[
		\mathfrak p_1 \ldots \mathfrak p_m \mathfrak p_1' \ldots \mathfrak p_m'
		\subset JJ' = (bb') + bI + b'I + I^2
		\subset I;
	\]
	a contradiction.
\end{proof}

\begin{theorem}[]
	Non-zero prime ideals in $R$ are invertible.
\end{theorem}

