%! TEX root = master.tex
\lecture{8}{5/11}

\begin{definition}[Ring of integers]
	For a number field $K$, we define
	\[
		\mathcal O_K = K \cap \overline\Z,
	\]
	the \emph{ring of integers of $K$}.
\end{definition}

$\mathcal O_K$ is indeed a ring as it is the intersection of a field
and a ring.

\begin{lemma}[]
	$\mathcal O_\Q = \Z$.
\end{lemma}

\begin{proof}
	As $\Z$ is contained within $\overline\Z$ and $\Q$, we have
	$\Z \subset \mathcal O_\Q$.
	Now let $\alpha \in \mathcal O_\Q$.
	Since $\alpha \in \Q$, there is minimal polynomaial $p(x) \in \Q[x]$ 
	with $\deg p = 1$;
	that is, $p(x) = x - \alpha$.
	As $\alpha \in \overline\Z$, we have that $p(x) \in \Z[x]$, and so
	$\alpha \in \Z$.
\end{proof}

\begin{lemma}[]
	Let $K$ be a number field. Then
	\[
		K =
		\left\{ 
			\frac{\alpha}m : 
			\alpha \in \mathcal O_K, m \in \Z \setminus \left\{ 0 \right\} 
		\right\}.
	\]
\end{lemma}

\begin{proof}
	Let $\beta \in K$.
	As $\beta$ is algebraic, there is $a_0, \ldots, a_n \in \Z$
	such that
	\[
		a_n\beta^n + \ldots + a_0 = 0,
	\]
	with $a_n \neq 0$.
	We multiply this by $a_n^{n-1}$ to get
	\[
		(a_n\beta)^n
		+ a_{n-1}(a_n\beta)^{n-1}
		+ \ldots
		+ a_1 a_n^{n-1} (\alpha_n\beta) + a_n^{n-1}a_0
		= 0
	\]
	and so $\alpha = a_n\beta$ is an algebraic integer.
	Paired with the fact that $\alpha \in K$, we see
	that $\alpha \in \mathcal O_K$.
	That is, $\beta = \frac{\alpha}{a_n}$, which shows
	our statement by taking $m = a_n$.
\end{proof}

\subsection{Quadratic fields}

\begin{definition}[Square-free integer]
	A \textbf{square-free integer} is an integer which is divisible by
	no perfect square other than 1.
\end{definition}

\begin{proposition}[]
	Let $K$ be an extension field of $\Q$ of degree 2.
	Then there is some square-free integer $d$ such that $K = \Q(\sqrt d)$.
\end{proposition}

\begin{definition}[Quadratic field]
	A \textbf{quadratic field} is a field $K = \Q(\sqrt d)$.
	If $d > 0$, $K$ is a \emph{real quadratic field}.
	If $d < 0$, $K$ is a \emph{imaginary quadratic field}.
\end{definition}

We will work towards determining the ring of integers for quadratic fields.

\begin{lemma}[]
	Let $K = \Q(\sqrt d)$ with $1 \neq d \equiv 1 \pmod 4$.
	Then
	\begin{enumerate}
		\item
			\[
				\Z \left[ \frac{1+\sqrt d}2 \right] 
				\subset \mathcal O_K;\;\text{and}
			\]

		\item
			\[
				\Z \left[
					\frac{1 + \sqrt d}2
				\right] = \left\{
					\frac{r + s\sqrt d}2:
					r,s \in \Z, r \equiv s \pmod 2
				\right\}.
			\]
	\end{enumerate}
\end{lemma}

\begin{proof}
	$(i)$: it is enough to show that $\frac{1 + \sqrt d}2 \in \mathcal O_K$
	since $\mathcal O_K$ is a ring.
	Clearly $\frac{1 + \sqrt d}2 \in K$ and it is a root of
	$x^2 - x - \frac{d-1}4 \in \Z[x]$.
	$(ii)$: let $\theta = \frac{1 + \sqrt d}2$.
	Observe that $\Z[\theta] = \Z + \theta\Z$ as
	$\theta^2 = \theta + \frac{d-1}4$.
	So if $\beta \in \Z[\theta]$ then there is $x,y \in \Z$ such that
	$\beta = x + y\theta$ or equivalently
	$
		\frac{(2x+y)+y\sqrt d}{2}.
	$
	Since $2x + y \equiv y \pmod 2$, we can take $r = 2x + y$ and
	$s = y$ to conclude the first inclusion.
	Conversely, if $\beta = \frac{r+s\sqrt d}2$ with $r,s \in \Z$ and
	$r \equiv s \pmod 2$ then 
	\[
		\beta = \frac{r-s}2 + s\left( 
			\frac{1 + \sqrt d}{2}  
		\right) \in \Z + \Z\theta = \Z[\theta].\qedhere
	\]
\end{proof}

We are ready to look at the ring of integers of quadratic fields.

\begin{theorem}[]
	Let $d$ be a square-free integer and $K = \Q(\sqrt d)$.
	Then
	\[
		\mathcal O_K =
		\begin{cases}
			\Z[\sqrt d] & d \equiv 2,3 \pmod 4, \\
			\Z\left[\frac{1 + \sqrt d}2\right] & d \equiv 1 \pmod 1. \\
		\end{cases}
	\]
\end{theorem}

\begin{proof}
	If $d \equiv 1 \pmod 4$, then 
	$\Z\left[\frac{1+\sqrt d}2\right] \subset \mathcal O_K$.
	If $d \equiv 2,3 \pmod 4$, then $\Z[\sqrt d] \subset \mathcal O_K$ as
	$\sqrt d \in \mathcal O_K$.

	Now we show the other inclusion.
	We let $\alpha = \frac{a + b\sqrt d}c \in \mathcal O_K$ with 
	$a,b,c \in \Z$.
	If $b = 0$, then $\alpha \in \Q \cap \overline\Z = \Z$.
	Now assume $b \neq 0$.
	Without loss of generality, we assume $\gcd(a,b,c) = 1$ and
	$c \in \N$.
	Then the minimum polynomial of $\alpha$ is given by
	\[
		p(x) = \left( 
			x - \frac{a + b\sqrt d}c 
		\right) \left( 
			x - \frac{a - b\sqrt d}c 
		\right)
		= x^2 - \frac{2a}c x + \frac{a^2 - b^2d}{c^2}
		\in \Z[x].
	\]
	Thus $\frac{2a}c, \frac{a^2 - b^2d}{c^2} \in \Z$.
	Now let $p$ be prime such that $p \mid a,c$.
	Then $p^2 \mid b^2d$.
	As $d$ is square free, $p \mid b^2$ and then
	$p \mid a,b,c$. 
	That is, $\gcd(a,b,c) > 1$; a contradiction.
	Hence $\gcd(a,c) = 1$.
	Thus $c \in \left\{
		1,2
	\right\}$.
	If $c=1$, then $\alpha = a+b\sqrt d \in \Z[\sqrt d]$ and
	$\alpha \in \Z\left[\frac{1 + \sqrt d}2\right]$ 
	since $a + b\sqrt d = (a - b) + 2b \frac{1 + \sqrt d}2$.
	If $c=2$, then $a$ and $b$ are odd.
	Furthermore, as $\frac{a^2 - b^2d}4 \in \Z$,
	$a^2 - b^2d \equiv 0 \pmod 4$.
	As $a$ and $b$ are odd, $a^2 \equiv b^2 \equiv 1 \pmod 4$.
	Hence $d \equiv 1 \pmod 4$.
\end{proof}

