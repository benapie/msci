%! TEX root = master.tex
\section{Dedekind domain}
\lecture{13}{20/11}

For this section, we will write $R$ to denote the ring of integers for some
number field $K$.
We have seen that for ideals $I$ and $J$ in $R$,
\[
	I \cdot J = \left\{
		\sum_{n \in \N} a_i b_j: a_i \in I, b_j \in J, n \in \N
	\right\}
\]
is again an ideal in $R$.
Note that
$I \cdot R = I$
and so
$I \cdot J \subset I \cdot R = I$
and similarly 
$I \cdot J \subset J$.
Recall that any non-proper ideal is prime if
$ab \in I \implies a \in I \;\text{or}\; b \in I$.

\begin{lemma}[]
	Let $\mathfrak p$ be a prime ideal in a ring $R$.
	Then
	\[
		\mathfrak p \supset IJ \iff
		\mathfrak p \supset I \;\text{or}\; \mathfrak p \supset J
	\]
	for ideals $I$ and $J$.
\end{lemma}

\begin{proof}
	$(\implies)$: 
	is obvious.
	$(\impliedby)$: 
	suppose 
	$\mathfrak p \supset IJ$, 
	but
	$\mathfrak p \not\supset I$.
	Then there is 
	$a \in I \setminus \mathfrak p$.
	Now for any $b \in J$, 
	$ab \in IJ \subset \mathfrak p$.
	So either $a \in \mathfrak p$ or $b \in \mathfrak p$.
	But $a \not\in \mathfrak p$ so $b \in \mathfrak p$.
	Thus $J \subset \mathfrak p$.
\end{proof}

We will now move on to define the divisibility of ideals.
That is, if for two ideals $I$ and $J$ in $R$ there is an ideal $S$ such that
$J = IS$, then we may write $I \mid J$.
This clearly implies that $I \supset J$, since $J = IS \subset IR = I$.
But the reverse statement is not at all obvious.
Our aim here is to show that a non-zero ideal $I$ in $R$ can be written as a
unique product of prime ideals.
Another interesting point is that we can use $R$ as an \emph{identity} when
multiplying ideals. 
This is key in how we define the inverse of an ideal.

\begin{definition}[Fractional ideal]
	A \emph{fractional ideal} of $R$ is a subgroup under addition of the form
	\[
		\lambda I = \left\{
			\lambda \alpha: \alpha \in I
		\right\} \subset K
	\]
	where $I$ is a non-zero ideal in $R$ and $\lambda \in K^\times$.
	We denote \emph{$\mathcal J(R)$} the set of fractional ideals in $R$.
\end{definition}

Clearly any ideal $I$ is also a fractional ideal, by taking $\lambda = 1$.

\begin{definition}[Principal fractional ideal]
	A fractional ideal of $R$ of the form $\lambda R$ is called a
	\emph{principal fractional ideal}, denoted $(\lambda)_R \subset K$.
	We write $\mathcal P(R)$ for the set of principal fractional ideals
	of $R$.
\end{definition}

Fractional ideals in $R$ act the same as integral ideals
(that is, non-fractional).
For fractional ideals $\mathfrak a$ and $\mathfrak b$ in $R$, we define
\begin{align*}
	\mathfrak a \cdot \mathfrak b &= \left\{
		\sum^{n}_{i=1} a_i b_i : a_i \in \mathfrak a, b_i \in \mathfrak b
	\right\}, \\
	\mathfrak a + \mathfrak b &= \left\{
		\sum^{n}_{i=1} (a_i + b_i): a_i \in \mathfrak a, b_i \in \mathfrak b
	\right\}.
\end{align*}

\begin{lemma}[]
	Let $\mathfrak a, \mathfrak b \in \mathcal J(R)$.
	Then $\mathfrak a + \mathfrak b$, $\mathfrak a \cdot \mathfrak b$, and
	$\mathfrak a \cap \mathfrak b$ are fractional ideals.
\end{lemma}

\begin{proof}
	%todo
\end{proof}

$\mathcal J(R)$ also has \emph{associativity}, 
\emph{commutativity}, \emph{distributivity}, and
for fractional ideals $\mathfrak a$, $\mathfrak b$, and $\mathfrak c$
if $\mathfrak a \subset \mathfrak b$ then
$\mathfrak c \mathfrak a \subset \mathfrak c \mathfrak b$.

\begin{definition}[Invertible ideal]
	A fractional ideal $\mathfrak a$ is \emph{invertible} if there is a
	fractional ideal $\mathfrak b$ of $R$ such that
	\[
		\mathfrak a \cdot \mathfrak b = R = (1)_R.
	\]
\end{definition}

Note that inverses of fractional ideals must be unique.
Indeed, if $\mathfrak a \neq \mathfrak c$ such that
$\mathfrak{ab} = \mathfrak{ac} = R$.
Then $\mathfrak{abc} = R$.
But $(\mathfrak{ac})\mathfrak b = \mathfrak{acb}$ and so
$\mathfrak b = \mathfrak c$. Thus $\mathfrak b = \mathfrak c$.
So we can safely write $\mathfrak a^{-1}$ as the inverse of any
$a \in \mathcal J(R)$.
Our aim is to show that for every $a \in \mathcal J(R)$, $a$ is invertible.
Infact, we show that $\mathcal J(R)$ is an abelian group.
This is why we do not include the zero fractional ideal in $\mathcal J(R)$.
If we are in a principal ideal domain, for every ideal $I$ in $R$
we have $I = (\lambda)_R$.
Indeed, $(\lambda I)(\lambda^{-1}I) = \lambda R \lambda^{-1} R = R$.
So $I$ is invertible.
But if we are not in a principal ideal domain, how do obtain $I^{-1}$?
For example, if $R = \Z[\sqrt{-6}]$ what is
$\left( 
	(5, 1 + 3\sqrt{-5})_R 
\right)^{-1}$?

\begin{theorem}[]
	Every non-zero ideal $I$ in $R$ is a finitely generated free abelian group.
\end{theorem}

That is, 
\[
	I 
	= \Z \alpha_1 + \ldots + \Z \alpha_r
	= (\alpha_1, \ldots, \alpha_r)_R
\]
and by \emph{free} we mean that for every $\alpha \in I$
there is unique integers $m_1, \ldots, m_r \in \Z$
such that
$\alpha = m_1 \alpha_1 + \ldots + m_r \alpha_r$.

