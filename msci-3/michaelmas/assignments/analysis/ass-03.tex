\documentclass[a4paper, answers]{exam}
\usepackage[utf8]{inputenc}

\usepackage{parskip}

\usepackage{amssymb}
\usepackage{amsmath}
\usepackage{amsfonts}
\usepackage{mathtools}

\usepackage{todonotes}

\usepackage{csquotes}

\usepackage{algpseudocode}
\usepackage{algorithm}

\DeclareMathOperator{\AQ}{AQ}
\DeclareMathOperator{\DAQ}{\Delta AQ}
\DeclareMathOperator{\Q}{Q}
\DeclareMathOperator{\HE}{HE}
\DeclareMathOperator*{\argmax}{arg\,max}
\DeclareMathOperator{\Ep}{Ep}

\usepackage{braket}

\addtolength{\oddsidemargin}{-.875in}
\addtolength{\evensidemargin}{-.875in}
\addtolength{\textwidth}{1.75in}
\addtolength{\topmargin}{-.875in}
\addtolength{\textheight}{1.75in}

\usepackage[backend=biber]{biblatex}
\addbibresource{ref.bib}

\title{Natural Computing Part A}
\author{Ben Napier}
\date{March 2022}

\begin{document}

\begin{center}
	\textbf{\textsc{Analysis III, Assignment 3}}
	\\
	\textsc{Ben Napier}
	\vspace{1em}
\end{center}

\begin{questions}
	\question 
	Let $C$ be the cantor set.
	Prove that $C$ is measurable and that is has measure zero.
	\begin{solution}
		Recall that the Cantor set $C = \bigcap_{n=1}^\infty C_n$
		and further recall that $C_n$ has $2^n$ closed intervals each with
		length $\left( 
			\frac13 
		\right)^n$.
		So $C_n$ and $C$ are measurable.
		Since the Lebesgue measure is finitely additive, $m(C_n) = \left( 
			\frac23 
		\right)^n$.
		As $C \subset C_n$, $m(C) \leq m(C_n)$
		and we must have $m(C) = 0$.
	\end{solution}

	\question
	\begin{parts}
		\part 
		Let $f_n$ be a sequence of continuous, real valued functions
		on $[0,1]$ which converges uniformly to $f$.
		Prove that $\lim_{n \to \infty} f_n(x_n) = f(\sfrac12)$ for any sequence
		$\left\{
			x_n
		\right\}$
		which converges to $\sfrac12$.
		\begin{solution}
			Let $\varepsilon > 0$.
			Then there is $N_1 \in \N$ such that for every $n \geq N_1$ and
			$x \in [0,1]$ we have
			\[
				\left\lvert f_n(x) - f(x) \right\rvert 
				< \sfrac{\varepsilon}{2} 
			\]
			by the uniform convergnce of $f_n$ to $f$.
			As $f$ is continuous, there is $\delta > 0$ such that
			\[
				\left\lvert x - \sfrac12 \right\rvert < \delta
				\implies \left\lvert f(x) - f\left( 
					\sfrac12 
				\right) \right\rvert < \sfrac{\varepsilon}{2}.
			\]
			By the convergence of $x_n$ to $\frac12$, there is $N_2 \in \N$
			such that for every $n \geq N_2$ we have
			\[
				\left\lvert x_n - \sfrac12 \right\rvert < \delta.
			\]
			Thus, for every $n \geq \max\left\{
				N_1, N_2
			\right\}$ we have
			\begin{align*}
				\left\lvert f_n(x_n) - f\left( 
					\sfrac12 
				\right) \right\rvert
				&\leq \left\lvert f_n(x_n) - f(x) \right\rvert
				+ \left\lvert f(x) - f\left( 
					\sfrac12 
				\right) \right\rvert \\
				&< \sfrac{\varepsilon}2 + \sfrac{\varepsilon}{2} \\
				&= \varepsilon.
			\end{align*}
			
		\end{solution}

		\part 
		Must the conclusion still hold if the converge is only pointwise?
		Explain.
		\begin{solution}
			Consider the sequence of continuous functions
			\[
				f_n(x) =
				\begin{cases}
					(2x)^n		& x \in \left[0,\sfrac12\right), \\
					(2(1-x))^n	& x \in \left(\sfrac12,1\right], \\
					1			& x = \frac12.
				\end{cases}
			\]
			Now $f_n \to f$ piecewise where
			\[
				f(x) =
				\begin{cases}
					0 & x \in [0,1] \setminus \frac12, \\
					1 & x = \frac12.
				\end{cases}
			\]
			But $f_n$ does not converge uniformly as it does not preserve
			continuity. 
			Let $x_n = \frac12 - \frac1n$, then
			\[
				f_n(x_n)
				= f_n\left( 
					\frac12 - \frac1n 
				\right)
				= \left( 
					1 - \frac2n 
				\right)^n
				\to \frac{1}{e^2}
			\]
			as $n \to \infty$.
			But $f(\sfrac12) = 1 \neq \sfrac{1}{e^2}$.
		\end{solution}
	\end{parts}

	\question
	Prove that the outer measure of the set of irrational numbers in the
	interval $[0,1]$ is equal to $1$.
	\begin{solution}
		As $\Q$ is countable, it has measure zero. 
		Thus it is measurable.
		Therefore
		\[
			1
			= m^\star([0,1])
			= m^\star([0,1] \cap \Q) + m^\star([0,1] \setminus \Q).
		\]
		Now as $[0,1] \cap \Q \subset \Q$,
		it also has measure zero and so
		$m^\star([0,1] \setminus \Q) = 1$.
	\end{solution}
\end{questions}

\end{document}
