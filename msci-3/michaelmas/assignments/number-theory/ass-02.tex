\documentclass[a4paper, answers]{exam}
\usepackage[utf8]{inputenc}

\usepackage{parskip}

\usepackage{amssymb}
\usepackage{amsmath}
\usepackage{amsfonts}
\usepackage{mathtools}

\usepackage{todonotes}

\usepackage{csquotes}

\usepackage{algpseudocode}
\usepackage{algorithm}

\DeclareMathOperator{\AQ}{AQ}
\DeclareMathOperator{\DAQ}{\Delta AQ}
\DeclareMathOperator{\Q}{Q}
\DeclareMathOperator{\HE}{HE}
\DeclareMathOperator*{\argmax}{arg\,max}
\DeclareMathOperator{\Ep}{Ep}

\usepackage{braket}

\addtolength{\oddsidemargin}{-.875in}
\addtolength{\evensidemargin}{-.875in}
\addtolength{\textwidth}{1.75in}
\addtolength{\topmargin}{-.875in}
\addtolength{\textheight}{1.75in}

\usepackage[backend=biber]{biblatex}
\addbibresource{ref.bib}

\title{Natural Computing Part A}
\author{Ben Napier}
\date{March 2022}

\begin{document}
\begin{center}
	\textbf{\textsc{Number Thoery III, Assignment 2}}
	\\
	\textsc{Ben Napier}
	\vspace{1em}
\end{center}
\begin{questions}
	\question 
	Let $\theta \in \C$ with $\theta^3 - 7 = 0$,
	and set $K = \Q(\theta)$.
	Calculate
	$\operatorname{Tr}_{K/\Q}(a + b\theta + c\theta^2)$
	and
	$\operatorname{N}_{K/\Q}(a + b\theta + c\theta^2)$,
	where $a,b,c \in \Q$.
	Your answer will depend on $a$, $b$, and $c$.
	\begin{solution}
		As $\theta$ is algebraic: $\Q(\theta) = \Q[\theta]$.
		By Eisenstein's criterion with $p = 7$,
		we see that $\theta^3 - 7$ is irreducible and
		as it is also monic, it is the minimal polynomial for $\theta$.
		Hence $[K : \Q] = 3$, and so we may fix the basis
		$ \left\{ 1, \theta, \theta^3 \right\}$.
		Now for $\alpha = a + b\theta + c\theta^2$
		we see
		\begin{align*}
			1 \cdot \alpha
				&= a + b\theta + c\theta^2 \\
			\theta \cdot \alpha
				&= a\theta + b\theta^2 + c\theta^3
				= 7c + a\theta + b \theta^2 \\
			\theta^2 \cdot \alpha
				&= a\theta^2 + b\theta^3 + c\theta^4
				= 7b + 7c\theta + a\theta^2,
		\end{align*}
		and so we get the following matrix that corresponds to our linear
		map:
		\[
			A =
			\begin{pmatrix}
				a & 7c & 7b \\
				b & a & 7c \\
				c & b & a \\ 
			\end{pmatrix}.
		\]
		Hence
		$
			\operatorname{Tr}_{K/\Q}(\alpha) = 3a
		$
		and
		\begin{align*}
			\operatorname{det}_{K/\Q}(\alpha)
				&= (a^3 - 7abc) - (7abc - 49c^3) + (7b^3 - 7abc) \\
				&= a^3 + 7b^3 + 49c^3 - 21abc.
		\end{align*}
		
	\end{solution}

	\question
	Let $f(x) \in \Z[x]$ be a monic polynomial and
	$\alpha \in \overline{\Q}$.
	Show that $f(\alpha) \in \overline{\Z}$
	if and only if $\alpha \in \overline{\Z}$.
	\begin{solution}
		Let $f(x) = a_0 + a_1x + \ldots + a_{n-1} x^{n-1} + x^n$
		and assume $f(\alpha) \in \overline\Z$.
		Then there is is a monic $g(x) \in \Z[x]$ such that
		\[
			g(f(\alpha)) = 0.
		\]
		Let $g(x) = b_0 + b_1x + \ldots + b_{m-1} x^{m-1} + x^m$.
		Then
		\begin{align*}
			g(f(\alpha))
			&= g(a_0 + \ldots a_{n-1}\alpha^{n-1} + \alpha^n) \\
			&= b_0 + b_1(a_0 + \ldots a_{n-1}\alpha^{n-1} + \alpha^n) + \ldots
				+ (a_0 + \ldots a_{n-1}\alpha^{n-1} + \alpha^n)^m \\
			&= (b_0 + b_1a_0 + \ldots) + \ldots + \alpha^{n+m} \in \Z[x],
		\end{align*}
		and so $\alpha \in \overline\Z$.
		
		Now assume $\alpha \in \overline\Z$.
		Hence there is a monic $g(x) \in \Z[x]$ such that $g(\alpha) = 0$.
		Let $g(x) = b_0 + \ldots + x^m$.
		Observe that $f(\alpha) = a_0 + \ldots + \alpha^n$.
		As $\alpha \in \overline\Z$ and $\overline\Z$ is a ring,
		then clearly $f(\alpha) \in \overline\Z$
		(since $f(x) \in \Z[x]$ and $\Z \subset \overline\Z$).
	\end{solution}

	\question
	Let $K$ be a number field and write $\mathcal O_K$ for it
	ring of integers.
	Let $I$ be a non-zero ideal in $\mathcal O_K$.
	Show that $I \cap \Z$ is a non-zero ideal of $\Z$.
	\begin{solution}
		Recall that
		\[
			K = \left\{ \frac{\alpha}{m}:
				\alpha \in \mathcal O_K, m \in \Z \setminus \left\{ 0 \right\}
			\right\}.
		\]
		Let $x \in I \cap \Z \subset K$.
		Then there is $\alpha \in \mathcal O_K$ and 
		$m \in \Z \setminus \left\{ 0 \right\}$
		such that $x = \frac{\alpha}{m}$.
		As $I$ is an ideal in $\mathcal O_K$ we have
		$\frac{n\alpha}{m} \in I$ for all $n \in \mathcal O_K$.
	\end{solution}
\end{questions}
\end{document}
