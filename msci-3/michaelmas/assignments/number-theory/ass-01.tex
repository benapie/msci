\documentclass[a4paper, answers]{exam}
\documentclass[a4paper, answers]{exam}

% tikzcd.yichuanshen.de/ tikcd diagrams
%======================%
%   Standard packages  %
%======================%
\usepackage[utf8]{inputenc}
\usepackage[T1]{fontenc}
\usepackage{lmodern}
\usepackage[UKenglish]{babel}
\usepackage{enumitem}
\usepackage{tasks}
\usepackage{graphicx}
\setlist[enumerate,1]{
  label={(\roman*)}
}
\usepackage{parskip}
\usepackage{hyperref}

%======================%
%        Maths         %
%======================%
\usepackage{amsfonts, mathtools, amsthm, amssymb}
\usepackage{xfrac}
\usepackage{bm}
\newcommand\N{\ensuremath{\mathbb{N}}}
\newcommand\R{\ensuremath{\mathbb{R}}}
\newcommand\Z{\ensuremath{\mathbb{Z}}}
\newcommand\Q{\ensuremath{\mathbb{Q}}}
\newcommand\C{\ensuremath{\mathbb{C}}}
\newcommand\F{\ensuremath{\mathbb{F}}}
\newcommand{\abs}[1]{\ensuremath{\left\lvert #1 \right\rvert}}
\newcommand\given[1][]{\:#1\vert\:}
\newcommand\restr[2]{{% we make the whole thing an ordinary symbol
  \left.\kern-\nulldelimiterspace % automatically resize the bar with \right
  #1 % the function
  \vphantom{\big|} % pretend it's a little taller at normal size
  \right|_{#2} % this is the delimiter
}}

\newcommand\corestr[2]{{% we make the whole thing an ordinary symbol
  \left.\kern-\nulldelimiterspace % automatically resize the bar with \right
  #1 % the function
  \vphantom{\big|} % pretend it's a little taller at normal size
  \right|^{#2} % this is the delimiter
}}
\usepackage{siunitx}

\usepackage{afterpage}

\usepackage{tikz-cd}
\usepackage{adjustbox}
\DeclareMathOperator{\norm}{N}
\DeclareMathOperator{\trace}{Tr}
\DeclareMathOperator*{\argmax}{arg\,max}
\DeclareMathOperator*{\argmin}{arg\,min}
\DeclareMathOperator*{\esssup}{ess\,sup}
\DeclareMathOperator*{\SL}{SL}
\DeclareMathOperator*{\GL}{GL}
\DeclareMathOperator*{\SO}{SO}
\DeclareMathOperator*{\aut}{Aut}
\DeclareMathOperator*{\id}{id}
\DeclareMathOperator*{\coker}{coker}
\DeclareMathOperator*{\im}{im}



%======================%
%       CompSci        %
%======================%
\usepackage{forest}
\usepackage{textgreek}
\usepackage{algpseudocode}

%======================%
%    Pretty tables     %
%======================%
\usepackage{booktabs}
\usepackage{caption}

\begin{document}
\begin{center}
	\textbf{\textsc{Number Thoery III, Assignment 1}}
	\\
	\textsc{Ben Napier}
	\vspace{1em}
\end{center}
\begin{questions}
	\question
		Show that the polynomial $f(x) = x^4 + 1$ is irreducible in $\Q[x]$
		(Hint: Eisenstein's criterion).
		\begin{solution}
			We cannot apply Eisenstein's criterion to $f$, but if we consider
			\[
				f(x + 1) = x^4 + 4x^3 + 6x^2 + 4x + 2
			\]
			and take $p = 2$, we see that
			\[
				p \mid 1, \qquad
				p \mid 4,6,4, \;\text{and}\; \qquad
				p^2 \nmid 2;
			\]
			hence, by Eisenstein's criterion, $f(x + 1)$ is irreducible.
			Now, for a contradiction, assume that $f(x)$ is reducible.
			So $f(x) = g(x) \cdot h(x)$ for non-constant $g(x)$ and $h(x)$.
			Therefore $f(x + 1) = g(x + 1) \cdot h(x + 1)$; a contradiction.
			Therefore, $f(x)$ is irreducible.
		\end{solution}

	\question
		Let $\sqrt[6]{2}$ denote the (positive) real $6^\text{th}$ root of
		$2$.
		\begin{parts}
			\part
				Show that $[\Q(\sqrt[6]{2}): \Q] = 6$.
				\begin{solution}
					Let $f(x) = x^6 - 2$.
					By Eisenstein's criterion (picking $p = 2$) we see that
					$f$ is irreducible.
					$f$ is also monic and $f(\sqrt[6]{2}) = 0$; therefore,
					$f$ is the minimal polynomial of $\sqrt[6]{2}$.
					We see that $\operatorname{deg}{f} = 6$ and so 
					\[
						[\Q(\sqrt[6]{2}): \Q] = 6.
					\]
				\end{solution}

			\part
				Show that $x^3 - \sqrt 2$ is irreducible over $\Q(\sqrt 2)$.
				\begin{solution}
					First, observe that $x^2 - 2$ is the minimal polynomial
					of $\sqrt 2$ over $\Q$; hence,
					\[
						[\Q(\sqrt 2) : \Q] = 2.
					\]
					Now $\sqrt 2 = (\sqrt[6]{2})^3 \in \Q(\sqrt[6]{2})$,
					so $\Q \subset \Q(\sqrt 2) \subset \Q(\sqrt[6]{2})$.
					By the tower theorem we have
					\[
						[\Q(\sqrt[6]{2}): \Q]
						= [\Q(\sqrt[6]{2}): \Q(\sqrt 2)]
							\cdot [\Q(\sqrt 2): \Q]
					\]
					and so, by (a), we have 
					$[\Q(\sqrt[6]{2}): \Q(\sqrt 2)] = 3$.
					Now we consider $g(x) = x^3 - \sqrt 2$.
					As $g(\sqrt[6]{2}) = 0$, $g$ is monic, and
					$\operatorname{deg}{g} = [\Q(\sqrt[6]{2}): \Q(\sqrt 2)]$
					then $g$ is the minimal polynomial for $\sqrt[6]{2}$ in
					$\Q(\sqrt 2)$.
					Thus $g$ is irreducible.
				\end{solution}
		\end{parts}

	\question
		Let $\alpha \in \C$ be algebraic over $\Q$ and assume that its
		minimal polynomial over $\Q$ is of odd degree.
		Show that $\Q(\alpha) = \Q(\alpha^2)$.
		\begin{solution}
			Let $d$ be the degree of the minimal polynomial for $\alpha$ over
			$\Q$, then
			\[
				[\Q(\alpha): \Q] = d.
			\]
			Now suppose, for a contradiction, that 
			$\Q(\alpha) \neq \Q(\alpha^2)$.
			Observe that $\alpha \not\in \Q(\alpha^2)$ and so the minimum
			polynomial of $\alpha$ over $\Q(\alpha^2)$ is 
			$x^2 - \alpha^2 \in \Q(\alpha^2)[x]$.
			Hence $[\Q(\alpha): \Q(\alpha^2)] = 2$.
			Clearly $\Q \subset \Q(\alpha^2) \subset \Q(\alpha)$, so by the
			tower theorem
			\begin{align*}
				[\Q(\alpha): \Q]
					&= [\Q(\alpha): \Q(\alpha^2)] \cdot
					[\Q(\alpha^2): \Q] \\
					&= 2 \cdot [\Q(\alpha^2): \Q];
			\end{align*}
			but $d = [\Q(\alpha): \Q]$ is odd; a contradiction.
			Therefore $\Q(\alpha) = \Q(\alpha^2)$.
		\end{solution}
\end{questions}
\end{document}
