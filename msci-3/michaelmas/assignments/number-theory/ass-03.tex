\documentclass[a4paper, answers]{exam}
\documentclass[a4paper, answers]{exam}

% tikzcd.yichuanshen.de/ tikcd diagrams
%======================%
%   Standard packages  %
%======================%
\usepackage[utf8]{inputenc}
\usepackage[T1]{fontenc}
\usepackage{lmodern}
\usepackage[UKenglish]{babel}
\usepackage{enumitem}
\usepackage{tasks}
\usepackage{graphicx}
\setlist[enumerate,1]{
  label={(\roman*)}
}
\usepackage{parskip}
\usepackage{hyperref}

%======================%
%        Maths         %
%======================%
\usepackage{amsfonts, mathtools, amsthm, amssymb}
\usepackage{xfrac}
\usepackage{bm}
\newcommand\N{\ensuremath{\mathbb{N}}}
\newcommand\R{\ensuremath{\mathbb{R}}}
\newcommand\Z{\ensuremath{\mathbb{Z}}}
\newcommand\Q{\ensuremath{\mathbb{Q}}}
\newcommand\C{\ensuremath{\mathbb{C}}}
\newcommand\F{\ensuremath{\mathbb{F}}}
\newcommand{\abs}[1]{\ensuremath{\left\lvert #1 \right\rvert}}
\newcommand\given[1][]{\:#1\vert\:}
\newcommand\restr[2]{{% we make the whole thing an ordinary symbol
  \left.\kern-\nulldelimiterspace % automatically resize the bar with \right
  #1 % the function
  \vphantom{\big|} % pretend it's a little taller at normal size
  \right|_{#2} % this is the delimiter
}}

\newcommand\corestr[2]{{% we make the whole thing an ordinary symbol
  \left.\kern-\nulldelimiterspace % automatically resize the bar with \right
  #1 % the function
  \vphantom{\big|} % pretend it's a little taller at normal size
  \right|^{#2} % this is the delimiter
}}
\usepackage{siunitx}

\usepackage{afterpage}

\usepackage{tikz-cd}
\usepackage{adjustbox}
\DeclareMathOperator{\norm}{N}
\DeclareMathOperator{\trace}{Tr}
\DeclareMathOperator*{\argmax}{arg\,max}
\DeclareMathOperator*{\argmin}{arg\,min}
\DeclareMathOperator*{\esssup}{ess\,sup}
\DeclareMathOperator*{\SL}{SL}
\DeclareMathOperator*{\GL}{GL}
\DeclareMathOperator*{\SO}{SO}
\DeclareMathOperator*{\aut}{Aut}
\DeclareMathOperator*{\id}{id}
\DeclareMathOperator*{\coker}{coker}
\DeclareMathOperator*{\im}{im}



%======================%
%       CompSci        %
%======================%
\usepackage{forest}
\usepackage{textgreek}
\usepackage{algpseudocode}

%======================%
%    Pretty tables     %
%======================%
\usepackage{booktabs}
\usepackage{caption}

\begin{document}
\begin{center}
	\textbf{\textsc{Number Theory III, Assignment 3}}
	\\
	\textsc{Ben Napier}
	\vspace{1em}
\end{center}
\begin{questions}
	\question 
	Let $p$ be an odd prime and set $K = \Q(\zeta)$
	with $\zeta = e^{\frac{2\pi i}p}$.
	Show that
	\[
		\frac{1 - \zeta^i}{1 - \zeta^j}
		\in \mathcal O^\times_K,
	\]
	for all $0 \neq i, j \in \Z$
	with $\gcd(i,p) = \gcd(j,p) = 1$.
	\begin{solution}
		Observe that
		\[
			\frac{1 - \zeta^i}{1 - \zeta} 
			= 1 + \zeta + \ldots + \zeta^{i-1}
			\in \Z[\zeta]
		\]
		and thus 
		$\frac{1 - \zeta^i}{1 - \zeta} \in \mathcal O_K$.
		Now let $i'$ be the modular inverse of $i$ (modulo $p$),
		which exists as $i$ and $p$ are coprime.
		Let $\varphi = \zeta^i$. 
		Then
		\begin{align*}
			\frac{1 - \zeta}{1 - \zeta^i}
			&= \frac{1 - \left( 
				\zeta^i
			\right)^{i'}}{1 - \left( 
				\zeta^i
			\right)} \\
			&= \frac{1 - \varphi^{i'}}{1 - \varphi} \\
			&= 1 + \varphi + \ldots + \varphi^{i' - 1} \\
			&\in \Z[\varphi] \subset \Z[\zeta].
		\end{align*}
		Thus $\left( 
			\frac{1 - \zeta^i}{1 - \zeta} 
		\right)^{-1} = \frac{1 - \zeta}{1 - \zeta^i}
		\in \mathcal O_K$.
		Therefore,
		$\frac{1 - \zeta^i}{1 - \zeta} \in \mathcal O_K^\times$.
		Similarly,
		$\frac{1 - \zeta^j}{1 - \zeta} \in \mathcal O_K^\times$.
		As $\mathcal O_K$ is a ring, $\mathcal O_K^\times$ is closed
		under multiplication.
		Hence,
		$
			\frac{1 - \zeta^i}{1 - \zeta^j} \in \mathcal O_K^\times
		$.
	\end{solution}

	\question
	Let $R = \Z[\sqrt{-13}]$,
	and set $\alpha = 1 + \sqrt{-13} \in R$.
	\begin{parts}
		\part Is $\alpha$ irreducible in $R$?
		\begin{solution}
			Assume that $\alpha$ is reducible.
			Then
			\[
				1 + \sqrt{-13} = (a + b\sqrt{-13})(c + d\sqrt{-13})
			\]
			for $a + b\sqrt{-13}, c + d\sqrt{-13} \in R \setminus R^\times$.
			Now recall that $N(p + q\sqrt{-13}) = p^2 + 13q^2$.
			Taking norms of both sides:
			\[
				(a^2 + 13b^2)(c^2 + 13d^2) = 14.
			\]
			The only possibilities are: $a = b = c = 1$ and $d = 0$;
			or $a = c = d = 1$ and $b = 0$.
			But in both situations either $a + b\sqrt{-13} \in R^\times$
			or $c + d\sqrt{-13} \in R^\times$; a contradiction.
			Hence $\alpha$ is irreducible.
		\end{solution}

		\part Is $\alpha$ a prime element of $R$?
		\begin{solution}
			Assume $1 + \sqrt{-13}$ is prime.
			Observe that
			\[
				(1 + \sqrt{-13})(1 - \sqrt{-13}) = 14 \mid 14 = 7 \cdot 2.
			\]
			Then either $1 + \sqrt{-13} \mid 7$
			or $1 + \sqrt{-13} \mid 2$.
			If $1 + \sqrt{-13} \mid 7$, then there is
			$a, b \in \Z$ such that
			$(1 + \sqrt{-13})(a + b\sqrt{13}) = 7$.
			Taking norms of both sides:
			$14(a^2 + 13b^2) = 49$, and so
			$2(a^2 + 13b^2) = 7$, which is impossible.
			Therefore $1 + \sqrt{-13} \nmid 7$.
			Thus we must have $1 + \sqrt{-13} \mid 2$.
			So there is $c, d \in \Z$ such that
			$(1 + \sqrt{-13})(c + d\sqrt{-13}) = 2$.
			Again, taking norms we get
			$7(a^2 + 13b^2) = 2$, which again is impossible.
			So $1 + \sqrt{-13} \nmid 2$.
			Hence $1 + \sqrt{-13}$ is not prime.
		\end{solution}

		\part Is $R$ a U.F.D.?
		\begin{solution}
			In unique factorisation domains, our notions of 
			irreducibility and primality coincide.
			As $\alpha$ is an element of $R$ which is irreducible but not prime,
			$R$ must not be a unique factorisation domain.
		\end{solution}
	\end{parts}

	\question
	Let $R = \Z[\sqrt{-5}]$ and $I = (1+\sqrt{-5}, 2)_R$.
	\begin{parts}
		\part
		Show that $I^2$ is a principal ideal.
		\begin{solution}
			Observe
			\begin{align*}
				I^2
				&= (1 + \sqrt{-5}, 2)_R^2 \\
				&= (1 + \sqrt{-5}, 2)_R \cdot (1 + \sqrt{-5}, 2)_R \\
				&= ((1 + \sqrt{-5})^2, 2(1 + \sqrt{-5}), 4)_R \\
				&= (-4 + 2\sqrt{-5}, 2 + 2\sqrt{-5}, 4)_R.
			\end{align*}
			Further note that
			\begin{align*}
				-4 + 2\sqrt{-5}
				&= 2(-2 + \sqrt{-5}) \in (2)_R, \\
				2 + 2\sqrt{-5}
				&= 2(1 + \sqrt{-5}) \in (2)_R, \\
				4
				&= 2(2) \in (2)_R.
			\end{align*}
			Thus $I^2 = (2)_R$ and so $I^2$ is principal.
		\end{solution}

		\part
		Show that $I$ is a maximal ideal.
		\begin{solution}
			First we observe that
			\[
				R/(2)_R = \left\{
					(2)_R,
					1 + (2)_R,
					\sqrt{-5} + (2)_R,
					1 + \sqrt{-5} + (2)_R
				\right\}
			\]
			and so $\left\lvert R/(2)_R \right\rvert = 4$.
			$R/I$ is an additive subgroup of $R/(2)_R$,
			so $\left\lvert R/I \right\rvert \mid 4$.
			Thus $\left\lvert R/I \right\rvert \in \left\{
				1,2,4
			\right\}$.
			If $\left\lvert R/I \right\rvert = 1$, then $I = R$.
			But then $I^2 = R^2 = R$, but we showed that
			$I^2 = (2)_R \neq R$ so this cannot be the case.
			If $\left\lvert R/I \right\rvert = 4$ then
			$I = (2)_R$ which is impossible since
			$1 + \sqrt{-5} \in I$ and $1 + \sqrt{-5} \not\in (2)_R$.
			Thus we must have $\left\lvert R/I \right\rvert = 2$.
			But there is only one ring (up to isomorphisms) with
			2 elements, and it is a field.
			So $R/I$ is a field, and so $I$ must be maximal.
		\end{solution}
	\end{parts}
\end{questions}
\end{document}
