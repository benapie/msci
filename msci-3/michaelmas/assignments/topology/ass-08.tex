\documentclass[a4paper, answers]{exam}
\documentclass[a4paper, answers]{exam}

% tikzcd.yichuanshen.de/ tikcd diagrams
%======================%
%   Standard packages  %
%======================%
\usepackage[utf8]{inputenc}
\usepackage[T1]{fontenc}
\usepackage{lmodern}
\usepackage[UKenglish]{babel}
\usepackage{enumitem}
\usepackage{tasks}
\usepackage{graphicx}
\setlist[enumerate,1]{
  label={(\roman*)}
}
\usepackage{parskip}
\usepackage{hyperref}

%======================%
%        Maths         %
%======================%
\usepackage{amsfonts, mathtools, amsthm, amssymb}
\usepackage{xfrac}
\usepackage{bm}
\newcommand\N{\ensuremath{\mathbb{N}}}
\newcommand\R{\ensuremath{\mathbb{R}}}
\newcommand\Z{\ensuremath{\mathbb{Z}}}
\newcommand\Q{\ensuremath{\mathbb{Q}}}
\newcommand\C{\ensuremath{\mathbb{C}}}
\newcommand\F{\ensuremath{\mathbb{F}}}
\newcommand{\abs}[1]{\ensuremath{\left\lvert #1 \right\rvert}}
\newcommand\given[1][]{\:#1\vert\:}
\newcommand\restr[2]{{% we make the whole thing an ordinary symbol
  \left.\kern-\nulldelimiterspace % automatically resize the bar with \right
  #1 % the function
  \vphantom{\big|} % pretend it's a little taller at normal size
  \right|_{#2} % this is the delimiter
}}

\newcommand\corestr[2]{{% we make the whole thing an ordinary symbol
  \left.\kern-\nulldelimiterspace % automatically resize the bar with \right
  #1 % the function
  \vphantom{\big|} % pretend it's a little taller at normal size
  \right|^{#2} % this is the delimiter
}}
\usepackage{siunitx}

\usepackage{afterpage}

\usepackage{tikz-cd}
\usepackage{adjustbox}
\DeclareMathOperator{\norm}{N}
\DeclareMathOperator{\trace}{Tr}
\DeclareMathOperator*{\argmax}{arg\,max}
\DeclareMathOperator*{\argmin}{arg\,min}
\DeclareMathOperator*{\esssup}{ess\,sup}
\DeclareMathOperator*{\SL}{SL}
\DeclareMathOperator*{\GL}{GL}
\DeclareMathOperator*{\SO}{SO}
\DeclareMathOperator*{\aut}{Aut}
\DeclareMathOperator*{\id}{id}
\DeclareMathOperator*{\coker}{coker}
\DeclareMathOperator*{\im}{im}



%======================%
%       CompSci        %
%======================%
\usepackage{forest}
\usepackage{textgreek}
\usepackage{algpseudocode}

%======================%
%    Pretty tables     %
%======================%
\usepackage{booktabs}
\usepackage{caption}

\begin{document}

\begin{center}
	\textbf{\textsc{Topology III, Assignment 8}}
	\\
	\textsc{Ben Napier}
	\vspace{1em}
\end{center}

\begin{questions}
	\question Suppose $X$ is a space and let $\sim$ be any equivalence
	relation on $X$.
	\begin{parts}
		\part Show that if $X$ is compact then $X/{\sim}$ is compact.
		\begin{solution}
			Let
			$
				\left\{
					U_i
				\right\}_{i=1}^\infty
			$
			be an open cover of $X/{\sim}$.
			Then
			$
				\left\{
					\pi^{-1}(U_i)
				\right\}_{i=1}^\infty
			$
			is an open cover of $X$.
			Thus there is $N \in \N$
			such that
			$
				\left\{
					\pi^{-1}(U_i)
				\right\}_{i=1}^N
			$
			is an open cover of $X$.
			Thus
			$
				\left\{
					U_i
				\right\}_{i=1}^N
			$
			is a finite cover of $X/{\sim}$.
			Thus $X/{\sim}$ is compact.
		\end{solution}

		\part Show that if $X$ is connected then $X/{\sim}$ is connected.
		\begin{solution}
			Let $U \subset X/{\sim}$ be clopen and
			$
				U \not\in
				\left\{
					\varnothing, X/{\sim}
				\right\}
			$.
			Then $\pi^{-1}(U) \subset X$ is also clopen.
			But as $X$ is conencted,
			$\pi^{-1}(U) \in \{\varnothing, X\}$.
			If $\pi^{-1}(U) = \varnothing$,
			$U = \varnothing$ as $U$ is non-empty.
			If $\pi^{-1}(U) = X$, then $U = X/{\sim}$;
			a contradiction.
			Thus $X/{\sim}$ is connected.
		\end{solution}
	\end{parts}

	\question This question uses the result of question $1$ from Problem Set 7.
	Let $X$ be a compact metric space, and let $f: X \to X$ be continuous.
	We are going to show that there is a non-empty subset $A \subset X$ such
	that $f(A) = A$.
	\begin{parts}
		\part Show that $A = \bigcap_{n=1}^\infty f^n(X) \neq \varnothing$,
		where we write $f^n$ for the composition of $f$ with itself $n$ times.
		(You may need to recall the fact that metric spaces are Hausdorff).
		\begin{solution}
			$X$ is compact, so $f(X)$ is also compact.
			In fact, as $X$ is also Hausdorff, $f(X)$ is closed.
			Let $n \in \N$.
			For every $x \in f^{n+1}(X)$, there must exist
			$x' \in f^n(X) \subset X$ such that $f^n(f(x')) = x$.
			Thus $x \in f^n(X)$ and so
			$f^{n+1}(X) \subset f^n(X)$ for all $n \in \N$.
			Clearly $f^n(X)$ is non-empty for every $n \in \N$,
			otherwise we would have a non-empty domain and an
			empty codomain.
			Hence $\left\{
				f^n(X)
			\right\}_{n \in \N}$
			is a collection of closed, nested, non-empty subsets of a
			compact space $X$; hence,
			\[
				\bigcap_{n=1}^\infty f^n(X) \neq \varnothing.
			\]
		\end{solution}

		\part Show that $f(A) \subset A$.
		\begin{solution}
			Let $a \in f(A)$.
			So there is $a' \in A$ such that $a = f(a')$.
			Then for each $n \in \N$, $a' \in f^n(X)$,
			so $a = f(a') \in f^{n+1}(X)$.
			So $a \in f^n(X)$ for each $n \in\{2, 3, \ldots\}$.
			For $n=1$, we observe that $a = f(a')$ and
			$a' \in A \subset X$, so $a \in f(X)$.
			Hence $a \in A$.
			Thus $f(A) \subset A$.
		\end{solution}

		\part Suppose $a \in A$.
		Show that there exists a sequence of points
		$b_n \in f^n(X)$
		such that $f(b_n) = a$.
		\begin{solution}
			Fix $n \in \N$.
			Then, as $a \in A$,
			$a \in f^{n+1}(X)$.
			So there is some $b_n \in f^n(X)$ such that
			$f(b_n) = a$.
			So
			$
				\left\{
					b_n
				\right\}
			$
			is our sequence.
		\end{solution}

		\part Recalling that infinite subsets of compact spaces always have
		limit points, show that there is a subsequence of $b_n$, say $b_{n_i}$,
		so that $b_{n_i} \to b$ for some $b \in X$.

		Observe that therefore we have $f(b) = a$.
		\begin{solution}
			If one point in $b_n$ occurs an infinite number of times,
			then clearly this constant sequence converges.
			Now suppose that every point in the sequence occurs finitely many
			times.
			Then this is an infinite subset of $X$, so there must be a limit
			point.
			That is, there is a convergent subsequence
			$
				\left\{
					b_{n_i}
				\right\}
			$
			that converges to some limit $b \in X$.
		\end{solution}

		\part Show that $b \in A$.
		\begin{solution}
			Let $\{b_n\}$ be some sequence as defined
			in (c) and let
			$
				\left\{
					b_{n_k}
				\right\}
			$
			be a convergent subsequence with limit $b \in X$.
			We then define
			\[
				j = \min\left\{
					i \in \N: n_i \geq n
				\right\}.
			\]
			Then
			$(b_{n_j}, b_{n_{j+1}}, \ldots)$
			is a convergent sequence contained within
			$f^n(X)$ with limit $b$.
			So $b$ is a limit point of $f^n(X)$.
			As $f^n(X)$ is closed, it contains all of its limit points.
			So $b \in f^n(X)$.
			As $n \in \N$ was chosen arbitrarily,
			$b \in f^n(X)$ for each $n \in \N$.
			That is, $b \in A$.
		\end{solution}

		\part Hence conclude that $f(A) = A$.
		\begin{solution}
			Let $a \in A$.
			Then there is a sequence
			$
				\left\{
					b_n
				\right\}
			$
			with convergent subsequence
			$
				\left\{
					b_{n_k}
				\right\}
			$
			that converges to some $b \in A$ with
			$a = f(b) \in f(A)$.
			Hence $A \subset f(A)$.
			By (b), $f(A) \subset A$.
			Therefore, $f(A) = A$.
		\end{solution}
	\end{parts}
\end{questions}

\end{document}
