\documentclass[a4paper, answers]{exam}
\documentclass[a4paper, answers]{exam}

% tikzcd.yichuanshen.de/ tikcd diagrams
%======================%
%   Standard packages  %
%======================%
\usepackage[utf8]{inputenc}
\usepackage[T1]{fontenc}
\usepackage{lmodern}
\usepackage[UKenglish]{babel}
\usepackage{enumitem}
\usepackage{tasks}
\usepackage{graphicx}
\setlist[enumerate,1]{
  label={(\roman*)}
}
\usepackage{parskip}
\usepackage{hyperref}

%======================%
%        Maths         %
%======================%
\usepackage{amsfonts, mathtools, amsthm, amssymb}
\usepackage{xfrac}
\usepackage{bm}
\newcommand\N{\ensuremath{\mathbb{N}}}
\newcommand\R{\ensuremath{\mathbb{R}}}
\newcommand\Z{\ensuremath{\mathbb{Z}}}
\newcommand\Q{\ensuremath{\mathbb{Q}}}
\newcommand\C{\ensuremath{\mathbb{C}}}
\newcommand\F{\ensuremath{\mathbb{F}}}
\newcommand{\abs}[1]{\ensuremath{\left\lvert #1 \right\rvert}}
\newcommand\given[1][]{\:#1\vert\:}
\newcommand\restr[2]{{% we make the whole thing an ordinary symbol
  \left.\kern-\nulldelimiterspace % automatically resize the bar with \right
  #1 % the function
  \vphantom{\big|} % pretend it's a little taller at normal size
  \right|_{#2} % this is the delimiter
}}

\newcommand\corestr[2]{{% we make the whole thing an ordinary symbol
  \left.\kern-\nulldelimiterspace % automatically resize the bar with \right
  #1 % the function
  \vphantom{\big|} % pretend it's a little taller at normal size
  \right|^{#2} % this is the delimiter
}}
\usepackage{siunitx}

\usepackage{afterpage}

\usepackage{tikz-cd}
\usepackage{adjustbox}
\DeclareMathOperator{\norm}{N}
\DeclareMathOperator{\trace}{Tr}
\DeclareMathOperator*{\argmax}{arg\,max}
\DeclareMathOperator*{\argmin}{arg\,min}
\DeclareMathOperator*{\esssup}{ess\,sup}
\DeclareMathOperator*{\SL}{SL}
\DeclareMathOperator*{\GL}{GL}
\DeclareMathOperator*{\SO}{SO}
\DeclareMathOperator*{\aut}{Aut}
\DeclareMathOperator*{\id}{id}
\DeclareMathOperator*{\coker}{coker}
\DeclareMathOperator*{\im}{im}



%======================%
%       CompSci        %
%======================%
\usepackage{forest}
\usepackage{textgreek}
\usepackage{algpseudocode}

%======================%
%    Pretty tables     %
%======================%
\usepackage{booktabs}
\usepackage{caption}

\begin{document}
\begin{center}
	\textbf{\textsc{Topology III, Assignment 2}}
	\\
	\textsc{Ben Napier}
	\vspace{1em}
\end{center}
\begin{questions}
	\question
		Decide which of the following subsets of $\mathcal P(X)$,
		where $X = \left\{ 1,2,4 \right\}$, defines a topology.
		\begin{enumerate}
			\item $
				\tau_1 = \{
					\varnothing,
					\left\{ 1 \right\},
					\left\{ 1, 2 \right\},
					\left\{ 1, 2, 4 \right\}
				\}
			$;

			\item $
				\tau_2 = \{
					\varnothing,
					\left\{ 1 \right\},
					\left\{ 2 \right\},
					\left\{ 1, 2 \right\},
					\left\{ 1, 2, 4 \right\}
				\}
			$;

			\item $
				\tau_3 = \{
					\varnothing,
					\left\{ 1 \right\},
					\left\{ 4 \right\},
					\left\{ 1, 2, 4 \right\}
				\}
			$;

			\item $
				\tau_4 = \{
					\varnothing,
					\left\{ 2 \right\},
					\left\{ 4 \right\},
					\left\{ 2, 4 \right\},
					\left\{ 1, 2, 4 \right\}
				\}
			$;

			\item $
				\tau_5 = \{
					\varnothing,
					\left\{ 2, 4 \right\},
					\left\{ 1, 2, 4 \right\}
				\}
			$;

			\item $
				\tau_6 = \{
					\varnothing,
					\left\{ 2 \right\},
					\left\{ 1, 2, 4 \right\}
				\}
			$; and

			\item $
				\tau_7 = \{
					\varnothing,
					\left\{ 1 \right\},
					\left\{ 2 \right\},
					\left\{ 2, 4 \right\},
					\left\{ 1, 2, 4 \right\}
				\}
			$.
		\end{enumerate}
	\begin{solution}
		All collections satisfy $X, \varnothing \in \tau$.
		As $\left\{ 1 \right\}, \left\{ 4 \right\} \in \tau_3$
		but $\left\{ 1,4 \right\} \not\in \tau_3$, $\tau_3$ is not a
		topology.
		Similarly, as $\left\{ 1 \right\}, \left\{ 2 \right\} \in \tau_7$
		but $\left\{ 1, 2 \right\} \not\in \tau_7$, $\tau_7$ is also
		not a topology.
		All other collections are closed under union and intersection,
		hence they define topologies.
	\end{solution}
		
	\question
		Let $X$ be a finite set, and $\tau$ a topology on $X$.
		Assume that $(X, \tau)$ is a Hausdorff space.
		Show that $\tau$ is the discrete topology
		(in other words,
		show that every subset of $X$ is open).
	\begin{solution}
		We claim that all singleton sets of $X$ are open.
		Let $x, y_1, \ldots, y_n \in X$ be all the points in $X$.
		Then as $X$ is Hausdorff there exists the open sets
		$U_1, \ldots, U_n, V_1, \ldots, V_n$ such that for all
		$i \in \left\{ 1, \ldots, n \right\}$ we have
		\begin{enumerate}
			\item $x \in U_i$;
			\item $y_i \in V_i$; and
			\item $U_i \cap V_i = \varnothing$.
		\end{enumerate}
		Now let $U = \bigcup_{i \in {1,\ldots,n}} U_i$.
		Clearly $y_i \not\in U$ for all $i \in {1, \ldots, n}$.
		Therefore $U = \left\{ x \right\}$ and $U$ is open
		as an arbitrary union of open sets is open.
		As $x$ was chosen arbitrarily, all singleton sets are open.
		Therefore, the union of any combination of singleton sets are open.
		Hence $\tau = \mathcal P(X)$.
	\end{solution}
	

	\question
		Let $X$ be a topological space, 
		and let $Y \subset X$ be a subspace of $X$
		(so $Y$ is given the subspace topology).
		Let $A \subset Y$.
		If $A$ is closed in $Y$, and if $Y$ is closed in $X$,
		show that $A$ is closed in $X$.

	\begin{solution}
		$A$ is closed in $Y$,
		so $Y \setminus  A$ is open with $\tau_A$
		and as $\tau_A \subset \tau$, $Y \setminus A$ is
		also open with $\tau$.
		Similarly as $Y$ is closed in $X$, 
		$X \setminus Y$ is open with $\tau$.
		Note that \[
			X \setminus A = (X \setminus Y) \cup (Y \setminus A)
		\]
		so $X \setminus A$ is a union of two open sets (with $\tau$);
		therefore, $X \setminus A$ is open with $\tau$.
		We conclude that $A$ is closed $X$.
	\end{solution}
\end{questions}
\end{document}
