\documentclass[a4paper, answers]{exam}
\documentclass[a4paper, answers]{exam}

% tikzcd.yichuanshen.de/ tikcd diagrams
%======================%
%   Standard packages  %
%======================%
\usepackage[utf8]{inputenc}
\usepackage[T1]{fontenc}
\usepackage{lmodern}
\usepackage[UKenglish]{babel}
\usepackage{enumitem}
\usepackage{tasks}
\usepackage{graphicx}
\setlist[enumerate,1]{
  label={(\roman*)}
}
\usepackage{parskip}
\usepackage{hyperref}

%======================%
%        Maths         %
%======================%
\usepackage{amsfonts, mathtools, amsthm, amssymb}
\usepackage{xfrac}
\usepackage{bm}
\newcommand\N{\ensuremath{\mathbb{N}}}
\newcommand\R{\ensuremath{\mathbb{R}}}
\newcommand\Z{\ensuremath{\mathbb{Z}}}
\newcommand\Q{\ensuremath{\mathbb{Q}}}
\newcommand\C{\ensuremath{\mathbb{C}}}
\newcommand\F{\ensuremath{\mathbb{F}}}
\newcommand{\abs}[1]{\ensuremath{\left\lvert #1 \right\rvert}}
\newcommand\given[1][]{\:#1\vert\:}
\newcommand\restr[2]{{% we make the whole thing an ordinary symbol
  \left.\kern-\nulldelimiterspace % automatically resize the bar with \right
  #1 % the function
  \vphantom{\big|} % pretend it's a little taller at normal size
  \right|_{#2} % this is the delimiter
}}

\newcommand\corestr[2]{{% we make the whole thing an ordinary symbol
  \left.\kern-\nulldelimiterspace % automatically resize the bar with \right
  #1 % the function
  \vphantom{\big|} % pretend it's a little taller at normal size
  \right|^{#2} % this is the delimiter
}}
\usepackage{siunitx}

\usepackage{afterpage}

\usepackage{tikz-cd}
\usepackage{adjustbox}
\DeclareMathOperator{\norm}{N}
\DeclareMathOperator{\trace}{Tr}
\DeclareMathOperator*{\argmax}{arg\,max}
\DeclareMathOperator*{\argmin}{arg\,min}
\DeclareMathOperator*{\esssup}{ess\,sup}
\DeclareMathOperator*{\SL}{SL}
\DeclareMathOperator*{\GL}{GL}
\DeclareMathOperator*{\SO}{SO}
\DeclareMathOperator*{\aut}{Aut}
\DeclareMathOperator*{\id}{id}
\DeclareMathOperator*{\coker}{coker}
\DeclareMathOperator*{\im}{im}



%======================%
%       CompSci        %
%======================%
\usepackage{forest}
\usepackage{textgreek}
\usepackage{algpseudocode}

%======================%
%    Pretty tables     %
%======================%
\usepackage{booktabs}
\usepackage{caption}

\begin{document}
\begin{center}
	\textbf{\textsc{Topology III, Assignment 4}}
	\\
	\textsc{Ben Napier}
	\vspace{1em}
\end{center}
\begin{questions}
	\question Let $X$ and $Y$ be topological spaces, and let $A, B \subset X$ be
	two closed sets such that $A \cup B = X$.
	Let $f: X \to Y$ be a function such that the restriction of $f$ to $A$ and
	the restriction of $f$ to $B$ are continuous (with respect to the induced
	subspace topologies).
	Show that $f$ is continuous.
	\begin{solution}
		Let $C \subset Y$ be closed.
		Observe that
		\[
			\left( \restr fA \right)^{-1}(C) = f^{-1}(C) \cap A, \qquad
			\left( \restr fB \right)^{-1}(C) = f^{-1}(C) \cap B;
		\]
		hence, 
		$\left( \restr fA \right)^{-1}(C)$ 
		and 
		$\left( \restr fB \right)^{-1}(C)$
		are both closed in $Y$.
		Now
		\begin{align*}
			f^{-1}(C)
			&= \left( f^{-1}(C) \cap A \right) \cup 
				\left( f^{-1}(C) \cap B \right) \\
			&= \left( \restr fA \right)^{-1}(C) 
				\cup \left( \restr fB \right)^{-1}(C),
		\end{align*}
		so $f^{-1}(C)$ is closed.
		Thus $f$ is continuous.
	\end{solution}

	\question Let $X$ be a topological space and let $\R$ have the standard
	topology.
	Suppose $f: X \to \R$ is a function that satisfies that $f^{-1}((-\infty,a))$
	and $f^{-1}((a,\infty))$ are both open for all $a \in \R$.

	Show that $f$ is a continuous map.
	\begin{solution}
		We claim that
		\[
			\mathcal B = \left\{ (a,b): a,b \in \R, a < b \right\}
		\]
		forms a basis for $\R$.
		Surely enough, $\R$ with the standard topology is a metric space,
		and we can express each open ball as an open interval as follows:
		\[
			B(x; \varepsilon) = (x - \varepsilon, x + \varepsilon)
		\]
		and so $\mathcal B$ is a basis for $\R$.
		Now, by assumption we have that
		$f^{-1}((-\infty, b))$
		and
		$f^{-1}((a, \infty))$
		are open, observe that
		\[
			f^{-1}((-\infty, b)) \cap f^{-1}((a, \infty))
			= f^{-1}((-\infty, b) \cap (a, \infty))
			= f^{-1}(a,b)
		\]
		and so $(a,b)$ is open in $X$.
		Hence, for any element $B \in \mathcal B$, $f^{-1}(B)$ is open.
		Thus $f$ is continuous.
	\end{solution}

	\question Give an example of a non-empty topological space $X$ and a subset
	$A \subset X$ such that the interior of $A$ is empty 
	($\mathring A = \varnothing$), and the closure of $A$ is the whole set $X$
	($\overline A = X$).
	\begin{solution}
		Consider $X = \Z$ with the trivial topology and $A = 5 \Z$.
		$\varnothing$ and $\Z$ are the only closed sets and open sets in this
		space.
		Now, the interior of $A$ describes the largest open set that is contained
		within $A$, and as $\Z \not\subset A$, we must have 
		$\mathring A = \varnothing$.
		Similarly, the exterior of $A$ describes the smallest closed that that
		contains $A$. 
		As $A \not\subset \varnothing$, we must have $\overline A = \Z$.
	\end{solution}
\end{questions}
\end{document}
