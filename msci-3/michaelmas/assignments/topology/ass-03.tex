\documentclass[a4paper, answers]{exam}
\documentclass[a4paper, answers]{exam}

% tikzcd.yichuanshen.de/ tikcd diagrams
%======================%
%   Standard packages  %
%======================%
\usepackage[utf8]{inputenc}
\usepackage[T1]{fontenc}
\usepackage{lmodern}
\usepackage[UKenglish]{babel}
\usepackage{enumitem}
\usepackage{tasks}
\usepackage{graphicx}
\setlist[enumerate,1]{
  label={(\roman*)}
}
\usepackage{parskip}
\usepackage{hyperref}

%======================%
%        Maths         %
%======================%
\usepackage{amsfonts, mathtools, amsthm, amssymb}
\usepackage{xfrac}
\usepackage{bm}
\newcommand\N{\ensuremath{\mathbb{N}}}
\newcommand\R{\ensuremath{\mathbb{R}}}
\newcommand\Z{\ensuremath{\mathbb{Z}}}
\newcommand\Q{\ensuremath{\mathbb{Q}}}
\newcommand\C{\ensuremath{\mathbb{C}}}
\newcommand\F{\ensuremath{\mathbb{F}}}
\newcommand{\abs}[1]{\ensuremath{\left\lvert #1 \right\rvert}}
\newcommand\given[1][]{\:#1\vert\:}
\newcommand\restr[2]{{% we make the whole thing an ordinary symbol
  \left.\kern-\nulldelimiterspace % automatically resize the bar with \right
  #1 % the function
  \vphantom{\big|} % pretend it's a little taller at normal size
  \right|_{#2} % this is the delimiter
}}

\newcommand\corestr[2]{{% we make the whole thing an ordinary symbol
  \left.\kern-\nulldelimiterspace % automatically resize the bar with \right
  #1 % the function
  \vphantom{\big|} % pretend it's a little taller at normal size
  \right|^{#2} % this is the delimiter
}}
\usepackage{siunitx}

\usepackage{afterpage}

\usepackage{tikz-cd}
\usepackage{adjustbox}
\DeclareMathOperator{\norm}{N}
\DeclareMathOperator{\trace}{Tr}
\DeclareMathOperator*{\argmax}{arg\,max}
\DeclareMathOperator*{\argmin}{arg\,min}
\DeclareMathOperator*{\esssup}{ess\,sup}
\DeclareMathOperator*{\SL}{SL}
\DeclareMathOperator*{\GL}{GL}
\DeclareMathOperator*{\SO}{SO}
\DeclareMathOperator*{\aut}{Aut}
\DeclareMathOperator*{\id}{id}
\DeclareMathOperator*{\coker}{coker}
\DeclareMathOperator*{\im}{im}



%======================%
%       CompSci        %
%======================%
\usepackage{forest}
\usepackage{textgreek}
\usepackage{algpseudocode}

%======================%
%    Pretty tables     %
%======================%
\usepackage{booktabs}
\usepackage{caption}

\begin{document}
\begin{center}
	\textbf{\textsc{Topology III, Assignment 3}}
	\\
	\textsc{Ben Napier}
	\vspace{1em}
\end{center}
\begin{questions}
	\question
		Consider the followinmg topologies on $X = \left\{ 1,2,4 \right\}$
		that we saw in the last problem sheets.
		Decide which pairs are homeomorphic.
		Write down a homeomorphism for the homeomorphic ones.
		\begin{enumerate}
			\item $\tau_2 = \left\{
					\varnothing,
					\left\{ 1 \right\},
					\left\{ 2 \right\},
					\left\{ 1,2 \right\},
					\left\{ 1,2,4 \right\}
				\right\}$;

			\item $\tau_4 = \left\{
					\varnothing,
					\left\{ 2 \right\},
					\left\{ 4 \right\},
					\left\{ 2,4 \right\},
					\left\{ 1,2,4 \right\}
				\right\}$;

			\item $\tau_5 = \left\{
					\varnothing,
					\left\{ 2,4 \right\},
					\left\{ 1,2,4 \right\}
				\right\}$; and

			\item $\tau_6 = \left\{
					\varnothing,
					\left\{ 2 \right\},
					\left\{ 1,2,4 \right\}
				\right\}$.
		\end{enumerate}

		\begin{solution}
			We know that neither $\tau_2$ or $\tau_4$ cannot be homeomorphic
			to $\tau_5$ or $\tau_6$ as 
			$
				\left\lvert \tau_2 \right\rvert
				= \left\lvert \tau_4 \right\rvert
				= 5
			$
			and 
			$
				\left\lvert \tau_5 \right\rvert
				= \left\lvert \tau_6 \right\rvert
				= 3.
			$
			We have left to check two pairs.
			$\tau_2 \cong \tau_4$ with map $h: X \to X$ such that:
			$1 \mapsto 2$, $2 \mapsto 4$, $3 \mapsto 3$, and $4 \mapsto 1$.
			As any bijection must map a singleton set to a singleton set, we
			have that $\tau_5 \not\cong \tau_6$ as there is no singleton set
			in $\tau_5$ to map $\left\{ 2 \right\} \in \tau_6$ to.
		\end{solution}

	\pagebreak
	\question
		Suppose that $f: X \to Y$ is a homeomorphism, and suppose
		$A \subset X$.
		Show that the induced map
		\[
			\restr{f}{A}: A \to f(A)
		\]
		is a homeomorphism between $A$ (with its topology induced from that 
		of $X$) and $f(A)$ (with its topology induced from that of $Y$).
		\begin{solution}
			Let $U \subset f(A)$ be open.
			Then, as $f$ is continuous, $f^{-1}(U)$ is open in $X$.
			Now $\left( \restr fA \right)^{-1}(U) = f^{-1}(U) \cap A$; hence,
			$\restr fA$ is continuous.
			Similarly, let $V \subset A$ be open.
			Then $f(V)$ is open (as $f^{-1}$ is continuous).
			As $V \subset A$, $\restr fA(V) = f(V)$ as $V \subset A$, so
			$\left( \restr fA \right)^{-1}$ is continuous.
			Clearly $\restr fA$ is a bijection; hence, it is a homeomorphism.
		\end{solution}

	\pagebreak
	\question
		Show that the following subset of $\R^2$ is neither open nor closed
		(in the standard topology on $\R^2$).
		\[
			A = \left\{
				(x,y) \in \R^2 : x \geq 0, y > 0
			\right\}.
		\]
		
		\begin{solution}
			Observe that $(0,1) \in A$ and let $\varepsilon > 0$.
			Now, see that 
			$A \not\ni (-\sfrac{\varepsilon}2, 1) \in B((0,1);\varepsilon)$; 
			hence, $B((0,1); \varepsilon) \not\subset A$.
			Therefore, $A$ is not open (since $\varepsilon$ was chosen 
			arbitrarily).
			Now, observe that
			\[
				\R^2 \setminus A = \left\{ 
					(x,y) \in \R^2: x < 0 \;\text{or}\;y \leq 0 
				\right\}
			\]
			and we see that $(1,0) \in A$. 
			Again, let $\varepsilon > 0$.
			We see that 
			$A \not\ni (1, \sfrac{\varepsilon}2) \in B((1,0); \varepsilon)$
			and so $B((1,0); \varepsilon) \not\subset A$;
			hence, $\R^2 \setminus A$ is not open.
			Therefore $A$ is not closed.
		\end{solution}

	\pagebreak
	\question
		Decide which of the following subsets of $\R^n$ is open, and which
		ones are closed (some might be both, and some might be neither).
		Justify your statements for each set.
		\begin{parts}
			\part $X_1 = \left\{ (x,y,z) \in \R^3 : x > y > z \right\}$;
			\begin{solution}
				Observe that $X_1 = O_1 \cup O_2$ where 
				\begin{align*}
					O_1
					&= \left\{ (x,y,z) \in \R^3 : x > y \right\},
					O_2
					&= \left\{ (x,y,z) \in \R^3 : y > z \right\}.
				\end{align*}
				We define $f_1(x,y,z) = x - y$ and $f_2(x,y,z) = y - z$.
				We see that $f_1^{-1}((0,\infty)) = O_1$ and
				$f_2^{-1}((0,\infty)) = O_2$ and both functions are
				continuous; hence, $O_1$ and $O_2$ are open.
				Therefore $X_1$ is open.
				Now see
				\[
					\R \setminus X_1 = \left\{ 
						(x,y,z) \in \R^3: x \leq y \;\text{or}\; y \leq z 
					\right\}.
				\]
				Let $\varepsilon > 0$.
				We have $(0,0,0) \in \R \setminus X_1$ but
				$
					(
						\sfrac{2\varepsilon}3, 
						\sfrac{\varepsilon}{3}, 
						0
					) \not\in \R \setminus X_1.
				$
				Therefore 
				$B((0,0,0); \varepsilon) \not\subset \R \setminus X_1$;
				hence, $\R \setminus X_1$ is not open and so $X_1$ is not
				closed.
			\end{solution}

			\part 
				$
					X_2 = \left\{ 
						(x,y,z,w) \in \R^4: x^4 + y^3 + z^2 = w
					\right\}
				$;
			\begin{solution}
				We define $f(x,y,z,w) = x^4 + y^3 + z^2 - w$.
				Clearly $f$ is continuous; hence, the preimage
				$f^{-1}(\left\{ 0 \right\}) = X_2$ is closed (since
				$\left\{ 0 \right\}$ is closed in $\R$).
				Now suppose $X_2$ is open.
				Let $(x_1,y_1,z_1,w_1) \in X_2$.
				There is $\delta > 0$ such that 
				\[
					(x_1,y_1,z_1,w_1 + \sfrac{\delta}2) 
					\in B((x_1,y_1,z_1,w_1); \delta)
					\subset X_2
				\]
				but if $x_1^4 + y_1^3 + z_1^2 = w_1$ and
				$x_1^4 + y_1^3 + z_1^2 = w_1 + \sfrac{\delta}2$
				then $\sfrac{\delta}2 = 0$; a contradiction.
				Therefore $X_2$ is not open.
			\end{solution}

			\part 
				$
					X_3 = \left\{ 
						(x_1, x_2) \in \R^2: x_1 \geq x_2^2 
					\right\}
				$; and
			\begin{solution}
				We define $h(x_1, x_2) = x_1 - x_2^2$, clearly continuous.
				Then $h^{-1}([0,\infty)) = X_3$ and so $X_3$ is closed.
				Now assume that $X_3$ is open.
				We see that $(0,0) \in X_3$ and so there is $\delta > 0$ such
				that
				\[
					(-\sfrac{\delta}2,0) 
					\in B((0,0); \delta)
					\subset X_3
				\]
				but clearly $(-\sfrac{\delta}2, 0) \not\in X_3$; a 
				contradiction. 
				Therefore $X_3$ is not open.
			\end{solution}

			\part 
				$
					X_4 = \left\{ 
						(x_1, x_2, x_3) \in \R^3: 
						x_1 < x-2 + x_3^3, x_3 \leq 5
					\right\}
				$.
			\begin{solution}
				Suppose $X_4$ is open and observe that $(0,0,5) \in X_4$.
				Then there is $\delta > 0$ such that
				\[
					(0,0,5 + \sfrac{\delta}2) 
					\in B((0,0,5); \delta)
					\subset X_4
				\]
				but clearly $(0,0,5 + \sfrac{\delta}2) \not\in X_4$; a
				contradiction. Hence $X_4$ is not open.
				Now suppose $X_4$ is closed.
				So $\R^3 \setminus X_4$ is open.
				Observe
				\[
					\R^3 \setminus X_4
					= \left\{
						(x_1,x_2,x_3) \in \R^3: x_1 \geq x_2 + x_3^3
						\;\text{or}\;
						x_3 > 5
					\right\}
				\]
				and $(0,0,0) \in \R^3 \setminus X^4$.
				Now, as $\R^3 \setminus X_4$ is open, there is $\delta > 0$
				such that
				\[
					(-\sfrac{\delta}2, 0, 0) 
					\in B((0,0,0); \delta)
					\subset X_4
				\]
				but we can see that $(-\sfrac{\delta}2, 0, 0) \not\in X_4$;
				a contradiction.
				Therefore $X_4$ is not closed.
			\end{solution}
		\end{parts}
\end{questions}
\end{document}
