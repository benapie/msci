\documentclass[a4paper, answers]{exam}
\documentclass[a4paper, answers]{exam}

% tikzcd.yichuanshen.de/ tikcd diagrams
%======================%
%   Standard packages  %
%======================%
\usepackage[utf8]{inputenc}
\usepackage[T1]{fontenc}
\usepackage{lmodern}
\usepackage[UKenglish]{babel}
\usepackage{enumitem}
\usepackage{tasks}
\usepackage{graphicx}
\setlist[enumerate,1]{
  label={(\roman*)}
}
\usepackage{parskip}
\usepackage{hyperref}

%======================%
%        Maths         %
%======================%
\usepackage{amsfonts, mathtools, amsthm, amssymb}
\usepackage{xfrac}
\usepackage{bm}
\newcommand\N{\ensuremath{\mathbb{N}}}
\newcommand\R{\ensuremath{\mathbb{R}}}
\newcommand\Z{\ensuremath{\mathbb{Z}}}
\newcommand\Q{\ensuremath{\mathbb{Q}}}
\newcommand\C{\ensuremath{\mathbb{C}}}
\newcommand\F{\ensuremath{\mathbb{F}}}
\newcommand{\abs}[1]{\ensuremath{\left\lvert #1 \right\rvert}}
\newcommand\given[1][]{\:#1\vert\:}
\newcommand\restr[2]{{% we make the whole thing an ordinary symbol
  \left.\kern-\nulldelimiterspace % automatically resize the bar with \right
  #1 % the function
  \vphantom{\big|} % pretend it's a little taller at normal size
  \right|_{#2} % this is the delimiter
}}

\newcommand\corestr[2]{{% we make the whole thing an ordinary symbol
  \left.\kern-\nulldelimiterspace % automatically resize the bar with \right
  #1 % the function
  \vphantom{\big|} % pretend it's a little taller at normal size
  \right|^{#2} % this is the delimiter
}}
\usepackage{siunitx}

\usepackage{afterpage}

\usepackage{tikz-cd}
\usepackage{adjustbox}
\DeclareMathOperator{\norm}{N}
\DeclareMathOperator{\trace}{Tr}
\DeclareMathOperator*{\argmax}{arg\,max}
\DeclareMathOperator*{\argmin}{arg\,min}
\DeclareMathOperator*{\esssup}{ess\,sup}
\DeclareMathOperator*{\SL}{SL}
\DeclareMathOperator*{\GL}{GL}
\DeclareMathOperator*{\SO}{SO}
\DeclareMathOperator*{\aut}{Aut}
\DeclareMathOperator*{\id}{id}
\DeclareMathOperator*{\coker}{coker}
\DeclareMathOperator*{\im}{im}



%======================%
%       CompSci        %
%======================%
\usepackage{forest}
\usepackage{textgreek}
\usepackage{algpseudocode}

%======================%
%    Pretty tables     %
%======================%
\usepackage{booktabs}
\usepackage{caption}

\begin{document}

\begin{center}
	\textbf{\textsc{Topology III, Assignment 9}}
	\\
	\textsc{Ben Napier}
	\vspace{1em}
\end{center}

\begin{questions}
	\question
	In this question we shall show that $\operatorname{SO}(n)$ is connected
	by induction.

	Recall that in lectures we have that if a connected topological group $G$
	acts on a space $X$ such that $X/G$ is connected then $X$ is connected.
	\begin{parts}
		\part
		Let $n \geq 0$.
		Show that inclusion of spaces 
		$
			i: \operatorname{SO}(n) 
			\xhookrightarrow{} \operatorname{SO}(n+1)
		$
		given by
		\[
			i(A) = \begin{pmatrix}
				1 & 0 \\ 0 & A
			\end{pmatrix}
		\]
		includes $\operatorname{SO}(n)$ as a subgroup of
		$\operatorname{SO}(n+1)$.

		\part
		Thinking of $\operatorname{SO}(n)$ as a subgroup, we let
		$\operatorname{SO}(n)$ acts of $\operatorname{SO}(n+1)$
		by left multiplication $A \cdot B = i(A)B$ for all
		$A \in \operatorname{SO}(n)$, $B \in \operatorname{SO}(n+1)$.
		Suppose $v \in S^n \subset \R^{n+1}$ is a point on the unit
		$n$-sphere expressed as a row vector and suppose that $C$ is an
		$n \times (n + 1)$ matrix such that
		$
			\begin{pmatrix}
				v \\ C
			\end{pmatrix}
			\in \operatorname{SO}(n+1)
		$.
		Show that the orbit of
		$
			\begin{pmatrix}
				v \\ C
			\end{pmatrix}
		$
		is given by
		\[
			\left\{
				\begin{pmatrix}
					v \\ D
				\end{pmatrix}:
				\begin{pmatrix}
					v \\ D
				\end{pmatrix}
				\in \operatorname{SO}(n+1)
			\right\}.
		\]

		\part
		Hence or otherwise give a continuous bijective map
		\[
			\phi: \operatorname{SO}(n+1)/\operatorname{SO}(n)
			\to S^n.
		\]

		\part Argue that $\phi$ is a homeomorphism.

		\part Using the connectedness of $S^n$ for $n \geq 1$,
		and the connectedness of $\operatorname{SO}(1)$
		(which just consists of a single point),
		give an induction showing that $\operatorname{SO}(n)$
		is connected for all $n \geq 1$.
	\end{parts}

	\question
	Let the group of $n \times n$ invertible matrices $\operatorname{GL}_n(\R)$
	act in the usual way on $\R^n$ by left multiplication.
	Show that $\R^n/\operatorname{GL}_n(\R)$ consists of two points,
	and give the topology on this quotient space.
\end{questions}

\end{document}
