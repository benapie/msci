\documentclass[a4paper, answers]{exam}
\usepackage[utf8]{inputenc}

\usepackage{parskip}

\usepackage{amssymb}
\usepackage{amsmath}
\usepackage{amsfonts}
\usepackage{mathtools}

\usepackage{todonotes}

\usepackage{csquotes}

\usepackage{algpseudocode}
\usepackage{algorithm}

\DeclareMathOperator{\AQ}{AQ}
\DeclareMathOperator{\DAQ}{\Delta AQ}
\DeclareMathOperator{\Q}{Q}
\DeclareMathOperator{\HE}{HE}
\DeclareMathOperator*{\argmax}{arg\,max}
\DeclareMathOperator{\Ep}{Ep}

\usepackage{braket}

\addtolength{\oddsidemargin}{-.875in}
\addtolength{\evensidemargin}{-.875in}
\addtolength{\textwidth}{1.75in}
\addtolength{\topmargin}{-.875in}
\addtolength{\textheight}{1.75in}

\usepackage[backend=biber]{biblatex}
\addbibresource{ref.bib}

\title{Natural Computing Part A}
\author{Ben Napier}
\date{March 2022}

\begin{document}

\begin{center}
	\textbf{\textsc{Analysis III, 2018}} \\
	\textsc{Ben Napier}
	\vspace{1em}
\end{center}

Credit will be given for the best \emph{four} answers from Section A
and the best \emph{three} answers from Section B.
Questions in Section B carry \emph{twice} as many marks as those in Section A.

\begin{center}
	\textsc{\textbf{Section A}}
\end{center}

\begin{questions}
	\question
	\begin{parts}
		\part
		Define what it means for a set to be countable.

		\part
		Let $\Q\left[\sqrt 2\right] = \left\{
			a + b\sqrt 2: a, b \in \Q
		\right\} \subset \R$.
		Show that $\Q\left[\sqrt 2\right]$ is countable.

		\part
		Show that the set of all infinite sequences of natural numbers is
		uncountable.
	\end{parts}

	\question
	Let $\left\{a_n\right\}$ be a sequence of real numbers, 
	and let $A$ be the set of all elements of $\left\{a_n\right\}$.
	\begin{parts}
		\part
		Define $\limsup a_n$.

		\part
		Show that $\sup A \geq \limsup a_n$.

		\part
		Assume that $\limsup a_n \in A$.
		Does this imply that $\sup A \in A$?
	\end{parts}

	\question
	\begin{parts}
		\part
		Define what it means for a set $E \subset \R$ to be measurable.

		\part
		Show that if $E \subset \R$ is measurable and bounded and
		$\varepsilon > 0$, then there exists a finite collection
		$\left\{E_1, \ldots, E_n\right\}$ of mutually disjoint measurable sets
		such that $\bigcup_{i=1}^n E_i = E$ and $m(E_i) \leq \varepsilon$
		for all $i = 1, \ldots, n$.

		\part
		Is the assertion of (b) true if $E$ is unbounded of finite measure?
	\end{parts}

	\question
	\begin{parts}
		\part
		Define what it means for $f: \R \to \R$ to be measurable.

		\part
		Prove that $f(x) \vcentcolon= \sin(3x)$ is measurable.
		You may use that open intervals are measurable.

		\part
		Let $f: \R \to \R$ be measurable, $a \in \R$ and define the function
		$g: \R \to \R$ by $g \vcentcolon= f(x - a)$.
		Prove that $g$ is measurable.
	\end{parts}

	\question
	\begin{parts}
		\part
		Define $\int f$ for a nonnegative measurable function $f: \R \to \R$.

		\part
		Prove that for measurable functions $f,g$
		with $0 \leq f(x) \leq g(x)$ we have
		$\int f \leq \int g$.

		\part
		Let $f$ be as in (a), assume that $\int f = 0$, $c > 0$, and
		$A_c \vcentcolon= \left\{x \in \R: f(x) > c\right\}$.
		Prove that $m(A_c) = 0$, where $m$ denotes the Lebesgue measure
		of $\R$.
	\end{parts}

	\question
	\begin{parts}
		\part
		State the Lemma of Fatou.

		\part
		Let
		\[
			f_n(x) =
			\begin{cases}
				0 & x \leq n; \\
				1 & x > n.
			\end{cases}
		\]
		Do the \textbf{assumptions} of the Lemma of Fatou apply to the
		sequence?
		Prove your answer.

		\part
		Does the \textbf{conclusion} of the Lemma of Fatou apply to the 
		sequence $f_n$ as in $(b)$? 
		Prove your answer.
	\end{parts}

	\begin{center}
		\textsc{\textbf{Section B}}
	\end{center}

	\question
	\begin{parts}
		\part 
		Define the outer measure $m^\star(E)$ of a set $E \subset \R$.

		\part
		Use the definition (a) to show that the outer measure is monotone, i.e.
		$m^\star(A) \leq m^\star(B)$ for $A \subset B$.

		\part
		Show that a set $E \subset \R$ is measurable if and only if
		for every $\varepsilon > 0$ there exists an open set $U$
		and a close set $F$ such that $F \subset E \subset U$
		and $m^\star(U \setminus F) < \varepsilon$.

		\part
		Let $E \subset \R$ have finite outer measure.
		Show that there exists a countable intersection $G$ of open sets such
		that $E \subset G$ and $m^\star(E) = m^\star(G)$.
		Does this imply that $m^\star(G \setminus E) = 0$?
	\end{parts}

	\question
	A set $A \subset \R$ is called \emph{nowhere dense} if every open
	non-empty subset $U_0 \subset U$ such that $U_0 \cap A = \varnothing$.
	A set $B \subset \R$ is called \emph{dense} if the closure of $B$
	conincides with $\R$.
	\begin{parts}
		\part
		Let $N \in \N$.
		Show that the set $\left\{\frac pq: p,q \in \N, p \leq N\right\}$
		is nowhere dense.

		\part
		Let $B \subset \R$ be dense.
		Is it true that $\R \setminus B$ has to be nowhere dense?

		\part
		Let $A \subset \R$ be nowhere dense.
		Is it true that $\R \setminus A$ has to be dense?

		\part
		Show that for every $\varepsilon > 0$ there exists a nowhere dense set
		$E \subset \R$ such that $m^\star(\R \setminus E) < \varepsilon$.
	\end{parts}

	\question
	\begin{parts}
		\part
		Define the collection of functions
		$L^1([0,2\pi])$
		and the collection of functions
		$L^2([0,2\pi])$.

		\part
		Prove or disprove by counterexample the following claim:
		$L^1([0,2\pi]) \subset L^2([0,2\pi])$.

		\part
		State the Dominated Convergence Theorem.
	\end{parts}

	\question
	\begin{parts}
		\part
		Define an inner product on $L^2([0,2\pi])$ and define what it means that
		$L^2([0,2\pi])$ isa  Hilbert space.

		\part
		State and prove the Bessel Inequality on the Hilbert space 
		$L^2([0,2\pi])$.

		\part
		Define functions $f_n: \R \to \R$ by 
		$f_n(x) = \left(\frac 1n\right)_{\chi_{[0,n]}}$
		and $f(x) = 0$.
		Here, $\chi_{[0,n]}$ has value $1$ on $[0, n]$ and vanishes
		elsewhere.
		Show that $f_n$ converges uniformly to $f$ but that
		$\lim \int f_n \neq \int f$.
		Why does this not contradict the Monotone Convergence Theorem?
	\end{parts}
\end{questions}

\end{document}
