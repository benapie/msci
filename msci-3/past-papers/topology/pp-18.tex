\documentclass[a4paper, answers]{exam}
\documentclass[a4paper, answers]{exam}

% tikzcd.yichuanshen.de/ tikcd diagrams
%======================%
%   Standard packages  %
%======================%
\usepackage[utf8]{inputenc}
\usepackage[T1]{fontenc}
\usepackage{lmodern}
\usepackage[UKenglish]{babel}
\usepackage{enumitem}
\usepackage{tasks}
\usepackage{graphicx}
\setlist[enumerate,1]{
  label={(\roman*)}
}
\usepackage{parskip}
\usepackage{hyperref}

%======================%
%        Maths         %
%======================%
\usepackage{amsfonts, mathtools, amsthm, amssymb}
\usepackage{xfrac}
\usepackage{bm}
\newcommand\N{\ensuremath{\mathbb{N}}}
\newcommand\R{\ensuremath{\mathbb{R}}}
\newcommand\Z{\ensuremath{\mathbb{Z}}}
\newcommand\Q{\ensuremath{\mathbb{Q}}}
\newcommand\C{\ensuremath{\mathbb{C}}}
\newcommand\F{\ensuremath{\mathbb{F}}}
\newcommand{\abs}[1]{\ensuremath{\left\lvert #1 \right\rvert}}
\newcommand\given[1][]{\:#1\vert\:}
\newcommand\restr[2]{{% we make the whole thing an ordinary symbol
  \left.\kern-\nulldelimiterspace % automatically resize the bar with \right
  #1 % the function
  \vphantom{\big|} % pretend it's a little taller at normal size
  \right|_{#2} % this is the delimiter
}}

\newcommand\corestr[2]{{% we make the whole thing an ordinary symbol
  \left.\kern-\nulldelimiterspace % automatically resize the bar with \right
  #1 % the function
  \vphantom{\big|} % pretend it's a little taller at normal size
  \right|^{#2} % this is the delimiter
}}
\usepackage{siunitx}

\usepackage{afterpage}

\usepackage{tikz-cd}
\usepackage{adjustbox}
\DeclareMathOperator{\norm}{N}
\DeclareMathOperator{\trace}{Tr}
\DeclareMathOperator*{\argmax}{arg\,max}
\DeclareMathOperator*{\argmin}{arg\,min}
\DeclareMathOperator*{\esssup}{ess\,sup}
\DeclareMathOperator*{\SL}{SL}
\DeclareMathOperator*{\GL}{GL}
\DeclareMathOperator*{\SO}{SO}
\DeclareMathOperator*{\aut}{Aut}
\DeclareMathOperator*{\id}{id}
\DeclareMathOperator*{\coker}{coker}
\DeclareMathOperator*{\im}{im}



%======================%
%       CompSci        %
%======================%
\usepackage{forest}
\usepackage{textgreek}
\usepackage{algpseudocode}

%======================%
%    Pretty tables     %
%======================%
\usepackage{booktabs}
\usepackage{caption}

\begin{document}

\begin{center}
	\textbf{\textsc{Topology III, 2018}} \\
	\textsc{Ben Napier}
	\vspace{1em}
\end{center}

\begin{center}
	\scshape\bfseries Section A
\end{center}

\begin{questions}
\question
\begin{parts}
	\part State the definition of a Hausdorff space.

	\part Let $X = \left\{ 1,2,3,4,5 \right\}$.
	Give an example of a Hausdorff topology on $X$, and an example of a
	non-Hausdorff topology on $X$.

	\part Assume that $X$ is a Hausdorff space.
	Show that
	\[
		\Delta(X)
		= \left\{ (x,x) \in X \times X: x \in X \right\}
	\]
	is a closed subset of $X \times X$.
\end{parts}

\question Let $X$ be a topological space and $A \subset X$.
\begin{parts}
	\part State the definition of a limit point of $A$.

	\part Show that if $x$ is a limit point of $A$, then $x \in \overline A$,
	the closure of $A$.

	\part Determine the limit points of 
	\[
		A = \left\{ 
			\cos\left( \frac1n \right) \in \R:
			n \in \Z \setminus \left\{ 0 \right\} 
		\right\}.
	\]
\end{parts}

\question
\begin{parts}
	\part State the definition of a path-connected space.

	\part Let
	\[
		X = \left\{ 
			\left( \cos\left( 
				\frac1x \right), 
				\sin\left( \frac1x \right), 
				x 
			\right) \in \R^3: x \in (0,1)
		\right\}.
	\]
	Determine whether $X$ is path-connected or not.
	Justify your statement.
\end{parts}

\question
\begin{parts}
	\part Suppose $\lambda: [0,1] \to X$ and $\mu: [0,1] \to X$ are two loops
	in the space $X$ based at the point $x_0$.
	Define the product loop $\lambda * \mu: [0,1] \to X$.

	\part Now suppose $\lambda$, $\mu$, $\kappa$, and $\nu$ are four loops in
	$X$ based at $x_0$.
	Prove that in general the loops $(\lambda * \mu) * (\kappa * \nu)$ and
	$\lambda * (\mu * (\kappa * \nu))$ are not the same.
	Prove, using a suitable diagram, that they represent however the same
	element of $\pi_1(X, x_0)$. (You do not have to give explicit formulas.)
\end{parts}

\question
\begin{parts}
	\part Let $S = \left\{ z \in \C: \left\lvert z \right\rvert = 1 \right\}$
	be the unit circle in the complex plane, and $f: \C \to \C$ be a
	continuous funciton.
	Let $W(f)$ be the winding number of the loop $f(S)$ about the origin:
	explain precisely what this means and what assumption must be made about
	$f$ for it to be defined.

	\part Say if each of the following statements are true or false in
	general, justifying your answers.
	In both statements, $r$ is a non-zero complex number.
	\begin{subparts}
		\subpart $W(rf) = W(f)$ where $rf$ is the function
		$z \mapsto r \cdot f(z)$.

		\subpart $W(fr) = W(f)$ where $fr$ is the function $z \mapsto f(rz)$.
	\end{subparts}
\end{parts}

\question
\begin{parts}
	\part If $A$ and $B$ are finite simplicial complexes, state a formula
	relating the Euler characteristics of $A$, $B$, $A \cup B$, and
	$A \cap B$.
	State, without proof, the Euler characteristics for the sphere $S^2$, the
	circle $S^1$, and the closed $2$-disc $D^2$.
	Use your formula to compute the Euler characteristic of the space $Y$
	which is given by removing three non-intersecting open discs from a
	sphere $S^2$.

	\part A closed surface is constructed by taking $26$ copies of the space
	$Y$ and identifying pairs of boundary circles in such a way that the
	final space is connected and without boundary.
	How many different closed surfaces (up to homeomorphism) can be 
	constructed in this way?
	Justify your answer, identifying the ones you think can be constructed in
	terms of the families of surfaces given in the Classification Theorem of
	Closed Surfaces considered in lectures, explaining any notation you use.
\end{parts}

\begin{center}
	\scshape\bfseries Section B
\end{center}

\question Let $X = \R^n \cup \left\{ \infty \right\}$, that is, the set
consisting of $\R^n$ and one extra element denoted $\infty$ (note that $X$ is
not a topological space yet).
In particular, we have $\R^n \subset X$, but $\R^n \neq X$.
\begin{parts}
	\part Let $U \subset \R^n$ be open and $C \subset \R^n$ compact.
	Show that $U \cap \left(X \setminus C \right)$ is an open subset of 
	$\R^n$.

	\part Let $U \subset \R^n$ be open and $C \subset \R^n$ compact.
	Show that $U \cup \left( X \setminus C \right) = X \setminus D$ for a
	compact set $D \subset \R^n$.

	\part Let
	\[
		\tau 
		= \left\{ U \subset \R^n: \text{$U$ open in $\R^n$} \right\} \cup
			\left\{ X \setminus C: \text{$C \subset \R^n$ compact} \right\}
	\]
	which is a subset of the power set of $X$, as $\R^n \subset X$.
	\begin{subparts}
		\subpart Show that $\tau$ is a topology.
		\subpart Show that $X$ with this topology is compact.
		\subpart For $x = (x_1, \ldots, x_n) \in \R^n$ let
		$\left\lvert x \right\rvert^2 = x_1^2 + \ldots + x_n^2$.
		Show that the function $f: X \to \R^n \times \R$ given by
		\[
			f(x) =
			\begin{cases}
				\left( 
					\frac{2}{1 + \left\lvert x \right\rvert^2} \cdot x,
					1 - \frac{2}{1 + \left\lvert x \right\rvert^2}
				\right) & \text{if $x \in \R^n$}, \\
					(0,1) & \text{if $x = \infty$}
			\end{cases}
		\]
		is a homeomorphism onto its image.
		(Remark: we write $\R^{n+1} = \R^n \times \R$ to emphasize that we
		only need two coordinates, an $\R^n$-coordinate and an
		$\R$-coordinate.
		It is poosible to do this with $(n+1)$ real coordinates, which only
		means you have to write more.)
	\end{subparts}
\end{parts}
	
\question Let 
$U(n) = \left\{ A \in \operatorname{GL}_n(\C) : AA^* = I \right\}$, where
$A^\star = \left( a^\star_{ij} \right)$ is the matrix whose entries
satisfy $a^\star_{ij} = \overline a_{ij}$, where 
$A = \left( a_{in} \right)$, and $\overline a$ is the complex conjugate
of $a \in \C$.
State any results from the lectures you are using.
You may use that $U(n)$ is compact.
\begin{parts}
	\part Show that for $A \in U(n-1)$ the matrix
	\[
		i(A) =
		\begin{pmatrix}
			&&& 0 \\
			& A && \vdots \\
			&&& 0 \\
			0 & \ldots & 0 & 1 \\
		\end{pmatrix}
	\]
	satisfies $i(A) \in U(n)$.

	\part Show that $U(n-1)$ acts on $U(n)$ by
	\[
		A \cdot B = Bi(A^\star)
	\]
	where $A \in U(n-1)$, $B \in U(n)$, and we use matrix multiplication on
	the right hand side.

	\part Show that $f: U(n) \to S^{2n-1}$ given by
	\[
		f(B) = Be_n
	\]
	where $e_n = (0, \ldots, 0, 1) \in \C^n$, induces a map
	$\overline f: U(n) / U(n-1) \to S^{2n-1}$ which is a homeomorphism.
	Here
	\[
		S^{2n-1} = \left\{
			(z_1, \ldots, z_n) \in \C^n: 
			z_1\overline z_1 + \ldots + z_n \overline z_n = 1
		\right\}.
	\]

	\part Show that $U(n)$ is connected for all $n \geq 1$.
\end{parts}

\question 
\begin{parts}
	\part Give a net for the Klein bottle $K$ as a quotient of a unit square.
	Triangulate your square so as to give a triangulation of $K$ and use it 
	to compute $\pi_1(K)$, briefly explaining any diagrams you use.

	\part Uisng two copies of your triangulated square, pieced together
	appropriately, show that there is a triangulated orientable closed
	surface $W$ and a map $f: W \to K$ with the property that for each
	$n$-simplex $\sigma$ in $K$, there are precisely two $n$-simplices,
	$\tau_1$, $\tau_2$ in $W$ such that $f(\tau_i) = \sigma$ for $i = 1,2$.

	\part Identify the surface $W$.

	\part Use your triangulation of $W$ to compute $\pi_1(W)$.

	\part For each of your spaces, pick a point to be its basepoint and
	interpret the generators of your fundamental groups as loops based at
	this point.
	Use this to describe the homomorphism $f_\star: \pi_1(W) \to \pi_1(K)$.
\end{parts}
\end{questions}

\end{document}
