\documentclass[a4paper, answers]{exam}
\usepackage[utf8]{inputenc}

\usepackage{parskip}

\usepackage{amssymb}
\usepackage{amsmath}
\usepackage{amsfonts}
\usepackage{mathtools}

\usepackage{todonotes}

\usepackage{csquotes}

\usepackage{algpseudocode}
\usepackage{algorithm}

\DeclareMathOperator{\AQ}{AQ}
\DeclareMathOperator{\DAQ}{\Delta AQ}
\DeclareMathOperator{\Q}{Q}
\DeclareMathOperator{\HE}{HE}
\DeclareMathOperator*{\argmax}{arg\,max}
\DeclareMathOperator{\Ep}{Ep}

\usepackage{braket}

\addtolength{\oddsidemargin}{-.875in}
\addtolength{\evensidemargin}{-.875in}
\addtolength{\textwidth}{1.75in}
\addtolength{\topmargin}{-.875in}
\addtolength{\textheight}{1.75in}

\usepackage[backend=biber]{biblatex}
\addbibresource{ref.bib}

\title{Natural Computing Part A}
\author{Ben Napier}
\date{March 2022}

\begin{document}

\begin{center}
	\textbf{\textsc{Number Thoery III, 2017}} \\
	\textsc{Ben Napier}
	\vspace{1em}
\end{center}

Credit will be given for the best \emph{two} answers
from Section A, the best \emph{three} answers from Section B,
and the answer to \emph{the} answer to the question from Section C.
Questions in Section B and C carry twice as many marks as those in
Section A.

\begin{center}
	\textsc{\textbf{Section A}}
\end{center}

\begin{questions}
	\question 
	\begin{parts}
		\part
		Prove that there are no integral solutions to the Diophantine 
		equation
		\[
			15x^2 - y^2 = 1.
		\]

		\part
		Let $R = \Z\left[\sqrt d\right]$, where $d < -1$ is an odd integer.
		\begin{subparts}
			\subpart
			Prove that 2 is an irreducible element in $R$.
			
			\subpart
			Using the previous part or otherwise, prove that $R$ is not a UFD.
			(Hint: You may consider the element $1 - d$).
		\end{subparts}
	\end{parts}

	\question
	\begin{parts}
		\part
		What is the full ring of integers $\mathcal O_{-11}$ in
		$K = \Q\left(\sqrt{-11}\right)$?

		\part
		Put $\alpha = -5 + \sqrt{-11}$ and $\beta = 7 + \sqrt{-11}$.
		By considering $\beta/\alpha$, or otherwise, find
		$\gamma \in \mathcal O_{-11}$ such that
		\[
			\operatorname{N}_{K/\Q}(\beta - \gamma\alpha)
			< \operatorname{N}_{K/\Q}(\alpha).
		\]

		\part
		Let $R = \Z\left[\sqrt{-11}\right]$ and put 
		$J = \left(3, 1 + \sqrt{-11}\right)_R$.
		Show that $J^2 = \left(9, a + \sqrt{-11}\right)_R$ for some integer $a$.
	\end{parts}

	\question
	\begin{parts}
		\part
		Let $K = \Q\left(\sqrt 3\right)$.
		Let $\mathcal O_3$ be the number ring of $K$.
		Then prove that $\alpha \in \mathcal O_3$ is a unit if and only if
		$
			\left\lvert \operatorname{N}_{K/\Q}(\alpha) \right\rvert = 1
		$.

		\part
		Let $K = \Q(\alpha)$,
		where $\alpha$ is a root of the polynomial $x^3 + 14x + 7$.
		Find the degree $[K: \Q]$ and calculate the trace 
		$
			\operatorname{Tr}_{K/\Q}(\beta)
		$
		for $\beta = \alpha^2 - \alpha - 1$.
	\end{parts}

	\question
	\begin{parts}
		\part
		\begin{subparts}
			\subpart
			Give the ring of integers $\mathcal O_{69}$ of 
			$\Q\left(\sqrt{69}\right)$
			and find its fundamental unit.

			\subpart
			Give all solutions in integers $x, y$, if any, of
			$x^2 - 69y^2 = 4$.
		\end{subparts}

		\part
		Let $R = \Z\left[\sqrt{-74}\right]$.
		Factorise the ideal $\left(13 - \sqrt{-74}\right)_R$ as a product of 
		prime ideals.
		Determine which, if any, of these are principal ideals and show that $R$
		has an ideal class of order $5$.
	\end{parts}

	\question
	\begin{parts}
		\part
		Let $R = \Z\left[\sqrt{-31}\right]$.
		Calculate the norm $\operatorname{N}(I)$ of the ideal in $R$ given by
		\[
			I = \left(2 - \sqrt{-31}, 3 + 2\sqrt{-31}\right)_R
		\]
		and exhibit a fractional ideal $F$ such that $IF = R$.

		\part
		Determine how many solutions there are to the equation
		\[
			x^2 + 2y^2 = 11^{11}
		\]
		such that $x$ is not divisible by $11$.
		(You may use that $\Z\left[\sqrt{-2}\right]$ is a UFD, but you should
		emphasise at which point you are invoking this.)
	\end{parts}

	\question
	Let $R = \mathcal O_{-23}$.
	Consider the ideals $I_1 = \left( 3, -2 + \sqrt{-23} \right)_R$
	and $I_2 = \left( 8, 1 + \sqrt{-23} \right)_R$ in $R$.
	\begin{parts}
		\part
		Do $I_1$ and $I_2$ lie in the same ideal class?

		\part
		Is either one of $I_1$ or $I_2$ principal?
	\end{parts}

	\begin{center}
		\textsc{\textbf{Section B}}
	\end{center}

	\question
	\begin{parts}
		\part
		Let $m \neq n$ be two square-free integers, i.e. they are products
		of distinct primes.
		Prove that $\Q\left(\sqrt m\right) \neq \Q\left(\sqrt n\right)$.

		\part
		Given any integer $n$, let 
		$\omega_n = \exp\left(\sfrac{2\pi i}n\right)$
		denote a primitive $n$-th root of unity.
		The fields $\Q\left(\omega_n\right) = \Q\left[\omega_n\right]$
		are called \emph{cyclotomic} extensions of $\Q$.
		\begin{subparts}
			\subpart
			For any positive integer $n$, prove that
			$
			\omega^2_{2n} = \omega_n
			$
			and
			$
			-\omega^{n+1}_{2n} = \omega_{2n}
			$.

			\subpart
			Using the above relations, or otherwise, prove that
			$
				\Q\left[\omega_n\right]
				= \Q\left[\omega_{2n}\right],
			$
			where $n$ is a positive odd integer.
		\end{subparts}

		\part
		Give an example of a quadratic extension of $\Q$ which is also a
		cyclotomic extension.
	\end{parts}

	\question
	\begin{parts}
		\part
		Show that
		$
		\Z\left[\sqrt{-11}\right]
		$
		contains elements of norm $53$ and $103$, respectively.
		How many integer solutions are there to
		\[
			x^2 + 11y^2 = 2^2 \cdot 53^5 \cdot 103^7?
		\]
		How many integers solutions with $x,y$ positive are there?
		(You may assume that
		$
		\Z\left[\frac{\left(1 + \sqrt{-11}\right)}2\right]
		$
		is a UFD and that the only units in this ring are $\pm 1$.)

		\part
		In this part, you may use that $\Z\left[i\right]$ is a UFD.
		\begin{subparts}
			\subpart
			Factorise $2$ as a product of primes in $\Z\left[i\right]$.

			\subpart
			Prove that equation
			\[
				\left(x + 16i\right) \left(x - 16i\right) = y^3
			\]
			would necessarily imply that $x + 16i$ is a cube in $\Z[i]$.

			\subpart
			Prove that the Diophantine equation
			\[
				x^2 + 256 = y^3
			\]
			has at most finitely many solutions where $x$ and $y$ are integers.
			(Carefully justify your method.)
		\end{subparts}
	\end{parts}

	\question
	Determine the ideal class group of $K = \Q\left(\sqrt{-39}\right)$.
	(You may assume that every ideal class contains an ideal of norm no more
	than $B_K$, where, in the usual notation,
	$B_K = \left(\frac4\pi\right)^t \frac{n!}{n^n} 
	\sqrt{\left\lvert \Delta_K \right\rvert}$.)

	\question
	Let $K = \Q(\theta)$ where $\theta$ is a root of 
	$f(x) = x^3 - 3x - 3$.
	\begin{parts}
		\part
		Find the norm and trace of $\theta^2 + 1$.

		\part
		Find the discriminant 
		$\Delta_K\left(\Z\left[\theta\right]\right)$ 
		and show that $\mathcal O_K = \Z\left[\theta\right]$.

		\part
		Show that the class number of $K$ is $1$.
		(You may use the formula from Question 9.)
	\end{parts}

	\begin{center}
		\textsc{\textbf{Section C}}
	\end{center}

	\question
	For this question, you may assume that, if $\omega$ is a primitive $k$-th
	root of unity, then
	$\left[\Q\left(\omega\right): \Q\right] > 2$, when $k \neq 2,3,4,6$.
	For an algebraic number $\theta$, let $K = \Q\left(\theta\right)$.
	\begin{parts}
		\part
		Let $\theta = \sqrt[4]{11}$.
		\begin{subparts}
			\subpart
			Find the minimal polynomial for $\theta$.
			Using this, or otherwise, find all the conjugates of $\theta$,
			and find all the embeddings $\sigma_i: K \to \C$.

			\subpart
			Use the morphisms $\sigma_i$ to find an embedding of $K$
			into a space $\R^r \times \C^s$.
			What are the integers $r$ and $s$?

			\subpart
			What is the cardinality of the set of units in $K$?
		\end{subparts}

		\part
		Let $\theta = \sqrt d$, where $d$ is a square-free integer.
		\begin{subparts}
			\subpart
			Explicitly compute the group of finite order units in $K$.
			How does this group depend on $d$?

			\subpart
			Find the group structure for the whole group of units of $K$.
		\end{subparts}
	\end{parts}
\end{questions}

\end{document}
