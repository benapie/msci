\paper{2014}
\setcounter{question}{0}

\question
\begin{parts}
    \part
    Compute the gradient, $\nabla f$, of 
    $f(x,y,z) = xy + y^2 + z^2$.
    At the point $(-1,2,3)$,
    in which direction is this function increasing the fastest,
    and what is the directional derivative of $f$ in this direction?

    \part
    Compute the divergence and curl of the vector field
    \[
        \bm A(x,y,z) = xyz \bm e_1 + xy \bm e_2 + x\bm e_3.
    \]
\end{parts}

\question
\begin{parts}
    \part
    Give the definition of the open ball $B_\delta(\bm a)$
    with centre $\bm a \in \R^n$
    and radius $\delta > 0$,
    and define what it means for a subset $S$ of $\R^n$ to be open.

    \part
    Which of the following subsets of $\R^2$ are open?
    In each case, 
    justify your answer in terms of the definition you gave in part (a).
    \begin{subparts}
        \subpart
        $S_1 = \{(x,y): x > 2\}$,

        \subpart
        $S_2 = \{(x,y): x > 2, y = 2\}$,

        \subpart
        $S_3 = \{(x,y): x > 2, y > 2\}$.
    \end{subparts}
\end{parts}

\question
\begin{parts}
    \part
    Write the scalar product
    $\bm a \cdot \bm b$
    of two vectors $\bm a$ and $\bm b$
    in index notation.

    \part 
    If $\bm v = v_1 \bm e_1 + v_2 \bm e_2 + v_3 \bm e_3$
    is a vector field in three dimensions,
    use $\varepsilon_{ijk}$ to write the
    $i$th component of the curl of $\bm v$, 
    $(\nabla \times \bm v)_i$,
    in index notation.

    \part
    If $\bm x = x_1 \bm e_1 + x_2 \bm e_2 + x_3 \bm e_3$
    is the position vector in three dimensions and
    $\bm v(\bm x) = \bm a \times \bm x$
    with $\bm a$ a constant vector,
    compute the curl of $\bm v$ using index notation.
\end{parts}

\question
\begin{parts}
    \part 
    State Fubini's Theorem.

    \part
    Show that Fubini's theorem holds for the following integral,
    by evaluating it in two ways.
    \[
        \iint_R dA,
    \]
    where the integration, $R$, is contained between
    the $y$-axis, $y=x+2$, and $y = x^2$ where $x \geq 0$.
\end{parts}

\question
\begin{parts}
    \part 
    $\bm F = kx \bm e_1 + ky \bm e_2 + kz \bm e_3$
    is a vector field in $\R^3$, where $k$ is a constant.
    Use the Divergence Theorem to show that
    \[
        V = \oint_S \bm F \cdot d\bm A,
    \]
    where $V$ is the volume enclosed by the surface $S$, 
    for an appropriate value of $k$ which you should find.

    \part
    Use the equation above and your value for $k$ from part (a)
    to calculate the volume of a sphere of radius $a$,
    with the equation $x^2 + y^2 + z^2 = a^2$.
\end{parts}

\question
By integrating both sides against an arbitrary test-function,
find the coefficients $a$, $b$, $c$, $\alpha$, $\beta$, and $\gamma$
in the following generalised function identities.
\begin{parts}
    \part $(x^3 + 2x - 1) \, \delta\left( x-\frac12 \right) = a\delta\left( x-\frac12 \right)$,
    \part $e^{2x} \, \delta'(x-1) = b \delta(x-1) + c\delta'(x-1)$,
    \part $\delta(x^2+x-6) = \alpha(\delta(x-\beta) + \delta(x-\gamma))$.
\end{parts}

\question
\begin{parts}
    \part
    Find the unit normal to the plane in $\R^3$
    defined by $f(\bm x) = \sfrac xa + \sfrac yb + \sfrac zc = 1$
    where $a$, $b$, and $c$ are three positive constants.

    \part
    The plane defined in part (a) intersects the $x$, $y$, and $z$
    axes at the points $\bm a = (a,0,0)$, $\bm b$, and $\bm c$
    respectively.
    What are the coordinates of $\bm b$ and $\bm c$?
    Find the area of the triangle $\bm a \bm b \bm c$.
    
    (Hint: recall that for any two vectors $\bm p$ and $\bm q$,
    $\bm p \times \bm q$ has magnitude $\norm{\bm p} \norm{\bm q} \abs{\sin\theta}$,
    where $\theta$ is the angle between $\bm p$ and $\bm q$.)

    \part 
    Without using Stokes' theorem,
    compute the line integral $\oint \bm A \cdot d\bm x$,
    where the closed contour $C$ is the union of the three edges
    of the triangle $\bm a \bm b \bm c$ from part (b),
    and $\bm A(x,y,z) = z\bm e_1 + x\bm e_2 + y\bm e_3$.

    \part
    Using your answers to part (a) and (b) (or otherwise)
    show that your result from part (c) is consistent with Stokes' theorem.
\end{parts}

\question
\begin{parts}
    \part
    Explain what it means for a function
    $f: \R^n \to \R$
    to be differentiable at a point $\bm a$,
    in terms of a vector $\bm v(\bm a)$ and a function $R(\bm h)$
    satisfying
    \[
        f(\bm a + \bm h) - f(\bm a) = \bm h \cdot \bm v(\bm a) + R(\bm h),
    \]
    stating how $R$ must behave as $\bm h \to \bm 0$.
    Using the definition of partial derivatives as limits,
    find a formula for $\bm v(\bm a)$ in terms of the partial derivatives
    of $f$.

    \part 
    What does it mean to say that a function is \emph{continuously differentiable}
    at a point $\bm a$ in $\R^n$?
    State (but do not prove) a theorem relating continuous differentiability
    to differentiability.
    % inverse function theorem

    \part 
    Determine the points in $\R^2$ where the function 
    $f(x,y) = \abs{xy + x + y + 1}$ is
    \begin{subparts}
        \subpart continuously differentiable; and
        \subpart differentiable.
    \end{subparts}
    (Hint: first factorise $f$.)
\end{parts}

\question
\begin{parts}
    \part 
    For any closed surface $S$,
    show that
    \[
        \oint_S (\nabla \times \bm F) \cdot d\bm A = 0
    \]
    \begin{subparts}
        \subpart by Divergence Theorem,

        \subpart by Stokes' Theorem.
    \end{subparts}
    Show all your working.

    \part
    Calculate
    \[
        \int_S (\nabla \times \bm F) \cdot d\bm A
    \]
    for $\bm F = (xz, x^2y^2z, xyz)$,
    when $S$ is the open surface given by the cylinder (with no ends)
    of radius $a$,
    centred on the $x$-axis with equation
    $y^2 + z^2 = a^2$
    and
    $x \in [0,5]$.
\end{parts}

\question
$T(x,t)$ satisfies the heat equation
\[
    \frac{\partial T}{\partial t} = \frac{\partial^2 T}{\partial x^2}, 
\]
with boundary conditions 
$T(0,t) = T(L,t)$, 
$\frac{\partial T}{\partial x}(0,t) = \frac{\partial T}{\partial x}(L,t)$,
and $T(x,0) = e^x$.

Find $T(x,t)$ for $t > 0$ and $x \in (0,L)$.
