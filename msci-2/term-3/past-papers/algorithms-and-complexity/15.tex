\paper{2015}

\question
\begin{parts}
    \part[6]
    For each of the following recursions,
    given $T(n)$ in asymptotic notation.
    Justify your answer.
    \begin{subparts}
        \subpart $T(n) = 2T(\sfrac n2) + \Theta(n)$; and

        \subpart $T(n) = T(n-1) + \Theta(n)$.
    \end{subparts}

    \part[4]
    Describe the two approaches to implementing dynamic programming.

    \part[6]
    Is Strassen's algorithm a divide-and-conquer algorithm?
    Explain the main ideas of Strassen's algorithm 
    and give its running time.

    \part[9]
    Give a brief explanation of why matrix squaring is as difficult
    as matrix multiplication.
    Expand this result to the following statement.
    \begin{center}
        \itshape
        \parbox{0.75\textwidth}{
            For any integer $k \geq 2$,
            computing the $k$-th power of a matrix is as difficult
            as matrix multiplication.
        }
    \end{center}
    In each case, no formal proof is required, and there is no need
    to give the exact formulation of the result.
\end{parts}

\question
\begin{parts}
    \part[2]
    Give pseudocode for the Euclidean algorithm.

    \part[5]
    Apply the Euclidean algorithm to determine
    $\gcd(2015,1066)$.

    \part[4]
    Modify the Euclidean algorithm so that it performs no integer
    division.
    Give psuedocode.

    \part[2]
    Define the rod cutting problem.

    \part[12]
    Give two greedy algorithms for the rod cutting problem.
    For each greedy algorithm, give an instance of the problem 
    of length at most three where the algorithm fails to find
    the optimal solution.
\end{parts}

\question
\begin{parts}
    \part[2]
    The output of Dijkstra's algorithm is two arrays $d$ and $\pi$.
    What do the values of these arrays represent?

    \part[8]
    Compute the arrays $d$ and $\pi$ when Dijkstra's algorithm
    is performed on the weighted directed graph $G$ represented
    by the adjacency matrix below,
    where the source vertex corresponds to the first row and first
    column of the matrix.
    \[
        \begin{pmatrix}
            0 & 6 & 1 & 2 & 5 & 0 & 0 & 0 \\
            0 & 0 & 0 & 0 & 8 & 0 & 4 & 0 \\
            0 & 0 & 0 & 2 & 0 & 3 & 0 & 0 \\
            0 & 1 & 0 & 0 & 0 & 0 & 0 & 0 \\
            0 & 9 & 0 & 3 & 0 & 1 & 9 & 0 \\
            0 & 0 & 0 & 0 & 0 & 0 & 2 & 0 \\
            0 & 0 & 0 & 0 & 0 & 0 & 0 & 3 \\
            0 & 0 & 5 & 0 & 2 & 0 & 0 & 0 \\
        \end{pmatrix}
    \]

    \part
    \begin{subparts}
        \subpart[2]
        Give the definition of the decision problem \textsc{Independent Set}.

        \subpart[2]
        Give the definition of the optimisation problem \textsc{Independent Set}.

        \subpart[3]
        Assume that you have discovered a polynomial-time algorithm $A$
        for the decision version of \textsc{Independent Set}.
        Prove that, using your algorithm $A$, you can solve the optimisation
        version of \textsc{Independent Set} in polynomial time.
    \end{subparts}

    \part[8]
    Describe a polynomial-time reduction from the problem \textsc{Satisfiability}
    to the problem \textsc{3-Satisfiability} and prove its correctness.
\end{parts}

\question
\begin{parts}
    \part
    Let $G = (V,E)$ be a network.
    Assume that for every edge $(u,v) \in E$, we have a non-negative capacity
    $c(u,v) \geq 0$ and, whenever $(u,v) \not\in E$, we have $c(u,v) = 0$.
    Let $f: V \times V \to \R_+$ be a real-valued funciton.
    Let $s,t \in V$ be two distinct nodes of $G$.
    \begin{subparts}
        \subpart[4]
        When do we call $f$ a flow from $s$ to $t$ in the network $G$?

        \subpart[1]
        If $f$ is a flow from $s$ to $t$, what is the total value $\abs f$ of $f$?
    \end{subparts}

    \part[4]
    Let $G = (V,E)$ be a flow network with source $s$ and sink $t$ and let $f$
    be a flow in $G$ from $s$ to $t$.
    What is a cut in $G$?
    Define the flow accross a cut and the capacity of a cut.

    \part[4]
    Consider the following flow network with source $s$ and sink $t$;
    where each edge is market with its capacity.
    \begin{center}
        \begin{tikzpicture}[node distance=1.5cm]
            \tikzstyle{v}=[circle, minimum size=2em,draw,thick]
            % NODES
            \node[v] (s) []                  {$s$}  ;
            \node[v] (1) [above right= of s] {$v_1$};
            \node[v] (2) [below right= of s] {$v_2$};
            \node[v] (3) [right      = of 1] {$v_3$};
            \node[v] (4) [right      = of 2] {$v_4$};
            \node[v] (t) [below right= of 3] {$t$}  ;

            % EDGES
            %% s
            \draw[->] (s) to []             node[above left]  {18} (1);
            \draw[->] (s) to []             node[below left]  {12} (2);
            %% 1
            \draw[->] (1) to [bend left=15] node[right]       {9}  (2);
            \draw[->] (1) to []             node[above]       {10} (3);
            %% 2
            \draw[->] (2) to []             node[below]       {11} (4);
            \draw[->] (2) to [bend left=15] node[left]        {4}  (1);
            %% 3
            \draw[->] (3) to []             node[below right] {6}  (2);
            \draw[->] (3) to []             node[above right] {20}  (t);
            %% 4
            \draw[->] (4) to []             node[left]        {8}  (3);
            \draw[->] (4) to []             node[above left]        {5}  (t);
        \end{tikzpicture}
    \end{center}
    What is the value of a maximum flow in this network?
    Justify your answer using the max-flow min-cut theorem.

    \part[4]
    Give the definition of the decision problem \textsc{Clique}
    and \textsc{$k$-Colouring}.

    \part[8]
    Consider the following decision problem.
    \begin{center}
        \parbox{0.75\textwidth}{
            \textsc{Clique Partition} \\
            \textbf{Instance:} an undirected graph $G = V,E)$
            and an integer $k$. \\
            \textbf{Question:} can we partition $V$ into at most
            $k$ subsets such that each of these subsets is a complete
            graph?
        }
    \end{center}
    Provide a polynomial-time reduction from \textsc{$k$-Colouring}
    to \textsc{Clique Partition}.
\end{parts}
