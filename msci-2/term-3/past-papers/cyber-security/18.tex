\paper{2018}

\question
\begin{parts}
    \part[4]
    Explain the difference between symmetric key and
    asymmetric key cryptograph, and state an example
    application where each would be used.
    \begin{solution}
        Symmetric key cryptography uses a single key to encrypt and decrypt the data.
        Assymetric key cryptography uses two keys: 
        \begin{enumerate}
            \item the public key, used to encrypt the data; and
            \item the private key, used to decrypt the data.
        \end{enumerate} 
        The public key is typically given out to many people who want to send data
        to a particular receiver and the private key is kept secret by the particular
        receiver.
        Private key cryptography may be used by a website that requires your credentials
        to gain priviledges to perform actions.
        Public key cryptography may be used in block cipher (used to encrypt files).
    \end{solution}

    \part[4]
    State four desirable properties of a cryptographic
    hash function.
    \begin{solution}
        Properties of an effective hash function:
        \begin{enumerate}
            \item determinstic;
            \item one-way: not easily invertible;
            \item no collisions: $f(x_1) = f(x_2) \implies x_1 = x_2$; and
            \item avalance effect: a small change in the input causes a large change in the output.
        \end{enumerate}
    \end{solution}

    \part[6]
    In the context of databases,
    explain and give an example of
    \begin{subparts}
        \subpart an obscure query; and
        \begin{solution}
            To make it harder for a system to identify whether a user is requesting data that they
            are not authorised to make, agents can construct overly complex queries.
            For example, instead of
            \begin{center}
                \ttfamily
                SELECT * FROM Students WHERE Sex = "F"
            \end{center}
            can be rewritten as
            \begin{center}
                \ttfamily
                SELECT * FROM Students WHERE       \\
                (IsSenior = "T" AND Sex <> "M") OR \\
                (IsSenior = "F" AND Sex = "F").
            \end{center}
        \end{solution}

        \subpart an inference attack.
        \begin{solution}
            An inference attack is a technique performed by analysing data in order to illegitimately
            gain knownledge about a subject or database.
            Inference attacks occur when agents execute queries that they are authorised to do
            to gain access to information that they are not authorised to have access to.
            For example, if you can search for how much money has been spent by customers in the last month
            and you can see the prices of all the articles on sale, we can deduce possible purchase lists
            of customers.
        \end{solution}
    \end{subparts}
    \emph{(You do not need to write SQL in your answers.)}

    \part[4]
    Explain what \emph{``IP spoofing''}
    is, and explain how you can protect against an
    IP spoofing attack.
    \begin{solution}
        IP spoofing is where we change the IP data in our packet headers to pretend to be someone else.
        There are many ways that we can protect against this,
        one way is to setup an authentication protocol and having encrypted sessions so that
        connections between two machines are authenticated and all data sent between them are
        encrypted and so cannot be intercepted.
    \end{solution}

    \part[7]
    A company specialises in annotating large geographic
    datasets for the oil industry.
    There are 70 exmployees (group A) who have no specialist
    knowledge of the data, and annotate it based on a few
    hours of training.
    There are also 10 specialist employees (group B)
    who have PhDs in Geology and can label the data to
    a high degree of accuracy.

    Unfortunately all 80 exmployees are reading and
    writing over each other data annotations and the
    overall annotation quality is inconsistent.
    You are implementing a system to fix this problem for
    future annotations.

    Would you recommend either the Bell-LaPadula Models or the
    Biba Model for the company?
    List the read and write policies as determined by your
    chosed model for each group accordingly.
    %todo
\end{parts}

