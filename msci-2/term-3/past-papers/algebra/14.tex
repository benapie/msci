\paper{2014}

\question
\begin{parts}
    \part 
    Find $d = \gcd(45,237)$
    and $n, m \in \Z$
    with $d = 45n + 237m$.
    \begin{solution}
        \begin{align*}
            45  &= (0)237 + 45 \\
            237 &= (5)45  + 12 \\
            45  &= (3)12  + 9  \\
            12  &= (1)9   + 3  \\
            9   &= (3)3   + 0;
        \end{align*}
        therefore, $d = 3$.
        We have
        \begin{align*}
            3 &= 12 - (1)9                  \\
              &= 12 - (1)(45 - (3)12)       \\
              &= (4)12 - (1)45              \\
              &= (4)(237 - (5)45) - (1)(45) \\
              &= (4)237 - (21)45
        \end{align*}
        so $n = -21$ and $m = 4$.
    \end{solution}

    \part
    Let $f(x) = x^7 + x^4 + x^3 + \overline 1$,
    $g(x) = \overline2 x^4 + \overline2 x^3 + \overline 1$
    be in $\Z_5[x]$.
    Find $\gcd(f(x),g(x))$ in monic form.
    \begin{solution}
        \begin{align*}
            f(x) &= (\ov3 x^3 + \ov2 x^2 + \ov3 x) g(x) + (\ov3 x^3 + \ov3 x^2 + \ov2 x + \ov1) \\
            g(x) &= (\ov4 x)(\ov3 x^3 + \ov3 x^2 + \ov2 x + \ov1) + (\ov2 x^2 + x) \\
            (\ov3 x^3 + \ov3 x^2 + \ov2 x + \ov1) &= (\ov4 x - \ov2)(\ov2 x^2 + x) + (\ov4 x + 1) \\
            (\ov2 x^2 + x) &= (\ov3 x + \ov2)(\ov4x + 1) + \ov3 \\
            (\ov4x + 1) &= (\ov3x)(\ov 3) + \ov1 \\
            (\ov3) &= (\ov3)(\ov1) + \ov 0.
        \end{align*}
        Therefore $\gcd(f(x),g(x)) = \overline 1$.
        %todo probably an error somewhere
    \end{solution}
\end{parts}

\question
Let $R = \Z[\sqrt{-3}]$.
For an integer $k > 1$ consider the map $\varphi: R \to \Z_k$
given by
\[
    \varphi(a+b\sqrt{-3}) = \overline{a + 4b}.
\]
\begin{parts}
    \part
    Find $k > 1$ such that $\varphi$ is a ring homomorphism.
    \begin{solution}
        $k = 19$ by the definition of a ring homomorphism.
    \end{solution}

    \part 
    Find an element $a + b\sqrt{-3} \in R$ such that
    $\ker\varphi = \langle a + b\sqrt{-3} \rangle$
    for this $k$.
    Show that the element does indeed generate $\ker\varphi$.
\end{parts}

\question
Show that
\[
    \Z_7[x] / \langle x^4 + 1 \rangle \cong
    \Z_7[x] / \langle x^2 + \overline3 x + \overline 1 \rangle
        \times Z_7[x] / \langle x^2 - \overline3 x + \overline 1 \rangle.
\]
You may quote theorems from the lecture to do this, but you should justify their use.

\question
Consider the permutation 
$\sigma = (1\;3)(2\;4\;6)(5\;7\;6\;4)$
in $S_7$.
\begin{parts}
    \part
    Determine the order of $\sigma$. (State carefully any result you use.)
    \begin{solution}
        \[
            \sigma = (1\;3)(2\;4\;5\;7).
        \]
        If $\sigma$ is a product of disjoint $k_i$-cycles,
        then $\abs \sigma = \operatorname{lcm}\{k_i\}$.
        So $\abs\sigma = 4$.
    \end{solution}

    \part
    Does $\sigma$ also lie in $A_7$? (Justify your reasoning.)
    \begin{solution}
        \[
            \sigma = (1\;3)(2\;4)(4\;5)(5\;7),
        \]
        $\sigma$ factorises into an even number of transpositions;
        hence, $\sigma \in A_7$.
    \end{solution}


    \part 
    Either give a permutation that conjugates $\sigma$
    into $\sigma' = (1\;5\;7\;4)(2\;3)$
    or show that $\sigma$ and $\sigma'$ are not conjugate to each
    other in $S_7$.
    \begin{solution}
        Let $g$ be the permutation such that
        $g\sigma g^{-1} = \sigma'$.
        Then
        \begin{align*}
            g\sigma g^{-1} &= \sigma' \\
            g (1\;3)(1\;5\;7\;4) g^{-1} &= \sigma' \\
            g (1\;3) g^{-1} g (1\;5\;7\;4) g^{-1} &= \sigma' \\
            (g(1)\;g(3)) (g(1)\;g(5)\;g(7)\;g(4)) &= (1\;5\;7\;4)(2\;3);
        \end{align*}
        which is satisfied by
        \[
            g = (1\;2)(4\;5\;7).
        \]
    \end{solution}
\end{parts}

\question
\begin{parts}
    \part
    Consider the pair $(S, \circ)$ 
    where $S$ denotes the set of $2\times2$-matrices with entries
    in $\Q$ which ahve determinant an integer power of $2$,
    and where $\circ$ denotes matrix multication.
    Does $(S, \circ)$ form a group? (Justify your answer.)

    \part
    Give a map from $\C \setminus \{0\}$ to 
    $\{x \in \C: \abs x = 1\}$
    that defines a \emph{surjective} group homomorphism.
    (Here $\abs x$ denotes the usual complex norm of $x$.
    You can assume that the former is a group under multiplication and that
    the latter is a subgroup thereof.)
    
    Furthermore, give the kernel of this homomorphism.
\end{parts}

\question
\begin{parts}
    \part
    Determine the centre $Z(D_4)$
    of the dihedral group $D_4$.

    \part
    Determine all the groups of order 289, and of order 202,
    up to isomorphism. (State carefully any result you use.)
\end{parts}

\question 
Let $R$ be a commutative ring with identity,
and let $I$ be an ideal in $R$.
\begin{parts}
    \part
    Show that $I[x]$, the polynomiaal ring with 
    coefficients in $I$,
    is an ideal in $R[x]$.

    \part
    Show that $R[x]/I[x]$ is isomorphic to $(R/I)[x]$.

    \part
    Show that if $I$ is a maximal ideal in $R$,
    then $I[x]$ is a prime ideal in $R[x]$,
    but not a maximal ideal in $R[x]$.
\end{parts}
(Carefully state any results you use.)

\question
\begin{parts}
    \part
    List all irreducible polynomials of degree $2$ in $\Z_2[x]$.

    \part
    Show that $f(x) = x^4 + x^3 + x^2 + x + \overline 1 \in \Z_2[x]$
    is irreducible.

    \part
    Show that
    $\varphi: \Z_2[x] \to \Z_2[x]/\langle f(x) \rangle$
    given by
    \[
        \varphi(g(x)) = \overline{g(x) \cdot g(x)}
    \]
    is a ring homomorphism.

    \part
    Show that $\ker\varphi = \langle f(x) \rangle$, and $\varphi$
    induces an automorphism
    \[
        \overline\varphi: \Z_2[x]/\langle f(x) \rangle \to
        \Z_2[x]/\langle f(z) \rangle.
    \]
    which is different from the identity.
\end{parts}

\question
Let the group $G$ act on a set $X$, and let $x \in X$.
\begin{parts}
    \part
    Define the $G$-orbit of $x$ and the stabiliser of $x$ in $G$,
    and state the Orbit-Stabiliser Theorem.
    \begin{solution}
        Let $G$ be a group acting on a set $X$.
        \begin{align*}
            \Orb(x)  &= \{gx: g \in G\} \\
            \Stab(x) &= \{g : gx = x \}.
        \end{align*}
        Let $x \in X$.
        The orbit-stabiliser theorem states that
        there exists a bijection
        \[
            \beta: \Orb(x) \to \{g\Stab(x): g \in G\},
            \qquad \beta(gx) = g\Stab(x).
        \]
    \end{solution}

    \part
    Show that, for a finite group $G$, 
    the order of a $G$-orbit divides the order of $G$.
    \begin{solution}
        $\Stab(x)$ is a subgroup of $G$;
        hence, by Lagrange's theorem $\abs G = m \abs{\Stab(x)}$
        where $m$ is the number of left cosets of $\Stab(x)$ in $G$.
        As $\beta$ is a bijection from an orbit to the set of
        left cosets, $\abs{\Orb(x)} = m$.
    \end{solution}

    \part 
    Show that $G_x$ is a subgroup of $G$.
    Futhermore, given $y \in G(x)$ show that $G_y$ is conjugate to $G_x$.
    \begin{solution}
        Let $a,b \in Stab(x)$.
        We have $ax = x$, $bx = x$.
        Then
        \[
            ab(x) = a(bx) = ax = x
        \]
        so $ab \in \Stab(x)$.
        We have $e \in \Stab(x)$ and
        \begin{align*}
            ax &=       x
            x  &= a^{-1}x
        \end{align*}
        so $a^{-1} \in \Stab(x)$.
        Therefore,
        $\Stab(x)$ is a subgroup of $G$.
        Now, let $y \in Orb(x)$.
        Then $y = gx$ for some $g \in G$.
        Consider the mapping $\varphi: \Stab(x) \to \Stab(y)$
        defined by $\varphi(h) = ghg^{-1}$.
        Then
        \[
            ghg^{-1}y = ghx = gx = y
        \]
        therefore $\varphi(h) \in \Stab(y)$,
        so $\varphi$ is well defined.
        Now $\varphi^{-1}(h) = g^{-1}hg$,
        hence, $\varphi$ is surjective.
        Let $h, h' \in \Stab(x)$.
        Then if $\varphi(h) = \varphi(h')$ we have
        \begin{align*}
            ghg^{-1} &= gh'g^{-1} \\
            h &= h',
        \end{align*}
        so $\varphi$ is injective.
        Also
        \[
            \varphi(hh') = ghh'g^{-1} = ghg^{-1} gh'g^{-1}
        \]
        so $\varphi$ is an isomorphism;
        therefore, $\Stab(y)$ is conjugate to $\Stab(x)$.
    \end{solution}

    \part
    Let $X = M_2(\R)$, the set of $2 \times 2$ matrices with real entries,
    and let $G = \operatorname{GL}_2(\R)$ be the multiplicative
    group of matrices in $M_2(\R)$ with non-zero determinant.
    \begin{subparts}
        \subpart
        Show that the formula $A * M = AMA^{t}$
        where $A \in G$, $M \in X$, and $A^t$ is the
        transpose of $A$, defines a group action of $G$ on $X$.
        \begin{solution}
            Let $B \in G$.
            Then
            \[
                A * (B * M) 
                = A * (BMB^t)
                = ABMB^tA^t
                = (AB) M (AB)^t
                = (AB) * M
            \]
            and
            \[
                I * A = IAI^t = A
            \]
            therefore the formula defines a group action.
        \end{solution}
        
        \subpart 
        Let $M \in X$ and let $H = G_M$, the stabiliser of $M$ in $G$
        (with the action define above).
        For $A \in G$, show that $G_{AMA^t} = AHA^{-1}$.
        \begin{solution}
            We want to show that $\Stab(A*M)$ is conjugate to $\Stab(M)$.
            We know that $A*M \in \Orb(M)$, so we can use part (c).
            Therefore, $\beta: \Stab(M) \to \Stab(A*M)$
            defined by $\beta(B) = ABA^{-1}$
            is an isomorphism and so $\Stab(M)$ is conjugate to $\Stab(A*M)$.
            Therefore
            \[
                AHA^{-1} = \Stab(AMA^t).
            \]
        \end{solution}
    \end{subparts}
\end{parts}

\question
\begin{parts}
    \part 
    Determine the possible cycle shapes for $A_6$.
    \begin{solution}
        \[
            (2,2),
            (3,3),
            (3),
            (4,2),
            (2,4),
            (5).
        \]
    \end{solution}

    \part
    Determine the conjugacy classes of $D_6$.
    \begin{solution}
        \begin{align*}
            \Orb(r)
            &= \{r, r^5\} \\
            \Orb(r^2)
            &= \{r^2, r^4\} \\
            \Orb(r^3)
            &= \{r^3\} \\
            \Orb(s)
            &= \{s,r^2s,r^4s\} \\
            \Orb(rs)
            &= \{rs, r^3s, r^5s\}.
        \end{align*}
    \end{solution}

    \part
    For each of $n \in \{3,4,5,6\}$ decide whether
    \begin{subparts}
        \subpart $A_n$ is isomorphic to a subgroup of $D_n$;
        
        \subpart $D_n$ is isomorphic to a subgroup of $A_n$; or

        \subpart neither $A_n$ is isomorphic to a subgroup of $D_n$
            not $D_n$ is isomorphic to a subgroup of $A_n$.
    \end{subparts}

    (Justify your answers.)
    \begin{solution}
        \begin{description}
            \item[$n = 3$] \hfill \\
                $\abs{A_3} = 3$ and $\abs{D_3} = 6$
                so ii cannot hold.
                Let $G = \langle r \rangle$.
                It is clear that $G \leq D_3$.
                Let $\varphi: A_3 \to G$ be defined as:
                \begin{align*}
                    \varphi(e)   &= e \\
                    \varphi(r)   &= (1\;2\;3) \\
                    \varphi(r^2) &= (1\;3\;2).
                \end{align*}
                $\varphi$ is certainly a homomorphism
                and is surjective;
                hence,
                it is an isomorphism and so i holds.

            \item[$n = 4$] \hfill \\
                $\abs{A_4} = 12$ and $\abs{D_4} = 8$.
                Therefore, neither are isomorphic a subgroup of the other.
                So iii holds.

            \item[$n = 5$] \hfill \\
                $\abs{A_5} = 60$, $\abs{D_5} = 10$.
                So i cannot hold.
                But indeed ii does hold.
                We define $\varphi: D_5 \to A_5$ as follows
                \begin{align*}
                    \varphi(r) &= (1\;2\;3\;4\;5) \\
                    \varphi(s) &= (2\;5)(3\;4).
                \end{align*}
                Then
                \begin{align*}
                    \varphi(r)\varphi(s)
                    &= (1\;2\;3\;4\;5)(2\;5)(3\;4) \\
                    &= (1\;2)(4\;5) \\
                    &= (4\;3)(5\;2)(5\;4\;3\;2\;1) \\
                    &= (\varphi(r)\varphi(s))^{-1}, \\
                    \varphi(r)^5 &= e, \\
                    \varphi(s)^2 &= e;
                \end{align*}
                therefore, $\varphi$ describes an isomorphism between
                $D_5$ and $G = \langle \varphi(r), \varphi(s) \rangle \leq A_5$.

            \item[$n = 6$]
                $\abs{A_6} = 360$ and $\abs{D_6} = 12$.
                Clearly i cannot hold.
                Assume that ii holds.
                Let $G$ be the subgroup that $D_6$ is isomorphic to.
                We will come to a contradiction by analysing the
                orders of elements in $D_6$ and $A_6$.
                $D_6$ has $2$ elements of order 6, namely $r$ and $r^5$;
                however, all permutations in $S_6$ of order $6$
                are not even; hence, there are no elements of order
                $6$ in $G$, a contradiction.
                Therefore iii holds.
        \end{description}
    \end{solution}
    
    \part 
    Show that $D_3 \times D_5$ cannot be isomorphic to $A_5$.
    \begin{solution}
        The highest element order in $A_5$ is 5;
        however $(r, r^4)$ has order $15$ in $D_3 \times D_5$;
        therefore,
        \[
            D_3 \times D_5 \not\cong A_5.
        \]
    \end{solution}
\end{parts}
