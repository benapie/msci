\paper{2014}
\setcounter{question}{0}
\question 
\begin{parts}
    \part
    Define what is means for a complex valued function 
    $f(z)$
    defined on the complex plane to be, at the point $z_0$,
    \begin{subparts}
        \subpart
        complex differentiable; and
        \begin{solution}
            $f(z)$ is complex differentiable at a point $z_0$ if
            \[
                \lim_{h \to 0}
                \left(
                    \frac{f(z_0 + h) - f(z_0)}{h}
                \right)
                = f'(z_0)
            \]
            exists.
        \end{solution}
        
        \subpart 
        holomorphic.
        \begin{solution}
            $f(z)$ is holomorphic at a point $z_0$ if
            there exists $\varepsilon > 0$ such that
            for all $z \in B_\varepsilon(z_0)$, 
            $f(z)$ is complex differentiable.
        \end{solution}
    \end{subparts}

    \part 
    State the Cauchy-Riemann equations,
    and use them to determine where the function
    \[
        f(x + iy) = x\sin(x + iy)
    \]
    is complex-differentiable and where it is holomorphic.
    State carefully any results from the module that you use.
    \begin{solution}
        Let $f(z) = f(x + iy) = u(x,y) + iv(x,y)$
        where $u, v: \R^2 \to \R$.
        The Cauchy-Riemann equations are as follows:
        \begin{align*}
            u_x(x,y) &=  v_y(x,y) \\
            u_y(x,y) &= -v_x(x,y).
        \end{align*}
        Now we will calculate $u$ and $v$ for $x \sin(x+iy)$.
        \begin{align*}
            x \sin(x+iy)
            &= x
            \left(
                \sin(x)\cos(iy) + \sin(iy)\cos(x)
            \right) \\
            &= x\sin(x)\cosh(y) + i\left( x\sinh(y)\cos(x) \right);
        \end{align*}
        so, we have
        \begin{align*}
            u(x,y)   &= x\sin(x)\cosh(y) \\
            v(x,y)   &= x\sinh(y)\cos(x).
        \end{align*}
        Calculating partial derivatives:
        \begin{align*}
            u_x(x,y) &= \sin(x)\cosh(y) + x \cos(x)\cosh(y) \\
            u_y(x,y) &= x\sin(x)\sinh(y) \\
            v_x(x,y) &= \sinh(y)\cos(x) - x\sinh(y)\sin(x) \\
            v_y(x,y) &= x\cosh(y)\cos(x).
        \end{align*}
        Now we will apply the Cauchy-Riemann equations.
        \begin{description}
            \item[$u_x = v_y$]
                \begin{align*}
                    \sin(x)\cosh(y) + x \cos(x)\cosh(y)
                    &= x\cosh(y)\cos(x) \\
                    \sin(x) \cosh(y)
                    &= 0
                \end{align*}
                and as $\cosh(y) \geq 1$ for all $y \in \R$,
                then $\sin(x) = 0$ which is satisfied only by
                $x = \pi \Z$.

            \item[$u_y = -v_x$]
                \begin{align*}
                    x\sin(x)\sinh(y)
                    &= -\sinh(y)\cos(x) + x\sinh(y)\sin(x) \\
                    \sinh(y)\cos(x) &= 0.
                \end{align*}
                So either $\sinh(y) = 0$ or $\cos(x) = 0$.
                If $\sinh(y) = 0$ then $y = 0$.
                If $\cos(x)$ then $x = \pi \Z + \frac\pi2$.
        \end{description}
        However, $\pi \Z \neq \pi \Z +  \frac\pi2$.
        Therefore $f$ is complex differentiable 
        for $x = \pi\Z$ and $y = 0$
        and holomorphic nowhere.
    \end{solution}
\end{parts}

\question
State the Weierstrass $M$-test,
and use it to prove that if $\rho$ is a positive real number
then the series
\[
    \sum^{\infty}_{n=1} \frac{n}{e^{nx}} 
\]
is uniformly convergent on $\{x+iy\in\C : x \geq \rho\}$.
\begin{solution}
    \hfill \\
    \textbf{Theorem} (Weierstrass $M$-test).
    \textit{
        \hspace{-7pt} Let $\{f_n\}_{n=1}^\infty$ 
        be a sequence of functions $f_n: X \to \C$
        on some set $X$.
        Then if there exists a sequence of postive numbers $M_n$
        satisfying
        \[
            \abs{f(x)} \leq M_n
        \]
        for all $x \in X$ and $M_n$ satisfies
        \[
            \sum^{\infty}_{n=1} M_n < \infty
        \]
        then $ \sum^{\infty}_{n=1} f_n(x) $
        uniformly converges to some limit function $f: X \to \C$.
    }
    Now, looking at this summation we
    let $f_n(x) = ne^{-nx}$.
    Then we have
    \[
        \abs{f_n(x)} = ne^{-nx} \leq ne^{-n\rho}.
    \]
    So we take $M_n = ne^{-n\rho}$.
    Now we use the ratio test:
    \[
        \abs{
            \frac{M_{n+1}}{M_n}
        }
        = \frac{n+1}{n} e^{-\rho}
        \to e^{-\rho}
    \]
    as $n \to \infty$. 
    As $e^{-\rho} < 1$, 
    we have that $M_n$ is absolutely convergent.
    Therefore, we have satisfied all the constraints of the
    Weierstrass $M$-test
    and thus $\sum_{n=1}^\infty ne^{-nx}$ uniformly converges
    to some limit function $f$ on
    $\{x + iy \in \C : x \geq \rho\}$.

    \emph{
        \hspace{-7pt} Interestingly, it is not much more hard work to calculate
        this function.
        If we let
        $g(x) = ne^{-nx}$
        and consider that
        $g(x) = - \frac{d}{dx} \left( e^{-nx} \right)$
        and then use the geometric series, 
        we easily find that the series converges to
        \[
            f(x) = \frac{e^x}{(e^x - 1)^2}.
        \]
    }
\end{solution}

\question
Let $\gamma$ be the circle centred at the origin with radius $R > 0$.
Find (stating any results you use):
\begin{parts}
    \part
    $\displaystyle \oint_\gamma z \,dz$;
    \begin{solution}
        $\gamma$ is a simple closed contour and $f(z) = z$
        is a holomorphic function; therefore
        $\oint_\gamma z \,dz = 0$.
        (Cauchy's theorem for simple closed contours).
    \end{solution}

    \part
    $\displaystyle \oint_\gamma \overline z \,dz$; and
    \begin{solution}
        Now $\overline z$ is not a holomorphic,
        so we can't use Cauchy's theorem.
        It also isn't meromorphic so now residues here.
        We just have to solve it directly, which isn't too
        much work.
        We let $\gamma(\theta) = Re^{i\theta}$, and so
        $\gamma'(\theta) = Ri e^{i\theta}$.
        Then
        \begin{align*}
            \oint_\gamma \overline z \,dz
            &= \int_0^{2\pi} 
                \overline{\gamma(\theta)} 
                \gamma'(\theta) \,d\theta \\
            &= R^2 i\int_0^{2\pi} 
                e^{-i\theta}
                e^{i\theta} \,d\theta \\
            &= R^2 i \int_0^{2\pi} d\theta \\
            &= 2R^2 \pi i.
        \end{align*}
    \end{solution}

    \part
    $\displaystyle \oint_\gamma \abs z \,dz$.
    \begin{solution}
        We define $\gamma$ as we did in the last part.
        \begin{align*}
            \oint_\gamma \abs z \,dz
            &= \int_0^{2\pi} 
                \abs{\gamma(\theta)} 
                \gamma'(\theta) \,d\theta \\
            &= \int_0^{2\pi} 
                (R)
                (Ri e^{i\pi})
                \,d\theta \\
            &= R^2 i \int_0^{2\pi} e^{i\pi} \,d\theta \\
            &= 0.
        \end{align*}
        
    \end{solution}
\end{parts}

\question
By using the substitution $z = e^{2i\theta}$, or otherwise, calculate
\[
    \int_{\theta = 0}^\pi \frac{1}{5 + \cos^2\theta} \,d\theta.
\]

\question
Let $f(z)$ be a holomorphic function defined for all $z \in \C$.
Suppose that $\abs{f(z)}$ tends to $0$ as $\abs z$ tends to infinity.
Find $f(z)$, stating any result that you use.
\begin{solution}
    As $\abs{f(z)} \to 0$ as $\abs z \to \infty$, we have
    for all $\varepsilon > 0$, there exists a $x > 0$
    such that
    \[
            \abs z > x \quad \implies \quad \abs{f(z)} < \varepsilon.
    \]
    Let $\varepsilon = 1$ and $N > 0$ such that it satisfies the above.
    $f(z)$ is defined on $\C$, so it is defined on $\overline B_N(0)$
    and so the supremum is defined on this set.
    Then we have
    \[
        \abs{f(z)} \leq \max\{1, \sup_{\overline B_N(0)} \abs f\}
    \]
    and so $f$ is bounded.
    As $f$ is entire and bounded, by Liouville's theorem, $f$ is constant.
    As $\abs{f(z)} = 0$ implies $f(z) = 0$, we have that $f(z) = 0$.
\end{solution}

\question
\begin{parts}
    \part 
    Show that the M\"obius transformation
    \[
        g(z) = \frac{-z+i}{z+i}
    \]
    maps the upper half plane $\Im(z) > 0$ onto the open unit disc
    of complex nubers of modulus less than $1$.

    \part
    Show that if $a$ and $b$ are complex numbers with
    $\overline ab$ having negative imaginary part, then
    \[
        f(z) = \frac{az + b}{\overline az + \overline b} 
    \]
    is a M\"obius transformation which maps the upper half plane onto 
    the open unit disc.

    \part 
    Assuming that all M\"obius transformations
    from the upper half plane onto the open disc may be
    written in the form given in (b), use the M\"obius
    transformation $g(z)$ defined in (a) to show that any 
    M\"obius transformation from the open unit disc onto itself
    may be written in the form
    \[
        h(z) = e^{i\theta} \left( \frac{z - z_0}{\overline{z_0} z - 1} \right)
    \]
    for some fixed $\theta$, $z_0$, with $\abs{z_0} < 1$.

    \part 
    For each complex number $z_0$ of modulus less than $1$,
    find a M\"obius transformation from the open unit disc onto
    itself mapping $z_0$ to $0$ and whose derivative at 
    $z_0$ is a positive real number.
\end{parts}

\question
\begin{parts}
    \part
    Find the poles and residues of
    \[
        \frac{1}{\cosh(z)}.
    \]
    \begin{solution}
        \begin{align*}
            \cosh(z)
            &= \cosh(x + iy) \\
            &= \cosh\left( i\left( y - ix \right) \right) \\
            &= \cos \left( y - ix \right) \\
            &= \cos(y) \cosh(x) + i\sin(y) \sinh(x).
        \end{align*}
        $ \frac{1}{\cosh(z)} $ 
        has a pole if $y = \pi\Z + \frac\pi2$ and $x=0$.
        All of these poles are simple as
        \[
            \frac{1}{\cosh' \left( i \left( \pi k + \frac\pi2 \right) \right) }
            = \frac{1}{i \sin \left( \pi k + \frac\pi2 \right) } 
            = i(-1)^{k+1}
        \]
        for $k \in \Z$ and, infact, these are the residues:
        \[
            \Res_{z=i\left(\pi k + \frac\pi2\right)} = i(-1)^{k+1}.
        \]
    \end{solution}

    \part
    Let $R>0$ be given.
    Let $\gamma_R$ be the straight line from $z = R$
    to $z = R + i\pi$;
    that is,
    $\gamma_R(t) = R + it$ with $t \in [0,\pi]$.
    Show that
    \[
        \abs{\int_{\gamma_R} \frac{1}{\cosh(z)} \,dz}
        \leq \frac{\pi}{\sinh(R)}.
    \]
    State any result that you use.
    \begin{solution}
        This is a clear application of the estimation lemma.
        Calculating the length of $\gamma$:
        \begin{align*}
            L(\gamma_R)
            &= \int_0^\pi \abs{\gamma_R'(t)} \,dt \\
            &= \int_0^\pi dt \\
            &= \pi.
        \end{align*}
        Now looking for an upper bound on $\abs f$:
        \begin{align*}
            \abs{f(\gamma_R(t))}
            &=    \frac{2}{\abs{e^{R+it} + e^{-R-it}}} \\
            &\leq \frac{2}{\abs{e^{R+it}} - \abs{e^{-R-it}}}
            \tag{reverse triangle inequality} \\
            &=    \frac{2}{e^R - e^{-R}} \\
            &=    \frac{1}{\sinh(R)}.
        \end{align*}
        Now applying the estimation lemma:
        \begin{align*}
            \abs{\int_{\gamma_R} \frac{1}{\cosh(z)} \,dz}
            &\leq L(\gamma_R) \, \sup_{\gamma_R} \abs{ \frac{1}{\cosh(z)} } \\
            &\leq L(\gamma_R) \left( \frac{1}{\sinh(R)} \right) \\ 
            &=    \frac{\pi}{\sinh(R)}.
        \end{align*}
    \end{solution}

    \part 
    Let $\delta_R$ be the straight line from $z = -R + i\pi$ to $-R$.
    Show that
    \[
        \lim_{R \to \infty}
        \int_{\gamma_R} \frac{1}{\cosh(z)} \,dz
        = \lim_{R \to \infty} 
        \int_{\gamma_R} \frac{1}{\cosh(z)} \,dz
    \]
    \begin{solution}
        Following similar steps to above we get
        \[
            \int_{\delta_R} \frac{1}{\cosh(z)} \,dz
            \leq \frac{\pi}{\sinh(R)} 
        \]
        and taking limits of these gives us the equality shown
        in the question.
    \end{solution}

    \part By integrating around the rectangle with vertices 
    $R$, $R + i\pi$, $-R + i\pi$, $-R$, or otherwise, find
    \[
        \infint \frac{1}{\cosh(x)} \,dx.
    \]
    \begin{solution}
        First I will start with an \emph{or otherwise} approach,
        which is quite clear when considering the common hyperbolic identities.
        It is know that $\cosh^2(x) = \sinh^2(x) + 1$.
        Consider the substitution $u = \sinh(x)$, 
        then $du = \cosh(x) \,dx$ and we have
        \[
            \int \frac{1}{\cosh(x)} \,dx
            = \int \frac{1}{u^2 + 1} \,du
            = \arctan(u) + c
            = \arctan(\sinh(x)) + c
        \]
        for some constant $c \in \R$.
        So
        \begin{align*}
            \infint \frac{1}{\cosh(x)} \,dx
            &= \lim_{x \to \infty}
            \left(
                \arctan(\sinh(x)) - \arctan(\sinh(-x))
            \right) \\
            &= \lim_{x \to \infty} 
            \left(
                \arctan(x) - \arctan(-x)
            \right) \\
            &= \pi.
        \end{align*}
        Nothing complex.
    \end{solution}
    \begin{solution}
        Now for the expected method.
        We define two contours:
        \begin{align*}
            \varphi_R(t)          &= R(2t-1) \\
            \overline\varphi_R(t) &= R(1-2t) + \pi i.
        \end{align*}
        So now we have $\alpha_R = \gamma_R + \varphi_R + \delta_R + \overline\varphi_R$.
        Using Cauchy's residue theorem, we get
        \[
            \int_\alpha \frac{1}{\cosh(z)} \,dz
            = 2\pi i \left( \Res_{z=\frac{i\pi}2} \left(\frac{1}{\cosh(z)}\right) \right) 
            = 2\pi.
        \]
        Looking into the integrals over $\varphi_R$ and $\overline\varphi_R$:
        \begin{align*}
            \int_{\varphi_R} \frac{1}{\cosh(z)} \,dz
            &= \int_0^1 \frac{2R}{\cosh(R(2t-1))} \, dt \\
            &= \int_{-R}^{R} \frac{1}{\cosh(u)} \,du \\
            \int_{\overline\varphi_R} \frac{1}{\cosh(z)} \,dz
            &= \int_0^1 \frac{-2R}{\cosh(R(1-2t) + i\pi)} \, dt \\
            &= \int_0^1 \frac{2R}{\cosh(R(1-2t))} \, dt \\
            &= -\int_{R}^{-R} \frac{1}{\cosh(u)} \,du \\
            &= \int_{-R}^{R} \frac{1}{\cosh(u)} \,du.
        \end{align*}
        Now
        \begin{align*}
            \int_{\alpha_R} \frac{1}{\cosh(z)} \,dz
            &= \int_{\gamma_R + \varphi_R + \delta_R + \overline\varphi_R}
                \frac{1}{\cosh(z)} \,dz \\
            &= \int_{\varphi_R} \frac{1}{\cosh(z)} \,dz
                + \int_{\overline\varphi_R} \frac{1}{\cosh(z)} \,dz \\
            &= 2\int_{-R}^R \frac{1}{\cosh(z)} \,dz \\
            &= 2\pi.
        \end{align*}
        And so finally we get
        \[
            \infint \frac{1}{\cosh(x)} \,dx
            = \lim_{R \to \infty} \int_{-R}^{R} \frac{1}{\cosh(u)} \,du
            = \pi.
        \]
    \end{solution}
\end{parts}

\question
\begin{parts}
    \part
    Find the poles and residues of
    \[
        \frac{1}{z^4 + 4}.
    \]
    \begin{solution}
        The poles are $1 + i$, $-1 + i$, $-1 - i$, $1 - i$
        and the residues, calculated with the cover up rule,
        are given by the formula
        \[
            \Res_{z=z_0} \left( \frac{1}{z^4 + 4} \right) = -\frac{1}{16}z_0.
        \]
    \end{solution}

    \part 
    Show that $z^4 + z + 4$ has one root in each quadrant
    and that all roots lie inside the circle of radius $2$ centred
    at the origin.
    
    State any result you use and give full details of any estimates involved.
    \begin{solution}
        We claim that for all $R \geq 0$, $R < R^4 + 4$:
        \begin{enumerate}
            \item 
                if $R \in [0,1)$,
                $R < 4 < R^4 + 4$; and

            \item
                if $R \in [1,\infty)$,
                $R < R^4 < R^4 + 4$;
        \end{enumerate}
        hence, we have proved our claim.
        
        Let $f(z) = z^4 + z + 4$ and $g(z) = z^4 + 4$.
        Now,
        we consider the contour
        $\alpha = \gamma + \tilde\varphi + \varphi$
        where
        \begin{enumerate}
            \item $\gamma(t) = e^{it}$, $t \in [0,\pi]$;
            \item $\tilde\varphi(t) = 2i(1 - t)$, $t \in [0,1]$; and
            \item $\varphi(t) = 2t$, $t \in [0,1]$.
        \end{enumerate}
        We will show that $\abs{f(z) - g(z)} < \abs{g(z)}$
        for each of these.
        \begin{enumerate}
            \item 
                If $z \in \gamma$, 
                then $z = 2e^{it}$ for some $t \in [0,\pi]$
                so:
                \[
                    \abs{f(z) - g(z)}
                    = \abs{z} 
                    = 2 
                    < 15 
                    = \abs{ \left( 2e^{it} \right)^4 } - \abs 1 
                    < \abs{ \left( 2e^{it} \right)^4 + 1 } 
                    = \abs{g(z)}.
                \]

            \item
                If $z \in \tilde\varphi(t)$ 
                then $z = ix$ for some $x \in \R$
                so
                \[
                    \abs{f(z) - g(z)}
                    = \abs{z} 
                    = \abs{ix} 
                    = \abs{x} 
                    = x 
                    < x^4 + 4 
                    = (ix)^4 + 4
                    = \abs{g(z)}.
                \]

            \item 
                If $z \in \varphi(t)$
                then $z = x$ for some $x \in \R$
                so
                \[
                    \abs{f(z) - g(z)}
                    = \abs z
                    = \abs x
                    < \abs{x^4 + 4}
                    = \abs{g(z)}.
                \]
        \end{enumerate}
        So we have shown for all $z \in \alpha$,
        $\abs{f(z) - g(z)} < \abs{g(z)}$;
        therefore, $f$ and $g$ have the same number of zeros inside
        $D_\alpha^{\text{int}}$.
        From above, we know that $\frac1{g(z)}$ has 1 pole
        at $1 + i \in D_\alpha^{\text{int}}$;
        hence $g(z)$ has a zero at $1 + i$.
        Therefore $f(z)$ has a root in the upper right quadrant.
        Now, roots to complex polynomials come in conjugate pairs;
        therefore, $f(z)$ has a root in the upper left quadrant too.
        Now we know that the two remaining roots
        lie either on the imaginary axis, the real axis, or the lower half plane.
        \begin{enumerate}
            \item 
                If $z = ix$ for $x \in \R$
                then
                \[
                    z^4 + z + 4
                    = x^4 + ix + 4
                    \neq 0.
                \]

            \item 
                If $z = x$ for $x \in \R$
                then
                \[
                    z^4 + z + 4
                    = x^4 + x + 4
                    > 0.
                \]
        \end{enumerate}
        So we know that the other two roots do \emph{not}
        lie on the imaginary or real axis;
        hence, they lie in the lower half plane.
        Again, roots come in conjugate pairs;
        therefore, one root must be in the 
        lower left quadrant
        and the othe rin the
        lower right quadrant.
    \end{solution}

    \part 
    Let $\gamma$ be the $D$-shaped contour made up of
    (a) the arc of the real axis between $-2$ and $2$ and
    (b) the arc of the circle centred at $0$ of radius $2$
    lying in the upper half plane. 
    Find
    \[
        \oint_\gamma \frac{4z^3 + 1}{z^4 + z + 4} \,dz.
    \]
    State any result that you use.
    \begin{solution}
        This is a simple application of the argument principle.
        For a meromorphic function $f(z)$ and a simple closed contour $\gamma$
        \[
            \frac{1}{2\pi i} \int_\gamma \frac{f'(z)}{f(z)} \,dz
            = Z_f - P_f
        \]
        where $Z_f$ is the number of zeros $f$ has in $D_\gamma^\text{int}$
        and $P_f$ is the number of poles $f$ has in $D_\gamma^\text{int}$
        (note that both of these number are counting multiplicities).
        
        We have already done most of the ground work to answer this question.
        We know that it has no poles, but we need to find the zeros of
        $z^4 + z + 4$.
        Well, infact, we see that there are $2$ zeroes from part (b);
        therefore,
        $Z_f - P_f = 2$.
        So we have
        \[
            \int_\gamma \frac{4z^3 + 1}{z^4+z+4} \,dz
            = 4\pi i.
        \]
    \end{solution}
\end{parts}
