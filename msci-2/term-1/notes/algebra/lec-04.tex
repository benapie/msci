\chapter{Integral domains}
\lecture{4}{17/10}

\begin{definition}[Integral domain]
    A commutative ring $R$ with at least two elements ($0 \neq 1$) is called an \textbf{integral domain} if it has no non-zero zero divisor. Alternatively, where $R$ satisfies
    \[ \;\forall\; a, b \in R: a \cdot b = 0 \implies a = 0 \;\text{or}\; b = 0. \]
\end{definition}

\begin{example}
    Any field $F$ has no non-zero zero divisor. Hence any field $F$ where $0 \neq 1$ is an integral domain. So clearly $\Q$, $\C$, and $\R$ are all integral domain over typical operations. $\Z$ is also an integral domain.
\end{example}

\begin{example}
    Consider
    \[ \Z/3 = \{ \bar 0, \bar 1,\bar 2 \}. \]
    Let's consider the multiplication table.
    \begin{center}
        \begin{tabular}{cccc}
            \toprule
            $+$ & $\bar 0$ & $\bar 1$ & $\bar 2$ \\
            \midrule
            $\bar 0$ & $\bar 0$ & $\bar 0$ & $\bar 0$ \\
            $\bar 1$ & $\bar 0$ & $\bar 1$ & $\bar 2$ \\
            $\bar 2$ & $\bar 0$ & $\bar 2$ & $\bar 1$ \\
            \bottomrule
        \end{tabular}
    \end{center}
    It is clear that $\Z/3$ has no non-zero divisor; hence, $\Z/3$ is an integral domain. In fact, $\Z/3$ is a field.
\end{example}

\begin{example}
    Consider
    \[ \Z/4 = \{ \bar 0, \bar 1, \bar 2, \bar3 \}. \]
    This is not an integral domain as $\bar 2 \cdot \bar 2 = \bar 0$.
\end{example}

\begin{example}
    Let $R$ be an integral domain. Let
    \[ R[x] = \left\{ \sum_{i = 0}^n a_i x_i : a_i \in R \right\}. \]
    $R[x]$ is an integral domain. Let's prove it. Let $f(x) = a_0 + a_1x + \ldots + a_nx^n$ and $g(x) = b_0 + b_1x + \ldots + b_mx^m$ such that $a_n \neq 0$ and $b_m \neq 0$. 
    \[ f(x)g(x) = a_0b_0 + \ldots + a_nb_m x^{n + m} \neq 0 \]
    hence an integral domain.
\end{example}

\section{The group of units in a ring}

\begin{definition}[Unit]
    Let $R$ be a ring. An element $u \in R$ is called a \textbf{unit} if there exists $u^{-1}$ such that
    \[ uu^{-1} = u^{-1}u = 1. \]
    Given a ring $R$,
    \[ R^\times = \{ u \in R : u \;\text{is a unit}\} \]
    is the set of all units in $R$.
\end{definition}

\begin{proposition}
    Let $R$ be a ring. $R^\times$ is a group under multiplication ($\cdot$).
\end{proposition}

\begin{proof}
    Let $a, b, c \in R^\times$. Then
    \begin{enumerate}
        \item (closure under $\cdot$)
            \[ (ab)(b^{-1}a^{-1}) = abb^{-1}a^{-1} = a \cdot 1 \cdot a^{-1} = aa^{-1} = 1; \]
        \item (multiplicative identity) $1 \in R^\times$;
        \item (associativity)
            \[ (ab)c = a(bc); \]
        \item (inverse)
            \[ aa^{-1} = 1 = a^{-1}a \implies a^{-1} \in R^\times. \]
    \end{enumerate}
\end{proof}

\begin{example}
    \begin{enumerate}
        \item $R = \Z/4 = \{ \bar 0, \bar 1, \bar 2, \bar 3 \}$. $\R^\times = \{ \bar 1, \bar 3 \}$.
        \item $\Z = \{0, \pm 1, \pm 2, \ldots \}$. $\Z^\times = \{ \pm 1 \}$.
        \item $M_2(\R)^\times = \operatorname{GL}_2(\R) = \text{all invertible matrices}$.
    \end{enumerate}
\end{example}

\begin{example}
    Let $R$ be an integral domain. Consider
    \begin{align*}
        f(x) &= a_0 + a_1x + \ldots + a_nx^n \in R[x] \\
        g(x) &= b_0 + b_1x + \ldots + b_nx^n \in R[x]
    \end{align*}
where $f(x) \neq 0$ and $g(x) \neq 0$. $f(x)g(x) \neq 1$ for $n \geq 1$, hence if $f(x)$ is a unit then $n = 0$. In fact, $R[x]^\times = R^\times$.
\end{example}

\begin{proposition}
    Let $\bar x \in \Z/n$. Then $\bar x$ is a unit if and only if 
    \[ \gcd{(x, n)} = 1. \]
\end{proposition}

\begin{proof}
    \begin{description}
        \item[$\implies$] Let $\bar x \in \Z/n^\times$. Then there exists some $y \in \Z/n^\times$ such that $\bar x \cdot \bar y = 1$. Then
            \begin{align*}
                \overline{x \cdot y} &= \bar 1 \\
                \overline{xy - 1} &= \bar 0 \\
                n &\mid xy - 1 \\
                xy - 1 &= kn, \quad k \in \Z \\
                1 &= xy - kn \\
                \gcd{(x, n)} &\mid 1 \\
                \gcd{(x, n)} &= 1§
            \end{align*}

        \item[$\impliedby$] Via Euclid's algorithm we have
            \begin{align*}
                1 &= xy + nz \\
                n &\mid xy - 1 \\
                \overline{xy - 1} &= \bar 0 \\
                \overline{xy} &= \bar 1 \\
                \bar x &\in \Z/n^\times.
            \end{align*}
    \end{description}
\end{proof}
