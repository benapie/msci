\section{Prime and maximal ideals}
\lecture{18}{12/12}

We have two main types of ideals of importance in ring theory.

\begin{definition}[Prime ideal]
    An ideal $I$ of a commutative ring $R$ is \textbf{prime} if
    \begin{enumerate}
        \item for all $a, b \in R$ such that $ab \in I$ we have
            $a \in I$ or $b \in I$; and
        \item $I \neq R$.
    \end{enumerate}
\end{definition}

\begin{example}
    \begin{enumerate}
        \item The set of real numbers is a prime ideal of $\Z$.

        \item $I = (2, x)$ is a prime ideal of $\Z[x]$.

        \item Let $p \in \Z$. 
            Then $p$ is prime iff $(p) = p\Z$ is a non-zero prime ideal.

        \item $(2)$ is \emph{not} a prime ideal of $\Z[i]$ as
            $(1 - i)(1 + i) = 2 \in (2)$
            \emph{but}
            $1 \pm i \not \in (2)$.
    \end{enumerate}
\end{example}

There is a similarity between prime elements and prime ideals, 
which may be seen in the above examples.
If $x \in R$ is a prime element,
then $I = (x) \subset R$ is a prime ideal.
Conversely, if $I = (x) \subset R$ is a prime ideal
then $x \in R$ is a prime element;
however, there exists prime ideals that are not principal.
Therefore, prime ideals are more general then prime elements.
We also see that
\[ x \in (a) \iff (x) \subset (a) \iff x \mid a. \]

\begin{definition}[Maximal ideal]
    An ideal $I$ of a ring $R$ is \textbf{maximal} if
    \begin{enumerate}
        \item the only ideals of $R$ containing $I$ are $R$ and $I$; and
        \item $I \neq R$.
    \end{enumerate}
\end{definition}

\begin{example}
    \begin{enumerate}
        \item If $F$ is a field, the only maximal ideal is $\{0\}$.

        \item In the ring $\Z$, 
            the maximal ideals are the principal ideals generated by a prime number.
    \end{enumerate}
\end{example}

\begin{proposition}[]
    Let $I$ be an ideal of a commutative ring $R$. Then
    \begin{enumerate}
        \item $I$ is prime iff $R/I$ is an integral domain; and
        \item $I$ is maximal iff $R/I$ is a field.
    \end{enumerate}
\end{proposition}

\begin{proof}
    \begin{enumerate}
        \item First let us prove that $I$ is prime 
            $\implies$ $R/I$ is an integral domain.
            Assume $I$ is prime.
            Let $\overline a, \overline b \in R/I$.
            Then if $\overline{ab} = \overline{0}$ then $ab \in I$.
            So either $a \in I$ or $b \in I$.
            Hence $\overline a = \overline 0$ or $\overline b = \overline 0$.
            Therefore, $R/I$ is an integral domain.
            Now, let us prove that $R/I$ is an integral domain 
            $\implies$ $I$ is prime.
            Assume that $R/I$ is an integral domain.
            Now let $a, b \in R$ such that $ab \in I$.
            Then $\overline{ab} = \overline 0$.
            So $\overline a = \overline 0$ or $\overline b = \overline 0$.
            Therefore $a \in I$ or $b \in I$;
            and so $I$ is prime.

        \item First, we will prove that $I$ being maximal
            $\implies$ $R/I$ field.
            Assume that $I$ is maximal. 
            Then we have a representation $x$ for every non-zero element in $R/I$
            such that $x \not\in I$ (clearly).
            So we have
            \[ (I, x) = R = (1). \]
            It is clearly closed under addition and multiplication by $r \in R$,
            so there exists $y \in R$ and $m \in I$ such that
            \[ xy + m = 1. \]
            Hence $\overline{xy} = \overline 1$
            and so $R/I$ is a field.
            Now to prove the other direction.
            Assume that $R/I$ is a field and let $x \in R$ such that $x \not \in I$
            (that is, $\overline x \neq \overline 0$).
            Then there exists $\overline y \in R/I$ such that
            \[ \overline{xy} = \overline 1. \]
            Hence
            \[ xy = 1 + m \]
            for some $m \in I$.
            So we have $1 \in (x, I)$.
            Hence, $(x, I) = R$.
            Now let $J$ be some other ideal such that $I \subset J$ and $I \neq J$.
            Now, there must exist some $x \in J$ such that $x \not \in I$. Hence
            \[ I \subset (I, x) \subset J \]
            but as $R = (I, x)$, we have that $J = R$ and so $I$ is maximal.
    \end{enumerate}
\end{proof}

\begin{lemma}[]
    If an ideal $R$ of a ring $R$ is maximal, then it is prime.
\end{lemma}

\begin{proof}
    $I \subset R$ is maximal $\iff$ 
    $R/I$ is a field $\iff$ 
    $R/I$ is an integral domain $\iff$
    $I$ is prime.
\end{proof}

\begin{example}
    \begin{enumerate}
        \item Recall that $R[x]/(x^2 + 1) \cong \C$.
            As $\C$ is a field, $(x^2 + 1)$ is maximal.

        \item Let $R = \Z[i]$ and $I = (2)$.
            $(\overline{1+i})(\overline{1+i}) = \overline 0$,
            so $R/I$ is \emph{not} an integral domain, so $I$ is not maximal.

        \item Let $R = \Z[i]$ and $I = (2-i)$.
            It can be shown that $R/I \cong \Z/5$.
            As $5$ is prime, $\Z/5$ is a field and so $I$ is a maximal ideal of $R$.
    \end{enumerate}
\end{example}
