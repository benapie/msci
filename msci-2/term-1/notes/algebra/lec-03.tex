\section{Subrings}
\lecture{3}{15/10}

\begin{definition}[Subring]
    A subring $S$ in a ring $R$ is a subset $S \subset R$ such that
    \begin{enumerate}
        \item (identity) $0, 1 \in S$;
        \item (closure under addition) $\;\forall\; a, b \in S$, $a + b \in S$;
        \item (closure under multiplication) $\;\forall\; a, b \in S$, $ab \in S$; and
        \item (addition inverse) $\;\forall\; a \in S$, $-a \in S$.
    \end{enumerate}
\end{definition}

\begin{remark}
    Any subring $S \subset R$ is equipped with the same $+$ and $\cdot$ operators of $R$.
\end{remark}

\begin{example}
    It can be easily shown that $\Q$ is a subring of $\Q[x]$.
\end{example}

\begin{example}
    Consider
    \[ \Z[\sqrt 2] = \{ a + b\sqrt 2 : a, b \in \Z \} \subset \R. \]
    This may not be immediately obvious, but this is a subring of $\R$.
\end{example}

\begin{example}
    Consider 
    \[ R = \Z / 6 = \{ \bar 0, \bar 1, \bar 2, \bar 3, \bar 4, \bar 5 \} \]
    and
    \[ S = \{ \bar 0, \bar 2, \bar 4 \} \subset R. \]
    As $\bar 1 \not \in S$, it is not a subring of $R$. This is the only criteria it fails. However, $S$ can be made into a ring itself with $\bar 0$ as the additive identity and $\bar 4$ as the multiplicative identity. Even though this is the case, it is not a subring of $R$ as it does not share the same multiplicative identity.
\end{example}

\begin{remark}
    The above example illustrates an interesting point that rings can be contained with other rings and not be considered a subring. 
\end{remark}

\section{Fields}

We saw earlier that $\Q$, $\R$, and $\C$ are fields but we did not formally define the concept.

\begin{definition}[Field] 
    A ring $R$ is called a \textbf{field} if
    \begin{enumerate}
        \item $R$ is a commutative ring;
        \item $\;\forall\; a \in R$ where $a \neq 0$ there exists $a^{-1} \in R$ such that $aa^{-1} = 1$.
    \end{enumerate}
\end{definition}

\begin{example}
    Consider the commutative ring $\Q$. For all $a, b \in \Q$ where $a, b \neq 0$ we have that
    \[ \frac ab \cdot \frac ba = 1 \] 
    and hence
    \[ \frac ab = \left(\frac ba\right)^{-1}; \]
    therefore, $\Q$ is a field.
\end{example}

Looking at the example above, we use the notation 
\[ \frac ab = ab^{-1} \] for $a, b \in F$ where $F$ is a field and $b \neq 0$. This operation is called \textbf{division} (surely introducted before).

\begin{definition}[Zero divisor]
    An element $a \in R$ of a ring $R$ is called a \textbf{zero divisor} if there exists $b \in R$ where $b \neq 0$ such that 
    \[ ab = 0. \]
\end{definition}

\begin{example}
    $\bar 2 \in \Z/6$ and $\bar 2 \cdot \bar 3 = \bar 6 = \bar 0$ hence $\bar 2$ is a zero divisor of $\Z / 6$. In fact, $\bar 3$ is too.
\end{example}

\begin{example}
    Consider
    \[ A = \begin{pmatrix} 0 & 1 \\ 0 & 0 \end{pmatrix} \in M_2(\Q). \]
    As
    \[ A^2 = \begin{pmatrix} 0 & 0 \\ 0 & 0 \end{pmatrix} \] 
    $A$ is a zero divisor of $M_2(\Q)$.
\end{example}

\begin{example}
    For all rings $R$, $0 \in R$ is a zero divisor as long as $0 \neq 1$.
\end{example}

\begin{example}
    Let $F$ be a field and $a \in F$ be a zero divisor. Let $b \in F$ where $b \neq 0$. Then
    \begin{align*}
        ab &= 0 \\
        (ab)b^{-1} &= 0 \cdot b^{-1} \\
        a \cdot 1 &= a = 0.
    \end{align*}
\end{example}
