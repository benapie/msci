\lecture{16}{10/12}

\begin{proof}
    This theorem is equivalent to saying that
    $F[x]/(f(x))$
    is a vector space over $F$ with basis
    \[ B = \{ \bar 1, \bar x, \bar x^2, \ldots, \bar x^{n -1} \} \]
    where $\deg f = n$.
    It is clear that $F[x]/(f(x))$ as it is an abelian group
    with scalar multiplication given by
    \[ \alpha \cdot (g(x)) = \alpha g(x) \]
    for $\alpha \in F$ and $g(x) \in F[x]$.
    Now we must show that $B$ spans $F[x]/(f(x))$.
    By long division, for any
    $g(x) \in F[x]$
    with
    $\deg g \geq 0$ we have
    \[ g(x) = q(x)f(x) + r(x) \]
    where $\deg r < n$;
    hence,
    \[ \overline{g(x)} = \overline{r(x)} \implies 
    g(x) - r(x) \in (f(x)) \]
    as $B$ spans all polynomials in $\bar x$ up to degree $n - 1$
    it will contain $\overline{r(x)}$ and so $\overline{g(x)}$.
    If
    $\sum_{i = 0}^{n - 1} a_i \bar x^i = \bar 0$
    then
    $\sum_{i = 0}^{n - 1} a_i x^i \in (f(x))$
    and so $f(x)$ divides
    $\sum_{i = 0}^{n - 1} a_i x_i$
    hence $\deg f = n \leq n - 1$;
    a contradiction. Therefore, $a_0 = a_1 = \ldots = 0$
    and so $B$ is a basis.
\end{proof}

\begin{example}
    Consider $\Q[x]$ with ideal $I = (x^2 + x + 1)$.
    By the above theorem
    \[ \Q[x]/(f(x)) = \{ \overline{r(x)} = \deg{(r(x))} \leq 1 \}. \]
    Then, for example, 
    $\overline{x^2 + x + 1} = \overline 0$
    then
    $\overline x^2 = \overline{-x-1}$.
    Suppose $p(x) = x^4 - 3x^2 + 2$. 
    We have that
    \[ \bar x^4 = 
    (\bar x^2)^2 = 
    (\overline{-x-1})^2 = 
    \overline{x^2 + 2x + 1} =
    \overline{x^2 + x + 1} + \overline x = \overline x. \]
    It can easily be shown from that 
    $4x + 5$ 
    is the coset representation of 
    $x^4 - 3x^2 + 2$
    as $\overline{p(x)} = \overline{4x + 5}$.
\end{example}

\begin{example}
    Let $R = \Z/3[x]$ and $I = (x^4 + x + 1)$. 
    Then $R/I$ is a $\Z/3$ vector space over 
    $\{\bar 1, \bar x, \bar x^2, \bar x^3\}$.
    Hence
    \[ R / I = 
    \left\{ \sum_{i = 0}^3 \overline{a_ix^i} : a_i \in \Z/3 \right\}. 
    \]
\end{example}

\begin{remark}
    Consider the quotient ring $R/I$. 
    $R/I$ can be commutative or non-commutative. 
    If $R$ is commutative, so is $R/I$.
    $R/R$ is isomorphic to the zero ring,
    which is commutative.
    On the other side of things, $R/0$ is isomorphic to $R$ (clearly).
\end{remark}

\begin{example}[Quotient map]
    Let $R$ be a ring and $I$ be an ideal.
    We have that the map $f: R \to R/I$ such that
    \[ f(r) = \bar r = r + I. \]
    By construction we have that
    \[ \ker f = I \qquad \text{and} \qquad \im f =  R/I. \]
    Hence, 
    we have proved that all ideals are the kernal of a homomorphism.
\end{example}

\begin{example}
    Let $R = \Z[i]$.
    Consider $I = (2)$.
    \begin{enumerate}
        \item Show that $R/I$ has exactly 4 elements.
        \item Give the tables for addition and multiplication in $R/I$.
        \item Is $R/I$ isomorphic to $\Z/2 \times \Z/2$ or $\Z/4$ as a ring?
    \end{enumerate}
\end{example}

\begin{solution}
    \begin{enumerate}
        \item Let $\alpha = a + bi \in \Z[i]$.
            We have that 
            \[ a = 2k + a' \qquad \text{and} \qquad b = 2l + b' \]
            where $a', b' \in \{0, 1\}$.
            Then
            \[ \alpha = 
                (2k + a') + (2l + b')i = 
                (a' + b'i) + 2(k + li). \]
            Therefore, $\bar \alpha = (a' + b'i) \in R/I$.
            Hence, there at at most $4$ elements in $R / I$.
            Now we just need to show that they are distinct. 
            We assume that 
            $\overline{a' + b'i} = \overline{c' + d'i}$
            where $a', b', c', d' \in \{0,1\}$.
            So 
            \begin{align*}
                \overline{(a' + b'i) - (c' + d'i)} &=   \overline 0 \\
                \overline{(a' - c') + (b' - d')i}  &=   \overline 0 \\
                \overline{(a' - c') + (b' - d')i}  &\in I = (2)     \\
                (a' - c') + (b' - d')i             &= 2(e + fi)
            \end{align*}
            for some $e, f \in \Z$.
            However, 
            $a' - c' \in \{0, \pm 1\}$
            so $a' = c'$.
            Similarly,
            $b' = d'$.

        \item
            \begin{center}
                \begin{tabular}{ccccc}
                    \toprule
                    $+$ & $\bar 0$ & $\bar 1$ & $\bar i$ & $\overline{1 + i}$ \\
                    \midrule
                    $\overline 0$ & $\overline 0$ & $\overline 1$ & $\overline i$ & $\overline{1 + i}$ \\
                    $\overline{1}$ & $\overline{1}$ & $\overline{0}$ & $\overline{1 + i}$ & $\overline{i}$ \\
                    $\overline{i}$ & $\overline{i}$ & $\overline{1 + i}$ & $\overline{0}$ & $\overline{1}$ \\
                    $\overline{1 + i}$ & $\overline{1 + i}$ & $\overline{i}$ & $\overline{1}$ & $\overline{0}$ \\
                    \bottomrule
                \end{tabular}
                \begin{tabular}{ccccc}
                    \toprule
                    $\cdot$ & $\overline{0}$ & $\overline{1}$ & $\overline{i}$ & $\overline{1 + i}$ \\
                    \midrule
                    $\overline{0}$ & $\overline{0}$ & $\overline{0}$ & $\overline{0}$ & $\overline{0}$ \\
                    $\overline{1}$ & $\overline{0}$ & $\overline{1}$ & $\overline{i}$ & $\overline{1 + i}$ \\
                    $\overline{i}$ & $\overline{0}$ & $\overline{i}$ & $\overline{1}$ & $\overline{1 + i}$ \\
                    $\overline{1 + i}$ & $\overline{0}$ & $\overline{1 + i}$ & $\overline{1 + i}$ & $\overline{0}$ \\
                    \bottomrule
                \end{tabular}
            \end{center}

        \item $x \in R/I \implies \overline{x + x} = \overline{0}$
            (which is preserved by a ring isomorphism)
            and in $\Z/4$ we have that $\bar 1 + \bar 1 = \bar 2 \neq \bar 0$.
            Hence, $R/I$ is not isomorphic to $\Z/4$.
            In $\Z/2 \times \Z/2$ we have that $\overline x^2 = \overline x$
            which is preserved by a ring isomorphism which is clearly not
            held in $R/I$; therefore, 
            $R/I$ is also not ismorphic to $\Z/2 \times \Z/2$.
    \end{enumerate}
\end{solution}
