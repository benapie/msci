\lecture{17}{11/12}

\begin{theorem}[First isomorphism theorem]
    Let $\varphi: R \to S$ be a ring homomorphism.
    Then the map
    \[ \bar \varphi: R/\ker\varphi \to \im\varphi \]
    defined by
    \[ \bar x \to \varphi(x) \]
    (that is, $\overline\varphi(\bar x) = \varphi(x)$)
    is a well-defined isomorphism.
    That is,
    \[ R / \ker\varphi \cong \im\varphi. \]
\end{theorem}

\begin{proof}
    First, we must show that $\bar \varphi$ is well-defined.
    Let $x, x' \in R$ such that $\bar x = \overline{x'}$.
    Then $x - x' \in \ker \varphi$ and so
    \[ \varphi(x - x') = 0. \]
    As $\varphi$ is a homomorphism, we have that
    $\varphi(x) = \varphi(x')$
    and so
    \[ \overline \varphi(\overline x) = \overline \varphi(\overline{x'}); \]
    hence $\bar\varphi$ is well-defined.
    It is clear that $\bar\varphi$ is a homomorphism as
    \[ \overline \varphi(\overline x + \overline y) =
        \overline \varphi(\overline{x + y}) =
        \varphi(x + y) =
        \varphi(x) + \varphi(y) =
        \overline\varphi(\overline x) + \overline\varphi(\overline y).
    \]
    Now, we must show that it is bijective and thus a isomorphism.
    The kernal of $\overline\varphi$ is zero as
    \[ \overline\varphi(\overline x) = 0 \iff
        \varphi(x) = 0 \iff
        x \in \ker\varphi \iff
        \bar x = \bar 0 \]
    and so $\overline\varphi$ is injective.
    $\overline\varphi$ is surjective as for all 
    $y \in \im\varphi$ 
    there exists an $x \in R$ such that
    $\varphi(x) = y$;
    therefore,
    \[ \overline\varphi(\overline x) = y \]
    as required.
\end{proof}

\begin{example}
    Let $\varphi: \R[x] \to \C$ be the evaluation map
    \[ \varphi(f(x)) = f(i) \]
    where $i^2 = -1$.
    If $i$ is a root, then so is its conjugate $-i$
    (this is a property of real polynomials).
    Hence, if $f(i) = 0$ then $f(x)$ has a factor $x^2 + 1$, and so
    \[ \ker\varphi = (x^2 + 1). \]
    Since $\varphi$ is surjective, the first isomorphism theorem yields
    \[ \R[x]/(x^2 + 1) \cong \C; \]
    we could have \emph{defined} $\C$ to be $\R/(x^2 + 1)$!
\end{example}
