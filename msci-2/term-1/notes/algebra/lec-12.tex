\chapter{Ideals and quotient rings}
\lecture{12}{14/11}

\begin{definition}[Ideal]
    A subset $I$ of a ring $R$ is called an \textbf{ideal} if it is closed under addition and for every $r \in R$ and $x \in I$ we have $rx \in I$ and $xr \in I$.
\end{definition}

We can think of ideals as \emph{black holes}. Once we are in an ideal, we cannot escape it by addition within the ideal or even multiplication with elements outside the ideal. We can also define \textbf{left ideals} which are closed under addition and closed only under left multiplications by elements in $R$ and similarly for \textbf{right ideals}. When $R$ is commutative, all of the notions coincide (ofcourse). 

\begin{remark}
    Note that if $I$ is an ideal, then $x \in I \implies (-1)x \in I$. Thus, ideals are \emph{almost} subrings, except that it usually does not contain the identity $1 \in R$.
\end{remark}

\begin{example}
    \begin{enumerate}
        \item For any $n \in \Z$, the set $(n) = n\Z = \{nk : k \in \Z\}$ is an ideal.

        \item Let $R$ be a commutative ring. Let $a \in R$. Then
            \[ (a) := \{ra : r \in R\} \]
            is a set which is closed under addition (clearly). For any $x \in R$, we have $x \cdot ra = xra \in (a)$ (so is an ideal). Infact, we call this the \textbf{principle ideal} generated by $a$.

        \item Similarly, let $R$ be commutative and let $a_1, \ldots, a_n \in R$. Then 
            \[ (a_1, \ldots, a_n) := \{ r_1a_1 + \ldots + r_na_n : r_i \in R \} \]
            is an ideal of $R$. We say that $(a_1, \ldots, a_n)$ is generated by elements $a_1, \ldots, a_n$. Thus a principle ideal is generated by a single element. It is clear to see that an ideal with contain its generator (consider $1 \in R$).

        \item Let $R$ be a ring. Let $I \subset R$ be an ideal and suppose that $I$ contains a unit $u \in R^\times$. Then $u^{-1} u = 1 \in I$ and so $r \cdot 1 = r \in r$ for any $r \in R$. Thus $R \subset I$ and so $R = I$.
            
            We can go further with this. If $F$ is a field, then any non-zero element is a unit. So an ideal in $F$ is either $0$ or $F$ itself. 
    \end{enumerate}
\end{example}

\begin{lemma}[]
    Let $R$ be a commutative ring. Then the ideals $I_1 = (a_1, \ldots, a_m)$ and $I_2 = (b_1, \ldots, b_n)$ are equal if and only if $a_1,\ldots, a_m \in I_2$ and $b_1, \ldots, b_n \in I_1$.
\end{lemma}

\begin{proof}
    It is clear to see that if $I_1 = I_2$, then $a_1, \ldots, a_m \in I_1 = I_2$ and similarly $b_1, \ldots, b_n \in I_2 = I_1$. To prove the other way we look at the definitions of our ideals:
    \[ I_1 = \{ r_1a_1 + \ldots + r_m a_m : r_i \in R \} \]
    and
    \[ I_2 = \{ r_1b_1 + \ldots + r_n b_n : r_i \in R \}. \]
    As $a_1, \ldots, a_m \in I_2$, then for all $r_1, \ldots, r_m \in R$, $a_1r_1, \ldots, a_mr_m \in I_2$. Using the fact that $I_2$ is closed under addition we see that $a_1r_1 + \ldots + a_m r_m \in I_2$. Thus $I_1 \subset I_2$. We can use this argument to prove the other statement, concluding $I_2 \subset I_1 \implies I_1 = I_2$.
\end{proof}

\begin{example}
    Let $R = \Z[\sqrt{-5}]$. Prove that $(1 - \sqrt{-5}, 2) = (1 + \sqrt{-5}, 2)$.
\end{example}

\begin{solution}
    From the last Lemma, it is enough to show that all of the generators exist in the ideal.
    \[ 1 - \sqrt{-5} = (-1)(1 + \sqrt{-5}) + (1)(2) \in (1 + \sqrt{-5}, 2). \]
    Similarly,
    \[ 1 + \sqrt{-5} = (1)(2) + (-1)(1 - \sqrt{-5}) \in (1 - \sqrt{-5}, 2) \]
    and so $(1 - \sqrt{-5}, 2) = (1 + \sqrt{-5}, 2)$.
\end{solution}

Ideals can also be seen as kernels of some homomorphism.

\begin{lemma}
    Let $R$ be a ring and let $I \subset R$ be a subset. Then the following are equivalent:
    \begin{enumerate}
        \item $I$ is an ideal;
        \item $I = \Ker{f}$ for some homomorphism $f: R \to S$.
    \end{enumerate}
\end{lemma}

\begin{proof}
    Assume that $I = \Ker{f}$ for some $f$ (homomorphism) and ket $x, y \in I$ and $r \in R$. Then $f(x + y) = f(x) + f(y) = 0$. So $x + y \in I$. Moreover, $f(rx) = f(r) f(x) = 0$ so $rx \in I$ (and similarly for $xr \in I$. We will only be able to prove the converse once we have developed a notoin of quotients of rings modulo ideals.
\end{proof}

\begin{example}
    \begin{enumerate}
        \item For any ring $R$ we have the trivial ideals $0 := \{ 0 \}$ and $R$ itself. The ideal $0$ is the kernel of the identity homomorphism and $R$ is the kernel of the zero map.

        \item $n\Z$ is an ideal of $Z$ for any $n \in \Z$. It is the kernal of the reduction map $\Z \to \Z/n$, $a \mapsto \bar a$.

        \item Consider the \emph{evaluation map} $\phi: \Q[x] \to \Q$ given by $\phi(f(x)) = f(1)$. It is easy to check that $\phi$ is a homomorphism. It is surjective because for any $a \in \Q$ the constant polynomial $a \in \Q[x]$ maps to $a$ under $\phi$. The kernel of $\phi$ is
            \[ \Ker\phi = \{ f(x) \in \Q[x] : f(1) = 0\} = \{(x - 1)g(x) : g(x) \in \Q[x]\};\]
            hence, $\Ker\phi$ is the principle ideal $(x - 1)$ in $\Q[x]$.
    \end{enumerate}
\end{example}
