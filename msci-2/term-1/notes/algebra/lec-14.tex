\lecture{14}{21/11}

\begin{definition}[Coset]
    Let $I \subset R$ be an ideal in the ring $R$. Then given any $x \in R$ we define the \textbf{coset} of $x$ to be the set
    \[ \bar x \coloneqq x + I \coloneqq \{ x + r: r \in I \} \subset R. \]
    $x$ is said to be a \textbf{representative} of $x + I$.
\end{definition}

The following lemam says that distinct cosets are either disjoint or equal.

\begin{lemma}
    Let $x, y \in R$. Then
    \[ x + I = y + I \iff x + I \cap y + I \neq \emptyset \iff x - y \in I. \]
\end{lemma}

\begin{proof}
    The first $\implies$ is obvious. 
    To prove the second $\implies$, if $x + I \cap y + I \neq \emptyset$ then this means there exists $r_1, r_2 \in I$ such that
    \[ x + r_1 = y + r_2 \implies x - y = r_2 - r_1 \in I. \]
    Now for the last (circular) $\implies$, we know that $x - y \in I$. 
    So $x - y = r'$ for $r' \in I$. 
    So $x = y + r'$.
    Therefore
    \[ x + I = \{ x + r: r \in R \} = \{ y + r' + r: r, r' \in R\} \subset y + I; \]
    a similar result can be obtained for $y + I$ to show that $x + I = y + I$.
\end{proof}

We are now going to (roughly) define our quotient ring. We define $R/I$ to be the set of all distinct cosets of $R$ by $I$, that is
\[ R/I \coloneqq \{ \bar x: x \in R \} = \{ x + I: x \in R \}. \]
We can now define addition and multiplication on the cosets by
\begin{align*}
    (x + I) + (y + I) &\coloneqq (x + y) + I \\
    (x + I)   (y + I) &\coloneqq xy + I.
\end{align*}

Initially, it is not obvious that this is well-defined. 
That is, independent of representatives. 
For addition to be well defined we need
\[ (x + y) + I = (x' + y') + I \]
where $x, y, x', y' \in R$, where $x, x'$ are representatives for $x + I$ and $y, y'$ are representatives for $y + I$.
Recall that we have $x + I = x' + I$ means that $x - x' \in I$, so similarly $y - y' \in I$. 
Since $I$ is an ideal, it is closed under addition so 
\[ x - x' + y - y' = (x + y) - (x' + y') \in I \]
and so
\[ (x + y) + I = (x' + y') + I(x + y) + I = (x' + y') + I. \]
Similarly, for multiplication we have that $x - x' \in I$ and $y - y' \in I$. 
As $I$ is closed under multiplication for elements in $R$,
\[ (x - x')y \in I \qquad (y - y')x' \in I \]
and by adding these together we get
\[ xy - x'y' \in I \]
as required.

\begin{definition}[Quotient]
    Let $R$ be a ring and $I \subset R$ be an ideal. 
    Then the ring $R/I$ is called the \textbf{quotient} of $R$ by $I$, or $R \bmod I$.
    Its elements $x + I$ for $x \in R$ are called \textbf{residue classes} ($\bmod \, I$), and are sometimes denoted $\bar x$.
\end{definition}

Now lets look at an example to try and make this look a little bit less abstract.

\begin{example}
    Let $R = \Z[i]$ and $I = (2 - i)$. Let's work out what the quotient $R / I$ looks like. First, we need to find representatives of its elements. Every element in $R / I$ is of the form
    \[ a + bi + (2 - i), \qquad a, b \in \Z \]
    but when are two such elements equal in $R / I$? 
    We have
    \[ \overline{a + bi} = \overline{c + di} \]
    if and only if 
    \[ 2 - i \mid (a + bi) - (c + di). \]
    So, for example, $2 - i + (2 - i) = 0 + (2 - i)$ (as $2 - i \mid 2 - i - 0$), so
    \[ \overline{2 - i} = \overline 0 \]
    and so
    \[ \overline 2 = \overline i. \]
    Hence
    \[ \overline{a + bi} = \overline{a} + \overline{2b} = \overline{a + 2b} \]
    and so every element in $R/I$ has a representative in $\Z$. 
    We can restrict the distinct equivalence classes $\bmod \, I$ further. 
    Namely, squaring both sides of $\overline 2 = \overline i$ we get
    \[ \overline 4 = \overline{-1} \]
    and so $\overline 5 = \overline 0$.
    Thus, the equivalence classes are only different $\bmod 5$, that is, we have at most five elements in $R/I$:
    \[ 0 + I, 1 + I, 2 + I, 3 + I, 4 + I. \]
    Now, is it possible to reduce this further? 
    We assume that $a, b \in \Z$ are such that $\overline a = \overline b$. 
    That is, 
    \[ a - b \in (2 - i) \implies a - b = (x + iy)(2 - i) \]
    for some $x, y \in \Z$.
    This is equivalent to 
    \[ a - b = 2x + y + (2y - x)i \]
    and so
    \[ 2x + y = a - b, \quad 2y - x = 0. \]
    These equations imply that $5y = a - b$, that is $a = b \bmod 5$; so our representatives are indeed distinct.
\end{example}
