\lecture{21}{9/12}

Recall that to check whether a matrix $H$ is positive definite 
we can calculate all the eigenvalues $\lambda_i$ of $H$.
If all $\lambda_i > 0$, then $H$ is positive definite.

Also, $H$ is positive definite if the determinant of all upper-left submatrices are positive.

Although the two above methods work, 
it is recommended that you use the following test.

\begin{proposition}[Second derivative test for functions of two variables]
    Suppose $f(x, y)$ is a function of two variable with 
    continuous second derivatives.
    Let
    \[ H =
        \begin{pmatrix}
            f_{xx} & f_{xy} \\
            f_{xy} & f_{xy} \\
        \end{pmatrix} \]
    and $(x_0, y_0)$ be a critical point (so $\nabla f = \bm 0$).
    Define 
    \[ D(x_0, y_0) = \det{(H)} = f_{xx} f_{yy} - f_{xy}^2. \]
    Then
    \begin{enumerate}
        \item if $D(x_0, y_0) > 0$ and $f_{xx}(x_0, y_0) > 0$ then
            $H$ is positive definite and so 
            $(x_0, y_0)$ is a strict local minimum;
        
        \item if $D(x_0, y_0) > 0$ and $f_{xx}(x_0, y_0) > 0$ then
            $H$ is negative definite and so
            $(x_0, y_0)$ is a strict local maximum;
            
        \item if $D(x_0, y_0) < 0$ then 
            $(x_0, y_0)$ is a \emph{saddle point}; and
        
        \item if $D(x_0, y_0 = 0$ then the test is inconclusive.
    \end{enumerate}
\end{proposition}

\begin{example}
    Find and classify all local extrema of
    \[ f(x, y) = x^4 + y^4 - 4x - 4y. \]
\end{example}

\begin{solution}
    \[ f_x = 4x^3 - 4, \qquad f_y = 4y^3 - 4. \]
    Hence, $\nabla f = \bm 0 \iff x_0 = y_0 = 1$. $f(1, 1) = -6$.
    \[ f_{xx} = 12x^2, \qquad f_{yy} = 12y^2 \]
    so
    \[ H = 
        \begin{pmatrix}
            12 x^2 & 0 \\
            0 & 12 y^2 \\
        \end{pmatrix} \]
    with 
    \[ H =
        \begin{pmatrix}
            12 & 0 \\
            0 & 12 \\
        \end{pmatrix} \]
    at $(x_0, y_0)$.
    $D(x_0, y_0) = 144 > 0$ and $f_{xx}(x_0, y_0) = 12 < 0$ therefore 
    $(x_0, y_0)$
    is a strict local minimum.
\end{solution}

\section{Global extrema}

\begin{theorem}[]
    Suppose $f$ is function that is continuously differentiable 
    in the interior of some region $X \subset \R^n$.
    If $f$ attains its global extreme value (maximum or minimum) in $X$,
    then the global extremum is either a local extremum of
    $f$ in the interior of $X$
    \emph{or}
    it lies on the boundary of $X$.
\end{theorem}

\begin{remark}
    $f$ may not always attain a global extreme value in $X$;
    for example, let
    \[ f(x) = \frac{1}{1 + x^2}. \]
    We have $\max f = 1$ at $x = 0$; however,
    $f$ does not attain a minimum for any finite $x \in \R$.
    Similarly, let
    \[ g(x) = x, \qquad x \in (0, 1) = X. \]
    $g$ gets arbitrarily close to $0$ and $1$ but does not attain them.
\end{remark}

\begin{definition}[Bounded]
    A region $X$ is \textbf{bounded} if there exists finite $R > 0$
    such that
    \[ X \subset B_R(\bm 0). \]
\end{definition}
