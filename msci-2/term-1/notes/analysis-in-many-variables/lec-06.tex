\lecture{6}{21/10}

\begin{remark}
    $\grad f$ is normal to the currves of constant $f$.
\end{remark}

\begin{proposition}[Properities of gradient]
    Let $f, g: \R^n \to \R$ be scalar fields, $a, b$ be constants, and $\phi: \R \to \R$. Then
    \begin{enumerate}
        \item $\nabla(af + bg) = a\nabla f + b\nabla g$;
        \item $\nabla(fg) = \nabla f g + f \nabla g$; and
        \item $\nabla(\phi(f)) = \nabla f \cdot \frac{d\phi}{df}$.
    \end{enumerate}
\end{proposition}

\begin{example}[For $n = 2$]
    \begin{enumerate}
        \item 
            \begin{align*}
                \nabla(af + bg) &= \bm e_1 \frac{\partial}{\partial x} (af + bg) + \bm e_2 \frac{\partial}{\partial y} (af + by) \\
                                &= \bm e_1 \left( a \frac{\partial f}{\partial x} + b \frac{\partial g}{\partial x} \right) + \bm e_2 \left( a \frac{\partial f}{\partial y} + b \frac{\partial g}{\partial y} \right) \\
                                &= a \left( \bm e_1 \frac{\partial f}{\partial x} + \bm e_2 \frac{\partial f}{\partial y} \right) + b \left( \bm e_1 \frac{\partial g}{\partial x} + \bm e_2 \frac{\partial g}{\partial y} \right) \\
                                &= a \nabla f + b \nabla g.
            \end{align*}

        \item
            \begin{align*}
                \nabla(fg) &= \bm e_1 \frac{\partial}{\partial x} (fg) + \bm e_2 \frac{\partial}{\partial y} (fg) \\
                           &= \bm e_1 \left( \frac{\partial f}{\partial x} g + f \frac{\partial g}{\partial x} \right) + \bm e_2 \left( \frac{\partial f}{\partial y} g + f \frac{\partial g}{\partial y} \right) \\
                           &= f \left( \bm e_1 \frac{\partial g}{\partial x} + \bm e_2 \frac{\partial g}{\partial y} \right) + g \left( \bm e_1 \frac{\partial f}{\partial x}  + \bm e_2 \frac{\partial f}{\partial y} \right) \\
                           &= \nabla f g + f \nabla g.
            \end{align*}

        \item Let $\phi(f) = f^2$ and $f(x, y) = x\sin{y}$. Then
            \begin{align*}
                \phi(f(x, y)) &= x^2\sin^2{y} \\
                \nabla \phi(f) &= \nabla(x^2\sin^2{y}) \\
                               &= \bm e_1 (2x\sin^2y) + \bm e_2 (2x^2\sin y\cos y) \\
                               &= 2 (\bm e_1 \sin{y} + \bm e_2 x \cos{y}) x\sin{y} \\
                               &= \nabla f \cdot 2f \\
                               &= \nabla f \cdot \frac{d\phi}{df}.
            \end{align*}
    \end{enumerate}
\end{example}

\begin{example}
    Let 
    $f(\bm x) = \bm a \cdot \bm x - \bm x \cdot \bm x.$ 
    Hence
    \[ f(x, y, z) = a_1x + a_2y + a_3z - (x^2 + y^2 + z^2). \]
    Then
    \begin{align*}
        \frac{\partial f}{\partial x} &= a_1 - 2x \\
        \frac{\partial f}{\partial y} &= a_2 - 2y \\
        \frac{\partial f}{\partial z} &= a_3 - 2z \\
        \nabla f &= \bm e_1(a_1 - 2x) + \bm e_2(a_2 - 2y) + \bm e_3(a_3 - 2z) = \bm a - 2\bm x.
    \end{align*}
    We can compare this result with $f(\bm x) = \bm a \cdot \bm x - \bm x \cdot \bm x$.
\end{example}

\begin{remark}
    The above example makes lots of shortcuts look possible; however. we must be careful as it is easy to screw up.
\end{remark}

\section{Directional derivatives}

% todo this bit confused me a lot.

Let the curve $C$ in $\R^n$ be defined by $\bm x = \bm x(t)$ and let $f: \R^n \to \R$ be a scalar field. Then $f(\bm x(t)): \R \to \R$ is $f$ restricted to $C$ and
\[ \frac{d}{dt} f(\bm x(t)) = \frac{d\bm x}{dt} \cdot \nabla f. \]
We know $\frac{d\bm x}{dt}$ is tangent to $C$ at point $P$. If we use  arc length $s$ as a parameter instead of $t$ then we have that
\[ \frac{d\bm x}{ds} = \hat{\bm n} \]
is a unit tangent to $C$ and
\[ \frac{df(\bm x(s))}{ds} = \hat{\bm n} \cdot \nabla f. \]
Now $\frac{df(\bm x(s))}{ds}$ is the rate of change with respect ot the arc lengthin the direction $\hat{\bm n}$. This is called the directional derivative of $f$ in the direction $\hat{\bm n}$. Notice
\[ \frac{df}{ds} = \hat{\bm n} \cdot \nabla f = \norm{\hat{\bm n}} \norm{\nabla f} \cos{\theta} = \norm{\nabla f} \cos{\theta} \leq \norm{\nabla f}. \]
Hence, $\norm{\nabla f}$ is maximum of all possible directions $\hat{\bm n}$ when $\theta = 0$. That is, when the tangent to the curve is parallel to the gradient. Therefore, $\nabla f$ points in the direction where $f$ increases fastest. 
