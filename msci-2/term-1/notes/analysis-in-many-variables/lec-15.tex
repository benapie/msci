\section{Differentiation}

Here we are looking at differential maps $\R^n \to \R$, that is, functions $f: \R^n \to \R$ which are differentiable.

Our basic idea of differentiation is to \emph{approximate} a function of a single variable $f(x)$ near to $x = a$ by a straight line with slope $f'(a)$, as shown in Figure \ref{fig:approximate-function}.

\begin{figure}
    \centering
    \begin{tikzpicture}
        \begin{axis}[axis lines=middle,
                     xmin=-1, xmax=5, ymin=-1, ymax=3]
            \addplot[color = scarletred, 
                     smooth] coordinates {(-1, 0.7) (5,1.)};
            \addplot[smooth] {0.1*(x)^2 + 1};
        \end{axis}
    \end{tikzpicture}
    \caption{Illustration of a tangent line of a function.}
    \label{fig:approximate-function}
\end{figure}

So we define
\[ R(h) = f(a + h) - f(a) - hf'(a). \]
Then
\[ \lim_{h \to 0} \left( \frac{R(h)}{h} \right) = \lim_{h \to 0} \left(\frac{f(a + h) - f(a)}{h}\right) - f'(a) = 0 \]
by the usual definition of a derivative. That is, $R(h)$ goes to $0$ faster than $h$ does. So for $h$ small enopugh, $R(h)$ can be neglected when compared to $hf'(a)$. This gives us 
\[ f(a+h) - f(a) = hf'(a) + R(h) \approx hf'(a), \]
this is linear in $h$, which is very useful.

We are now going to generalise the idea of differentiability to $\R^n$.

\begin{definition}[Differentiability]
    Let $U$ be an open set in $\R^n$ and $f: U \to \R$ is a scalar field. $f$ is \textbf{differentiable} at $\bm a \in U$ if we can find a vector $\bm v(\bm a)$ such that
    \[ f(\bm a + \bm h) - f(\bm a) = \bm h \cdot \bm v(\bm a) + R(\bm h) \tag{$\star$} \]
    with 
    \[ \lim_{\bm a \to \bm 0} \left(\frac{R(\bm h)}{\norm{\bm h}} \right) = 0. \tag{$\star\star$} \]
\end{definition}

\begin{remark}
    \begin{enumerate}
        \item $(\star)$ defines what $R(h)$ is, $(\star\star)$ is the important bit.
        \item $\bm h \cdot \bm v(\bm a)$ is linear in $\bm h$.
    \end{enumerate}
\end{remark}

\begin{theorem}
    From the above definition of differentiability, if $\bm v$ exists, it is equal to $\nabla f$.
\end{theorem}

\begin{proof}
    Take $\bm h = h \bm e_i$ for some $h \in \R$ and $i = 1, \ldots, n$. Then $\norm{\bm h} = \lvert h \rvert$ and by definition
    \begin{align*}
        \left( \nabla f \right)_i = \frac{\partial f}{\partial x_i} &= \lim_{h \to 0} \left(\frac{\bm a + h\bm e_i) - f(\bm a)}{h}\right) \\
        &= \lim_{h \to 0} \left( \frac{f(\bm a) + h\bm e_i \cdot \bm v(\bm a) + R(\bm h) - f(\bm a)}{h} \right) \\
        &= \lim_{h \to 0} \left( \bm e_i \cdot \bm v(\bm a) + \frac{R(\bm h)}{h} \right) \\
        &= \bm e_i \cdot \bm v(\bm a) + \lim_{h \to 0} \left(\frac{R(\bm h)}{\norm{\bm h}} \right) \\
        &= \bm e_i \cdot \bm v(\bm a) = v_i(\bm a).
    \end{align*}
\end{proof}

\begin{remark}
    The components of $\nabla f$ may be well-defined at $\bm x = \bm a$ but $f$ could be non-differentiable.
\end{remark}

\begin{example}
    Let $f:\R^2 \to \R$ defined by
    \[ 
        f(x, y) = 
        \begin{cases}
            \frac{xy}{x^2 + y^2} & \bm x \neq \bm 0 \\
            0                    & \bm x = \bm 0. \\
        \end{cases}
    \]
    At the origin, we have that
    \[ \frac{\partial f}{\partial x} = \lim_{h \to 0} \left( \frac{f(h, 0) - f(0,0)}{h} \right) = 0, \]
    likewise $\sfrac{\partial f}{\partial y} = 0$ at the origin. So $\nabla f(0,0) = \bm 0$. If $f$ is differentiable, $\bm v$ is equal to this as well. Setting $\bm h = (h_1, h_2)$,
    \[ f(h_1, h_2) - f(0,0) = \bm h \cdot \bm v(\bm 0) + R(\bm h) = R(h_1, h_2) \]
    so
    \[ R(h) = \frac{h_1h_2}{h_1^2 + h_2^2} \]
    if $f$ is differentiable at the origin. So we can test whether $R(h)$ tends to zero faster than $h$:
    \[ \frac{R(\bm h)}{\bm h} = \frac{h_1h_2}{h_1^2 + h_2^2} \cdot \norm{h}^{-1} = \frac{h_1h_2}{(h_1^2 + h_2^2)^{\sfrac32}}. \]
    If we take this limit with $h_1 = 0$ or $h_2 = 0$, we get $0$; however, if $h_1 = h_2 = h$ 
    \[ \frac{R(\bm h)}{\bm h} = \frac{1}{2\sqrt 2 \cdot h} \not \to 0\]
    as $\bm h \to 0$; therefore, $f$ is not differentiable.
\end{example}
