\lecture{22}{10/12}

\begin{theorem}[Extreme value theorem]
    For any closed and bounded region $X \subset \R^n$,
    any continuous function $f: X \to \R$ 
    always attains its maximum and minimum values in $X$.
\end{theorem}

\begin{proof}
    The proof from this is seen in Complex Analysis:
    \[ \text{``closed and bounded''} 
    \qquad \iff \qquad
    \text{``compact''}.\]
\end{proof}

We now have a recipe for finding global extrema of $f$
in a closed and bounded (compact) region of $X$:

\begin{enumerate}
    \item find all critical points in $\int X$;
    \item compute the value at each critical point;
    \item restrict $f$ to the boundary and find extreme values there; then
    \item compare the value of $f$ at all critical points in $\int X$ and the boundary,
        the smallest is the global minimum and the largest is the 
        global maximum.
\end{enumerate}

(iii) can be difficult and must be argued on a case by case basis.
Note that if $X$ is \emph{not} compact, we can try express $X$ as
\[ X = Y \cup Z \]
where $Y$ is compact and $f(y) \geq f(z)$ or $f(y) \leq f(z)$ for all
$y \in Y$ and $z \in Z$.

\begin{example}
    Find the global extrema of
    \[ f(x, y) = x^4 + y^4 - 4x - 4y \]
    in $X = [0,2] \times [0,2] \subset \R^2$.
\end{example}

\begin{solution}
    We have $\int X = (0,2) \times (0,2)$.
    We found earlier that $f$ has one critical point at $(1, 1)$,
    with $f(1, 1) = -6$.
    Now we look at the boundary.
    At $x = 0$:
    \[ f(0, y) = y^4 - 4y \]
    which has a minimum of $-3$ at $y - 1$ and a maximum of $8$ at $y = 2$.
    At $x = 2$:
    \[ f(2, y) = 8 + y^4 - 4y \]
    which similarly has a minimum of $5$ at $y = 1$ and $16$ at $y = 2$.
    By symmetry, we have the same minimum and maximums for 
    $y = 0$ and $y = 2$ on $x$;
    therefore, the global minimum is $-6$ at $(1, 1)$ and 
    the global maximum is $16$ at $(2, 2)$.
\end{solution}

\section{Lagrange multipliers}

Suppose the boundary of a region $X \subset \R^2$ is a level set,
then (iii) (from our recipe for finding global extrema from last section) becomes
\begin{align*}
    \text{maximise}\quad   & f(x, y) \\
    \text{subject to}\quad & g(x, y) = 0.
\end{align*}
