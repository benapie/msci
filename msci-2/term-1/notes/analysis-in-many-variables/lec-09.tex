\lecture{9}{28/10}
\begin{proposition}[Properities of $\curl$]
    Let $a, b$ be constant, $\bm u, \bm v$ be vector fields in $\R^3$, and $f$ be a scalar field in $\R^3$:
    \begin{enumerate}
        \item $\nabla \times (a\bm u + b\bm v) = a \nabla \times u + b \nabla \times \bm v$; and
        \item $\nabla \times (f \bm v) = (\nabla f) \times \bm v + (\nabla \times \bm v)f$.
    \end{enumerate}
\end{proposition}

\begin{proof}
    \begin{enumerate}
        \item Follows from the lienarity of derivatives.

        \item This proof is not so easy... it just requires expanding out all the terms and then grouping together. It would take far too long to write out.
    \end{enumerate}
\end{proof}

\section{Applying $\nabla$ twice}

\begin{definition}[Laplace operator]
    If $f$ is a twice-differentiable real-valued function then the Lapacian of $f$ is defined by
    \begin{align*}
        \Delta f = \nabla^2 f &= \nabla \cdot \nabla \\
                              &= \frac{\partial^2f}{\partial x_1^2} + \frac{\partial^2f}{\partial x_2^2} + \ldots + \frac{\partial^2f}{\partial x_n^2}
    \end{align*}
    in Cartesian coordinates.
\end{definition}

\begin{example}
    Let $f(x, y) = \log{(x^2 + y^2)} = \log{(\bm x \cdot \bm x)}$. 
    \begin{align*}
        \frac{\partial f}{\partial x} &= \frac{2x}{x^2 + y^2} & \frac{\partial f}{\partial y} &= \frac{2y}{x^2 + y^2} \\
        \frac{\partial^2 f}{\partial x^2} &= \frac{2(y^2 - x^2)}{(x^2 + y^2)^2} & \frac{\partial^2 f}{\partial y^2} &= \frac{2(x^2 - y^2)}{(x^2 + y^2)^2}; 
    \end{align*}
    hence,
    \[ \nabla^2f = \frac{\partial^2 f}{\partial x^2} + \frac{\partial^2 f}{\partial y^2} = \frac{2}{(x^2 + y^2)^2} (y^2 - x^2 + x^2 - y^2) = 0, \]
    provided $\bm x \neq 0$.
\end{example}

\begin{example}
    Let $f(x, y) = x^3 - 3xy^2$.
    \begin{align*}
        \frac{\partial^2 f}{\partial x^2} &= 6x & \frac{\partial^2 f}{\partial y^2} &= -6x.
    \end{align*}
    So
    \[ \nabla^2f = 6x + (-6x) = 0. \]
\end{example}

We have seen two scenarios where $\nabla^2f = 0$, this is not always the case!. 

\begin{example}[Two more examples of applying $\nabla$ twice]
    We are now working $\R^3$.
    \begin{enumerate}
        \item Consider
            \[ \nabla \times \nabla f. \]
            Expanding this we find that this is always $\bm 0$ as long as the second partial derivates of $f$ are continuous.

        \item Consider
            \[ \nabla \cdot (\nabla \times \bm v). \]
            Again, expanding this out we find that this is always equal to $0$ given that all the partial second derivates are continuous.
    \end{enumerate}
\end{example}
