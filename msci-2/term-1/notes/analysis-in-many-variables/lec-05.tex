\section{Partial differentiation with vector notation}
\lecture{5}{15/10}

For
$f(x, y) : \R^2 \to \R$
we had
\[ \frac{\partial f}{\partial x} = \lim_{h \to 0} \left( \frac{f(x + h, y) - f(x, y)}{h} \right). \]
In vector notation we write 
\[ \frac{\partial f}{partial x} = \lim_{h \to 0} \left( \frac{f(\bm x + h\bm e_1) - f(\bm x)}{h} \right). \]
Sometimes we use
$\partial_a f(\bm x)$
as a shorthand for this.

\section{A chain rule in vector notation}

If we restrict a function 
$f(x, y) : \R^2 \to R$ 
to the parametric curve
$C : (x(t), y(t))$
we get a fucntion of $t$
\[ F(t) = f(x(t), y(t)). \]
We know that
\[ \frac{dF}{dt} = \frac{dx}{dt} \frac{\partial F}{\partial x} + \frac{dy}{dt} \frac{\partial F}{dy}. \]
in vector notation $C$ is
\[ \bm x(t) = x_1(t) \bm e_1 + x_2(t) \bm e_2 \]
where $x_1 = x(t)$ and $x_2 = y(t)$. Let $f(\bm x) = f(x, y)$. Then
\[ \frac{df}{dt} (\bm x(t)) = \frac{d x_1}{dt} \frac{\partial f}{\partial x_1} + \frac{dx_2}{dt} \frac{\partial f}{\partial x_2} \]
and as an operator we write
\[ \frac{d}{dt} = \frac{dx_1}{dt} \frac{\partial}{\partial x_1}  + \frac{dx_2}{dt} \frac{\partial f}{\partial x_2}. \]
We can rewrite this expression as a scalar product, that is
\[ 
    \frac{dF}{dt} =
    \left(
        \bm e_1 \frac{dx_2}{dt} + \bm e_2 \frac{dx_2}{dt}
    \right)
    \cdot
    \left(
        \bm e_1 \frac{\partial f}{\partial x_1} + \bm e_2 \frac{\partial f}{\partial x_2}.
    \right).
\]
We can also write this as an operator:
\[ 
    \frac{d}{dt} =
    \left(
        \bm e_1 \frac{dx_2}{dt} + \bm e_2 \frac{dx_2}{dt}
    \right)
    \cdot
    \left(
        \bm e_1 \frac{\partial}{\partial x_1} + \bm e_2 \frac{\partial}{\partial x_2}.
    \right).
\]
The left side of the scalar product is the tangent to the curve and the right side of the scalar product is called the \emph{gradient of $f$}, deonted $\nabla f$ or $\grad f$. From this we have
\[ \frac{dF}{dt} = \frac{d\bm x}{dt} \cdot \nabla f. \]
This applies to $n$-dimensions as well, given $f(\bm x) : \R^n \to \R$ and $F(t) = f(\bm x(t))$ we have
\[ \frac{dF}{dt} = \frac{d}{dt} f(\bm x(t)) = \frac{dx_i}{dt} \frac{\partial f}{\partial x_i} = \frac{d\bm x}{dt} \cdot \nabla{f} \] 
via the \emph{Einstein Summation Convention}.

\chapter{The gradient of a scalar function}

\begin{definition}[Gradient]
    In $n$-dimensions, we define the \emph{del} (or \emph{nabla}) operator $\nabla$ as
    \[ \nabla \equiv \bm e_1 \frac{\partial}{\partial x_1} + \bm e_2 \frac{\partial}{\partial x_2} + \ldots + \bm e_n \frac{\partial}{\partial x_n} = \bm e_i \frac{\partial}{\partial x_i}. \]
    If $f : \R^n \to R$ is a scalar field, then we define its \textbf{gradient} to be given by
    \[ \nabla{f} \equiv \grad{f} \equiv \bm e_1 \frac{\partial f}{\partial x_1} + \bm e_2 \frac{\partial f}{\partial x_2} + \ldots + \bm e_n \frac{\partial f}{\partial x_n} = \bm e_i \frac{\partial f}{\partial x_i}. \] 
    It is a vector field.
\end{definition}

\begin{example}
    In two dimensions, set 
    $\bm x = x \bm e_1 + y \bm e_2$
    and 
    \[ f(\bm x) = \frac{x^2 + y^2}{4}. \]
    Then
    \[ \frac{\partial f}{\partial x} = \frac{x}{2}, \qquad \frac{\partial f}{\partial y} = \frac{y}{2} \]
    and
    \[ \nabla{f} = \frac{1}{2} x \bm e_1 + \frac{1}{2} y \bm e_2 = \frac{1}{2} \bm x.\]
\end{example}
