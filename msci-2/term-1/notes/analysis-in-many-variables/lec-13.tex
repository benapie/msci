\lecture{13}{7/11}

Recall the property of $\curl$ introduced earlier:
\[ \nabla \times (f\bm v) = (\nabla f) \times \bm v + f(\nabla \times \bm v) \]
for a scalar field $f$ and vector field $\bm v$. We will now prove this.

\begin{proof}
    \begin{align*}
        (\nabla \times (f\bm v))_i &= \varepsilon_{ijk} \frac{\partial}{\partial x_j}(fv_k) \\
        &= \varepsilon_{ijk} \frac{\partial f}{\partial x_j} v_k + \varepsilon_{ijk}f\frac{\partial v_k}{\partial x_j} \\
        &= (\nabla f \times \bm v)_i + f\cdot(\nabla \times \bm v)_i.
    \end{align*}
\end{proof}

\chapter{Differentiability of a single variable}

Recall that a function of a single variable $f(x)$ is differentiable at $a$ if its derivative
\[ \frac{df}{dx} = \lim_{h \to 0} \left(\frac{f(x + h) - f(x)}{h}\right) \]
exists at $x = a$. We will now generalise this to $\R^n$.

\section{Continuity and open sets}

There will be a lot of repeated content from Complex Analysis II here, just a warning.

\begin{definition}[Limit]
    Let $f: \R^n \to \R$, $l \in \R$, and $\bm x, \bm a \in \R^n$. We say that $f$ tends to $l$ if
    \[ \;\forall\; \varepsilon > 0 \;\exists\; \delta > 0 : \quad 0 < \norm{\bm x - \bm a} < \delta \implies \lvert f(\bm x) - l \rvert < \varepsilon. \]
    We denote this as
    \[ \lim_{\bm x \to \bm a} f(\bm x) = l. \]
\end{definition}

\begin{definition}[Continuous]
    Let $f: \R^n \to \R$ and $\bm x, \bm a \in \R^n$. $f$ is \textbf{continuous} at $\bm a$ if
    $\lim_{\bm x \to \bm a} f(\bm x)$
    exists and is equal to $f(\bm a)$.
\end{definition}

\begin{example}
    Let $f(\bm x) = x^2 + y^2$. Show that $\lim_{\bm x \to \bm 0} = 0$.
\end{example}

\begin{solution}
    Let $\varepsilon > 0$, pick $\delta = \sqrt{\varepsilon}$. Then we let $0 < \norm{\bm x} < \sqrt{\varepsilon}$. Then
    \[ x^2 + y^2 < \varepsilon \implies \lvert x^2 + y^2 \rvert < \varepsilon \implies \lvert f(\bm x) - 0 \rvert < \varepsilon \]
    and hence $\lim_{\bm x \to \bm 0} = 0$.
\end{solution}

\begin{example}
    Show that $f(\bm x) = \frac{xy}{x^2 + y^2}$ has no limit as $\bm x \to \bm 0$ even though $\lim_{x \to 0}f(x,0)$ and $\lim_{y\to 0}f(0, y)$ both exist.
\end{example}
