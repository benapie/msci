\chapter{Introduction}
\lecture{1}{8/10}

\begin{definition}[Computer network]
    A \textbf{computer network} is a group of devices that are connected together in order to exchange information or share resources.
\end{definition}

We refer to all computerised systems that are intended to be used by a user as a \textbf{host} or \textbf{end system}.

Within a network, we have a number of different types of \textbf{communication links} that connect our devices. For example, fiber, copper, radio, etc. We will be interested in the \textbf{transmission rate} (bandwidth) of these links.

We have \textbf{packet switches} in networks that forward \textbf{packets} (chunks of data) on to other devices in the network. These devices include routers and switches.

\begin{definition}[Protocol]
    \textbf{Protocols} define the format and order of messages send and received among network entities as well as the actions done on transmission receipt. It is a set of rules that govern communication.
\end{definition}

\begin{definition}[Access network]
    An \textbf{access network} is a type of network that connects subscribers to a service provider.
\end{definition}

\textbf{Digital subscribe line} (DSL) is a technology that uses existing phone line infrastructure to form an access number. In this situation, we have asymmetric access (such that downstream and upstream rates differ) with typically a higher downstream rate.

\textbf{Cable network} is a network connectec via coaxial cables. These are common for television but have poor transmission rates.

\textbf{Enterprise access networks} (ethernet) is widely used in companies, universities, etc. and they consist of wired (copper cables) connections between devices. They typical have much higher transmission rates over technologies such as DSL or cable.

Within a home network, we typically have a wireless access point (WiFi), a router and firewall, and then a cable or DSL modem to convert data into a form that can be used by an ISP. These tasks used to have separate devices that performed them, but now they are done by a single device.

\begin{definition}[Wireless access networks]
   A \textbf{wireless access network} is a access network where end systems connects to the router wirelessly. 
\end{definition}

The standard for wireless data transmission is \textbf{WiFi} (802.11).

\begin{definition}[Physical media]
   We refer to the physical materials that is used to transmit data as \textbf{physical media}. 
\end{definition}

There are two commonly used type of copper-based physical media:
\begin{enumerate}
    \item twisted pair, two insulated cpoper wires (such as ethernet); and
    \item coaxial cable, two concentrial copper conductors (can achieve high data transmission rates).
\end{enumerate}

Copper cables are prone to electromagnetic noise which can cause data to become corrupt. This issue is not present in fibre as it is a glass fibre carrying light pulses (representing bits). Fibre has very high transmission rate and low error rate.

Radio physical media are signals carried in the electromagnetic spectrum with no physical wire. We classify radio into three groups:
\begin{enumerate}
    \item very short distance over \SI{5}{\metre} - \SI{10}{\metre} (Bluetooth);
    \item local area over \SI{10}{\metre} - \SI{1000}{\metre} (WiFi); and
    \item wide area over miles (cellular/mobile).
\end{enumerate}
