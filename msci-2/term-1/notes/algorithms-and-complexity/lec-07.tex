\section{Longest common subsequence}
\lecture{7}{18/11}

\begin{definition}[DNA strand]
    A \textbf{strand of DNA} consists of a string of molecules called \textbf{bases}: 
    that is, \textbf{adenine}, \textbf{guanine}, \textbf{cytosine}, and \textbf{thymine}. 
    For us, this a string over the finite string $\{A, C, G, T\}$.
\end{definition}

We want to know how \emph{similar} two DNA strands are. 
To do this, 
we treat the strands as sequences and define a measure of the similarity as the length of the largest shared subsequence.

\begin{example}
    Let $s_n = (A, G, G, T)$ and $t_n = (A, A, G, T)$. 
    Then the largest shared subsequence is $u_n = (A, G, T)$ which has size 3.
\end{example}

\begin{definition}[Common subsequence]
    Let $s_n$ and $t_n$ be sequences. $u_n$ is a \textbf{common subsequence} 
    of $s_n$ and $t_n$ if it is a subsequence of $s_n$ and $t_n$.
\end{definition}

\begin{example}
    Let $s_n = (A, B, C, B, D, A, B)$ and $t_n = (B, D, C, A, B, A)$. 
    Some common subsequences of $s_n$ and $t_n$ are
    \begin{enumerate}
        \item $(B, C, A)$;
        \item $(B, C, B, A)$; and
        \item $(B, D, A, B)$.
    \end{enumerate}
\end{example}

\begin{problem}[Longest common subsequence]
    Given sequences $s_n$ and $t_n$, 
    what is the longest common subsequence of $s_n$ and $t_n$ ($\operatorname{LCS}{(s_n, t_n)}$)?
\end{problem}

Now that we have introduced this problem, 
we can see how to apply dynamic programming to solve it.

\begin{theorem}[]
    Let 
    $S = (s_1, \ldots, s_k)$ 
    be a longest common subsequence of 
    $X = (x_1, \ldots, x_m)$ and $Y = (y_1, \ldots, y_n)$. 
    Then
    \begin{enumerate}
        \item if $x_m = y_n$, 
            then $z_k = x_m = y_n$ 
            and $(z_1, \ldots, z_{k - 1})$ 
            is a longest common subsequence of 
            $(x_1, \ldots, x_{m -1})$ 
            and 
            $(y_1, \ldots, y_{n - 1})$;
        \item if $x_m \neq y_n$ and $z_k \neq x_m$ 
            then $z$ is a longest common subsequence of 
            $x_1, \ldots, x_{n - 1}$ and $Y$; and
        \item if $x_m \neq y_n$ and $z_k \neq y_n$ then 
            $z$ is a longest common subsequence of 
            $X$ and $(y_1, \ldots, y_{n - 1})$.
    \end{enumerate}
\end{theorem}

\begin{proof}
    Proof by contradiction, this is quite easy to follow. 
\end{proof}

\begin{corollary}
    Let $X, Y$ be sequences of size $n$ and $m$. 
    Let $C_{XY}(i,j)$ be the length of the longest common subsequence of the first $i$ elements of $X$ and the first $j$ characters of $Y$. 
    Then
    \[ 
        C_{XY}(i, j) =
        \begin{cases}
            0 & i = 0 \;\text{or} j = 0 \\
            c(i - 1, j - 1) + 1 & i,j > 0, x_i = y_i \\
            \max\{c(i, j - 1), c(i - 1, j)\} & \text{otherwise}.
        \end{cases}
    \]
\end{corollary}

Lets look at a bottom-up approach to a dynamic programming solutioin.

\begin{algorithm}[Bottom-up longest common subsequence]
    We have input $X = (x_1, \ldots, x_m)$ and $Y = (y_1, \ldots, y_n)$. 
    We have two two-dimensional arrays 
    $c[0..m, 0..n]$ and $b[1..m, 1..n]$. $c$ 
    stores the length of the longest common subsequence where the index shows how many characters of $X$ and $Y$ to compare. 
    $b$ shows us which pair to be consider while reconstructing the longest common subsequence.
\end{algorithm}
