\section{Differentiation and integration of power series}
\lecture{21}{13/12}

\begin{proposition}[]
    Let
    \[ f(z) = \sum_{n = 0}^\infty a_n (z - c)^n \]
    be a power series with radius of convergence $0 < R \leq \infty$. 
    Then the \emph{formal} derivatives and anti-derivatives
    \[ 
        f'(z) = \sum_{n = 1}^\infty na_n(z - c)^{n - 1} 
        \quad \text{and} \quad 
        F(z) = \sum_{n = 0}^\infty \left(\frac{a_n}{n + 1} (z - c)^{n + 1}\right)
    \]
    have the same radius of convergence $R$.
\end{proposition}

Above are our expected formula for the derivative and and anti-derivative of a power series; 
however, we must confirm that they genuinely represent this. 

\begin{theorem}[]
    \label{the:power-series-derivative}
    Define $f: B_R(c) \to \C$ where
    \[ f(z) = \sum_{n = 0}^\infty a_n (z - c)^n \]
    be a power series with radius of convergence $0 < R \leq \infty$.
    Then $f$ is holomorphic on $B_R(c)$ with
    \[ f'(z) = \sum_{n = 1}^\infty na_n(z - c)^{n - 1} \]
    for $z \in B_R(c)$.
\end{theorem}

\begin{proof}
    For simplicity, we will assume $c = 0$; 
    the proof for arbitrary $c$ is essentially the same.
    So, we want to show that for all $w \in B_R(c)$ that
    \[ \lim_{x \to c} \left(\frac{f(z) - f(w)}{z - w}\right) \]
    exists and agrees with expression above.
    Since the convergence of $f(z)$ is absolute, we can reorder the sums.
    Hence, we have
    \[ f(z) - f(w) =
        \sum_{n = 1}^\infty a_n (z^n - w^n) =
        \sum_{n = 1}^\infty a_n (z - w) q_n(z) \]
    where $q_n(z) = \sum_{k = 0}^{n - 1} w^k z^{n - 1 - k}$.
    So, for $z \neq w$ we have
    \[ \frac{f(z) - f(w)}{z - w} = \sum_{n = 1}^\infty a_n q_n(z) = h(z). \]
    This series still makes sense if $z = w$, so we will view $h$ as being defined here too.
    We claim that the series defining $h(z)$ converges to a continuous function on $B_R(0)$. 
    We will use the local $M$-test to prove this.
    Given $z_0 \in B_R(0)$, we choose $r < R$ such that $w, z_0 \in B_r(0)$.
    We have for $z \in B_r(0)$
    \begin{align*}
        \left\lvert a_n q_n(z) \right\rvert 
        &=    \left\lvert a_n \sum_{k = 0}^{n - 1} w^k z^{n - 1 - k} \right\rvert \\
        &\leq \lvert a_n \rvert \sum_{n = 0}^{n - 1} \lvert w \rvert^k \lvert z \rvert^{n - 1 - k} \\
        &<    \lvert a_n \rvert \sum_{n = 0}^{n - 1} r^k r^{n - 1 - k} \\
        &=    n \lvert a_n \rvert r^{n - 1} = M_n.
    \end{align*}
    We have 
    \[ \sum_{n = 1}^\infty M_n = \sum_{n = 0}^\infty n \lvert a_n \rvert r^{n - 1} \]
    which we now is convergent as 
    $\sum_{n = 1}^\infty n a_n z^{n - 1}$ 
    has radius of convergence $R$ 
    (and therefore converges absolutely on $B_R(0)$, in particular, at $r$).
    From the local $M$-test, 
    we can conclude that the series defining $h$ converges uniformly
    to a continuous function on $B_R(0)$.
    Hence,
    \begin{align*}
        \lim_{z \to w} \left(\frac{f(z) - f(w)}{z - w}\right) 
        &= \lim_{z \to w} h(z) 
         = h(w) \\
        &= \sum_{n = 1}^\infty a_n q_n(w) \\
        &= \sum_{n = 1}^\infty \left(a_n \sum_{k = 0}^{n - 1} w^k w^{n - 1 - k}\right) \\
        &= \sum_{n = 1}^\infty \left(a_n \sum^{n - 1}_{k = 0} w^{n - 1}\right) \\
        &= \sum_{n = 1} n a_n w^{n - 1}
    \end{align*}
    as required.
\end{proof}

\begin{corollary}
    A power series $f$ as in Theorem \ref{the:power-series-derivative} with postive radius of convergence $R$ can be differentiated infinitely many times and
    \[ f^{(k)}(z) = \sum_{n = k}^\infty \left(k! \binom{n}{k} a_n (z - c)^{n - k}\right) \]
    for $z \in B_R(c)$.
    This implies $f^{(k)}(c) = k! a_k$.
\end{corollary}

\begin{corollary}
    A power series $f$ as in Theorem \ref{the:power-series-derivative} with positive radius of convergence $R$ has a holomorphic anti-derivative $F: B_R(c) \to \C$, 
    that is, $F'(z) = f(z)$,
    and $F$ is given by
    \[ F(z) = \sum_{n = 0}^\infty \left(\frac{a_n}{n + 1} (z - c)^{n + 1}\right) \]
    for $z \in B_R(c)$.
\end{corollary}

\begin{example}
    \begin{enumerate}
        \item The expected power series for $\sin$, $\cos$, $\sinh$, and 
            $\cosh$ all converge locally uniformly to continuous functions on $\C$.
            The derivatives/anti-derivatives match those expected.
            We will later see that they genuinely do represent the functions in question.
        
        \item More generally, 
            when can a holomorphic function be represented via a power series? 
            For example, what about $\log(z)$? 
            It is only defined (and is holomorphic) on 
            $\C \setminus \R_{\leq 0}$ 
            so cannot be defined as a power series on, say, $\D$.
            We can show that
            \[ \sum_{n = 1}^\infty \left((-1)^{n + 1} \frac{z^n}n\right) \]
            converges on $\D$,
            and we expect this to converge to $\log{(1 + z)}$,
            but what is so special about taking a power series about the point $c = 1$?
            Well, nothing.
            It turns out that \emph{holomorphic functions are precisely the functions that can be locally represented by power series at every point in their domain}.
            Hence, we can see that every holomorphic function is infinitely many times complex differentiable.

        \item Consider the geometric series
            \[ \sum_{n = 0}^\infty z^n. \]
            By the ratio test, 
            this sum converges when $\lvert z \rvert < 1$, 
            so the radius of convergence is $R = 1$.
            Also, the series converges to a continuous function on $\D$.
            \begin{align*}
                \sum_{n = 0}^\infty z^n
                &= \lim_{N \to \infty} \left(\sum_{n = 0}^N z^n\right) \\
                &= \lim_{N \to \infty} \left(\frac{1 - z^{N + 1}}{1 - z}\right) \\
                &= \frac1{1 - z}
            \end{align*}
            for $\lvert z \rvert < 1$.
            This limit function is defined and continuous for all $z \in \C \setminus \{1\}$.
            So, in some sense the convergence of the series in the cmoplex plane is limited to the unit disc $\D$
            because it can't pass the \emph{pole} at $z = 1$.

        \item We can determine the convergence of new power series by substitution.
            For example,
            notice that 
            \[ \lvert z \rvert < 1 \iff \lvert z^2 \rvert < 1 \iff \lvert -z^2 \rvert < 1 \]
            so by the substitution $z \mapsto z^2$ we have that
            $\sum_{n = 0}^\infty z^{2n}$
            converges locally uniformly on $\D$ to
            $\frac1{1 - z^2}$.
            Similarly, by the substitution $z \mapsto -z^2$ we have
            $\sum_{n = 0}^\infty (-1)^n z^{2n}$
            converges locally uniformly to $\frac1{1 + z^2}$.

        \item The examples above give us some real insight into the reasons for convergence of corresponding \emph{real} power series.
            Consider the graph of the real function $y = \frac1{1-x^2}$.
            It is obvious why its real interval of convergence is the unit interval;
            there are asymptotes at $x = \pm 1$ that we can't \emph{get past} continousuly.
            On the other hand,
            $y = \frac1{1+x^2}$ is a nice smooth looking graph everywhere on the real line.
            So why is the interval of convergence also restricted to the unit interval?
            We can now see the answer;
            the interval of convergence of the \emph{real} power series is restricted by the disc of convergence of the corresponding \emph{complex} power series!
            The issue being the poles in the complex plane at $z = \pm i$ that we couldn't see when considering only the real version of the function.
            So, hidden inside the real power series of nice continuous real functions is actually some meaningful and significant complex analysis.
    \end{enumerate}
\end{example}

\begin{remark}
    The key remark of the last example above is that complex analysis can give us new information about real functions.
\end{remark}

