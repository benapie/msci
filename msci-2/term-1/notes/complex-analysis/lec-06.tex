\lecture{6}{21/10}

\begin{lemma}[Open balls are open]
    Let $(X, d)$ be a metric space. Then for all $x \in X$ and $r > 0$, $B_r(x)$ is open.
\end{lemma}

\begin{proof}
    Given
    $y \in B_r(x),$
    we need to find
    $\varepsilon > 0$
    such that
    $B_{\varepsilon}(y)$
    is inside the original ball, that is 
    $B_{\varepsilon}(y) \subset B_r(x).$
    Here we pick 
    $\varepsilon = r - d(x, y).$
    Given 
    $z \in B_{\varepsilon}(y)$ 
    we have
    \begin{align*}
        d(x, z) &\leq d(x, y) + d(y, z) \\
                &< d(x, y) + \varepsilon \\
                &= d(x, y) + r - d(x, y) \\
                &= r;
    \end{align*}
    hence, $z \in B_r(x)$ and therefore $B_r(x)$ is open.
\end{proof}

\begin{example}
    $\mathbb H$, $\mathbb D$, $\C^\star$, and $\C \setminus \R_{\leq 0}$ are all open in $\C$ with the standard metric. 
\end{example}

\begin{example}
    Consider $\R$ with the standard metric. $(0, 1]$ is neither open or closed. This is because there does not exist a $\varepsilon > 0$ where $B_{\varepsilon}(1) \subset (0, 1]$. The complement of this interval is also not open by similar logic.
\end{example}

\begin{example}[Discrete metric]
    In the discrete metric
    \[ B_{\frac12}(x) = \{ x \} \]
    hence $B_{\frac12}(x)$ is open.
\end{example}

\begin{lemma}[Unions and intersections of open sets]
    Let $(X, d)$ be a metric space. If
    $\{ u_i \}_{i \in I}$
    is any set of open sets then
    \[ \bigcup_{i \in I} u_i \qquad \text{is open}. \]
    If
    $\{ v_i \}_{i \in I}$
    is a finite set of open sets then
    \[ \bigcap_{i \in I} v_i \qquad \text{is open}. \]
\end{lemma}

\begin{proof}
    \begin{enumerate}
        \item Given
            $x \in \bigcup_{i \in I} u_i,$
            $x \in u_{i_0}$
            for some 
            $i_0 \in I.$
            Since $u_{i_0}$ is open there is $\varepsilon > 0$ such that
            \[ B_{\varepsilon}(X) \subset u_{i_0} \subset \bigcup_{i \in I} u_i. \]

        \item Given $x \in \bigcap_{i \in I} v_i$. Then for each $i \in I$ there exists $\epsilon_i$ such that
            \[ B_{\varepsilon_i}(x) \subset v_i. \]
            Then let $\varepsilon = \min_{i \in I} \{ \varepsilon_i \}$. Then
            \[ B_{\varepsilon} \subset B_{\varepsilon_i} \subset v_i \]
            for all $\varepsilon \in I$. So
            \[ B_{\varepsilon} \subset \bigcap_{i \in I} v_i. \]
    \end{enumerate}
\end{proof}

\begin{corollary}
    Let $(X, d)$ be a metric space. 
    \begin{enumerate}
        \item Arbitrary intersections of closed sets are closed.
        \item Finite unions of closed sets are closed.
    \end{enumerate}
\end{corollary}

\begin{proof}
    The proof for this follows directly from De Morgan's Law and the previous lemma.
\end{proof}

\begin{definition}[Interior, closure, boundary, and exterior points]
    Let $A$ be a subset of a $(X, d)$.
    \begin{enumerate}
        \item The \textbf{interior} $A^0$ of $A$ is
            \[ A^0 = \{ x \in A : \;\exists\; \text{open set} \; U \subset A \;\text{with}\; x \in U \}. \]
        \item The \textbf{closure} $\bar A$ of $A$ is
            \[ \bar A = \{ x \in X : U \cap A \neq \emptyset \;\forall\; \;\text{open}\; U \;\text{with}\; x \in U \}. \]
        \item The \textbf{boundary} $\partial A$ of $A$ is
            \[ \partial A = \bar A \setminus A^0. \]
        \item The \textbf{exterior} $A^e$ of $A$ is
            \[ A^e = X \setminus \bar A. \]
    \end{enumerate}
\end{definition}

\begin{example}[Discrete metric]
    Consider the discrete metric space $(X, d)$.
    \begin{align*}
        \bar B_1(x) &= X \\
        B_1^0(x) &= \{ x \} \\
        \partial B_1(x) &= \bar B_1(x) \setminus B_1^0(x) = X \setminus \{ x \}.
    \end{align*}
\end{example}

\section{Convergence}

\begin{definition}[Convergence]
    Let $\{ x_n \}_{n = 1}^\infty$ be a sequence of points in the metric space $(X, d)$. We say $x_n \to x \in X$ a $n \to \infty$ if
    \[ \lim_{n \to \infty} \left( d(x_n, x) \right) = 0. \]
    That is, for all $\varepsilon > 0 $ there exists $N \in \N$ such that $d(x_n, x) < \varepsilon$ for all $n \geq N$. We say $x_n$ converges to $x$.
\end{definition}
