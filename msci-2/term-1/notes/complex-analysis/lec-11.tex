\lecture{11}{4/11}
\begin{theorem}
    Let $f = u + iv$ be defined on an open subset $U$ of $\C$. Assume that the partial derivatives $u_x, u_y, v_x, v_y$ exist, are continuous, and satisfy the Cauchy-Riemann equations at $z_0 \in U$. Then $f$ is complex differentiable at $z_0$.
\end{theorem}

\begin{proof}
    Let $h = h_1 + ih_2$. Then
    \begin{align*}
        f(z_0 + h) &= 
            \begin{pmatrix}
                u(x_0 + h_1, y_0 + h_2) \\
                v(x_0 + h_1, y_0 + h_2) \\
            \end{pmatrix}
            \\
        &=
            \begin{pmatrix}
                u(x_0, y_0) \\
                v(x_0, y_0) \\
            \end{pmatrix}
            +
            \begin{pmatrix}
                u_x(x_0, y_0) & u_y(x_0, y_0) \\
                v_x(x_0, y_0) & v_y(x_0, y_0) \\
            \end{pmatrix}
            \begin{pmatrix}
                h_1 \\
                h_2 \\
            \end{pmatrix}
            + \varepsilon(h)
    \end{align*}
    where $\lim_{h \to 0} \left(\frac{\varepsilon}{h}\right) = 0$. We write $u_x = u_x(x_0, y_0)$ and similar for other partial derivatives to make this a bit cleaner. Then
    \begin{align*}
        \begin{pmatrix}
            u_x & u_y \\
            v_x & v_y \\
        \end{pmatrix}
        \begin{pmatrix}
            h_1 \\
            h_2 \\
        \end{pmatrix}
            &=
            \begin{pmatrix}
                u_x & -v_x \\
                v_x & u_x  \\
            \end{pmatrix}
            \begin{pmatrix}
                h_1 \\
                h_2 \\
            \end{pmatrix}
            \\
            &= 
            \begin{pmatrix}
                u_x h_1 - v_x h_2 \\
                v_x h_1 + u_x h_2 \\
            \end{pmatrix}
            \\
            &= (u_x + iv_x)(h_1 + ih_2).
    \end{align*}
    Hence we have shown that $f(z_0 + h) = f(z_0) + f'(z_0)h + \varepsilon(h)$. So
    \[ \frac{f(z_0 + h) - f(z_0)}{h} = f'(z_0) + \frac{\varepsilon(h)}{h}. \]
    We have that
    \[ \frac{\varepsilon(h)}{h} = \frac{\varepsilon(h)}{\lvert h \rvert} \frac{\lvert h \rvert}{h} \]
    and we know that $\sfrac{\lvert h \rvert}h$ is bounded so
    \[ \lim_{h \to 0} \left(\frac{f(z_0 + h) - f(z_0)}{h}\right) = f'(z_0) = \lim_{h \to 0} \left(\frac{\varepsilon(h)}{h}\right) = f'(z_0). \]
\end{proof}

\begin{example}
    \begin{enumerate}
        \item Consider $\exp{z} = \exp{(x + iy)} = e^x(\cos{y} + i \sin{y})$. If we write $\exp{z} = u + iv$ we get
            \[ u(x, y) = e^x\cos{y}, \qquad v(x, y) = e^x\sin{y}. \]
            Hence
            \begin{align*}
                u_x(x, y) &= e^x\cos{y} & u_y(x, y) &= -e^x\sin{y} \\
                v_x(x, y) &= e^x\sin{y} & v_y(x, y) &= e^x\cos{y}.
            \end{align*}
            Hence the Cauchy-Riemann equations hold and $u_x, u_y, v_x, v_y$ are continuous everywhere. Therefore, $\exp$ is complex differentiable everywhere in $\C$.

        \item Chain rule shows us that $e^{iz}$ and $e^{-iz}$ are complex differentiable; hence, $\cos$, $\sin$, $\sinh$, $\cosh$ are complex differentiable.

        \item All polynomials are complex differentiable everywhere. So are all functions that are ratios of polynomials as long as they defined (that is, the denominator is not 0).

        \item Continuing part (i), we know that $\exp$ is complex differentiable everywhere. From a previous proposition, we can conclude that
            \[ \exp'{z_0} = \exp{z_0}, \]
            which is what we would expect from real analysis. 
    \end{enumerate}
\end{example}

\begin{definition}[Holomorphic]
    If $U \subset \C$ is an open set and $f: U \to \C$, we say that $f$ is \textbf{holomorphic} on $U$ if $f$ is complex differentiable at all points in $U$. We say that $f$ is \textbf{holomorphic} at a point $z_0 \in U$ if $f$ is holomorphic on some ball $B_\varepsilon(z_0) \subset U$ where $\varepsilon > 0$.
\end{definition}

\section{Connected sets (and zero derivatives)}

\begin{example}
    Let $U = \{ z \in \C: \lvert z \rvert \neq 1\}$ and $f: U \to \C$ given by
    \[ 
        f(z) =
        \begin{cases}
            1 & \lvert z \rvert < 1 \\
            2 & \lvert z \rvert > 1. \\
        \end{cases} 
    \]
    $f$ is holomorphic on $U$. In fact, $f'(z_0) = 0$ for all $z_0 \in U$; however, $f$ is not constant. The point here is that a zero derivative does \emph{not} imply that a function is constant if it is holomorphic. This is where we need a sense of \emph{connectedness} for functions.
\end{example}

\begin{definition}[Path]
    A \textbf{path} (or a \textbf{curve}) from $a \in \C$ to $b \in \C$ is a continuous function $\gamma: [0, 1] \to \C$ such that $\gamma(0) = a$ and $\gamma(1) = b$. The path is called \textbf{closed} if $\gamma(0) = \gamma(1)$.
\end{definition}

\begin{definition}[Smooth]
    A path $\gamma: [0, 1] \to \C$ is called \textbf{smooth} if $\gamma'(t)$ exists for all $t \in [0, 1]$ and $\gamma'(t)$ is continuous. That is, $\gamma$ is \textbf{continually differentiable} for $t \in [0, 1]$.
\end{definition}

\begin{definition}[Path-connected]
    A subset $U \subset \C$ is \text{path-connected} if for all $a, b \in U$ there is a smooth path $\gamma: [0, 1] \to U$ with $\gamma(0) = a$ and $\gamma(1) = b$. 
\end{definition}
