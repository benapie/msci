\chapter{Convergence and power series}
\lecture{18}{6/12}

Just a forward thought onto what we are studying this chapter, we are looking at functions of the form
\[ f(z) = \sum^{\infty}_{n=0} c_n (z - a)^n; \]
this is called a \emph{power series} (you should remember from Analysis I). For example, does
\[ e^z = \sum_{n = 0}^\infty \frac{z^n}{n!} \]
converge how we expect for $z \in \C$ (that is, the same as for the reals)?

\section{Convergence}

\begin{definition}[Pointwise convergence]
    Let $(X, d_X)$ and $(Y, d_Y)$ be two metric spaces. 
    A sequence $\{f_n\}_{n = 1}^\infty$ of functions $f_n: X \to Y$ converges \textbf{pointwise} on $X$ to $f: X \to Y$ if for all $x \in X$ we have
    \[ \lim_{n \to \infty} f_n(x) = f(x). \]
    That is,
    \[ \;\forall\; \varepsilon > 0 \;\forall\; x \in X \;\exists\; N \in \N \;\forall\; n > N: d_Y(f_n(x), f(x)) < \varepsilon. \]
\end{definition}

\begin{example}
    \begin{enumerate}
        \item Let $f_n: [0, 1] \to \R$, $f_n = x^n$. 
            Then we have $f_n \to f$ pointwise where
            \[ 
                f(x) =
                \begin{cases}
                    1 & x \in (0, 1] \\
                    0 & x = 0. \\
                \end{cases}
            \]

        \item Let $f_n : \C \to \C$ where $f_n = z^n$. Then we have three scenarios:
            \begin{enumerate}
                \item if $\abs{z} < 1$ then, given $\varepsilon > 0$, we choose $N$ such that 
                    \[ N > \frac{\log{\varepsilon}}{\log\abs z}, \]
                    then for $n  > N$ we have
                    \[ \abs{f_n(x) - 0} = \abs{z^n} = \abs{z}^n \leq \abs{z}^N < \varepsilon \]
                    so $f_n \to 0$ pointwise on $\D$;

                \item if $\abs{z} = 1$ and if $z = 1$ then $f_n \to 1$ pointwise; however, if $\abs z = 1$ and $z \neq 1$ then $f_n$ does not converge, it just constantly spins around the unit circle (cool!); and

                \item if $\abs z > 1$ then $f_n$ is unbounded and hence does not converge.
            \end{enumerate}
            Therefore, $f_n \to 0$ pointwise on $\D$.
    \end{enumerate}
\end{example}

\begin{definition}[Uniform convergence]
    Let $X$ and $Y$ be metric spaces as above. 
    A sequence $\{f_n\}_{n = 1}^\infty$ converges \textbf{uniformly} if
    \[ \;\forall\; \varepsilon > 0 \;\exists\; N \in \N \;\forall\; n > N \;\forall\; x \in X: d_Y(f_n(x), f(x)) < \varepsilon. \]
\end{definition}

\begin{lemma}[]
    If $f_n \to f$ uniformly then $f_n \to f$ pointwise.
\end{lemma}

\begin{remark}
    The above Lemma shows us that uniform convergence is a \emph{stronger notion of convergence} than pointwise convergence.
\end{remark}

\begin{theorem}[]
    Uniform limits of continuous functions are continuous.
\end{theorem}

\begin{proof}
    The proof for this follows a similar reasoning found in Analysis I, but with complex numbers, so will not be repeated here.
\end{proof}

The following Lemma is a useful tool for determining whether or not a function is uniform convergent. 
Remember, as uniform convergence is a stronger notion of convergence then pointwise convergence, 
it is useful to check whether a sequence of functions is pointwise convergence before looking at using the following techniques (given that it is so easy). 
This can normally be done just be looking at the sequence. 

\begin{lemma}[Test for uniform convergence (or lack of)]
    Let $f_n : X \to \C$ for a metric space $X$ be a sequence of functions such that $f_n \to f: X \to \C$ pointwise.
    Then
    \begin{enumerate}
        \item if there exists $\{s_n\}_{n = 1}^\infty \subset \R$ with $s_n \to 0$ and $\abs{f_n(x) - f(x)} < s_n$ for all $n$ and $x$ then $f_n \to f$ uniformly; and
        \item if there exists a sequence $\{x_n\}_{n = 1}^\infty$ and constant $c > 0$ such that
            \[ \abs{f_n(x_n) - f(x_n)} \geq c \]
            for all $n$ and $x$ then $f_n$ does \emph{not} converge uniformly to $f$.
    \end{enumerate}
\end{lemma}

\begin{proof}
    Part (i) falls through by writing the definitions of $s_n \to 0$ and uniform convergence. 
    (ii) happens to be the negation of uniform convergence.
\end{proof}

\begin{example}
    Let $f_n(z) = e^z + \frac 1n$ and $g(z) = e^z + \frac zn$ where $f, g: \C \to \C$. Both $f_n, g_z \to e^z$ pointwise. To see if $f_n \to f$ uniformly, take $s_n = \frac 1n$. Then
    \[ \abs{f_n - f} = \abs{e^z + \frac 1n - e^z} = \frac 1n \leq s_n \]
    for all $z \in \C$ so $f_n \to f$ uniformly.
    To see if $g_n \to f$ uniformly, we take $z_n = n$. Then 
    \[ \abs{g_n(x_n) - f(x_n)} = \abs{e^{z_n} + \frac{z_n}n - e^{z_n}} = 1 \] and by taking $c = 1$ we see that $g_n \not \to f$ uniformly.
\end{example}
