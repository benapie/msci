\pc{1}{16/10}
\setcounter{question}{1}
\question Consider the set of polynomials $\Q[x]$ with its normal addition operation $+$, but instead of the normal multiplication, define a new kind of \emph{multiplication}: for any $f, g \in \Q[x]$, let $f \circ g$ be the polynomial obtained by substituting $g(x)$ into $f(x)$, that is,
\[ (f \circ g)(x) = f(g(x)). \]
Does $\Q[x]$ equipped with the operations $+$ and $\circ$ form a ring?
\begin{solution}
    To check if this is a ring, first we must confirm that it is an abelian group under addition which it clearly is.

    $x \in \Q[x]$ is our multiplicative identity.    
    \begin{align*}
        f \circ ( (g \circ h)(x)) &= f \circ (g \circ (h(x))) \\
                                  &= f(g(h(x))) \\
                                  &= f\circ g(h(x)) \\
                                  &= (f \circ g) h(x)
    \end{align*}
    hence multiplication is associative.

    However,
    \[ f \circ(h + g) \neq f \circ h + f \circ g \]
    so this is not a ring.
\end{solution}

\setcounter{question}{3}
\question Let $R$ be a ring. Show, using only the definition of ring, that for any two elements $a, b \in R$ we have
\begin{parts}
    \part $0 \cdot a = a \cdot 0 = 0$;
    \begin{solution}
        Consider $0 + 0 = 0$. Then
        \begin{align*}
            a(0 + 0)              &= a \cdot 0 \\
            a \cdot 0 + a \cdot 0 &= a \cdot 0 \\
            a \cdot 0             &= 0;
        \end{align*}
        similarily
        \begin{align*}
            (0 + 0)a              &= 0 \cdot a     \\
            0 \cdot a + 0 \cdot a &= 0 \cdot a     \\
            0 \cdot a             &= a \cdot 0 = 0.
        \end{align*}
    \end{solution}

    \part $(-a)b = -(ab)$; and
    \begin{solution}
        Well $(-a)b = -(ab)$ iff $(-a)b+ab = 0$. We have $-a + a = 0$ as $R$ is an abelian group with addition. Using this with the distributivity law
        \begin{align*}
            0           &= 0 \cdot b            \\
                        &= (-a + a) \cdot b     \\
                        &= (-a)b + ab = 0,
        \end{align*}
        as required.
    \end{solution}

    \part $(-a)(-b) = ab$.
    \begin{solution}
        \begin{align*}
            (-a)(-b)        &= ab               \\
            a(-b)           &= -(-a)(-b)        \\
            a(-b + (-a)(-b) &= 0                \\
            (a + (-a))(-b)  &= 0 \cdot (-b) = 0.
        \end{align*}
    \end{solution}
\end{parts}

\setcounter{question}{4}
\question Let $R$ be a ring such that $r^2 = r$ for all $r \in R$. Show that $r + r = 0$ for all $r$ in $R$, and that $R$ is commutative. Can you find an example of such a ring?
\begin{solution}
    \begin{align*}
        (r + r)^2               &= r + r                \\
        (r + r)(r + r)          &= r + r                \\
        r^2 + r^2 + r^2 + r^2   &= r + r                \\
        r + r                   &= 0
    \end{align*}
    as required.
    \begin{align*}
        (r + s)^2               &= r + s                \\
        r^2 + s^2 + rs + sr     &= r + s                \\
        rs + sr                 &= 0                    \\
        rs                      &= sr
    \end{align*}
    hence commutative. An example of this kind of ring is $\Z/2$.
\end{solution}

\setcounter{question}{6}
\question
\begin{parts}
    \part Find all the solutions $\bar x \in \Z/12$ to the equation $\bar 3 \bar x = \bar 0$.
    \begin{solution}
        This could be brute forced, but following is a nicer solution.
        \begin{align*}
            \bar 3 \bar x           &=    \overline{3x}                 \\
            \overline{3x}           &=    \bar 0                        \\
            12                      &\mid 3x                            \\
            4                       &\mid x                             \\
            x                       &\in  \{ \bar 0, \bar 4, \bar 8 \}. 
        \end{align*}
    \end{solution}
\end{parts}
