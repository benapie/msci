\assignment{3}{23/11}

\setcounter{question}{23}
\question Show that the following are irreducible in $\Q[x]$.
\begin{parts}
    \setcounter{partno}{1}
    \part $3x^3 + x + 5$;
    \begin{solution}
        $f$ is irreducible in $\Q[x]$ if and only if it has any roots.
        $f(\sfrac pq) = 0$ implies that $p \mid 5$ and $q \mid 3$. Hence
        \[ \frac pq \in \{ \pm 1, \pm 5, \pm \frac13, \pm \frac53 \}. \]
        By directly checking, we see that $f$ has no roots and therefore is irreducible.
    \end{solution}

    \setcounter{partno}{2}
    \part $x^4 + 35x - 21$; and
    \begin{solution}
        Here we will apply the Eisenstein's criterion with prime $p = 7$ to conclude that $f$ is irreducible.
    \end{solution}

    \setcounter{partno}{3}
    \part $3x^4 + 3x^3 + 1$. (Hint: reduce mod some prime $p$. Write your argument carefully and motivate each step.)
    \begin{solution}
        Here form $\overline f(x) \in \Z/2[x]$, so
        \[ \overline f(x) = x^4 + x^3 + \overline 1. \]
        We see that $\overline f$ if and only if it has no roots and is not the product of two irreducible quadratics.
        As the only irreducible quadratic in $\Z/2[x]$ is $x^2 + x + \overline 1$ and 
        \[ (x^2 + x + \overline 1)^2 = x^4 + x^2 + \overline 1 \neq \overline f, \]
        we see that it only stands to prove that $\overline f$ has no roots, which can be done directly:
        \[ \overline f(\overline 0) = \overline 1, \qquad \overline f(\overline 1) = \overline 1; \]
        therefore, $\overline f$ is irreducible in $\Z/2[x]$ and hence $f$ is irreducible in $\Q[x]$.
    \end{solution}
\end{parts}

\setcounter{question}{28}
\question
\begin{parts}
    \part Show that the elements $3$, $1 - \sqrt{-5}$, and $1 + \sqrt{-5}$ are irreducible in $\Z[\sqrt{-5}]$. (Hint: try to deal with $1 \pm \sqrt{-5}$ in one go.)
    \begin{solution}
        An element $r \in \Z[\sqrt{-5}]$ if and only if $r$ is not a unit and $r = xy$ implies that $x$ or $y$ is a unit, where $x, y \in \Z[\sqrt{-5}]$. 
        So we consider
        \[3 = (a + b\sqrt{-5})(c + d\sqrt{-5}).\]
        We will also consider the homomorphism $N: \Z[\sqrt{-5}] \to \Z$ defined as
        \[ N(a + b\sqrt{-5}) = (a + b\sqrt{-5})(a - b\sqrt{-5}) = a^2 + 5b^2. \]
        Now, we apply this homomorphism to both sides of the equation above:
        \[ 9 = (a^2 + 5b^2)(c^2 + 5d^2). \]
        This implies that $a + 5b^2 \mid 9$, so
        \[ a^2 + 5b^2 \in \{ \pm 1, \pm 3, \pm 9 \}. \]
        But clearly $a^2 + 5b^2 \geq 0$ and $a^2 + 5b^2 \neq 3$, 
        so we are left with $a^2 + 5b^2 \in \{1, 9\}$. 
        If $a^2 + 5b^2 = 1$, then $b = 0$ and $a = \pm 1$ which implies that $a + b\sqrt{-5}$ is a unit.
        If $a^2 + 5b^2 = 9$, then $a = \pm 2$ and $b = \pm 1$.
        Hence either
        \[ 3 = (2 \pm \sqrt{-5})(c + d\sqrt{-5}) \implies c + d\sqrt{-5} = \frac3{2 \pm \sqrt{-5}} = \frac23 \mp \frac{\sqrt{-5}}3 \not \in \Z[\sqrt{-5}] \]
        or
        \[ 3 = (-2 \pm \sqrt{-5})(c + d\sqrt{-5}) \implies c + d\sqrt{-5} = \frac3{-2 \pm \sqrt{-5}} = -\frac23 \mp \frac{\sqrt{-5}}3 \not \in \Z[\sqrt{-5}]. \]
        Therefore, $3$ is irreducible. 
        Again, we consider
        \[ 1 \pm \sqrt{-5} = (a + b\sqrt{-5})(c + d\sqrt{-5}). \]
        Applying $N$ to both sides again we get
        \[ 6 = (a^2 + 5b^2)(c^2 + 5d^2). \]
        This implies that $a^2 + 5b^2 \mid 6$ and so
        \[ a^2 + 5b^2 \in \{\pm 1, \pm 2, \pm 3, \pm 6\} \]
        but as we know $a^2 + 5b^2 \geq 0$ and $a^2 + 5b^2 \not \in \{2, 3\}$ it can either equal $1$ or $6$.
        If $a^2 + 5b^2 = 1$ then $a = \pm 1$, $b = 0$ and so $a + b\sqrt{-5}$ is a unit.
        If $a^2 + 5b^2 = 6$ then $a = \pm 1$, $b = \pm 1$ so
        \[ 6 = 6(c^2 + 5d^2) \]
        by applying $N$ to both sides. 
        Hence, $c^2 + 5d^2 = 1$ which implies that $d = 0$ and $c = \pm 1$ so $c + d\sqrt{-5}$ is a unit. 
        Therefore, $1 \pm \sqrt{-5}$ is irreducible.
    \end{solution}

    \part Show that $1 - \sqrt{-5}$ is not a prime element in $\Z[\sqrt{-5}]$. (Hint: try to use the norm map whenever possible).
    \begin{solution}
        We know that $1 - \sqrt{-5} \mid 6 = 2 \cdot 3$; however, it does not divide $2$ or $3$. This can be shown by writing the definition of a number dividing another out and applying $N$ to both sides. Therefore, $1 - \sqrt{-5}$ is not a prime element.
    \end{solution}
\end{parts}

\setcounter{question}{31}
\question Let $f: R \to S$ be a homomorphism of rings where $S$ has at least two elements.
\begin{parts}
    \part Show that $\ker f$ is \emph{not} a subring of $R$ (according to the definitions we use).
    \begin{solution}
        We define $\ker f = \{ x \in R: f(x) = 0_S \}$. 
        By the definition of a homomorphism, we have that $f(1_R) = 1_S \neq 0_S$ given that $S$ is not the zero ring. 
        Hence, $1_R \not \in \ker f$.
        Therefore, $\ker f$ is not a subring of $R$.
    \end{solution}

    \part Show that $\im f$ \emph{is} a subring of $S$.
    \begin{solution}
        We define $\im f = \{ f(x) \in S: x \in R \} \subset S$.
        Using the definitions of a subring and homomorphism along with the following facts:
        \[ f(0_R) = 0_S, \qquad f(-x) = -f(x) \]
        for all $x \in R$ we conclude that $\im f$ is indeed a subring of $S$.
    \end{solution}
\end{parts}

\setcounter{question}{35}
\question Check the following statement about ideals in $\Z[\sqrt 5]$:
\[ I_1 = (3 - \sqrt 5, 3 + \sqrt 5) = (2, 1 - \sqrt 5) = I_2. \]
\begin{solution}
    \begin{align*}
        (3 - \sqrt 5)(3 + \sqrt 5) &= 9 - 5 = 4 \in I_1 \\
        (3 - \sqrt 5) + (3 + \sqrt 5) &= 6 \in I_1 \\
        4 (-1) &= -4 \in I_1 \\
        6 + (-4) &= 2 \in I_1 \\
        2 (-1) &= -2 \in I_1 \\
        (3 - \sqrt 5) + (-2) &= 1 - \sqrt 5 \in I_1 \\
        (1 - \sqrt 5) + (2) &= 3 - \sqrt 5 \in I_2 \\
        (3 - \sqrt 5) (-1) &= -3 + \sqrt 5 \in I_2 \\
        2 + 2 &= 4 \in I_2 \\
        4 + 2 &= 6 \in I_2 \\
        (-3 + \sqrt 5) + 6 &= 3 + \sqrt 5 \in I_2.
    \end{align*}
    Therefore, $I_1 = I_2$.
\end{solution}
