\assignment{4}{11/12}

\setcounter{question}{34}
\question Let $R$ be a commutative ring, and let $a, b \in R$ be elements such that $\gcd(a, b)$ exists.
\begin{parts}
    \part Prove that $(a, b) \subset (\gcd(a, b))$.
    \begin{solution}
        Recall that $(a, b) = \{ax + by: x, y \in R\}$.
        Now $a = xd$ and $b = yd$ where $d = \gcd(a, b)$ for some $x, y \in R$.
        Hence, $a, b \in (d)$.
        As $(d)$ is closed under multiplication and addition, we have
        \[ (a, b) \subset (d). \]
    \end{solution}

    \part In the special case $R = \Z$, prove that $(a, b) = (\gcd(a, b))$ for any $a, b \in \Z$.
    \begin{solution}
        As $R \in \{\Z, F[x]\}$ where $F$ is a field, 
        we have that $d = ax + by$ for some $x, y \in R$. 
        Hence $d \in (a, b)$ and so
        \[ (d) = (a, b) \]
        using the previous part.
    \end{solution}
\end{parts}

\setcounter{question}{38}
\question Find a linear polynomial $r(x) \in (\Z/5)[x]$ such that
\[ \overline{r(x)} = \overline{2x^4 + x^3 +2x + 1} \]
in $(\Z/5)[x]/(x^2 + x + \overline 2)$.
\begin{solution}
    A simple application of the division algorithm shows that
    \[ \overline 2 x^4 + x^3 + \overline 2 x + 1 = (\overline 2 x^2 - x - \overline 2) (x^2 + x + \overline 2) + (\overline 2 x + \overline 2) \]
    where we have $f(x) = \overline 2 x^4 + x^3 + \overline 2 x + 1$ and $r(x) = \overline 2 x + \overline 2$. It is clear that $f(x) - r(x) \in I$, and so $\overline f(x) = \overline r(x)$.
\end{solution}

\setcounter{question}{39}
\question Which of $\overline 2$, $\overline{1 - i}$, and $\overline{-1 - i}$ is the element $\overline{2 + i} \cdot \overline{1 + i}$ equal to in the ring $\Z[i]/(3 + i)$.
\begin{solution}
    We have that
    \[ \overline{2 + i} \cdot \overline{1 + i} = \overline{(2 + i)(1 + i)} = \overline{1 + 3i}. \]
    $\overline{3 + i} = \overline 0 \implies \overline i = -\overline 3$, so $\overline{1 + 3i} = \overline{-8}$, $\overline{1 - i} = \overline 4$, and $\overline{-1 - i} = \overline 2$. 
    We can either have $\overline 2 = \overline{-8}$ or $\overline 4 = \overline{-8}$.
    If $\overline 2 = \overline{-8}$, we have $\overline{10} = \overline 0$ and so $(3 + i) \mid 10$ which we can confirm as $(3 + i)(3 - i) = 10$. In the other case where $\overline 4 = \overline{-8}$, we have that $\overline{12} = \overline 0$; however, it is easy to see that this cannot occur. Hence,
    \[ \overline 2 = \overline{-1 - i} = \overline{2 + i} \cdot \overline{1 + i}. \]
\end{solution}

\setcounter{question}{41}
\question Let $R = \Z[\sqrt 7]$. Let $\phi: R \to \Z/9$ given by $\phi(a + b\sqrt 7) = \overline{a + 4b}$.
\begin{parts}
    \part Show that $\phi$ is a homomorphism.
    \begin{solution}
        All that needs to be shown is that the additive and multiplicative identity is mapped to the additive and multiplicative identity, and that there is additivity and multiplicitivity.
    \end{solution}

    \part Show that $\ker \phi = (9, 4 - \sqrt 7)$.
    \begin{solution}
        We have that $\ker\phi = \{a + b\sqrt 7 : \overline{a + 4b} = 0\}$.
        $\phi(9) = \overline 9 = \overline 0$ and $\phi(4 - \sqrt 7) = \overline{4 - 4} = \overline 0$;
        hence, $9, 4 - \sqrt 7 \in \ker\phi$ and so $(9, 4 - \sqrt 7) \subset \ker\phi$.
        Now we will do a bit of manipulation to show the other way round.
        If $a + b\sqrt 7 \in \ker\phi$, then $9 \mid a + 4b$.
        Therefore,
        \[ a + b\sqrt7 = a + 4b - b(4 - \sqrt7) \in (9, 4 - \sqrt 7 \]
        and so $\ker\phi \subset (9, 4 - \sqrt 7)$.
        Hence,
        \[ \ker\phi = (9, 4 - \sqrt 7). \]
    \end{solution}

    \part Show that $R/(9, 4 - \sqrt 7)$ is isomorphic to $\Z/9$.
    \begin{solution}
        Here we will use the First Isometric Theorem.
    \end{solution}
\end{parts}
