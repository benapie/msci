\documentclass[a4paper]{article}

\usepackage{booktabs, longtable}
\usepackage{geometry}

\title{Cyber Security Coursework}
\date{Due: 6th December 14:00}

\begin{document}
\maketitle

\paragraph{The mystery, solved}

Jess (the `bleeder') is in love with Mark; however, Mark is in love with Alice (who does not love him back). 
Out of jealously, Jess murdered a girl for dancing too close to Mark in Klute. 
Jess was also seen by Karl leaving a note threatening to kill Alice out of jealously of Mark loving her, so she murdered Karl (which is on CCTV).
The crime to be prevented is Jess killing Alice, which can be done by arresting Jess or by communicating to Jess that Alice \emph{does not} love Mark; infact, she thinks he's annoying.
As for relationship advice, Jess needs professional help and the rest of the employees should probably learn to communicate their feelings a little better.

\begin{longtable}{p{0.3\textwidth} p{0.3\textwidth} p{0.3\textwidth}}        
    \toprule
    Vulnerability & Exploit & Mitigation \\
    \midrule

    The (weak) root password is `123456789'. &
    An agent with access to the server could guess the password `12345678' and then gain root access. &
    The vulnerability can be secured by selecting a more secure password, including specials characters and numbers. \\
    \addlinespace

    Database is stored insecurely. & 
    Anyone with access to the database file can open and see the data, which contains sensitive data (such as credit card information). & 
    Encrypt the database file or ensure that a username and secure password is required to access it. \\
    \addlinespace

    The web server uses HTTP. &
    An agent sitting on the Wi-Fi being used to communicate with the web server can see all data transfered between them. &
    Implement HTTPS on the web server. \\
    \addlinespace

    There is read access for the bitcoin wallet executable. &
    An agent can read the executable and see the password and the private key for the bitcoin wallet. &
    Do not give the user read access for the bitcoin wallet executable. \\
    \addlinespace

    Password input is not sanitised. &
    An agent can inject malicious SQL code (such as getting all information or dropping tables). & 
    Sanitise user input. \\
    \addlinespace

    The database backup executable has its \texttt{setuid} bit set to 1 and was created by root, giving it root permissions. It also takes an argument which is unsanitised and is executed as a parameter in bash. & 
    An agent can execute the backup file with an argument of the form \texttt{"./Backup /root/backups/b5.db; ls"} to be able to run commands with root permissions. &
    Ensuring that there is not root access on this file or santising the user input such that it is not malicious. \\
    \addlinespace

    Port 8888 is left open with root access. &
    An agent with access to the server can run the command \texttt{netcat 127.0.0.1 8888} and gain root access. &
    Close the port or put a password on the port. \\
    \addlinespace

    Chat messages are not sanitised. &
    An agent can perform cross-site scripting and inject malicious Javascript code. &
    Sanitise user input. \\
    \addlinespace

    Passwords are stored in plain text, not hashed. & 
    If an agent gets access to the database, he has access to their passwords (which may be used for other services). &
    Hash (and salt) the passwords. \\
    \addlinespace

    Boot loader is not password protected. &
    An agent can login to a root terminal and exploit his privileges. &
    Password protect the boot loader. \\
    \addlinespace

    All users have read access on the Linux password file. & 
    This makes the user passwords suspectible to a dicitonary. &
    Remove read access on the file for all users. \\
    \addlinespace

    Paths are not properly verified on the web server. &
    Agents can gain access to the other files on the system (known as a path traversal attack). &
    Properly verify paths, this can be done by validating the URL before serving files. \\
    \addlinespace

    A picture in Jess' files is encrypted using ECB. &
    Agents can still see what the picture is representing, even though it is meant to be encrypted. & 
    Encrypt the image with non-ECB methods. \\
    \bottomrule
\end{longtable}

\end{document}
