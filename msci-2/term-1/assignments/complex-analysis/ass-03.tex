\assignment{3}{1/11}

\setcounter{question}{5}
\question We call a map $f: (X, d_1) \to (Y, d_2)$ between two metric spaces \textbf{Lipschitz} if there exists a positive constant $C$ such that
\[ d_2(f(x_1), f(x_2)) \leq C d_1(x_1, x_2) \]
for all $x_1, x_2 \in X$. Show that if $f$ is Lipschitz then $f$ is continuous.

\setcounter{question}{10}
\question Let $X$ be any metric space. We call a subset $A \subset X$ \textbf{discrete} if for every point $x \in A$ there is an open set $U$ containing $x$ that does not intersect any other point of $A$ (in other words, $U \cap A = \{x\}$).
\begin{parts}
    \part Show that $\Z$ is discrete inside $\R$.

    \part Show that 
    $\{\frac1n: n \in \Z, n \neq 0\}$ 
    is discrete in $\R$, but 
    $\{\frac1n: n \in \Z, n \neq 0\} \cup \{0\}$
    is not.

    \part Let $A$ be a closed discrete set inside a compact set $K$. Show that $A$ is finite.

    \part Use part (ii) to explain that one needs the \emph{closed} hypothesis in part (iii).

    \part Show that every subset of a discrete metric space is discrete.
\end{parts}

\setcounter{question}{11}
\question We call a metric space $X$ \textbf{connected} if the only subsets which are simultaneously open and closed (that is, \emph{clopen}) are $X$ and the empty set $\emptyset$. \emph{Note that this is the opposite situation to a discrete metric space, where every subset is clopen.}
\begin{parts}
    \part Let $X$ be the union of the intervals $[0, 1)$ and $[2, 3]$ together with the subspace metric from the standard metric on $\R^n$. Show that $X$ is not connected.
\end{parts}
