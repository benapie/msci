\tutorial{1}{21/1}

\question The \emph{quaternion group} $Q_8$ is defined by
\[
    Q_8 = \langle -1, i, j, k: (-1)^2 = 1, i^2 = j^2 = k^2 = ijk = -1 \rangle,
\]
that is, $Q_8$ is generated by the symbols $-1, i, j, k$ satisfying the
above relations.
\begin{parts}
    \part Show that $(-1)i = i(-1)$ (and similarly for $j$ and $k$).
    \begin{solution}
        \[
            (-1)i = i^2 i = i^3 = i i^2 = i(-1).
        \]
    \end{solution}

    \part Show that $Q_8$ is not abelian by showing that $ij \neq ji$.
    \begin{solution}
        We show this by manipulating the algebra of $Q_8$.
    \end{solution}

    \part Show that $xy = -yx$, for any $x, y \in \{i, j, k\}$ such that $x \neq y$.
    \begin{solution}
        Simple manipulation of algebra.
    \end{solution}

    \part Find all the elements in $Q_8$.
    \begin{solution}
        Let $S = \{\pm 1, \pm i, \pm j, \pm k\}$.
        We see that $\langle S \rangle = S$.
        As $\{-1,i,j,k\} \subset S$, $S = Q_8$.
    \end{solution}
\end{parts}

\question Find the centre $Z(Q_8)$ of the quaternion group $Q_8$
(you may use the results in the previous question).
\begin{solution}
    \[
        Z(Q_8) = \{\pm 1\}.
    \]
\end{solution}

\question Write down the order of each element of
\begin{parts}
    \part $\Z/9$;
    \begin{solution}
        $\overline 0$ has order $1$, $\overline 1, \overline 2, \overline 4, \overline 5, \overline 7$ has order
        $9$, and $\overline 3, \overline 6$ have order $3$.
    \end{solution}

    \part $\Z/8$;
    \begin{solution}
        $\overline 0$ has order 1, $\overline 1, \overline 3, \overline 5, \overline 7$ has order $8$,
        $\overline 2$ has order 4, and $\overline 4$ has order 2.
    \end{solution}

    \part $D_6$; and
    \begin{solution}
        $r_1, r^5$ has order 6,
        $r_2, r^4$ has order 3,
        $r_3, r^is$ has order 2, and
        $1$ has an order 1.
    \end{solution}

    \part $\C^\times$.
    \begin{solution}
        Let $z \in \C^\times$.
        \begin{description}
            \item[$\abs z > 1$]
                If $z^n = 1$ then $\abs z^n = 1$.
                But if $\abs z > 1$, then $\abs z^n > 1$ for all $n \in \N$.
                So $\ord(z) = \infty$.

            \item[$\abs z < 1$]
                Similarly, $\abs z^n < 1$ for all $n \in \N$.
                So $\ord(z) = \infty$.

            \item[$\abs z = 1$]
                Let $z = e^{2i\pi\theta}$
                with $\theta = [0,1)$.
                If $\theta \not\in \Q$, then $\ord(z) = \infty$
                as we will keep rotating around the unit circle.
                If $\theta \in \Q$, then $\theta = \frac{a}{b}$ where $a,b \in \Z$
                and $\gcd(a,b) = 1$.
                Then $\ord(z) = b$.
        \end{description}
    \end{solution}
\end{parts}

\question Show that the order of the element $\overline a$ of $\Z/n$ is $n/\gcd(n, a)$.
\begin{solution}
    Let $k = \ord(\overline a)$.
    Then $k$ is the smallest positive integer such that $k\overline a = \overline 0$.
    That is, $n \mid k a$.
    So $nx = ka$ for some $x \in \Z$.
    Now let $d = \gcd(n,a)$. Then $a = db$ and $n = dm$ for some $b,m \in \Z$.
    We have $dmx = kdb$, so $mx=kb$. 
    That is, $m \mid kb$.
    Now $d = \gcd(n,a) = \gcd(dm,db) = d\gcd(m,b)$ and so $\gcd(m,b) = 1$.
    Hence $m \mid kb$ implies that $m \mid k$.
    The smallest integer $k$ that satisfies this is $m$ and so
    \[
        \ord(\bm a) = m = \frac nd = \frac{n}{\gcd(n,a)}. 
    \]
\end{solution}

\question
\begin{parts}
    \part Show that if $x$ and $y$ are elements of finite order of a group $G$,
    and $xy = yx$,
    then $xy$ is also an element of finite order.
    What can you say about the order of $xy$ in terms of the orders of $x$ and $y$?
    \begin{solution}
        We have $x,y \in Z(G)$.
        Let $\ord(x) = m$ and $\ord(y) = n$.
        Then $(xy)^{mn} = 1$
        hence $\ord(xy) \leq mn$.
        In fact, $\ord(xy) \leq \operatorname{lcm}(m,n)$.
    \end{solution}

    \part Show that $\ord(x) = \ord(x^{-1})$.
    \begin{solution}
        \begin{align*}
            x^{\ord(x)} &= 1 \\
            (x^{\ord(x)})^{-1} &= 1 \\
            (x^{-1})^{\ord(x)} &= 1.
        \end{align*}
        We know that $\ord(x^{-1}) = \ord(x)$ is the smallest integer possible
        as otherwise we could reverse this and find an integer smaller than $\ord(x)$, a contradiction.
    \end{solution}

    \part Find a group $G$ and elements $x, y$ of $G$ such that $x$ and $y$ have
    finite order yet $xy$ has infinite order.
    \begin{solution}
        Let $G = \{f: \R \to \R, \;\text{$f$ is a bijection}\}$.
        Consider $f,g \in G$ where $f(t) = -t$ and $g(t) = 1-t$.
        Clearly $\ord(f) = \ord(g) = 2$.
        Consider $h(t) = (f \circ g)(t) = t-1$.
        Then $h^n(t) = t-n \neq t$ for any $n \in \N$.
        Therefore, $\ord(h) = \infty$.
    \end{solution}
\end{parts}
