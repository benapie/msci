\tutorial{2}{4/2}

\question Let $x \in G$ be an element in a group such that $x^n = 1$ for some $n \in \N$.
Show that $\ord(x)$ divides $n$.
\begin{solution}
    Let $n = \ord(x)q + r$ where $r < \ord(x)$
    by the division algorithm.
    Then 
    \[
        x^n = x^{\ord(x)q + r} = x^r = 1
    \]
    and as $k = \ord(x)$ is the smallest positive integer such that $x^k = 1$, $r = 0$.
    Hence $\ord(x) \mid n$.
\end{solution}

\question Let $G$ be an abelian group, and $x,y \in G$ be two elements of finite orders $m,n$ respectively.
Show that if $\gcd(m,n)=1$, then
\[
    \ord(xy) = mn.
\]
\begin{solution}
    $x^m = 1$ and $y^n = 1$, so $(xy)^{mn} = 1$ since $G$ is abelian.
    So $\ord(xy) \leq mn$.
    On the other side,
    \begin{align*}
        (xy)^{\ord(xy)}         &= 1                \\
        1 = ((xy)^{\ord(xy)})^m &= (xy)^{m\ord(xy)} \\
                                &= y^{m\ord(xy)}    \\
        n                       &\mid m\ord(xy),
    \end{align*}
    and since $\gcd(m,n) = 1$ we have $n \mid \ord(xy)$ (this is intuitive).
    By symmetry, $n \mid \ord(xy)$.
    Again, by $\gcd(m,n) = 1$ we have
    \[
        mn \mid \ord(xy).
    \]
    We already know $\ord(xy) \mid mn$ so we must have
    \[
        \ord(xy) = mn.
    \]
\end{solution}

\question
\begin{parts}
    \part Write
    $
        \begin{pmatrix}
            1 & 2 & 3 & 4 & 5 & 6 & 7 \\
            3 & 7 & 4 & 1 & 6 & 5 & 2 \\
        \end{pmatrix}
        \in S_7
    $
    as a product of disjoint cycles.
    \begin{solution}
        \[
            \begin{pmatrix}
                1 & 2 & 3 & 4 & 5 & 6 & 7 \\
                3 & 7 & 4 & 1 & 6 & 5 & 2 \\
            \end{pmatrix}
            = (1\;3\;4)(2\;7)(5\;6).
        \]
    \end{solution}

    \part Write $(1\;4\;3\;6)(4\;3\;1)(3\;5) \in S_6$ as a product of disjoint cycles.
    \begin{solution}
        \[
            (1\;4\;3\;6)(4\;3\;1)(3\;5) = (1\;3\;5\;4\;6).
        \]
    \end{solution}

    \part Let $\sigma = (6\;2\;5\;3)^{-1}(1\;3\;2\;4) \in S_{10}$.
    Determine the order $\ord(\sigma)$.
    \begin{solution}
        \[
            (6\;2\;5\;3)^{-1}(1\;3\;2\;4) = (3\;5\;2\;6)(1\;3\;2\;4) = (1\;5\;2\;4)(3\;6)
        \]
        therefore $\ord(\sigma) = \operatorname{lcm}(\{4,2\})$.
    \end{solution}
\end{parts}

\question List all the transpositions in $S_4$
(make sure not to include the same transposition twice).
Show that $(1\;2)$ and $(1\;2\;3\;4)$ together generate a set which contains
all the transpositions of $S_4$.
\begin{solution}
    \[
        (1\;2), (1\;3), (1\;4), (2\;3), (2\;4), (3\;4).
    \]
    Let $\lambda = (1\;2\;3\;4)$.
    \begin{align*}
        \lambda (1\;2) \lambda^{-1} &= (2\;3) \\
        \lambda (2\;3) \lambda^{-1} &= (3\;4) \\
        \lambda (3\;4) \lambda^{-1} &= (4\;1) \\
        (2\;3)(1\;2)(2\;3)^{-1}     &= (1\;3) \\
        \lambda (1\;3) \lambda^{-1} &= (2\;4).
    \end{align*}
\end{solution}
