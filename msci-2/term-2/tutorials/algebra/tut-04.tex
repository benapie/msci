\tutorial{4}{3/3}

\setcounter{question}{30}
\question
\begin{parts}
    \part 
    Show that if a subgroup $H$ of $A_4$ contains at least $7$
    different elements, then $H = A_4$.

    \begin{solution}
        By Lagrange's theorem, if $H$ is a subgroup of $A_4$ then
        $\abs H \mid \abs{A_4}$
        So
        \[
            \abs H \in \{1, 3, 4, 6, 12\}
        \]
        as $\abs{A_4} = \frac12 \abs{S_4} = 12$.
        If $\abs H \geq 7$, then $\abs H = 12$
        and so $H = A_4$.
    \end{solution}

    \part
    Find all the cyclic subgroups of $A_4$.
    \begin{solution}
        We have all products of transpositions in $A_4$:
        \begin{align*}
            \langle (1\;2)(3\;4) \rangle&, \\
            \langle (1\;3)(2\;4) \rangle&, \\
            \langle (1\;4)(2\;3) \rangle&.
        \end{align*}
        Now we have all $3$-cycles:
        \begin{align*}
            \langle (1\;2\;3) \rangle&, \\
            \langle (1\;2\;4) \rangle&, \\
            \langle (1\;3\;4) \rangle&, \\
            \langle (2\;3\;4) \rangle&.
        \end{align*}
        We have no elements of order $3$ (or above)
        since $4$-cycles are odd permutations.
        Clearly, we also have our trivial subgroup $\langle e \rangle$.
    \end{solution}

    \part 
    Show that
    \[
        \langle (1\;2\;3), (1\;2\;4) \rangle = A_4
    \]
    and
    \[
        \langle (1\;2\;3), (1\;2)(3\;4) \rangle = A_4.
    \]
    (Hint: use part (a).)
    \begin{solution}
        Let $B_1 = \langle (1\;2\;3), (1\;2\;4) \rangle$.
        We have
        \begin{align*}
            (1\;2\;3)^2 &= (1\;3\;2) \\
            (1\;2\;4)^2 &= (1\;4\;2) \\
            (1\;2\;3)(1\;2\;4) &= (1\;3)(2\;4) \\
            (1\;2\;4)(1\;2\;3) &= (1\;4)(2\;3);
        \end{align*}
        and including $e$ that is $7$ elements.
        Using part (a) we have that $B_1 = A_4$.
        Now let $B_2 = \langle (1\;2\;3), (1\;2)(3\;4) \rangle$.
        We have
        \begin{align*}
            (1\;2\;3)^2 &= (1\;3\;2) \\
            (1\;2\;3)(1\;2)(3\;4) &= (1\;3\;4) \\
            (1\;2)(3\;4)(1\;2\;3) &= (2\;4\;3) \\
            (1\;3\;4)^2 &= (1\;4\;3),
        \end{align*}
        and including $e$ that is $7$ elements.
        Using a similar argument $B_2 = A_4$.
    \end{solution}
\end{parts}

\setcounter{question}{33}
\question 
Show that $A_n$, $n \geq 3$ is generated by its 3-cycles.

\begin{solution}
    Consider a permutation
    \[
        (\sigma_1 \; \sigma_2)(\sigma_3 \; \sigma_4).
    \]
    We have that
    \begin{enumerate}
        \item if $\sigma_2 \neq \sigma_3$ then
            \[
                (\sigma_1 \; \sigma_2)(\sigma_3 \; \sigma_4)
                = (\sigma_1 \; \sigma_2 \; \sigma_3)
                  (\sigma_2 \; \sigma_3 \; \sigma_4);
            \]

        \item if $\sigma_2 = \sigma_3$ and $\sigma_1 \neq \sigma_4$ then
            \[
                (\sigma_1 \; \sigma_2)(\sigma_3 \; \sigma_4)
                = (\sigma_1 \sigma_2 \; \sigma_4); \;\text{and}
            \]

        \item if $\sigma_2 = \sigma_3$ and $\sigma_1 = \sigma_4$ then
            \[  
                (\sigma_1 \; \sigma_2)(\sigma_3 \; \sigma_4) = e.
            \]
    \end{enumerate}
    Therefore, the product of two transpositions
    can be generated by $3$-cycles in $S_n$.
    Now, let $\sigma \in A_4$. 
    We can express $\sigma$ as a product of an even amount of non-disjoint transpositions
    \[
        \sigma = (\sigma_1     \; \sigma_2) 
                 (\sigma_3     \; \sigma_4) \ldots
                 (\sigma_{k-1} \; \sigma_k)
    \]
    with $k \in 4\Z$.
    Hence, from the work before, we can generate $\sigma$ by $3$-cycles.
\end{solution}

\setcounter{question}{34}
\question 
By arguing with orders of elements, or otherwise,
show that $A_4$ has no subgroup of order $6$.
Why does this show that the converse of Lagrange's Theorem is false.

\begin{solution}
    Let $G = A_4$ and $H \subset G$ be a subgroup of order $6$.
    $\frac{\abs G}{\abs H} = 2$, so there are two cosets of $H$ in $G$,
    $aH$ (for $a \not \in H$) and $eH = H$.
    Let $a \in G$, we will prove that $a^2 \in H$.
    If $a \in H$, this is clear.
    Now let $a \not \in H$.
    Suppose $a^2 \in aH$, then $a^2 = ah$ for some $h \in H$.
    Then $a = h$, a contradiction.
    So $a^2 \in H$ for all $a \in G$.
    We have
    \[
        (\sigma^2)^2 = (\sigma^{-1})^2 = (\sigma^-1)^{-1} = \sigma
    \]
    where $\sigma$ is a $3$-cycle in $G$, 
    so every $3$-cycle is the square of an element in $G$.
    As such, every $3$-cycle in $G$ must be in $H$;
    however, there are $8$ $3$-cycles in $A_4$ and $H$ has only $6$ elements,
    a contradiction.
    Hence no such subgroup $H$ can exist.
\end{solution}
