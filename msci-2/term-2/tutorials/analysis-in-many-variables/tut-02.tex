\tutorial{2}{6/2}

\question Write the line lintegral
\[
    \int_C x\,dx + y\,dy + (xz - y)\,dz
\]
in the form $\int_C \bm v \cdot d\bm x$ for a suitable vector field $\bm v(\bm x)$,
and compute its value when $C$ is the curve given by 
\[
    \bm x(t) = t^2 \bm e_1 + 2t \bm e_2 + 4t^3 \bm e_3, \qquad t \in [0,1].
\]

\question Let $\bm A(\bm x)$ be the vector field
\[
    \bm A(x,y,z) = x\bm e_1 + y\bm e_2 + z\bm e_3.
\]
\begin{parts}
    \part Compute the line integral $\int_C \bm A \cdot d\bm x$ where $C$ is the straight line
    from the origin to the point $(1,1,1)$.
    \part Show (by finding $f$) that the vector field $\bm A$ from part (a) is equal to
    $\nabla f(\bm x)$ for some scalar field $f$, and that your answer to part (a)
    is equal to $f(1,1,1) - f(0,0,0)$.
\end{parts}

\question Show that the result from question 78 applies in general: if the vector field $\bm v(\bm x)$
in $\R^n$ is the gradient of a scalar field $f(\bm x)$, so that $\bm v = \nabla f$,
and if $C$ is a curve in $\R^n$ running from $\bm x = \bm a$ to $\bm x = \bm b$,
then $\int_C \bm v \cdot d\bm x = f(\bm b) - f(\bm a)$.

\question Let $\bm v$ be the radial vector field $\bm v(\bm x) = \bm x$.
\begin{parts}
    \part Compute $l_1 = \int_{C_1} \bm x \cdot d\bm x$ where $C_1$ is the straight-line
    contour from the origin to the point $(2,0,0)$.

    \part Compute $l_2 = \int_{C_2} \bm v \cdot d\bm x$ where $C_2$ is the semi-circular contour
    from the origin to the point $(2,0,0)$ defined by 
    $x \in [0,2]$,
    $y = \sqrt{1 - (x-1)^2}$,
    $z = 0$.

    \part You should have found that $l_1 = l)2$.
    Explain this result using Stokes' theorem.
\end{parts}

\question The paraboloid of equation $z = x^2 + y^2$ intersects the plane $z = y$ in a curve $C$.
Calcualte 
$\oint_C \bm x \cdot d\bm x$ 
for 
$\bm v = 2z \bm e_1 + x \bm e_2 + y \bm e_3$ 
using Stokes' theorem.
Check your answer by evaluating the line integral directly.
