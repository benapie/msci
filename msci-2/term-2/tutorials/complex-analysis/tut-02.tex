\tutorial{2}{13/2}

\question Compute the Taylor series expansions for the following function at $z = 0$:
\[
    f(z) = \frac{1}{(1-z)^2}. 
\]
\begin{solution}
    \[
        f'(z) = \frac{2}{(1-z)^3}, \qquad f''(z) = \frac{2 \cdot 3}{(1-z)^4} 
    \]
    so we guess
    \[
        f^{(n)}(z) = \frac{(n+1)!}{(1-z)^{n+2}}
    \]
    which is easy to confirm by induction.
    Hence
    \[
        f(z) 
        = \sum^{\infty}_{n=0} \frac{f^{(n)}(0)}{n!} \, z^n
        = \sum^{\infty}_{n=0} (n+1) \, z^n.
    \]
\end{solution}

\question Compute the following integral:
\[
    \int_{\abs z = 2} \frac{\sin^2z}{z^2} \,dz.
\]
\begin{solution}
    \begin{align*}
        I
        &= \int_{\abs z = 2} \frac{\sin^2z}{z^2} \,dz \\
        &= \frac{2 \pi i}{1!} \cdot 2\sin 0 \cos 0 \\
        &= 0.
    \end{align*}
\end{solution}

\question Compute the Taylor series expansion of $f(z) = e^z$ about an arbitrary point $z_0 \in \C$.
\begin{solution}
    \[
        f(z)
        = \sum^{\infty}_{n=0} \frac{f^{(n)}(z_0)}{n!} \, (z-z_0)^n
        = \sum^{\infty}_{n=0} \frac{e^{z_0}}{n!} \, (z-z_0)^n.
    \]
\end{solution}

\question Let $P(z)$ be a polynomial of degree at most $d$, for some $d \in \N$, and assume that
\[
    \int_{\abs z = 2} \frac{P(z)}{(n+1)z - 1} \,dz = 0,
\]
for $n = 0,1,2,\ldots,d$. 
Show that $P(z) = 0$; that is, it is the zero polynomial.
\begin{solution}
    Let $P(z) = a_dz^d + a_{d-1}z^{d-1} + \ldots + a_1z + a_0$ where $a_d \neq 0$.
    We will use the fact that
    \[
        P(z) = a_d(z-z_1)(z-z_2)\ldots(z-z_d)
    \]  
    for $z_i \in \C$ and so $\deg P \leq d$ and $P$ can have at most
    $d$ distinct complex roots.
    \[
        \int_{\abs z = 2} \frac{P(z)}{(n+1)z - 1} \,dz 
        = \int_{\abs z = 2} \frac{f(z)}{z - \frac1{n+1}} \,dz 
    \]
    where $f(z) = \frac{P(z)}{n+1}$.
    So
    \begin{align*}
        2i \pi i \, f\left(\frac1{n+1}\right) 
        &= \frac{2\pi i}{n+1} \, P\left(\frac1{n+1}\right) \\
        &= \frac{2 \pi i a_d}{n + 1}
            \left(
                \frac{1}{n+1} - z_1
            \right)
            \left(
                \frac{1}{n+1} - z_2
            \right)
            \ldots
            \left(
                \frac{1}{n+1} - z_d
            \right) \\
        &= 0
    \end{align*}
    for $n=0,1,\ldots,d$.
    Hence, $P(z)$ has $d+1$ roots but we already know that $P(z)$ can have at most $d$ roots,
    hence $P \equiv 0$.
\end{solution}

\question Find the maximum of $\abs{\sin z}$ on the square
\[
    \{z = x+iy: 0 \leq x \leq 2 \pi, 0 \leq y \leq 2\pi\} \subset \C.
\]
\begin{solution}
    We set $f(z) = \sin z$ which is holomorphic in the domain
    \[
        D = \{z = x+iy: x,y \in (0,2\pi)\}.
    \]
    Moreover, $\abs{f(z)}$ is continuous on
    \[
        \overline D = \{z = x+iy: x,y \in [0,2\pi]\}
    \]
    which is comapct; hence $\abs{f(z)}$ does obtain a maximum value.
    By the maximum modulus principle, we know this cannot happen in $D$,
    so it occurs on $\partial \overline D$.
    We observe $f$ on the boundary to see that it has a maximu mvalue of $\cosh(2\pi)$.
\end{solution}
