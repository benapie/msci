\question Duplicate the above server and run this duplicated server 
using a different socket number.
Modify the client program such that it will submit the 5 numbers to one 
of the two servers each time for computation. 
Now, you will have two identical components forming the distributed system.
\begin{parts}
    \part What extra instructions you have to implement at the client, 
    allowing it to access one of the two distributed system components?
    \begin{solution}
        When connecting to the socket, instead of using a fixed port number
        I randomly pick from a list of port numbers:
        \begin{center}
            \ttfamily
            random.choice([1200,1201]).
        \end{center}
    \end{solution}

    \part The distributed system is not replication transparent, 
    since the client explicitly knows there are two duplicated components 
    and have to establish separate socket connections to them. 
    Suggest a possible solution to enable replication transparency.
    \begin{solution}
        To ensure that the client cannot see the multiple components,
        we could create a third server that will take requests from the client
        and then forward it on to the servers that actually do the work.
        This would allow us to scale the amount of servers doing the work
        without the client knowing.
    \end{solution}
\end{parts}

\question Change the socket number of one of the server programs. 
This represents migration (or relocation) of a component in distributed system.
How do you modify the client program to cope with this change?
\begin{solution}
    One solution is to create the third server for routing traffic (as before)
    and ensuring that this server is kept updated with migrations.
    Alternatively, the client could keep a collection of \emph{possible}
    port numbers and should try them at random until a successful connection
    is made.
\end{solution}

\question Assume a server can be shutdown randomly. 
You can manually shutdown any server or both when you do the tests. 
This represents the component availability issue with a distributed system.
How do you modify your client program to cope with such a situation?
\begin{solution}
    Again, a server routing traffic would solve this problem.
    Alternatively again, having a client \emph{attempt} to connect to a server
    then error handling if a successful connection is not made would be
    sufficient, but would not be very transparent.
\end{solution}