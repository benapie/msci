\pc{1}{29/1}

\question Write down the order of each element of
\begin{parts}
    \part $\Z/9$;
    \begin{solution}
        The following elements all have order 9:
        \[
            \overline 1, \overline 2, \overline 4, \overline 5,
            \overline 7, \overline 8.
        \]
        $\overline 3$ and $\overline 6$ have order 2 and
        $\overline 0$ has order 1.
    \end{solution}

    \part $D_6$; and
    \begin{solution}
        $r$ and $r^5$ both have order $6$,
        $r^2$ and $r^4$ have order $3$, and
        $r^3$ has order $2$.
        $s$ also has order $2$.
        $r^is$ has order $2$ as $(r^is)^2 = r^isr^is = r^i r^{-i} = 1$.
    \end{solution}

    \part $\C^\times$.
    \begin{solution}
        If $\lvert z \rvert > 1$, then $\lvert z \rvert^n > 1$
        and similarly if $\lvert z \rvert < 1$ then
        $\lvert z \rvert^n < 1$; hence in both cases $\ord(z) = \infty$.
        Now consider $\lvert z \rvert = 1$.
        Clearly, if $z = 1$, then $\ord(z) = 1$.
        Let $z = e^{2i\pi\theta}$ where $\theta \in [0,1]$.
        If $\theta \not \in \Q$ then $\ord(z) = \infty$
        (we will just keep rotation around the unit circle).
        Now if $\theta = \frac{p}{q} \in Q$ such that
        $\gcd(p, q) = 1$ then $\ord(z) = q$.
    \end{solution}
\end{parts}

\question Show that the order of the element $\overline a$ of $\Z/n$
is $n/\gcd(n,a)$.
\begin{solution}
    $k = \ord(\overline a)$ iff
    $k$ is the minimum value such that $k \overline a = \overline 0$ iff
    $n \mid ka$.
    Let $d = \gcd(a, n)$, $a = db$, and $n = dm$ where $b,m \in \Z$.
    We have $n \mid ka$ iff $nx = ka$ for some $x$ iff
    $dmx = kdb$ iff $mx = kb$ iff $m \mid kb$.
    Now $\gcd(m,b) = 1$ as if a prime $p$ divided $m$ and $b$ then $db$
    would be a common divisor bigger than $d$).
    Thus $m \mid K$ and so $k = m$.
    So
    \[
        \ord(\overline a) = m = \frac{n}{d} = \frac{n}{\gcd(a,n)}.
    \]
\end{solution}

\question
\begin{parts}
    \part Show that if $x$ and $y$ are elements of finite order of a group $G$,
    and $xy = yx$, then $xy$ is also an element of finite order.
    What can you say about the order of $xy$ in terms of the 
    orders of $x$ and $y$?
    \begin{solution}
        $(xy)^r = x^r y^r$ as $xy = yx$ so if $\ord(x) = m$ and
        $\ord(y) = n$ then 
        \[
            (xy)^nm = x^{nm} y^{nm} = (x^m)^n (y^n)^m = 1
        \]
        so $\ord(xy) \leq nm$.
    \end{solution}

    \part Show that $\ord(x) = \ord(x^{-1})$.
    \begin{solution}
        \[
            \left(x^{\ord(x)}\right)^{-1} = (1)^{-1} \iff (x^{-1})^m = 1.
        \]
    \end{solution}

    \part Find a group $G$ and elements $x,y$ of $G$ such that $x$ 
    and $y$ have
    finite order yet $xy$ has infinite order.
    \begin{solution}
        Let $G$ be the set of bijections $\R \to \R$ under composition.
        Let $f(t) = -t$ and $g(t) = 1 - t$. Then $f^2 = t$ (identiy)
        hence $\ord(f) = 2$ and $g^2 = t$ (identity) so
        $\ord(g) = 2$.
        Now consider
        \[
            (f \circ g)(t) = t - 1.
        \]
        Clearly $(f \circ g)^m(t) = t - m \neq t$ so
        $\ord(f \circ g) = \infty$.
    \end{solution}
\end{parts}
