\assignment{6}{6/3}

\setcounter{question}{1}
\question 
Let $f$ be a non-constant entire function, and set $g(z) = f\left(\frac1z\right)$.
Note that $g(z)$ gives a holomorphic function on $\C^\star$.
\begin{parts}
    \part 
    Show that $g$ \emph{cannot} be bounded near $z = 0$, so in particular does
    not extend to a holomorphic function on all of $\C$.
    \begin{solution}
        If $g$ is bounded near $z = 0$, $f$ is bounded near $z = \infty$
        and by Liouville's theorem would imply that $f$ is constant,
        which is a contradiction.
    \end{solution}

    \part 
    Let $f$ be a polynomial of degree $n$.
    Show that $g$ has a pole of order $n$ at $z = 0$.
    \begin{solution}
        Let $f(z) = a_0 + a_1z + \ldots + a_nz^n$.
        Then
        \[
            g(z) 
            = a_0 + \frac{a_1}z + \ldots + \frac{a_n}{z^n}
            = \sum^{0}_{i = -n} a_{-i} z^n
        \]
        and so $g$ has a pole of order $n$ at $z = 0$.
    \end{solution}

    \part Assume that $f$ is \emph{transcendental}, that is, $f$ is not
    a polynomial, so in the power series expansion 
    \[
        f(z) = \sum^{\infty}_{n=0} a_nz^n
    \]
    we have infinitely many $a_n \neq 0$.
    Show that $g$ has an essential singularity at $z = 0$.
    \begin{solution}
        \[
            g(z) = \sum^{\infty}_{n=0} a_n \left(\frac1z\right)^n = \sum^{0}_{n = -\infty} a_{-n} z^n.
        \]
        As $f$ has infinitely many $a_n \neq 0$ for $n > 0$, 
        then $g$ has infinitely many $a_{-n} \neq 0$ for $n < 0$;
        therefore, $g$ has an essential singularity at $z = 0$.
    \end{solution}

    \part Apply the Big Picard Theorem to $g$ to deduce that if $f$ is not a polynomial,
    then the image of $f$ contains $\C \setminus \{b\}$ for some $b \in \C$.
    If $f$ is a non-constant polynomial, what is the image of $f$?
    \begin{solution}
        $f$ is not a polynomial hence $g$ has an essential singularity at $0$.
        By the Big Picard Theorem, there exists a $b \in \C$ such that
        \[
            g(\C^\star) \supset \C \setminus \{b\}.
        \]
        But $g(\C^\star) = \C = \im f$ so
        \[
            \im f \supset \C \setminus \{b\}
        \]
        as required.
        If $f$ is a non-constant polynomial, the image of $f$ is $\C$; proof as follows.
        Let $w \in \C$. Then $g(z) = f(z) - w$ is also a non-constant polynomial,
        and by the fundamental theorem of algebra we have that there exists
        a $z_0$ such that $g(z_0) = 0$.
        Hence $f(z_0) = w$.
        Thus we have shown that the image of any non-constant polynomial is $\C$.
    \end{solution}
\end{parts}

\question Determine all zeros and pole with their orders of
\begin{parts}
    \setcounter{partno}{3}
    \part
    \[
        \frac{1}{z^2} + \frac{1}{z^2 + 1}.
    \]
    \begin{solution}
        Let
        \[
            f(z) = \frac1{z^2} + \frac1{z^2+1}.
        \]
        $f$ has a pole of order $2$ at $0$
        and two simple poles at $i$ and $-i$.
        We have
        \begin{align*}
            \Res_{z = 0} (f)
            &= \Res_{z=0} \left(\frac1{z^2}\right) + \Res_{z=0} \left(\frac1{z^2 + 1}\right)
            = 0 \\
            \Res_{z=i} (f)
            &= \Res_{z=i} \left(\frac1{z^2}\right) + \Res_{z=i} \left(\frac1{z^2 + 1}\right)
            = 0 + \lim_{z \to i} \left(\frac1{z+i}\right)
            = \frac{-i}2 \\
            \Res_{z=-i} (f)
            &= \Res_{z=-i} \left(\frac1{z^2}\right) + \Res_{z=-i} \left(\frac1{z^2 + 1}\right)
            = 0 + \lim_{z \to -i} \left(\frac1{z-i}\right)
            = \frac i2.
        \end{align*}
    \end{solution}
\end{parts}
Find the residues at the poles.

\question Determine the integral over the unit circle of
\begin{parts}
    \part
    \[
        \frac{1 - \cos z}{z^2}.
    \]
    \begin{solution}
        \begin{align*}
            \int_{\abs z = 1} \frac{1 - \cos z}{z^2} \,dz = 0.
        \end{align*}
        This is clear as we have a pole of order $2$ at $0$,
        and our residue at this point is $\Res_{z = 0} = 0$
        (from our fourth method of calculating residues).
        We can also apply Cauchy's integral formula for derivatives
        to get the same answer.
    \end{solution}
\end{parts}
