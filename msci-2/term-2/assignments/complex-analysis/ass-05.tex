\assignment{5}{28/2}

\setcounter{question}{0}
\question Determine all zeros and poles with their orders of the following functions:
\begin{parts}
    \part $\frac{z^3 - z^2}{z+1}$.
    \begin{solution}
        \[
            f(z) = \frac{z^3 - z^2}{z + 1} = \frac{z^2(z-1)}{z+1}.
        \]
        $f$ has a pole of order $1$ at $-1$,
        $f$ has a zero of order $2$ at $0$, and
        $f$ has a zero of order $1$ at $1$.
    \end{solution}
\end{parts}

\setcounter{question}{2}
\question Assume that $f$ has a pole of order $m$ at $a$ and $g$ has a pole of order $n$ at $a$.
What kind of singularity at $a$ is it possible for $f + g$ to have?
Give examples to show that all the possibilities that you list can occur.
\begin{solution}
    Let
    \begin{align*}
        f(z) &= (z-a)^{-m}f_1(z) \\
        g(z) &= (z-a)^{-n}g_1(z).
    \end{align*}
    Let us consider some different cases.
    \begin{enumerate}
        \item 
            If $m > n$, then we have
            \[
                (f+g)(z) = (z-a)^{-m} \left(f_1(z) + (z-a)^{m-n} g_1(z)\right)
            \]
            and so $f+g$ has a pole of order $m$ at $a$.

        \item 
            If $m < n$, similarly we see that $f + g$ has a pole of order $n$ at $a$.

        \item 
            If $m = n$, this becomes a bit more specific.
            We have
            \[
                (f+g)(z) = (z-a)^{-m} (f_1(z)+g_1(z)).
            \]
            The singularities of $(f+g)(z)$ at $a$ is determined
            by the behaviour of $(f_1 + g_1)(z)$.
            \begin{enumerate}
                \item If $(f_1+g_1)(z)$ does not have a zero at $a$,
                    then $(f+g)(z)$ has a pole of order $m$ at $a$.

                \item If $(f_1+g_1)(z)$ has a zero of order $k < m$ at $a$,
                    then $(f+g)(z)$ has a pole of order $m - k$ at $a$.

                \item If $(f_1+g_1)(z)$ has a zero of order $k \geq m$ at $a$,
                    then $(f+g)(z)$ has a removable singularity at $a$.
            \end{enumerate}
    \end{enumerate}
\end{solution}

\question What are the annuli of convergence of the following Laurent series?
\begin{parts}
    \setcounter{partno}{1}
    \part $ \displaystyle\sum^{\infty}_{n = -\infty} \frac{z^n}{\abs n!} $.
    \begin{solution}
        \begin{align*}
            \sum^{\infty}_{n=-\infty} \frac{z^n}{\abs n!} 
            &= \sum^{\infty}_{n=1} \frac{z^{-n}}{n!} + \sum^{\infty}_{n=0} \frac{z^n}{n!} \\
            &= -1 + \sum^{\infty}_{n=0} \frac{z^{-n}}{n!} + \sum^{\infty}_{n=0} \frac{z^n}{n!} \\
            &= -1 + \exp\left(\frac1z\right) + \exp(z)
        \end{align*}
        and so the annulus of convergence is $\C^\star$ (with an essential singularity at $z = 0$.
    \end{solution}
\end{parts}

\question Determine the Laurent series expansion of $\displaystyle f(z) = \frac{z}{1+z^2}$
\begin{parts}
    \part on the annulus $0 < \abs z < 1$.
    \begin{solution}
        \begin{align*}
            f(z) 
            &= z\left(\frac{1}{1 - \left(-z^2\right)}\right) \\
            &= z (1 + (-z^2) + (-z^2)^2 + \ldots) \\
            &= z - z^3 + z^5 - \ldots \\
            &= \sum^{\infty}_{n=0} z^{2n+1}.
        \end{align*}
    \end{solution}
\end{parts}
