\assignment{3}{27/2}

\setcounter{question}{24}
\question Prove that none of the following groups are isomorphic to one another:
\[
    \Z/8, \qquad \Z/4 \times \Z/2, \qquad D_4, \qquad S_4.
\]

\begin{solution}
    \begin{description}
        \item[$\Z/8 \not \cong \Z/4 \times \Z/2$] \hfill

            $\Z/8$ has four elements 
            $\overline 1$, 
            $\overline 3$, 
            $\overline 5$, and
            $\overline 7$ of order $8$.
            $\Z/4 \times \Z/2$ has no such elements.

        \item[Abelian to non-abelian] \hfill

            As $\Z/8$ and $\Z/4 \times \Z/2$ are abelian
            and $D_4$ and $S_4$ is not, neither of the abelian groups
            can be isomorphic to either of the non-abelian groups.

        \item[$D_4 \not \cong S_4$] \hfill

            $D_4$ has $8$ elements while $S_4$ has $24$ elements.
    \end{description}
\end{solution}

\setcounter{question}{28}
\question Determine the well-known group which the quotient
\[
    (\Z/2 \times \Z/4) / \langle (\overline 1, \overline 2) \rangle
\]
is isomorphic to.

(Hint: write down the elements as cosets, compute their orders, and deduce the
structure of the group.)

\begin{solution}
    \begin{align*}
        (\overline 0, \overline 0) + H &= \{(\overline 1,\overline 2), (\overline 0, \overline 0)\} \\
        (\overline 0, \overline 1) + H &= \{(\overline 1,\overline 3), (\overline 0, \overline 1)\} \\
        (\overline 0, \overline 2) + H &= \{(\overline 1,\overline 0), (\overline 0, \overline 2)\} \\
        (\overline 0, \overline 3) + H &= \{(\overline 1,\overline 1), (\overline 0, \overline 3)\},
    \end{align*}
    this includes all of our elements and so we are looking for a group of 4 elements each with
    order 2.
    That is,
    the Klein 4-group $K_4$.
\end{solution}

\setcounter{question}{29}
\question Let $G$ be a group, 
$H \subset G$ an arbitrary subgroup, 
and $N \subset G$ a normal subgroup.
Let
\[
    HN = \{ hn: h \in H, n \in N \}.
\]

\begin{parts}
    \part Show that $HN$ is a subgroup of $G$.

    \begin{solution}
        \begin{description}
            \item[Identity] \hfill

                Let $e$ be the identity element of $G$.
                $e \in H$ and $e \in N$ as they are both subgroups,
                hence $e = ee \in HN$.
            
            \item[Closure]

                Let $h_1, h_2 \in H$ and $n_1, n_2 \in N$.
                \[
                    (h_1n_1)(h_2n_2)
                    = h_1 h_2 (h_2^{-1}n_1h_2) n_2
                \]
                as $H$ is a subgroup, $h_1 h_2 \in H$.
                As $N$ is a normal subgroup, $h_2^{-1}n_1h_2 \in N$
                and as it is a subgroup $(h_2^{-1}n_2h_2)n_2 \in N$;
                hence, we have closure.

            \item[Existence of inverse] \hfill 

                Let $h \in H$ and $N \in N$.
                Then
                \[
                    (hn)^{-1} = n^{-1} h^{-1} = h^{-1} h n^{-1} h^{-1} = h^{-1} (hn^{-1}h^{-1}) \in HN.
                \]
        \end{description}
    \end{solution}

    \part Show that $H \cap N$ is a normal subgroup of $H$.

    \begin{solution}
        $H \cap N$ is clearly a subgroup of $H$, but we must show that it is \emph{normal}.
        Let $x \in H \cap N$ and $h \in H$.
        Then
        \[
            hxh^{-1} \in H
        \]
        as $H \cap N \subset H$, and we also have
        \[
            hxh^{-1} \in N
        \]
        as $H \cap N \subset N$;
        therefore, $H \cap N$ is a normal subgroup of $H$.
    \end{solution}

    \part Show that $HN/N$ is isomorphic to $H/(H \cap N)$.

    (Hint: define a map from the latter group to the first and hope that it's an isomorphism.)

    \begin{solution}
        Let us define $\varphi: H \to HN/N$ by $\varphi(h) = hN$.
        Let $h_1, h_2 \in H$. 
        \[
            \varphi(h_1h_2) = (h_1h_2)N = (h_1N)(h_2N) = \varphi(h_1)\varphi(h_2)
        \]
        and so $\varphi$ is a homomorphism.
        We have
        \begin{align*}
            \ker\varphi 
            &= \{ h \in H : \varphi(h) = e_{HN/N} \} \\
            &= \{ h \in H : \varphi(h) = N \} \\
            &= \{ h \in H : hN = N \} \\ 
            &= \{ h \in H: h \in N \} \\
            &= H \cap N.
        \end{align*}
        Therefore, by the first isomorphism
        \[
            HN/N \cong H/(H \cap N).
        \]
    \end{solution}
\end{parts}
