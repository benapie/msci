\assignment{2}{10/2}

\setcounter{question}{14}
\question Express each of the following permutations in $S_8$
as a product of disjoint cycles and a product of transpositions:
\begin{parts}
    \part
        $
            \begin{pmatrix}
                1 & 2 & 3 & 4 & 5 & 6 & 7 & 8 \\
                7 & 6 & 4 & 1 & 8 & 2 & 3 & 5 \\
            \end{pmatrix};
        $
    \begin{solution}
        \begin{align*}
            \begin{pmatrix}
                1 & 2 & 3 & 4 & 5 & 6 & 7 & 8 \\
                7 & 6 & 4 & 1 & 8 & 2 & 3 & 5 \\
            \end{pmatrix}
            &= (1\;7\;3\;4)(2\;6)(5\;8) \\
            &= (1\;7)(7\;3)(3\;4)(2\;6)(5\;8).
        \end{align*}
    \end{solution}

    \part
        $
            (4\;5\;6\;8)(1\;2\;4\;5);
        $
    \begin{solution}
        \begin{align*}
            (4\;5\;6\;8)(1\;2\;4\;5)
            &= (1\;2\;5)(4\;6\;8) \\
            &= (1\;2)(2\;5)(4\;6)(6\;8).
        \end{align*}
    \end{solution}

    \part
        $
            (6\;2\;4)(2\;5\;3)(8\;7\;6)(4\;5).
        $
    \begin{solution}
        \begin{align*}
            (6\;2\;4)(2\;5\;3)(8\;7\;6)(4\;5)
            &= (2\;5\;6\;8\;7)(4\;3) \\
            &= (2\;5)(5\;6)(6\;8)(8\;7)(4\;3).
        \end{align*}
    \end{solution}
\end{parts}

\setcounter{question}{17}
\question Write down all the elements of $S_4$ (in cycle form),
and compute the order of each element.
\begin{solution}
    Let's categorise the elements by order.
    Let $\sigma \in S_4$.
    We have $\ord{\sigma} \in \{1,2,3,4\}$.
    \begin{description}
        \item[$\ord{\sigma} = 1$]
            Here we only have the identity permutation, so
            \[
                (1)(2)(3)(4).
            \]

        \item[$\ord{\sigma} = 2$]
            We have the single tranposition elements:
            \[
                (1\;2), (1\;3), (1\;4), (2\;3), (2\;4), (3\;4).
            \]
            and also the permutations made up of multiple transpositions:
            \[
                (1\;2)(3\;4), (1\;3)(2\;4), (1\;4)(2\;3).
            \]

        \item[$\ord\sigma = 3$]
            We have
            \[
                (1\;2\;3), (1\;3\;2),
                (1\;2\;4), (1\;4\;2),
                (1\;3\;4), (1\;4\;3),
                (2\;3\;4), (2\;4\;3).
            \]

        \item[$\ord\sigma = 4$]
            We have
            \[
                (1\;2\;3\;4), (1\;2\;4\;3),
                (1\;3\;2\;4), (1\;3\;4\;2),
                (1\;4\;2\;3), (1\;4\;3\;2).
            \]
    \end{description}
\end{solution}

\question Let $G$ be a finite group of order $n$.
Use Lagrange's theorem to show that for every $g \in G$ we have
\[
    g^n = 1.
\]
\begin{solution}
    It is known that the order of an element $g \in G$
    is equal to the order of the cyclic group it generates.
    So let $H = \{1, g, g^2, \ldots\} \subset G$ be a subgroup.
    Then by Lagrange's theorem
    \[
        n = m \ord(H) = m\ord(g)
    \]
    for some $m \in \Z$.
    Then
    \[
        g^n = g^{m\ord(g)} = (g^{\ord(g)})^m = 1^m = 1.
    \]
\end{solution}

\question Let 
$H = \{1, (1\;2\;3\;4), (1\;3)(2\;4), (1\;4\;3\;2)\} \subset S_4$.
\begin{parts}
    \part Show that $H$ is a subgroup of $S_4$.
    \begin{solution}
        We have $1 \in H$.
        Let
        \[
            \sigma_1 = (1\;2\;3\;4), 
            \qquad \sigma_2 = (1\;3)(2\;4),
            \qquad \sigma_3 = (1\;4\;3\;2).
        \]
        Then
        \begin{align*}
            \sigma_1\sigma_2 &= \sigma_3 &
            \sigma_2\sigma_1 &= \sigma_3 \\
            \sigma_2\sigma_3 &= \sigma_1 &
            \sigma_3\sigma_2 &= \sigma_1 \\
            \sigma_1\sigma_3 &= 1 &
            \sigma_3\sigma_1 &= 1 \\
            \sigma_1^{-1}    &= \sigma_3 &
            \sigma_2^{-1}    &= \sigma_2 \\
            \sigma_3^{-1}    &= \sigma_1;
        \end{align*}
        hence, $H$ is closed under composition and
        there exists an inverse for every element.
        Therefore, $H$ is a subgroup of $S_4$.
    \end{solution}
\end{parts}

\setcounter{question}{23}
\question Decompose $D_6$ into left cosets with respect to
the subgroup
$\{e, r^3, s, sr^3\}$.
Is every left coset also a right coset?
\begin{solution}
    Let $H = \{e, r^3, s, sr^3\}$.
    We know that cosets are either disjoint or equal and that
    the union of all cosets of a subgroup form the entire group.
    With this in mind, we have
    \[
        H = eH = \{e, r^3, s, sr^3\}
    \]
    and
    \[
        rH = \{r, r^4, rs, r^4s\}
    \]
    and
    \[
        r^2H = \{r^2,r^5,r^2s,r^5s\}.
    \]
    As 
    $
        \lvert eH \rvert + \lvert rH \rvert + \lvert r^2H \rvert 
        = 12 = \lvert D_6 \rvert,
    $
    we have all our disjoint cosets.
    Any other coset is not unique.
    We also have
    \[
        rH = Hr^2, \qquad r^2H = Hr
    \]
    so yes, all left cosets correspond to an equivalent right coset.
\end{solution}
