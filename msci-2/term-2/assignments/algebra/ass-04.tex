\assignment{4}{9/3}

\setcounter{question}{31}
\question 
Find two elements in $A_4$ which generate a non-trivial
normal subgroup $N$.
\begin{solution}
    \[
        (1\;2)(3\;4), (1\;3)(2\;4).
    \]
    Why is normal? Well
    \[
        \langle (1\;2)(3\;4), (1\;3)(2\;4) \rangle =
        \langle
            e,
            (1\;2)(3\;4),
            (1\;3)(2\;4),
            (1\;4)(2\;3)
        \rangle.
    \]
    Now we know for all $\lambda \in A_4$
    \[
        \lambda (a\;b)(c\;d) \lambda^{-1}
        = (\lambda(a)\;\lambda(b))(\lambda(c)\;\lambda(d)) \in A_4.
    \]
    hence, it must be normal.
\end{solution}

\setcounter{question}{32}
\question
Let $G$ be a finite group and $H \subset G$ a subgroup.
Suppose that $\abs H = \frac12 \abs G$.
\begin{parts}
    \part 
    How many left cosets with respect to $H$ are there in $G$,
    and how many right cosets? 

    (Justify your answer.)
    \begin{solution}
        Cosets always have the same cardinality;
        hence, there are two left cosets
        (the same applies to right cosets).
    \end{solution}

    \part
    Using the above, show that $H$ must be normal in $G$.
    \begin{solution}
        Consider our two cosets $g_1H$ and $g_2H$.
        We set $g_1 = 1$.
        Now we see that for all $h \in H$
        $hH = 1H = H = H1 = Hh$
        and that for all $g_2 \in G \setminus H$. 
        $g_2H = G \setminus H = Hg_2$.
        Therefore, $H$ must be normal.
    \end{solution}
\end{parts}

\setcounter{question}{36}
\question
Let $H$ be a subgroup of a group $G$.
Verify that the formula
\[
    (h,h')(x) = hxh'^{-1}
\]
defines an action of $H \times H$ on $G$.
Find the orbit and stabiliser of each element of $G$
when $G = D_4$ and $H = \{e, s\}$.
\begin{solution}
    First we prove that the formula defines an action:
    \begin{align*}
        (h, h')((k,k')(x))
        &= (h,h')(kxk'^{-1}) \\
        &= hkxk^{-1}h^{-1} \\
        &= (hk,h'k')(x) \\
        &= ((h,h')(k,k'))(x);
    \end{align*}
    therefore, this defines a group action.
    \begin{center}
        \begin{tabular}{lll}
            \toprule
            $x$ & $\Orb(x)$ & $\Stab(x)$ \\
            \midrule
            $e$ & $\{e,s\}$ & $\{(e,e),(s,s)\}$ \\
            $r$ & $\{r,r^3,rs,r^3s\}$ & $\{(e,e)\}$ \\
            $r^2$ & $\{r^2,r^2s\}$ & $\{(e,e),(s,s)\}$ \\
            $r^3$ & $\{r,r^3,rs,r^3s\}$ & $\{(e,e)\}$ \\
            $s$ & $\{e,s\}$ & $\{(e,e),(s,s)\}$ \\
            $rs$ & $\{r,r^3,rs,r^3s\}$ & $\{(e,e)\}$ \\
            $r^2s$ & $\{r^2,r^2s\}$ & $\{(e,e),(s,s)\}$ \\
            $r^3s$ & $\{r,r^3,rs,r^3s\}$ & $\{(e,e)\}$ \\
            \bottomrule
        \end{tabular}
    \end{center}
\end{solution}

\setcounter{question}{39}
\question
Let $G$ be the subgroup of $S_8$ generated by 
$(1\;2\;3)(4\;5)$ and $(7\;8)$.
Then $G$ acts on the set $X = \{1, 2, \ldots, 8\}$ by
$n \mapsto \sigma(n)$, for $\sigma \in G$ and $n \in X$.
Work out the elements of $G$, 
and calculate the orbit and the stabiliser of every element in $X$.
\begin{solution}
    \begin{align*}
        G = \{
        &e, (1\;2\;3)(4\;5), (7\;8), (1\;3\;2), (4\;5), (1\;2\;3)(4\;5)(7\;8), \\
        &(1\;3\;2)(7\;8), (4\;5)(7\;8), (1\;2\;3), (1\;2\;3)(7\;8), (1\;3\;2)(4\;5)
        \}.
    \end{align*}
    Now for orbits we have
    \begin{align*}
        \Orb(1) = \Orb(2) = \Orb(3) 
        &= \{1,2,3\} \\
        \Orb(4) = \Orb(5) 
        &= \{4,5\} \\
        \Orb(7) = \Orb(7)
        &= \{7,8\} \\
        \Stab(1) = \Stab(2) = \Stab(3)
        &= \{e, (4\;5), (7\;8), (4\;5)(7\;8)\} \\
        \Stab(4) = \Stab(5)
        &= \{e, (1\;2\;3), (1\;3\;2), (7\;8), (1\;2\;3)(7\;8), (1\;3\;2)(7\;8)\} \\
        \Stab(7) = \Stab(8)
        &= \{e, (1\;2\;3), (1\;3\;2), (4\;5), (1\;2\;3)(4\;5), (1\;3\;2)(4\;5)\}.
    \end{align*}
\end{solution}
