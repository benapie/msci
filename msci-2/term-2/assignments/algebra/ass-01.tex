\assignment{1}{27/1}

\setcounter{question}{0}
\question
\begin{parts}
    \part Let $G$ be a group and $g, h \in G$.
    Show that
    \[
        (gh)^{-1} = h^{-1}g^{-1}.
    \]
    \begin{solution}
        As $G$ is a group, for all $g \in G$
        there exists $g^{-1}$ such that $gg^{-1} = 1$.
        Then
        \begin{align*}
            (gh)(gh)^{-1} &= 1 \\
            h(gh)^{-1}    &= g^{-1} \\
            (gh)^{-1}     &= h^{-1}g^{-1}.
        \end{align*}
    \end{solution}

    \part Let $G$ be a group such that $x^2 = 1$ for all $x \in G$.
    Show that $G$ is abelian.
    \begin{solution}
        $xx = 1 \implies x = x^{-1}$.
        Hence, using the fact that $(gh)^{-1} = h^{-1}g^{-1}$ we see that
        \[
            xy = yx
        \]
        for all $x, y \in G$ and so $G$ is abelian.
    \end{solution}
\end{parts}

\setcounter{question}{1}
\question
\begin{parts}
    \part Let $r$ and $s$ be the generators in $D_{10}$ 
    ($r$ is the rotation, $s$ is the reflection).
    Find the positive integers $i, j$ such that
    \[
        r^5sr^{-4}s^4sr^{-3}s = r^is^j.
    \]
    \begin{solution}
        \begin{align*}
            r^5sr^{-4}s^4sr^{-3}s
            &= r^5sr^{-4}s^4r^3 \\
            &= r^5r^4s^3r^3 \\
            &= r^5r^4sr^3 \\
            &= r^5r^4sr^3ss \\
            &= r^5r^4r^{-3}s \\
            &= r^6s.
        \end{align*}
    \end{solution}

    \part Determine the order of the element $r^is \in D_n$, where $i \in \Z$.
    \begin{solution}
        $r^is$ is a rotation followed by a reflection,
        performing this twice will return us to our original position;
        hence, $\ord{(r^is)} = 2$.
    \end{solution}
\end{parts}

\setcounter{question}{2}
\question Show that any element in $D_n$ can be written \emph{uniquely} in the form
$r^as^b$ with $0 \leq a \leq n - 1$ and $0 \leq s \leq 1$.
\begin{solution}
    Assume $r^as_b=r^cs^d$. 
    Then
    \[
        r^{a - c} = s^{b - d}
    \]
    which, if $a - c \neq 0$ and $b - d \neq 0$ is \emph{impossible}
    because a reflection can never equal a rotation
    (otherwise $D_n$ would be generated by $r$).
    If $b - d = 0$, we would have $r^{a - c} = 1$
    which can only occur if $a - c = 0$ 
    (then follow the same reasoning as above).
    Finally, if $a - c = 0$,
    then $s^{b - d} = 1$
    with $b,d \in \{0,1\}$,
    so clearly $b = d$.
    Hence we have shown that any element in $D_n$ can be written
    uniquely in the form above.
\end{solution}

\setcounter{question}{3}
\question The \emph{centre} $Z(G)$ of a group $G$ is the set of elements
in $G$ which commute with all other elements, that is,
\[
    Z(G) = \{g \in G: gh = hg, \;\forall\; h \in G\}.
\]
\begin{parts}
    \part Show that $Z(G)$ is an abelian group.
    \begin{solution}
        Let $g, h \in Z(G)$.
        $Z(G)$ is a subgroup of $G$ as
        \begin{enumerate}
            \item $1 \cdot g = g \cdot 1 \implies 1 \in Z(G)$,
            \item clearly $Z(G)$ is closed; and
            \item as $gg^{-1} = g^{-1}g$ for all $g \in G$,
                then $g^{-1} \in Z(G)$ for all $g \in Z(G)$.
        \end{enumerate}

        From question 1:
        \begin{align*}
            (gh)^{-1}    &= (hg)^{-1} \\
            h^{-1}g^{-1} &= g^{-1} h^{-1} \\
            g^{-1}, h^{-1} &\in Z(G).
        \end{align*}
        Commutativity carries over from $G$; hence,
        $Z(G)$ is an abelian group.
    \end{solution}

    \part Find $Z(D_6)$.
    \begin{solution}
        If $r_i \in D_6$ we have
        \begin{align*}
            r^ir^js &= r^jsr^i \\
            r^i &= r^jsr^isr^{-j} \\
                &= r^jr^{-i}r^{-j} \\
                &= r^{-i}
        \end{align*}
        so we have $i \in \{0,3\}$.
        So
        \[
            Z(D_6) = \{1,r^3\}.
        \]
    \end{solution}
    
    \part Find $Z(D_n)$ for any $n \geq 3$.
    \begin{solution}
        Following a similar reasoning to the previous part,
        we have
        \[
            Z(D_n) =
            \begin{cases}
                \{1, \frac n2\} & \text{if $n$ is even}, \\
                \{1\} & \text{otherwise}. \\
            \end{cases}
        \]
    \end{solution}
\end{parts}
