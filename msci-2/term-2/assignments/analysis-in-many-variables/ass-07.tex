\assignment{7}{12/3}

\setcounter{question}{105}
\question Let $A$ be the region $0 \leq y \leq x$ and $0 \leq x \leq 1$.
Evaluate
\[
    \int_A (x+y) \,dx\,dy
\]
by making the change of variables $x = u + v$, $y = u - v$.
Check your answer by evaluating the integral directly.
\begin{solution}
    We have $x(u,v) = u + v$ and $y(u,v) = u - v$.
    \[
        f(x,y) = x + y, \qquad f(u,v) = 2u.
    \]
    So 
    \[
        J =
        \begin{vmatrix}
            x_u & x_v \\
            y_u & y_v \\
        \end{vmatrix}
        =
        \begin{pmatrix}
            1 & 1 \\
            1 & -1 \\
        \end{pmatrix}
        = -2.
    \]
    In our orignial coordinate system, our region is bounded by 
    the positive $x$ axis,
    the line $x = 1$, 
    and the line $y = x$.
    Applying our transformation we get
    \begin{align*}
        y &= 0 && \to & u &= v     \\
        x &= 1 && \to & u &= 1 - v \\
        y &= x && \to & v &= 0.        
    \end{align*}
    Hence
    \begin{align*}
        I
        &= \int_A (x + y) \,dx\,dy \\
        &= \int_B 2u \abs J \,du\,dv \\
        &= 4
            \left( 
                \int_0^{\frac12} 
                \int_0^{u} u \,dv\,du
                + \int_{\frac12}^1
                \int_0^{1-u} u \,dv\,du
            \right) \\
        &= 4
            \left(
                \int_0^{\frac12} 
                u^2 \,du
                + \int_{\frac12}^1
                (u - u^2) \,du
            \right) \\
        &= 4
            \left(
                \frac13 \left(\frac12\right)^3 +
                \left(
                    \frac12 - \frac13
                \right) -
                \left(
                    \frac12 \left(\frac12\right)^2
                    - \frac13 \left(\frac12\right)^3
                \right)
            \right) \\
        &= \frac12.
    \end{align*}
\end{solution}

\setcounter{question}{110}
\question $T(x, t)$ is given by
\[
    T(x,t)
    = \sum^{\infty}_{n = 1} 
    c_n \exp\left(-t\left(\frac{n\pi}L\right)^2\right)
    \sin\left(\frac{n\pi x}L\right)
\]
for $0 \leq x \leq L$.
Assume that you can interchange summation and differentation.
\begin{parts}
    \part Show that $T$ satisfies the heat (diffusion) equation
    \[
        \frac{\partial T}{\partial t} = \frac{\partial^2 T}{\partial x^2}.
    \]
    \begin{solution}
        \begin{align*}
            \frac{\partial T}{\partial t} 
            &= \sum^{\infty}_{n=1} c_n \exp\left(-t\left(\frac{n\pi}L\right)^2\right)
            \left(-\left(\frac{n\pi}L\right)^2\right) \sin\left(\frac{n\pi x}L\right) \\
            \frac{\partial T}{\partial x}
            &= \sum^{\infty}_{n=1} c_n \exp\left(-t\left(\frac{n\pi}L\right)^2\right)
            \cos\left(\frac{n\pi x}L\right) \left(\frac{n\pi}L\right) \\
            \frac{\partial^2 T}{\partial x^2}
            &= \sum^{\infty}_{n=1} c_n \exp\left(-t\left(\frac{n\pi}L\right)^2\right)
            \left(-\sin\left(\frac{n\pi x}L\right)\right)\left(\frac{n\pi}L\right)^2
        \end{align*}
        so clearly 
        $ \frac{\partial T}{\partial t} = \frac{\partial^2 T}{\partial x^2} $.
    \end{solution}

    \part Use the Fourier trick to find the constants $c_n$ if, at $t = 0$,
    $T$ is given by $T = \cos\left(\frac{\pi x}L\right)$.
    For $t > 0$, what boundary condition does this solution satisfy at $x = 0$
    and $x = L$?
    (You might want to use the identity $\sin A \cos B = \frac12(\sin(A + B) + \sin(A - B))$
    to find the coefficients in the Fourier expansion.
    \begin{solution}
        \[
            c_n 
            = \frac 2L \int_0^L 
                \cos\left(\frac{\pi x}L\right) 
                \sin\left(\frac{n \pi x}L\right) 
                \, dx.
        \]
        We have
        \[
            \sin\left(\frac{n \pi x}L\right) 
            \cos\left(\frac{\pi x}L\right)
            = \frac12
                \left(
                    \sin\left(\frac{(n+1)\pi x}L\right)
                    + \sin\left(\frac{(n-1)\pi x}L\right)
                \right)
        \]
        so
        \begin{align*}
            c_n 
            &= \frac 2L \int_0^L 
                \cos\left(\frac{\pi x}L\right) 
                \sin\left(\frac{n \pi x}L\right) 
                \,dx \\
            &= \frac 1L \int_0^L
                \left(
                    \sin\left(\frac{(n+1)\pi x}L\right)
                    + \sin\left(\frac{(n-1)\pi x}L\right)
                \right) \,dx \\
            &= \frac 1L
                \left(
                    \left(\frac L{(n+1)\pi}\right)
                    \left(-\cos\left(\frac{(n+1)\pi x}L\right)\right)
                    +
                    \left(\frac L{(n-1)\pi}\right)
                    \left(-\cos\left(\frac{(n-1)\pi x}L\right)\right)
                \right)_0^L \\
            &= \frac 1L
                \left(
                    \left(\frac L{(n+1)\pi}\right)
                    \left(1 - \cos\left((n+1)\pi\right)\right)
                    +
                    \left(\frac L{(n-1)\pi}\right)
                    \left(1 - \cos\left((n-1)\pi\right)\right)
                \right) \\
            &= \frac{1 - (-1)^{n+1}}{\pi}
                \left(
                    \frac1{n+1} + \frac1{n-1}
                \right) \\
            &= \frac{2n \, (1 - (-1)^{n+1})}{\pi(n^2-1)}.
        \end{align*}
    \end{solution}
\end{parts}
