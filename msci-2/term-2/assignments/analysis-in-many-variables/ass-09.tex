\assignment{9}{?}

\setcounter{question}{73}
\question Let $B$ be the region bounded by the five planes
$x = 0$, $y = 0$, $z = 0$, $x + y = 1$, and $z = x + y$.
\begin{parts}
    \part Find the volume of $B$.
    \begin{solution}
        \begin{align*}
            \text{volume}_B
            &= \int_B \,dV \\
            &= \int_0^1 dx \int_0^{1-x} dy \int_0^{x+y} dz \\
            &= \int_0^1 dx \int_0^{1-x} dy (x+y) \\
            &= \int_0^1 \left(xy + \frac12y^2\right)^{y=1-x}_{y=0} dx \\
            &= -\frac12 \int_0^1 (x^2 - 1) \,dx \\
            &= \frac13.
        \end{align*}    
    \end{solution}

    \part Evaluate $\int_B x \,dV$.
    \begin{solution}
        \begin{align*}
            \int_B x \,dV
            &= \int_0^1 dx \int_0^{1-x} dy \int_0^{x+y} dz \, (x) \\
            &= \int_0^1 dx \int_0^{1-x} dy \, x(x+y) \\
            &= \int_0^1 dx \left(x^2y+\frac12xy^2\right)^{y={1-x}}_{y=0} \\
            &= -\frac12\int_0^1 x(x^2 - 1) \,dx \\
            &= \frac18.
        \end{align*}
    \end{solution}

    \part Evaluate $\int_B y \,dV$.
    \begin{solution}
        \begin{align*}
            \int_B y \,dV
            &= \int_0^1 dx \int_0^{1-x} dy \int_0^{x+y} dz \, (y) \\
            &= \int_0^1 dx \int_0^{1-x} dy \, y(x+y) \\
            &= \int_0^1 dx \, \left(\frac12xy^2 + \frac13y^3\right)^{y=1-x}_{y=0} \\
            &= \frac16 \int_0^1 (x^3 - 3x + 2) \,dx \\
            &= \frac18.
        \end{align*}
    \end{solution}
\end{parts}

\setcounter{question}{86}
\question Integrate $\nabla \times \bm F$ where
\[
    \bm F = 3y \bm e_1 - xz\bm e_2 + yz^2 \bm e_3,
\]
over the portion $S$ of the surface 
$2z = x^2 + y^2$ below the plane $z = 2$,
both directly and by using Stokes' theorem.
Take the area elements of $S$ to point outwards, 
so that their $z$ components are negative.
\begin{solution}
    We will compute this integral directly first.
    We calculate the curl of $\bm F$:
    \[
        \nabla \times \bm F = (z^2 + x, 0, -(z+3)).
    \]
    We can describe our surface with the level set
    \[
        f(x,y,z) = x^2 + y^2 - 2z, \qquad \nabla f(x,y,z) = (2x,2y,-2).
    \]
    Now
    \[
        d\bm A 
        = \frac{-\nabla f}{\bm e_3 \cdot \nabla f} 
        = (x,y,-1),
    \]
    note that the minus sign here comes from the restriction that
    $z < 0$.
    Let $A$ be the disk centered at the origin on the $x$-$y$ plane
    with radius $2$
    (that is, the projection of $S$ to the $x$-$y$ plane).
    Then
    \begin{align*}
        \int_S \nabla \times \bm F \cdot d\bm A
        &= \int_A (z^2+x, 0, -(z+3)) \cdot (x,y,-1) \,dx\,dy \\
        &= \int_A \frac14x(x^2 + y^2 + x) + \frac12(x^2 + y^2) + 3 \,dx\,dy \\
        &= \int_0^{2\pi} d\theta \int_0^2 dr \,
            \left(
                \frac14 r^6 \cos\theta + r^3 \cos^2\theta + \frac12r^3 + 3r 
            \right) \\
        &= \int_0^{2\pi}
            \left(
                \frac{2^5}{7} \cos\theta + 2\cos{2\theta} + 10
            \right) \,d\theta \\
        &= 20\pi.
    \end{align*}
    Now we look at using Stokes' theorem.
    We define
    \[
        \bm x(\theta) = (2\cos\theta, -2\sin\theta, -2), 
        \qquad \bm x'(\theta) = (-2\sin\theta, -2\cos\theta, 0),
    \]
    we traverse clockwise such that we agree with the area element
    having a negative $z$ component.
    Now
    \begin{align*}
        \int_S \nabla \times \bm F \cdot d\bm A
        &= \int_{\partial S} \bm F \cdot d\bm x \\
        &= \int_{0}^{2\pi} \bm F(\bm x(\theta)) \, \bm x'(\theta) \,d\theta \\
        &= \int_{0}^{2\pi} (10 - 2\cos2\theta) \,d\theta \\
        &= 20\pi.
    \end{align*}
\end{solution}

\setcounter{question}{114} %todo
\question Consider the series
\[
    \varphi(x,y)
    = \sum^{\infty}_{n=1} B_n \sin(n\pi x) \sinh(n\pi y).
\]
Show that $\varphi$ solves Laplace's equation $\nabla^2 \varphi = 0$
in the unit square and obeys the Dirichlet boundary conditions
\[
    \varphi(0,y) = \varphi(1,y) = \varphi(x,0) = 0 
    \quad\text{for}\quad
    x, y \in (0,1).
\]
\begin{parts}
    \part 
    If in addition it is given that $\varphi(x,y) = 1$ for $x \in (0,1)$,
    find the coefficients $B_n$.

    \part 
    Using your solution to part (a), find a function $\tilde{\varphi}(x,y)$
    solving Laplace's equation on the unit square with the Dirichlet boundary conditions
    \[
        \tilde{\varphi}(x,0)
        = \tilde{\varphi}(0,y)
        = 0, 
        \qquad \tilde{\varphi}(x,1)
        = \tilde{\varphi}(1,y)
        = 1
    \]
    for $x,y \in (0,1)$.

    \part 
    If $\tilde{\varphi}(x,y)$ is a solution to the problem posed in part (b),
    show that $1 - \tilde{\varphi}(1-y,1-x)$ is a solution to the same problem.
    Given that the solution to this problem is unique,
    this implies that $\tilde{\varphi}(x,y) = 1 - \tilde{\varphi}(1-y,1-x)$.

    \part 
    Deduce the curious fact that, for $x \in (0,1)$,
    \[
        \sum^{\infty}_{n=0} 
        \frac
        {\sin((2n+1)\pi x) \cosh((2n+1)\pi(x-\sfrac12))}
        {(2n+1)\cosh((2n+1)\sfrac\pi2)} 
        = \frac\pi8.
    \]
    (Hint: consider setting $y = 1 - x$.)
\end{parts}

\setcounter{question}{116}
\question By integrating both sides against an arbitrary test-function, find the
coefficients $c$ or $c_m$ in the following generalised function identities
($\delta^{(m)}(x)$ is the $m$th derivative of $\delta(x)$ with respect
to $x$):
\begin{parts}
    \part 
    $x^n \delta^{(m)}(x) = c\delta^{(m - n)}(x)$, where $m \geq n$;
    \begin{solution}
        For $f \in K$, we first consider the LHS:
        \begin{align*}
            \infint x^n \delta^{(m)} f(x) \,dx
            &= \infint \delta^{(m)} (x) (x^nf(x))\,dx \\
            &= (-1)^m \infint \delta(x) (x^nf(x))^{(m)} \,dx \\
            (x^nf(x))^{(m)}
            &= \sum^{m}_{k=0} 
                \binom mk
                \frac{d^{m-k}}{dx^{m-k}} (x^n)
                \frac{d^k}{dx^k} (f(x)),
        \end{align*}
        For $k < m - n$ we will get $0$ as the $x^n$ will
        differentiate to $0$.
        Additionally, as we are evaluating at $x = 0$, 
        for $k > m - n$ we will also get $0$;
        therefore, we only need to evaluate this term at $k = m - n$.
        So
        \begin{align*}
            (x^nf(x))^{(m)} \rvert_{x=0}
            &= \binom m{m-n} f^{(m-n)}(0) \\
            &= \binom mn f^{(m-n)}(0).
        \end{align*}
        Now consider the RHS:
        \begin{align*}
            \infint c \delta^{(m-n)}(x) f(x) \,dx
            &= c \infint \delta^{(m-n)}(x) (f(x)) \,dx \\
            &= c (-1)^{(m-n)} \infint \delta(x) f^{(m-n)}(x) \,dx \\
            &= c (-1)^{(m-n)} f^{(m-n)}(0).
        \end{align*}
        Equating the LHS and RHS we get
        \begin{align*}
            \binom mn f^{(m-n)}(0)
            &= c(-1)^{(m-n)} f^{(m-n)}(0) \\
            c &= \binom mn (-1)^{(m-n)}.
        \end{align*}
    \end{solution}

    \part 
    $\delta(x^3 + x) = c\delta(x)$;
    \begin{solution}
        For $f \in K$, we consider the LHS.
        Let $g(x) = x^3 + x$. 
        Then $g(x) = 0$ for $x \in \R$
        if and only if $x = 0$.
        Therefore
        \[
            \delta(x^3 + x) 
            = \frac{\delta(x - 0)}{g'(0)} 
            = \delta(x).
        \]
        Hence $c = 1$.
    \end{solution}

    \part 
    $e^{-\lambda x}\delta^{(n)}(x) = \sum_{m = 0}^{n} c_m \delta^{(m)}(x)$,
    where $\lambda$ is a real constant; and
    \begin{solution}
        For $f \in K$, we initially check the LHS:
        \begin{align*}
            \infint e^{-\lambda x} \delta^{(n)}(x) f(x) \,dx
            &= \infint \delta^{(n)}(x) \left(e^{-\lambda x} f(x)\right) \,dx \\
            &= (-1)^n \infint \delta(x) \left(e^{-\lambda x} f(x)\right)^{(n)} \,dx \\
            &= (-1)^n \sum^{n}_{k=0} \binom nk
                \frac{d^{n-k}}{dx^{n-k}}\left(e^{-\lambda x}\right)
                \frac{d^k}{dx^k} \left(f^{(k)}\right) \rvert_{x = 0} \\
            &= (-1)^n \sum^{n}_{k=0} \binom nk
                \left((-\lambda)^{n-k}\right)
                \left(f^{(k)}(0)\right).
        \end{align*}
        Now we look at the RHS:
        \begin{align*}
            \infint \sum^{n}_{k=0} 
                c_k \delta^{(k)}(x) f(x) \,dx
            &= \sum^{n}_{k=0} c_k \infint
                \delta^{(k)}(x) f(x) \,dx \\
            &= \sum^{n}_{k=0} c_k (-1)^k f^{(k)}(0).
        \end{align*}
        Equating the sides:
        \begin{align*}
            (-1)^n \sum^{n}_{k=0} \binom nk
                \left((-\lambda)^{n-k}\right)
                \left(f^{(k)}(0)\right)
            &= \sum^{n}_{k=0} c_k (-1)^k f^{(k)}(0) \\
            c_k &= (-1)^k \lambda^{(n-k)} \binom nk.
        \end{align*}
    \end{solution}

    \part
    $\delta(\tanh(x) - \lambda) = c\delta(x - \tanh^{-1}(\lambda))$,
    where $\abs{\lambda} < 1$.
    \begin{solution}
        For $f \in K$, we consider the LHS.
        Let $g(x) = \tanh(x) - \lambda$.
        If $g(x) = 0$ then $\tanh(x) = \lambda$ so $x = \tanh^{-1}(\lambda)$.
        Note that the domain of $\tanh^{-1}$ 
        is equal to the image of $\tanh$ which is $[0,1)$; 
        hence, as $\lvert \lambda \rvert < 1$,
        $\tanh^{-1} (\lambda)$ is well defined.
        Additionally, $\tanh: \R \to [0,1)$ is a bijection;
        hence, only one $\lambda$ exists such that $x = \tanh^{-1}(\lambda)$.
        $g'(x) = 1 - \tanh^2(x)$, so
        \[
            \delta(\tanh(x) - \lambda)
            = \frac{\delta(x - \tanh^{-1}(\lambda)}{g'(\tanh^{-1}(\lambda)}
            = \frac{1}{1-\lambda^2} \delta(x - \tanh^{-1}(\lambda));
        \]
        therefore, $c = \frac{1}{1-\lambda^2}$.
    \end{solution}
\end{parts}
