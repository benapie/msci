\chapter{Odds and logistic regreesion}
\section{Odds}
\lecture{5}{10/2}

\begin{definition}[Odds]
    \textbf{Odds} are a numerical expression expressed as a pair of numbers.
\end{definition}

The \textbf{odds for} (or \textbf{odds of}) of some \emph{event}
reflects the likelihood that the event will take place,
while the \textbf{odds against} reject the likelihood that it will not.

\begin{example}
    We may say that the odds in favour of students to graduate with $1$st class
    honours is $1$ to $4$.
    This means that out of $5$ students, $1$ will graduate with $1$st class
    honours and $4$ of them will graduate without $1$st class honours.
    We can write this as a fraction: $\frac14 = 0.25$.
    But note, odds are \emph{not} probabilities.
    The odds are $\frac14$, but the probability is $\frac15$.
\end{example}

Let $p$ be the probability in favour of a binary event.
Therefore the probability against is $1 - p$.
\emph{The odds} of the event are the quotient of the two:
\[
    \frac{p}{1-p}.
\]

If there is a greater probability of success than failure ($p > 0.5$),
then the odds are between $1$ and $\infty$.
If there is a lesser probability of success than failure ($p < 0.5$),
then the odds are between $0$ and $1$.
There is a problem here as this is not symmetric, the interval
for odds of success is much larger than the interval for odds of failure.
We can solve this problem by takign $\log$ of the odds.
This centers ourselve at $0$ on the real number line.

\begin{definition}[Logit function]
    The \textbf{logit} function (or the \textbf{log-odds})
    is the logarithm of the odds $\frac{p}{1 - p}$, where
    $p$ is the probability:
    \[
        \logit(p) = \log\left(\frac{p}{1 - p}\right).
    \]
\end{definition}

\begin{example}
    Consider the following contingency table for occurence of cancer
    and a \emph{mutated gene}.
    \begin{center}
        \begin{tabular}{ccc}
            \toprule
              & \multicolumn{2}{c}{Cancer} \\
            Mutated gene  & P  & N \\
            \midrule
            P & 23 & 117 \\
            N & 6  & 210 \\
            \bottomrule
        \end{tabular}
    \end{center}
    Can we use a ratio of odds to determine if there is a \emph{relationship} between
    the mutated gene and cancer?
    If someone has the mutated gene,
    are odds higher that they will get cancer?
    The odds that a person has cancer given that they have a mutated gene
    is $\frac{23}{117}$
    and the odds that they have cancer given that they don't have the mutated gene
    $\frac{6}{210}$.
    The ratio of these odds is
    \[
        \frac{\left(\frac{23}{117}\right)}{\left(\frac{6}{210}\right)}
        = \frac{0.2}{0.03}
        = 6.88.
    \]
    We have
    \[
        \log(6.88) = 1.93.
    \]
    Larger values of the $\log$ of the odds ratio 
    mean that the mtuated gene is a good
    predictor of cancer.
    Smaller values mean that the mutated gene is not a good predictor of cancer.
\end{example}

%todo
