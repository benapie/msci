\lecture{11}{4/2}

We need to check whether $\phi(\bm x)$ satisfies $\bm F = \nabla \phi$.
\begin{align*}
    \frac{\partial\phi}{\partial x}
    &= \lim_{\delta x \to 0} 
        \frac{\phi(\bm x + \bm e_1 \delta x) - \phi(\bm x)}{\delta x} \\
    &= \lim_{\delta x \to 0} 
        \left(
            \frac{1}{\delta x} 
            \left(
                \int_{\bm x_0}^{\bm x + \bm e_1 \delta x}
                \bm F \cdot d\bm x
                + \int_{\bm x_0}^{\bm x} 
                \bm F \cdot d\bm x
            \right)
        \right).
\end{align*}
We see that
\[
    \int_{\bm x_0}^{\bm x + \bm e_1 \delta x} \bm F \cdot d\bm x
    = \int_{\bm x_0}^{\bm x} \bm F \cdot d\bm x 
    = \int_{\bm x}^{\bm x + \bm e_1 \delta x} \bm F \cdot d\bm x
\]
hence
\[
    \frac{\partial \phi}{\partial x}
    = \lim_{\delta x \to 0}
    \left(
        \frac{1}{\partial x} 
        \int_{\bm x}^{\bm x + \bm e_1 \delta x}
        \bm F \cdot d\bm x
    \right).
\]
We can parametrise this as 
\[
    \bm x(t) = (x + t, y, z), \qquad t \in [0,\delta x]
\]
where
$\frac{d\bm x}{dt} = \bm e_1$.
So
\begin{align*}
    \frac{1}{\delta x}
        \int_{\bm x}^{\bm x + \bm e_1 \delta x} \bm F \cdot d\bm x
    &= \frac{1}{\delta x} \int_0^{\delta x} 
        \bm F(x + t, y, z) \cdot \frac{d\bm x}{dt} \,dt \\
    &= \frac{1}{\delta x} \int_0^{\delta x} 
        F_1(x + t, y, z) \,dt \\
    &= \frac{1}{\delta x} \int_0^{\delta x} 
        \left(
            F_1 \delta x 
            + \frac12 \delta x^2 \frac{\partial F_1}{\partial x} 
            + \ldots
        \right)
        \,dt \\
    &= \frac{1}{\delta x} \int_0^{\delta x}
        F_1(x,y,z) + \frac{\delta x}{2} \frac{\partial F_1}{\partial x} + \ldots \\
    &\to \frac{1}{\delta x} \int_0^{\delta x} F_1(x,y,z)
\end{align*}
as $\delta x \to 0$ so 
\[
    \frac{\partial\phi}{\partial x} = F_1
\]
and similarly
\[
    \frac{\partial\phi}{\partial y} = F_2, \qquad
    \frac{\partial\phi}{\partial z} = F_3.
\]

$\phi$ (or $-\phi$) is called the \textbf{scalar potential}.

\begin{definition}[Closed]
    Let $\bm F$ be a vector field.
    $F$ is \textbf{closed} if $\nabla \times \bm F = \bm 0$.
\end{definition}

\begin{definition}[Exact]
    Let $\bm F$ be a vector field.
    $F$ is \textbf{exact} if there exists some scalar field $\phi$
    such that $F = \nabla\phi$.
\end{definition}

\begin{remark}
    Let $\bm F$ be a vector field.
    Then
    \[
        F \;\text{is exact}\; \implies F \;\text{is closed}.
    \]
    If we are in a simply connected domain, then
    \[
        \implies F \;\text{is closed}\; \implies F \;\text{is exact}.
    \]
    If our region has suitable \emph{holes}, this may fail.
\end{remark}

Finally, lets note that if $\bm F = \nabla\phi$
then path independence follows directly.

\begin{theorem}[]
    If $\bm F = \nabla\phi$ for some scalar field $\phi$
    and $C$ is a curve from $\bm x = \bm a$ to $\bm x = \bm b$
    then
    \[
        \int_C \bm F \cdot d\bm x = \phi(b) - \phi(a).
    \]
\end{theorem}

\begin{proof}
    Follows from the chain rule.
\end{proof}

\begin{example}
    Compute
    \[
        I = \int_C \bm F \cdot d\bm x
    \]
    where
    \[
        \bm F =
        \begin{pmatrix}
            y\cos(xy)             \\
            x\cos(xy) - z\sin(yz) \\
            -y\sin(yz)            \\
        \end{pmatrix}
    \]
    and $C$ is specified by $t \mapsto \bm x(t)$ where
    \[
        \bm x(t) =
        \begin{pmatrix}
            \frac{\sin t}{\sin t} \\
            \frac{\log(t + t)}{\log 2} \\
            \frac{1 - e^t}{1 - e} \\ 
        \end{pmatrix}
    \]
    with $t \in [0,1]$.
\end{example}

\begin{solution}
    Note that $\bm x(0) = (0,0,0)$ and $\bm x(1) = (1,1,1)$.
    We compute $\nabla \times \bm F = 0$ in all of $\R^3$ (trust me).
    So $\bm F = \nabla \phi$ for some $\phi$.
    So we have $\phi_x = F_1$, $\phi_y = F_2$, and $\phi_z = F_3$.
    Consider $\phi_x$, we have
    \[
        \frac{\partial\phi}{\partial x} = y\cos(xy)
        \implies \phi(x,y,z) = \sin(xy) + f(y,z).
    \]
    Then
    \begin{align*}
        \frac{\partial}{\partial y} (\sin(xy) + f(y,z))
        &= x\cos(xy) - z\sin(yz) \\
        x\cos(xy) + \frac{\partial f}{\partial y}
        &= x\cos(xy) - z\sin(yz) \\
        \frac{\partial f}{\partial y}
        &= -z\sin(yz) \\
        f(y,z) 
        &= \cos(yz) + g(z) \\
        \frac{\partial}{\partial z} (\sin(xy) + \cos(yz) + g(z))
        &= -y\sin(yz) \\
        -y\sin(yz) + \frac{\partial g}{\partial z}
        &= -y\sin(yz) \\
        \frac{\partial g}{\partial z}
        &= 0 \\
        g &= A.
    \end{align*}
    So
    \[
        \phi(x,y,z) = \sin(xy) + \cos(yz) + A
    \]
    solves $\bm F = \nabla\phi$.
    So
    \[
        I = \int_C \bm F \cdot d\bm x 
        = \phi(1,1,1) - \phi(0,0,0) 
        = \sin 1 + \cos 1 - 1.
    \]
\end{solution}
