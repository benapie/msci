\chapter{Partial differentiable equations}
\section{The heat equation}
\lecture{13}{10/2}

Consider a volume $V$ enclosed by a surface $S$.
The total heat in $V$ at time $t$ is
\[
    H(t) = \int_V T(\bm x, t) \,dV,
\]
a 3-dimensional integral over $\bm x$ of $T(\bm x, t)$,
the temperature at point $\bm x$ at time $t$.
If there are no soruces or sinks of heat in $V$, then
\[
    \text{rate of change of total heat in $V$}
    =
    \text{total flux of heat into $V$ through $S$}.
\]
That is,
\[
    \frac{d}{dt} \int_V T(\bm x, t) \,dV
    = - \int_S \bm v \cdot d\bm A
\]
where $\bm v$ is the \textbf{local heat flux}.
Furthermore, $\bm v$ is proportional to the gradient
of the temperature
\[
    \bm v = -k\nabla T \tag{$\star$}
\]
with $k > 0$. The minus sign is as heat flows from hot to cold.
The gradient in $(\star)$ is a 3-dimensional gradient,
not including $\frac{\partial}{\partial t}$.
Hence
\begin{align*}
    \frac{\partial}{\partial t} \int_V T \,dV
    &= k\int_S \nabla T \cdot d\bm A \\
    \int_V \frac{\partial T}{\partial t} \,dV
    &= k\int_V \nabla^2 T \,dV
\end{align*}
using the divergence theorem where 
$\nabla^2 T = \nabla \cdot (\nabla T)$.
So
\[
    \int_V
    \left(
        \frac{\partial T}{\partial t} - k\nabla^2 T
    \right)
    \,dV = 0.
\]
If there are no heat sources anywhere, $V$ could be
any volume. Hence
\[
    \frac{\partial T}{\partial t} - k\nabla^2T = 0
\]
everywhere. That is,
\[
    \frac{\partial T}{\partial t} = k\nabla^2 T
\]
(this is the heat (or diffusion) equation).

\begin{remark}
    \hfill
    \begin{enumerate}
        \item Derivation for the heat equation is like that 
            for the continuity equation. 
            $(\star)$ allows the system to be closed.

        \item In following, we usually set $k = 1$.

        \item Looking at $1$-dimensional space,
            the heat equation becomes
            \[
                \frac{\partial T(x, t)}{\partial t} 
                = \frac{\partial^2 T(x,t)}{\partial x^2}.
            \]
    \end{enumerate}
\end{remark}

\section{Separation of variables}

Separation of variables is a general method for solving linear 
partial differential equations.
We start by looking for simple solutions for when the
partial differential equation reduces to an ordinary
differential equation and then add them up.

\begin{example}
    A rod of length $L$ is insulated along its length.
    For all $t < 0$, one end has temperature $T = 0$
    and the other end has temperature $T = 100$.
    \begin{enumerate}
        \item What is $T(x,t)$ for $t \leq 0$?

        \item At $t \geq 0$, the ends of the rod satisfy
            \[
                T(0,t) = T(L,t) = 0.
            \]
            What is $T(x,t)$ for all $t > 0$?
    \end{enumerate}
\end{example}

\begin{solution}
    \hfill
    \begin{enumerate}
        \item
            Since
            $\frac{\partial T}{\partial t} = 0$
            for $t<0$,
            $T(x,t) = X(x)$ for some $X(x)$.
            \[
                \frac{\partial T}{\partial t}
                = \frac{\partial^2 T}{\partial x^2} 
            \]
            gets
            \[
                0 = \frac{\partial^2 X}{\partial x^2}.
            \]
            Hence
            \[
                X(x) = ax + b.
            \]
            $X(0) = 0$ and $X(L) = 100$ so
            \[
                X(x) = \frac{100}{L} x = T(x,t)
            \]
            for $t \leq 0$.

        \item Next lecture.
    \end{enumerate}
\end{solution}
