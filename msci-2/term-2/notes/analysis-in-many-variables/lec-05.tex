\lecture{5}{21/1}

A \textbf{line integral} is an integral where the function to be
integrated is evaluated along a curve.
The term \emph{contour integral} is also used,
but this is typically reserved for line integrals in the complex plane.

\begin{definition}[Line integral]
    For a vector field $\bm v: U \subset \R^n \to \R^n$, 
    the \textbf{line integral} along a regular arc $C \subset U$,
    in the direction $\bm r$ is defined as
    \[
        \int_C \bm v(\bm r) \cdot d\bm r =
        \int_a^b \bm v(\bm r(t)) \cdot \bm r'(t) \, dt.
    \]
\end{definition}

\begin{remark}
    As suggested by the notation,
    the line integral does not depend on the choice of parameters.
    Suppose $t = t(u)$, with $u$ being the alternative parameter.
    Then
    \[
        \frac{d\bm x}{du} = \frac{d\bm x}{dt} \frac{dt}{du} \tag{chain rule}
    \]
    and
    \begin{align*}
        \int \bm v(\bm x(t(u))) \cdot \frac{d\bm x}{du} \, du
        &= \int \bm v(\bm x(t(u))) \cdot \frac{d\bm x}{dt} \frac{dt}{du} \, du \\
        &= \int \bm v(\bm x(t)) \cdot \frac{d\bm x}{dt} \, dt
    \end{align*}
    which is equal to our original expression.
\end{remark}

If $C$ is a regular curve, then we still write
\[
    \int_C \bm v \cdot d\bm x;
\]
that is, we sum over the arcs.
If $C$ is a \emph{closed regular} curve or arc we sometimes write this as
\[ \oint_C \bm v \cdot d\bm x. \]

\begin{example}
    Let
    \[
        \bm u(\bm x) =
        (x^2z, xyz, x)
    \]
    in $\R^3$ and let $C$ be the circle with centre
    $\bm a = (a_1, a_2, a_3)$ and radius $r$ in the $x$-$y$ plane
    given by
    \[
        \bm x(t) = \bm a + r\cos t \bm e_1 + r\sin t \bm e_2
    \]
    where $t \in [0, 2\pi]$.
    Calculate
    \[
        \oint_C \bm u \cdot d\bm x.
    \]
\end{example}

\begin{solution}
    \begin{align*}
        \frac{d\bm x}{dt} 
        &= -r\sin t\bm e_1 + r\cos t \bm e_2 \\
        &= (-r\sin t, r\cos t, 0) \\
        \bm x(t) 
        &= (a_1 + r\cos t, a_2 + r\sin t, a_3) \\
        \bm u(\bm x(t))
        &= ((a_1 + r\cos t)^2a_3, (a_1 + r\cos t)(a_2 + r\sin t)a_3,
        a_1 + r\cos t) \\
        \bm u(\bm x(t)) \cdot \frac{d\bm x}{dt}
        &= -r\sin t(a_1 + r\cos t)^2a_3 + 
        r\cos t(a_1 + r\cos t)(a_2 + r\sin t)a_3.
    \end{align*}
    Therefore
    \begin{align*}
        \oint_C \bm u \cdot d\bm x
        &= \int_0^{2\pi}
        \big(
            -r\sin t(a_1 + r\cos t)^2a_3
            \\& \qquad\qquad
            + r\cos t(a_1 + r\cos t)(a_2 + r\sin t)a_3
        \big)
        \, dt \\
        &= \int_0^{2\pi}
        \big(
            -r\sin t(a_1 + r\cos t)^2 a_3
            +r\cos t(a_1 + r\cos t)a_2a_3
            \\& \qquad\qquad
            +r\cos t(a_1 + \cos t)r\sin t a_3
        \big)
        \, dt \\
        &= \int_0^{2\pi} -\cos t (a_1 + r\cos t) a_2 a_3 \, dt \\
        &= \int_0^{2\pi} \left( r^2\cos^2{t} + a_2 a_3 \right) \, dt \\
        &= \pi r^2 a_2 a_3.
    \end{align*}
\end{solution}

\section{Surface integrals: defining a surface}

To integrate over a surface we must define the surface.
There are (at least) two ways to do this.

\paragraph{Method 1} 
We can directly generalise from line integrals; that is,
define a surface in parametric form $\bm x(u, v)$ in $\R^2$.

\begin{example}
    Points on a sphere of radius $a$ centred on the origin can be
    parametrised in spherical (polar) coordinates,
    with $u$ and $v$ usually written as $\theta$ and $\phi$,
    \[
        \bm x(\theta, \phi) = (x(\theta, \phi), 
        y(\theta, \phi), 
        z(\theta, \phi))
    \]
    where
    \[
        x(\theta, \phi) = a\sin\theta\cos\phi, \qquad
        y(\theta, \phi) = a\sin\theta\sin\phi, \qquad
        z(\theta, \phi) = a\cos\theta,
    \]
    and $\theta \in [0, \pi]$ and $\phi \in [0, 2\pi]$.
\end{example}
