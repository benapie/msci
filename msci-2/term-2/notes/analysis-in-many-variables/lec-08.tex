\lecture{8}{28/1}

\section{Summary}

To summarise, we have our two methods for defining surfaces areas for
surface integrals:
\begin{enumerate}
    \item 
        \[
            \int_S \bm F \cdot d \bm A 
            = \int_U \bm F(\bm x) \cdot
            \left(
                \frac{\partial \bm x}{\partial u} 
                \times \frac{\partial \bm x}{\partial v}
            \right)
            \,du\,dv;
        \]

    \item
        \[
            \int_S \bm F \cdot d\bm A
            = \int_A \frac{\bm F \cdot \nabla f}{\bm e_3 \cdot \nabla f} \,dx\,dy
        \]
        where $A$ is $S$ projected onto the $x$-$y$ plane.
\end{enumerate}

\begin{example}
    Compute the integral of $\bm F = \bm e_3$ over the surface $S$
    given by the hemisphere of radius $1$ centred at the origin with $z > 0$.
\end{example}

\begin{solution}
    We have $S = \{(x,y,z) \in \R^3: x^2 + y^2 + z^2 = 1, z > 0\}$
    and let $f(x,y,z) = x^2 + y^2 + z^2$.
    Then $\nabla f = (2x,2y,2z)$ and so
    \[
        \int_S \bm F \cdot d\bm A
        = \int_A \frac{\bm F \cdot \nabla f}{\bm e_3 \cdot \nabla f} \,dx\,dy
        = \int_A \frac{\bm e_3 \cdot \nabla f}{\bm e_3 \cdot \nabla f} \,dx\,dy
        = \int_A \,dx\,dy = \pi
    \]
    as if we project our hemisphere to the $x$-$y$ plane, we get the unit disk.
\end{solution}

\chapter{Three big theorems}

\begin{theorem}[Green's theorem]
    Let $C$ be the boundary of some region $A$ (traversed anticlockwise) and let
    $P$ and $Q$ be continuously differentiable scalar functions on $\R^2$.
    Then
    \[
        \oint_C \left(P(x, y)\,dx + Q(x,y)\,dy\right)
        = \int_A 
        \left(
            \frac{\partial Q}{\partial x} - \frac{\partial P}{\partial y}
        \right)
        \,dA
    \]
\end{theorem}

\begin{remark}
    We can also write Green's theorem in vector form;
    by setting 
    \[
        F(x, y, z) = (P(x,y),Q(x,y),R)
    \]
    for $R$ arbritrary we get
    \[
        \oint_C \bm F \cdot d\bm x 
        = \int_A (\nabla \times \bm F) \cdot \bm e_3 \,dA.
    \]
\end{remark}

\begin{theorem}[Stokes' theorem]
    Let $\bm F(x,y,z)$ be a vector field in $\R^3$ and $S$ be a surface in $\R^3$
    with area elements $d\bm A = \hat{\bm n} dA$.
    Then
    \[
        \oint_{\partial S} \bm F \cdot d\bm x = \int_S (\nabla \times \bm F) \,d\bm A.
    \]
\end{theorem}

\begin{theorem}[Divergence theorem]
    Let $\bm F$ be a continuously differentiable vector field over some volume $V$ 
    with a bounding surface $S$.
    Then
    \[
        \iiint_V \nabla \cdot \bm F \,dV = \iint_S \bm F \cdot d\bm A.
    \]
\end{theorem}

\begin{remark}
    It is important to note here that we are effectively just extending the
    fundamental theorem of calculus.
\end{remark}

\section{Examples}

\begin{example}
    Check Green's theorem where 
    $P(x,y) = y^2 - 7y$, 
    $Q(x,y) = 2xy + 2x$, and
    the area of integration is the unit circle.
\end{example}

\begin{solution}
    Let $R$ be unit circle and
    \[
        \gamma(\theta) = (\cos\theta, \sin\theta), \qquad
        \gamma'(\theta) = (-\sin\theta, \cos\theta).
    \]
    First we will look at the LHS.
    \begin{align*}
        \text{LHS} 
        &= \oint_{\partial R} (P(x,y)\,dx + Q(x,y)\,dy) \\
        &= \oint_{\partial R} 
            ((\sin^2\theta - 7\sin\theta) (-\sin\theta) +
            (2\cos\theta\sin\theta + 2\cos\theta) (\cos\theta))\,d\theta \\
        &= 9\pi.
    \end{align*}
    Now onto the RHS.
    We have that
    \[
        \frac{\partial P}{\partial y} = 2y + 2,
        \qquad \frac{\partial Q}{\partial x} = 2y - 7.
    \]
    So
    \begin{align*}
        \text{RHS}
        &= \int_R \left( \frac{\partial Q}{\partial x}
            - \frac{\partial P}{\partial y}\right) \,dx\,dy \\
        &= \int_R 9 \,dx\,dy \\
            &= 9\pi.
    \end{align*}
\end{solution}
