\lecture{2}{14/1}

The \emph{inverse function theorem} says that this invertibility can extend
beyond the linear approximation.

\begin{theorem}[Inverse function theorem]
    Let $U \subset \R^n$ be open, $\bm v: U \to \R^n$ be a differentiable
    vector field with continuous partial derivatives, and $\bm a \in U$.
    Then provided $\det(\bm J_{\bm v}(\bm a)) \neq 0$, there exists an open set
    $\tilde U \subset U$ containing $\bm a$ such that
    \begin{enumerate}
        \item $\bm v(\tilde U)$ is open; and
        \item the mapping $\bm v$ from $\tilde U$ to $\bm v(\tilde U)$ has a
            differentiable inverse.
    \end{enumerate}
\end{theorem}

When we say that a mapping $\bm v$ from $\bm U$ to $\bm v(\tilde U)$ has a
differentiable inverse, 
we mean that there exists a differential vector field 
$\bm w: \bm v(\tilde U) \to U$ 
such that 
$\bm w(\bm v(\bm x)) = \bm x$
and 
$\bm v(\bm w(\bm y)) = \bm y$ 
for all 
$x \in \tilde U$ and $y \in \bm v(\tilde U)$.

\begin{definition}[Diffeomorphism]
    Let $U, V \subset \R^n$ be open.
    A differentiable mapping $\bm F: U \to V$ is called a \textbf{diffeomorphism}
    if
    \begin{enumerate}
        \item it is a bijection; and
        \item its inverse $\bm F^{-1}$ is differentiable.
    \end{enumerate}
    If such a differentiable mapping exists, $U$ and $V$ are said to be \textbf{diffeomorphic}.
\end{definition}

We can extend this idea to be more general.

\begin{definition}[Local diffeomorphism]
    Let $U, V \subset \R^n$ be open.
    A differntiable mapping $\bm F: U \to V$ is called a \textbf{local diffeomorphism}
    if for all $\bm x \in U$ there exists and open set $U' \subset U$ containing $x$ such that
    $V' = \bm F(U')$ is open in $V$ and $\bm F$ is a diffeomorphism from $U'$ to $V'$.
    If such a mapping exists we say that $U$ and $V$ are locally diffeomorphic.
\end{definition}

\begin{remark}
    In general, suppose
    \[ \bm v: U \to V \subset \R^n, \qquad \bm w: V \to W \subset \R^n \]
    with $U,V$ both open in $\R^n$ and that $\bm v, \bm w$ are continuously
    differentiable vector fields.
    Then $\bm x \to \bm w(\bm v(\bm x))$ is a mapping $U \to W \subset \R^n$
    and its differential can be found using the chain rule:
    \begin{align*}
        D\bm w(\bm v(\bm x))
        &= \begin{pmatrix}
            \frac{\partial w_1(\bm v(\bm x))}{\partial x_1} & 
            \frac{\partial w_1(\bm v(\bm x))}{\partial x_2} & 
            \ldots & 
            \frac{\partial w_1(\bm v(\bm x))}{\partial x_n} \\            
            \frac{\partial w_2(\bm v(\bm x))}{\partial x_1} & 
            \frac{\partial w_2(\bm v(\bm x))}{\partial x_2} & 
            \ldots & 
            \frac{\partial w_2(\bm v(\bm x))}{\partial x_n} \\
            \vdots & \vdots & \ddots & \vdots \\
            \frac{\partial w_n(\bm v(\bm x))}{\partial x_1} & 
            \frac{\partial w_n(\bm v(\bm x))}{\partial x_2} & 
            \ldots & 
            \frac{\partial w_n(\bm v(\bm x))}{\partial x_n} \\
        \end{pmatrix} \\
        &= \begin{pmatrix}
            \sum_j \left(\frac{\partial w_1}{\partial v_j}\frac{\partial v_j}{\partial x_1}\right) &
            \ldots &
            \sum_j \left(\frac{\partial w_1}{\partial v_j}\frac{\partial v_j}{\partial x_n}\right) \\
            \vdots & \ddots & \vdots \\
            \sum_j \left(\frac{\partial w_n}{\partial v_j}\frac{\partial v_j}{\partial x_1}\right) &
            \ldots &
            \sum_j \left(\frac{\partial w_n}{\partial v_j}\frac{\partial v_j}{\partial x_n}\right) \\
        \end{pmatrix} \\
        &= \begin{pmatrix}
            \frac{\partial w_1}{\partial v_1} &
            \ldots &
            \frac{\partial w_1}{\partial v_n} \\
            \vdots & \ddots & \vdots \\
            \frac{\partial w_n}{\partial v_1} &
            \ldots &
            \frac{\partial w_n}{\partial v_n} \\
        \end{pmatrix}
        \begin{pmatrix}
            \frac{\partial v_1}{\partial x_1} &
            \ldots &
            \frac{\partial v_1}{\partial x_n} \\
            \vdots & \ddots & \vdots \\
            \frac{\partial v_n}{\partial x_1} &
            \ldots &
            \frac{\partial v_n}{\partial x_n} \\
        \end{pmatrix} \\
        &= D\bm w(\bm v) D\bm v(\bm x).
    \end{align*}
    For the special case when $\bm v$ is a local diffeomorphism and
    $\bm w$ is its inverse map, 
    \[\bm w(\bm v(\bm x)) = 
        \bm x \implies D\bm w \cdot D\bm x = 
        D\bm x(\bm x) = I_n. \]
    Likewise,
    \[ \bm v(\bm w(\bm y)) = 
        \bm y \implies D\bm w \cdot D\bm x = 
        D\bm x(\bm x) = I_n. \]
    Hence, $D\bm v$ is an invertible matrix with inverse $D\bm w$. 
    Therefore, 
    \[
        \det(D\bm w) = \frac1{\det(D\bm v)}.
    \]
\end{remark}

\begin{definition}[]
    A diffeomorphism $\bm v$ is called
    \begin{enumerate}
        \item \textbf{orientation preserving} if $\det(\bm J_{\bm v}) > 0$: and
        \item \textbf{orientation reversing} if $\det(\bm J_{\bm v}) < 0$.
    \end{enumerate}
\end{definition}

\begin{example}
    Let
    \[
        \bm v(\bm x) =
        \begin{pmatrix}
            x^2 - y^2 \\
            2xy \\
        \end{pmatrix}
        .
    \]
    Then
    \[ \det(\bm J_{\bm v}) = 4(x^2 + y^2) > 0 \]
    for all $(x, y) \neq \bm 0$.
    Then $\bm v: \R^2 \setminus \{\bm 0\} \to \R^2$ is an orientation
    preserving local diffeomorphism;
    however, it is not global since
    \[ \bm v(-\bm x) = \bm v(\bm x) \]
    there is no global inverse mapping.
    But $\bm v$ does map $\{(x, y): x > 0\}$ onto
    $\R^2 \setminus \{(x, 0): x \leq 0\}$
    diffeomorphically.
\end{example}
