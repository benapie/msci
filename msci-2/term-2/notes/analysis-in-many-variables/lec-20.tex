\section{Formal definitions}
\lecture{20}{17/3}

We will now be a bit more careful.
Suppose we have 2 ordinary continuous functions $g(x)$ and $h(x)$ and
that we want to check whether they are equal as functions.
We have two methods:
\begin{enumerate}
    \item (E1) check that for all $x \in \R$ that $g(x) = h(x)$; and
    \item (E2) check that for all functions $f(x)$, 
        in some suitable class of functions $K$,
        that 
        \[
            \int_{-\infty}^{\infty} f(x) g(x) \,dx 
            = \int_{-\infty}^{\infty} f(x) h(x) \,dx.
        \]
\end{enumerate}

For $K$ we will take \emph{all} functions $f$ which satisfy:
\begin{enumerate}
    \item (K1) $f$ has finite derivatives of all orders; and
    \item (K2) $f$ vanishes outside some bounded region
        (that is, for every $f \in K$ there exists some $M$ 
        such that $f(x) = 0$ for all $\abs x > M$).
\end{enumerate}

In fact, (E1) and (E2) are equivalent!
\begin{enumerate}
    \item Clearly, $\text{(E1)} \implies \text{(E2)}$.
    \item For $\text{(E2)} \implies \text{(E1)}$,
        suppose that there is an $x_0$ such that
        \[
            g(x_0) \neq h(x_0).
        \]
        Lets say $g(x_0) > h(x_0)$.
        Then 
        \[
            g(x_0) - h(x_0) > 0
        \]
        and by continuity there exists an interval $(a,b)$ with $x_0 \in (a,b)$
        \emph{and}
        \[
            g(x) - h(x) > 0 \qquad \;\forall\; x \in (a,b).
        \]
        Now we pick $f(x)$ such that $f(x) > 0$ for all $x \in (a,b)$
        and $f(x) = 0$ for $x \not\in (a,b)$.
        Then
        \[
            \int_{-\infty}^{\infty} f(x) (g(x) - h(x)) \,dx
            = \int_{a}^{b} f(x) (g(x) - h(x)) \,dx > 0.
        \]
        So
        \[
            \int_{-\infty}^{\infty} f(x) g(x) \,dx
            \neq \int_{-\infty}^{\infty} f(x) h(x) \,dx.
        \]
\end{enumerate}

Now, for generalised functions such as $\delta(x)$,
the \emph{limit} of the delta sequence $h_N(x)$ is the value
of the function at a particular point is \emph{not} well defined.
Instead they are defined by the values that they give when they are
multiplied by another function and integrated over $\R$.

Thus, (E1) does not work.
But (E2) still makes sense so we use it to \emph{define} what it means
for two generalised functions to be equal.

Thus $\delta(x)$ is defined by the formula 
\[
    \int_{-\infty}^{\infty} f(x) \delta(x) \,dx
    = f(0)
    \qquad \forall\; f \in K.
\]
This allows us to forget the delta convergent sequence $h_N(x)$
and instead work with the generalised function directly
using the further property that the operations of substitution
and integration by parts work as for integrals of ordinary functions.

\section{Proving generalised function identities}

\begin{example}
    Show that $x \delta(x) = 0$ as a distribution.
\end{example}

\begin{solution}
    For all $f \in K$ we have
    \begin{align*}
        \int_{-\infty}^{\infty} f(x) x \delta(x) \,dx
        &= \int_{-\infty}^{\infty} (f(x) x) \delta(x) \,dx \\
        &= \left(f(x) x\right) \rvert_{x = 0} \\
        &= 0.
    \end{align*}
    While
    \begin{align*}
        \int_{-\infty}^{\infty} f(x) 0 \,dx &= 0 \\
        \int_{-\infty}^{\infty} f(x) x \delta(x) \,dx 
        &= \int_{-\infty}^{\infty} f(x) 0 \,dx \\
        x \delta(x) &= 0
    \end{align*}
    as distributions.
\end{solution}

\begin{example}
    Show that $x \delta'(x) = -\delta(x)$.
\end{example}

\begin{solution}
    For all $f \in K$
    \begin{align*}
        \int_{-\infty}^{\infty} f(x) x \delta'(x) \,dx
        &= \int_{-\infty}^{\infty} \delta'(x) \left(xf(x)\right) \,dx \\
        &= -\int_{-\infty}^{\infty} \delta'(x) \left(xf(x)\right)' \,dx \tag{by parts} \\
        &= -(xf(x))' \rvert_{x = 0} \\
        &= -(xf'(x) + f(x))' \rvert_{x = 0} \\
        &= -f(0).
    \end{align*}
    While
    \begin{align*}
        \int_{-\infty}^{\infty} f(x) (-\delta'(x)) \,dx 
        &= -\int_{-\infty}^{\infty} f(x) \delta'(x) \,dx \\
        \int_{-\infty}^{\infty} f(x) x \delta'(x) \,dx 
        &= \int_{-\infty}^{\infty} f(x) (-\delta(x)) \,dx \\
        x\delta'(x) &= -\delta(x)
    \end{align*}
    as distributions.
\end{solution}

\begin{example}
    Evaluate
    \[
        \int_{-\infty}^{\infty} \delta(x-y) f(y) \,dy.
    \]
\end{example}

\begin{solution}
    The technique here is to substitute $z = x - y$ with $dz = -dy$.
    \begin{align*}
        \int_{-\infty}^{\infty} \delta(x-y) f(y) \,dy
        &= -\int_{\infty}^{-\infty} \delta(z) f(x-z) \,dz \\
        &= \int_{-\infty}^{\infty} \delta(z) f(x-z) \,dz \\
        &= f(x-z)\rvert_{z = 0} \\
        &= f(x).
    \end{align*}
\end{solution}

\begin{remark}
    Sometimes we \emph{define} $\delta(x)$ as above.
\end{remark}

\begin{example}
    Evaluate
    \[
        \int_{-\infty}^{\infty} \delta(2x) f(x) \,dx.
    \]
\end{example}

\begin{solution}
    As before, we substitute. This time our substitution is
    $z = 2x \implies dz=2dx$.
    \begin{align*} 
        \int_{-\infty}^{\infty} \delta(2x) f(x) \,dx
        &= \int_{-\infty}^{\infty} \frac12 \delta(z) f\left(\frac z2\right) \,dz \\
        &= \frac12 f\left(\frac02\right) \\
        &= \frac12 f(0).
    \end{align*}
\end{solution}

\begin{remark}
    In the solution above, we have also shown that 
    \[
        \delta(2x) = \frac12 \delta(x).
    \]
\end{remark}
