\lecture{6}{23/1}

To evaluate integrals constructed in this method,
we construct area elements $\Delta A_k$ by approximating them as parallelograms.
Remember, the cross product of two vectors is equal to the area of the parallelgram
that they span.
Therefore
\begin{align*}
    \hat{\bm n} \Delta A_k 
    &\approx (\bm x(u_i + \Delta u_i, v_j) - \bm x(u_i, v_j))
        \times (\bm x(u_i, v_j + \Delta v_j) - \bm x(u_i, v_j)) \\
    &\approx \left( \frac{\partial x}{\partial u} \Delta u_i \right)
        \times \left( \frac{\partial x}{\partial v} \Delta v_j \right) \\
    &= \left( 
            \frac{\partial \bm x}{\partial u} \times \frac{\partial \bm x}{\partial v}
        \right)
        \Delta u_i \Delta v_j
\end{align*}
and thus
\[
    \int_S \bm F \cdot d\bm A
    = \lim_{\Delta A_k \to 0}
        \sum_{k = 0}^N \bm F(\bm x_k^\star) \cdot 
        \left(
            \frac{\partial \bm x(u_i, v_j)}{\partial u} 
            \times \frac{\partial \bm x(u_i, v_j)}{\partial v}
        \right)
        \Delta u_i \Delta v_j.
\]

\begin{example}
    Find the surface integral of $\bm F = \bm e_3$ over the surface $S$
    of a hemisphere centred on the origin with radius 1, $z \geq 0$.
\end{example}

\begin{solution}
    We parametrise with
    $\theta \in \left[0,\frac{\pi}2\right]$
    and
    $\phi \in [0, 2\pi]$.
    Then (from the previous example)
    \begin{align*}
        \int_S \bm F \cdot d\bm A
        &= \int_0^{\frac{\pi}2} d\theta \int_0^{2\pi} d\phi \,
            \bm F \cdot 
            \left(
                \frac{\partial x}{\partial \theta}
                \times
                \frac{\partial x}{\partial \phi}
            \right) \\
        &= \int_0^{\frac{\pi}2} d\theta \int_0^{2\pi} d\phi
            \sin\theta \cos\theta \\
        &= 2\pi \int_0^{\frac{\pi}2} \frac12 \sin(2\theta) \,d\theta \\
        &= \pi \left(\frac{-\cos(2\theta)}2\right)_0^{\frac{\pi}2} = \pi.
    \end{align*}
\end{solution}

