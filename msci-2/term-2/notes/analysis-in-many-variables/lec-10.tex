\section{Conservation laws and the continuity equation}
\lecture{10}{3/2}

This is an application of the divergence theorem.
Suppose that $\rho(x,y,z,t)$ is the \emph{density} of something
and that $\bm j(x,y,z,t)$ is the corresponding \emph{current} or \emph{flux}.

\begin{example}[Bees]
    Let $\rho$ be the density of bees in a volume
    and $\bm v$ be the average velocity of the bees.
    Then $\bm j = \rho \bm v$.
    That is,
    the number of bees passing through the area element $d\bm A$
    at $(x,y,z)$ is
    \[
        \bm j \cdot d \bm A.
    \]
    Let us apply the conservation of bees in some fixed region $F$ at time $t$:
    \begin{align*}
        B                                          &= \int_V \rho \,dV \\
        \text{rate of change of $B$}               &= \frac{\partial}{\partial t} 
            \int_V \rho \,dV = \int_V \frac{\partial\rho}{\partial t} \,dV \\
        \text{rate of flow of bees out of $V$} = R &= \int_{S=\partial V}
            \bm j \cdot d\bm A.
    \end{align*}
    If no bees are being born or killed then we have
    \[
        \frac{dB}{dt} = -R
    \]
    hence
    \begin{align*}
        \int_V \frac{\partial\rho}{\partial t} \,dV
        &= -\int_S \bm j \cdot d\bm A \\
        &= -\int_V \nabla \cdot \bm j \cdot d\bm A \tag{div thm} \\
        \int_V 
        \left(
            \frac{\partial\rho}{\partial t} + \nabla \cdot \bm j
        \right)
        \,dV &= 0
    \end{align*}
    for all fixed volumes $V$ and so we get 
    \[
        \frac{\partial \rho}{\partial t} = -\bm v \cdot \bm j
    \]
    which is known as a \emph{continuity equation} and is a consequence of a
    \emph{conservation law}.
\end{example}

\section{Path independence of line integrals}

If $C$ is the boundary of a surface $S$ on which 
$\nabla \times \bm F = \bm 0$
then
\[
    \oint_C \bm F \cdot d\bm x 
    = \int_S \nabla \times \bm F \cdot d\bm A 
    = \int_S \bm 0 \cdot d\bm A 
    = 0.
\]

If a vector field $\bm F$ has zero curl in some simply connected region $D$
and if $C_1$ and $C_2$ are two paths connecting the points $\bm a, \bm b \in D$
then
\[
    \int_{C_1} \bm F \cdot d\bm x = \int_{C_2} \bm F \cdot d\bm x.
\]
So the integral depends only on the endpoints.

\begin{remark}
    \emph{Simply connected} just means that all closed curves in $D$ 
    can be shrunk down to a point in $D$ (topology).
\end{remark}

\paragraph{The scalar potential}

If $\bm F = \nabla \phi$ for some scalar field $\phi$ then
$\nabla \times \bm F = \bm 0$ is automatic.
Does this work the other way? That is, does
\[
    \nabla \times \bm F = \bm 0 \implies \bm F = \nabla \phi \;\text{for some $\phi$}.
\]
If $D$ is simply connected, then yes.
Pick $\bm x_0 \in D$ and set
\[
    \phi(\bm x) = \int_{\bm x_0}^{\bm x} \bm F \cdot d\bm x,
\]
this is well-defined as 
$\nabla \bm F = 0$
and $D$ is simply connected.
By the fundamental theorem of calculus, $\nabla \phi = \bm F$.
