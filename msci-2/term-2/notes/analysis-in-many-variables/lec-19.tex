\chapter{Generalised functions and identities}
\section{Preamble}
\lecture{19}{16/3}
In section 12 of the online notes we looked at the convergence
of Fourier series.
The following function was examined:
\[
    h_N(x) = \frac{\sin\left(\left(N + \frac12\right)x\right)}{\pi x}
\]
and its behaviour as $N \to \infty$ considered.
It is \emph{not} defined in this limit;
however, its behaviour inside an integral can be well defined!

For example, it was shown that
\[
    \lim_{N \to \infty} \int_a^b f(y) h_N(y) \,dy =
    \begin{cases}
        f(0) & \text{if $a < 0 < b$}, \\
        0    & \text{otherwise}.
    \end{cases}
    \tag{$\star$}
\]
That is, it \emph{selects} only the value of $f(y)$ at $y = 0$.
We sometimes write the LHS of $(\star)$ as
\[
    \int_a^b f(y) \delta(y) \,dy
\]
where $\delta(y)$ is the \textbf{Dirac delta function};
however, it is \emph{not} a function as we can't assign a value
to it at each $x$.
It only makes sense when it is multiplied by another function
and then integrated.
It is called a \textbf{generalised function},
or a \textbf{distribution}.

\section{Informal definitions}

Generalised functions are interesting objects in their own right.
It is worthwhile to see how to manipulate them.
They are defined by their effects within integrals
rather than by their values at all values of $x$. 
For example
\[
    \int f(x) \delta(x) \,dx = \lim_{N \to \infty} \int f(x) h_N(x) \,dx.
\]
Similarly, we can define equality between two generalised functions $g$ and $h$
by saying: $g = h$ as generalised fucntions if
\[
    \int_{-\infty}^\infty f(x) g(x) \,dx = \int_{-\infty}^\infty f(x) h(x) \,dx
\]
for \emph{all} functions $f$ in some class $K$ (see below).
Note that we integrate over all of $\R$ for simplicity.
We will take $K$, sometimes called the \textbf{space of test functions},
to be the set of all functions which satisfy the following:
\begin{enumerate}
    \item 
        (K1) the functions have finite derivatives of all orders; and

    \item 
        (K2) they vanish outside some bounded region of $\R$ 
        (so we can neglect contributions from $\pm \infty$ when
        integrating by parts).
\end{enumerate}
Furthermore, we will suppose (or define) the usual operations on integrals
(that is, substitution and integration by parts)
to work for generalised functions just as for ordinary functions.
This allows the derivative of any generalised function to be derived as follows.

Say $g$ is a generalised function, then for $f \in K$
\begin{align*}
    \int_{-\infty}^{\infty} f(x) g'(x) \,dx
    &=
    \left[
        f(x) g(x)
    \right]_{-\infty}^{\infty}
    - \int_{-\infty}^{\infty} f'(x) g(x) \,dx \\
    &= -\int_{-\infty}^{\infty} f'(x) g(x) \,dx
\end{align*}
using (K1) and (K2).
This effectively \emph{defines} the derivative $g'(x)$ of any generalised function $g(x)$.

\begin{example}
    Find $\theta'(x)$ where
    \[
        \theta(x) =
        \begin{cases}
            1 & x > 0, \\
            0 & x < 0.
        \end{cases}
    \]
\end{example}

\begin{solution}
    We have for $f \in K$
    \begin{align*}
        \int_{-\infty}^{\infty} f(x) \theta'(x) \,dx
        &= -\int_{-\infty}^{\infty} f'(x) \theta(x) \,dx \\
        &= -\int_0^{\infty} f'(x) \,dx \\
        &= -(f(\infty) - f(0)) \\
        &= f(0).
    \end{align*}
    Now we know
    \[
        \int_{-\infty}^{\infty} f(x) \delta(x) \,dx = f(0)
    \]
    from our definition of $\delta(x)$ from before;
    hence, $\theta'(x) = \delta(x)$ as
    \[
        \int_{-\infty}^{\infty} f(x) \theta'(x) \,dx 
        = \int_{-\infty}^{\infty} f(x) \delta(x) \,dx.
    \]
\end{solution}
