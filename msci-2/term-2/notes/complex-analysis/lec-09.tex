\section{Maximum modulus principle}
\lecture{9}{5/2}

\begin{theorem}[Local maximum principle]
    Let $B = B_r(a) \subset \C$ and $f: B \to \C$ be holomorphic with $\bm a \in B$.
    If
    \[
        \abs{f(z)} \leq \abs{f(a)}
    \]
    for all $z \in B$, then $f$ is constant on $B$.
\end{theorem}

\begin{proof}
    \hfill
    \begin{enumerate}
        \item Firstly, we will prove that $\lvert f(z) \rvert$ is constant
            for all $z \in B$.
            Let $z \in B$, $\rho < r$, and $\gamma(t) = a + \rho e^{i2\pi t}$.
            Then
            \begin{align*}
                \abs{f(a)}
                &= \frac{1}{2\pi} \abs{
                    \int_\gamma \frac{f(z)}{z - a} \,dz
                } \\
                &= \frac{1}{2\pi} \abs{
                    \int_0^1
                    \frac{f(\gamma(t))}{\gamma(t) - a} \gamma'(t)
                    \,dt
                } \\
                &= \frac{1}{2\pi} \abs{
                    \int_0^1
                    \frac{f(\gamma(t))}{\rho e^{i2\pi t}} 
                    2\pi i \rho e^{i2\pi t}
                    \,dt
                } \\
                &= \frac{1}{2\pi} \abs{
                    \int_0^1 f(\gamma(t)) \,dt
                } \\
                &\leq \abs{
                    \int_0^1 f(\gamma(t))\,dt
                } \\
                &\leq \abs{f(a)}.
            \end{align*}
            Hence all of our inequalities are actually equalities.
            Hence
            \[
                \int_0^1 \abs{f(a)} - \abs{f(\gamma(t))} \,dt = 0
            \]
            and so
            \[
                \abs{f(a)} = \abs{f(\gamma(t))}
            \]
            for all $t$ and $\rho$.

        \item Now we will prove that $f(z)$ is a constant in $B$.
            From the first step, we have that $\lvert f(z) \rvert = c$.
            If $c = 0$, then clearly $f(z) = 0$.
            If $c \neq 0$, then $f(z) \cdot \overline{f(z)} = c^2$. So
            \[
                \overline{f(z)} = \frac{c^2}{f(z)} 
            \]
            which is well-defined as $f(z) \neq 0$ and is indeed holomorphic.
            So we claim that both $f$ and $\overline f$ are holomorphic, so if
            \[
                \overline f(z) = u(x,y) - iv(x,y)
            \]
            we see that consider $f$, by the Cauchy-Riemann equations, that
            \[
                u_x = v_y, \qquad u_y = -v_x.
            \]
            But if we consider $\overline f$ we get
            \[
                u_x = -v_y, \qquad u_y = v_x.
            \]
            Hence $u_x = v_y = 0$ and so $f$ must be constant.
    \end{enumerate}
\end{proof}

\begin{theorem}[Maximum principle]
    Let $D \subset \C$ be a domain and
    $f: D \to \C$ be holomorphic.
    If there exists $a \in D$ with
    \[
        \abs{f(a)} \geq \abs{f(z)}
    \]
    for all $z \in D$
    then $f$ is constant on $D$.
\end{theorem}

\begin{remark}
    If $U \subset \C$ is open and path connected,
    then there does not exist $V_1, V_2 \subset \C$
    such that
    \[
        U = V_1 \cup V_2.
    \]
\end{remark}

\begin{proof}[Proof of maximum principle]
    Assume
    \[
        \abs{f(a)} \geq \abs{f(z)}
    \]
    for all $z \in D$ and let
    \[
        z \in U_1 = \{z \in D: f(z) = f(a)\} \neq \varnothing.
    \]
    For small $r$, $B_r(z) \subset D$ is open since $D$ is open.
    Since 
    \[
        \abs{f(z')} \leq \abs{f(z)} = \abs{f(a)}
    \]
    for all $z' \in B_r(z)$, the local maximum principle theorem
    implies that 
    \[
        f(z') = f(a)
    \]
    for all $z' \in B_r(z)$.
    Hence $U_1$ is oepn.
    Let $U_2 = D \subset U_1$. Then
    \[
        U_2 = f^{-1}(\C \setminus \{f(a)\})
    \]
    and is open since $f$ is holomorphic.
    Since $D$ is path connected, $U_2$ must be empty so $U_1 = d$
    and $f(z) = f(a)$ for all $z \in D$.
\end{proof}

\begin{example}
    Let $f(z) = z^2 + 2z - 3$. 
    Find the maximum value of $\abs{f(z)}$ in $B_1(0)$.
\end{example}

\begin{solution}
    $f: B_1(0) \to \C$ is holomorphic.
    By the maximum principle theorem, the extermal points of
    $|abs{f(z)}$ on $\overline{B_1(0)}$ is attained on $\partial B_1(0)$.
    Let $z = e^{it} \in \partial B_1(0)$.
    Then
    \begin{align*}
        \abs{f(z)}^2 
        &= (z^2 + 2z - 3)(\overline z^2 + 2\overline z - 3) \\
        &= (e^{2it} + 2e^{it} - 3)(e^{-2it} + 2e^{-it} - 3) \\
        &= -12\cos^2 t - 8\cos t + 20
    \end{align*}
    which attains its maximum for $\cos t = -\frac13$.
    Therefore
    \[
        \max{\abs{f(z)}^2} = 
        -\frac1{12} \left(-\frac13\right)^2
        - 8\left(-\frac13\right)
        + 20
        = \frac{64}{3} 
    \]
    hence $\max\abs{f} = \frac{8}{\sqrt 3}$. 
\end{solution}
