\lecture{7}{31/1}

\begin{theorem}[Cauchy's integral formula for derivatives]
    Let $r > 0$, $a \in \C$, and $f: B_r(a) \to \C$ be holomorphic.
    Then for any $0 < \rho < r$ we have
    \[
        \int_{\abs{z-a}=\rho} \frac{f(z)}{(z-a)^{n+1}} \,dz
        = \frac{2\pi i}{n!} \, f^{(n)}(a).
    \]
\end{theorem}

\begin{proof}
    We know that $f$ has a convergent power series in $B_r(a)$ given by
    \[
        f(z) = \sum^{\infty}_{n=0} c_n(z-a)^n
    \]
    where
    \[
        c_n = \frac{1}{2\pi i} \int_{\abs{z-a}=\rho} \frac{f(z)}{(z-a)^{n+1} \;dz} 
    \]
    but on the other hand, we also have that $c_n = \frac{f^{(n)}(a)}{n!}$,
    since we know about this power series from a previous corollary.
    Equating these two expressions gives us the statement of the theorem.
\end{proof}

\begin{example}
    Consider the function $f(z) = e^z$.
    We know by the Cauchy-Riemann equations that $f$ is holomorphic in $\C$.
    Therefore, its Taylor series is convergent in all of $\C$.
    We know that $f^{(n)}(z) = e^z$ for all $n \geq 0$,
    so $f^{(n)}(0) = e^0 = 1$ for all $n \in \N$ and
    \[
        f(z) = \sum^{\infty}_{n=0} \frac{f^{(n)}(0)}{n!} z^n = \sum^{\infty}_{n=0} \frac{z^n}{n};
    \]
    we have shown that the two definitions of the exponential function
    are \emph{equivalent}.
\end{example}

\begin{remark}
    The Cauchy-Taylor theorem told us roughly that a function with a compelx derivative at
    every point of a ball has a Taylor series on the ball.
    This is \emph{not} the case in real analysis.
\end{remark}

\begin{corollary}[Holomorphic functions have infinitely many derivatives]
    If $U \subset \C$ is an open set and $f: U \to \C$ is holomorphic,
    then $f$ has derivatives of all orders on $U$ and they are all holomorphic.
\end{corollary}

\begin{proof}
    Let $a \in U$ and $r > 0$ such that $B_r(a) \subset U$.
    Then by the teorem above we have
    \[
        f(z) = \sum^{\infty}_{n=0} (z-a)^n
    \]
    for all $z \in B_r(a)$.
    But then
    \[
        f'(z) = \sum^{\infty}_{n=1} n c_n (z-a)^{n-1}
    \]
    for $z \in B_r(z)$.
    That is to say, around any point $a$ of $U$ we have a Taylor series representation;
    that is, $f'$ is holomorphic.
    We can repeat this replacing $f'$ with $f$ to see that $f$ must have infinite derivatives,
    each being holomorphic.
\end{proof}

\begin{remark}
    This illuminates \emph{another} difference between real and complex analysis.
    Consider the real function $f(x) = \abs{x} \, x^n$ for $x \in \R$.
    $f$ is \emph{exactly} $n$-times differentiable at $x = 0$, but the derivative
    $f^{(n+1)}$ is \emph{not} defined at $x = 0$.
\end{remark}

The above corollary allows us to prove a nice theorem called \emph{Morera's theorem},
which can be viewed as a converse to Cauchy's theorem.
