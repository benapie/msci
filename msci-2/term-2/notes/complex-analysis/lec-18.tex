\section{The argument principle and Rouch\'e's theorem}
\lecture{18}{18/3}

\begin{lemma}[]
    Let $f$ be meromorphic on a domain $D$ 
    with a zero or pole of order $k > 0$ at $a \in D$.
    Then the function $\frac{f'(z)}{f(z)}$ has a simple pole at $a$ and
    \[
        \Res_{z=a} \left(\frac{f'(z)}{f(z)}\right) =
        \begin{cases}
            k  & \text{if $f$ has a zero at $a$}, \\
            -k & \text{if $f$ has a pole at $a$}.
        \end{cases}
    \]
\end{lemma}

\begin{proof}
    Assume that $f$ has a zero of order $k > 0$ at $a$.
    Then we may write
    \[
        f(z) = (z-a)^k g(z), \qquad g(a) \neq 0
    \]
    for all $z \in B_R^\star(a)$, $R>0$, and $g$ is holomorphic on $B_R(a)$.
    But then
    \[
        \frac{f'(z)}{f(z)} = \frac{k}{z - a} + \frac{g'(z)}{g(z)} 
    \]
    for all $z \in B_R^\star(a)$.
    Since $g(z)$ is holomorphic on $B_R(a)$ and $g(a) \neq 0$,
    $\frac{g'(z)}{g(z)}$ has a removable singularity at $a$ and so we have that
    \[
        \Res_{z=a} \left(\frac{f'(z)}{f(z)}\right)
        = \Res_{z=a} \left(\frac{k}{z-a} + \frac{g'(z)}{g(z)}\right) 
        = k.
    \]
    Similarly one shows the case of $f$ having a pole of order $k > 0$ at $z = a$
    where we write
    \[
        f(z) = (z-a)^{-k} g(z)
    \]
    with $g$ holomorphic around $a$ and $g(a) \neq 0$.
\end{proof}

\begin{theorem}[Argument princple]
    Let $\gamma$ be a simple closed postively oriented contour
    and $f$ be meromorphic on $D_\gamma^{\text{int}} \cup \gamma$.
    Assume $f$ has no zeros or poles on $\gamma$ 
    and let
    \begin{enumerate}
        \item $Z_f$ denote the number of zeros of $f$ in $D_\gamma^{\text{int}}$; and
        \item $P_f$ denote the number of poles of $f$ in $D_\gamma^{\text{int}}$
    \end{enumerate}
    both counted with multiplicities.
    Then
    \[
        \frac{1}{2\pi i} \int_\gamma \frac{f'(z)}{f(z)} \,dz = Z_f - P_f.
    \]
\end{theorem}

\begin{remark}
    When we say \emph{`counted with multiplicities'} 
    we mean that when we have a zero of order $k$ we count $k$ towards $Z_f$
    (and similarly for poles).
\end{remark}

\begin{example}
    Let
    \[
        f(z) = \frac{(z-3)^3(z-1)^7z^3}{(z-i)^4(z+4)^5(z-3i)^7}
    \]
    and $\gamma(\theta) = \frac72 e^{i\theta}$, $\theta \in [0, 2\pi]$.
    Evaluate the integral
    \[
        \int_{\gamma} \frac{f'(z)}{f(z)} \,dz.
    \]
\end{example}

\begin{solution}
    $f$ has $13$ zeros and $11$ poles, hence
    \[
        \int_\gamma \frac{f'(z)}{f(z)} \,dz = 2\pi i (13 - 11) = 4\pi i.
    \]
\end{solution}

\begin{theorem}[Rouch\'e's]
    Let $\gamma$ be a simple closed contour
    and $f$ and $g$ be holomorphic functions on $D_\gamma^\text{int} \cup \gamma$.
    Suppose that
    \[
        \abs{f(z) - g(z)} < \abs{g(z)}
    \]
    for all $z \in \gamma$.
    Then $f(z)$ and $g(z)$ have the same number of zeros
    (counted with multiplicities) inside $\gamma$.
\end{theorem}

\begin{example}
    We can use Rouch\'e's theorem to give us more information of the location
    of zeros of functions.
    For example, consider
    \[
        P(z) = z^4 + 6z + 3.
    \]
    We know that $P$ has 4 zeros, but we can find out more than that.
    Consider $\gamma: \lvert z \rvert = 2$ and set $g(z) = z^4$.
    Then for all $z \in \gamma$,
    \[
        \abs{g(z)} 
        = \abs z^4 
        = 16 > 15 
        = 6 \abs z + 3 \geq \abs{6z + 3} 
        = \abs{P(z) - g(z)};
    \]
    therefore, $P(z)$ and $g(z)$ have the same number of zeros.
    Clearly $g$ has a zero when $z = 0$ with order $4$ within $\gamma$
    hence we conclude that the number of zeros of $P$ inside $\gamma$ 
    (counted with multiplicity)
    is $4$.
    Now we consider the curve $\gamma: \abs = 1$ and set $g(z) = 6z$. 
    Then for any $z \in \gamma$ we have
    \[
        \abs{g(z)}
        = 6\abs z
        = 6 > 4
        = 1 + 3
        \geq \abs z^4 + 3
        \geq \abs{z^4 + 3}
        = \abs{P(z) - g(z)}.
    \]
    By Rouch\'e's theorem we have that $g(z)$ and $P(z)$ have the same
    number of zeroes inside $\gamma$,
    but clearly $6z$ has one zero at $z = 0$.
    Therefore $P(z)$ has one zero of modulus less than $1$ and has three zeros
    of modulus between $1$ and $2$.
\end{example}

\begin{example}
    Consider the function
    \[
        f(z) = \cos(\pi z) - \alpha z^m
    \]
    where $\alpha > e^\pi$ and $m \in \N$.
    We will show that this function has $m$ zeros in $B_1(0)$.
    We set $g(z) = -\alpha z^m$ and consider the curve $\gamma: \abs z = 1$.
    Then for all $z \in \gamma$ we have
    \[
        \abs{g(z)} = \alpha > e^\pi.
    \]
    On the other hand, for all $z \in \gamma$ we have
    \[
        \abs{f(z) - g(z)} 
        = \abs{\cos(\pi z)} 
        = \left\lvert \frac{e^{i\pi z} + e^{-i\pi z}}{2} \right\rvert
        \leq \frac{\abs{e^{i\pi z}} + \abs{e^{-i\pi z}}}{2}.
    \]
    Since $z \in \gamma$, if we write $z = x + iy$ we have $x,y \in [-1,1]$.
    In particular, we have that for all $z \in \gamma$
    \[
        \abs{e^{i\pi z}} = \abs{e^{i\pi(x + iy)}} = e^{-\pi y} \leq e^\pi
    \]
    and similarly
    \[
        \abs{e^{-i\pi z}} = \abs{e^{-i\pi(x + iy)}} = e^{-\pi y} \leq e^\pi.
    \]
    Hence we have that for all $z \in \gamma$
    \[
        \abs{f(z) - g(z)} \leq e^\pi < \alpha = \abs{g(z)}.
    \]
    By Rouch\'e's theorem we have that the functions $f(z)$ and $g(z)$ have the same number of zeros,
    counted with multiplicities, inside the unit disc.
    Clearly $g(z) = \alpha z^m$ only has $z = 0$ as a zero with multiplicity $m$.
    Therefore, $f$ has the same $m$-many zeros inside the unit disc.
\end{example}
