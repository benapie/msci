\lecture{12}{14/2}

\begin{theorem}[Cauchy's theorem for simple closed contours]
    Let $\gamma$ be a simple closed positively oriented contour,
    and $f: D_\gamma^\text{int} \cup \gamma \to \C$.
    Then we have
    \[
        \int_\gamma f(z) \,dz = 0.
    \]
\end{theorem}

\begin{theorem}[CIF for simple closed contours]
    Let $\gamma$ be a simple closed contour.
    For $w \in D^\text{int}_\gamma$
    \[
        \int_\gamma \frac{f(z)}{z - w} \,dz
        = 2\pi i f(w).
    \]
\end{theorem}

\chapter{Holomorphic functions on punctured domains}

Recall with Cauchy's theorem, we were considering domains
with a finite set of points removed. 
We will continue this idea this chapter, studying
the behaviour of functions $f: D \to \C$ on a domain $D$
that we know are holomorphic on $D$ except for at a finite set
of points.
The simplest example of this is when we are given a holomorphic
function $f$ on a punctured ball
$B_r^\star(a) = B_r(a) \setminus \{a\}$.

\section{Laurent series}

\begin{definition}[Laurent series]
    A \textbf{Laurent series} is an infinite series
    of the form
    \[
        \sum_{n=-\infty}^\infty c_n(z-a)^n
    \]
    where $c_n$ and $a$ are complex numbers.
    $a$ is called the \emph{center} of the Laurent series.
\end{definition}

\begin{definition}[Annulus]
    We write
    \[
        A_{r,R}(a) = \{z \in \C: r < \abs{z - a} < R\}
    \]
    and call this an \textbf{annulus} of center $a$,
    internal radius $r$, and external radius $R$.
\end{definition}

\begin{remark}
    Since we allow $r = 0$ or $R = \infty$, our definition
    of annulus $A_{r,R}(a)$ includes the sets
    $B_r^\star(a)$, 
    $\C^\star$, and 
    $\C \setminus \overline B_r(a)$.
\end{remark}

\begin{proposition}[]
    Given a Laurent series 
    $\sum_{n=-\infty}^\infty c_n(z-a)^n$
    that is not a power series
    (that is, $c_n \neq 0$ for some $n < 0$)
    then either
    \begin{enumerate}
        \item the Laurent series never converges; or
        \item there are $0 \leq r < R \leq \infty$
            such that
            \begin{enumerate}
                \item the series converges absolutely when
                    $r < \abs{z-a} < R$ and diverges when
                    either $\abs{z-a} < r$ or
                    $\abs{z-a} > R$,
                    in this case the annulus $A_{r,R}(a)$
                    is called the \textbf{annulus of convergence}
                    of the Laurent series; and
                    
                \item the series converges uniformly on any annulus
                    $A_{\rho_1,\rho_2}(a)$ with
                    $r < \rho_1 < \rho_2 < R$;
                    hence, the Laurent series converges
                    locally uniformly in their annulus of convergence
                    and so \emph{Laurent series give holomorphic functions
                    in their annulus of convergence}.
            \end{enumerate}
    \end{enumerate}
\end{proposition}

\begin{proof}
    By definition, the Laurent series convergences absolutely
    if and only if both of
    \[
        F_1(z) = \sum^{\infty}_{n=0} c_n(z-a)^n, \qquad
        F_2(z) = \sum^{-1}_{n=-\infty} c_n(z-a)^n
    \]
    converges absolutely.
    The first expression is a power series that we already understand.
    If $F_1(z)$ only converges when $z = a$, then the Laurent series
    never converges since $F_2$ is not defined at $a$.
    Otherwise, let $0 < R \leq \infty$ be the radius of convergence of
    $F_1$.
    For $F_2$, we introduce a new variable $\zeta = (z - a)^{-1}$.
    Then
    \[
        F_2(z) = \sum^{\infty}_{n = 1} c_{-n} \zeta^n
    \]
    is a power series in the variable $\gamma$ with center $0$.
    If this power series converges only at $\gamma = 0$, then
    $F_2$ never converges. 
    So suppose that doesn't happen and let 
    $0 < R' \leq \infty$ be the radius of convergence of this 
    power series.
    Let
    \[
        r=
        \begin{cases}
            R'^{-1} & \text{if}\; 0<R'<\infty \\
            \infty  & \text{if}\; R'=0.
        \end{cases}
    \]
    Then the series $\sum_{n = -\infty}^{-1} c_n(z-a)^n$ 
    converges absolutely when $\abs{\zeta} < R'$.
    That is, $\abs{z-a} > r$.
    And such, diverges when $\abs{z-a} < r$.
    This proves either case of (i) or (ii)(a) in the statement
    of the proposition.
    We now show that if (ii)(a) is true, so is (ii)(b).
    As a power series is locally uniformly convergent on any
    open ball on the center of the power series with radius less
    than the radius of convergence,
    $F_1(z)$ converges locally uniformly on $B_{\rho_2}(a)$ and
    $\tilde F(\zeta) = F_2(z)$ as a power series in $\zeta$
    with radius of convergence $r$ converges uniformly on 
    $\abs{\zeta} < \rho_1^{-1}$, since $\rho_1^{-1} < R'$.
    That is, it converges uniformly when $\abs{z-a} > \rho_1$.
    Hence both halves of the Laurent series converge uniformly
    on $A_{\rho_1,\rho_2}(a)$.
\end{proof}

\begin{definition}[]
    Given a Laurent series 
    \[
        \sum_{n = -\infty}^\infty c_n (z - a)^n,
    \]
    the sum 
    \[
        \sum_{n = -\infty}^{-1} c_n (z-a)^n
    \]
    is called the \textbf{principal part} of the Laurent series.
\end{definition}
