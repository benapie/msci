\lecture{2}{15/1}

\begin{proof}
    \begin{align*}
        \int_\delta f(z) \,dz
        &= \int_{a'}^{b'} f(\delta(t)) \delta'(t) \,dt \\
        &= \int_{a'}^{b'} f(\gamma(\phi(t))(\gamma \circ \phi)'(t) \,dt \\
        &= \int_{a'}^{b'} f(\gamma(\phi(t)) \gamma'(\phi(t)) \phi'(t) \,dt \\
        &= \int_a^b f(\gamma(t)) \gamma'(t) \,dt \\
        &= \int_\gamma f(z) \,dz
    \end{align*}
    where $s = \phi(t)$, so $\phi'(t)\,dt = ds$.
\end{proof}

\begin{definition}[Contour]
    Let $\gamma : [a, b] \to \C$ be a curve.
    Suppose that there are
    \[ a = a_0 < a_1 < \ldots < a_{n-1} < a_n = b \]
    such that
    \[ \gamma_i:[a_{i - 1}, a_i], \qquad \gamma_i(t) = \gamma(t) \]
    are $C^1$, then $\gamma$ is called \textbf{piecewise $C^1$} or a
    \textbf{contour}.
\end{definition}

If $U \subset \C$ is open and $f: U \to \C$ is continuous then we have
\[ \int_\gamma f(z) \, dz = \sum_{i = 1}^n \left(\int_{\gamma_i} f(z) \,dz\right). \]
All properties of integrals over $C^1$ curves also hold for contours.

Now suppose $\gamma: [a, b] \to \C$ and $\delta:[c, d] \to \C$ are contours
with $\gamma(b) = \delta(c)$. Then we define the contour
\[ (\gamma + \delta): [a, b + d - c] \to \C \]
by
\[
    (\gamma + \delta) =
    \begin{cases}
        \gamma(t) & t \in [a,b] \\
        \delta(t + c - b) & t \in [b, b + d - c] \\
    \end{cases}
    .
\]
We have
\[ 
    \int_{\gamma + \delta} f(z) \,dz =
    \int_\gamma f(z) \,dz + \int_\delta f(z) \,dz.
\]

\section{Fundamental theorem of calculus}

\begin{theorem}[Fundamental theorem of calculus]
    Let $U \subset \C$ be an open set,
    $f: U \to \C$ be a holomorphic function, and
    $\gamma: [a, b] \to U$ be a contour.
    Then
    \[ \int_\gamma f'(z) \,dz = f(\gamma(b)) - f(\gamma(a)). \]
    If $\gamma$ is closed (that is, $\gamma(a) = \gamma(b)$)
    then $\int_\gamma f'(z) \,dz = 0$.
\end{theorem}

\begin{proof}
    \[ \int_\gamma f'(z) \,dz 
        = \int_a^b f'(\gamma(t)) \gamma'(t) \,dt 
        = \int_a^b (f \circ \gamma)'(t) \,dt \tag{$\star$}. \]
    If $u + iv$ then
    \begin{align*}
        (\star)
        &= \int_a^b (u \circ \gamma)'(t) \, dt + i\int_a^b (v \circ \gamma)'(t) \,dt \\
        &= u(\gamma(b)) - u(\gamma(a)) + i(v(\gamma(b)) - v(\gamma(a))) \\
        &= f(\gamma(b)) - f(\gamma(a)).
    \end{align*}
    This applies for $\gamma$ being a $C^1$ curve;
    however, to extend this for $\gamma$ being a contour
    we simply sum the integrals over all the $C^1$ curves that make up
    $\gamma$.
\end{proof}

\begin{remark}
    The fundamental theorem of calculus shows us that $\int_\gamma f'(z) \,dz$
    depends only on the endpoints of $\gamma$ if $f$ holds on an open set containing
    $\gamma$.
\end{remark}

\begin{example}
    Calculate
    \[ \int_{\lvert z \rvert = r} z^n \,dz. \]
\end{example}

\begin{solution}
    $\int_{\lvert z \rvert = r} z^n \,dz = \int_\gamma z^n \,dz$ where
    $\gamma(\theta) = re^{i\theta}: [0, 2\pi] \to \C$.
    If $n \neq 1$, then $z^n = f'(z)$ where $f(z) = \frac1{n+1}z^{n+1}$.
    $f$ is holomorphic on $\C^\star$; hence we meet the requirements
    for the fundamental theorem of calculus (FTC).
    Therefore
    \[
        \int_{\lvert z \rvert = r} z^n \, dz 
        = f(\gamma(b)) - f(\gamma(a))
        = f(\gamma(2\pi)) - f(\gamma(0))
        = 0.
    \]
    If $n = 1$, there is no obvious holomorphic antiderivative of
    $z^{-1}$ on $\C^\star$ or any set containing the circle $S$.
    \begin{align*}
        \int_{\lvert z \rvert = r} z^n \,dz
        &= \int_\gamma z^{-1} \,dz \\
        &= \int_0^{2\pi} \left(re^{i\theta}\right)^{-1} 
           \left(re^{i\theta}\right)' \,dz \\
        &= \int_0^{2\pi} \left(\frac{rie^{i\theta}}{re^{i\theta}}\right) \,d\theta \\
        &= \int_0^{2\pi} i \,d\theta = 2\pi i \neq 0;
    \end{align*}
    therefore, $z^{-1}$ has no holomorphic antiderivative on $\C^\star$ by the
    fundamental theorem of calculus.
\end{solution}
