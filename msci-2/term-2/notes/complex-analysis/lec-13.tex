\lecture{13}{17/2}

\begin{proposition}[]
    Let
    \[
        f(z) = \sum^{\infty}_{n=-\infty} c_n(z-a)^n
    \]
    be convergent on $A_{r,R}(z)$ with $a \in \C$ and
    $0 \leq r < R \leq \infty$.
    Then for any $\rho$ with $0 < r < \rho < R$
    \[
        c_n = \frac1{2\pi i} \int_{\abs{z-a}=\rho} \frac{f(z)}{(z-a)^{n+1}} \,dz.
    \]
\end{proposition}

\begin{theorem}[Laurent's]
    Let $a \in \C$, $0 \leq r < R \leq \infty$, and
    $f: A_{r,R}(a) \to \C$ be holomorphic.
    Then there exists an unique $c_n$ such that
    \[
        f(z) = \sum^{\infty}_{n = -\infty} c_n (z-a)^n
    \]
    converges for all $z \in A_{r,R}(a)$.
\end{theorem}

\section{Classifications of singularities}

Given $f: B_r^\star(a) \to \C$ holomorphic, what can happen at $a$?
Let us write
\[
    f(z) = \sum_{n=-\infty}^\infty c_n(z-a)^n
\]
as our unique representation of $f$ in $B_r^\star(a)$.
Then we have the following cases.

\begin{description}
    \item[$f$ has a removable singularity at $a$.]
        If $c_n = 0$ for all $n < 0$ 
        (that is, the principal part of $f$ is zero)
        then we say that $f$ has a removable singularity.
        \begin{example}
            Let
            \[
                f(z) = \frac{e^z - 1}{z}.
            \]
            $f$ is holomorphic on $B_1^\star(0)$ but it is \emph{not} defined at $0$.
            \[
                f(z) = \frac1z
                \left(
                    \sum^{\infty}_{n=0} \frac{z^n}{n!}
                \right)
                = \frac1z \sum^{\infty}_{n=0} \frac{z^n}{n!} 
                = \sum^{\infty}_{n=0} \frac{z^n}{(n+1)!} 
            \]
            and so $f$ has a power series;
            therefore, $f$ has a removable singularity at $0$.
        \end{example}

    \item[$f$ has a pole at $z=a$.]
        If there exists a $k>0$ such that $c_{-k} \neq 0$
        and $c_n = 0$ for all $n < -k$
        then we say that $f$ has a \textbf{pole} of order $k$ at $z=a$.
        Note that this means that the principal part of $f$ is not zero,
        but contains finitely many non-zero terms.
        \begin{example}
            Let
            \[
                g(z) = \frac{e^z - 1}{z^2} 
            \]
            be holomorphic on $B_1^\star(0)$.
            Then
            \[
                g(z) = \frac{f(z)}{z} = \sum^{\infty}_{n = -1} \frac{z^n}{(n+2)!}
            \]
            so this is \emph{not} a power series and therefore $g$ has a pole of order $1$
            at $0$.
        \end{example}

    \item[$f$ has an essential singularity at $z = a$]
        If there are infinitely many $n<0$ such that $c_n \neq 0$ then
        $f$ has an \textbf{essential singularity} at $z = a$.
        \begin{example}
            Let $f(z) = e^{\frac1z}$ be holomorphic on $\C^\star$.
            Then
            \[
                f(z) 
                = e^{\frac1z} 
                = \sum^{\infty}_{n=0} \frac{\left(\frac1z\right)^n}{n!}
                = \sum^{\infty}_{n=0} \frac{1}{z^n\,n!} 
                = \sum^{0}_{n=-\infty} \frac{z^n}{(-n)!}  
            \]
            is the Laurent series of $f$ and therefore $f$ has an essential singularity at
            $0$.
        \end{example}
\end{description}

\section{Removable singularities}

\begin{lemma}[]
    Let $a \in \C$, $r > 0$, and $f$ be holomorphic on $B_r^\star(a)$.
    Then $f$ has a removable singularity at $a$ if and only if $f$ extends to
    a holomorphic function on $B_r(a)$.
\end{lemma}

\begin{proposition}[]
    Let $a \in \C$, $r > 0$, and $f: B_r^\star(a) \to \C$ be holomorphic.
    Then $f$ has a removable singularity at $a$ if and only if
    \[
        \lim_{z \to a} (z-a) \, f(z) = 0.
    \]
\end{proposition}

\begin{example}
    Let
    \[
        f(z) = \frac{e^z - 1}{z}.
    \]
    $f$ is holomorphic on $B_r^\star(0)$ for all $r > 0$.
    We have
    \[
        \lim_{z \to 0} z f(z) 
        = \lim_{z \to 0} \left(e^z-1\right)
        = 0
    \]
    and so $f$ has a removable singularity at $0$.
\end{example}

\begin{theorem}[Riemann extension]
    If $f: B_r^\star(a) \to \C$ is bounded then $f$ has a removable singularity.
\end{theorem}

\section{Poles}

\begin{proposition}[]
    Let $a \in C$, $r>0$, and $f: B_r^\star(a) \to \C$ be holomorphic.
    Then the following are equivalent.
    \begin{enumerate}
        \item $f$ has a pole of order $k$ at $a$.
        \item $f(z) = (z-a)^{-k}g(z)$ where $g:B_r(a) \to \C$ is holomorphic and $g(a) \neq 0$.
        \item There exists $\varepsilon$ such that $0 < \varepsilon < r$
            and there exists $h: B_\varepsilon(a) \to \C$ with a zero of order $k$ at $a$
            such that 
            \[
                f(z) = \frac{1}{h(z)} 
            \]
            for all $z \in B_{\varepsilon}^\star(a)$.
    \end{enumerate}
\end{proposition}

