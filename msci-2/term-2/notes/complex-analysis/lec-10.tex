\section{Analytic continuation and the identity theorem}
\lecture{10}{10/2}

Let $f: B_r(a) \to \C$ be holomorphic and
$f \not\equiv 0$. 
We can write
\[
    f(z) = \sum^{\infty}_{n = 0} c_n(z-a)^n.
\]
As $f \not\equiv 0$, then not all $c_k = 0$.
$c_0 = f(a)$, so 
\[
    f(a) = 0 \iff c_0 = 0.
\]

\begin{definition}[]
    Let $f: B_r(a) \to \C$ be holomorphic, $f \not\equiv 0$,
    and $m = \min\{n : c_n \neq 0\}$.
    Then we say $f$ has a zero of order $m$ at $a$.
\end{definition}

\begin{remark}
    Since $c_n = f^{(n)}(a) \cdot \frac1{n!}$,
    $f$ has a zero of order $m$ at $a$ if and only if
    \[
        f(a) = 0 
        = f^{(1)}(a) 
        = f^{(2)}(a) 
        = \ldots
        = f^{(m-1)}(a)
    \]
    \emph{but} $f^{(m)}(a) \neq 0$.
\end{remark}

\begin{example}
    Let $f(z) = z(e^z - 1)$.
    $f(0) = 0$.
    \[
        f(z) = z
        \left(
            \sum^{\infty}_{n=0} \frac{z^n}{n!} - 1
        \right)
        = z \sum^{\infty}_{n = 1} \frac{z^n}{n!} 
        = \sum^{\infty}_{n=1} \frac{z^{n+1}}{n!} 
        = \sum^{\infty}_{n=2} \frac{z^n}{(n-1)!}.
    \]
    So $f$ has a zero of order $2$ at $0$.
\end{example}

\begin{theorem}[Principle of isolated zeros]
    Let $f:B_r(a) \to \C$ be holomorphic
    such that $f \not\equiv 0$.
    Then there exists $\rho$ where $0 < \rho \leq r$
    such that $f(z) \neq 0$ for all
    $z \in B_\rho(a) \setminus \{a\} = B^\star_\rho(a)$.
\end{theorem}

\begin{proof}
    \hfill
    \begin{description}
        \item[$f(a) \neq 0$]
            $f$ is continuous, so there exists
            $0 < \rho \leq r$ such that
            \[
                f(B_\rho(a)) \leq B_{\frac{\abs{f(a)}}{2}}(f(a))
                \subset \C^\star
            \]
            hence $f(z) \neq 0$ for all $z \in B_\rho(a)$.

        \item[$f(a) = 0$]
            We write 
            \[
                f(z) = \sum_{n = m}^\infty c_n (z-a)^n
            \]
            where $m$ is the order of the zero of $f$
            at $a$ ($c_m \neq 0$).
            So
            \[
                f(z)
                = (z-a)^m
                \sum^{\infty}_{n = 0} c_{n+m} (z-a)^n
            \]
            and we set
            \[
                h(z)
                = \sum^{\infty}_{n = 0} c_{n+m} (z-a)^n
                = \frac{f(z)}{(z-a)^m} 
            \]
            for all $z \in B_r^\star(a)$.
            The power series for $h(z)$ converges on 
            $B_r^\star(a)$.
            Hence, it converges on $B_r(a)$.
            So $h$ is holomorphic on $B_r(a)$.
            Also $h(a) = c_m \neq 0$.
            By case one, there exists a $\rho$
            with $0 < \rho \leq r$
            such that $h(z) \neq 0$
            for all $B_\rho(a)$.
            So $f(z) \neq 0$ for all $z \in B_\rho^\star(a)$
            as $(z - a)^m \neq 0$ and $h(z) \neq 0$.
    \end{description}
\end{proof}

\begin{theorem}[Uniqueness of analytic continuation]
    Let $D' \subset D$ be non-empty domains in $\C$.
    Suppose $f: D' \to \C$ is holomorphic.
    Then there is \emph{at most} one holomorphic function
    $f: G \to \C$ such that $g(z) = f(z)$ for all 
    $z \in D$.
    If $g$ exists, it is called the 
    \textbf{analytical continuation} of $f$ to $D$.
\end{theorem}

\begin{example}
    Let $f(z) = \sum_{n=0}^\infty z^n$.
    $f$ is holomorphic on $B_1(0)$.
    Now let $g(z) = \frac{1}{1 - z}$.
    $g$ agrees with $f$ on $B_1(0)$ but is holomorphic on
    $\C \setminus \{1\}$.
    $g$ is the analytic continuation of $f$.
\end{example}

\begin{proof}
    Suppose $g_1, g_2: D \to \C$ be holomorphic and suppose
    $g_1(z) = g_2(z) = f(z)$ for all $z \in D'$.
    Let $h(z) = g_1(z) - g_2(z)$.
    Then $h = 0$ for all $z \in D'$.
    $h$ is holomorphic on $D$.
    We want to show that $h \equiv 0$ on $D$.
    We are going to use the fact that $D$ is connected,
    so $D$ cannot be partitioned into disjoint 
    non-empty open sets.
    Let
    \begin{align*}
        D_0 &= 
        \{
            w \in D: \;\exists\; r>0: h(z)=0
            \;\forall\; z\in B_r(w)
        \} \\
        D_1 &=
        \{w \in D: h^{(n)}(w) \neq 0, n \geq 0\}.
    \end{align*}
    We will prove that $D_0$ is open and non-empty,
    $D_1$ is open, $D_0 \cup D_1 = D$,
    and $D_0 \cap D_1 = \varnothing$ next lecture, but we
    will assume it to finish the proof.
    So $D_1$ is empty, so $D_0 = D$. Therefore,
    $h(w) = 0$ for all $w \in D$ and thus
    $g_1(w) = g_2(w)$ for all $w \in D$.
\end{proof}
