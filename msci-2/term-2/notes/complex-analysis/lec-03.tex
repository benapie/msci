\lecture{3}{20/1}

\paragraph{Problem}
Given $U \subset \C$ open and $f: U \to \C$,
when does $f$ have a holomorphic antiderivative $F$ on $U$?

We call $F$ the \textbf{primitive} of $f$.

\begin{definition}[Contour length]
    Let $\gamma: [a, b] \to \C$ be a contour.
    The \textbf{length} of $\gamma$ is
    \[ 
        L(\gamma) = \int_a^b \abs{\gamma'(t)} \, dt. 
    \]
\end{definition}

\begin{lemma}[Estimation lemma]
    Let $U \subset \C$ be open,
    $f: U \to \C$ be continuous, and
    $\gamma: [a, b] \to U$ be a contour.
    Then
    \[
        \left\lvert
            \int_\gamma f(z) \,dz
        \right\rvert
        \leq L(\gamma) \sup_\gamma \lvert f \rvert.
    \]
\end{lemma}

\begin{proof}
    This proof is for $C^1$ curves,
    proving this for contours is left as an exercise %todo
    but it is not that hard.
    
    \paragraph{Claim}
    If $g: [a, b] \to \C$ is continuous then
    \[
        \left\lvert
            \int_a^b g(t) \,dt
        \right\rvert
        \leq \int_a^b \lvert g(t) \rvert \, dt.
    \]
    You can think of this as the triangle equality where instead
    of having sum, you have an integral.
    Let
    \[
        \int_a^b g(t) \,dt = re^{i\theta};
    \]
    so $\left\lvert\int_a^b g(t) \,dt \right\rvert = r$.
    Then
    \begin{align*}
        \left\lvert\int_a^b g(t) \,dt \right\rvert
        &=    e^{-i\theta} \int_a^b g(t) \,dt \\
        &=    \Re\left(e^{-i\theta}\int_a^bg(t)\,dt\right) \\
        &=    \int_a^b \Re\left(e^{-i\theta} g(t)\right) \,dt \\
        &\leq \int_a^b \left\lvert e^{-i\theta} g(t) \right\rvert \,dt \\
        &=    \int_a^b \left\lvert g(t) \right\rvert \,dt.
    \end{align*}
    This proves our claim.
    Then
    \begin{align*}
        \left\lvert \int_\gamma f(z)\,dz \right\rvert
        &=    \left\lvert \int_a^b f(\gamma(t)) \gamma'(t) \,dt \right\rvert \\
        &\leq \int_a^b \lvert f(\gamma(t)) \rvert \lvert \gamma'(t) \rvert \,dt \\
        &\leq \sup_\gamma \lvert f \rvert \int_a^b \lvert \gamma'(t) \rvert \,dt \\
        &=    L(\gamma) \sup_\gamma \lvert f \rvert
    \end{align*}
\end{proof}

\begin{theorem}[Converse to FTC]
    Let $D \subset \C$ be a domain,
    $f: D \to \C$ be continuous, and
    \[
        \int_\gamma f(z) \, dz = 0
    \]
    for all closed contours $\gamma$.
    Then $f$ has a holomorphic antiderivative
    $F: D \to \C$
    such that
    $F'(z) = f(z)$
    for all $z \in D$.
\end{theorem}

\begin{proof}
    Fix $a \in D$.
    Then for all $w \in D$ there exists a smooth path connecting $a$ and $w$,
    from the definition of a domain.
    Let $\gamma_w$ be this path, define
    \[ F(w) = \int_{\gamma_w} f(z) \, dz. \]
    We will skip the rest of this proof into two parts.
    \begin{enumerate}
        \item
            First we will show that the choice of $\gamma$ does \emph{not}
            affect $F$.

            Consider $\tilde\gamma_w$, another contour connecting $a$ and $w$.
            Now consider the closed contour
            $\gamma_w + (-\tilde\gamma_w)$.
            Now
            \[ 
                0 
                = \int_{\gamma_w + (-\tilde\gamma_w}) f(z) \, dz
                = \int_{\gamma_w} f(z) \, dz - \int_{\tilde\gamma_w} f(z) \, dz; 
            \]
            hence,
            \[
                \int_{\gamma_w} f(z) \, dz = \int_{\tilde\gamma_w} f(z) \, dz.
            \]
            Therefore, $F$ does not depend on $\gamma$.

        \item
            Now we will show that $F$ is holomorphic with deriative $f$.
            That is, for all $w \in D$
            \[
                \lim_{h \to 0} \left( \frac{F(w + h) - F(w)}{h} \right) = f(w).
            \]
            Now we pick $r > 0$ such that $B_r(w) \subset D$.
            This ball exists since $D$ is open.
            Then for any $h \in \C$ with $\lvert h \rvert < r$
            we consider the straight line $\delta_h$ that connects 
            the point $w$ to $w + h$.
            A parametrisation of such a line is given by
            \[
                \delta_h: [0, 1] \to D, \qquad t \mapsto w + th.
            \]
            Then
            \begin{align*}
                F(w + h)
                &= \int_{\gamma + \delta_h} f(z) \, dz \\
                &= \int_\gamma f(z) \, dz + \int_{\delta_h} f(z) \, dz \\
                &= F(w) + \int_{\delta_h} f(z) \, dz \\
                \frac{F(w + h) - F(w)}{h} 
                &= \frac1h \int_{\delta_h} f(z) \, dz \\
                \frac{F(w + h) - F(w)}{h} - f(w)
                &= \frac1h \int_{\delta_h} (f(z) - f(w)) \, dz
            \end{align*}
            as
            \[
                \int_{\delta_h} f(w) \, dz
                = f(w) \int_{\delta_h} 1 \, dz
                = hf(w).
            \]
            Now
            \begin{align*}
                \left\lvert \frac{F(w + h) - F(w)}{h} - f(w) \right\rvert
                &=    \left\lvert \frac1h \int_{\delta_h} (f(z) - f(w)) \, dz \right\rvert \\
                &=    \frac1{\lvert h \rvert}
                    \left\lvert
                        \int_{\delta_h} (f(z) - f(w)) \, dz
                    \right\rvert \\
                &\leq \frac1{\lvert h \rvert} L(\delta_h) 
                    \sup_{\delta_h}\lvert f(z) - f(w) \rvert
                \tag{est. lemma} \\
                &=    \sup_{\delta_h} \lvert f(z) - f(w) \rvert.
            \end{align*}
            As $h \to 0$, $\delta_h$ just becomes the point $w$.
            Therefore
            \[
                \left\lvert
                    \frac{F(w + h) - F(w)}{h} - f(w)
                \right\rvert
                \to 0
            \]
            as $h \to 0$.
    \end{enumerate}
\end{proof}
