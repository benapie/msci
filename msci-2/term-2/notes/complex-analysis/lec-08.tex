\lecture{8}{3/2}

\begin{theorem}[Morera]
    Let $U \subset \C$ be open and $f: U \to \C$.
    If
    \[
        \int_\gamma f(z) \,dz = 0
    \]
    for all closed contour $\gamma \subset U$, then $f$ is holomorphic.
\end{theorem}

\begin{proof}
    By the converse to the FTC, 
    we see that there exists a holomorphic function
    $F: U \to \C$
    such that $F'(z) = f(z)$.
    As holomorphic functions have infinitely many holomorphic derivatives,
    hence $f$ is holomorphic.
\end{proof}

\begin{example}
    \[
        \int_{\lvert z \rvert = 1} \frac{\sin{z}}{z^2} \,dz
        = \int_{\lvert z \rvert = 1} \frac{f(z)}{(z - 0)^2} \,dz
        = \frac{2\pi i}{1!} f'(0)
        = 2\pi i.
    \]
\end{example}

\begin{example}
    \begin{align*}
        I
        &= \int_{\lvert z \rvert = 3} \frac{\exp(z)}{z^2(z-1)} \,dz \\
        &= \int_{\lvert z \rvert = 3} \frac{-\exp(z)}{z} \,dz
            + \int_{\lvert z \rvert = 3} \frac{-\exp(z)}{z^2} \,dz
            + \int_{\lvert z \rvert = 3} \frac{\exp(z)}{z-1} \,dz \\
        &= -2\pi i e^0 + 2\pi i e^1 - 2\pi ie^0 \\
        &= 2\pi i(e - 2).
    \end{align*}
\end{example}

\section{Liouville's theorem}

\begin{definition}[Entire]
    A function $f: \C \to \C$ is \textbf{entire} if
    $f$ is holomorphic and defined on $\C$.
\end{definition}

\begin{theorem}[Liouville]
    Any bounded entire function is constant.
\end{theorem}

\begin{proof}
    Let $f$ be entire and bounded.
    So $\lvert f(z) \rvert \leq M$ for all $z \in \C$, $M \in \R$.
    Let $w \in \C$.
    For $\rho > \lvert w \rvert$ we have
    \begin{align*}
        \lvert f(w) - f(0) \rvert
        &=
        \frac{1}{2\pi} 
        \left\lvert
            \int_{\lvert z \rvert = \rho} \frac{f(z)}{z-w} \,dz - \int_{\lvert z \rvert = \rho} \frac{f(z)}{z} \,dz
        \right\rvert
        \tag{CIF} \\
        &= \frac{1}{2\pi}
        \left\lvert
            \int_{\lvert z \rvert = \rho} f(z)
            \left(
                \frac{1}{z - w} - \frac 1z
            \right)
            \,dz
        \right\rvert
        \\
        &= \frac{1}{2\pi} \int_{\lvert z \rvert = \rho} f(z) \frac{w}{(z - w)z} \,dz \\
        &\leq \frac{1}{2\pi} \lvert w \rvert 2 \pi \rho \sup_{\lvert z \rvert = \rho}
        \left(
            \frac{\lvert f(z) \rvert}{\lvert z \rvert \lvert z - w \rvert} 
        \right) \\
        &\leq \frac{1}{2\pi} 2 \pi \rho \lvert w \rvert \frac{M}{\rho(\rho - \lvert w \rvert)}  \to 0
    \end{align*}
    as $\rho \to \infty$;
    hence, $f(w) = f(0)$.
\end{proof}


\begin{example}
    Let $f: \C \to \C$ be a function defined by
    \[
        f(z) = \frac{1}{1 + \abs{z}^2}.
    \]
    We see that $\abs{f(z)} \leq 1$, 
    so $f$ is bounded but it is \emph{not} holomorphic by the Cauchy-Riemann equations.
\end{example}

\begin{theorem}[Fundamental theorem of algebra]
    Every non-constant polynomial with complex coefficients has a complex root.
\end{theorem}

\begin{proof}
    Let $P(z) = a_0 + a_1z + a_2z^2 + \ldots + a_dz^d$.
    Then
    \begin{align*}
        \abs{P(z)}
        &=    \abs{a_0 + a_1z + a_2z^2 + \ldots + a_dzd} \\
        &\geq \abs{a_d} \abs{z}^d - \abs{a_{d - 1}} \abs{z}^{d - 1} - \ldots - \abs{a_0} \\
        &\to  \infty.
    \end{align*}
    as $z \to \infty$.
    Therefore, there exists $R>0$ such that 
    $\abs{P(z)} > 1$
    for all 
    $\abs{z} > R$.
    We now assume that there does not exist
    $z_0 \in \C$
    such that
    $P(z_0) = 0$.
    We set
    \[
        f(z) = \frac{1}{P(z)}. 
    \]
    $f$ is holomorphic for all $z \in \C$ and it is entire.
    Now let
    \[
        L = \max_{\abs z \leq R} \lvert f(z) \rvert < \infty.
    \]
    Since $\abs{f(z)}$ is continuous and
    $\abs z \leq R$
    since 
    \[
        \abs{f(z)} = \frac{1}{\abs{P(z)}} \leq 1 
    \]
    for all $z \in \C$.
    We have $\abs{f(z)} \leq \max\{1,L\}$
    so by Liouville's theorem $f$ is constant and so $P$ is constant; contraction. 
\end{proof}

