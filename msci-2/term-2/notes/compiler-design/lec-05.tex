\chapter{Syntax analysis}
\section{Notation}
\lecture{5}{11/2}

We express the syntax of a language as a context-free grammar.
The alphabet is a set of \emph{tokens}.

Let $A \to \gamma$ be a production rule. 
Let $\alpha, \beta$ be strings.
Then
\begin{enumerate}
    \item $\alpha A \beta \implies \alpha \gamma \beta$
        means \emph{derives in one step};
    \item $\gamma_1 \overset\star\implies \gamma_2$
        means \emph{derives in zero or more steps}; and
    \item $\gamma_1 \overset{+}\implies \gamma_2$
        means \emph{derives in one or more steps}.
\end{enumerate}

\begin{example}
    Consider the following production rules for a context-free grammar.
    \begin{align*}
        E &\to E + T & T &\to T \times F & F &\to (E) \\
        E &\to T     & T &\to F          & F &\to x   \\
          &          &   &               & F &\to y.
    \end{align*}
    Consider the string $(x + y) \times x$.
    We derive this string by
    \begin{align*}
        E
        &\implies            T                \\
        &\implies            T \times F       \\
        &\implies            F \times F       \\
        &\implies            (E) \times F     \\
        &\implies            (E + T) \times F \\
        &\implies            (T + T) \times F \\
        &\overset{+}\implies (F + F) \times F \\
        &\overset{+}\implies (x + y) \times x.
    \end{align*}
\end{example}

The purpose of a parser is to identify whether a string of tokens
is in a language.
If it is, it must produce the derivation for the string.
Otherwise, it will report \emph{syntax errors}.

\begin{center}
    \begin{tikzpicture}
        \tikzstyle{rect} = [
            rectangle, 
            minimum width=2cm, 
            minimum height=1cm,
            text centered, 
            draw=black, 
        ]
        \node[rect] (lex) {Lexical analysis};
        \node[rect] (par) [right=4em of lex] {Parser};
        \node[rect] (res) [right=4em of par] {Rest of compilation};
        \node[rect] (sym) [below=2em of par] {Symbol table};
        \draw[<->] (lex) -- (sym);
        \draw[<->] (par) -- (sym);
        \draw[<->] (res) -- (sym);
        \draw[->]  (lex) -- (par);
        \draw[->]  (par) -- (res);
    \end{tikzpicture}
\end{center}

\section{Parse trees and abstract syntax trees}

A \textbf{parse tree} is a visual representation of a derivation.
In a parse tree, internal nodes are non-terminals 
and leaves are terminals.

\begin{example}
    Recall our derivation of $(x+y)\times x$.
    We have the following parse tree.
    \begin{center}
        \begin{forest}
            [$E$
                [$T$
                    [$T$
                        [$F$
                            [$($]
                            [$E$
                                [$E$
                                    [$E$[$T$[$F$[$x$]]]]
                                ]
                                [$+$]
                                [$T$[$F$[$y$]]]
                            ]
                            [$)$]
                        ]
                    ]
                    [$x$]
                    [$F$
                        [$x$]
                    ]
                ]
            ]
        \end{forest}
    \end{center}
\end{example}

Parse trees represent a derivation of a string
and are often very complex.
We may use \textbf{abstract syntax trees} (AST) as they 
are much simpler.
Every internal node represents an operation and leaves
represent operands.

\begin{example}
    Recall the string $(x+y)\times x$.
    We get the following abstract syntax tree.
    \begin{center}
        \begin{forest}
            [$\times$
                [$+$
                    [$x$]
                    [$y$]
                ]
                [$x$]
            ]
        \end{forest}
    \end{center}
\end{example}

\begin{example}
    Consider the string $b \times b - 4 \times a \times c$.
    The parse tree for this is \emph{chaos} but the AST is as
    follows.
    \begin{center}
        \begin{forest}
            [$-$
                [$\times$
                    [$b$]
                    [$b$]
                ]
                [$\times$
                    [$\times$
                        [$4$]
                        [$a$]
                    ]
                    [$c$]
                ]
            ]
        \end{forest}
    \end{center}
    An annotated AST has a type and location in memory annotated
    on each node.
\end{example}
