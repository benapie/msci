\lecture{6}{18/2}

\begin{definition}[Ambiguous grammar]
    Let $G$ be a formal grammar.
    Then $G$ is \textbf{ambiguous}
    if there is more than one leftmost derivations
    for the same string (or more than one parse tree).
\end{definition}

\begin{example}
    \hfill
    \begin{enumerate}
        \item
            Consider a grammar with the following production rule:
            \[
                \text{string} \to
                \text{string} + \text{string} 
                \mid \text{string} - \text{string} 
                \mid 0 \mid 1 \mid 2 \mid 3 \mid 4 \mid 5 \mid 6 \mid 7 \mid 8 \mid 9.
            \]
            Now consider the string $9-5+2$, it has the following two parse trees:
            \begin{center}
                \begin{forest}
                    [string
                        [string
                            [string
                                [9]
                            ]
                            [$-$]
                            [string
                                [5]
                            ]
                        ]
                        [$+$]
                        [string
                            [2]
                        ]
                    ]
                \end{forest}
                \hspace{1em}
                \begin{forest}
                    [string
                        [string
                            [9]
                        ]
                        [$-$]
                        [string
                            [string
                                [5]
                            ]
                            [$+$]
                            [string
                                [2]
                            ]
                        ]
                    ]
                \end{forest}
            \end{center}
            Therefore, it is an ambiguous grammar.

        \item 
            Consider the grammar with the following production rules:
            \[
                E \to
                E + E \mid
                E * E \mid
                (E) \mid
                \bm{\text{id}}.
            \]
            Then, for the string $\bm{\text{id}} + \bm{\text{id}} * \bm{\text{id}}$
            we have the following parse trees.
            \begin{center}
                \begin{forest}
                    [$E$
                        [$E$
                            [\textbf{id}]
                        ]
                        [$+$]
                        [$E$
                            [$E$
                                [\textbf{id}]
                            ]
                            [$*$]
                            [$E$
                                [\textbf{id}]
                            ]
                        ]
                    ]
                \end{forest}
                \hspace{1em}
                \begin{forest}
                    [$E$
                        [$E$
                            [$E$
                                [\textbf{id}]
                            ]
                            [$+$]
                            [$E$
                                [\textbf{id}]
                            ]
                        ]
                        [$*$]
                        [$E$
                            [\textbf{id}]
                        ]
                    ]
                \end{forest}
            \end{center}
            Therefore, it is an ambiguous grammar.
    \end{enumerate}
\end{example}

There are two solutions to resolve ambiguity:
\begin{enumerate}
    \item use \emph{disambiguating rules} that \emph{throw away} undesired
        parse trees; or

    \item construct an equivalent unambiguous grammar.
\end{enumerate}
We have no general algorithm, it is case dependent.

\begin{example}
    \hfill
    \begin{enumerate}
        \item 
            The simplest example is the following ambiguous grammar for
            the trivial language:
            \[
                A \to A \mid \varepsilon.
            \]
            The empty string has a derivation of every length,
            depending on hwo many times the rule $A \to A$ is used.
            This language also has the unambiguous grammar, 
            consiting of a single produciton rule:
            \[
                A \to \varepsilon.
            \]

        \item
            Consider the unambiguous grammar with production rules
            \[
                A \to aA \mid \varepsilon.
            \]
            We can construct the ambiguous grammar which describes
            the same formal language with production rules
            \[
                A \to aA \mid Aa \mid \varepsilon.
            \]

        \item
            The grammar with production rules
            \[
                A \to A - A \mid A + A \mid a
            \]
            is clearly ambiguous,
            an equivalent unambiguous grammar has production rules
            \[
                A \to a + A \mid a - A \mid a.
            \]
    \end{enumerate}
\end{example}

\begin{remark}
    It is important to note that it is not sufficient for a string
    to have multiple derivations for it to be ambiguous;
    only \emph{leftmost derivations}.
    For example, consider the simple grammar
    \begin{align*}
        S &\to A + A \\
        A &\to 0 \mid 1.
    \end{align*}
    The string $0 + 0$ has the derivations
    \[
        S \implies A + A \implies 0 + A \implies 0 + 0
    \]
    and
    \[
        S \implies A + A \implies A + 0 \implies 0 + 0
    \]
    but only the former derivation is a leftmost one.
\end{remark}

%todo
