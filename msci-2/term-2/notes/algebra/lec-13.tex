\lecture{13}{17/3}

\begin{example}
    Recall how $\Z$ acts on $\R$ as a translation.
    Let $x \in \R$.
    Then
    \begin{align*}
        \Orb(x)  &= \{n + x : n \in \Z\} = \Z + x, \\
        \Stab(x) &= \{n \in \Z: n + x = x\} = \{0\}.
    \end{align*}
\end{example}

\begin{example}
    Recall the other way that showed that $\Z$ could act on $\R$ ($n * x = (-1)^n x$).
    Then for $x \in \R$
    \begin{align*}
        \Orb(x)  &= \{(-1)^n x: n \in \Z\} = \{\pm x\}, \\
        \Stab(x) &= \{n \in \Z: (-1)^nx = x\} =
        \begin{cases}
            2\Z & \text{if}\; x \neq 0, \\
            \Z  & \text{otherwise}.
        \end{cases}
    \end{align*}
\end{example}

%\begin{proposition}[]
%    Let $G$ be a group acting on a set $X$ and $x,y \in X$.
%    Then the following hold.
%    \begin{enumerate}
%        \item $\Orb(x)$ is non-empty, that is, $\Orb(x) \neq \varnothing$.
%
%        \item For any two orbits $\Orb(x)$, $\Orb(y)$, we have either
%            $\Orb(x) \cap \Orb(y) = \varnothing$
%            or
%            $\Orb(x) = \Orb(y)$.
%
%        \item $X$ is a disjoint union of the orbits.
%    \end{enumerate}
%\end{proposition}

\section{Cosets and conjugacy classes as orbits}

Any group $G$ acts on itself:
for $g \in G$ we have $x \mapsto gx$ for $x \in G$.
Here
\begin{align*}
    \Orb(x)  &= G, \\
    \Stab(x) &= 1.
\end{align*}
Let $H \subset G$ be a subset of $G$ and let it act on $G$ as above.
For $x \in G$ we have
\[
    \Orb(x) = \{hx: h \in H\} = Hx,
\]
a right coset of $H$.

Consider the action $x \mapsto gxg^{-1}$ for $g,x \in G$ of $G$ on itself
(easy exercise to check that this is an action).
We call this action \textbf{conjugation}.
Under tis action, the orbit
\[
    \Orb(x) = \{gxg^{-1}: g \in G\}
\]
is called a \textbf{conjugacy class} (of $x$).
The stabiliser
\[
    \Stab(x) = \{g \in G: gxg^{-1} = x\}
\]
is called the \textbf{centraliser} (of $x$), and is usually denoted $C_G(x)$.

\begin{example}
    Find the conjugacy classes in $D_5$.
\end{example}

\begin{solution}
    \hfill
    \begin{enumerate}
        \item 
            $1 \in D_5$ is \emph{always} fixed by any conjugations, 
            hence $\Orb(1) = \{1\}$.

        \item 
            Now take $r \in D_5$, it is fixed by any power of $r$
            (that is, $r^i r r^{-i} = r$)
            and
            \[
                (r^is) r (r^is)^{-1} = r^{-1} = r^4
            \]
            so
            \[
                \Orb(r) = \{r, r^4\}.
            \]

        \item 
            Now lets consider $r^2 \in D_5$. 
            Again, conjugation by $r^i$ on $r^2$ fixes $r^2$, but
            \[
                (r^is) r^2 (r^is)^{-1} = r^{-2} = r^3
            \]
            so 
            \[
                \Orb(r^2) = \{r^2, r^3\}.
            \]

        \item
            Now finally we will consider $s \in D_5$.
            We have
            \begin{align*}
                (r^i) s (r^{-i})     &= r^{2i}s, \\
                (r^is) s (r^is)^{-1} &= r^{2i}s.
            \end{align*}
            Therefore,
            \[
                \Orb(s) = \{s, r^2s, r^4s, rs, r^3s\}
            \]
            and we have exhausted all elements in $D_5$, 
            hence our conjugacy classes are
            \[
                \{1\}, \quad \{r, r^4\}, \quad \{r^2, r^3\}, \quad \{s, rs, r^2s, r^3s, r^4s\}.
            \]
    \end{enumerate}
\end{solution}

\chapter{The Orbit-Stabiliser theorem}

\begin{theorem}[]
    Let $G$ be a group acting on a set $X$, and let $x \in X$.
    Then there is a bijection
    \[
        \beta: \Orb(x) \to \{g\Stab(x): g\in G\}, \qquad \beta(gx) = g\Stab(x).
    \]
    In particular, if $G$ is finite then
    \[
        \abs{\Orb(x)} = \frac{\abs{G}}{\abs{\Stab(x)}}.
    \]
\end{theorem}

\begin{example}
    Recall the conjugacy classes of $D_5$.
    We saw that
    \[
        \Orb(r) = \{r, r^4\}.
    \]
    Now
    \begin{align*}
        \Stab(r) 
        &= \{r^i s^j \in D_5: (r^i s^j) r (r^i s^j)^{-1} = r\} \\
        &= \langle r \rangle \cup \{r^is \in D_5: r^is r sr^{-i} = r^{-1} = r\} \\
        &= \langle r \rangle
    \end{align*}
    so $\abs{\Orb(r)} = 2$ and $\abs{\Stab(x)} = 5$, 
    which agrees wth the Orbit-Stabiliser theorem.
    Moreover, $\Orb(s)$ has five elements, 
    so $\Stab(x)$ must have $2$ elements.
    Indeed
    \begin{align*}
        \Stab(s)
        &= \{r^i : r^isr^{-i} = s\} \cup \{r^is: r^iss(r^is)^{-1}=s\} \\
        &= \{r^i : r^{2i} = 1\} \cup \{r^is : r^{2i} = 1\} \\
        &= \{1, s\}.
    \end{align*}
\end{example}

\section{Cauchy's theorem}

\begin{theorem}[Cauchy]
    Let $G$ be a finite group and $p$ be a prime such that $p$ divides
    the order of $G$.
    Then $G$ has a cyclic subgroup of order $p$
    (equivalently, $G$ has an element of order $p$).
\end{theorem}

\begin{example}
    $D_{20}$ has a subgroup of order $5$ as $\abs{D_{20}} = 40 = 8 \cdot 5$.
\end{example}
