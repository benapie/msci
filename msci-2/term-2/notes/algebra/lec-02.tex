\chapter{Dihedral groups}
\lecture{2}{16/1}

A \textbf{dihedral group} is the group of symmetries of a regular polygon,
which includes \emph{rotations} and \emph{reflections}.

We have the dihedral group $D_3$, or the symmetry group of the regular triangle,
defined as
\[
    D_3 = \{ 1, r, r^2, s, rs, r^2s \}.
\]
To relate this to transformations on a triangle: $r$ is a rotation of
$\frac{2\pi}3$ and $s$ is a reflection (in any of the lines of symmetry).

\begin{remark}
    \begin{enumerate}
        \item 
            $D_3$ is a non-abelian group.

        \item
            You may object that not \emph{all} symmetries are in the group;
            for example, the product $r^2(r^2s)$.
            However,
            \[
                r^2(r^2s) = r^3rs = 1rs = rs \in D_3.
            \]

        \item To describe $D_3$ completely, we need only $r$ and $s$ and
            three fundamental relations from which everything else follows:
            \[
                D_3 = \langle r, s : r^3 = 1, s^2 = 1, srs = r^2 \rangle.
            \]
    \end{enumerate}
\end{remark}

We can do a similar analysis for a square,
and the group of symmetries here are called $D_4$.
Similarly, for a regular $n$-gon ($n \geq 3$)
we get the dihedral group $D_n$.
We define this as
\[
    D_n = \langle r, s: r^n = 1, s^2 = 1, srs = r^{-1} \rangle.
\]
You can see the main difference here is that we have $n$ rotations.
This algebraic definition of $D_n$ makes sense for $n \in \{1, 2\}$;
however, $D_n$ clearly is not the symmetry group of an $n$-gon.
