\lecture{12}{27/2}

\begin{example}
    How many $3$-cycles are there in $S_{10}$?
\end{example}

\begin{solution}
    Consider the cycle $(a\;b\;c) \in S_{10}$.
    We have 10 choices for $a$, 9 choices for $b$, and 8 choices for $c$.
    The number of $3$-cycles is
    \[
        \frac{10\cdot9\cdot8}{3}
    \]
    as we can shift each $3$-cycle along twice, so we divide by $3$ to get rid of duplicates.
\end{solution}

\chapter{Linear groups}

\begin{center}
    \emph{Omitted due to strikes.}
\end{center}

\chapter{Group actions}

\begin{example}
    The group $S_n$ acts on the set $X = \{1,2,\ldots,n\}$; that is,
    if $x \in X$, then an element $\sigma \in S_n$ sends $x$ to $\sigma(x)$.
\end{example}

\begin{example}
    $D_3$ acts on a regular triangle which we can encode as a set of ordered triples
    \[
        \{(1,2,3), (1,3,2), \ldots\}
    \]
    where $1$, $2$, and $3$ denote the vertices of the triangle.
    This set is then all the different positions that the triangle can take.
    For example, we saw that $r \in D_3$ acts on $(1,2,3)$ by sending it to $(3,2,1)$.
\end{example}

\begin{remark}
    One important notion of group actions is that if we first act by an element
    $h \in G$ and then by an element $g \in G$, that is the same as acting by the element
    $gh \in G$.
\end{remark}

\begin{definition}[Group action]
    Let $G$ be a group and $X$ a set.
    An \textbf{action} of $G$ on $X$ is a function
    \[
        G \times X \to X, \qquad (g,x) \mapsto g * x
    \]
    such that for all $g,h \in G$ and $x \in X$:
    \begin{enumerate}
        \item $g * (h * x) = (gh) * x$; and
        \item $1 * x = x$.
    \end{enumerate}
\end{definition}

We usually write $g * x$ simply as $gx$.

\begin{remark}
    We can think of a group actions as 
    \emph{multiplying elements of the group onto points of a space}.
\end{remark}

\begin{example}
    $\Z$ can act on $\R$ by translation. 
    That is,
    for $n \in \Z$ and $x \in \R$ we can define
    \[
        n * x = n + x.
    \]
    We have
    \begin{enumerate}
        \item 
            \[
                m * (n * x) = m + (n + x) = (m + n) + x = (m * n) * x; \;\text{and}
            \]

        \item
            \[
                0 * x = 0 + x = x.
            \]
    \end{enumerate}
\end{example}

\begin{example}
    $\Z$ can also act on $\R$ in another way:
    \[
        n * x = (-1)^n x.
    \]
    We check that this is indeed an action:
    \begin{enumerate}
        \item
            \[
                m * (n * x) = m * (-1)^n x = (-1)^m (-1)^n * x = (-1)^{m + n} x = (m + n) * x;
                \;\text{and}
            \]
            
        \item 
            \[
                0 * x = (-1)^0 x = x.
            \]
    \end{enumerate}
\end{example}

\section{Orbits and stabilisers}

\begin{definition}[]
    Let $G$ be a group acting on a set $X$.
    For any $x \in X$, we define the \textbf{orbit of $x$} as
    \[
        \Orb(x) = \{gx : g \in G\},
    \]
    and the \textbf{stabiliser of $x$} as
    \[
        \Stab(x) = \{g \in G: gx = x\}.
    \]
\end{definition}

\begin{remark}
    $\Stab(x)$ is a subgroup of $G$, but in general there is no reason for $\Orb(x)$
    to be a group (it should be thought of as a space).
\end{remark}
