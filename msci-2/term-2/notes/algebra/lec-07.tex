\lecture{7}{4/2}

\begin{example}
    Find all subgroups of $S_3$.
\end{example}

\begin{solution}
    We have $\lvert S_3 \rvert = 6$; hence,
    if $H \subset S_3$ is a subgroup then
    $\lvert H \rvert \in \{1,2,3,6\}$.
    \begin{description}
        \item[$\lvert H \rvert = 1$]
            This must be the trivial subgroup: $\{1\} \subset S_3$.

        \item[$\lvert H \rvert = 2$]
            We must have $H = \{1,x\}$ where $x\in S_3$ has order $2$.
            Thus $x$ must have a representation as a product of disjoint
            transpositions; however, as we are in $S_3$ we cannot have
            a product of more than $1$ transpositions. Hence
            \[
                x \in \{(1\;2), (1\;3), (2\;3)\}.
            \]
        
        \item[$\lvert H \rvert = 3$]
            We must have $H = \{1, x, x^2\}$ where $x \in S_3$ has order $3$.
            Here the only two distinct $x$ are $(1\;2\;3)$ and $(1\;3\;2)$,
            but $(1\;2\;3)^2 = (1\;3\;2)$ so we have one subgroup.
            
        \item[$\lvert H \rvert = 3$] $H = S_3$.
    \end{description}
    Hence we get the following subgroup structure for $S_3$.
    \begin{center}
        \begin{tikzpicture}
            % NODES
            %% TOP NODE
            \node (S3)     {$S_3$};
            %% MIDDLE ROW NODES
            \node (12) [below left = 0.5cm and 1.5cm of S3]  
                {$\langle (1\;2) \rangle$};
            \node (13) [below left = 0.5cm and 0cm of S3]
                {$\langle (1\;3) \rangle$};
            \node (23) [below right = 0.5cm and 0cm of S3]
                {$\langle (2\;3) \rangle$};
            \node (123) [below right = 0.5cm and 1.5cm of S3] 
                {$\langle (1\;2\;3) \rangle$};
            %% BOTTOM NODE
            \node (1) [below = 1.5cm of S3] {$\{1\}$};
            % EDGES
            %% TOP TO MIDDLE
            \draw (S3) -- (12);
            \draw (S3) -- (13);
            \draw (S3) -- (23);
            \draw (S3) -- (123);
            %% MIDDLE TO BOTTOM
            \draw (12) -- (1);
            \draw (13) -- (1);
            \draw (23) -- (1);
            \draw (123) -- (1);
        \end{tikzpicture}
    \end{center}
\end{solution}

\section{An application to number theory}

An application of Lagrange's theorem (and some ring theory) is 
the \emph{Fermat-Euler} theorem.
We must define $\varphi(n)$ first, however.

\begin{definition}[Euler's totient function]
    \textbf{Euler's totient function} $\varphi(n): \N \to \N$
    is the number of positive integers up to $n$ that are coprime to $n$.
\end{definition}

\begin{theorem}[Fermat-Euler]
    Let $a,n \in \N$ be coprime, then
    \[
        a^{\varphi(n)} \equiv 1 \pmod n.
    \]
\end{theorem}

\begin{proof}
    We have 
    \[
        a^{\varphi(n)} \equiv 1 \pmod n
        \iff a^{\varphi(n)} - 1 \in n\Z
        \iff \overline{a^{\varphi(n)}} = \overline 1 \qquad \text{in}\; 
        (\Z/n)^\times.
    \]
    From last term we have
    \[
        \left\lvert (\Z/n)^\times \right\rvert = \varphi(n)
    \]
    By Lagrange's theorem we have that 
    $\lvert \overline a \rvert$
    divides
    $\varphi(n)$.
    Thus $\ord(\overline a)m = \varphi(n)$ for some $m \in \Z$.
    Therefore
    \[
        \overline a^{\varphi(n)}
        = ((\overline a)^{\ord(\overline a)})^m
        = (\overline 1)^m
        = \overline 1.
    \]
\end{proof}

\chapter{Homomorphisms and isomorphisms}

This content will be very familiar from the definitions for rings.

\begin{definition}[Homomorphism]
    Let $G$ and $H$ be groups.
    A \textbf{homomorphism} $\varphi: G \to H$ is a function
    such that
    \[
        \varphi(xy) = \varphi(x)\varphi(y)
    \]
    for all $x, y\in G$.
\end{definition}

If a homomorphism $\varphi: G \to H$ is bijective, it is an \textbf{isomorphism}.
We denote $G \cong H$.

\begin{remark}
    A point on notation: in the definition for a homomorphism we are using the
    binary operation in $G$ for $xy$ and the binary operation in $H$ for
    $\varphi(x)\varphi(y)$.
    These are not necessarily the same.
\end{remark}

\begin{example}
    Any group of order $2$ is isomorphic to $\Z/2$.
    To show that, we consider the group $\{e, x\}$ where $e$
    is the identity and $x \neq e$.
    If we define the function
    $e \mapsto \overline 0$
    and
    $x \mapsto \overline 1$
    we see that it is indeed an isomorphism, so $\{e, x\} \cong \Z/2$.
\end{example}

\begin{example}
    Similarly to the last example, any group of order $3$ is
    isomorphic to $\Z/3$.
    To see this, take the group $\{e,x,y\}$ with $e$ identity.
    It is clear to see that $xy = e$ as if $xy = x$ then $y = 1$ and
    if $xy = y$ then $x = 1$.
    Furthermore, we have that $x^2 = y$ as if $x^2 = 1$ then
    $x^2y = y$, but $x^2y = x(xy) = x = y$; a contradiction.
    Similarly $y^2 = x$.
    We define the map $\varphi:\{e,x,y\} \to \Z/3$ by
    \[
        e \mapsto \overline 0, \qquad x \mapsto \overline 1,
        \qquad y \mapsto \overline 2.
    \]
    It is trivial to see that this is an isomorphism.
\end{example}

\begin{example}
    We can relate $D_n$ to $\Z/n$ by defining the homomorphism 
    \[
        \varphi: D_n \to \Z/2, \qquad \varphi(r^is^j) = \overline j \pmod 2.
    \]
    To confirm this, we just need to check that
    \[
        \varphi(xy) = \varphi(x) + \varphi(y)
    \]
    for all $x,y \in D_n$ 
    (don't be confused by the notation, we use the additive notation for
    $\Z/2$ but the multiplicative notation for $D_n$ as usual):
    \begin{align*}
        \varphi(r^as^b\cdot r^cs^d)
        &= \varphi(r^as^br^c(s^bs^b)s^d)
        = \varphi(r^{a-c}s^{b+d})
        = \overline{b + d} \\
        \varphi(r^as^b)\varphi(r^cs^d)
        &= \overline b + \overline d
        = \overline{b + d}.
    \end{align*}
\end{example}

\begin{example}
    Prove that the map
    \[
        \varphi: \Z/6 \to S_3, \qquad \varphi(\overline a) = (1\;2\;3)^a
    \]
    is a homomorphism.
\end{example}

\begin{solution}
    First, we must show that $\varphi$ is well-defined.
    That is,
    if 
    $\overline a = \overline b$
    then
    $\varphi(\overline a) = \varphi(\overline b)$.
    If 
    $\overline a = \overline b$ 
    then
    $a - b = 6m$
    for some $m \in \Z$.
    So
    \[
        \phi(\overline a)
        = (1\;2\;3)^{\overline a}
        = (1\;2\;3)^{b + 6m}
        = (1\;2\;3)^b((1\;2\;3)^{3})^{2m}
        = (1\;2\;3)^b
        = \varphi(\overline b)
    \]
    as a $3$-cycle has order $3$.
    So we have shown that $\varphi$ is well-defined,
    now we must show that it is a homomorphism:
    \[
        \varphi(\overline a + \overline b)
        = (1\;2\;3)^{a+b}
        = (1\;2\;3)^a(1\;2\;3)^b
        = \varphi(\overline a)\varphi(\overline b).
    \]
\end{solution}

We define the \textbf{kernel} and \textbf{image} of a homomorphism
identically to how we did last term in ring theory.
