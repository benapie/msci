\chapter{Symmetric group}
\lecture{4}{23/1}

\begin{definition}[Permutation]
    A \textbf{permutation} of a set $S = \{1,2,\ldots,n\}$ is a bijection 
    $\sigma: S \to S$, often written
    \[
        \begin{pmatrix}
            1 & 2 & \ldots & n \\
            \sigma(1) & \sigma(2) & \ldots & \sigma(n) \\
        \end{pmatrix}
        .
    \]
\end{definition}

\begin{definition}[Cycle]
    A $k$-cycle is a permutation $\sigma$ such that
    \[
        \sigma(a_1) = a_2, \quad \sigma(a_2) = a_3, \quad \ldots, 
        \quad \sigma(a_k) = a_1
    \]
    and $\sigma(a) = a$ for all $a \neq \{a_1, \ldots, a_k\}$.
    We use the shorthand
    \[
        \begin{pmatrix}
            a_1 & a_2 & \ldots & a_k \\
            a_2 & a_3 & \ldots & a_1 \\
        \end{pmatrix}
        =
        \begin{pmatrix}
            a_1 & a_2 & \ldots & a_k
        \end{pmatrix}
        .
    \]
\end{definition}

\begin{example}
    \hfill
    \begin{enumerate}
        \item A 3-cycle:
            \[
                \begin{pmatrix}
                    1 & 2 & 3 & 4 & 5 & 6 \\
                    1 & 5 & 3 & 4 & 6 & 2 \\
                \end{pmatrix}
                =
                \begin{pmatrix}
                    2 & 5 & 6
                \end{pmatrix}
                = 
                \begin{pmatrix}
                    5 & 6 & 2
                \end{pmatrix}
                =
                \begin{pmatrix}
                    6 & 2 & 5
                \end{pmatrix}
                .
            \]

        \item A 2-cycle:
            \[
                \begin{pmatrix}
                    1 & 2 & 3 & 4 \\
                    3 & 2 & 1 & 4 \\
                \end{pmatrix}
                =(1\;3);
            \]
            we call $2$-cycles \textbf{transpositions}.
    \end{enumerate}
\end{example}

\begin{definition}[Disjoint cycle]
    Two cycles are \textbf{disjoint} if there members do not intersect.
\end{definition}

\begin{example}
    $(1\;2\;3)$ and $(4\;5)$ are disjoint but
    $(1\;2\;3)$ and $(2\;4)$ are not.
\end{example}

\begin{proposition}[]
    For any $n \in \N$, the set of permutations 
    \[
        \sigma: \{1,2,\ldots,n\} \to \{1,2,\ldots,n\}
    \]
    is a group under composition, called the symmetric group $S_n$.
\end{proposition}

\begin{proof}
    Let $\sigma$ and $\tau$ be two bijections.
    Then $\sigma \circ \tau$ is also a bijection.
    Composition of functions is always associative.
    The identity bijection is $\operatorname{Id}: a \to a$.
    Finally, every bijection $\sigma$ has an inverse $\sigma^{-1}$
    such that $\sigma \circ \sigma^{-1} = 1$.
\end{proof}

\begin{remark}
    $\ord(S_n) = n!$. This is because we have $n$ choices for the
    position of the first value, then $n-1$ for the next, and so on.
\end{remark}

\begin{example}
    $\sigma = (1\;2\;3) \in S_4$, and $\tau = (2\;4) \in S_4$.
    Then
    \[
        \sigma \circ \tau = (2 \; 4 \; 3 \; 1).
    \]
\end{example}

\begin{example}
    Consider $S_{10}$ with
    \[
        \sigma =
        \begin{pmatrix}
            1 & 2 & 3 & 4 & 5 & 6 & 7 & 8 & 9 & 10 \\
            5 & 3 & 2 & 1 & 4 & 8 & 9 & 7 & 6 & 10 \\
        \end{pmatrix}
        .
    \]
    Then we have
    \[
        \sigma = (1\;5\;4)\circ(2\;3)\circ(6\;8\;7\;9)\circ(10).
    \]
\end{example}

\begin{example}
    Let $\sigma = (1\;2)$ and $\tau = (1\;3)$ in $S_3$.
    Then
    \[
        (1\;2)(1\;3) = (1\;3\;2).
    \]
\end{example}

\begin{proposition}[]
    \hfill
    \begin{enumerate}
        \item Disjoint cycles commute with each other.
        \item Every permutation is a product of disjoint cycles,
            and this is unique up to the order of the cycles
            and the different ways we can write the cycle.
    \end{enumerate}
\end{proposition}
