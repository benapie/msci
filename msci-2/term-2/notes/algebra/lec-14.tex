\chapter{Finite abelian cnd cyclic groups}
\lecture{14}{19/3}

\begin{theorem}[]
    Let $G$ be a finite abelian group.
    Then $G$ is isomorphic to
    \[
        \Z/a_1 \times \Z/a_2 \times \ldots \times \Z/a_t
    \]
    for some $t$, $a_i \in \N$, $a_1 \geq 2$ such that
    \[
        a_1 \mid a_2 \mid \ldots \mid a_t.
    \]
    Moreover, the integers $a_i$ are uniquely determined by $G$.
\end{theorem}

\begin{proof}
    Omitted.
\end{proof}

\begin{theorem}[CRT]
    Suppose that $m, n \in \N$ are coprime.
    Then 
    \[
        \Z/mn \cong \Z/m \times \Z/n.
    \]
\end{theorem}

\begin{proof}
    Omitted.
\end{proof}

\begin{example}
    \begin{enumerate}
        \item 
            $\Z/6 \cong \Z/2 \times \Z/3$.

        \item
            $\Z/4 \not\cong \Z/2 \times \Z/4$.

        \item 
            Let $G = \Z/12$.
            We know that
            \[
                \Z/12 \cong \Z/a_1 \times \ldots \times \Z/a_t
            \]
            for some $t \in \N$ and $a_i \in \Z$.
            Moreover, we have
            \[
                \Z/12 \cong \Z/4 \times \Z/3,
            \]
            but $4 \not\mid 3$ and $3 \not\mid 4$ so $3$ and $4$ cannot take our values of $a_i$
            stated above.
            We see that $\Z/12 \cong \Z/a_1$ where $a_i = 12$, and is unique. 
            Hence there are no other decompositions of $\Z/12$.
            For example, if $a_1 = 2$ and $a_2 = 6$ then we know
            \[
                \Z/12 \not\cong \Z/2 \times \Z/6.
            \]
    \end{enumerate}
\end{example}

\begin{example}
    Find (up to isomorphisms) all the abelian groups of order $16$.
\end{example}

\begin{solution}
    By the first theorem presented in the chapter, we need to find the groups of the form
    \[
        \Z/2^{a_1} \times \Z/2^{a_2} \times \ldots \times \Z/2^{a_n}
    \]
    with $a_1 \geq 1$ such that $a_1 + \ldots + a_n = 4$.
    We have five possibilities:
    \begin{enumerate}
        \item $1 + 1 + 1 + 1 = 4$;
        \item $1 + 1 + 2 = 4$;
        \item $1 + 3 = 4$;
        \item $2 + 2 = 4$; and
        \item $4 = 4$.
    \end{enumerate}
    Hence are possible abelian groups are
    \begin{enumerate}
        \item $\Z/2 \times \Z/2 \times \Z/2 \times \Z/2$;
        \item $\Z/2 \times \Z/2 \times \Z/4$;
        \item $\Z/2 \times \Z/8$;
        \item $\Z/4 \times \Z/4$; and
        \item $\Z/16$.
    \end{enumerate}
\end{solution}

\section{Cyclic groups}

\begin{theorem}[]
    Let 
    $G$ be a finite cyclic group, 
    $x \in G$, 
    and $a \in \Z$ with $a \neq 0$.
    Then the following hold.
    \begin{enumerate}
        \item 
            If $n = \ord(x)$, then $\ord(x^a) = \frac{n}{\gcd(n,a)}$.

        \item 
            $\langle x \rangle = \langle x^a \rangle$ if and only if $\gcd(n,a) = 1$.
            Thus the number of generators of $\langle x \rangle$ is $\varphi(n)$
            (Euler's totient function).
    \end{enumerate}
\end{theorem}

\begin{proof}
    Omitted.
\end{proof}

\begin{example}
    We have $\Z/20 = \langle \overline 1 \rangle$ 
    and an element $\overline a = a \cdot \overline 1$ 
    generates the whole group if and only if
    \[
        \ord(\overline a) = \frac{20}{\gcd(20,a)} = 20 \qquad \iff \qquad \gcd(20,a) = 1.
    \]
    Thus $\Z/20$ has $\varphi(20) = \varphi(4) \varphi(5) = 2\cdot 4 = 8$ generators, namely
    \[
        \overline 1, \overline 3, \overline 7, \overline 9, 
            \overline{11}, \overline{13}, \overline{17}, \overline{19}.
    \]
\end{example}

\begin{theorem}[]
    Let $H = \langle x \rangle$ be a fintie cyclic group of order $n$.
    Then every subgroup of $H$ is cyclc 
    and for each $a \in \N$ dividing $n$ 
    there is a unique subgroup of order $a$,
    namely $\langle x^{\sfrac na} \rangle$.
\end{theorem}

\begin{example}
    Find all the subgroups of $\Z/12$.
\end{example}

\begin{solution}
    We know that every subgroup is cyclic
    and that for any positive divisor $d$ of $12$, 
    there is a unique subgroup of order $d$ generated by $\overline{12/d}$.
    The possible divisors are
    \[
        1,2,3,4,6,12.
    \]
    We thus have the corresponding six subgroups:
    \begin{align*}
        \langle \overline{12} \rangle    
            &= \{\overline 0\} \\
        \langle \overline{\sfrac{12}{2}} \rangle  
            &= \{\overline 0, \overline 6\} \\
        \langle \overline{\sfrac{12}{3}} \rangle  
            &= \{\overline 4, \overline 8, \overline 0\} \\
        \langle \overline{\sfrac{12}{4}} \rangle  
            &= \{\overline 3, \overline 6, \overline 9, \overline 0\} \\
        \langle \overline{\sfrac{12}{6}} \rangle  
            &= \{\overline 2, \overline 4, \overline 6, \overline 8, \overline{10}, \overline 0\} \\
        \langle \overline{\sfrac{12}{12}} \rangle 
            &= \{\overline 1\} = \Z/12.
    \end{align*}
\end{solution}

