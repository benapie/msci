\lecture{3}{21/1}

\begin{lemma}[]
    In $D_n$, any
    \[
        r^{a_1}s^{b_1}r^{a_2}s^{b_2}\ldots r^{a_m}s^{b_m}
    \]
    can be written as $r^as^b$ for
    $0 \leq a \leq n - 1$ and $0 \leq b \leq 1$.
\end{lemma}

\begin{proof}
    We use induction on the length $m$.
    We consider $m = 1$, the base case, then we have
    \[
        r^n = 1, \qquad s^2 = 1
    \]
    so it is true.
    Now suppose the lemma is true for some $m \geq 1$.
    We want to prove it is true $m + 1$.
    Consider an expression of length $m + 1$
    \[
        x = r^{a_1}s^{b_1}\ldots r^{a_m}s^{b_m}r^{a_{m+1}}b^{b_{m+1}}.
    \]
    Using $s^2 = 1$, we can reduce to the cases $b_i = 0$ or $b_i = 1$
    for all $i = 1, 2, \ldots, m+1$.
    Cases:
    \begin{enumerate}
        \item 
            if $b_{m+1} = 0$, we can write
            (using $srs = r^{-1} \implies sr^is = r^{-i}$)
            $x = r^{a_1}s^{b_1}\ldots r^{a_m+a_{m+1}}$
            if $b_m = 0$ or
            $x = r^{a_1}s^{b_1}\ldots r^{a_m+a_{m+1}}$
            if $b_m = 1$; and

        \item
            if $b_{m + 1} = 1$, we can write
            $x = r^{a_1}s^{b_1}\ldots r^{a_m+a_{m+1}}s$
            if $b_m = 0$ and
            $x = r^{a_1}s^{b_1}\ldots r^{a_m-a_{m+1}}$
            if $b_m = 1$.
    \end{enumerate}
    All of these expressions have length $m$,
    so by induction $x$ can be written in the required form.
\end{proof}

\chapter{Generators and cyclic groups}

\begin{definition}[]
    A set $S$ of elements of a group $G$ is said to be a set of
    generators if any element in $G$ can be written as a product
    of elements in $S$ (possibly together with inverses).
    We then write $G = \langle S \rangle$.
\end{definition}

\begin{example}
    \[
        D_n = \langle r, s \rangle.
    \]
\end{example}

\begin{definition}[Cyclic]
    A group which can be generator by a single element is called \textbf{cyclic}.
\end{definition}

\begin{example}
    \hfill
    \begin{enumerate}
        \item 
            $\Z/n = \langle \overline 1 \rangle$.

        \item
            $\Z = \langle 1 \rangle$.
    \end{enumerate}
\end{example}

\section{Orders of groups of elements}

The word \emph{order} can mean different properties depending on whether
you are referring to a \emph{group} or an \emph{element}.

\begin{definition}[Order]
    The \textbf{order} of a finite group $G$ is the number of elements,
    written $\lvert G \rvert$ or $\#G$.
\end{definition}

\begin{definition}[Order of an element]
    The \textbf{order} of an element $g$ of a group $G$, 
    written $\ord(g)$,
    is the smallest integer $n \geq 1$ such that $g^n = 1$.
    If no such $n$ exists, $\ord(g) = \infty$.
\end{definition}

\begin{remark}
    If $G$ is finite, then any element $g \in G$ has finite order as
    if $\{1, g, g^2, \ldots\}$ were all distinct then $G$ would be infinite;
    therefore, $g^i = g^j$ for some $0 \leq i \leq j$, and hence
    $g^{j - i} = 1$. 
    Note $\ord(g) = 1 \iff g = 1$.
\end{remark}

We can look at the order of an element $g \in G$ as the order of the 
cyclic group that it generates:
\[
    \langle g \rangle = \{1, g, g^2, \ldots\}.
\]

\begin{example}
    \begin{enumerate}
        \item 
            In $\Z/6$, $\overline 1$ and $\overline 5$ both have order $6$.

        \item
            In $D_n$, $r$ has order $n$ and $s$ has order $2$.

        \item 
            The order of $D_n$ is $2n$.
    \end{enumerate}
\end{example}
