\chapter{Relating and identifying finite groups}
\lecture{9}{11/2}

Recall that we have seen that any group of order $2$ must be isomorphic to $\Z/2$,
and similarly any group of order $3$ must be isomorphic to $\Z/3$.

\begin{lemma}[]
    \hfill
    \begin{enumerate}
        \item Let $G$ be a cyclic group of order $n$.
            Then $G \cong \Z/n$.

        \item Let $G$ be a group of prime order $p$.
            Then $G$ is cyclic, so that $G \cong \Z/p$.
    \end{enumerate}
\end{lemma}

\begin{proof}
    \hfill
    \begin{enumerate}
        \item Let $G$ be cyclic of order $n$, and let $g$ be a generator.
            It is easy to check that $g^i \mapsto \overline i \in \Z/n$
            defines an isomorphism.

        \item Let $G$ be a group of prime order $p$.
            Then the subgroup $\langle g \rangle \subset G$
            for any $g \neq 1$ must be of order $p$ by Lagrange's theorem
            and thus $\langle g \rangle = G$, so $G$ is cyclic.
            By the previous part, $G \cong \Z/p$.
    \end{enumerate}
\end{proof}

Among the finite groups, we have seen 
the cyclic groups $\Z/n$, 
the dihedral groups $D_n$, and
the symmetric groups $S_n$.
Note that there is some overlap between these, for example 
$\Z/1 \cong S_1$, $D_1 \cong \Z/2$, and $D_3 \cong S_3$.

\section{Direct products}

\begin{definition}[Direct product]
    Let $G,H$ be groups.
    Their \textbf{direct product} is
    \[
        G \times H = \{(g,h) \in G \times H\}
    \]
    where the binary operation is defined componentwise, 
    in terms of the operations in $G$ and $H$.
\end{definition}

\begin{example}
    Suppose that $G$ is a group of order 4. If $G$ is cyclic, then $G \cong \Z/4$.
    If $G$ is not cyclic, then every element has order $1$ or $2$
    (because then no element has order $4$).
    Since only 1 has order 1, we have
    \[
        G = \{1,a,b,c\},
    \]
    where $a,b,c$ are distinct and have order $2$.
    Since none of $a,b,c$ are the identity,
    we must have $ab \in \{1,c\}$.
    If $ab = 1$ then $a^2b = a$ and so $b = a$; a contradiction.
    Thus $ab = c$.
    Similarly, $ba = c$. So $ab = ba$. By symmetry $ac = ca$ and $bc = cb$;
    thus $G$ is abelian.
    We can show (by constructing an isomorphism) that 
    $G \cong \Z/2 \times \Z/2$.
    This group is often called the \emph{Klein 4-group}. 
    Since $D_2$ has order 4 and is not cyclic, we have proved that
    $D_2 \cong \Z/2 \times \Z/2$.
\end{example}

Isomorphisms preserve the order of elements.
Therefore, one can easily prove that groups are not isomorphic by
showing that one has an element of a certain order, while another does not.

\begin{example}
    Is $\Z/3 \times S_3$ isomorphic to $D_9$?
\end{example}

\begin{solution}
    $\abs{\Z/3 \times S_3} = 3 \cdot 6 = 18 = \abs{D_9}$, so maybe.
    Every element of $D_9$ must have an order that divdes $9$ or $2$;
    on the other hand, in $\Z/3 \times S_3$ we have the element
    \[
        (\overline 1, (1\;2))
    \]
    that has order $6$.
    Hence, $\Z/3 \times S_3 \not\cong D_9$.
\end{solution}

\section{$D_3 \cong S_3$}

In this section, we will provide two ways of proving this.
One is a more direct hands-on approach while the other is more
\emph{structural}.
The first method helps us to understand how to define a homomorphism using
generators, while the second method helps to understand the \emph{structural}/
\emph{geometrical} connecting between dihedral groups and symmetric groups.

\paragraph{Method 1}
We know
\[
    D_3 = \{1,r,r^2,s,rs,r^2s\}
\]
and
\[
    S_3 = \{1, (1\;2), (2\;3), (1\;3), (1\;2\;3), (1\;3\;2)\}.
\]
Any homomorphism $\varphi: D_3 \to S_3$ is completely determined by the values
$\varphi(r)$ and $\varphi(s)$ since all other values of $D_3$ are
products of $r$s and $s$s, and $\varphi$ is multiplicative.
If $\varphi$ is an isomorphism, it must preserve the order of elements.
So $\varphi(r)$ must have order $3$ and $\varphi(s)$ must have order $2$.
So we define $\varphi: D_3 \to S_3$ such that
\[
    \varphi(r) = (1\;2\;3), \qquad \varphi(s) = (1\;2).
\]
If $\varphi$ is a homomorphism, it follows that $\varphi(1) = 1$ and
\begin{align*}
    \varphi(r^2)  &= \varphi(r)^2 = (1\;3\;2) \\
    \varphi(r^2s) &= \varphi(r)^2\varphi(s) = (1\;3\;2)(1\;2) = (2\;3) \\
    \varphi(rs)   &= \varphi(r)\varphi(s) = (1\;2\;3)(1\;2) = (1\;3).
\end{align*}
What remains to check is that $\varphi$ is \emph{well-defined}.
We have the relation $srs = r^2$, so in order for $\varphi$ to be a function
we must have $\varphi(srs) = \varphi(r^2)$.
But
\[
    \varphi(srs) = (1\;2)(1\;2\;3)(1\;2) = (1\;3\;2),
\]
so this is true.
We see that from its values that it is bijective, and hence is an isomorphism.

\paragraph{Second method}

$D_3$ acts on a triangle which was labelled by its vertices $(1,2,3)$.
Applying an element of $D_3$ to the triangle, we obtain a new triangle
whose vertices have been permuted. Hence we obtain a function
$\varphi: D_3 \to S_3$ which sends any symmetry in $D_3$ to the corresponding
permutation in $S_3$.
This map is well defined by construction, because given a n element in $D_3$ we
have uniquely associated a permutation.
The map is also a homomorphism by construction because composing two symmetries
corresponds to composing the two corresponding permutations.
The kernel of $\varphi$ is $\{1\}$ because if a symmetry gives rise
to the identity permutation, it means it is the identity symmetry.
Thus $\varphi$ is injective.
Since $\abs{D_6} = 6 = \abs{S_3}$, any injection must be a bijection.
Therefore, $\varphi$ is an isomorphism.
