\chapter{The alternating groups}
\lecture{11}{18/2}

This chapter will introduce a family of normal subgrounds of the symmetric groups,
called the \emph{alternating groups}.

Let $x_1, \ldots, x_n$ be indeterminates and consider the polynomial
\[
    F(x_1, \ldots, x_n) = \prod_{1 \leq i < j \leq n} (x_j - x_i).
\]
For example, for $n = 3$ we have
\[
    F(x_1,x_2,x_3) = (x_2 - x_1)(x_3 - x_1)(x_3 - x_2).
\]
Given $\sigma \in S_n$, we define $F^\sigma$ by appling $\sigma$ to the indices:
\[
    F^\sigma(x_1, \ldots, x_n) = \prod_{1 \leq i < j \leq n} (x_{\sigma(j)} - x_{\sigma(i)}).
\]
For example, if $\sigma = (2\;3)$ then
\[
    F^\sigma(x_1, x_2, x_3) = (x_3 - x_1)(x_2 - x_1)(x_2 - x_3) = -F(x_1, x_2, x_3).
\]
We always have
\[
    F^\sigma(x_1, \ldots, x_n) = (-1)^s F(x_1, \ldots, x_n)
\]
for some integer $s$, and we call $(-1)^s$ the \textbf{sign} of $\sigma$
and write $\sgn(\sigma) = (-1)^s$.
We thus have a function $\sgn: S_n \to \{\pm 1\}$.
The result of applying some $\sigma_1$ and then $\sigma_2$ to $F$
is just the result of applying $\sigma_2\sigma_1$ to $F$.
That is,
\begin{align*}
    (F^{\sigma_1}(x_1,\ldots,x_2))^{\sigma_2}
    &= (\sgn(\sigma_1) F(x_1,\ldots,x_n))^{\sigma_2} \\
    &= \sgn(\sigma_1)\sgn(\sigma_2) F(x_1, \ldots, x_n) \\
    &= F^{\sigma_2\sigma_1}(x_1,\ldots,x_n).
\end{align*}
Hence $\sgn(\sigma_2\sigma_1) = \sgn(\sigma_2)\sgn(\sigma_1)$
and thus $\sgn$ is a group homomorphism.

\begin{theorem}[]
    For a given permutation $\sigma \in S_n$,
    the number of factors in any factorisation of $\sigma$ into transpositions
    is even if $\sgn(\sigma) = 1$ and odd if $\sgn(\sigma) = -1$.
\end{theorem}

\begin{proof}
    \hfill
    \paragraph{Claim}
    We claim that for any transposition $\tau \in S_n$ we have
    \[
        F^\tau(x_1,\ldots,x_n) = -F(x_1,\ldots,x_n),
    \]
    and thus if $\sigma$ is the product of $s$ transpositions then
    \[
        F^\sigma(x_1, \ldots, x_n) = (-1)^s F(x_1, \ldots x_n).
    \]

    \paragraph{Proof of claim}
    Let $\tau = (a\;b)$ such that $a > b$.
    Then 
    \begin{align*}
        F^\tau(x_1, \ldots, x_n)
        &= \prod_{1 \leq i < j \leq n} \left(x_{\tau(j)} - x_{\tau(i)}\right) \\
        &= \prod_{j \in \{2, \ldots, n\}} \prod_{i \in \{1, \ldots, j-1\}}
            \left(x_{\tau(j)} - x_{\tau(i)}\right) \\
        &= \left(x_{\tau(2)} - x_{\tau(1)}\right) \ldots
            \left(x_{\tau(a)} - x_{\tau(b)}\right) \ldots
            \left(x_{\tau(n)} - x_{\tau(n-1)}\right) \\
        &= \left(x_{\tau(2)} - x_{\tau(1)}\right) \ldots
            (-1)\left(x_{a} - x_{b}\right) \ldots
            \left(x_{\tau(n)} - x_{\tau(n-1)}\right) \\
        &= -F(x_1, \ldots, x_n).
    \end{align*}
    This proves the claim.
    So if $\sigma \in S_n$ is factorised into transpositions, then
    \[
        \sgn(\sigma) =
        \begin{cases}
            1 & \text{$\sigma$ has an even number of factors} \\
            -1 & \text{otherwise}.
        \end{cases}
    \]
\end{proof}

If $\sgn(\sigma) = 1$ for $\sigma \in S_n$ then we 
say that $\sigma$ is an \textbf{even} permutation.
If $\sgn(\sigma) = -1$ then we say that it is an \textbf{odd} permutation.

\begin{example}
    \begin{enumerate}
        \item $(1\;2\;5) = (1\;2)(2\;5)$ is even.
        \item $(4\;8\;9\;6)=(4\;8)(8\;9)(9\;6)$ is odd.
        \item For \emph{any} $k$-cycle $\sigma \in S_n$, we have
            \[
                \sgn(\sigma) = (-1)^{k - 1}
            \]
            as every permutation is a product of transpositions.
    \end{enumerate}
\end{example}

We can now define the \emph{alternating groups}.

\begin{definition}[]
    The \textbf{alternating group} $A_n$ is the subgroup of $S_n$
    consisting of even permutations.
\end{definition}

\begin{remark}
    In other words, $A_n$ is the kernel of $\sgn$.
\end{remark}

\begin{example}
    Recall the subgroups of $S_3$, we have
    \[
        A_3 = \langle (1\;2\;3) \rangle = \{1, (1\;2\;3), (1\;3\;2)\}.
    \]
    We know that $A_4$ has 12 elements, so $A_4$ is the subgroup consisting
    of $1$, all $3$-cycles, and all products of $2$-cycles.
\end{example}
