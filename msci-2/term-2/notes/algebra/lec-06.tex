\lecture{6}{30/1}

\begin{example}
    $S_3$ is generated by $(1\;2)$ and $(2\;3)$.
    We see:
    \begin{align*}
        (1\;2)^2 &= 1 \\
        (1\;2)(2\;3) &= (1\;2\;3) \\
        (1\;2\;3)^2 &= (1\;3\;2) \\
        (1\;3\;2)(1\;2) &= (1\;3)
    \end{align*}
    and as $\ord(S_3) = 3! = 6$, we have all the elements.
\end{example}

\chapter{Subgroups, cosets, and Lagrange}

\begin{definition}[Subgroup]
    Let $G$ be a group.
    $H \subset G$ is a \textbf{subgroup} of $G$ if
    \begin{enumerate}
        \item $1 \in H$ where $1$ is the identity of $G$;
        \item for all $h_1, h_2 \in H$, $h_1h_2 \in H$; and
        \item for any $h \in H$, $h^{-1} \in H$.
    \end{enumerate}
\end{definition}

We refer to a subgroup $H \subset G$ as \textbf{proper} if
$H \neq G$.

\begin{example}[Examples of subgroups]
    \hfill
    \begin{enumerate}
        \item $\langle \overline 2 \rangle = 
            \{ \overline 0, \overline 2, \overline 4 \}
            \subset \Z/6$;
        
        \item $\langle r \rangle =
            \{ 1, r, r^2, \ldots, r^n \}
            \subset D_n$; and

        \item $\langle 2 \rangle = 2 \Z \subset \Z$.
    \end{enumerate}
\end{example}

\begin{remark}
    It is interesting to note that,
    in the examples above,
    the order of the subgroup always divides the order of the group
    it is contained within.
    This is not a coincidence, and we will show that this always holds.
\end{remark}

\section{Cosets}

\begin{definition}[Coset]
    Let $G$ be a group, $H \subset G$ be a subgroup, and $g \in G$.
    Then
    \[
        gH = \{gh : h \in H\}
    \]
    is called the \textbf{left coset} of $H$ with respect to $g$.
    Similarly,
    \[
        Hg = \{hg: h \in H\}
    \]
    is called the \textbf{right coset} of $H$ with respect to $g$.
\end{definition}

\begin{remark}
    If $G$ is abelian, $gH = Hg$.
\end{remark}

\begin{example}
    Let $G = \Z$ and $H = 2\Z$.
    We will write our cosets additively here, so $g + H$ (instead of $gH$).
    Now
    \begin{align*}
        0 + H &= H \\
        1 + H &= \{1 + 2n: n \in \Z\} \\
        2 + H &= H \\
        \vdots
    \end{align*}
    so we see that there is only two distinct cosets:
    $2\Z$ and $1 + 2\Z$.
\end{example}

\begin{example}
    Take $G = D_3$ and $H = \langle s \rangle = \{1,s\}$.
    The left cosets are
    \[
        1H = H, \qquad rH, \qquad r^2H
    \]
    as $rsH = rH$ and $r^2sH = r^2H$ (from the fact that $s \in H$).
    We know that $rH$ and $r^2H$ are distinct as otherwise there would exist
    $h, h' \in H$ such that $rh = r^2h'$, but then $h(h')^{-1} = r$
    which is a contradiction as $r \not \in H$.
\end{example}

\begin{remark}
    The last example provides the neat fact that if $xH = yH$ then
    \[
        y^{-1}x \in H.
    \]
    The converse is also true; therefore
    \[
        xH = yH \iff y^{-1}x \in H.
    \]
\end{remark}

\begin{lemma}[]
    Let $G$ be a group and $H \subset G$ be a subgroup.
    Then
    \[
        G = \bigcup_{g \in G} gH
    \]
    and for any two cosets $gH$ and $g'H$ either
    $gH = g'H$ or $gH \cap g'H = \varnothing$.
\end{lemma}

\begin{proof}
    For all $g \in G$, $g \in gH$; hence, 
    \[
        G = \bigcup_{g \in G} gH.
    \]
    Now let $gH$ and $g'H$ be two distinct cosets and consider
    $x \in gH \cap g'H$. 
    Then 
    \[
        x = gh = g'h'
    \]
    where $g, g' \in G$ and $h, h' \in H$.
    Then
    \[
        xH = ghH = gH = g'h'H = g'H
    \]
    but $gH$ and $g'H$ are distinct; contradiction.
    Therefore, no such $x$ can exist
    and so $gH \cap g'H = \varnothing$.
\end{proof}

\begin{theorem}[Lagrange]
    Let $G$ be a finite group and $H \subset G$ be a subgroup.
    Then
    \[
        \lvert G \rvert = m \cdot \lvert H \rvert
    \]
    where $m$ is the number of left cosets of $H$ in $G$.
\end{theorem}

\begin{proof}
    Let $g_1H, g_2H, \ldots, g_mH$ be the distinct left cosets of $H$ in $G$
    where $g_i \in G$.
    Then
    \[
        G = \bigcup_{i = 1}^m g_iH
    \]
    and we know that each $g \in G$ lies in exactly one of these cosets.
    Thus
    \[
        \lvert G \rvert = \sum_{i = 1}^m \lvert g_iH \rvert.
    \]
    We define the function
    \[
        f: g_iH \to H, \qquad f(g_ih) = h.
    \]
    $f$ is clearly surjective and
    $f(g_ih) = f(g_ih') \implies h = h' \implies g_ih = g_ih'$ 
    so $f$ is injective; therefore, $f$ is a bijection.
    Therefore $\lvert g_iH \rvert = \lvert H \rvert$ and so
    \[
        \lvert G \rvert = \sum_{i = 1}^m \lvert g_i H \rvert
        = m \cdot \lvert H \rvert.
    \]
\end{proof}

\begin{definition}[Index]
    Let $G$ be a group and $H \subset G$ be a subgroup.
    Then we define the \textbf{index} of $H$ in $G$ as
    $
        \frac{\lvert G \rvert}{\lvert H \rvert}
    $
    and it represents the number of cosets of $H$ in $G$.
\end{definition}
