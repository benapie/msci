\lecture{5}{28/1}

\begin{example}
    Write
    $
        (1\;2\;5)(4\;6\;8\;9)
    $
    as a product of transpositions.
\end{example}

\begin{solution}
    \begin{align*}
        \begin{pmatrix}
            1 & 2 & 5
        \end{pmatrix}
        \begin{pmatrix}
            4 & 6 & 8 & 9
        \end{pmatrix}
        &=
        \begin{pmatrix}
            1 & 2
        \end{pmatrix}
        \begin{pmatrix}
            2 & 5
        \end{pmatrix}
        \begin{pmatrix}
            4 & 6 & 8 & 9
        \end{pmatrix}
        \\
        &=
        \begin{pmatrix}
            1 & 2
        \end{pmatrix}
        \begin{pmatrix}
            2 & 5
        \end{pmatrix}
        \begin{pmatrix}
            4 & 6
        \end{pmatrix}
        \begin{pmatrix}
            6 & 8 & 9
        \end{pmatrix}
        \\
        &=
        \begin{pmatrix}
            1 & 2
        \end{pmatrix}
        \begin{pmatrix}
            2 & 5
        \end{pmatrix}
        \begin{pmatrix}
            4 & 6
        \end{pmatrix}
        \begin{pmatrix}
            6 & 8
        \end{pmatrix}
        \begin{pmatrix}
            8 & 9
        \end{pmatrix}
        .
    \end{align*}
\end{solution}

\begin{proposition}
    Every permutation is a product of transpositions
    (but not uniquely).
\end{proposition}

\begin{proof}
    We know that every permutation is a product of disjoint
    cycles, so it suffices to show that any cycle is a product
    of cycles.
    Now
    \[
        \begin{pmatrix}
            a_1 & a_2 & \ldots & a_k
        \end{pmatrix}
        =
        \begin{pmatrix}
            a_1 & a_2
        \end{pmatrix}
        \begin{pmatrix}
            a_2 & a_3
        \end{pmatrix}
        \ldots
        \begin{pmatrix}
            a_{k-1} & a_k
        \end{pmatrix}
        .
    \]
\end{proof}

\begin{example}
    Write 
    $
        \begin{pmatrix}
            1 & 2 & 3
        \end{pmatrix}
        \begin{pmatrix}
            2 & 3 & 4
        \end{pmatrix}
    $
    as a product of transpositions in two different ways.
\end{example}

\begin{solution}
    \begin{align*}
        \begin{pmatrix}
            1 & 2 & 3
        \end{pmatrix}
        \begin{pmatrix}
            2 & 3 & 4
        \end{pmatrix}
        &=
        \begin{pmatrix}
            1 & 2
        \end{pmatrix}
        \begin{pmatrix}
            2 & 3
        \end{pmatrix}
        \begin{pmatrix}
            2 & 3
        \end{pmatrix}
        \begin{pmatrix}
            3 & 4
        \end{pmatrix}
        \\
        \begin{pmatrix}
            1 & 2 & 3
        \end{pmatrix}
        \begin{pmatrix}
            2 & 3 & 4
        \end{pmatrix}
        &=
        \begin{pmatrix}
            1 & 2
        \end{pmatrix}
        \begin{pmatrix}
            3 & 4
        \end{pmatrix}
        .
    \end{align*}
\end{solution}

Note that the number of products in both factorisation are
even, this is not a coincidence.

\section{Computations with permutations}

\begin{example}
    Let 
    $
        \sigma =
        \begin{pmatrix}
            2 & 4 & 1
        \end{pmatrix}
        \in S_5
    $.
    Then $\sigma(1) = 2$, $\sigma(2) = 4$, and $\sigma(4) = 1$.
    So if we take $\sigma \circ \sigma = \sigma^2$ we get the
    permutation: 
    $1 \mapsto 4$, $2 \mapsto 1$, and $4 \mapsto 2$.
    Thus 
    $
        \sigma^2 =
        \begin{pmatrix}
            1 & 4 & 2
        \end{pmatrix}
    $.
    Considering $\sigma^3$ we have
    $1 \mapsto 1$, $2 \mapsto 2$, and $4 \mapsto 4$.
    Thus $\sigma^3 = 1$ so $\ord\sigma = 3$.
    More generally, a $k$-cycle ($k \in \N$) has order $k$.
\end{example}

\begin{lemma}
    Let $\sigma = \sigma_1 \ldots \sigma_m$
    be a product of disjoint cycles of length $k_u$.
    Then 
    \[
        \ord(\sigma) = \lcm(k_1, \ldots, k_m).
    \]
\end{lemma}

\begin{proof}
    Let $L = \lcm(k_1, \ldots, k_m)$.
    We know that $\ord(\sigma_i) = k_i$ (see earlier example).
    So 
    \[
        \sigma^L = \sigma_1^L = \ldots = \sigma_m^L = 1
    \]
    and since disjoint cycles commute, $\ord(\sigma) \leq L$.
    Suppose $\sigma_i^N = 1$ for some $N \in \N$.
    Then we can express 
    $N = k_i q + r$
    for some $q, r \in \N_0 = \N \cup \{0\}$ where
    $r < k_i$.
    So $\sigma_i^N = \sigma_i^r = 1$
    but $k_i$ is the smallest non-zero number with
    this property so $r = 0$.
    Therefore $k_i \mid N$.
    Now let $\sigma = \ord(\sigma).$.
    Since disjoint cycles commute
    \[
        1 = \sigma^N = \sigma_1^N \ldots \sigma_m^N = 1
    \]
    and $\sigma_1^N, \ldots, \sigma_m^N$ are still disjoint
    so $\sigma_i^N = 1$ for $1 \leq i \leq n$.
    Thus $k_i \mid N$ and $L \mid N$.
    Therefore
    \[
        L \leq N = \ord(\sigma) \leq L \implies N = L.
    \]
\end{proof}

\begin{example}[Inverses]
    If
    $
        \sigma =
        \begin{pmatrix}
            2 & 4 & 1
        \end{pmatrix}
    $
    then
    $
        \sigma^{-1} =
        \begin{pmatrix}
            2 & 4 & 1
        \end{pmatrix}
    $.
    In general, the inverse of a cycle is the cycle read backwards.
    That is,
    \[
        \begin{pmatrix}
            a_1 & a_2 & \ldots & a_k
        \end{pmatrix}
        ^{-1}
        =
        \begin{pmatrix}
            a_k & a_{k-1} & \ldots & a_1
        \end{pmatrix}
        ^{-1}.
    \]
    For a product $\sigma = \sigma_1 \ldots \sigma_m$ then
    \[
        \sigma^{-1} = \sigma_m^{-1} \ldots \sigma_1^{-1}.
    \]
\end{example}

\begin{lemma}
    Let
    $
        \sigma =
        \begin{pmatrix}
            a_1 & a_2 & \ldots & a_k
        \end{pmatrix}
        \in S_n
    $
    be a cycle and $\lambda \in S_n$.
    Then
    \[
        \lambda \sigma \lambda^{-1}
        = 
        \begin{pmatrix}
            \lambda(a_1) & \lambda(a_2) & \ldots & \lambda(a_k)
        \end{pmatrix}
        .
    \]
\end{lemma}

\begin{proof}
    \begin{align*}
        \lambda \sigma \lambda^{-1}(\lambda(a_1))
        &= \lambda\sigma(a_1)
        = \lambda(a_2) \\
        &\vdots \\
        \lambda \sigma \lambda^{-1}(\lambda(a_k))
        &= \lambda\sigma(a_k)
        = \lambda(a_1).
    \end{align*}
\end{proof}
%todo not really necessary but change pmatrix to (x\;x\;x)
