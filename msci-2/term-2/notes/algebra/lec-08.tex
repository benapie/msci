\lecture{8}{6/2}

Just as for rings, we see that a homomorphism $\varphi$ is injective if and only if
$\ker{\varphi} = \{1\}$
(the proof for this is the same again).

\section{Normal subgroups, quotients, and the FIT}

Recall that \emph{ideals} are special kinds of subsets of a ring $R$
such that the quotient $R/I$ is again a ring.
Moreover, we have seen that ideals are exactly the kernels of ring homomorphisms.

Analogously, we will define certain subgroups called \emph{normal} such that we can
form quotient groups.
Moreover, the normal subgroups will be exactly the kernels of group homomorphisms.

\begin{definition}[Normal]
    The subgroup $H \subset G$ is said to be \textbf{normal} (in $G$)
    if for every $g \in G$ and $h \in H$ we have
    \[
        g^{-1} h g \in H.
    \]
\end{definition}

\begin{example}
    The following examples are easy to show.
    \begin{enumerate}
        \item If $G$ is abelian, any subgroup $H$ is normal.

        \item The subgroup $\langle r \rangle$ of $D_n$ is normal.
    \end{enumerate}
\end{example}

\begin{definition}[Quotient group]
    Let $N$ be a normal subgroup of a group $G$.
    The \textbf{quotient group} $G/N$ is the group
    \[
        G/N = \{gN : g\in G\}
    \]
    with the operation $(gN)(g'N) = gg'N$.
    The identity in $G/N$ is $1\cdot N = N$ and $(gN)^{-1} = g^{-1}N$.
\end{definition}

Just ling for rings modulo ideals, we have a canonical surjective homomorphism
\[
    G \to G/N, \qquad g \mapsto gN
\]
whose kernel is exactly $N$.
Just like ideals are kernels, we thus deduce that normal subgroups are kernels.

\begin{lemma}[]
    Let $\varphi: G \to H$ be a homomorphism of groups.
    Then $\ker\varphi$ is normal in $G$.
    Conversely, if $N$ is a normal subgroup of $G$, 
    then $N$ is the kernel of a homomorphism.
\end{lemma}

\begin{proof}
    For the first statement, let $x \in \ker\varphi$ and $g \in G$.
    Then
    \[
        \varphi(gxg^{-1}) = \varphi(g)\varphi(g)^{-1} = 1
    \]
    so $gxg^{-1} \in \ker\varphi$ and so $\varphi$ is normal.
    The second statement comes directly from what we have noted about 
    $G \to G/n$.
\end{proof}

\begin{lemma}[]
    Let $G$ be a finite group and $N$ be a normal subgroup.
    Then
    \[
        \abs{G/N} = \frac{\abs G}{\abs N}. 
    \]
\end{lemma}

\begin{proof}
    By Lagrane's theorem, we have that $\frac{\abs G}{\abs N}$ is the number of left
    cosets of $N$.
    By the definition of $G/N$, its order is the number of cosets of $N$.
\end{proof}

\begin{theorem}[FIT for groups]
    Let $\varphi:G \to H$ be a group homomorphism.
    Then
    \[
        G/\ker\varphi \cong \im\varphi.
    \]
\end{theorem}

\begin{proof}
    The proof is very similar to that of rings, so we omit most of it.
    We note that the isomorphism is given by the map $g\ker\varphi \to \varphi(g)$.
\end{proof}

\begin{example}
    Let $\varphi$ be a homomorphism defined by
    \[
        \varphi: D_n \to \Z/2, \qquad \varphi(r^is^j) = \overline j \pmod 2.
    \]
    This map is clearly surjective and has kernel $\langle r \rangle$.
    Thus, $\langle r \rangle$ is normal in $D_n$
    (which we already knew).
    Moreover, FIT imlpies that
    \[
        D_n/\langle r \rangle \cong \Z/2.
    \]
\end{example}

\begin{example}
    Let
    \[
        \varphi: \Z/6 \to S_3, \qquad \varphi(\overline a) = (1\;2\;3)^a
    \]
    be a homomorphism.
    We see that
    \[
        \im\varphi = \langle(1\;2\;3)\rangle, \qquad \ker\varphi = \{\overline0, \overline3\}
    \]
    so
    \[
        (\Z/6)/\{\overline 0, \overline 3\} \cong \langle (1\;2\;3) \rangle.
    \]
\end{example}
