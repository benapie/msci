\lecture{5}{10/2}

\begin{definition}[Big $O$]
    Let 
    $R \subset \R_{>0}$ 
    be unbounded and
    $f,g: R \to \R$
    such that there exists 
    $x' \in R$ 
    such that
    $g(x) > 0$
    for all 
    $x > x'$.
    Then 
    \[
        f(x) = O(g(x)) \qquad \text{as}\;x \to \infty
    \]
    if and only if there exists $M > 0$ and $x_0 \in \R$ such that
    \[
        \abs{f(x)} \leq Mg(x)
    \]
    for all $x \geq x_0$.
\end{definition}

Big $O$ notation is about describing the \emph{asymptotic} behaviour
of functions; 
that is, how a function $f(x)$ acts as $x \to \infty$.

\begin{example}
    \hfill
    \begin{enumerate}
        \item $\log{n} \in O(n)$;
        \item $n \in O(n)$;
        \item $n \not\in O(\log n)$;
        \item $3n + 7 \in O(n^2)$;
        \item $2^n \not\in O(n^k)$ for $k \in \N$; and
        \item $n^k \in O(2^n)$ for $k \in \N$.
    \end{enumerate}
\end{example}

\begin{example}[Comparing different bases of $\log$]
    What can we say about, say, $\log_2{n}$ and $\log_{2^{100}} n$?
    Well, we have the following formula for $\log$:
    \[
        \log_an = (\log_ab)(\log_bn).
    \]
    So, we have
    \[
        \log_2n = \log_2(2^{100})\log_{2^{100}}n
    \]
    hence
    \[
        \log_2n = O(\log_{2^{100}}n) 
        \quad\text{and}\quad
        \log_{2^{100}}n = O(\log_2n).
    \]
    That is, $\log_2n = \Theta(\log_{2^{100}}n)$.
\end{example}

\begin{definition}[Time complexity]
    Let $R \subset \R_{>0}$ be unbounded
    and $f: R \to \R$.
    We say that the \textbf{time complexity} of a decidable language
    $\mathcal L$ is $O(f)$ (or $\mathcal L$ is decidable in $O(f)$ time)
    if there exists a Turing machine $T$ which decides $\mathcal L$
    such that there exists $c, n_0 \in \N$ such that for all $x$ with 
    $\abs{x} > n_0$ we have
    \[
        \TIME_T(x) \leq cf(\abs{x}).
    \]
\end{definition}

\begin{definition}[Time complexity class]
    The \textbf{time complexity class} $\operatorname{TIME}[f]$
    is the class of all languages for which
    there exists an algorithm with time complexity $O(f)$.
\end{definition}

\begin{remark}
    We sometimes use the notation $\operatorname{DTIME}[f]$ for
    \emph{determinstic time}.
\end{remark}

\begin{definition}[$\poly$]
    We define $\poly$ as
    \[
        \poly = \bigcup_{k \geq 0} \TIME[n^k].
    \]
\end{definition}

$\poly$ is a good model for \emph{tractable} (or \emph{solvable}) 
problems. 
We need to take caution though, as if $\deg f$ is large
or the constants involved are large, 
then our algorithm may take a long time to terminate.
This typically only arises in artifically constructed problems,
however.

\begin{lemma}[]
    We can simulate $t$ steps of a $k$-tape Turing machine
    with an equivalent $1$-tape Turing machine in $O(t^2)$.
\end{lemma}

\begin{lemma}[]
    We can simulate $t$ steps of a two-way infinite $k$-tape
    Turing machine with an equivalent $k$-tape Turing machine
    in $O(t)$.
\end{lemma}

\begin{lemma}[]
    Let $b_1, b_2 \in \N$ and $x \in \R$.
    Let $\operatorname{enc}(x,b)$ be the encoding of $x$ in base $b$.
    If $b_1, b_2 \geq 2$ then
    \[
        \operatorname{enc}(x,b_1) = c\operatorname{enc}(x,b_2)
    \]
    for some $c \in \R$.
\end{lemma}

The above lemma shows us that $\poly$ is equivalent
regardless of the base of encoding.
Hence $\poly$ is \emph{robust} to the encoding or computation method.
The best way to show that a problem is in $\poly$ 
is to find an algorithm that solves it in polynomial time.
Finding this algorithm typically provides us with lots of insight
as it is typically more efficient than brute-force.

\begin{definition}[Conjunctive normal form]
    A boolean formula $f$ is in \textbf{conjunctive normal form} (CNF)
    if it is the conjunction of one or more \textbf{clauses},
    which are the disjunction of literals.
    That is,
    \[
        f = C_1 \land C_2 \land \ldots \land C_n
    \]
    where each $C_i$ is a clause of the form
    \[
        C_i = l_1 \lor l_2 \lor \ldots \lor l_k
    \]
    where each $l_j$ is a literal, 
    which is a variable or its negation.
\end{definition}

\begin{definition}[$k$-CNF]
    If a boolean formula $f$
    has at most $k$ literals in each clause
    then it is in $k$-CNF.
\end{definition}

\begin{example}
    Let
    \[
        f(x_1,x_2,x_3,x_4,x_5) =
        (x_1 \lor x_2 \lor \neg x_5)
        \land
        (\neg x_2 \lor \neg x_4 \lor \neg x_5)
        \land
        (x_2 \lor x_3 \lor x_4).
    \]
    $f$ is in $3$-CNF.
\end{example}

\begin{definition}[Satifiable]
    Let $f: \{0,1\}^n \to \{0,1\}$ be a boolean formula.
    We say that $f$ is \textbf{satifiable} if there exists
    $x_1,x_2,\ldots,x_n$
    such that
    $f(x_1,x_2,\ldots,x_n) = 1$.
\end{definition}

\begin{problem}[$\sat$]
    Let $f$ be in CNF.
    Is $f$ satisfiable?
\end{problem}

\begin{problem}[$k$-$\sat$]
    Let $f$ be in $k$-CNF.
    Is $f$ satisfiable?
\end{problem}

\begin{proposition}[]
    $2$-$\sat$ is $\poly$.
\end{proposition}

\begin{definition}[]
    A language $\mathcal L_1$ is \textbf{polynomially reducable} to 
    $\mathcal L_2$, denoted $\mathcal L_1 \leq \mathcal L_2$,
    if there exists a polynomial time function $f$ such that
    \[
        x \in \mathcal L_1 \implies f(x) \in \mathcal L_2.
    \]
\end{definition}

\begin{lemma}[]
    Let $\mathcal L_1$ and $\mathcal L_2$ be languages
    such that $\mathcal L_1 \leq \mathcal L_2$.
    Then
    \[
        \mathcal L_2 \in \poly \implies \mathcal L_1 \in \poly.
    \]
\end{lemma}

\begin{proof}
    Trivial.
\end{proof}
