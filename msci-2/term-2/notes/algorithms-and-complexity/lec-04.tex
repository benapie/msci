\chapter{Decision problems}
\lecture{4}{3/2}

We have up to now been focusing on \emph{optimisation problems}, such as
\begin{enumerate}
    \item shortest path between two vertices;
    \item minimum spanning tree of a graph;
    \item maximum flow; and
    \item maximum matching.
\end{enumerate}
We will now introduce \emph{decision problems} where the answer is
a yes or no.

\begin{problem}[Shortest path, decision problem]
    Let $G = (V,E)$ be a graph.
    Does there exist a path between $u,v \in V$
    with length at most $k$?
\end{problem}

\begin{problem}[Minimum spanning tree, decision problem]
    Does there exists a minimum spanning tree of a graph $G$
    with weight of at most $k$?
\end{problem}

\begin{problem}[Graph colouring, decision problem]
    Can a graph $G$ be coloured with at most $k$ colours?
\end{problem}

In these examples, we can use this problem to solve the related
optimisation problem.

\begin{theorem}[]
    An optimisation algorithm has a polynomial algorithm
    if and only if the corresponding decision problem
    has a polynomial algorithm.
\end{theorem}

\begin{definition}[Decision problem]
    We define a \textbf{decision problem} as an \emph{instance} 
    and a yes or no question.
\end{definition}

\begin{example}
    To redefine our shortest path problem more formally:
    \begin{enumerate}
        \item (instance) a finite directed graph $G = (V,E)$ with $s,t \in V$; and
        \item (question) does there exist a path $P \subset E$
            from $s$ to $t$.
    \end{enumerate}
\end{example}

Recall alphabets, strings, and languages.

For a problem $\Pi$ and an encoding sceheme $e$ with alphabet $\Sigma$,
the set of instances corresopnding with an answer \emph{yes} is denoted
$\mathcal L(\Pi, e)$
and is called the \textbf{language} asociated with $\Pi$ and $e$.

All problems we have found are \emph{tractable}; however,
there are some \emph{intractable} problems.
We are looking to \emph{classify} problems. We have seen that some languages are undecidable.
Every deciable problem has a turing machine that can solve it.
The following are some potentials measures that we can have on
these turing machines:
\begin{enumerate}
    \item difficulty of constructing the algorithm;
    \item static complexity measure; and
    \item dynamic complexity measure.
\end{enumerate}
A static complexity measure is something that will not be influenced
by the input given,
while a dynamic complexity measure will typically take the size of the
input.

\begin{problem}
    Let $G$ be a finite graph. Does $G$ have an Eulerian circuit?
\end{problem}

\begin{problem}
    Let $G$ be a finite graph. Does $G$ have a hamiltonian cycle?
\end{problem}

\begin{remark}
    Recall that an Eulerian circuit will visit every edge once
    and a hamiltonian cycle will visit every vertex once.
\end{remark}

Although these two problems look similar, there are no
similar characteristics that would allow us to \emph{reduce}
one to another. 
A hamiltonian cycle finder (as far as we know) must have exponetial time.

\begin{problem}[Independent set]
    Let $G$ be a graph and $k \in \N$.
    Does $G$ have an independent set of size of atleast $k$?
\end{problem}

\begin{problem}[Vertex cover]
    Let $G$ be a graph and $k \in \N$.
    Does $G$ have a vertex cover of size at most $k$?
\end{problem}

Now, we \emph{can} link these two problems.

\begin{theorem}[]
    Let $G = (V,E)$ be a graph.
    A set $I \subset V$ is independent if and only if
    $V \setminus I$ is a vertex cover.
\end{theorem}

\begin{proof}
    Suppose $I$ is independent and assume $V \setminus I$ is not a vertex cover.
    Then there exists $(u,v) \in E$ such that $u,v \not \in V \setminus I$.
    Therefore $u,v \in I$; contradiction. 
\end{proof}

\begin{problem}[Clique problem]
    Let $G = (V,E)$ be a graph and $k \in \N$.
    Does $G$ have a clique of size of at least $k$?
\end{problem}

\begin{remark}
    Recall that a clique is a subset of vertices such that 
    any two distinct vertices are adjacent.
\end{remark}

\begin{definition}[]
    Let $G = (V,E)$ be a graph.
    The \textbf{complement} of $G$, denoted $\overline G = (V, E')$, 
    is such that two vertices are adjacent in $\overline G$ if
    there are not adjacent in $G$.
\end{definition}

\begin{theorem}[]
    A graph $G$ has an indendent set of size $k$ if and only if
    $\overline G$ has a clique of size $k$.
\end{theorem}

The above theorem shows us that the independent set problem and
the clique problem are equivalent.

\begin{problem}[Set cover]
    Let $U$ be a set with $\lvert U \rvert = n$ where
    \[
        U = \bigcup_{i = 1}^t S_i
    \]
    and $k \in \N$.
    Does there exist a collection of $S_i$ such that
    at most $k$ of these sets union equals $U$.
\end{problem}

Can we use the previous examples to solve this one?
The anser is the vertex cover problem.
