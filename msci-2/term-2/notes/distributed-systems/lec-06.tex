\chapter{Concurrent events and consistency controls}
\lecture{6}{17/2}

\begin{definition}[Shared object]
    A \textbf{shared object} is an object that may be simultaneously accessed by
    multiple \emph{events}.
\end{definition}

Servers managing shared objects are responsible for maintaining consistency.
Such a control is called \textbf{concurrency control}.
This is typically done by \textbf{locking}.
If event $T$ happens before event $U$ in their conflicting access to
objects at one of the servers then they must be in that order at all of the
servers whose objects are accessed in a conflicting manner by both $T$ and $U$.
\emph{Locking} is one way to deal with concurrency control.

\begin{definition}[Locking]
    A \textbf{lock} (or \textbf{mutex}) is a synchronisation mechanism for
    limiting access to a resource in an environmentment with many threads of execution.
\end{definition}

A local lock manager will decide whether to grant a lock or to make a transaction wait.
Lockaing has an issue with distrubuted transactions:
if a lock is applied to a resource, they cannot be released until the transaction
has been committed \emph{at all servers}.
Objects cannot be accessed by anyone else while locked.

\emph{Timestamp ordering} is a non-locl concurrency control method.
\begin{definition}[Timestamp ordering]
    In timestamp-based concurrency control,
    each transaction $T_i$ is an ordered list of actions $A_{ix}$.
    Before the transaction performs its first action ($A_{i1}$)
    it is marked with the current timestamp.
    \textbf{Serial equivalence} is enforced by commiting versions of objects
    in the order of the timestamps.
\end{definition}

Servers of distributed transactions are responsible for enforcing serial equivalence;
coordinators \emph{must} agree on the timestamps.
In practise, timestamps are generated locally.
This makes it hard to sort by timestamp as we have \emph{no global clock}.
But, we can use the \emph{network-time protocol} to synchronise our clocks.

\textbf{Deadlock} can occur within a single server when locking is used for concurrency control.
A clumsy solution to this is \textbf{timeouts}; however,
it is difficult to pick suitable intervals for timeout and some transactions may be
aborted unnecessarily.

We can make use of a \textbf{wait-for graph} for detecting deadlocks.
In a wait-for graph, processes are represented by nodes
and a edge between processes $P_i$ and $P_j$ indicates that
$P_j$ is holding a resource that $P_i$ needs to execute.
It is clear that the possibility of a deadlock is implied by graph cycles. 
