\lecture{2}{20/1}

\begin{remark}
    We have many issues that we must overcome when designing distributed
    systems, such as
    \begin{enumerate}
        \item highly variable bandwidth;
        \item possibly large and variable latency;
        \item no native support for clock synchronisation;
        \item difficult to ensure data consistency;
        \item system failure; and
        \item security.
    \end{enumerate}

    To add on to point (v), as we add more components to a distributed system
    the probability of a component failing increases.
    This is a property of all distributed systems, we cannot prevent it.
    Because of this, we design our system to be robust against individual
    component failure (by using redudancy).

    Additionally, with regards to point (vi), when we have a single centralised
    system we can rely on physical security to prevent people having access
    to things they shouldn't; however, this is not possible in a widely
    distributed system.
\end{remark}

\begin{definition}[Transparency]
    A \textbf{transparency} is some aspect of a distributed system
    that is hidden from the user.
\end{definition}

\begin{remark}
    In this context, the user can be a programmer, system developer,
    or the end user of an operating system.
\end{remark}

\begin{example}[Examples of transparencies]
    \hfill
    \begin{enumerate}
        \item Access transparency:
            hide differences in data representation / how the components are 
            accessed.
            
        \item Location transparency:
            hide where the components are located.

        \item Migration transparency:
            hide that a component or resource may be moved to another location.

        \item Relocation transparency:
            hide that a component or resource may be moved
            \emph{while it is in use}.
        
        \item Replication transparency:
            hide that a component or resource may have many copied.
            
        \item Concurrency transparency:
            hide that a resource may be shared by several users.

        \item Failure transparency:
            hide the failure/recovery of a component or resource.
    \end{enumerate}
\end{example}

\chapter{Middleware technology}

Middleware hides the implementation and \emph{distributed} nature of
distributed systems from the user.
It provides a common programming abstraction for constructing distributed
applications for developers, 
while still hiding low-level detail.

\begin{definition}[Middleware]
    \textbf{Middleware} is software that facilitates components of
    distributed systems to communicate and work together.
\end{definition}

\begin{example}[Examples of middleware]
    \hfill
    \begin{enumerate}
        \item 
            \textbf{Java Database Connectivity} (JDBC) is an API that provides connectivity
            between Java and many different database systems.

        \item 
            \textbf{Remote component integration} is a piece of middleware 
            (that we will define soon) 
            that allows remote procedural calls to be made to other computers.
            We have \textbf{remote procedure call} which allows procedures to
            be executed remotely and \textbf{remote method invocation} which
            allows objects to be used remotely.

        \item A \textbf{service broker} is a service that provides matching
            services to the ones requested. 
            They identify suitable components to serve a purpose and provides
            features such as
            \begin{enumerate}
                \item concurrency control;
                \item load distribution; and
                \item fault tolerance.
            \end{enumerate}
    \end{enumerate}
\end{example}

\begin{remark}
    When developing a distributed system, our main goals are to:
    \begin{enumerate}
        \item focus on application logics; and
        \item avoid implementating socket communications or working with any
            network architectures due to its dynamic changes.
    \end{enumerate}
\end{remark}

\section{Remote procedure call}

\begin{definition}[Remote procedure call]
    A \textbf{remote procedure call} (RPC) is when a computer program causes
    a procedure to execute in a different address space,
    typically a different computer on the network,
    without the programmer esxplicitly coding the details for the remote
    interaction.
\end{definition}

\begin{remark}
    In a RPC, the programmer would write the same code
    whether the subroutine is local or remote.
\end{remark}

Now we will look into a remote procedure call would work.

\begin{definition}[Stub]
    A \textbf{stub} is a piece of code that converts parameters passed
    between client and server during a RPC.
\end{definition}

\paragraph{Sequence of events in a RPC}
\begin{enumerate}
    \item 
        The client will call the client stub.

    \item 
        The client stub will pack the parameters into a message 
        (known as \textbf{marshalling})
        and make a system call to send the message to the server.

    \item 
        The client's operating system sends the message from the client
        machine to the server machine.

    \item 
        The server's operating system passes the incoming packets
        to the server stub.

    \item 
        The server stub unpacks the parameters
        (known as \textbf{unmarshalling}).
    \item 
        Finally, the server stub calls the server procedure and traces the same
        steps backwards for the reply.
\end{enumerate}

\begin{definition}[Interface description language]
    An \textbf{interface description language} (IDL) (or \textbf{interface definition language})
    is a specification language (see formal language)
    used to to describe a software component's API.
\end{definition}

\begin{remark}
    IDLs are language-independent, 
    this allows communication between software components that do not share 
    a language.
\end{remark}

Stubs can be generated automatically using a \textbf{stub compiler}.
Stub compilers use IDLs to generate the client stub and the server stub.
